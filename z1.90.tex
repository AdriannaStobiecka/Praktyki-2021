\documentclass[12pt, a4paper]{article}
\usepackage[utf8]{inputenc}
\usepackage{polski}
\usepackage{amsthm}  %pakiet do tworzenia twierdzeń itp.
\usepackage{amsmath} %pakiet do niektórych symboli matematycznych
\usepackage{amssymb} %pakiet do symboli mat., np. \nsubseteq
\usepackage{amsfonts}
\usepackage{graphicx} %obsługa plików graficznych z rozszerzeniem png, jpg
\theoremstyle{definition} %styl dla definicji
\newtheorem{zad}{} 
\title{Multizestaw zadań}
\author{Robert Fidytek}
%\date{\today}
\date{}
\newcounter{liczniksekcji}
\newcommand{\kategoria}[1]{\section{#1}} %olreślamy nazwę kateforii zadań
\newcommand{\zadStart}[1]{\begin{zad}#1\newline} %oznaczenie początku zadania
\newcommand{\zadStop}{\end{zad}}   %oznaczenie końca zadania
%Makra opcjonarne (nie muszą występować):
\newcommand{\rozwStart}[2]{\noindent \textbf{Rozwiązanie (autor #1 , recenzent #2): }\newline} %oznaczenie początku rozwiązania, opcjonarnie można wprowadzić informację o autorze rozwiązania zadania i recenzencie poprawności wykonania rozwiązania zadania
\newcommand{\rozwStop}{\newline}                                            %oznaczenie końca rozwiązania
\newcommand{\odpStart}{\noindent \textbf{Odpowiedź:}\newline}    %oznaczenie początku odpowiedzi końcowej (wypisanie wyniku)
\newcommand{\odpStop}{\newline}                                             %oznaczenie końca odpowiedzi końcowej (wypisanie wyniku)
\newcommand{\testStart}{\noindent \textbf{Test:}\newline} %ewentualne możliwe opcje odpowiedzi testowej: A. ? B. ? C. ? D. ? itd.
\newcommand{\testStop}{\newline} %koniec wprowadzania odpowiedzi testowych
\newcommand{\kluczStart}{\noindent \textbf{Test poprawna odpowiedź:}\newline} %klucz, poprawna odpowiedź pytania testowego (jedna literka): A lub B lub C lub D itd.
\newcommand{\kluczStop}{\newline} %koniec poprawnej odpowiedzi pytania testowego 
\newcommand{\wstawGrafike}[2]{\begin{figure}[h] \includegraphics[scale=#2] {#1} \end{figure}} %gdyby była potrzeba wstawienia obrazka, parametry: nazwa pliku, skala (jak nie wiesz co wpisać, to wpisz 1)

\begin{document}
\maketitle


\kategoria{Wikieł/Z1.90}
\zadStart{Zadanie z Wikieł Z1.90 moja wersja nr [nrWersji]}
%[z]:[10,15,20,25,30,35,40,45,50,55]
%[x]:[12,14,16,18,22,24,26,28,32,34]
%[p1]=int([x]/2)
%[p2]=int([z]*2)
%[l1]=round(math.log([p1], 14),2)
%[p3]=int([p2]/5)
%[l2]=round(math.log([p3], 14),2)
Obliczyć $ \log_{[z]} [x] $, jeżeli wiadomo, że $ \log_{14} 2 = a $ i $ \log_{14} 5 = b $.\\
\zadStop
\rozwStart{Martyna Czarnobaj}{}
Wykorzystamy własność $ \log_a b = \frac{\log_c b}{\log_c a} $.\\
\begin{center}
	$ \log_{[z]} [x] = \frac{\log_{14} [x]}{\log_{14} [z]} = \frac{\log_{14} (2*[p1])}{\log_{14} (\frac{[p2]}{2})} = \frac{\log_{14} 2 + \log_{14} [p1]}{\log_{14} [p2] - \log_{14} 2} = \frac{\log_{14} 2 + [l1] }{\log_{14} (5*[p3]) - \log_{14} 2} = \frac{\log_{14} 2 + [l1] }{\log_{14} 5 - \log_{14} 2 + [l2]} = \frac{a + [l1]}{b - a + [l2]} $\\
\end{center}

Koniec rozwiązania.\\
\rozwStop
\odpStart
$ x \frac{a + [l1]}{b - a + [l2]} $\\
\odpStop
\testStart
A.$ \frac{a + [l1]}{b - a + [l2]} $\\
B.$ \frac{a + [l1]}{a - b + [l2]} $\\
C.$ \frac{b + [l1]}{b - a + [l2]} $\\
\testStop
\kluczStart
A
\kluczStop



\end{document}