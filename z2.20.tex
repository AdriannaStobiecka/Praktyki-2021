\documentclass[12pt, a4paper]{article}
\usepackage[utf8]{inputenc}
\usepackage{polski}

\usepackage{amsthm}  %pakiet do tworzenia twierdzeń itp.
\usepackage{amsmath} %pakiet do niektórych symboli matematycznych
\usepackage{amssymb} %pakiet do symboli mat., np. \nsubseteq
\usepackage{amsfonts}
\usepackage{graphicx} %obsługa plików graficznych z rozszerzeniem png, jpg
\theoremstyle{definition} %styl dla definicji
\newtheorem{zad}{} 
\title{Multizestaw zadań}
\author{Robert Fidytek}
%\date{\today}
\date{}
\newcounter{liczniksekcji}
\newcommand{\kategoria}[1]{\section{#1}} %olreślamy nazwę kateforii zadań
\newcommand{\zadStart}[1]{\begin{zad}#1\newline} %oznaczenie początku zadania
\newcommand{\zadStop}{\end{zad}}   %oznaczenie końca zadania
%Makra opcjonarne (nie muszą występować):
\newcommand{\rozwStart}[2]{\noindent \textbf{Rozwiązanie (autor #1 , recenzent #2): }\newline} %oznaczenie początku rozwiązania, opcjonarnie można wprowadzić informację o autorze rozwiązania zadania i recenzencie poprawności wykonania rozwiązania zadania
\newcommand{\rozwStop}{\newline}                                            %oznaczenie końca rozwiązania
\newcommand{\odpStart}{\noindent \textbf{Odpowiedź:}\newline}    %oznaczenie początku odpowiedzi końcowej (wypisanie wyniku)
\newcommand{\odpStop}{\newline}                                             %oznaczenie końca odpowiedzi końcowej (wypisanie wyniku)
\newcommand{\testStart}{\noindent \textbf{Test:}\newline} %ewentualne możliwe opcje odpowiedzi testowej: A. ? B. ? C. ? D. ? itd.
\newcommand{\testStop}{\newline} %koniec wprowadzania odpowiedzi testowych
\newcommand{\kluczStart}{\noindent \textbf{Test poprawna odpowiedź:}\newline} %klucz, poprawna odpowiedź pytania testowego (jedna literka): A lub B lub C lub D itd.
\newcommand{\kluczStop}{\newline} %koniec poprawnej odpowiedzi pytania testowego 
\newcommand{\wstawGrafike}[2]{\begin{figure}[h] \includegraphics[scale=#2] {#1} \end{figure}} %gdyby była potrzeba wstawienia obrazka, parametry: nazwa pliku, skala (jak nie wiesz co wpisać, to wpisz 1)

\begin{document}
\maketitle


\kategoria{Wikieł/Z2.20}
\zadStart{Zadanie z Wikieł Z 2.20 moja wersja nr [nrWersji]}
%[a1]:[1,2,3,4,5,6,7,8,9,10,11,12]
%[a2]:[1,2,3,4,5,6,7,8,9,10,11,12]
%[b1]=[a1]+2
%[b2]=[a2]+2
%[ab1]=[b1]-[a1]
%[ab2]=[b2]-[a2]
%[x]=[b1]+[a1]
%[y]=[b2]+[a2]
%[c1]=[x]/2
%[c2]=[y]/2
%[pc1]=int([c1])
%[pc2]=int([c2])
%[abc1]=[ab1]*[pc1]
%[abc2]=[ab2]*[pc2]
%[xy]=[abc2]+[abc1]
%[xy2]=[xy]/2
%[cxy2]=int([xy2])
%[c1].is_integer()==True and [c2].is_integer()==True and [ab1]>0 and [ab2]>0 
Podać równanie symetralnej odcinka $\overline{AB}$, gdzie A([a1],[a2]), B([b1],[b2]).
\zadStop
\rozwStart{Aleksandra Pasińska}{}
$$\overline{AB}, C\biggl(\frac{[b1]+[a1]}{2},\frac{[b2]+[a2]}{2}\biggr)\Rightarrow C([pc1],[pc2])$$
$$\vec n=\overrightarrow{AB}=[[b1]-[a1],[b2]-[a2]]=[[ab1],[ab2]]$$
$$[ab1](x-[pc1])+[ab2](y-[pc2])=0$$
$$[ab1]x-[abc1]+[ab2]y-[abc2]=0$$
$$[ab1]x+[ab2]y-[xy]=0$$
$$[ab2]y=-[ab1]x+[xy]$$
$$y=-x+[cxy2]$$
\rozwStop
\odpStart
$y=-x+[cxy2]$\\
\odpStop
\testStart
A.$y=-x+[cxy2]$
B.$[ab1]x-[xy]=0$
C.$[ab1]x-[ab2]y=7$
D.$[ab1]x-[ab2]y+[xy]=9$
E.$[ab2]y+[xy]=0$
F.$[ab1]x+[xy]=0$
G.$[ab1]x-[ab2]y=0$
H.$[ab1]x-[ab2]y+[xy]=4$
I.$[ab1]x-[ab2]y+[xy]=2$
\testStop
\kluczStart
A
\kluczStop



\end{document}