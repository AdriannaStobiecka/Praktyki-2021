\documentclass[12pt, a4paper]{article}
\usepackage[utf8]{inputenc}
\usepackage{polski}
\usepackage{amsthm}  %pakiet do tworzenia twierdzeń itp.
\usepackage{amsmath} %pakiet do niektórych symboli matematycznych
\usepackage{amssymb} %pakiet do symboli mat., np. \nsubseteq
\usepackage{amsfonts}
\usepackage{graphicx} %obsługa plików graficznych z rozszerzeniem png, jpg
\theoremstyle{definition} %styl dla definicji
\newtheorem{zad}{} 
\title{Multizestaw zadań}
\author{Robert Fidytek}
%\date{\today}
\date{}
\newcounter{liczniksekcji}
\newcommand{\kategoria}[1]{\section{#1}} %olreślamy nazwę kateforii zadań
\newcommand{\zadStart}[1]{\begin{zad}#1\newline} %oznaczenie początku zadania
\newcommand{\zadStop}{\end{zad}}   %oznaczenie końca zadania
%Makra opcjonarne (nie muszą występować):
\newcommand{\rozwStart}[2]{\noindent \textbf{Rozwiązanie (autor #1 , recenzent #2): }\newline} %oznaczenie początku rozwiązania, opcjonarnie można wprowadzić informację o autorze rozwiązania zadania i recenzencie poprawności wykonania rozwiązania zadania
\newcommand{\rozwStop}{\newline}                                            %oznaczenie końca rozwiązania
\newcommand{\odpStart}{\noindent \textbf{Odpowiedź:}\newline}    %oznaczenie początku odpowiedzi końcowej (wypisanie wyniku)
\newcommand{\odpStop}{\newline}                                             %oznaczenie końca odpowiedzi końcowej (wypisanie wyniku)
\newcommand{\testStart}{\noindent \textbf{Test:}\newline} %ewentualne możliwe opcje odpowiedzi testowej: A. ? B. ? C. ? D. ? itd.
\newcommand{\testStop}{\newline} %koniec wprowadzania odpowiedzi testowych
\newcommand{\kluczStart}{\noindent \textbf{Test poprawna odpowiedź:}\newline} %klucz, poprawna odpowiedź pytania testowego (jedna literka): A lub B lub C lub D itd.
\newcommand{\kluczStop}{\newline} %koniec poprawnej odpowiedzi pytania testowego 
\newcommand{\wstawGrafike}[2]{\begin{figure}[h] \includegraphics[scale=#2] {#1} \end{figure}} %gdyby była potrzeba wstawienia obrazka, parametry: nazwa pliku, skala (jak nie wiesz co wpisać, to wpisz 1)

\begin{document}
\maketitle


\kategoria{Wikieł/Z1.97c}
\zadStart{Zadanie z Wikieł Z1.97 c) moja wersja nr [nrWersji]}
%[a]:[2,3,4,5,6,7,8,9,10]
%[p]=int(3**[a])
%[d]=[p]+1
Rozwiąż nierównośc.\\
Podana nierówność $ \log_{\frac{1}{3}} (|x| - 1) > -[a]$\\
\zadStop
\rozwStart{Martyna Czarnobaj}{}
Dziedzina: $ |x|-1 > 0 \rightarrow |x|>1 \rightarrow x>1$ lub $x<-1 \rightarrow x \in (-\infty,-1)$ lub $(1,\infty) $\\
\begin{center}
	$ \log_{\frac{1}{3}} (|x| - 1) > -[a]$\\
	$ \log_{a} b = c \Leftrightarrow a^{c}=b$\\
	$ a = \frac{1}{3}, c = -[a] \rightarrow b = (\frac{1}{3})^{-[a]} = 3^{[a]} = 
	[p] $\\
	$ \log_{\frac{1}{3}} (|x| - 1) > \log_{\frac{1}{3}} [p], a < 1 $\\
	$ |x| - 1 < [p] $\\
	$ |x| < [d] $\\
	$ x < [d] $ i $ x > -[d] $\\
	$ x \in (-[d],[d]) $\\
	Biorąc dziedzinę oraz powyższy wynik otrzymujemy $ x \in (-[d], -1) $ lub $ x \in (1,[d]) $\\
\end{center}

Koniec rozwiązania.\\
\rozwStop
\odpStart
$ x \in (-[d], -1) $ lub $ x \in (1,[d])$ \\
\odpStop
\testStart
A.$ x \in (-[d], -1) $ lub $ x \in (1,[d])$ \\
B.$ x \in (-[d], 0) $ lub $ x \in (0,[d])$ \\
C.$ x \in (-[d], 1) $ lub $ x \in (-1,[d])$ \\
\testStop
\kluczStart
A
\kluczStop



\end{document}