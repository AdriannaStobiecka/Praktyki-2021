\documentclass[12pt, a4paper]{article}
\usepackage[utf8]{inputenc}
\usepackage{polski}

\usepackage{amsthm}  %pakiet do tworzenia twierdzeń itp.
\usepackage{amsmath} %pakiet do niektórych symboli matematycznych
\usepackage{amssymb} %pakiet do symboli mat., np. \nsubseteq
\usepackage{amsfonts}
\usepackage{graphicx} %obsługa plików graficznych z rozszerzeniem png, jpg
\theoremstyle{definition} %styl dla definicji
\newtheorem{zad}{} 
\title{Multizestaw zadań}
\author{Robert Fidytek}
%\date{\today}
\date{}
\newcounter{liczniksekcji}
\newcommand{\kategoria}[1]{\section{#1}} %olreślamy nazwę kateforii zadań
\newcommand{\zadStart}[1]{\begin{zad}#1\newline} %oznaczenie początku zadania
\newcommand{\zadStop}{\end{zad}}   %oznaczenie końca zadania
%Makra opcjonarne (nie muszą występować):
\newcommand{\rozwStart}[2]{\noindent \textbf{Rozwiązanie (autor #1 , recenzent #2): }\newline} %oznaczenie początku rozwiązania, opcjonarnie można wprowadzić informację o autorze rozwiązania zadania i recenzencie poprawności wykonania rozwiązania zadania
\newcommand{\rozwStop}{\newline}                                            %oznaczenie końca rozwiązania
\newcommand{\odpStart}{\noindent \textbf{Odpowiedź:}\newline}    %oznaczenie początku odpowiedzi końcowej (wypisanie wyniku)
\newcommand{\odpStop}{\newline}                                             %oznaczenie końca odpowiedzi końcowej (wypisanie wyniku)
\newcommand{\testStart}{\noindent \textbf{Test:}\newline} %ewentualne możliwe opcje odpowiedzi testowej: A. ? B. ? C. ? D. ? itd.
\newcommand{\testStop}{\newline} %koniec wprowadzania odpowiedzi testowych
\newcommand{\kluczStart}{\noindent \textbf{Test poprawna odpowiedź:}\newline} %klucz, poprawna odpowiedź pytania testowego (jedna literka): A lub B lub C lub D itd.
\newcommand{\kluczStop}{\newline} %koniec poprawnej odpowiedzi pytania testowego 
\newcommand{\wstawGrafike}[2]{\begin{figure}[h] \includegraphics[scale=#2] {#1} \end{figure}} %gdyby była potrzeba wstawienia obrazka, parametry: nazwa pliku, skala (jak nie wiesz co wpisać, to wpisz 1)

\begin{document}
\maketitle


\kategoria{Wikieł/Z3.12d}
\zadStart{Zadanie z Wikieł Z 3.12 d) moja wersja nr [nrWersji]}
%[a]:[2,3,4,5,6,7]
%[b]:[2,3,4,5,6,7]
%[c]:[2,3,4,5,6,7]
%[d]:[2,3,4,5,6,7]
%[2a]=2*[a]
%[2c]=2*[c]
%[e]=[a]+[2a]+[b]
%[f]=[2a]+[b]
%[g]=[c]+[2c]
%math.gcd([a],[c])==1 and math.gcd([b],[d])==1
Obliczyć granicę ciągu 
$$a_n=\frac{[a](n+2)!+[b](n+1)!}{[c](n+2)!-[d](n-1)!}.$$
\zadStop
\rozwStart{Adrianna Stobiecka}{}
$$\lim_{n\to\infty}\frac{[a](n+2)!+[b](n+1)!}{[c](n+2)!-[d](n-1)!}$$
$$=\lim_{n\to\infty}\frac{[a](n+2)(n+1)n(n-1)!+[b](n+1)n(n-1)!}{[c](n+2)(n+1)n(n-1)!-[d](n-1)!}$$
$$=\lim_{n\to\infty}\frac{(n-1)![[a](n+2)(n+1)n+[b](n+1)n]}{(n-1)![[c](n+2)(n+1)n-[d]]}$$
$$=\lim_{n\to\infty}\frac{[a](n+2)(n+1)n+[b](n+1)n}{[c](n+2)(n+1)n-[d]}$$
$$=\lim_{n\to\infty}\frac{[a](n^2+n+2n+2)n+[b](n^2+n)}{[c](n^2+n+2n+2)n-[d]}$$
$$=\lim_{n\to\infty}\frac{[a]n^3+[a]n^2+[2a]n^2+[2a]n+[b]n^2+[b]n}{[c]n^3+[c]n^2+[2c]n^2+[2c]n-[d]}$$
$$=\lim_{n\to\infty}\frac{[a]n^3+[e]n^2+[f]n}{[c]n^3+[g]n^2+[2c]n-[d]}$$
$$=\lim_{n\to\infty}\frac{[a]+\frac{[e]}{n}+\frac{[f]}{n^2}}{[c]+\frac{[g]}{n}+\frac{[2c]}{n^2}-\frac{[d]}{n^3}}=\frac{[a]}{[c]}$$
\rozwStop
\odpStart
 $\frac{[a]}{[c]}$
\odpStop
\testStart
A.$\infty$
B. $\frac{[c]}{[a]}$
C.$0$
D.$\frac{[a]}{[c]}$
E. $-\frac{[c]}{[a]}$
F.$-\frac{[a]}{[c]}$
G.$-\infty$
H.$-\frac{[b]}{[d]}$
I.$-\frac{[d]}{[b]}$
\testStop
\kluczStart
D
\kluczStop



\end{document}
