\documentclass[12pt, a4paper]{article}
\usepackage[utf8]{inputenc}
\usepackage{polski}
\usepackage{amsthm}  %pakiet do tworzenia twierdzeń itp.
\usepackage{amsmath} %pakiet do niektórych symboli matematycznych
\usepackage{amssymb} %pakiet do symboli mat., np. \nsubseteq
\usepackage{amsfonts}
\usepackage{graphicx} %obsługa plików graficznych z rozszerzeniem png, jpg
\theoremstyle{definition} %styl dla definicji
\newtheorem{zad}{} 
\title{Multizestaw zadań}
\author{Robert Fidytek}
%\date{\today}
\date{}
\newcounter{liczniksekcji}
\newcommand{\kategoria}[1]{\section{#1}} %olreślamy nazwę kateforii zadań
\newcommand{\zadStart}[1]{\begin{zad}#1\newline} %oznaczenie początku zadania
\newcommand{\zadStop}{\end{zad}}   %oznaczenie końca zadania
%Makra opcjonarne (nie muszą występować):
\newcommand{\rozwStart}[2]{\noindent \textbf{Rozwiązanie (autor #1 , recenzent #2): }\newline} %oznaczenie początku rozwiązania, opcjonarnie można wprowadzić informację o autorze rozwiązania zadania i recenzencie poprawności wykonania rozwiązania zadania
\newcommand{\rozwStop}{\newline}                                            %oznaczenie końca rozwiązania
\newcommand{\odpStart}{\noindent \textbf{Odpowiedź:}\newline}    %oznaczenie początku odpowiedzi końcowej (wypisanie wyniku)
\newcommand{\odpStop}{\newline}                                             %oznaczenie końca odpowiedzi końcowej (wypisanie wyniku)
\newcommand{\testStart}{\noindent \textbf{Test:}\newline} %ewentualne możliwe opcje odpowiedzi testowej: A. ? B. ? C. ? D. ? itd.
\newcommand{\testStop}{\newline} %koniec wprowadzania odpowiedzi testowych
\newcommand{\kluczStart}{\noindent \textbf{Test poprawna odpowiedź:}\newline} %klucz, poprawna odpowiedź pytania testowego (jedna literka): A lub B lub C lub D itd.
\newcommand{\kluczStop}{\newline} %koniec poprawnej odpowiedzi pytania testowego 
\newcommand{\wstawGrafike}[2]{\begin{figure}[h] \includegraphics[scale=#2] {#1} \end{figure}} %gdyby była potrzeba wstawienia obrazka, parametry: nazwa pliku, skala (jak nie wiesz co wpisać, to wpisz 1)

\begin{document}
\maketitle


\kategoria{Wikieł/Z1.91}
\zadStart{Zadanie z Wikieł Z1.91 moja wersja nr [nrWersji]}
%[x]:[6,7,8,9]
%[y]:[4,5,6,7]
%[s]=int([x]**(1/3))
%[l]=int(math.log([y],2))
%[w]=[l]-1
%[n]=[s]+1
%[n]%3!=0
Podać przykład liczby wymiernej i niewymiernej zawartje między liczbami $\sqrt[3]{[x]}$ i $ \log_{2} [y] $
\zadStop
\rozwStart{Martyna Czarnobaj}{}
Sprawdzamy z jakiego zakresu mamy podawać liczby.\\
$\sqrt[3]{[x]} $ wynosi około $ [s] $.\\
$ \log_{2} [y] $ wynosi około $ [l] $.\\
Więc mamy zakres $ ([s],[l]) $. Teraz z tego zakresu wybieramy liczbę wymierną i niewymierną.\\
Liczba wymierna - to taka liczba, którą można zapisać w postaci ułamka zwykłego, więc jako przykład możemy podać $ [w] $.\\
Liczba niewymierna - to taka liczba, której nie można zapisać za pomocą ułamka zwykłego, więc jako przykład możemy podać $ \sqrt[3]{[n]} $.\\
Koniec rozwiązania.\\
\rozwStop
\odpStart
 $ [w] $ i $ \sqrt[3]{[n]} $.\\
\odpStop
\testStart
A.$ [w] $ i $ \sqrt[3]{[n]} $.\\
B.$ 1 $ i $ 0 $.\\
C.$ 0 $ i $ 1 $.\\
\testStop
\kluczStart
A
\kluczStop



\end{document}