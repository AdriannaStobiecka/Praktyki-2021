\documentclass[12pt, a4paper]{article}
\usepackage[utf8]{inputenc}
\usepackage{polski}

\usepackage{amsthm}  %pakiet do tworzenia twierdzeń itp.
\usepackage{amsmath} %pakiet do niektórych symboli matematycznych
\usepackage{amssymb} %pakiet do symboli mat., np. \nsubseteq
\usepackage{amsfonts}
\usepackage{graphicx} %obsługa plików graficznych z rozszerzeniem png, jpg
\theoremstyle{definition} %styl dla definicji
\newtheorem{zad}{} 
\title{Multizestaw zadań}
\author{Robert Fidytek}
%\date{\today}
\date{}
\newcounter{liczniksekcji}
\newcommand{\kategoria}[1]{\section{#1}} %olreślamy nazwę kateforii zadań
\newcommand{\zadStart}[1]{\begin{zad}#1\newline} %oznaczenie początku zadania
\newcommand{\zadStop}{\end{zad}}   %oznaczenie końca zadania
%Makra opcjonarne (nie muszą występować):
\newcommand{\rozwStart}[2]{\noindent \textbf{Rozwiązanie (autor #1 , recenzent #2): }\newline} %oznaczenie początku rozwiązania, opcjonarnie można wprowadzić informację o autorze rozwiązania zadania i recenzencie poprawności wykonania rozwiązania zadania
\newcommand{\rozwStop}{\newline}                                            %oznaczenie końca rozwiązania
\newcommand{\odpStart}{\noindent \textbf{Odpowiedź:}\newline}    %oznaczenie początku odpowiedzi końcowej (wypisanie wyniku)
\newcommand{\odpStop}{\newline}                                             %oznaczenie końca odpowiedzi końcowej (wypisanie wyniku)
\newcommand{\testStart}{\noindent \textbf{Test:}\newline} %ewentualne możliwe opcje odpowiedzi testowej: A. ? B. ? C. ? D. ? itd.
\newcommand{\testStop}{\newline} %koniec wprowadzania odpowiedzi testowych
\newcommand{\kluczStart}{\noindent \textbf{Test poprawna odpowiedź:}\newline} %klucz, poprawna odpowiedź pytania testowego (jedna literka): A lub B lub C lub D itd.
\newcommand{\kluczStop}{\newline} %koniec poprawnej odpowiedzi pytania testowego 
\newcommand{\wstawGrafike}[2]{\begin{figure}[h] \includegraphics[scale=#2] {#1} \end{figure}} %gdyby była potrzeba wstawienia obrazka, parametry: nazwa pliku, skala (jak nie wiesz co wpisać, to wpisz 1)

\begin{document}
\maketitle


\kategoria{Wikieł/Z1.36k}
\zadStart{Zadanie z Wikieł Z 1.36k) moja wersja nr [nrWersji]}
%[f]:[1,2,3,4,5,9,10,11,12,13,14,15,16,17,18,19,20]
%[c]:[1,2,4,5,6,7,8,9,10,11,12,17,18,19,20,21,22]
%[b]=random.randint(3,50)
%[a]=random.randint(1,[b]-1)
%[n]=random.randint(4,15)
%[r]=[b]-[a]
%[s]=int((2*[a]+([n]-1)*[r])*[n]/2)
%[d]=[b]+[r]
%[e]=[a]-[r]
%[g]=[a] + [e]
%[g2]=-[g]
%[h]=2*[s]
%[de]=[g]*[g] +4*[r]*[h]
%[dep]=int(math.sqrt(abs([de])))
%[n2]=int((-[g]+[dep])/(2*abs([r])))
%[o]=2*[n2]
%[p]=int(([g2]+[dep])/2)
%[p2]=[r]
%[p3]=[p]/[p2]
%[p4]=int([p3])
%[a]<[b] and [r]>[a] and [dep]-(math.sqrt([de]))==0 and [dep]>[g] and [n2]-((-[g]+[dep])/(2*[r]))==0 and [f]!=[n2] and [g]<0 and [s]-(2*[a]+([n]-1)*[r])*[n]/2==0 and [p3].is_integer()


Rozwiąż równanie $[r] x^4 + ([g]) x^2- [h]=0$.
\zadStop
\rozwStart{Joanna Świerzbin}{}

$$[r] x^4 + ([g]) x^2- [h]=0$$
$$ t=x^2 \land t \geq 0 $$
$$[r]t^{2}+([g])t-[h]=0$$
$$\Delta= {([g])}^2 + 4\cdot[r]\cdot[h]= [de], \hspace{0.2 cm} \sqrt{\Delta}=[dep]$$
$$\left(t_1= \frac{[g2]-[dep]}{2\cdot[r]}, \hspace{0.2 cm} t_2=\frac{[g2]+[dep]}{2\cdot[r]}\right) \land t \geq 0 $$
$$  t=\frac{[p]}{[p2]}$$
$$ x^2=\frac{[p]}{[p2]}$$
$$ x_1=\sqrt{\frac{[p]}{[p2]}} \vee  x_2=-\sqrt{\frac{[p]}{[p2]}}$$
$$ x_1=\sqrt{[p4]} \vee  x_2=-\sqrt{[p4]}$$
\rozwStop
\odpStart
$ x_1=\sqrt{[p4]} \vee  x_2=-\sqrt{[p4]}$
\odpStop
\testStart
A. $x_1=\sqrt{[p4]} \vee  x_2=-\sqrt{[p4]}$\\
B. $x_1=\sqrt{[p4]} $\\
C. $x_2=-\sqrt{[p4]}$\\
D. $ x_1=\sqrt{[p2]} \vee  x_2=-\sqrt{[p2]}$\\
E. $ x_1=\sqrt{[p]} \vee  x_2=-\sqrt{[p]}$ \\
F. $ x_1=\sqrt{[p2]} \vee  x_2=-\sqrt{[p]}$
\testStop
\kluczStart
A
\kluczStop



\end{document}