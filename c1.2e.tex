\documentclass[12pt, a4paper]{article}
\usepackage[utf8]{inputenc}
\usepackage{polski}

\usepackage{amsthm}  %pakiet do tworzenia twierdzeń itp.
\usepackage{amsmath} %pakiet do niektórych symboli matematycznych
\usepackage{amssymb} %pakiet do symboli mat., np. \nsubseteq
\usepackage{amsfonts}
\usepackage{graphicx} %obsługa plików graficznych z rozszerzeniem png, jpg
\theoremstyle{definition} %styl dla definicji
\newtheorem{zad}{} 
\title{Multizestaw zadań}
\author{Robert Fidytek}
%\date{\today}
\date{}
\newcounter{liczniksekcji}
\newcommand{\kategoria}[1]{\section{#1}} %olreślamy nazwę kateforii zadań
\newcommand{\zadStart}[1]{\begin{zad}#1\newline} %oznaczenie początku zadania
\newcommand{\zadStop}{\end{zad}}   %oznaczenie końca zadania
%Makra opcjonarne (nie muszą występować):
\newcommand{\rozwStart}[2]{\noindent \textbf{Rozwiązanie (autor #1 , recenzent #2): }\newline} %oznaczenie początku rozwiązania, opcjonarnie można wprowadzić informację o autorze rozwiązania zadania i recenzencie poprawności wykonania rozwiązania zadania
\newcommand{\rozwStop}{\newline}                                            %oznaczenie końca rozwiązania
\newcommand{\odpStart}{\noindent \textbf{Odpowiedź:}\newline}    %oznaczenie początku odpowiedzi końcowej (wypisanie wyniku)
\newcommand{\odpStop}{\newline}                                             %oznaczenie końca odpowiedzi końcowej (wypisanie wyniku)
\newcommand{\testStart}{\noindent \textbf{Test:}\newline} %ewentualne możliwe opcje odpowiedzi testowej: A. ? B. ? C. ? D. ? itd.
\newcommand{\testStop}{\newline} %koniec wprowadzania odpowiedzi testowych
\newcommand{\kluczStart}{\noindent \textbf{Test poprawna odpowiedź:}\newline} %klucz, poprawna odpowiedź pytania testowego (jedna literka): A lub B lub C lub D itd.
\newcommand{\kluczStop}{\newline} %koniec poprawnej odpowiedzi pytania testowego 
\newcommand{\wstawGrafike}[2]{\begin{figure}[h] \includegraphics[scale=#2] {#1} \end{figure}} %gdyby była potrzeba wstawienia obrazka, parametry: nazwa pliku, skala (jak nie wiesz co wpisać, to wpisz 1)

\begin{document}
\maketitle
\kategoria{Wikieł/C1.2e}
\zadStart{Zadanie z Wikieł C 1.2e moja wersja nr [nrWersji]}
%[a]:[2,3,4,5,6,7,8,9,10,11,12,13,14,15]
%[b]:[2,3,4,5,6,7,8,9,10,11,12,13,14,15]
%[c]=[b]-[a]
%[a]<[b] and [c]!=1 and [c]!=[a]
Oblicz całkę $$\int ([a]ctg^{2}(x)+[b]) dx.$$
\zadStop
\rozwStart{Justyna Chojecka}{}
Korzystamy ze wzorów trygonometrycznych.
$$\int ([a]ctg^{2}(x)+[b]) dx=\int \left([a]\cdot\frac{cos^{2}(x)}{sin^{2}(x)}+[b]\right)dx=\int \left([a]\cdot\frac{1-sin^{2}(x)}{sin^{2}(x)}+[b]\right)dx$$$$=[a]\left(\int\frac{1}{sin^{2}(x)}dx-\int 1dx\right)+\int [b]dx=[a]\left(ctg(x)-x\right)+[b]x+C$$$$=[a]ctg(x)-[a]x+[b]x+C=[a]ctg(x)+[c]x+C$$
\rozwStop
\odpStart
$[a]ctg(x)+[c]x+C$
\odpStop
\testStart
A.$[a]ctg(x)+[c]x+C$\\
B.$[a]ctg(x)-[c]x+C$\\
C.$[a](ctg(x)+x)+C$\\
D.$[c](ctg(x)+x)+C$\\
E.$-[a]ctg(x)+[c]x+C$\\
F.$[c]ctg(x)+[a]x+C$\\
G.$[a](ctg(x)-x)+C$\\
H.$-[a]ctg(x)-[c]x+C$\\
I.$[c](ctg(x)-x)+C$\\
\testStop
\kluczStart
A
\kluczStop



\end{document}