\documentclass[12pt, a4paper]{article}
\usepackage[utf8]{inputenc}
\usepackage{polski}

\usepackage{amsthm}  %pakiet do tworzenia twierdzeń itp.
\usepackage{amsmath} %pakiet do niektórych symboli matematycznych
\usepackage{amssymb} %pakiet do symboli mat., np. \nsubseteq
\usepackage{amsfonts}
\usepackage{graphicx} %obsługa plików graficznych z rozszerzeniem png, jpg
\theoremstyle{definition} %styl dla definicji
\newtheorem{zad}{} 
\title{Multizestaw zadań}
\author{Robert Fidytek}
%\date{\today}
\date{}
\newcounter{liczniksekcji}
\newcommand{\kategoria}[1]{\section{#1}} %olreślamy nazwę kateforii zadań
\newcommand{\zadStart}[1]{\begin{zad}#1\newline} %oznaczenie początku zadania
\newcommand{\zadStop}{\end{zad}}   %oznaczenie końca zadania
%Makra opcjonarne (nie muszą występować):
\newcommand{\rozwStart}[2]{\noindent \textbf{Rozwiązanie (autor #1 , recenzent #2): }\newline} %oznaczenie początku rozwiązania, opcjonarnie można wprowadzić informację o autorze rozwiązania zadania i recenzencie poprawności wykonania rozwiązania zadania
\newcommand{\rozwStop}{\newline}                                            %oznaczenie końca rozwiązania
\newcommand{\odpStart}{\noindent \textbf{Odpowiedź:}\newline}    %oznaczenie początku odpowiedzi końcowej (wypisanie wyniku)
\newcommand{\odpStop}{\newline}                                             %oznaczenie końca odpowiedzi końcowej (wypisanie wyniku)
\newcommand{\testStart}{\noindent \textbf{Test:}\newline} %ewentualne możliwe opcje odpowiedzi testowej: A. ? B. ? C. ? D. ? itd.
\newcommand{\testStop}{\newline} %koniec wprowadzania odpowiedzi testowych
\newcommand{\kluczStart}{\noindent \textbf{Test poprawna odpowiedź:}\newline} %klucz, poprawna odpowiedź pytania testowego (jedna literka): A lub B lub C lub D itd.
\newcommand{\kluczStop}{\newline} %koniec poprawnej odpowiedzi pytania testowego 
\newcommand{\wstawGrafike}[2]{\begin{figure}[h] \includegraphics[scale=#2] {#1} \end{figure}} %gdyby była potrzeba wstawienia obrazka, parametry: nazwa pliku, skala (jak nie wiesz co wpisać, to wpisz 1)

\begin{document}
\maketitle


\kategoria{Wikieł/Z3.21b}
\zadStart{Zadanie z Wikieł Z 3.21 b) moja wersja nr [nrWersji]}
%[a]:[2,3,4,5,6,7,8,9]
%[b]:[2,3,4,5,6,7,8,9]
%[c]:[2,3,4,5,6,7,8,9]
%[d]=math.gcd([a],99)
%[a2]=int([a]/[d])
%[m]=int((99)/[d])
%[mb]=[m]*[b]
%[a2c]=[c]*[a2]
%[d2]=math.gcd([mb],[a2c])
%[mb2]=int([mb]/[d2])
%[a2c2]=int([a2c]/[d2])
%[calosci]=[mb2]//[a2c2]
%[reszta]=[mb2]%[a2c2]
%math.gcd([b],[c])==1 and [a2]!=1 and [a2c2]!=1 and [m]!=1 and [mb2]!=1
Rozwiązać równanie, którego lewa strona jest sumą nieskończonego ciągu geometrycznego.
$$0,0[a]x+0,000[a]x+0,00000[a]x+\dots=\frac{[b]}{[c]}$$
\zadStop
\rozwStart{Adrianna Stobiecka}{}
Zapiszemy powyższe równanie w inny sposób.
$$\frac{[a]x}{100}+\frac{[a]x}{10000}+\frac{[a]x}{1000000}+\dots=\frac{[b]}{[c]}$$
Lewa strona równania jest sumą nieskończonego ciągu geometrycznego o pierwszym wyrazie $a_1=\frac{[a]x}{100}$ oraz ilorazie $q=\frac{1}{100}$. Widzimy, że 
$$|q|=\bigg|\frac{1}{100}\bigg|=\frac{1}{100}<1.$$
Mamy zatem:
$$S=\frac{a_1}{1-q}=\frac{\frac{[a]x}{100}}{1-\frac{1}{100}}=\frac{[a]x}{100}\cdot\frac{100}{99}=\frac{[a2]x}{[m]}$$
Obliczoną sumę wstawiamy do równania.
$$\frac{[a2]x}{[m]}=\frac{[b]}{[c]}\Leftrightarrow [c]\cdot[a2]x=[m]\cdot[b]\Leftrightarrow[a2c]x=[mb]\Leftrightarrow x=\frac{[mb2]}{[a2c2]}\Leftrightarrow x=[calosci]\frac{[reszta]}{[a2c2]}$$
Rozwiązaniem równania jest $x=[calosci]\frac{[reszta]}{[a2c2]}$.
\rozwStop
\odpStart
$x=[calosci]\frac{[reszta]}{[a2c2]}$
\odpStop
\testStart
A.$x=-[calosci]$
B.$x=[calosci]\frac{[reszta]}{[a2c2]}$
C.$x=-1$
D.$x=[calosci]$
E.$x=0$
F.$x=-\frac{[reszta]}{[a2c2]}$
G.$x=\frac{[reszta]}{[a2c2]}$
H.$x=1$
I.$x=-[calosci]\frac{[reszta]}{[a2c2]}$
\testStop
\kluczStart
B
\kluczStop



\end{document}
