\documentclass[12pt, a4paper]{article}
\usepackage[utf8]{inputenc}
\usepackage{polski}
\usepackage{amsthm}  %pakiet do tworzenia twierdzeń itp.
\usepackage{amsmath} %pakiet do niektórych symboli matematycznych
\usepackage{amssymb} %pakiet do symboli mat., np. \nsubseteq
\usepackage{amsfonts}
\usepackage{graphicx} %obsługa plików graficznych z rozszerzeniem png, jpg
\theoremstyle{definition} %styl dla definicji
\newtheorem{zad}{} 
\title{Multizestaw zadań}
\author{Patryk Wirkus}
%\date{\today}
\date{}
\newcommand{\kategoria}[1]{\section{#1}}
\newcommand{\zadStart}[1]{\begin{zad}#1\newline}
\newcommand{\zadStop}{\end{zad}}
\newcommand{\rozwStart}[2]{\noindent \textbf{Rozwiązanie (autor #1 , recenzent #2): }\newline}
\newcommand{\rozwStop}{\newline}                                           
\newcommand{\odpStart}{\noindent \textbf{Odpowiedź:}\newline}
\newcommand{\odpStop}{\newline}
\newcommand{\testStart}{\noindent \textbf{Test:}\newline}
\newcommand{\testStop}{\newline}
\newcommand{\kluczStart}{\noindent \textbf{Test poprawna odpowiedź:}\newline}
\newcommand{\kluczStop}{\newline}
\newcommand{\wstawGrafike}[2]{\begin{figure}[h] \includegraphics[scale=#2] {#1} \end{figure}}

\begin{document}
\maketitle

\kategoria{Wikieł/Z1.16e}


\zadStart{Zadanie z Wikieł Z 1.16 e) moja wersja nr 1}

Obliczyć symbol Newtona ${63 \choose 2}$.
\zadStop
\rozwStart{Patryk Wirkus}{}
$${63 \choose 2} = \frac{63!}{(63-2)! \cdot 2!} = \frac{(63-1) \cdot 63}{2} = \frac{3906}{2}$$
\rozwStop
\odpStart
$\frac{3906}{2}$
\odpStop
\testStart
A.$\frac{3906}{2}$ B.$-\frac{3906}{2}$ C.$0$ D.$\frac{2}{3906}$ E.$-\frac{2}{3906}$
\testStop
\kluczStart
A
\kluczStop



\zadStart{Zadanie z Wikieł Z 1.16 e) moja wersja nr 2}

Obliczyć symbol Newtona ${67 \choose 2}$.
\zadStop
\rozwStart{Patryk Wirkus}{}
$${67 \choose 2} = \frac{67!}{(67-2)! \cdot 2!} = \frac{(67-1) \cdot 67}{2} = \frac{4422}{2}$$
\rozwStop
\odpStart
$\frac{4422}{2}$
\odpStop
\testStart
A.$\frac{4422}{2}$ B.$-\frac{4422}{2}$ C.$0$ D.$\frac{2}{4422}$ E.$-\frac{2}{4422}$
\testStop
\kluczStart
A
\kluczStop



\zadStart{Zadanie z Wikieł Z 1.16 e) moja wersja nr 3}

Obliczyć symbol Newtona ${73 \choose 2}$.
\zadStop
\rozwStart{Patryk Wirkus}{}
$${73 \choose 2} = \frac{73!}{(73-2)! \cdot 2!} = \frac{(73-1) \cdot 73}{2} = \frac{5256}{2}$$
\rozwStop
\odpStart
$\frac{5256}{2}$
\odpStop
\testStart
A.$\frac{5256}{2}$ B.$-\frac{5256}{2}$ C.$0$ D.$\frac{2}{5256}$ E.$-\frac{2}{5256}$
\testStop
\kluczStart
A
\kluczStop



\zadStart{Zadanie z Wikieł Z 1.16 e) moja wersja nr 4}

Obliczyć symbol Newtona ${79 \choose 2}$.
\zadStop
\rozwStart{Patryk Wirkus}{}
$${79 \choose 2} = \frac{79!}{(79-2)! \cdot 2!} = \frac{(79-1) \cdot 79}{2} = \frac{6162}{2}$$
\rozwStop
\odpStart
$\frac{6162}{2}$
\odpStop
\testStart
A.$\frac{6162}{2}$ B.$-\frac{6162}{2}$ C.$0$ D.$\frac{2}{6162}$ E.$-\frac{2}{6162}$
\testStop
\kluczStart
A
\kluczStop



\zadStart{Zadanie z Wikieł Z 1.16 e) moja wersja nr 5}

Obliczyć symbol Newtona ${51 \choose 2}$.
\zadStop
\rozwStart{Patryk Wirkus}{}
$${51 \choose 2} = \frac{51!}{(51-2)! \cdot 2!} = \frac{(51-1) \cdot 51}{2} = \frac{2550}{2}$$
\rozwStop
\odpStart
$\frac{2550}{2}$
\odpStop
\testStart
A.$\frac{2550}{2}$ B.$-\frac{2550}{2}$ C.$0$ D.$\frac{2}{2550}$ E.$-\frac{2}{2550}$
\testStop
\kluczStart
A
\kluczStop



\zadStart{Zadanie z Wikieł Z 1.16 e) moja wersja nr 6}

Obliczyć symbol Newtona ${57 \choose 2}$.
\zadStop
\rozwStart{Patryk Wirkus}{}
$${57 \choose 2} = \frac{57!}{(57-2)! \cdot 2!} = \frac{(57-1) \cdot 57}{2} = \frac{3192}{2}$$
\rozwStop
\odpStart
$\frac{3192}{2}$
\odpStop
\testStart
A.$\frac{3192}{2}$ B.$-\frac{3192}{2}$ C.$0$ D.$\frac{2}{3192}$ E.$-\frac{2}{3192}$
\testStop
\kluczStart
A
\kluczStop



\zadStart{Zadanie z Wikieł Z 1.16 e) moja wersja nr 7}

Obliczyć symbol Newtona ${61 \choose 2}$.
\zadStop
\rozwStart{Patryk Wirkus}{}
$${61 \choose 2} = \frac{61!}{(61-2)! \cdot 2!} = \frac{(61-1) \cdot 61}{2} = \frac{3660}{2}$$
\rozwStop
\odpStart
$\frac{3660}{2}$
\odpStop
\testStart
A.$\frac{3660}{2}$ B.$-\frac{3660}{2}$ C.$0$ D.$\frac{2}{3660}$ E.$-\frac{2}{3660}$
\testStop
\kluczStart
A
\kluczStop



\zadStart{Zadanie z Wikieł Z 1.16 e) moja wersja nr 8}

Obliczyć symbol Newtona ${69 \choose 2}$.
\zadStop
\rozwStart{Patryk Wirkus}{}
$${69 \choose 2} = \frac{69!}{(69-2)! \cdot 2!} = \frac{(69-1) \cdot 69}{2} = \frac{4692}{2}$$
\rozwStop
\odpStart
$\frac{4692}{2}$
\odpStop
\testStart
A.$\frac{4692}{2}$ B.$-\frac{4692}{2}$ C.$0$ D.$\frac{2}{4692}$ E.$-\frac{2}{4692}$
\testStop
\kluczStart
A
\kluczStop



\zadStart{Zadanie z Wikieł Z 1.16 e) moja wersja nr 9}

Obliczyć symbol Newtona ${71 \choose 2}$.
\zadStop
\rozwStart{Patryk Wirkus}{}
$${71 \choose 2} = \frac{71!}{(71-2)! \cdot 2!} = \frac{(71-1) \cdot 71}{2} = \frac{4970}{2}$$
\rozwStop
\odpStart
$\frac{4970}{2}$
\odpStop
\testStart
A.$\frac{4970}{2}$ B.$-\frac{4970}{2}$ C.$0$ D.$\frac{2}{4970}$ E.$-\frac{2}{4970}$
\testStop
\kluczStart
A
\kluczStop



\zadStart{Zadanie z Wikieł Z 1.16 e) moja wersja nr 10}

Obliczyć symbol Newtona ${77 \choose 2}$.
\zadStop
\rozwStart{Patryk Wirkus}{}
$${77 \choose 2} = \frac{77!}{(77-2)! \cdot 2!} = \frac{(77-1) \cdot 77}{2} = \frac{5852}{2}$$
\rozwStop
\odpStart
$\frac{5852}{2}$
\odpStop
\testStart
A.$\frac{5852}{2}$ B.$-\frac{5852}{2}$ C.$0$ D.$\frac{2}{5852}$ E.$-\frac{2}{5852}$
\testStop
\kluczStart
A
\kluczStop





\end{document}
