\documentclass[12pt, a4paper]{article}
\usepackage[utf8]{inputenc}
\usepackage{polski}

\usepackage{amsthm}  %pakiet do tworzenia twierdzeń itp.
\usepackage{amsmath} %pakiet do niektórych symboli matematycznych
\usepackage{amssymb} %pakiet do symboli mat., np. \nsubseteq
\usepackage{amsfonts}
\usepackage{graphicx} %obsługa plików graficznych z rozszerzeniem png, jpg
\theoremstyle{definition} %styl dla definicji
\newtheorem{zad}{} 
\title{Multizestaw zadań}
\author{Robert Fidytek}
%\date{\today}
\date{}
\newcounter{liczniksekcji}
\newcommand{\kategoria}[1]{\section{#1}} %olreślamy nazwę kateforii zadań
\newcommand{\zadStart}[1]{\begin{zad}#1\newline} %oznaczenie początku zadania
\newcommand{\zadStop}{\end{zad}}   %oznaczenie końca zadania
%Makra opcjonarne (nie muszą występować):
\newcommand{\rozwStart}[2]{\noindent \textbf{Rozwiązanie (autor #1 , recenzent #2): }\newline} %oznaczenie początku rozwiązania, opcjonarnie można wprowadzić informację o autorze rozwiązania zadania i recenzencie poprawności wykonania rozwiązania zadania
\newcommand{\rozwStop}{\newline}                                            %oznaczenie końca rozwiązania
\newcommand{\odpStart}{\noindent \textbf{Odpowiedź:}\newline}    %oznaczenie początku odpowiedzi końcowej (wypisanie wyniku)
\newcommand{\odpStop}{\newline}                                             %oznaczenie końca odpowiedzi końcowej (wypisanie wyniku)
\newcommand{\testStart}{\noindent \textbf{Test:}\newline} %ewentualne możliwe opcje odpowiedzi testowej: A. ? B. ? C. ? D. ? itd.
\newcommand{\testStop}{\newline} %koniec wprowadzania odpowiedzi testowych
\newcommand{\kluczStart}{\noindent \textbf{Test poprawna odpowiedź:}\newline} %klucz, poprawna odpowiedź pytania testowego (jedna literka): A lub B lub C lub D itd.
\newcommand{\kluczStop}{\newline} %koniec poprawnej odpowiedzi pytania testowego 
\newcommand{\wstawGrafike}[2]{\begin{figure}[h] \includegraphics[scale=#2] {#1} \end{figure}} %gdyby była potrzeba wstawienia obrazka, parametry: nazwa pliku, skala (jak nie wiesz co wpisać, to wpisz 1)

\begin{document}
\maketitle


\kategoria{Wikieł/P3.20a}
\zadStart{Zadanie z Wikieł P 3.20 a) moja wersja nr [nrWersji]}
%[p1]:[4,5,6,7,8,9,10,11,12]
%[p2]:[2,3,4,5,6,7,8,9,10,11,12]
%[p3]:[2,3,4,5,6,7,8,9,10,11,12]
%[a]=random.randint(2,10)
%[e]=random.randint(2,10)
%[c]=random.randint(1,10)
%[d]=random.randint(2,10)
%[b]=random.randint(2,10)
%[f]=random.randint(1,10)
%[p1p2m]=[p1]-[p2]
%[p1p3m]=[p1]-[p3]
%[p1]>[p2] and [p1]>[p3] and [p2]!=[p3] and math.gcd([a],[d])==1 and [p1p2m]>1 and [p1p3m]>1
Obliczyć granicę ciągu $a_{n}=\frac{[a]n^{[p1]}-[b]n^{[p2]}+[c]}{[d]n^{[p1]}+[e]n^{[p3]}-[f]}$.
\zadStop
\rozwStart{Robert Fidytek}{}
$$\lim\limits_{n\to\infty}\frac{[a]n^{[p1]}-[b]n^{[p2]}+[c]}{[d]n^{[p1]}+[e]n^{[p3]}-[f]}=$$ 
$$=\lim\limits_{n\to\infty}\frac{n^{[p1]}\left(\frac{[a]n^{[p1]}}{n^{[p1]}}-\frac{[b]n^{[p2]}}{n^{[p1]}}+\frac{[c]}{n^{[p1]}}\right)}{n^{[p1]}\left(\frac{[d]n^{[p1]}}{n^{[p1]}}+\frac{[e]n^{[p3]}}{n^{[p1]}}-\frac{[f]}{n^{[p1]}}\right)}=$$ 
$$=\lim\limits_{n\to\infty}\frac{n^{[p1]}\left([a]-\frac{[b]}{n^{[p1p2m]}}+\frac{[c]}{n^{[p1]}}\right)}
{n^{[p1]}\left([d]+\frac{[e]}{n^{[p1p3m]}}-\frac{[f]}{n^{[p1]}}\right)}=$$ 
$$=\lim\limits_{n\to\infty}\frac{[a]-\frac{[b]}{n^{[p1p2m]}}+\frac{[c]}{n^{[p1]}}}{[d]+\frac{[e]}{n^{[p1p3m]}}-\frac{[f]}{n^{[p1]}}}=\frac{[a]}{[d]}$$
\rozwStop
\odpStart
$\frac{[a]}{[d]}$
\odpStop
\testStart
A.$\frac{[a]}{[d]}$
B.$0$
C.$\infty$
D.$\frac{[d]}{[a]}$
E.$\frac{1}{[a]}$
F.$\frac{1}{[d]}$
G.$[d]$
H.$[a]$
I.$-\infty$
\testStop
\kluczStart
A
\kluczStop



\end{document}