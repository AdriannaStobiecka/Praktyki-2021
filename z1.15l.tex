\documentclass[12pt, a4paper]{article}
\usepackage[utf8]{inputenc}
\usepackage{polski}

\usepackage{amsthm}  %pakiet do tworzenia twierdzeń itp.
\usepackage{amsmath} %pakiet do niektórych symboli matematycznych
\usepackage{amssymb} %pakiet do symboli mat., np. \nsubseteq
\usepackage{amsfonts}
\usepackage{graphicx} %obsługa plików graficznych z rozszerzeniem png, jpg
\theoremstyle{definition} %styl dla definicji
\newtheorem{zad}{} 
\title{Multizestaw zadań}
\author{Robert Fidytek}
%\date{\today}
\date{}
\newcounter{liczniksekcji}
\newcommand{\kategoria}[1]{\section{#1}} %olreślamy nazwę kateforii zadań
\newcommand{\zadStart}[1]{\begin{zad}#1\newline} %oznaczenie początku zadania
\newcommand{\zadStop}{\end{zad}}   %oznaczenie końca zadania
%Makra opcjonarne (nie muszą występować):
\newcommand{\rozwStart}[2]{\noindent \textbf{Rozwiązanie (autor #1 , recenzent #2): }\newline} %oznaczenie początku rozwiązania, opcjonarnie można wprowadzić informację o autorze rozwiązania zadania i recenzencie poprawności wykonania rozwiązania zadania
\newcommand{\rozwStop}{\newline}                                            %oznaczenie końca rozwiązania
\newcommand{\odpStart}{\noindent \textbf{Odpowiedź:}\newline}    %oznaczenie początku odpowiedzi końcowej (wypisanie wyniku)
\newcommand{\odpStop}{\newline}                                             %oznaczenie końca odpowiedzi końcowej (wypisanie wyniku)
\newcommand{\testStart}{\noindent \textbf{Test:}\newline} %ewentualne możliwe opcje odpowiedzi testowej: A. ? B. ? C. ? D. ? itd.
\newcommand{\testStop}{\newline} %koniec wprowadzania odpowiedzi testowych
\newcommand{\kluczStart}{\noindent \textbf{Test poprawna odpowiedź:}\newline} %klucz, poprawna odpowiedź pytania testowego (jedna literka): A lub B lub C lub D itd.
\newcommand{\kluczStop}{\newline} %koniec poprawnej odpowiedzi pytania testowego 
\newcommand{\wstawGrafike}[2]{\begin{figure}[h] \includegraphics[scale=#2] {#1} \end{figure}} %gdyby była potrzeba wstawienia obrazka, parametry: nazwa pliku, skala (jak nie wiesz co wpisać, to wpisz 1)

\begin{document}
\maketitle



\kategoria{Wikieł/Z1.15l}
\zadStart{Zadanie z Wikieł Z 1.15 l) moja wersja nr [nrWersji]}
%[a]:[2,3,4,5,6,7,8]
%[b]:[2,3,4,5,6,7,8]
%[a]=random.randint(2,8)
%[b]=random.randint(2,8)
%[bpa]=round((-[b]/[a]),2)
%[2a]=2*[a]
%[bp2a]=round([b]/[2a],2)
Rozwiązać nierówność $\big|\frac{[a]x}{[a]x+[b]}\big|>1$
\zadStop
\rozwStart{Pascal Nawrocki}{Jakub Ulrych}
Wyznaczamy dziedzinę: $x\in\mathbb{R}\symbol{92}\{[bpa]\}$
Korzystamy z własności wartości bezwględnej:
$$\big|\frac{[a]x}{[a]x+[b]}\big|>1$$
$$\frac{|[a]x|}{|[a]x+[b]|}>1$$
$$|[a]x|>|[a]x+[b]|$$
$$|[a]x|-|[a]x+[b]|>0$$
Rozpatrujemy 3 przypadki i bierzemy sumę rozwiązań:
\begin{enumerate}
\item (obie wartości bezwględne z przeciwnym znakiem) $$x\in(-\infty,[bpa])$$ 
$$|[a]x|-|[a]x+[b]|>0$$
$$-[a]x-(-[a]x-[b])>0$$
$$-[a]x+[a]x+[b]>0$$
$$[b]>0$$
Rozwiązaniem jest cały przedział, na którym operujemy, czyli:
$$x\in(-\infty,[bpa])$$
\item (druga bez zmiany pierwsza z przeciwnym)$$x\in[[bpa],0)$$ 
$$|[a]x|-|[a]x+[b]|>0$$
$$-[a]x-([a]x+[b])>0$$
$$-[2a]x-[b]>0$$
$$-[2a]x>[b]$$
$$x<-[bp2a]$$
Biorąc pod uwagę przedział w jakim się znajdujemy mamy:
$$x\in[[bpa],-[bp2a])$$
\item (obie bez zmiany) $$x\in[0,\infty)$$ 
$$|[a]x|-|[a]x+[b]|>0$$
$$[a]x-([a]x+[b])>0$$
$$-[b]>0\Rightarrow x\in\emptyset$$
\end{enumerate}
Podsumowując, rozwiązaniem jest suma tych 3 przypadków (pamiętamy też o dziedzine), zatem: $$x\in(-\infty,-[bp2a])\symbol{92}\{[bpa]\}$$
\odpStop
\testStart
A.$x\in(-\infty,-[bp2a])\symbol{92}\{[bpa]\}$
B.$x\in\mathbb{R}$
C.$x\in(-\infty,[bp2a]]$
D.$x\in\emptyset$
\testStop
\kluczStart
A
\kluczStop


\end{document}