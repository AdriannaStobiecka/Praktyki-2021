\documentclass[12pt, a4paper]{article}
\usepackage[utf8]{inputenc}
\usepackage{polski}

\usepackage{amsthm}  %pakiet do tworzenia twierdzeń itp.
\usepackage{amsmath} %pakiet do niektórych symboli matematycznych
\usepackage{amssymb} %pakiet do symboli mat., np. \nsubseteq
\usepackage{amsfonts}
\usepackage{graphicx} %obsługa plików graficznych z rozszerzeniem png, jpg
\theoremstyle{definition} %styl dla definicji
\newtheorem{zad}{} 
\title{Multizestaw zadań}
\author{Robert Fidytek}
%\date{\today}
\date{}
\newcounter{liczniksekcji}
\newcommand{\kategoria}[1]{\section{#1}} %olreślamy nazwę kateforii zadań
\newcommand{\zadStart}[1]{\begin{zad}#1\newline} %oznaczenie początku zadania
\newcommand{\zadStop}{\end{zad}}   %oznaczenie końca zadania
%Makra opcjonarne (nie muszą występować):
\newcommand{\rozwStart}[2]{\noindent \textbf{Rozwiązanie (autor #1 , recenzent #2): }\newline} %oznaczenie początku rozwiązania, opcjonarnie można wprowadzić informację o autorze rozwiązania zadania i recenzencie poprawności wykonania rozwiązania zadania
\newcommand{\rozwStop}{\newline}                                            %oznaczenie końca rozwiązania
\newcommand{\odpStart}{\noindent \textbf{Odpowiedź:}\newline}    %oznaczenie początku odpowiedzi końcowej (wypisanie wyniku)
\newcommand{\odpStop}{\newline}                                             %oznaczenie końca odpowiedzi końcowej (wypisanie wyniku)
\newcommand{\testStart}{\noindent \textbf{Test:}\newline} %ewentualne możliwe opcje odpowiedzi testowej: A. ? B. ? C. ? D. ? itd.
\newcommand{\testStop}{\newline} %koniec wprowadzania odpowiedzi testowych
\newcommand{\kluczStart}{\noindent \textbf{Test poprawna odpowiedź:}\newline} %klucz, poprawna odpowiedź pytania testowego (jedna literka): A lub B lub C lub D itd.
\newcommand{\kluczStop}{\newline} %koniec poprawnej odpowiedzi pytania testowego 
\newcommand{\wstawGrafike}[2]{\begin{figure}[h] \includegraphics[scale=#2] {#1} \end{figure}} %gdyby była potrzeba wstawienia obrazka, parametry: nazwa pliku, skala (jak nie wiesz co wpisać, to wpisz 1)

\begin{document}
\maketitle


\kategoria{Wikieł/Z3.7e}
\zadStart{Zadanie z Wikieł Z 3.7 e)  moja wersja nr [nrWersji]}
%[p1]:[2,3,4,5]
%[p2]:[2,3,4,5]
%[p0]:[1]
%[a1]=[p0]/[p1]
%[a2]=round([a1]*[a1]*[p2],2)
%[a3]=round(pow([a2],2)*pow([p2],2),2)
%[a4]=round(pow([a3],2)*pow([p2],3),2)
%[a5]=round(pow([a4],2)*pow([p2],4),2)

Wypisać pięć początkowych wyrazów ciągu określonego rekurencyjnie.
$$\left\{ \begin{array}{ll}
a_{1}=\frac{1}{[p1]}\\
a_{n+1}=(a_{n})^{2}\cdot[p2]^{n}& \textrm{dla n$\geq$1} 
\end{array} \right.
$$
\zadStop
\rozwStart{Maja Szabłowska}{}
$$a_{1}=\frac{1}{[p1]}$$
$$a_{2}=a_{1}^{2}\cdot [p2]^{1}=[a2]$$
$$a_{3}=a_{2}^{2}\cdot [p2]^{2}=[a3]$$
$$a_{4}=a_{3}^{2}\cdot [p2]^{3}=[a4]$$
$$a_{5}=a_{4}^{2}\cdot [p2]^{4}=[a5]$$
\rozwStop
\odpStart
$[p1],[a2],[a3],[a4],[a5]$
\odpStop
\testStart
A.$[p1],[a2],[a3],[a4],[a5]$
B.$[p1],[a2],[a3],[a4],[p1]$
C.$[p1],[a2],[a2],[a4],[a5]$
D.$[p1],[a2],[a2],[a2],[a5]$
E.$[p1],[p1],[a3],[a4],[a5]$
F.$[p1],[a2],[a3],[a2],[a3]$
G.$[p1],[a2],[a3],[a2],[a2]$
H.$[a2],[a2],[a3],[a4],[a5]$
I.$[a5],[a2],[a3],[a4],[a5]$
\testStop
\kluczStart
A
\kluczStop



\end{document}