\documentclass[12pt, a4paper]{article}
\usepackage[utf8]{inputenc}
\usepackage{polski}

\usepackage{amsthm}  %pakiet do tworzenia twierdzeń itp.
\usepackage{amsmath} %pakiet do niektórych symboli matematycznych
\usepackage{amssymb} %pakiet do symboli mat., np. \nsubseteq
\usepackage{amsfonts}
\usepackage{graphicx} %obsługa plików graficznych z rozszerzeniem png, jpg
\theoremstyle{definition} %styl dla definicji
\newtheorem{zad}{} 
\title{Multizestaw zadań}
\author{Robert Fidytek}
%\date{\today}
\date{}
\newcounter{liczniksekcji}
\newcommand{\kategoria}[1]{\section{#1}} %olreślamy nazwę kateforii zadań
\newcommand{\zadStart}[1]{\begin{zad}#1\newline} %oznaczenie początku zadania
\newcommand{\zadStop}{\end{zad}}   %oznaczenie końca zadania
%Makra opcjonarne (nie muszą występować):
\newcommand{\rozwStart}[2]{\noindent \textbf{Rozwiązanie (autor #1 , recenzent #2): }\newline} %oznaczenie początku rozwiązania, opcjonarnie można wprowadzić informację o autorze rozwiązania zadania i recenzencie poprawności wykonania rozwiązania zadania
\newcommand{\rozwStop}{\newline}                                            %oznaczenie końca rozwiązania
\newcommand{\odpStart}{\noindent \textbf{Odpowiedź:}\newline}    %oznaczenie początku odpowiedzi końcowej (wypisanie wyniku)
\newcommand{\odpStop}{\newline}                                             %oznaczenie końca odpowiedzi końcowej (wypisanie wyniku)
\newcommand{\testStart}{\noindent \textbf{Test:}\newline} %ewentualne możliwe opcje odpowiedzi testowej: A. ? B. ? C. ? D. ? itd.
\newcommand{\testStop}{\newline} %koniec wprowadzania odpowiedzi testowych
\newcommand{\kluczStart}{\noindent \textbf{Test poprawna odpowiedź:}\newline} %klucz, poprawna odpowiedź pytania testowego (jedna literka): A lub B lub C lub D itd.
\newcommand{\kluczStop}{\newline} %koniec poprawnej odpowiedzi pytania testowego 
\newcommand{\wstawGrafike}[2]{\begin{figure}[h] \includegraphics[scale=#2] {#1} \end{figure}} %gdyby była potrzeba wstawienia obrazka, parametry: nazwa pliku, skala (jak nie wiesz co wpisać, to wpisz 1)

\begin{document}
\maketitle


\kategoria{Wikieł/Z1.52d}
\zadStart{Zadanie z Wikieł Z 1.52 d)  moja wersja nr [nrWersji]}
%[p1]:[2,3,4,5,6,7,8,9,10]
%[p2]:[2,3,4,5,6,7,8,9,10]
%[p3]:[2,3,4,5,6,7,8,9,10]
%[p4]=random.randint(2,10)
%[p5]=random.randint(2,10)
%[p6]:[2,3,4,5,6,7,8,9,10]
%[p7]=random.randint(2,10)
%[a]=round([p1]/[p6],2)
%[b]=round([a]*[p7],2)
%[p2b]=round(-[p2]+[b],2)
%[c]=round([p2b]/[p6],2)
%[d]=round([c]*[p7],2)
%[p3d]=round(-[p3]+[d],2)
%[e]=round([p3d]/[p6],2)
%[f]=round([e]*[p7],2)
%[p4f]=round([p4]+[f],2)
%[g]=round([p4f]/[p6],2)
%[p7g]=round([g]*[p7],2)
%[r]=-[p5]+[p7g]
%(-[p5]+[p7g])==0 and [a]!=0 and [c]!=0 and [e]!=0 and [g]!=0 
Obliczyć iloraz wielomianów $$([p1]x^{4}-[p2]x^{3}-[p3]x^{2}+[p4]x-[p5]):([p6]x-[p7]).$$

\zadStop

\rozwStart{Maja Szabłowska}{}
$$\begin{array}{lll}
([p1]x^{4}-[p2]x^{3}-[p3]x^{2}+[p4]x-[p5]):([p6]x-[p7])  =  [a]x^3 +([c])x^2 + ([e])x +[g] \\
\underline{-[p1]x^4 + [b]x^3} & &  \\
\qquad [p2b]x^3  -[p3]x^2 +[p4]x -[p5] & & \\
\qquad \ \ \underline{-([p2b])x^3 +[d]x^2} & &\\
\qquad \qquad \qquad [p3d]x^2 + [p4]x - [p5] & & \\
\qquad \qquad \quad \underline{-([p3d])x^2 + [f]x}  & & \\
\qquad \qquad \qquad \qquad [p4f]x - [p5] & & \\
\qquad \qquad \qquad \qquad \ \ \underline{-([p4f])x + [p7g]} & & \\
\qquad \qquad \qquad \qquad \quad R = [r] & &
\end{array}$$
\rozwStop


\odpStart
$[a]x^3 +([c])x^2 + ([e])x +[g]$
\odpStop
\testStart
A.$[a]x^3 +([c])x^2 + ([e])x +[g]$
B.$[a]x^3 - ([e])x +[g]$
D.$[a]x^3 +([c])x^2 +[p1]$
E.$([c])x^2 + [p3]x +[g]$
F.$[p3]x^3 $
G.$[p2]x^2 + ([e])x +[g]$
H.$[p1]x^3 +[p2]x^2 + [p3]x +[p4]$
\testStop
\kluczStart
A
\kluczStop



\end{document}
