\documentclass[12pt, a4paper]{article}
\usepackage[utf8]{inputenc}
\usepackage{polski}

\usepackage{amsthm}  %pakiet do tworzenia twierdzeń itp.
\usepackage{amsmath} %pakiet do niektórych symboli matematycznych
\usepackage{amssymb} %pakiet do symboli mat., np. \nsubseteq
\usepackage{amsfonts}
\usepackage{graphicx} %obsługa plików graficznych z rozszerzeniem png, jpg
\theoremstyle{definition} %styl dla definicji
\newtheorem{zad}{} 
\title{Multizestaw zadań}
\author{Robert Fidytek}
%\date{\today}
\date{}\documentclass[12pt, a4paper]{article}
\usepackage[utf8]{inputenc}
\usepackage{polski}

\usepackage{amsthm}  %pakiet do tworzenia twierdzeń itp.
\usepackage{amsmath} %pakiet do niektórych symboli matematycznych
\usepackage{amssymb} %pakiet do symboli mat., np. \nsubseteq
\usepackage{amsfonts}
\usepackage{graphicx} %obsługa plików graficznych z rozszerzeniem png, jpg
\theoremstyle{definition} %styl dla definicji
\newtheorem{zad}{} 
\title{Multizestaw zadań}
\author{Robert Fidytek}
%\date{\today}
\date{}
\newcounter{liczniksekcji}
\newcommand{\kategoria}[1]{\section{#1}} %olreślamy nazwę kateforii zadań
\newcommand{\zadStart}[1]{\begin{zad}#1\newline} %oznaczenie początku zadania
\newcommand{\zadStop}{\end{zad}}   %oznaczenie końca zadania
%Makra opcjonarne (nie muszą występować):
\newcommand{\rozwStart}[2]{\noindent \textbf{Rozwiązanie (autor #1 , recenzent #2): }\newline} %oznaczenie początku rozwiązania, opcjonarnie można wprowadzić informację o autorze rozwiązania zadania i recenzencie poprawności wykonania rozwiązania zadania
\newcommand{\rozwStop}{\newline}                                            %oznaczenie końca rozwiązania
\newcommand{\odpStart}{\noindent \textbf{Odpowiedź:}\newline}    %oznaczenie początku odpowiedzi końcowej (wypisanie wyniku)
\newcommand{\odpStop}{\newline}                                             %oznaczenie końca odpowiedzi końcowej (wypisanie wyniku)
\newcommand{\testStart}{\noindent \textbf{Test:}\newline} %ewentualne możliwe opcje odpowiedzi testowej: A. ? B. ? C. ? D. ? itd.
\newcommand{\testStop}{\newline} %koniec wprowadzania odpowiedzi testowych
\newcommand{\kluczStart}{\noindent \textbf{Test poprawna odpowiedź:}\newline} %klucz, poprawna odpowiedź pytania testowego (jedna literka): A lub B lub C lub D itd.
\newcommand{\kluczStop}{\newline} %koniec poprawnej odpowiedzi pytania testowego 
\newcommand{\wstawGrafike}[2]{\begin{figure}[h] \includegraphics[scale=#2] {#1} \end{figure}} %gdyby była potrzeba wstawienia obrazka, parametry: nazwa pliku, skala (jak nie wiesz co wpisać, to wpisz 1)

\begin{document}
\maketitle


\kategoria{Wikieł/Z2.40}
\zadStart{Zadanie z Wikieł Z 2.40  moja wersja nr [nrWersji]}
%[p1]:[2,3,4,5,6,7,8,9,10]
%[p2]:[2,3,4,5,6,7,8,9,10]
%[p3]:[2,3,4,5,6,7,8,9,10]
%[p4]=random.randint(2,10)
%[p5]=random.randint(1,10)
%[p4p1]=-[p4]*[p1]
%[p4p3]=-[p4]*[p3]
%[p5p2]=[p5]*[p2]
%[p5p1]=[p5]*[p1]
%[m1]=round([p4p1]/[p2],2)
%[m2]=round([p5p2]/[p4p3],2)
%[m3]=round(math.sqrt([p5p1]/[p3]),2)
%math.gcd([p5p1],[p3])==1 

Rozwiązać układ równań
$$\left\{\begin{array}{ccc}
[p1]x+[p2]y&=&[p3]m\\
mx-[p4]y&=&[p5]
\end{array} \right.$$
i przeprowadzić dyskusję rozwiązań w zależności od parametru $m$.
\zadStop
\rozwStart{Maja Szabłowska}{}
Powyższemu układowi równań odpowiadają wyznaczniki:
$$W=\left| \begin{array}{lccr} [p1] & [p2] \\ m & -[p4] \end{array}\right| = -[p4]\cdot[p1] - m\cdot[p2]=[p4p1]-[p2]m$$

$$W_{x}=\left| \begin{array}{lccr} [p3]m & [p2] \\ [p5] & -[p4] \end{array}\right| = -[p4]\cdot[p3]m - [p5]\cdot[p2]=[p4p3]m-[p5p2]$$

$$W_{y}=\left| \begin{array}{lccr} [p1] & [p3]m \\ m & [p5] \end{array}\right| = [p1]\cdot[p5] - m\cdot[p3]m=[p5p1]-[p3]m^{2}$$
\begin{enumerate}
    \item Układ ma dokładnie jedno rozwiązanie, gdy
    $$W\neq0 \iff [p4p1]-[p2]m \neq 0\iff m\neq\frac{[p4p1]}{[p2]}\neq [m1]$$
    Rozwiązaniem jest para liczb $x=\frac{[p4p3]m-[p5p2]}{[p4p1]-[p2]m}, y=\frac{[p5p1]-[p3]m^{2}}{[p4p1]-[p2]m}.$
    
    \item Układ równań jest nieoznaczony, gdy
    $$W=W_{x}=W_{y}=0 \iff m=[m1] \land m=\frac{[p5p2]}{[p4p3]}=[m2] \land$$
    $$ \land \quad \left( m=\sqrt{\frac{[p5p1]}{[p3]}}=[m3] \lor m=-\sqrt{\frac{[p5p1]}{[p3]}}=-[m3] \right)$$
    
    \item Układ równań jest sprzeczny, gdy
    $$W=0 \land (W_{x}\neq0 \lor W_{y}\neq0)\iff m=[m1]\ \land\ (m\ \neq\ [m2]\ \lor\  m\ \neq\ \pm\ [m3]\)$$
\end{enumerate}

\rozwStop
\odpStart
Jedno rozwiązanie dla $m\neq[m1]$, układ nieoznaczony dla $m=[m2] \land m=[m3] \land m=-[m3]$, 
układ sprzeczny dla $m\=[m1] \land (m\neq[m2]\ \lor m\neq\pm[m3])$.
\odpStop
\testStart
A.Jedno rozwiązanie dla $m\neq[m1]$, układ nieoznaczony dla $m=[m2] \land m=[m3] \land m=-[m3]$, 
układ sprzeczny dla $m\=[m1] \land (m\neq[m2]\ \lor m\neq\pm[m3])$.
B.Jedno rozwiązanie dla $m=[m1]$, układ nieoznaczony dla $m=[m2] \land m=[m3] \land m=-[m3]$, 
układ sprzeczny dla $m\=[m1] \land (m\neq[m2]\ \lor m\neq\pm[m3])$.
C.Jedno rozwiązanie dla $m\neq[m1]$, układ nieoznaczony dla $m=[p4p1] \land m=[m3] \land m=-[m3]$, 
układ sprzeczny dla $m\=[m1] \land (m\neq[m2]\ \lor m\neq\pm[m3])$.
D.Jedno rozwiązanie dla $m\neq[m1]$, układ nieoznaczony dla $m=[m2] \land m=[m3] \lor m=-[m3]$, 
układ sprzeczny dla $m\=[m1] \land (m\neq[m2]\ \lor m\neq\pm[m3])$.
E.Jedno rozwiązanie dla $m\neq[m1]$, układ nieoznaczony dla $m=[p4p3] \land m=[m3] \land m=-[m3]$, 
układ sprzeczny dla $m\=[m1] \land (m\neq[m2]\ \lor m\neq\pm[m3])$.
F.Jedno rozwiązanie dla $m\neq[m3]$, układ nieoznaczony dla $m=[m2] \land m=[m3] \land m=-[m3]$, 
układ sprzeczny dla $m\=[m1] \land (m\neq[m2]\ \lor m\neq\pm[m3])$.
G.Jedno rozwiązanie dla $m=[m1]$, układ nieoznaczony dla $m=[m2] \lor m=[m3] \land m=-[m3]$, 
układ sprzeczny dla $m\=[m1] \land (m\neq[m2]\ \lor m\neq\pm[m3])$.
H.Jedno rozwiązanie dla $m\neq[m1]$, układ nieoznaczony dla $m=[m2] \land m=[m3] $, 
układ sprzeczny dla $m\=[m1] \land (m\neq[m2]\ \lor m\neq\pm[m3])$.
I.Jedno rozwiązanie dla $m\neq[m1]$, układ nieoznaczony dla $m=[m2] \land m=[m3] \land m=-[m3]$, 
układ sprzeczny dla $m\=[m1]$.
\testStop
\kluczStart
A
\kluczStop



\end{document}
