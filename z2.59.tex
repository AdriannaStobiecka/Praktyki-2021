\documentclass[12pt, a4paper]{article}
\usepackage[utf8]{inputenc}
\usepackage{polski}

\usepackage{amsthm}  %pakiet do tworzenia twierdzeń itp.
\usepackage{amsmath} %pakiet do niektórych symboli matematycznych
\usepackage{amssymb} %pakiet do symboli mat., np. \nsubseteq
\usepackage{amsfonts}
\usepackage{graphicx} %obsługa plików graficznych z rozszerzeniem png, jpg
\theoremstyle{definition} %styl dla definicji
\newtheorem{zad}{} 
\title{Multizestaw zadań}
\author{Robert Fidytek}
%\date{\today}
\date{}
\newcounter{liczniksekcji}
\newcommand{\kategoria}[1]{\section{#1}} %olreślamy nazwę kateforii zadań
\newcommand{\zadStart}[1]{\begin{zad}#1\newline} %oznaczenie początku zadania
\newcommand{\zadStop}{\end{zad}}   %oznaczenie końca zadania
%Makra opcjonarne (nie muszą występować):
\newcommand{\rozwStart}[2]{\noindent \textbf{Rozwiązanie (autor #1 , recenzent #2): }\newline} %oznaczenie początku rozwiązania, opcjonarnie można wprowadzić informację o autorze rozwiązania zadania i recenzencie poprawności wykonania rozwiązania zadania
\newcommand{\rozwStop}{\newline}                                            %oznaczenie końca rozwiązania
\newcommand{\odpStart}{\noindent \textbf{Odpowiedź:}\newline}    %oznaczenie początku odpowiedzi końcowej (wypisanie wyniku)
\newcommand{\odpStop}{\newline}                                             %oznaczenie końca odpowiedzi końcowej (wypisanie wyniku)
\newcommand{\testStart}{\noindent \textbf{Test:}\newline} %ewentualne możliwe opcje odpowiedzi testowej: A. ? B. ? C. ? D. ? itd.
\newcommand{\testStop}{\newline} %koniec wprowadzania odpowiedzi testowych
\newcommand{\kluczStart}{\noindent \textbf{Test poprawna odpowiedź:}\newline} %klucz, poprawna odpowiedź pytania testowego (jedna literka): A lub B lub C lub D itd.
\newcommand{\kluczStop}{\newline} %koniec poprawnej odpowiedzi pytania testowego 
\newcommand{\wstawGrafike}[2]{\begin{figure}[h] \includegraphics[scale=#2] {#1} \end{figure}} %gdyby była potrzeba wstawienia obrazka, parametry: nazwa pliku, skala (jak nie wiesz co wpisać, to wpisz 1)

\begin{document}
\maketitle


\kategoria{Wikieł/Z2.59}
\zadStart{Zadanie z Wikieł Z 2.59 moja wersja nr [nrWersji]}
%[a]:[2,3,4,5,6,7,8,9,10,11,12,13,14,15,16,17,18,19,20,21,22,23,24,25,26,27,28,29,30]
%[a1]:[2,3,5,6,7,10,11,13,14,15,17,19,21,22,23,26,29,30,31,33,34,35,37,38,39,41]
%[b]=[a1]
%[aa1]=[a1]*[a1]
%[c]:[2,3,4,5,6,7,8,9,10,11,12,13,14,15,16,17,18,19,20,21,22,23,24,25,26,27,28,29,30]
%[b2]=[b]*2
%[ba1]=[b2]/[a1]
%[cba1]=int([ba1])
%[1a]=[a]+1
%[k2ba1]=[cba1]*[cba1]*[a1]
%[41ab]=4*[1a]*[b]
%[41ac]=4*[1a]*[c]
%[k]=[41ab]-[k2ba1]
%[ka]=[41ac]/[k]
%[cka]=int([ka])
%[bb1]=math.sqrt([cka])
%[cbb1]=int([bb1])
%[ba1].is_integer()==True and [ka].is_integer()==True and [bb1].is_integer()==True and [k]>0
Napisać równania stycznych do elipsy $[a]x^2+[b]y^2=[c]$ i prostopadłych do prostej $y=\sqrt{[a1]}x$.
\zadStop
\rozwStart{Aleksandra Pasińska}{}
$$y=ax+b$$
$$a_1=\sqrt{[a1]}$$
$$a_2=\frac{-1}{\sqrt{[a1]}}=-\frac{\sqrt{[a1]}}{[a1]}$$
$$y=-\frac{\sqrt{[a1]}}{[a1]}x+b$$
$$[a]x^2+[b]\bigg(-\frac{\sqrt{[a1]}}{[a1]}x+b\bigg)^2=[c]$$
$$[a]x^2+[b]\bigg(\frac{[a1]}{[aa1]}x^2-\frac{2\sqrt{[a1]}bx}{[a1]}+b^2\bigg)-[c]=0$$
$$[a]x^2+x^2-[cba1]\sqrt{[a1]}bx+[b]b^2-[c]=0$$
$$[1a]x^2-[cba1]\sqrt{[a1]}bx+[b]b^2-[c]=0$$
$$\Delta=[k2ba1]b^2-4\cdot[1a]([b]b^2-[c])$$
$$[k2ba1]b^2-[41ab]b^2+[41ac]=0$$
$$-[k]b^2=-[41ac]$$
$$b^2=[cka]$$
$$b=\pm \sqrt{[cka]}$$
$$b_1=-[cbb1],b_2=[cbb1]$$
$$ y=-\frac{\sqrt{[a1]}}{[a1]}x-[cbb1], y=-\frac{\sqrt{[a1]}}{[a1]}x+[cbb1]$$
\rozwStop
\odpStart
$ y=-\frac{\sqrt{[a1]}}{[a1]}x-[cbb1], y=-\frac{\sqrt{[a1]}}{[a1]}x+[cbb1]$\\
\odpStop
\testStart
A.$ y=-\frac{\sqrt{[a1]}}{[a1]}x-[cbb1], y=-\frac{\sqrt{[a1]}}{[a1]}x+[cbb1]$
B.$ y=-x-[cbb1], y=x-[cbb1]$
C.$ y=[cbb1], y=-[cbb1]$
D.$ y=x-[cbb1], y=0$
E.$ y=0, y=-x+[cbb1]$
F.$ y=-[cbb1], y=-x+[cbb1]$
G.$ y=x, y=[cbb1]$
H.$ y=x-[cbb1], y=-x$
I.$ y=x, y=x+[cbb1]$
\testStop
\kluczStart
A
\kluczStop



\end{document}