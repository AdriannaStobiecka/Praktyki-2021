\documentclass[12pt, a4paper]{article}
\usepackage[utf8]{inputenc}
\usepackage{polski}

\usepackage{amsthm}  %pakiet do tworzenia twierdzeń itp.
\usepackage{amsmath} %pakiet do niektórych symboli matematycznych
\usepackage{amssymb} %pakiet do symboli mat., np. \nsubseteq
\usepackage{amsfonts}
\usepackage{graphicx} %obsługa plików graficznych z rozszerzeniem png, jpg
\theoremstyle{definition} %styl dla definicji
\newtheorem{zad}{} 
\title{Multizestaw zadań}
\author{Robert Fidytek}
%\date{\today}
\date{}
\newcounter{liczniksekcji}
\newcommand{\kategoria}[1]{\section{#1}} %olreślamy nazwę kateforii zadań
\newcommand{\zadStart}[1]{\begin{zad}#1\newline} %oznaczenie początku zadania
\newcommand{\zadStop}{\end{zad}}   %oznaczenie końca zadania
%Makra opcjonarne (nie muszą występować):
\newcommand{\rozwStart}[2]{\noindent \textbf{Rozwiązanie (autor #1 , recenzent #2): }\newline} %oznaczenie początku rozwiązania, opcjonarnie można wprowadzić informację o autorze rozwiązania zadania i recenzencie poprawności wykonania rozwiązania zadania
\newcommand{\rozwStop}{\newline}                                            %oznaczenie końca rozwiązania
\newcommand{\odpStart}{\noindent \textbf{Odpowiedź:}\newline}    %oznaczenie początku odpowiedzi końcowej (wypisanie wyniku)
\newcommand{\odpStop}{\newline}                                             %oznaczenie końca odpowiedzi końcowej (wypisanie wyniku)
\newcommand{\testStart}{\noindent \textbf{Test:}\newline} %ewentualne możliwe opcje odpowiedzi testowej: A. ? B. ? C. ? D. ? itd.
\newcommand{\testStop}{\newline} %koniec wprowadzania odpowiedzi testowych
\newcommand{\kluczStart}{\noindent \textbf{Test poprawna odpowiedź:}\newline} %klucz, poprawna odpowiedź pytania testowego (jedna literka): A lub B lub C lub D itd.
\newcommand{\kluczStop}{\newline} %koniec poprawnej odpowiedzi pytania testowego 
\newcommand{\wstawGrafike}[2]{\begin{figure}[h] \includegraphics[scale=#2] {#1} \end{figure}} %gdyby była potrzeba wstawienia obrazka, parametry: nazwa pliku, skala (jak nie wiesz co wpisać, to wpisz 1)

\begin{document}
\maketitle


\kategoria{Wikieł/Z4.16e}
\zadStart{Zadanie z Wikieł Z 4.16 e) moja wersja nr [nrWersji]}
%[c]:[2,3,4,5,6,7,8,9,10,11,12,13,14,15,16,17,18]
%[b]:[2,3,4,5,6,7,8,9,10,11,12,13,14,16,17,18,19]
%[d]=2*[b]
%[xo]=0
%[cd]=2*[c]
%[ccdk]=[c]*[cd]
%[ct]=3*[c]
%[ccz]=4*[c]
%[u]=int([ccdk]/2)
%[a]=int([u]/3)
%[l]=[a]*3
%[m]=[b]*2
%[w]=[l]/[m]
%[l1]=[u]
%[m1]=[d]
%math.gcd([a],[b])==1 and math.gcd([l],[m])==1 and math.gcd([l1],[m1])==1
Zbadać, czy istnieje następująca granica $\lim\limits_{x\to [xo]}f(x)$ dla funkcji 
$$
f(x)=\left\{\begin{array}{ll}
\frac{[a](cos(x)-cos(2x))}{[b]x^2} & \textrm{dla $x<[xo]$}\\
\frac{sin^{2}([c]x)}{[d]x^{2}} & \textrm{dla $x>[xo]$}
\end{array} \right.
$$
w punkcie $x_{0}=[xo]$. Jeśli tak, to obliczyć ją.
\zadStop
\rozwStart{Justyna Chojecka}{}
Korzystając z reguły de l'Hospitala obliczamy granicę lewostronną funkcji $f$ w punkcie $x_{0}=[xo]$.
$$\lim\limits_{x\to [xo]^{-}}f(x)=\lim\limits_{x\to [xo]^{-}}\frac{[a](cos(x)-cos(2x))}{[b]x^2} \overset{l'H}{=}\frac{[a]}{[b]} \lim\limits_{x\to [xo]^{-}}\frac{-sin(x)+2sin(2x)}{2x}$$$$\overset{l'H}{=}\frac{[a]}{[b]} \lim\limits_{x\to [xo]^{-}}\frac{-cos(x)+4cos(2x)}{2}=\frac{[a]}{[b]}\cdot \frac{-cos([xo])+4cos(2\cdot [xo])}{2}$$$$=\frac{[a]}{[b]}\cdot \frac{-1+4\cdot 1}{2}=\frac{[a]}{[b]}\cdot\frac{3}{2}=\frac{[l]}{[m]}$$
Następnie korzystając ponownie z reguły de l'Hospitala obliczamy granicę prawostronną funkcji $f$ w punkcie $x_{0}=[xo]$.
$$\lim\limits_{x\to [xo]^{-}}f(x)=\lim\limits_{x\to [xo]^{+}}\frac{sin^{2}([c]x)}{[d]x^{2}} \overset{l'H}{=}\frac{1}{[d]}\lim\limits_{x\to [xo]^{+}}\frac{[c]sin([cd]x)}{2x}$$$$\overset{l'H}{=}\frac{1}{[d]}\lim\limits_{x\to [xo]^{+}}\frac{[ccdk]cos([cd]x)}{2}=\frac{1}{[d]}\cdot \frac{[ccdk]}{2} \cdot cos([cd]\cdot [xo])$$$$=\frac{1}{[d]}\cdot [u]\cdot 1=\frac{[l1]}{[m1]}$$
Skoro 
$$\lim\limits_{x\to [xo]^{-}}f(x)=\lim\limits_{x\to [xo]^{+}}f(x)=\frac{[l]}{[m]},$$
to granica $\lim\limits_{x\to [xo]}f(x)$ istnieje i jest równa $\frac{[l]}{[m]}$.
\rozwStop
\odpStart
$\frac{[l]}{[m]}$
\odpStop
\testStart
A.$\frac{[l]}{[m]}$
B.nie istnieje
C.$\frac{[m]}{[l]}$
D.0
E.$-\frac{[m]}{[l]}$
F.$\frac{[a]}{[b]}$
G.$\frac{[b]}{[a]}$
H.$-\frac{[a]}{[b]}$
I.$-\frac{[l]}{[m]}$
\testStop
\kluczStart
A
\kluczStop



\end{document}
