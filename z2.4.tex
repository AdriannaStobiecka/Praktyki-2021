\documentclass[12pt, a4paper]{article}
\usepackage[utf8]{inputenc}
\usepackage{polski}

\usepackage{amsthm}  %pakiet do tworzenia twierdzeń itp.
\usepackage{amsmath} %pakiet do niektórych symboli matematycznych
\usepackage{amssymb} %pakiet do symboli mat., np. \nsubseteq
\usepackage{amsfonts}
\usepackage{graphicx} %obsługa plików graficznych z rozszerzeniem png, jpg
\theoremstyle{definition} %styl dla definicji
\newtheorem{zad}{} 
\title{Multizestaw zadań}
\author{Robert Fidytek}
%\date{\today}
\date{}\documentclass[12pt, a4paper]{article}
\usepackage[utf8]{inputenc}
\usepackage{polski}

\usepackage{amsthm}  %pakiet do tworzenia twierdzeń itp.
\usepackage{amsmath} %pakiet do niektórych symboli matematycznych
\usepackage{amssymb} %pakiet do symboli mat., np. \nsubseteq
\usepackage{amsfonts}
\usepackage{graphicx} %obsługa plików graficznych z rozszerzeniem png, jpg
\theoremstyle{definition} %styl dla definicji
\newtheorem{zad}{} 
\title{Multizestaw zadań}
\author{Robert Fidytek}
%\date{\today}
\date{}
\newcounter{liczniksekcji}
\newcommand{\kategoria}[1]{\section{#1}} %olreślamy nazwę kateforii zadań
\newcommand{\zadStart}[1]{\begin{zad}#1\newline} %oznaczenie początku zadania
\newcommand{\zadStop}{\end{zad}}   %oznaczenie końca zadania
%Makra opcjonarne (nie muszą występować):
\newcommand{\rozwStart}[2]{\noindent \textbf{Rozwiązanie (autor #1 , recenzent #2): }\newline} %oznaczenie początku rozwiązania, opcjonarnie można wprowadzić informację o autorze rozwiązania zadania i recenzencie poprawności wykonania rozwiązania zadania
\newcommand{\rozwStop}{\newline}                                            %oznaczenie końca rozwiązania
\newcommand{\odpStart}{\noindent \textbf{Odpowiedź:}\newline}    %oznaczenie początku odpowiedzi końcowej (wypisanie wyniku)
\newcommand{\odpStop}{\newline}                                             %oznaczenie końca odpowiedzi końcowej (wypisanie wyniku)
\newcommand{\testStart}{\noindent \textbf{Test:}\newline} %ewentualne możliwe opcje odpowiedzi testowej: A. ? B. ? C. ? D. ? itd.
\newcommand{\testStop}{\newline} %koniec wprowadzania odpowiedzi testowych
\newcommand{\kluczStart}{\noindent \textbf{Test poprawna odpowiedź:}\newline} %klucz, poprawna odpowiedź pytania testowego (jedna literka): A lub B lub C lub D itd.
\newcommand{\kluczStop}{\newline} %koniec poprawnej odpowiedzi pytania testowego 
\newcommand{\wstawGrafike}[2]{\begin{figure}[h] \includegraphics[scale=#2] {#1} \end{figure}} %gdyby była potrzeba wstawienia obrazka, parametry: nazwa pliku, skala (jak nie wiesz co wpisać, to wpisz 1)

\begin{document}
\maketitle


\kategoria{Wikieł/Z2.4}
\zadStart{Zadanie z Wikieł Z 2.32  moja wersja nr [nrWersji]}
%[p1]:[1,2,3,4,5,6,7,8,9,10]
%[p2]:[1,2,3,4,5,6,7,8,9,10]
%[p3]:[1,2,3,4,5,6,7,8,9,10]
%[p4]:[1,2,3,4,5,6,7,8,9,10]
%[p0]:[0]
%[p1p2]=[p1]*[p2]
%[p1p4]=[p1]*[p4]
%[m1]=[p1p4]+[p1p2]
%[p2k]=[p2]*[p2]
%[p1k]=[p1]*[p1]
%[p1p3]=[p1]*[p3]
%[m2]=[p2k]+[p1k]
%[m2g]=[p1p2]+[p1p3]
%[w1]=[m1]/[p1p2]
%[w2]=[m2g]/[m2]
%[p2p4]=[p2]*[p4]
%[m3]=[p2p4]-[p1p3]-[p1k]
%[r3]=[p1p4]-[p1p3]
%([p1p4]+[p1p2])==([p1p2]+[p1p3]) and [p1p2]==[m2]

Wyznaczyć wartości parametru $m$, dla których wektory $\vec{d}+m\vec{b}$ oraz $\vec{c}$ są równoległe, gdy $\vec{a}=[[p1],[p1],[p1]], \vec{b}=[[p2],[p1],[p0]], \vec{c}=[[p1]m+[p3],[p4],-[p2]].$
\zadStop

\rozwStart{Maja Szabłowska}{}
$$\vec{a}+m\vec{b}=[[p1],[p1],[p1]]+m[[p2],[p1],[p0]]=[[p2]m+[p1], [p1]m+[p1], [p1]]$$

$$[[p2]m+[p1], [p1]m+[p1], [p1]] \times [[p1]m+[p3],[p4],-[p2]]=$$
$$\left| \begin{array}{ccc}
\vec{i} & \vec{j} & \vec{k}\\
[p2]m+[p1] & [p1]m+[p1] & [p1] \\
[p1]m+[p3] & [p4] & -[p2]
\end{array} \right|=$$
$$=\vec{i}(([p1]m+[p1])\cdot(-[p2])-[p1]\cdot[p4])-\vec{j}(([p2]m+[p1])\cdot(-[p2])-([p1]m+[p3])\cdot[p1])+$$
$$+\vec{k}(([p2]m+[p1])\cdot[p4]-([p1]m+[p1])\cdot([p1]m+[p3]))=0$$

$$([p1]m+[p1])\cdot(-[p2])-[p1]\cdot[p4]=0 $$
$$-[p1p2]m-[p1p2]-[p1p4]=0$$
$$-[p1p2]m=[p1p4]+[p1p2]$$
$$m=-\frac{[m1]}{[p1p2]}$$

$$([p2]m+[p1])\cdot(-[p2])-([p1]m+[p3])\cdot[p1]=0$$
$$-[p2k]m-[p1p2]-[p1k]m-[p1p3]=0$$
$$-[m2]m=[p1p2]+[p1p3]$$
$$m=-\frac{[m2g]}{[m2]}$$

$$([p2]m+[p1])\cdot[p4]-([p1]m+[p1])\cdot([p1]m+[p3])=0$$
$$[p2p4]m+[p1p4]-[p1k]m^{2}-[p1p3]m-[p1k]m-[p1p3]=0$$
$$[p1k]m^{2}+[m3]m+[r3]=0$$


\rozwStop


\odpStart
$x=\frac{[a]}{[p5b]}$
\odpStop
\testStart
A.$x=\frac{[a]}{[p5b]}$
B.$x=\frac{[a]}{[p5]}$
D.$x=[b]$
E.$x=[ab]$
F.$x=[p2p4]$
G.$x=[p5b]$
H.$x=\frac{[p5b]}{[a]}$

\testStop
\kluczStart
A
\kluczStop



\end{document}
