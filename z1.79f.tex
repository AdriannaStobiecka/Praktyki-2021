\documentclass[12pt, a4paper]{article}
\usepackage[utf8]{inputenc}
\usepackage{polski}

\usepackage{amsthm}  %pakiet do tworzenia twierdzeń itp.
\usepackage{amsmath} %pakiet do niektórych symboli matematycznych
\usepackage{amssymb} %pakiet do symboli mat., np. \nsubseteq
\usepackage{amsfonts}
\usepackage{graphicx} %obsługa plików graficznych z rozszerzeniem png, jpg
\theoremstyle{definition} %styl dla definicji
\newtheorem{zad}{} 
\title{Multizestaw zadań}
\author{Robert Fidytek}
%\date{\today}
\date{}
\newcounter{liczniksekcji}
\newcommand{\kategoria}[1]{\section{#1}} %olreślamy nazwę kateforii zadań
\newcommand{\zadStart}[1]{\begin{zad}#1\newline} %oznaczenie początku zadania
\newcommand{\zadStop}{\end{zad}}   %oznaczenie końca zadania
%Makra opcjonarne (nie muszą występować):
\newcommand{\rozwStart}[2]{\noindent \textbf{Rozwiązanie (autor #1 , recenzent #2): }\newline} %oznaczenie początku rozwiązania, opcjonarnie można wprowadzić informację o autorze rozwiązania zadania i recenzencie poprawności wykonania rozwiązania zadania
\newcommand{\rozwStop}{\newline}                                            %oznaczenie końca rozwiązania
\newcommand{\odpStart}{\noindent \textbf{Odpowiedź:}\newline}    %oznaczenie początku odpowiedzi końcowej (wypisanie wyniku)
\newcommand{\odpStop}{\newline}                                             %oznaczenie końca odpowiedzi końcowej (wypisanie wyniku)
\newcommand{\testStart}{\noindent \textbf{Test:}\newline} %ewentualne możliwe opcje odpowiedzi testowej: A. ? B. ? C. ? D. ? itd.
\newcommand{\testStop}{\newline} %koniec wprowadzania odpowiedzi testowych
\newcommand{\kluczStart}{\noindent \textbf{Test poprawna odpowiedź:}\newline} %klucz, poprawna odpowiedź pytania testowego (jedna literka): A lub B lub C lub D itd.
\newcommand{\kluczStop}{\newline} %koniec poprawnej odpowiedzi pytania testowego 
\newcommand{\wstawGrafike}[2]{\begin{figure}[h] \includegraphics[scale=#2] {#1} \end{figure}} %gdyby była potrzeba wstawienia obrazka, parametry: nazwa pliku, skala (jak nie wiesz co wpisać, to wpisz 1)

\begin{document}
\maketitle


\kategoria{Wikieł/Z1.79f}
\zadStart{Zadanie z Wikieł Z 1.79 f) moja wersja nr [nrWersji]}
%[a]:[2,3,4,5,6,7,8,9,10,11,12,13,14,15,16,17,18,19,20]
%[e]:[1,2,3,4,5,6]
%[w]=random.randint(2,6)
%[b]=[a]*[w]
%[c]=random.randint([w]+1,30)
%[c1]=2*[c]
%[c2]=[c1]*[c1]
%[d]=([a]-[c1])*([a]-[c1])+4*([b]-[c2])
%[d]<0
Rozwiązać nierówność $\sqrt{[a]x+[b]}>x+[c]$
\zadStop
\rozwStart{Barbara Bączek}{}
Zaczniemy od wyznaczenia dziedziny.
$$D: [a]x+[b] \geq 0$$
$$D: x \in [-[w], \infty)$$
Obie strony nierówności: $\sqrt{[a]x+[b]}>x+[c]$ są nieujemne w dziedzinie.
$$[a]x+[b]>x^2+ [c1]x + [c2]$$
$$-x^2 + ([a]-[c1])x +[b]-[c2]>0$$
Skoro $\Delta= [d]<0$ oraz współczynnik przy $x^2$ jest ujemny, to powyższa nierówność jest sprzeczna. Więc $x \in \emptyset$. 
\rozwStop
\odpStart
$x \in \emptyset$
\odpStop
\testStart
A.$x \in [[a],[c1])$
B.$x \in [[a],\infty)$
C.$x \in \emptyset$
D.$x \in (-[w],\infty)$
E.$x \in  ([b],[c2])$
G.$x \in (-\infty,[b]]$
H.$x \in (-\infty,[c2])$
\testStop
\kluczStart
C
\kluczStop



\end{document}