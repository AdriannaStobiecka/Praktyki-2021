\documentclass[12pt, a4paper]{article}
\usepackage[utf8]{inputenc}
\usepackage{polski}

\usepackage{amsthm}  %pakiet do tworzenia twierdzeń itp.
\usepackage{amsmath} %pakiet do niektórych symboli matematycznych
\usepackage{amssymb} %pakiet do symboli mat., np. \nsubseteq
\usepackage{amsfonts}
\usepackage{graphicx} %obsługa plików graficznych z rozszerzeniem png, jpg
\theoremstyle{definition} %styl dla definicji
\newtheorem{zad}{} 
\title{Multizestaw zadań}
\author{Robert Fidytek}
%\date{\today}
\date{}
\newcounter{liczniksekcji}
\newcommand{\kategoria}[1]{\section{#1}} %olreślamy nazwę kateforii zadań
\newcommand{\zadStart}[1]{\begin{zad}#1\newline} %oznaczenie początku zadania
\newcommand{\zadStop}{\end{zad}}   %oznaczenie końca zadania
%Makra opcjonarne (nie muszą występować):
\newcommand{\rozwStart}[2]{\noindent \textbf{Rozwiązanie (autor #1 , recenzent #2): }\newline} %oznaczenie początku rozwiązania, opcjonarnie można wprowadzić informację o autorze rozwiązania zadania i recenzencie poprawności wykonania rozwiązania zadania
\newcommand{\rozwStop}{\newline}                                            %oznaczenie końca rozwiązania
\newcommand{\odpStart}{\noindent \textbf{Odpowiedź:}\newline}    %oznaczenie początku odpowiedzi końcowej (wypisanie wyniku)
\newcommand{\odpStop}{\newline}                                             %oznaczenie końca odpowiedzi końcowej (wypisanie wyniku)
\newcommand{\testStart}{\noindent \textbf{Test:}\newline} %ewentualne możliwe opcje odpowiedzi testowej: A. ? B. ? C. ? D. ? itd.
\newcommand{\testStop}{\newline} %koniec wprowadzania odpowiedzi testowych
\newcommand{\kluczStart}{\noindent \textbf{Test poprawna odpowiedź:}\newline} %klucz, poprawna odpowiedź pytania testowego (jedna literka): A lub B lub C lub D itd.
\newcommand{\kluczStop}{\newline} %koniec poprawnej odpowiedzi pytania testowego 
\newcommand{\wstawGrafike}[2]{\begin{figure}[h] \includegraphics[scale=#2] {#1} \end{figure}} %gdyby była potrzeba wstawienia obrazka, parametry: nazwa pliku, skala (jak nie wiesz co wpisać, to wpisz 1)

\begin{document}
\maketitle


\kategoria{Wikieł/Z2.43}
\zadStart{Zadanie z Wikieł Z 2.43  moja wersja nr [nrWersji]}
%[p1]:[2,3,4,5,6,7,8,9,10]
%[p2]:[2,3,4,5,6,7,8,9,10]
%[p3]:[2,3,4,5,6,7,8,9,10]
%[p4]:[2,3,4,5,6,7,8,9,10]
%[p5]=random.randint(2,10)
%[p6]=random.randint(2,10)
%[p7]=random.randint(2,10)
%[p1p5]=[p1]*[p5]
%[p2p4]=[p2]*[p4]
%[p2p7]=[p2]*[p7]
%[p1p6]=[p1]*[p6]
%[p1p7]=[p1]*[p7]
%[p3p6]=[p3]*[p6]
%[p3p4]=[p3]*[p4]
%[p3p5]=[p3]*[p5]
%[p2p6]=[p2]*[p6]
%[w]=-[p1p5]-[p2p4]
%[wp]=-[w]
%[wx]=-[p3p5]-[p2p6]
%[r1]=[w]/2-[p2p7]
%[r2]=[wp]/2-[p2p7]
%[m1]=round([r1]/[wx],2)
%[m2]=round([r2]/[wx],2)
%math.gcd([p2p7],[w])==1 and [w]!=0  and [p1]*[p6]-[p3]*[p4]==0 and math.gcd([p1p7],[wp])==1

Podać, dla jakich wartości parametru $m$ rozwiązaniem układu równań
$$\left\{\begin{array}{ccc}
[p1]x+[p2]y&=&[p3]m\\
\ [p4]x-[p5]y&=&[p6]m-[p7]
\end{array} \right.$$
jest para x, y spełniająca warunki $|x|\leq\frac{1}{2}, |y|\leq\frac{1}{2}.$
\zadStop

\rozwStart{Maja Szabłowska}{}
Powyższemu układowi równań odpowiadają wyznaczniki:
$$W=\left| \begin{array}{lccr} [p1] & [p2] \\ \ [p4] & -[p5] \end{array}\right| = [p1]\cdot(-[p5]) - [p2]\cdot[p4]=-[p1p5]-[p2p4]=[w]$$

$$W_{x}=\left| \begin{array}{lccr} [p3]m & [p2] \\ \ [p6]m-[p7] & -[p5] \end{array}\right| = [p3]m\cdot(-[p5]) - [p2]\cdot([p6]m-[p7])=-[p3p5]m-[p2p6]m+[p2p7]=[wx]m+[p2p7]$$

$$W_{y}=\left| \begin{array}{lccr} [p1] & [p3]m \\ \ [p4] & [p6]m-[p7] \end{array}\right| = [p1]\cdot([p6]m-[p7]) - [p3]m\cdot[p4]=[p1p6]m-[p1p7]-[p3p4]m=-[p1p7]$$

$$W\neq 0 \iff [w] \neq 0, m\in\mathbb{R} $$

$$x=\frac{W_{x}}{W}=\frac{[wx]m+[p2p7]}{[w]}$$

$$y=\frac{W_{y}}{W}=\frac{[p1p7]}{[wp]}$$
Sprawdzenie warunków:
$$|x|\leq\frac{1}{2}$$

$$\left|\frac{[wx]m+[p2p7]}{[w]}\right|\leq\frac{1}{2}$$

$$\frac{[wx]m+[p2p7]}{[w]}\leq\frac{1}{2} \quad \land \quad \frac{[wx]m+[p2p7]}{[w]}\geq -\frac{1}{2}$$

$$[wx]m+[p2p7]\geq \frac{[w]}{2} \quad \land \quad [wx]m+[p2p7]\leq \frac{[wp]}{2} $$

$$[wx]m\geq \frac{[w]}{2} -[p2p7] \quad \land \quad [wx]m\leq \frac{[wp]}{2}-[p2p7] $$

$$m\leq \frac{[r1]}{[wx]} \quad \land \quad m\geq \frac{[r2]}{[wx]} $$

$$m\leq [m1] \quad \land \quad m\geq [m2]$$

$$m\in [[m2], [m1]]$$
\rozwStop
\odpStart
$m\in [[m2], [m1]]$
\odpStop
\testStart
A.$m\in [[m2], [m1]]$
B.$m\in [[r1], [m1]]$
C.$m\in [[m1], [w]]$
D.$m\in [[wx], [p1p7]]$
E.$m\in [[w], [p2]]$
F.$m\in [0, 1]$
G.$m\in [[p5], [p6]]$
H.$m\in [[p2p4], [p6]]$
I.$m\in \emptyset$
\testStop
\kluczStart
A
\kluczStop
\end{document}
