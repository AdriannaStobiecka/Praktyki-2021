\documentclass[12pt, a4paper]{article}
\usepackage[utf8]{inputenc}
\usepackage{polski}

\usepackage{amsthm}  %pakiet do tworzenia twierdzeń itp.
\usepackage{amsmath} %pakiet do niektórych symboli matematycznych
\usepackage{amssymb} %pakiet do symboli mat., np. \nsubseteq
\usepackage{amsfonts}
\usepackage{graphicx} %obsługa plików graficznych z rozszerzeniem png, jpg
\theoremstyle{definition} %styl dla definicji
\newtheorem{zad}{} 
\title{Multizestaw zadań}
\author{Robert Fidytek}
%\date{\today}
\date{}
\newcounter{liczniksekcji}
\newcommand{\kategoria}[1]{\section{#1}} %olreślamy nazwę kateforii zadań
\newcommand{\zadStart}[1]{\begin{zad}#1\newline} %oznaczenie początku zadania
\newcommand{\zadStop}{\end{zad}}   %oznaczenie końca zadania
%Makra opcjonarne (nie muszą występować):
\newcommand{\rozwStart}[2]{\noindent \textbf{Rozwiązanie (autor #1 , recenzent #2): }\newline} %oznaczenie początku rozwiązania, opcjonarnie można wprowadzić informację o autorze rozwiązania zadania i recenzencie poprawności wykonania rozwiązania zadania
\newcommand{\rozwStop}{\newline}                                            %oznaczenie końca rozwiązania
\newcommand{\odpStart}{\noindent \textbf{Odpowiedź:}\newline}    %oznaczenie początku odpowiedzi końcowej (wypisanie wyniku)
\newcommand{\odpStop}{\newline}                                             %oznaczenie końca odpowiedzi końcowej (wypisanie wyniku)
\newcommand{\testStart}{\noindent \textbf{Test:}\newline} %ewentualne możliwe opcje odpowiedzi testowej: A. ? B. ? C. ? D. ? itd.
\newcommand{\testStop}{\newline} %koniec wprowadzania odpowiedzi testowych
\newcommand{\kluczStart}{\noindent \textbf{Test poprawna odpowiedź:}\newline} %klucz, poprawna odpowiedź pytania testowego (jedna literka): A lub B lub C lub D itd.
\newcommand{\kluczStop}{\newline} %koniec poprawnej odpowiedzi pytania testowego 
\newcommand{\wstawGrafike}[2]{\begin{figure}[h] \includegraphics[scale=#2] {#1} \end{figure}} %gdyby była potrzeba wstawienia obrazka, parametry: nazwa pliku, skala (jak nie wiesz co wpisać, to wpisz 1)

\begin{document}
\maketitle


\kategoria{Wikieł/Z5.10a}
\zadStart{Zadanie z Wikieł Z 5.10 a) moja wersja nr [nrWersji]}
%[a]:[2,3,4,5,6,7,8,9]
%[b]:[2,3,4,5,6,7,8,9]
%[c]:[2,3,4,5,6,7,8,9]
%[a1]=random.randint(2,10)
%[c1]=random.randint(2,10)
%[r]=([a1] + [b])/(2*[a])
%[s]=int([r])
%[t]=([a1] + [b])
%[t1]=-[t]+[b]
%[g]=[s]*[s]
%[d]=-[a]*[g]+[c]
%[r].is_integer()==True and [r]!=0 and [d]<0
Wyznaczyć równanie stycznej do krzywej $y=f(x)$, wiedząc, że styczna ta jest prostopadła do prostej $-x+[a1]y+[c1]=0$, jeżeli $f$ wyraża się wzorem:\\
$f(x)=[a]x^2+[b]x+[c] $
\zadStop
\rozwStart{Joanna Świerzbin}{}
$$
-x+[a1]y+[c1]=0 \quad \Rightarrow \quad y=\frac{x-[c1]}{[a1]}
$$
Ponieważ proste są prostopadłe, to ich współczynniki kierunkowe spełniają warunek: $ a_1 \cdot a_2 =-1$, a więc skoro $a_1 = \frac{1}{[a1]}$ to $ a_2= - [a1]$. \\
Równanie stycznej do wykresu funkcji $y=f(x)$ w punkcie $P(x_0,f(x_0))$:
$$
y-f(x_0)=f'(x_0)(x-x_0)
$$

$$f'(x)=2\cdot[a]x+[b]$$
$$ y = (2\cdot[a]x_0+[b])(x-x_0)+[a]x_0^2+[b]x_0+[c] $$
$$ y = 2\cdot[a]x_0x+[b]x-2\cdot[a]x_0^2-[b]x_0+[a]x_0^2+[b]x_0+[c] $$
$$ y = 2\cdot[a]x_0x+[b]x-[a]x_0^2+[c] $$ \\
$$ 2\cdot [a]x_0+[b] = - [a1] $$
$$ x_0 = - \frac{[a1] + [b]}{2\cdot[a]}$$
$$ x_0 = - [s]$$\\

$$ y = -[t]x+[b]x-[a]\cdot(-[s])^2+[c] $$
$$ y = [t1]x-[a]\cdot[g]+[c] $$
$$ y = [t1]x [d] $$
\rozwStop
\odpStart
$f(x)= [t1]x [d] $
\odpStop
\testStart
A. $f(x)=-[t1]x  [d]$\\
B. $f(x)=[t1]x + [c]$\\
C. $f(x)=[t1]x+[b]$\\
D. $f(x)=-[t]x - [c]$\\
E. $f(x)=[a1]x+ [c]$\\
F. $f(x)=[a1]x [d]$
\testStop
\kluczStart
B
\kluczStop



\end{document}