\documentclass[12pt, a4paper]{article}
\usepackage[utf8]{inputenc}
\usepackage{polski}

\usepackage{amsthm}  %pakiet do tworzenia twierdzeń itp.
\usepackage{amsmath} %pakiet do niektórych symboli matematycznych
\usepackage{amssymb} %pakiet do symboli mat., np. \nsubseteq
\usepackage{amsfonts}
\usepackage{graphicx} %obsługa plików graficznych z rozszerzeniem png, jpg
\theoremstyle{definition} %styl dla definicji
\newtheorem{zad}{} 
\title{Multizestaw zadań}
\author{Robert Fidytek}
%\date{\today}
\date{}
\newcounter{liczniksekcji}
\newcommand{\kategoria}[1]{\section{#1}} %olreślamy nazwę kateforii zadań
\newcommand{\zadStart}[1]{\begin{zad}#1\newline} %oznaczenie początku zadania
\newcommand{\zadStop}{\end{zad}}   %oznaczenie końca zadania
%Makra opcjonarne (nie muszą występować):
\newcommand{\rozwStart}[2]{\noindent \textbf{Rozwiązanie (autor #1 , recenzent #2): }\newline} %oznaczenie początku rozwiązania, opcjonarnie można wprowadzić informację o autorze rozwiązania zadania i recenzencie poprawności wykonania rozwiązania zadania
\newcommand{\rozwStop}{\newline}                                            %oznaczenie końca rozwiązania
\newcommand{\odpStart}{\noindent \textbf{Odpowiedź:}\newline}    %oznaczenie początku odpowiedzi końcowej (wypisanie wyniku)
\newcommand{\odpStop}{\newline}                                             %oznaczenie końca odpowiedzi końcowej (wypisanie wyniku)
\newcommand{\testStart}{\noindent \textbf{Test:}\newline} %ewentualne możliwe opcje odpowiedzi testowej: A. ? B. ? C. ? D. ? itd.
\newcommand{\testStop}{\newline} %koniec wprowadzania odpowiedzi testowych
\newcommand{\kluczStart}{\noindent \textbf{Test poprawna odpowiedź:}\newline} %klucz, poprawna odpowiedź pytania testowego (jedna literka): A lub B lub C lub D itd.
\newcommand{\kluczStop}{\newline} %koniec poprawnej odpowiedzi pytania testowego 
\newcommand{\wstawGrafike}[2]{\begin{figure}[h] \includegraphics[scale=#2] {#1} \end{figure}} %gdyby była potrzeba wstawienia obrazka, parametry: nazwa pliku, skala (jak nie wiesz co wpisać, to wpisz 1)

\begin{document}
\maketitle


\kategoria{Wikieł/Z5.5i}
\zadStart{Zadanie z Wikieł Z 5.5 i) moja wersja nr [nrWersji]}
%[x]:[2,3,4,5,6,7,8,9,10,11,12,13]
%[y]:[2,3,4,5,6,7,8,9,10,11,12,13]
%[a]=random.randint(2,10)
%[b]=random.randint(2,10)
%[c]=random.randint(2,10)
%[d]=random.randint(2,10)
%[e]=random.randint(2,10)
%[m]=(-1)*[b]*2*[c]
%[n]=[d]*[b]*2
%[r]=3*[c]*[a]
%[u]=3*[c]*[b]
%[o]=[d]*[a]
%[s]=[d]*[b]
%[g]=(-1)*([n]+[r])+[s]
%[h]=[m]+[u]
%[p]=[e]*[b]*2
%[g]<0
Korzystając z podstawowych twierdzeń i wzorów, wyznaczyć pochodną funkcji (bez określania zakresu zmienności $x$).\\ $f(x)=\frac{[a]-[b]x^2}{[c]x^3+[d]x-[e]}$.
\zadStop
\rozwStart{Katarzyna Filipowicz}{}
$$f(x)=\frac{[a]-[b]x^2}{[c]x^3+[d]x-[e]}$$
$$f'(x)=\left(\frac{[a]-[b]x^2}{[c]x^3+[d]x-[e]}\right)' = $$
$$ = \frac{\left(-[b]\cdot 2x\right)([c]x^3+[d]x-[e])- \left([a]-[b]x^2\right) (3\cdot [c]x^2+[d])}{([c]x^3+[d]x-[e])^2} = $$
$$
=\frac{-[b]\cdot 2 \cdot [c]x^4+[d]\cdot(-[b])\cdot 2 x^2+[e]\cdot [b]\cdot 2 \cdot x}{([c]x^3+[d]x-[e])^2}-
$$ $$
-\frac{(3\cdot [c]\cdot [a]x^2-3\cdot [c]\cdot[b]\cdot  x^4+[d]\cdot[a]-[d]\cdot [b] \cdot x^2) }{([c]x^3+[d]x-[e])^2}=
$$ $$
=\frac{[m]x^4-[n]x^2+[p]x-[r]x^2+[u]x^4-[o]+[s]x^2 }{([c]x^3+[d]x-[e])^2}=
$$ $$
=\frac{[h]x^4 [g]x^2+[p]x-[o]}{([c]x^3+[d]x-[e])^2}
$$
\rozwStop
\odpStart
$ f'(x)=\frac{[h]x^4 [g]x^2+[p]x-[o]}{([c]x^3+[d]x-[e])^2}$
\odpStop
\testStart
A. $ f'(x)=\frac{[h]x^4 [g]x^2+[p]x-[o]}{([c]x^3+[d]x-[e])^2}$\\
B. $ f'(x)=\frac{[h]x^4 [g]x^2+[p]x-[o]}{[c]x^3+[d]x-[e]}$\\
C. $ f'(x)=\frac{[a]x^4 [b]x^2+[c]x}{([c]x^3+[d]x-[e])^2}$ \\
D. $ f'(x)=[h]x^4 [g]x^2+[p]x-[o]$\\
E. $ f'(x)=\frac{[n]x^4 [g]x^2+[p]x-[o]}{([c]x^3+[d]x-[e])^2}$\\
F. $ f'(x)=\frac{[h]x^4 [g]x^2+[p]x-[o]}{([c]x^3+[d]x-[e])^3}$
\testStop
\kluczStart
A
\kluczStop



\end{document}