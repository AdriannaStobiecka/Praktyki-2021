\documentclass[12pt, a4paper]{article}
\usepackage[utf8]{inputenc}
\usepackage{polski}

\usepackage{amsthm}  %pakiet do tworzenia twierdzeń itp.
\usepackage{amsmath} %pakiet do niektórych symboli matematycznych
\usepackage{amssymb} %pakiet do symboli mat., np. \nsubseteq
\usepackage{amsfonts}
\usepackage{graphicx} %obsługa plików graficznych z rozszerzeniem png, jpg
\theoremstyle{definition} %styl dla definicji
\newtheorem{zad}{} 
\title{Multizestaw zadań}
\author{Robert Fidytek}
%\date{\today}
\date{}\documentclass[12pt, a4paper]{article}
\usepackage[utf8]{inputenc}
\usepackage{polski}

\usepackage{amsthm}  %pakiet do tworzenia twierdzeń itp.
\usepackage{amsmath} %pakiet do niektórych symboli matematycznych
\usepackage{amssymb} %pakiet do symboli mat., np. \nsubseteq
\usepackage{amsfonts}
\usepackage{graphicx} %obsługa plików graficznych z rozszerzeniem png, jpg
\theoremstyle{definition} %styl dla definicji
\newtheorem{zad}{} 
\title{Multizestaw zadań}
\author{Robert Fidytek}
%\date{\today}
\date{}
\newcounter{liczniksekcji}
\newcommand{\kategoria}[1]{\section{#1}} %olreślamy nazwę kateforii zadań
\newcommand{\zadStart}[1]{\begin{zad}#1\newline} %oznaczenie początku zadania
\newcommand{\zadStop}{\end{zad}}   %oznaczenie końca zadania
%Makra opcjonarne (nie muszą występować):
\newcommand{\rozwStart}[2]{\noindent \textbf{Rozwiązanie (autor #1 , recenzent #2): }\newline} %oznaczenie początku rozwiązania, opcjonarnie można wprowadzić informację o autorze rozwiązania zadania i recenzencie poprawności wykonania rozwiązania zadania
\newcommand{\rozwStop}{\newline}                                            %oznaczenie końca rozwiązania
\newcommand{\odpStart}{\noindent \textbf{Odpowiedź:}\newline}    %oznaczenie początku odpowiedzi końcowej (wypisanie wyniku)
\newcommand{\odpStop}{\newline}                                             %oznaczenie końca odpowiedzi końcowej (wypisanie wyniku)
\newcommand{\testStart}{\noindent \textbf{Test:}\newline} %ewentualne możliwe opcje odpowiedzi testowej: A. ? B. ? C. ? D. ? itd.
\newcommand{\testStop}{\newline} %koniec wprowadzania odpowiedzi testowych
\newcommand{\kluczStart}{\noindent \textbf{Test poprawna odpowiedź:}\newline} %klucz, poprawna odpowiedź pytania testowego (jedna literka): A lub B lub C lub D itd.
\newcommand{\kluczStop}{\newline} %koniec poprawnej odpowiedzi pytania testowego 
\newcommand{\wstawGrafike}[2]{\begin{figure}[h] \includegraphics[scale=#2] {#1} \end{figure}} %gdyby była potrzeba wstawienia obrazka, parametry: nazwa pliku, skala (jak nie wiesz co wpisać, to wpisz 1)

\begin{document}
\maketitle


\kategoria{Wikieł/Z1.129d}
\zadStart{Zadanie z Wikieł Z 1.129 d) moja wersja nr [nrWersji]}
%[p1]:[2,3,4,5,6,7,8,9,10]
%[p2]:[2,3,4,5,6,7,8,9,10]
%[p0]:[0]
%[del]=[p1]*[p1]-4*[p2]
%[pdel]=math.sqrt(abs([del]))
%[t1]=round(([p1]-[pdel])/2,2)
%[t2]=round(([p1]+[pdel])/2,2)
%[x1]=round(math.log(abs([t1])+0.0000001),2)
%[x2]=round(math.log(abs([t2])+0.0000001),2)
%[del]>[p0] and [t1]>[p0] and [t2]>[p0] and [t1]<[t2]

Wyznaczyć dziedzinę naturalną funkcji.
$$f(x)=\sqrt{e^{2x}-[p1]e^{x}+[p2]}$$

\zadStop

\rozwStart{Maja Szabłowska}{}
$$e^{2x}-[p1]e^{x}+[p2]\geq 0$$
Wykonujemy podstawienie $e^{x}=t,\quad t>0.$
$$t^{2}-[p1]t+[p2]>0$$
$$\Delta=(-[p1])^{2}-4\cdot1\cdot[p2]=[del] \Rightarrow \sqrt{\Delta}=[pdel]$$
$$t_{1}=\frac{[p1]-[pdel]}{2}=[t1], \quad t_{2}=\frac{[p1]+[pdel]}{2}=[t2]$$
$$t\in(-\infty,[t1])\cup([t2],\infty)$$
$$e^{x}=[t1], \quad e^{x}=[t2]$$
$$\ln[t1]=[x1], \quad \ln[t2]=[x2]$$
$$x\in(-\infty,[x1])\cup([x2],\infty)$$
\rozwStop
\odpStart
$x\in(-\infty,[x1])\cup([x2],\infty)$
\odpStop
\testStart
A.$x\in(-\infty,[x1])\cup([x2],\infty)$
B.$x\in(-\infty,[t1])\cup([t2],\infty)$
C.$x\in(-\infty, 0)$
D.$x\in(-\infty, -[p2]] \cup [[p1],\infty)$
E.$x\in(-\infty, -[x1]] \cup [[x2],\infty)$
F.$x\in(-\infty, -[t1]) \cup ([x2],\infty)$
G.$x\in\emptyset$

\testStop
\kluczStart
A
\kluczStop



\end{document}
