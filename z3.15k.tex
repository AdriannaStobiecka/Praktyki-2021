\documentclass[12pt, a4paper]{article}
\usepackage[utf8]{inputenc}
\usepackage{polski}
\usepackage{amsthm}  %pakiet do tworzenia twierdzeń itp.
\usepackage{amsmath} %pakiet do niektórych symboli matematycznych
\usepackage{amssymb} %pakiet do symboli mat., np. \nsubseteq
\usepackage{amsfonts}
\usepackage{graphicx} %obsługa plików graficznych z rozszerzeniem png, jpg
\theoremstyle{definition} %styl dla definicji
\newtheorem{zad}{} 
\title{Multizestaw zadań}
\author{Patryk Wirkus}
%\date{\today}
\date{}
\newcommand{\kategoria}[1]{\section{#1}}
\newcommand{\zadStart}[1]{\begin{zad}#1\newline}
\newcommand{\zadStop}{\end{zad}}
\newcommand{\rozwStart}[2]{\noindent \textbf{Rozwiązanie (autor #1 , recenzent #2): }\newline}
\newcommand{\rozwStop}{\newline}                                           
\newcommand{\odpStart}{\noindent \textbf{Odpowiedź:}\newline}
\newcommand{\odpStop}{\newline}
\newcommand{\testStart}{\noindent \textbf{Test:}\newline}
\newcommand{\testStop}{\newline}
\newcommand{\kluczStart}{\noindent \textbf{Test poprawna odpowiedź:}\newline}
\newcommand{\kluczStop}{\newline}
\newcommand{\wstawGrafike}[2]{\begin{figure}[h] \includegraphics[scale=#2] {#1} \end{figure}}

\begin{document}
\maketitle

\kategoria{Wikieł/Z3.15k}


\zadStart{Zadanie z Wikieł Z 3.15 k) moja wersja nr 1}
Obliczyć granicę ciągu $a_{n}=(\frac{2n+1}{n})^{5n+3}$.
\zadStop
\rozwStart{Patryk Wirkus}{}
$$\lim\limits_{n\to\infty} a_{n} = \lim\limits_{n\to\infty}(\frac{2n+1}{n})^{5n+3}=$$
$$=\lim\limits_{n\to\infty}\frac{(2+\frac{1}{2n})}{1} \cdot (\frac{2n+1}{n})^{3} = \lim\limits_{n\to\infty} 2^{3} \cdot \lim\limits_{n\to\infty} (2+\frac{1}{2n})^{n} = \infty$$
\rozwStop
\odpStart
$\infty$
\odpStop
\testStart
A.$\infty$
B.$-\infty$
C.$2^{3}$
D.$-2^{3}$
E.$0$
F.$2^{-3}$
G.$-2^{-3}$
H.$2$
I.$-2$
\testStop
\kluczStart
A
\kluczStop



\zadStart{Zadanie z Wikieł Z 3.15 k) moja wersja nr 2}
Obliczyć granicę ciągu $a_{n}=(\frac{2n+1}{n})^{5n+7}$.
\zadStop
\rozwStart{Patryk Wirkus}{}
$$\lim\limits_{n\to\infty} a_{n} = \lim\limits_{n\to\infty}(\frac{2n+1}{n})^{5n+7}=$$
$$=\lim\limits_{n\to\infty}\frac{(2+\frac{1}{2n})}{1} \cdot (\frac{2n+1}{n})^{7} = \lim\limits_{n\to\infty} 2^{7} \cdot \lim\limits_{n\to\infty} (2+\frac{1}{2n})^{n} = \infty$$
\rozwStop
\odpStart
$\infty$
\odpStop
\testStart
A.$\infty$
B.$-\infty$
C.$2^{7}$
D.$-2^{7}$
E.$0$
F.$2^{-7}$
G.$-2^{-7}$
H.$2$
I.$-2$
\testStop
\kluczStart
A
\kluczStop



\zadStart{Zadanie z Wikieł Z 3.15 k) moja wersja nr 3}
Obliczyć granicę ciągu $a_{n}=(\frac{2n+1}{n})^{5n+11}$.
\zadStop
\rozwStart{Patryk Wirkus}{}
$$\lim\limits_{n\to\infty} a_{n} = \lim\limits_{n\to\infty}(\frac{2n+1}{n})^{5n+11}=$$
$$=\lim\limits_{n\to\infty}\frac{(2+\frac{1}{2n})}{1} \cdot (\frac{2n+1}{n})^{11} = \lim\limits_{n\to\infty} 2^{11} \cdot \lim\limits_{n\to\infty} (2+\frac{1}{2n})^{n} = \infty$$
\rozwStop
\odpStart
$\infty$
\odpStop
\testStart
A.$\infty$
B.$-\infty$
C.$2^{11}$
D.$-2^{11}$
E.$0$
F.$2^{-11}$
G.$-2^{-11}$
H.$2$
I.$-2$
\testStop
\kluczStart
A
\kluczStop



\zadStart{Zadanie z Wikieł Z 3.15 k) moja wersja nr 4}
Obliczyć granicę ciągu $a_{n}=(\frac{2n+1}{n})^{5n+13}$.
\zadStop
\rozwStart{Patryk Wirkus}{}
$$\lim\limits_{n\to\infty} a_{n} = \lim\limits_{n\to\infty}(\frac{2n+1}{n})^{5n+13}=$$
$$=\lim\limits_{n\to\infty}\frac{(2+\frac{1}{2n})}{1} \cdot (\frac{2n+1}{n})^{13} = \lim\limits_{n\to\infty} 2^{13} \cdot \lim\limits_{n\to\infty} (2+\frac{1}{2n})^{n} = \infty$$
\rozwStop
\odpStart
$\infty$
\odpStop
\testStart
A.$\infty$
B.$-\infty$
C.$2^{13}$
D.$-2^{13}$
E.$0$
F.$2^{-13}$
G.$-2^{-13}$
H.$2$
I.$-2$
\testStop
\kluczStart
A
\kluczStop



\zadStart{Zadanie z Wikieł Z 3.15 k) moja wersja nr 5}
Obliczyć granicę ciągu $a_{n}=(\frac{2n+1}{n})^{5n+17}$.
\zadStop
\rozwStart{Patryk Wirkus}{}
$$\lim\limits_{n\to\infty} a_{n} = \lim\limits_{n\to\infty}(\frac{2n+1}{n})^{5n+17}=$$
$$=\lim\limits_{n\to\infty}\frac{(2+\frac{1}{2n})}{1} \cdot (\frac{2n+1}{n})^{17} = \lim\limits_{n\to\infty} 2^{17} \cdot \lim\limits_{n\to\infty} (2+\frac{1}{2n})^{n} = \infty$$
\rozwStop
\odpStart
$\infty$
\odpStop
\testStart
A.$\infty$
B.$-\infty$
C.$2^{17}$
D.$-2^{17}$
E.$0$
F.$2^{-17}$
G.$-2^{-17}$
H.$2$
I.$-2$
\testStop
\kluczStart
A
\kluczStop



\zadStart{Zadanie z Wikieł Z 3.15 k) moja wersja nr 6}
Obliczyć granicę ciągu $a_{n}=(\frac{2n+1}{n})^{5n+19}$.
\zadStop
\rozwStart{Patryk Wirkus}{}
$$\lim\limits_{n\to\infty} a_{n} = \lim\limits_{n\to\infty}(\frac{2n+1}{n})^{5n+19}=$$
$$=\lim\limits_{n\to\infty}\frac{(2+\frac{1}{2n})}{1} \cdot (\frac{2n+1}{n})^{19} = \lim\limits_{n\to\infty} 2^{19} \cdot \lim\limits_{n\to\infty} (2+\frac{1}{2n})^{n} = \infty$$
\rozwStop
\odpStart
$\infty$
\odpStop
\testStart
A.$\infty$
B.$-\infty$
C.$2^{19}$
D.$-2^{19}$
E.$0$
F.$2^{-19}$
G.$-2^{-19}$
H.$2$
I.$-2$
\testStop
\kluczStart
A
\kluczStop



\zadStart{Zadanie z Wikieł Z 3.15 k) moja wersja nr 7}
Obliczyć granicę ciągu $a_{n}=(\frac{2n+1}{n})^{5n+23}$.
\zadStop
\rozwStart{Patryk Wirkus}{}
$$\lim\limits_{n\to\infty} a_{n} = \lim\limits_{n\to\infty}(\frac{2n+1}{n})^{5n+23}=$$
$$=\lim\limits_{n\to\infty}\frac{(2+\frac{1}{2n})}{1} \cdot (\frac{2n+1}{n})^{23} = \lim\limits_{n\to\infty} 2^{23} \cdot \lim\limits_{n\to\infty} (2+\frac{1}{2n})^{n} = \infty$$
\rozwStop
\odpStart
$\infty$
\odpStop
\testStart
A.$\infty$
B.$-\infty$
C.$2^{23}$
D.$-2^{23}$
E.$0$
F.$2^{-23}$
G.$-2^{-23}$
H.$2$
I.$-2$
\testStop
\kluczStart
A
\kluczStop



\zadStart{Zadanie z Wikieł Z 3.15 k) moja wersja nr 8}
Obliczyć granicę ciągu $a_{n}=(\frac{2n+1}{n})^{5n+29}$.
\zadStop
\rozwStart{Patryk Wirkus}{}
$$\lim\limits_{n\to\infty} a_{n} = \lim\limits_{n\to\infty}(\frac{2n+1}{n})^{5n+29}=$$
$$=\lim\limits_{n\to\infty}\frac{(2+\frac{1}{2n})}{1} \cdot (\frac{2n+1}{n})^{29} = \lim\limits_{n\to\infty} 2^{29} \cdot \lim\limits_{n\to\infty} (2+\frac{1}{2n})^{n} = \infty$$
\rozwStop
\odpStart
$\infty$
\odpStop
\testStart
A.$\infty$
B.$-\infty$
C.$2^{29}$
D.$-2^{29}$
E.$0$
F.$2^{-29}$
G.$-2^{-29}$
H.$2$
I.$-2$
\testStop
\kluczStart
A
\kluczStop



\zadStart{Zadanie z Wikieł Z 3.15 k) moja wersja nr 9}
Obliczyć granicę ciągu $a_{n}=(\frac{2n+1}{n})^{5n+31}$.
\zadStop
\rozwStart{Patryk Wirkus}{}
$$\lim\limits_{n\to\infty} a_{n} = \lim\limits_{n\to\infty}(\frac{2n+1}{n})^{5n+31}=$$
$$=\lim\limits_{n\to\infty}\frac{(2+\frac{1}{2n})}{1} \cdot (\frac{2n+1}{n})^{31} = \lim\limits_{n\to\infty} 2^{31} \cdot \lim\limits_{n\to\infty} (2+\frac{1}{2n})^{n} = \infty$$
\rozwStop
\odpStart
$\infty$
\odpStop
\testStart
A.$\infty$
B.$-\infty$
C.$2^{31}$
D.$-2^{31}$
E.$0$
F.$2^{-31}$
G.$-2^{-31}$
H.$2$
I.$-2$
\testStop
\kluczStart
A
\kluczStop





\end{document}
