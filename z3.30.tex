\documentclass[12pt, a4paper]{article}
\usepackage[utf8]{inputenc}
\usepackage{polski}

\usepackage{amsthm}  %pakiet do tworzenia twierdzeń itp.
\usepackage{amsmath} %pakiet do niektórych symboli matematycznych
\usepackage{amssymb} %pakiet do symboli mat., np. \nsubseteq
\usepackage{amsfonts}
\usepackage{graphicx} %obsługa plików graficznych z rozszerzeniem png, jpg
\theoremstyle{definition} %styl dla definicji
\newtheorem{zad}{} 
\title{Multizestaw zadań}
\author{Robert Fidytek}
%\date{\today}
\date{}
\newcounter{liczniksekcji}
\newcommand{\kategoria}[1]{\section{#1}} %olreślamy nazwę kateforii zadań
\newcommand{\zadStart}[1]{\begin{zad}#1\newline} %oznaczenie początku zadania
\newcommand{\zadStop}{\end{zad}}   %oznaczenie końca zadania
%Makra opcjonarne (nie muszą występować):
\newcommand{\rozwStart}[2]{\noindent \textbf{Rozwiązanie (autor #1 , recenzent #2): }\newline} %oznaczenie początku rozwiązania, opcjonarnie można wprowadzić informację o autorze rozwiązania zadania i recenzencie poprawności wykonania rozwiązania zadania
\newcommand{\rozwStop}{\newline}                                            %oznaczenie końca rozwiązania
\newcommand{\odpStart}{\noindent \textbf{Odpowiedź:}\newline}    %oznaczenie początku odpowiedzi końcowej (wypisanie wyniku)
\newcommand{\odpStop}{\newline}                                             %oznaczenie końca odpowiedzi końcowej (wypisanie wyniku)
\newcommand{\testStart}{\noindent \textbf{Test:}\newline} %ewentualne możliwe opcje odpowiedzi testowej: A. ? B. ? C. ? D. ? itd.
\newcommand{\testStop}{\newline} %koniec wprowadzania odpowiedzi testowych
\newcommand{\kluczStart}{\noindent \textbf{Test poprawna odpowiedź:}\newline} %klucz, poprawna odpowiedź pytania testowego (jedna literka): A lub B lub C lub D itd.
\newcommand{\kluczStop}{\newline} %koniec poprawnej odpowiedzi pytania testowego 
\newcommand{\wstawGrafike}[2]{\begin{figure}[h] \includegraphics[scale=#2] {#1} \end{figure}} %gdyby była potrzeba wstawienia obrazka, parametry: nazwa pliku, skala (jak nie wiesz co wpisać, to wpisz 1)

\begin{document}
\maketitle


\kategoria{Wikieł/Z3.30}
\zadStart{Zadanie z Wikieł Z 3.30 moja wersja nr [nrWersji]}
%[f]:[1,2,3,4,5,9,10,11,12,13,14,15,16,17,18,19,20]
%[c]:[1,2,4,5,6,7,8,9,10,11,12,17,18,19,20,21,22]
%[b]=random.randint(3,50)
%[a]=random.randint(1,[b]-1)
%[n]=random.randint(4,15)
%[r]=[b]-[a]
%[s]=int((2*[a]+([n]-1)*[r])*[n]/2)
%[d]=[b]+[r]
%[e]=[a]-[r]
%[g]=[a] + [e]
%[h]=2*[s]
%[de]=[g]*[g] +4*[r]*[h]
%[dep]=int(math.sqrt(abs([de])))
%[n2]=int((-[g]+[dep])/(2*abs([r])))
%[o]=2*[n2]
%[o1]=[n2]-1
%[o2]=[n2]+1
%[o3]=[n2]+2
%[o4]=[n2]+3
%[m]=2*[r]
%[a]<[b] and [r]>[a] and [dep]-(math.sqrt([de]))==0 and [dep]>[g] and [n2]-((-[g]+[dep])/(2*[r]))==0 and [f]!=[n2] and [g]<0 and [s]-(2*[a]+([n]-1)*[r])*[n]/2==0
Obliczyć $n$ z równania: [a] + [b] + [d] + \ldots + ([r]n [e]) = [s].
\zadStop
\rozwStart{Barbara Bączek}{}
Zgodnie ze wzorem:
$$S_n=n \cdot \frac{a_1+a_n}{2} $$ 
$$[s]= n \cdot \frac{[a] + [r]n [e]}{2}$$
$$[h]=[g]n + [r]n^2$$
$$[r] n^2  [g]n- [h]=0$$
$$\Delta= {([g])}^2 + 4[r][h]= [de], \hspace{0.2 cm} \sqrt{\Delta}=[dep]$$
$$n_1= \frac{[g]-[dep]}{[m]}<0 , \hspace{0.2 cm} n_2=\frac{[g]+[dep]}{[m]}=[n2] \in \mathbb{N}$$
\rozwStop
\odpStart
$[n2]$
\odpStop
\testStart
A.$[o3]$
B.$[o1]$
C.$[o4]$
D.$[n2]$
E.$[o]$
G.Nie można znaleźć takiego $n$
H.$[o2]$
\testStop
\kluczStart
D
\kluczStop



\end{document}