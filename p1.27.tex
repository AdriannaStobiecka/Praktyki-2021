\documentclass[12pt, a4paper]{article}
\usepackage[utf8]{inputenc}
\usepackage{polski}

\usepackage{amsthm}  %pakiet do tworzenia twierdzeń itp.
\usepackage{amsmath} %pakiet do niektórych symboli matematycznych
\usepackage{amssymb} %pakiet do symboli mat., np. \nsubseteq
\usepackage{amsfonts}
\usepackage{graphicx} %obsługa plików graficznych z rozszerzeniem png, jpg
\theoremstyle{definition} %styl dla definicji
\newtheorem{zad}{} 
\title{Multizestaw zadań}
\author{Robert Fidytek}
%\date{\today}
\date{}
\newcounter{liczniksekcji}
\newcommand{\kategoria}[1]{\section{#1}} %olreślamy nazwę kateforii zadań
\newcommand{\zadStart}[1]{\begin{zad}#1\newline} %oznaczenie początku zadania
\newcommand{\zadStop}{\end{zad}}   %oznaczenie końca zadania
%Makra opcjonarne (nie muszą występować):
\newcommand{\rozwStart}[2]{\noindent \textbf{Rozwiązanie (autor #1 , recenzent #2): }\newline} %oznaczenie początku rozwiązania, opcjonarnie można wprowadzić informację o autorze rozwiązania zadania i recenzencie poprawności wykonania rozwiązania zadania
\newcommand{\rozwStop}{\newline}                                            %oznaczenie końca rozwiązania
\newcommand{\odpStart}{\noindent \textbf{Odpowiedź:}\newline}    %oznaczenie początku odpowiedzi końcowej (wypisanie wyniku)
\newcommand{\odpStop}{\newline}                                             %oznaczenie końca odpowiedzi końcowej (wypisanie wyniku)
\newcommand{\testStart}{\noindent \textbf{Test:}\newline} %ewentualne możliwe opcje odpowiedzi testowej: A. ? B. ? C. ? D. ? itd.
\newcommand{\testStop}{\newline} %koniec wprowadzania odpowiedzi testowych
\newcommand{\kluczStart}{\noindent \textbf{Test poprawna odpowiedź:}\newline} %klucz, poprawna odpowiedź pytania testowego (jedna literka): A lub B lub C lub D itd.
\newcommand{\kluczStop}{\newline} %koniec poprawnej odpowiedzi pytania testowego 
\newcommand{\wstawGrafike}[2]{\begin{figure}[h] \includegraphics[scale=#2] {#1} \end{figure}} %gdyby była potrzeba wstawienia obrazka, parametry: nazwa pliku, skala (jak nie wiesz co wpisać, to wpisz 1)

\begin{document}
\maketitle


\kategoria{Wikieł/p1.27}
\zadStart{Zadanie z Wikieł P 1.27 moja wersja nr [nrWersji]}
%[z]:[1,2,3,4,5,6,7,8,9,10,11,12,13,14]
%[y]:[1,2,3,4,5,6,7,8,9,10,11,12,13,14]
%[b2]:[4,9,16,25,36,49,64,81,100,121,144,169,196]
%[a]=random.randint(2,25)
%[b]=random.randint(1,100)
%[d]=[b2]/(4*[a])
%[c]=int([d])
%[b3]=int(math.sqrt([b2]))
%[w1]=-[b3]-[b]
%[w2]=[b3]-[b]
%[d].is_integer()==True
Wyznaczyć wartości parametru $m$, dla których trójmian kwadratowy
$$
y=-[a]x^2+(m+[b])x-[c]
$$
ma jeden pierwiastek.
\zadStop
\rozwStart{Katarzyna Filipowicz}{}
Rozważany trójmian będzie miał dokładnie jeden pierwiastek wtedy i tylko wtedy gdy $\Delta=0$, tj. gdy
$$
\Delta=(m+[b])^2-[b2]=0 \quad \Leftrightarrow \quad (m+[b])^2=[b2] \quad \Leftrightarrow \quad |m+[b]|=[b3] 
\quad \Leftrightarrow \quad
$$ $$
\quad \Leftrightarrow \quad m+[b]=-[b3] \lor  m+[b]=[b3] \quad \Leftrightarrow \quad m=[w1] \lor m=[w2]
$$
\rozwStop
\odpStart
$ m=[w1] \lor m=[w2]$
\odpStop
\testStart
A.$ m=[w1] \lor m=[w2]$
B.$m=[w1]$
C.$m=0$
D.$m=-[b]$
E.$m \in R$
F.$7$
G.$m \in R \backslash \{[w1];[w2]\}$
H.$-5$
I.$\infty$
\testStop
\kluczStart
A
\kluczStop



\end{document}