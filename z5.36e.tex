\documentclass[12pt, a4paper]{article}
\usepackage[utf8]{inputenc}
\usepackage{polski}

\usepackage{amsthm}  %pakiet do tworzenia twierdzeń itp.
\usepackage{amsmath} %pakiet do niektórych symboli matematycznych
\usepackage{amssymb} %pakiet do symboli mat., np. \nsubseteq
\usepackage{amsfonts}
\usepackage{graphicx} %obsługa plików graficznych z rozszerzeniem png, jpg
\theoremstyle{definition} %styl dla definicji
\newtheorem{zad}{} 
\title{Multizestaw zadań}
\author{Robert Fidytek}
%\date{\today}
\date{}
\newcounter{liczniksekcji}
\newcommand{\kategoria}[1]{\section{#1}} %olreślamy nazwę kateforii zadań
\newcommand{\zadStart}[1]{\begin{zad}#1\newline} %oznaczenie początku zadania
\newcommand{\zadStop}{\end{zad}}   %oznaczenie końca zadania
%Makra opcjonarne (nie muszą występować):
\newcommand{\rozwStart}[2]{\noindent \textbf{Rozwiązanie (autor #1 , recenzent #2): }\newline} %oznaczenie początku rozwiązania, opcjonarnie można wprowadzić informację o autorze rozwiązania zadania i recenzencie poprawności wykonania rozwiązania zadania
\newcommand{\rozwStop}{\newline}                                            %oznaczenie końca rozwiązania
\newcommand{\odpStart}{\noindent \textbf{Odpowiedź:}\newline}    %oznaczenie początku odpowiedzi końcowej (wypisanie wyniku)
\newcommand{\odpStop}{\newline}                                             %oznaczenie końca odpowiedzi końcowej (wypisanie wyniku)
\newcommand{\testStart}{\noindent \textbf{Test:}\newline} %ewentualne możliwe opcje odpowiedzi testowej: A. ? B. ? C. ? D. ? itd.
\newcommand{\testStop}{\newline} %koniec wprowadzania odpowiedzi testowych
\newcommand{\kluczStart}{\noindent \textbf{Test poprawna odpowiedź:}\newline} %klucz, poprawna odpowiedź pytania testowego (jedna literka): A lub B lub C lub D itd.
\newcommand{\kluczStop}{\newline} %koniec poprawnej odpowiedzi pytania testowego 
\newcommand{\wstawGrafike}[2]{\begin{figure}[h] \centering \includegraphics[scale=#2] {#1} \end{figure}} %gdyby była potrzeba wstawienia obrazka, parametry: nazwa pliku, skala (jak nie wiesz co wpisać, to wpisz 1)

\begin{document}
\maketitle

\kategoria{Wikieł/Z5.36e}

\zadStart{Zadanie z Wikieł Z 5.36 e) moja wersja nr [nrWersji]}
%[a]:[2,3,4,5,6,7,8,9,10,11]
%[b]:[2,3,5,6,7,8,10,11]
%[e]=[a]*2
%[f]=[b]*3
Wyznaczyć przedziały wypukłości i wklęsłości podanej funkcji.
$$y =\frac{[a]x}{x^2 - [b]}$$
\zadStop

\rozwStart{Natalia Danieluk}{}
Postępujemy następująco:
\begin{enumerate}
\item Określamy dziedzinę funkcji: $\quad \mathcal{D}_f=\mathbb{R} \backslash \{-\sqrt{[b]},\sqrt{[b]}\}$. \\
\item Obliczamy pochodne: 
$$\quad f'(x) = \frac{[a]x'(x^2 - [b]) - (x^2 - [b])'[a]x}{(x^2 - [b])^2} = \frac{-[a](x^2+[b])}{(x^2 - [b])^2},$$
$$\quad f''(x) = \frac{-[a](x^2+[b])'(x^2 - [b])^2 - ((x^2 - [b])^2)'(-[a](x^2+[b]))}{(x^2 - [b])^4} = \frac{[e]x(x^2+[f])}{(x^2 - [b])^3}$$
i określamy ich dziedziny: $\quad \mathcal{D}_{f'}=\mathcal{D}_{f''}=\mathbb{R} \backslash \{-\sqrt{[b]},\sqrt{[b]}\}$. \\
\item Badamy znak $f''$. \\
Zauważmy, że dla każdego $x \in \mathcal{D}_f$ mamy $[e](x^2+[f]) > 0$. \\
Wystarczy zatem zbadać znak czynnika $\frac{x}{(x^2 - [b])^3}$. \\
$$ \frac{x}{(x^2 - [b])^3} = 0 \Leftrightarrow x = 0 $$
Należy jeszcze sprawdzić, czy w punktach nieciągłości, tj. $x=-\sqrt{[b]}$ i $x=\sqrt{[b]}$ występują asymptoty pionowe. Aby to zrobić liczymy ich granice z prawej i lewej strony:
$$\lim_{x \rightarrow -\sqrt{[b]}^+} f(x) = \lim_{x \rightarrow -\sqrt{[b]}^+} \frac{[a]x}{x^2 - [b]} = \frac{-[a]\sqrt{[b]}}{[b]^{+} - [b]} 
= \Bigg[ \frac{-[a]\sqrt{[b]}}{0^{+}}\Bigg] = -\infty$$ 
$$\lim_{x \rightarrow -\sqrt{[b]}^-} f(x) = \lim_{x \rightarrow -\sqrt{[b]}^-} \frac{[a]x}{x^2 - [b]} = \frac{-[a]\sqrt{[b]}}{[b]^{-} - [b]} 
= \Bigg[ \frac{-[a]\sqrt{[b]}}{0^{-}}\Bigg] = \infty$$ 
$$\lim_{x \rightarrow \sqrt{[b]}^+} f(x) = \lim_{x \rightarrow \sqrt{[b]}^+} \frac{[a]x}{x^2 - [b]} = \frac{[a]\sqrt{[b]}}{[b]^{+} - [b]} 
= \Bigg[ \frac{[a]\sqrt{[b]}}{0^{+}}\Bigg] = \infty$$ 
$$\lim_{x \rightarrow \sqrt{[b]}^-} f(x) = \lim_{x \rightarrow \sqrt{[b]}^-} \frac{[a]x}{x^2 - [b]} = \frac{[a]\sqrt{[b]}}{[b]^{-} - [b]} 
= \Bigg[ \frac{[a]\sqrt{[b]}}{0^{-}}\Bigg] = -\infty$$ 
Otrzymujemy dwie asymptoty obustronne. Wówczas wykres pochodnej wygląda mniej więcej jak poniżej.
\wstawGrafike{wykres_z5_36e.png}{0.75}
	\begin{enumerate}
	\item $f''(x) > 0 \Leftrightarrow x \in (-\sqrt{[b]},0) \cup (\sqrt{[b]},\infty)$ i w tym przedziale wykres funkcji $f$ jest wypukły (wypukły w dół) $ \smile $ \\
	\item $f''(x) < 0 \Leftrightarrow x \in (-\infty,-\sqrt{[b]}) \cup (0,\sqrt{[b]})$ i w tym przedziale wykres funkcji $f$ jest wklęsły (wypukły w górę) $ \frown $
	\end{enumerate}
\end{enumerate}
.
\rozwStop

\odpStart
Funkcja jest wypukła w $(-\sqrt{[b]},0) \cup (\sqrt{[b]},\infty)$ i wklęsła w $(-\infty,-\sqrt{[b]}) \cup (0,\sqrt{[b]})$.
\odpStop

\testStart
A. Funkcja jest wypukła w całej dziedzinie.
B. Funkcja jest wklęsła w całej dziedzinie.
C. Funkcja nie jest ani wypukła, ani wklęsła.
D. Funkcja jest wypukła w $(-\infty,-\sqrt{[b]}) \cup (0,\sqrt{[b]})$ i wklęsła w $(-\sqrt{[b]},0) \cup (\sqrt{[b]},\infty)$.
E. Funkcja jest wypukła w $(-\sqrt{[b]},0) \cup (\sqrt{[b]},\infty)$ i wklęsła w $(-\infty,-\sqrt{[b]}) \cup (0,\sqrt{[b]})$.
F. Funkcja jest wypukła w $(0,\infty)$ i wklęsła w $(-\infty,0)$.
\testStop

\kluczStart
E
\kluczStop

\end{document}