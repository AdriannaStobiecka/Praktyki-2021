\documentclass[12pt, a4paper]{article}
\usepackage[utf8]{inputenc}
\usepackage{polski}

\usepackage{amsthm}  %pakiet do tworzenia twierdzeń itp.
\usepackage{amsmath} %pakiet do niektórych symboli matematycznych
\usepackage{amssymb} %pakiet do symboli mat., np. \nsubseteq
\usepackage{amsfonts}
\usepackage{graphicx} %obsługa plików graficznych z rozszerzeniem png, jpg
\theoremstyle{definition} %styl dla definicji
\newtheorem{zad}{} 
\title{Multizestaw zadań}
\author{Jacek Jabłoński}
%\date{\today}
\date{}
\newcounter{liczniksekcji}
\newcommand{\kategoria}[1]{\section{#1}} %olreślamy nazwę kateforii zadań
\newcommand{\zadStart}[1]{\begin{zad}#1\newline} %oznaczenie początku zadania
\newcommand{\zadStop}{\end{zad}}   %oznaczenie końca zadania
%Makra opcjonarne (nie muszą występować):
\newcommand{\rozwStart}[2]{\noindent \textbf{Rozwiązanie (autor #1 , recenzent #2): }\newline} %oznaczenie początku rozwiązania, opcjonarnie można wprowadzić informację o autorze rozwiązania zadania i recenzencie poprawności wykonania rozwiązania zadania
\newcommand{\rozwStop}{\newline}                                            %oznaczenie końca rozwiązania
\newcommand{\odpStart}{\noindent \textbf{Odpowiedź:}\newline}    %oznaczenie początku odpowiedzi końcowej (wypisanie wyniku)
\newcommand{\odpStop}{\newline}                                             %oznaczenie końca odpowiedzi końcowej (wypisanie wyniku)
\newcommand{\testStart}{\noindent \textbf{Test:}\newline} %ewentualne możliwe opcje odpowiedzi testowej: A. ? B. ? C. ? D. ? itd.
\newcommand{\testStop}{\newline} %koniec wprowadzania odpowiedzi testowych
\newcommand{\kluczStart}{\noindent \textbf{Test poprawna odpowiedź:}\newline} %klucz, poprawna odpowiedź pytania testowego (jedna literka): A lub B lub C lub D itd.
\newcommand{\kluczStop}{\newline} %koniec poprawnej odpowiedzi pytania testowego 
\newcommand{\wstawGrafike}[2]{\begin{figure}[h] \includegraphics[scale=#2] {#1} \end{figure}} %gdyby była potrzeba wstawienia obrazka, parametry: nazwa pliku, skala (jak nie wiesz co wpisać, to wpisz 1)

\begin{document}
\maketitle


\kategoria{Wikieł/P5.2a}
\zadStart{Zadanie z Wikieł P 5.2a) moja wersja nr [nrWersji]}
%[p1]:[2,4,8,16,32,64,128,256,512,1024]
%[p2]:[2,4,8,16,32,64,128,256,512,1024]
%[a]=random.randint(2,8)
%[b]=random.randint(2,6)
%[c]=random.randint(2,6)
%[d]=random.randint(2,6)
%[w1]=[a]*2
%[f1]=[b]+[d]
%[f2]=random.randint(1,14)
%[f3]=[w1]+[f2]
Korzystając z podanych wzorów i twierdzeń, wyznacz pochodną funkcji:
a) $f(x)=[a]x^2 + [b]x + \frac{[c]}{x} + [d]$
\zadStop
\rozwStart{Jacek Jabłoński}{}
$$f'(x) = ([a]x^2 + [b]x + \frac{[c]}{x} + [d])' = [a](x^2)' + [b](x)' + [c](x^{-1})' + ([d])' = $$
$$= [w1]x + [b] - [c]x^{-2} + 0 = [w1]x - \frac{[c]}{x^2} + [b] $$
\rozwStop
\odpStart
$[w1]x - \frac{[c]}{x^2} + [b]$
\odpStop
\testStart
A. $$[w1]x - \frac{[c]}{x^2} + [b]$$
B. $$[a]x + \frac{[c]}{x^{-2}} + [f1]$$
C. $$[w1]x - \frac{[c]}{x^3} + [f1]$$
D. $$[w1]x^2 + \frac{[c]}{x} + [f2]$$
E. $$[c]{x^2} + [b]$$
F. $$[f1]x + [f2]$$
G. $$[f3]x + \frac{[c]}{x^2} + [b]$$
H. $$[d]x - \frac{[c]}{x} + [b]$$
I. $$ [b]x$$
\testStop
\kluczStart
A
\kluczStop



\end{document}