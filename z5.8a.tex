\documentclass[12pt, a4paper]{article}
\usepackage[utf8]{inputenc}
\usepackage{polski}

\usepackage{amsthm}  %pakiet do tworzenia twierdzeń itp.
\usepackage{amsmath} %pakiet do niektórych symboli matematycznych
\usepackage{amssymb} %pakiet do symboli mat., np. \nsubseteq
\usepackage{amsfonts}
\usepackage{graphicx} %obsługa plików graficznych z rozszerzeniem png, jpg
\theoremstyle{definition} %styl dla definicji
\newtheorem{zad}{} 
\title{Multizestaw zadań}
\author{Robert Fidytek}
%\date{\today}
\date{}
\newcounter{liczniksekcji}
\newcommand{\kategoria}[1]{\section{#1}} %olreślamy nazwę kateforii zadań
\newcommand{\zadStart}[1]{\begin{zad}#1\newline} %oznaczenie początku zadania
\newcommand{\zadStop}{\end{zad}}   %oznaczenie końca zadania
%Makra opcjonarne (nie muszą występować):
\newcommand{\rozwStart}[2]{\noindent \textbf{Rozwiązanie (autor #1 , recenzent #2): }\newline} %oznaczenie początku rozwiązania, opcjonarnie można wprowadzić informację o autorze rozwiązania zadania i recenzencie poprawności wykonania rozwiązania zadania
\newcommand{\rozwStop}{\newline}                                            %oznaczenie końca rozwiązania
\newcommand{\odpStart}{\noindent \textbf{Odpowiedź:}\newline}    %oznaczenie początku odpowiedzi końcowej (wypisanie wyniku)
\newcommand{\odpStop}{\newline}                                             %oznaczenie końca odpowiedzi końcowej (wypisanie wyniku)
\newcommand{\testStart}{\noindent \textbf{Test:}\newline} %ewentualne możliwe opcje odpowiedzi testowej: A. ? B. ? C. ? D. ? itd.
\newcommand{\testStop}{\newline} %koniec wprowadzania odpowiedzi testowych
\newcommand{\kluczStart}{\noindent \textbf{Test poprawna odpowiedź:}\newline} %klucz, poprawna odpowiedź pytania testowego (jedna literka): A lub B lub C lub D itd.
\newcommand{\kluczStop}{\newline} %koniec poprawnej odpowiedzi pytania testowego 
\newcommand{\wstawGrafike}[2]{\begin{figure}[h] \includegraphics[scale=#2] {#1} \end{figure}} %gdyby była potrzeba wstawienia obrazka, parametry: nazwa pliku, skala (jak nie wiesz co wpisać, to wpisz 1)

\begin{document}
\maketitle


\kategoria{Wikieł/Z5.8a}
\zadStart{Zadanie z Wikieł Z 5.8 a) moja wersja nr [nrWersji]}
%[x]:[2,3,4,5,6,7,8,9,10,11,12,13]
%[y]:[2,3,4,5,6,7,8,9,10,11,12,13]
%[a]=random.randint(2,10)
%[b]=random.randint(2,10)
%[c]=random.randint(2,10)
%[d]=random.randint(2,10)
%[m]=([a]*[d]*[d])+([b]*[d])-[c]
%[n]=2*[a]
%[r]=([n]*[d])+[b]
%[p]=[r]*[d]
%[t]=[m]-[p]
%[t]<0
Wyznaczyć równanie stycznej do krzywej $y=f(x)$ w punkcie $P(x_0,f(x_0))$, jeśli\\
$f(x)=[a]x^2+[b]x-[c]$, $x_0=[d]$
\zadStop
\rozwStart{Katarzyna Filipowicz}{}
$$
f([d])=[a]\cdot ([d])^2+[b]\cdot[d]-[c]=[m]
$$ $$
f'(x)=2\cdot[a]\cdot x+[b]=[n]x+[b]
$$ $$
f'([d])=[n]\cdot [d]+[b]=[r]
$$
Równanie stycznej do wykresu funkcji $y=f(x)$ w punkcie $P(x_0,f(x_0))$:
$$
y-f(x_0)=f'(x_0)(x-x_0)
$$
A więc:
$$
y-[m]=[r](x-[d])
$$
$$
y-[m]=[r]x-[p]
$$ $$
y=[r]x [t]
$$
\rozwStop
\odpStart
$f(x)=[r]x [t]$
\odpStop
\testStart
A. $ f(x)=[r]x [t] $\\
B. $ f(x)=-[r]x [t]$\\
C. $ f(x)=[r]x $ \\
D. $ f(x)=[n]x +[b]$\\
E. $ f(x)=[a]x+[c]$\\
F. $ f(x)=[m]x [t]$
\testStop
\kluczStart
A
\kluczStop



\end{document}