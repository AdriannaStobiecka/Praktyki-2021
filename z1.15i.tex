\documentclass[12pt, a4paper]{article}
\usepackage[utf8]{inputenc}
\usepackage{polski}

\usepackage{amsthm}  %pakiet do tworzenia twierdzeń itp.
\usepackage{amsmath} %pakiet do niektórych symboli matematycznych
\usepackage{amssymb} %pakiet do symboli mat., np. \nsubseteq
\usepackage{amsfonts}
\usepackage{graphicx} %obsługa plików graficznych z rozszerzeniem png, jpg
\theoremstyle{definition} %styl dla definicji
\newtheorem{zad}{} 
\title{Multizestaw zadań}
\author{Robert Fidytek}
%\date{\today}
\date{}
\newcounter{liczniksekcji}
\newcommand{\kategoria}[1]{\section{#1}} %olreślamy nazwę kateforii zadań
\newcommand{\zadStart}[1]{\begin{zad}#1\newline} %oznaczenie początku zadania
\newcommand{\zadStop}{\end{zad}}   %oznaczenie końca zadania
%Makra opcjonarne (nie muszą występować):
\newcommand{\rozwStart}[2]{\noindent \textbf{Rozwiązanie (autor #1 , recenzent #2): }\newline} %oznaczenie początku rozwiązania, opcjonarnie można wprowadzić informację o autorze rozwiązania zadania i recenzencie poprawności wykonania rozwiązania zadania
\newcommand{\rozwStop}{\newline}                                            %oznaczenie końca rozwiązania
\newcommand{\odpStart}{\noindent \textbf{Odpowiedź:}\newline}    %oznaczenie początku odpowiedzi końcowej (wypisanie wyniku)
\newcommand{\odpStop}{\newline}                                             %oznaczenie końca odpowiedzi końcowej (wypisanie wyniku)
\newcommand{\testStart}{\noindent \textbf{Test:}\newline} %ewentualne możliwe opcje odpowiedzi testowej: A. ? B. ? C. ? D. ? itd.
\newcommand{\testStop}{\newline} %koniec wprowadzania odpowiedzi testowych
\newcommand{\kluczStart}{\noindent \textbf{Test poprawna odpowiedź:}\newline} %klucz, poprawna odpowiedź pytania testowego (jedna literka): A lub B lub C lub D itd.
\newcommand{\kluczStop}{\newline} %koniec poprawnej odpowiedzi pytania testowego 
\newcommand{\wstawGrafike}[2]{\begin{figure}[h] \includegraphics[scale=#2] {#1} \end{figure}} %gdyby była potrzeba wstawienia obrazka, parametry: nazwa pliku, skala (jak nie wiesz co wpisać, to wpisz 1)

\begin{document}
\maketitle


\kategoria{Wikieł/Z1.15 i}
\zadStart{Zadanie z Wikieł Z 1.15 i) moja wersja nr [nrWersji]}
%[a]:[1,2,3,4,5,6,7,8,9]
%[b]:[2,3,4,5,6,7,8,9]
%[c]:[1,2,3,4,5,6,7,8,9]
%[d]:[1,2,3,4,5,6,7,8,9]
%[acm]=-[a]+[c]
%[acp]=[a]+[c]
%[dcm]=-[d]+[c]
%[dcp]=[d]+[c]
%math.gcd([acm],[b])==1 and math.gcd([acp],[b])==1 and math.gcd([dcm],[b])==1 and math.gcd([dcp],[b])==1 and [a]<[d] and [acm]!=0 and [acp] !=0 and [dcm]!=0 and [dcp]!=0


Rozwiązać nierówności  $[a] \leq |{[b]x-[c]}| < [d]$.
\zadStop
\rozwStart{Joanna Świerzbin}{}
$$[a] \leq |{[b]x-[c]}| < [d]$$
$$ |{[b]x-[c]}| \geq [a] \land |{[b]x-[c]}| < [d]$$
$$ ( {[b]x-[c]} \leq -[a] \vee {[b]x-[c]} \geq [a] ) \land ( {[b]x-[c]} > -[d] \land {[b]x-[c]} < [d] )$$
$$ ( {[b]x} \leq [acm] \vee {[b]x} \geq [acp] ) \land ( {[b]x} > [dcm] \land {[b]x} < [dcp] )$$
$$ \left( {x} \leq \frac{[acm]}{[b]} \vee {x} \geq \frac{[acp]}{[b]} \right) \land \left( {x} > \frac{[dcm]}{[b]} \land {x} < \frac{[dcp]}{[b]} \right)$$
$$ x \in \left( -\infty, \frac{[acm]}{[b]} \right] \cup \left[ \frac{[acp]}{[b]}, \infty \right) \land x \in \left( \frac{[dcm]}{[b]} ,\frac{[dcp]}{[b]} \right)$$
$$ x \in \left( \frac{[dcm]}{[b]}, \frac{[acm]}{[b]} \right] \cup \left[ \frac{[acp]}{[b]}, \frac{[dcp]}{[b]} \right)$$
\rozwStop
\odpStart
$ x \in \left( \frac{[dcm]}{[b]}, \frac{[acm]}{[b]} \right] \cup \left[ \frac{[acp]}{[b]}, \frac{[dcp]}{[b]} \right)$
\odpStop
\testStart
A. $ x \in \left( \frac{[dcm]}{[b]}, \frac{[acm]}{[b]} \right] \cup \left[ \frac{[acp]}{[b]}, \frac{[dcp]}{[b]} \right)$\\
B. $ x \in \left( \frac{[dcm]}{[b]}, \frac{[acm]}{[b]} \right] $\\
C. $ x \in  \left[ \frac{[acp]}{[b]}, \frac{[dcp]}{[b]} \right)$\\
D. $ x \in \left( -\infty, \frac{[acm]}{[b]} \right] \cup \left[ \frac{[acp]}{[b]}, \frac{[dcp]}{[b]} \right)$\\
E. $ x \in \emptyset $
\testStop
\kluczStart
A
\kluczStop



\end{document}