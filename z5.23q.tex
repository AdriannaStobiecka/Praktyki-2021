\documentclass[12pt, a4paper]{article}
\usepackage[utf8]{inputenc}
\usepackage{polski}

\usepackage{amsthm}  %pakiet do tworzenia twierdzeń itp.
\usepackage{amsmath} %pakiet do niektórych symboli matematycznych
\usepackage{amssymb} %pakiet do symboli mat., np. \nsubseteq
\usepackage{amsfonts}
\usepackage{graphicx} %obsługa plików graficznych z rozszerzeniem png, jpg
\theoremstyle{definition} %styl dla definicji
\newtheorem{zad}{} 
\title{Multizestaw zadań}
\author{Radosław Grzyb}
%\date{\today}
\date{}
\newcounter{liczniksekcji}
\newcommand{\kategoria}[1]{\section{#1}} %olreślamy nazwę kateforii zadań
\newcommand{\zadStart}[1]{\begin{zad}#1\newline} %oznaczenie początku zadania
\newcommand{\zadStop}{\end{zad}}   %oznaczenie końca zadania
%Makra opcjonarne (nie muszą występować):
\newcommand{\rozwStart}[2]{\noindent \textbf{Rozwiązanie (autor #1 , recenzent #2): }\newline} %oznaczenie początku rozwiązania, opcjonarnie można wprowadzić informację o autorze rozwiązania zadania i recenzencie poprawności wykonania rozwiązania zadania
\newcommand{\rozwStop}{\newline}                                            %oznaczenie końca rozwiązania
\newcommand{\odpStart}{\noindent \textbf{Odpowiedź:}\newline}    %oznaczenie początku odpowiedzi końcowej (wypisanie wyniku)
\newcommand{\odpStop}{\newline}                                             %oznaczenie końca odpowiedzi końcowej (wypisanie wyniku)
\newcommand{\testStart}{\noindent \textbf{Test:}\newline} %ewentualne możliwe opcje odpowiedzi testowej: A. ? B. ? C. ? D. ? itd.
\newcommand{\testStop}{\newline} %koniec wprowadzania odpowiedzi testowych
\newcommand{\kluczStart}{\noindent \textbf{Test poprawna odpowiedź:}\newline} %klucz, poprawna odpowiedź pytania testowego (jedna literka): A lub B lub C lub D itd.
\newcommand{\kluczStop}{\newline} %koniec poprawnej odpowiedzi pytania testowego 
\newcommand{\wstawGrafike}[2]{\begin{figure}[h] \includegraphics[scale=#2] {#1} \end{figure}} %gdyby była potrzeba wstawienia obrazka, parametry: nazwa pliku, skala (jak nie wiesz co wpisać, to wpisz 1)

\begin{document}
\maketitle


\kategoria{Wikieł/Z5.23q}
\zadStart{Zadanie z Wikieł Z 5.23 q) moja wersja nr [nrWersji]}
%[a]:[2,3,4,5,6,7,8,9,10,11,12]
%[b]:[2,3,4,5,6,7,8,9,10,11,12]
%[c]=2*[a]
%[d]=-([b]**2)
%[e]=4*[a]
%math.gcd([b],[c])==1
Znaleźć ekstrema lokalne funkcji:
$$f(x)=[a]ln^2(x)-[b]ln(x)=ln(x)([a]ln(x)-[b])$$
\zadStop
\rozwStart{Klaudia Klejdysz}{}
Funkcja $f(x)$ jest różniczkowalna w swojej dziedzinie, czyli w zbiorze $D_f=\mathbb{R}^{+}$. Obliczamy pochodną:
$$f'(x)=\frac{1}{x}([a]ln(x)-[b])+ln(x)(\frac{[a]}{x})=\frac{[a]}{x}ln(x)-\frac{[b]}{x}+\frac{[a]}{x}ln(x)=$$
$$=\frac{[c]}{x}ln(x)-\frac{[b]}{x}=\frac{1}{x}([c]ln(x)-[b])$$
\noindent Z warunku koniecznego istnienia ekstremum mamy do rozwiązania równanie $f'(x)=0$. Stąd otrzymujemy tylko jeden punkt stacjonarny:
$$\frac{1}{x}([c]ln(x)-[b])=0\Leftrightarrow[c]ln(x)-[b]=0\Leftrightarrow ln(x)=\frac{[b]}{[c]}\Leftrightarrow e^{ln(x)}=e^{\frac{[b]}{[c]}}\Leftrightarrow x=e^{\frac{[b]}{[c]}}$$
Na podstawie twierdzenia Fermata wnioskujemy, że funkcja $f(x)$ może mieć ekstremum tylko w punkcie $x_0=e^{\frac{[b]}{[c]}}$.\\
Aby zbadać znak pochodnej, rozwiązujemy nierówności $f'(x)>0$ i $f'(x)<0$:
$$f'(x)>0\Leftrightarrow\frac{1}{x}([c]ln(x)-[b])>0\Leftrightarrow x>e^{\frac{[b]}{[c]}}\Leftrightarrow x\in(e^{\frac{[b]}{[c]}},\infty)$$
$$f'(x)<0\Leftrightarrow\frac{1}{x}([c]ln(x)-[b])<0\Leftrightarrow x<e^{\frac{[b]}{[c]}}\Leftrightarrow x\in(-\infty,e^{\frac{[b]}{[c]}})$$

\noindent Ustalamy znaki pochodnych w otoczeniu punktu $x_0=e^{\frac{[b]}{[c]}}$. Zestawimy wyniki w tabeli:
\begin{table}[h]
\centering
\begin{tabular}{|l|l|l|l|}
\hline
x     & $(-\infty,e^{\frac{[b]}{[c]}})$ & $e^{\frac{[b]}{[c]}}$ & $(e^{\frac{[b]}{[c]}},\infty)$ \\ \hline
$f'(x)$ & -                          & 0 & +                         \\ \hline
$f(x)$  & $\searrow$    & $\frac{[d]}{[e]}$ & $\nearrow$   \\ \hline
\end{tabular}
\end{table}\\
Zgodnie z twierdzeniem funkcja osiąga minimum lokalne w punkcie $x_0=e^{\frac{[b]}{[c]}}$.
\rozwStop
\odpStart
Minimum lokalne w punkcie $x_0=e^{\frac{[b]}{[c]}}$.
\odpStop
\testStart
A. Minimum lokalne w punkcie $x_0=e^{\frac{[b]}{[c]}}$.\\
B. Maksimum lokalne w punkcie $x_0=-e^{\frac{[b]}{[c]}}$.\\
C. Minimum lokalne w punkcie $x_0=[a]$.\\
D. Maksimum lokalne w punkcie $x_0=e^{\frac{[b]}{[c]}}$.\\
E. Maksimum lokalne w punkcie $x_0=-e^{\frac{[b]}{[c]}}$.\\
F. Minimum lokalne w punkcie $x_0=[b]$
\testStop
\kluczStart
A
\kluczStop


\end{document}
