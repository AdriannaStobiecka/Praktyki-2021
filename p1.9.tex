\documentclass[12pt, a4paper]{article}
\usepackage[utf8]{inputenc}
\usepackage{polski}

\usepackage{amsthm}  %pakiet do tworzenia twierdzeń itp.
\usepackage{amsmath} %pakiet do niektórych symboli matematycznych
\usepackage{amssymb} %pakiet do symboli mat., np. \nsubseteq
\usepackage{amsfonts}
\usepackage{graphicx} %obsługa plików graficznych z rozszerzeniem png, jpg
\theoremstyle{definition} %styl dla definicji
\newtheorem{zad}{} 
\title{Multizestaw zadań}
\author{Jacek Jabłoński}
%\date{\today}
\date{}
\newcounter{liczniksekcji}
\newcommand{\kategoria}[1]{\section{#1}} %olreślamy nazwę kateforii zadań
\newcommand{\zadStart}[1]{\begin{zad}#1\newline} %oznaczenie początku zadania
\newcommand{\zadStop}{\end{zad}}   %oznaczenie końca zadania
%Makra opcjonarne (nie muszą występować):
\newcommand{\rozwStart}[2]{\noindent \textbf{Rozwiązanie (autor #1 , recenzent #2): }\newline} %oznaczenie początku rozwiązania, opcjonarnie można wprowadzić informację o autorze rozwiązania zadania i recenzencie poprawności wykonania rozwiązania zadania
\newcommand{\rozwStop}{\newline}                                            %oznaczenie końca rozwiązania
\newcommand{\odpStart}{\noindent \textbf{Odpowiedź:}\newline}    %oznaczenie początku odpowiedzi końcowej (wypisanie wyniku)
\newcommand{\odpStop}{\newline}                                             %oznaczenie końca odpowiedzi końcowej (wypisanie wyniku)
\newcommand{\testStart}{\noindent \textbf{Test:}\newline} %ewentualne możliwe opcje odpowiedzi testowej: A. ? B. ? C. ? D. ? itd.
\newcommand{\testStop}{\newline} %koniec wprowadzania odpowiedzi testowych
\newcommand{\kluczStart}{\noindent \textbf{Test poprawna odpowiedź:}\newline} %klucz, poprawna odpowiedź pytania testowego (jedna literka): A lub B lub C lub D itd.
\newcommand{\kluczStop}{\newline} %koniec poprawnej odpowiedzi pytania testowego 
\newcommand{\wstawGrafike}[2]{\begin{figure}[h] \includegraphics[scale=#2] {#1} \end{figure}} %gdyby była potrzeba wstawienia obrazka, parametry: nazwa pliku, skala (jak nie wiesz co wpisać, to wpisz 1)

\begin{document}
\maketitle


\kategoria{Wikieł/P1.9}
\zadStart{Zadanie z Wikieł P 1.9) moja wersja nr [nrWersji]}
%[p1]:[2,4,8,16,32,64,128]
%[p2]:[2,4,8,16,32,64,128]
%[p3]:[2,4,8,16,32,64,128]
%[a]=random.randint(2,8)
%[b]=random.randint(2,6)
%[c]=random.randint(2,6)
%[d]=random.randint(2,6)
%[e]=random.randint(2,6)
%[f]=random.randint(2,6)
%[w1]=int(math.pow(2,[b]))
%[w2]=int(math.pow(2,[d]))
%[w3]=int(math.pow(2,[f]))
%[e1]=([b]*[c])+[d]
%[r1]=[c]*[e]
%[e2]=( (([b]*[c])+[d]) *[e])-([f]*[c])
%[r2]=[c]*[e]*[a]
%[e3]=( (([b]*[c])+[d]) *[e])-([f]*[c])
%[dz]=int(math.gcd([e3],[r2]))
%[e4]=int([e3]/[dz])
%[r4]=int([r2]/[dz])
%[f1]=random.randint(2,128)
%[f2]=random.randint(2,128)
%[f3]=random.randint(2,128)
%[f4]=random.randint(2,128)
Zapisać liczbę $\sqrt[[a]]{\frac{[w1] \sqrt[[c]]{[w2]}}{\sqrt[[e]]{[w3]}}}$ w postaci potęgi liczby 2.
\zadStop
\rozwStart{Jacek Jabłoński}{}
$$\sqrt[[a]]{\frac{[w1] \sqrt[[c]]{[w2]}}{\sqrt[[e]]{[w3]}}} = (\frac{2^{[b]} \cdot 2^{\frac{[d]}{[c]}}}{2^{\frac{[f]}{[e]}}})^{\frac{1}{[a]}} = $$
$$(\frac{2^{\frac{[e1]}{[c]}}}{2^{\frac{[f]}{[e]}}})^{\frac{1}{[a]}} = (2^{\frac{[e2]}{[r1]}})^{\frac{1}{[a]}} = $$
$$2^{\frac{[e3]}{[r2]}} = 2^{\frac{[e4]}{[r4]}}$$
\rozwStop
\odpStart
\odpStop
\testStart
A. $2^{\frac{[e4]}{[r4]}}$
B. $2^{[f1]}$
C. $2^{\frac{[a]}{[b]}}$
D. $2^{[f2]}$
E. $2^{\frac{[b]}{[c]}}$
F. $2^{[f3]}$
G. $2^{\frac{[c]}{[d]}}$
H. $2^{[f4]}$
I. $2^{\frac{[e]}{[f]}}$
\testStop
\kluczStart
A
\kluczStop



\end{document}