\documentclass[12pt, a4paper]{article}
\usepackage[utf8]{inputenc}
\usepackage{polski}

\usepackage{amsthm}  %pakiet do tworzenia twierdzeń itp.
\usepackage{amsmath} %pakiet do niektórych symboli matematycznych
\usepackage{amssymb} %pakiet do symboli mat., np. \nsubseteq
\usepackage{amsfonts}
\usepackage{graphicx} %obsługa plików graficznych z rozszerzeniem png, jpg
\theoremstyle{definition} %styl dla definicji
\newtheorem{zad}{} 
\title{Multizestaw zadań}
\author{Robert Fidytek}
%\date{\today}
\date{}
\newcounter{liczniksekcji}
\newcommand{\kategoria}[1]{\section{#1}} %olreślamy nazwę kateforii zadań
\newcommand{\zadStart}[1]{\begin{zad}#1\newline} %oznaczenie początku zadania
\newcommand{\zadStop}{\end{zad}}   %oznaczenie końca zadania
%Makra opcjonarne (nie muszą występować):
\newcommand{\rozwStart}[2]{\noindent \textbf{Rozwiązanie (autor #1 , recenzent #2): }\newline} %oznaczenie początku rozwiązania, opcjonarnie można wprowadzić informację o autorze rozwiązania zadania i recenzencie poprawności wykonania rozwiązania zadania
\newcommand{\rozwStop}{\newline}                                            %oznaczenie końca rozwiązania
\newcommand{\odpStart}{\noindent \textbf{Odpowiedź:}\newline}    %oznaczenie początku odpowiedzi końcowej (wypisanie wyniku)
\newcommand{\odpStop}{\newline}                                             %oznaczenie końca odpowiedzi końcowej (wypisanie wyniku)
\newcommand{\testStart}{\noindent \textbf{Test:}\newline} %ewentualne możliwe opcje odpowiedzi testowej: A. ? B. ? C. ? D. ? itd.
\newcommand{\testStop}{\newline} %koniec wprowadzania odpowiedzi testowych
\newcommand{\kluczStart}{\noindent \textbf{Test poprawna odpowiedź:}\newline} %klucz, poprawna odpowiedź pytania testowego (jedna literka): A lub B lub C lub D itd.
\newcommand{\kluczStop}{\newline} %koniec poprawnej odpowiedzi pytania testowego 
\newcommand{\wstawGrafike}[2]{\begin{figure}[h] \includegraphics[scale=#2] {#1} \end{figure}} %gdyby była potrzeba wstawienia obrazka, parametry: nazwa pliku, skala (jak nie wiesz co wpisać, to wpisz 1)

\begin{document}
\maketitle


\kategoria{Wikieł/Z5.19h}
\zadStart{Zadanie z Wikieł Z 5.19h) moja wersja nr [nrWersji]}
%[a]:[2,3,4,5,6,7,8,9,10,11,12,13,14,15,16,17,18,19,20,21]
%[ap]=2*[a]
%[b]=[a]-1
%[c]=[a]+1
%[d]=[a]+2
Obliczyć granicę $\lim\limits_{x\to\infty}[a]x\left(e^{\frac{1}{x}}-1\right)$ (sprawdzić, czy spełnione są założenia twierdzenia de l'Hospitala).
\zadStop
\rozwStart{Justyna Chojecka}{}
Dokonujemy przekształceń.
$$\lim\limits_{x\to\infty}[a]x\left(e^{\frac{1}{x}}-1\right)=[a]\cdot\lim\limits_{x\to\infty}\frac{e^{\frac{1}{x}}-1}{\frac{1}{x}}=\left[\frac{0}{0}\right]$$
Zauważmy, że założenia twierdzenia de l'Hospitala są spełnione, zatem korzystamy z tego twierdzenia.
$$[a]\cdot\lim\limits_{x\to\infty}\frac{e^{\frac{1}{x}}-1}{\frac{1}{x}}\overset{l'H}{=}[a]\cdot\lim\limits_{x\to\infty}\frac{-\frac{e^{\frac{1}{x}}}{x^{2}}}{-\frac{1}{x^{2}}}=[a]\cdot\lim\limits_{x\to\infty}e^{\frac{1}{x}}=[a]\cdot 1=[a]$$
\rozwStop
\odpStart
$[a]$
\odpStop
\testStart
A.$[a]$
B.$0$
C.$[c]$
D.$-[b]$
E.$[d]$
F.$[b]$
G.$-[c]$
H.$-[a]$
I.$-[d]$
\testStop
\kluczStart
A
\kluczStop



\end{document}