\documentclass[12pt, a4paper]{article}
\usepackage[utf8]{inputenc}
\usepackage{polski}

\usepackage{amsthm}  %pakiet do tworzenia twierdzeń itp.
\usepackage{amsmath} %pakiet do niektórych symboli matematycznych
\usepackage{amssymb} %pakiet do symboli mat., np. \nsubseteq
\usepackage{amsfonts}
\usepackage{graphicx} %obsługa plików graficznych z rozszerzeniem png, jpg
\theoremstyle{definition} %styl dla definicji
\newtheorem{zad}{} 
\title{Multizestaw zadań}
\author{Robert Fidytek}
%\date{\today}
\date{}
\newcounter{liczniksekcji}
\newcommand{\kategoria}[1]{\section{#1}} %olreślamy nazwę kateforii zadań
\newcommand{\zadStart}[1]{\begin{zad}#1\newline} %oznaczenie początku zadania
\newcommand{\zadStop}{\end{zad}}   %oznaczenie końca zadania
%Makra opcjonarne (nie muszą występować):
\newcommand{\rozwStart}[2]{\noindent \textbf{Rozwiązanie (autor #1 , recenzent #2): }\newline} %oznaczenie początku rozwiązania, opcjonarnie można wprowadzić informację o autorze rozwiązania zadania i recenzencie poprawności wykonania rozwiązania zadania
\newcommand{\rozwStop}{\newline}                                            %oznaczenie końca rozwiązania
\newcommand{\odpStart}{\noindent \textbf{Odpowiedź:}\newline}    %oznaczenie początku odpowiedzi końcowej (wypisanie wyniku)
\newcommand{\odpStop}{\newline}                                             %oznaczenie końca odpowiedzi końcowej (wypisanie wyniku)
\newcommand{\testStart}{\noindent \textbf{Test:}\newline} %ewentualne możliwe opcje odpowiedzi testowej: A. ? B. ? C. ? D. ? itd.
\newcommand{\testStop}{\newline} %koniec wprowadzania odpowiedzi testowych
\newcommand{\kluczStart}{\noindent \textbf{Test poprawna odpowiedź:}\newline} %klucz, poprawna odpowiedź pytania testowego (jedna literka): A lub B lub C lub D itd.
\newcommand{\kluczStop}{\newline} %koniec poprawnej odpowiedzi pytania testowego 
\newcommand{\wstawGrafike}[2]{\begin{figure}[h] \includegraphics[scale=#2] {#1} \end{figure}} %gdyby była potrzeba wstawienia obrazka, parametry: nazwa pliku, skala (jak nie wiesz co wpisać, to wpisz 1)

\begin{document}
\maketitle


\kategoria{Wikieł/Z5.20 b}
\zadStart{Zadanie z Wikieł Z 5.20 b) moja wersja nr [nrWersji]}
%[z]:[3,5,7,9,11,13,15,17,19]
%[y]:[2,3,4,5,6,7,8,9,10,11,12,15,17]
%[a]=random.randint(2,20)
%[b]=random.randint(2,20)
%[c]=2*[a]
%[d]=[a]*[a]
Znaleźć równania asymptot wykresu funkcji $f$ danej wzorem.\\
 $f(x)=\frac{(x+[a])^2}{x^2+[b]}$
\zadStop
\rozwStart{Katarzyna Filipowicz}{}
1. Badamy istnienie asymptot pionowych.\\
Funkcja $f$ jest ciągła w $D_f=R$, stąd wynika, że nie ma asymptot pionowych.\\
2. Badamy istnienie asymptot ukośnych.\\
Ponieważ
$$ 
\lim_{x\rightarrow-\infty} \frac{f(x)}{x}=\lim_{x\rightarrow-\infty}\frac{(x+[a])^2}{x(x^2+[b])}=
$$ $$
=\lim_{x\rightarrow-\infty} \frac{x^2(1+\frac{[c]}{x}+\frac{[d]}{x^2})}{x^3(1+\frac{[b]}{x^3})}=\lim_{x\rightarrow-\infty} \frac{(1+\frac{[c]}{x}+\frac{[d]}{x})}{x(1+\frac{[b]}{x^3})}=0
$$ $$
=\lim_{x\rightarrow-\infty}(f(x)-0\cdot x)=\lim_{x\rightarrow-\infty}\frac{(x+[a])^2}{x^2+[b]}=
$$ $$
=\lim_{x\rightarrow-\infty} \frac{x^2(1+\frac{[c]}{x}+\frac{[d]}{x^2})}{x^2(1+\frac{[b]}{x^2})}=\lim_{x\rightarrow-\infty} \frac{(1+\frac{[c]}{x}+\frac{[d]}{x})}{(1+\frac{[b]}{x^2})}=1
$$
Więc prosta o równaniu $y=0\cdot x+1 \Rightarrow y=1$ jest asymptotą ukośną lewostronną.
Ponadto
$$ 
\lim_{x\rightarrow\infty} \frac{f(x)}{x}=\lim_{x\rightarrow\infty}\frac{(x+[a])^2}{x(x^2+[b])}=
$$ $$
=\lim_{x\rightarrow\infty} \frac{x^2(1+\frac{[c]}{x}+\frac{[d]}{x^2})}{x^3(1+\frac{[b]}{x^3})}=\lim_{x\rightarrow\infty} \frac{(1+\frac{[c]}{x}+\frac{[d]}{x})}{x^3(1+\frac{[b]}{x^3})}=0
$$ $$
=\lim_{x\rightarrow\infty}(f(x)-0\cdot x)=\lim_{x\rightarrow\infty}\frac{(x+[a])^2}{x^2+[b]}=
$$ $$
=\lim_{x\rightarrow\infty} \frac{x^2(1+\frac{[c]}{x}+\frac{[d]}{x^2})}{x^2(1+\frac{[b]}{x^2})}=\lim_{x\rightarrow\infty} \frac{(1+\frac{[c]}{x}+\frac{[d]}{x})}{(1+\frac{[b]}{x^2})}=1
$$
Więc prosta o równaniu $y=0\cdot x+1 \Rightarrow y=1$ jest asymptotą ukośną prawostronną.
Zatem  prosta o równaniu y=1 jest asymptotą ukośną obustronną.
\rozwStop
\odpStart
$y=1$
\odpStop
\testStart
A.$y=1$\\
B.$y=0$\\
C.$y=1,x=1$\\
D.$x=1$\\
E.$y=-1$\\
F.$y=1,x=0$\\
G.$y=x+1$\\
H.$y=x-1,x=1$\\
I.$y=2$
\testStop
\kluczStart
A
\kluczStop



\end{document}