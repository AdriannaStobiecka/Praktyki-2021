\documentclass[12pt, a4paper]{article}
\usepackage[utf8]{inputenc}
\usepackage{polski}

\usepackage{amsthm}  %pakiet do tworzenia twierdzeń itp.
\usepackage{amsmath} %pakiet do niektórych symboli matematycznych
\usepackage{amssymb} %pakiet do symboli mat., np. \nsubseteq
\usepackage{amsfonts}
\usepackage{graphicx} %obsługa plików graficznych z rozszerzeniem png, jpg
\theoremstyle{definition} %styl dla definicji
\newtheorem{zad}{} 
\title{Multizestaw zadań}
\author{Robert Fidytek}
%\date{\today}
\date{}
\newcounter{liczniksekcji}
\newcommand{\kategoria}[1]{\section{#1}} %olreślamy nazwę kateforii zadań
\newcommand{\zadStart}[1]{\begin{zad}#1\newline} %oznaczenie początku zadania
\newcommand{\zadStop}{\end{zad}}   %oznaczenie końca zadania
%Makra opcjonarne (nie muszą występować):
\newcommand{\rozwStart}[2]{\noindent \textbf{Rozwiązanie (autor #1 , recenzent #2): }\newline} %oznaczenie początku rozwiązania, opcjonarnie można wprowadzić informację o autorze rozwiązania zadania i recenzencie poprawności wykonania rozwiązania zadania
\newcommand{\rozwStop}{\newline}                                            %oznaczenie końca rozwiązania
\newcommand{\odpStart}{\noindent \textbf{Odpowiedź:}\newline}    %oznaczenie początku odpowiedzi końcowej (wypisanie wyniku)
\newcommand{\odpStop}{\newline}                                             %oznaczenie końca odpowiedzi końcowej (wypisanie wyniku)
\newcommand{\testStart}{\noindent \textbf{Test:}\newline} %ewentualne możliwe opcje odpowiedzi testowej: A. ? B. ? C. ? D. ? itd.
\newcommand{\testStop}{\newline} %koniec wprowadzania odpowiedzi testowych
\newcommand{\kluczStart}{\noindent \textbf{Test poprawna odpowiedź:}\newline} %klucz, poprawna odpowiedź pytania testowego (jedna literka): A lub B lub C lub D itd.
\newcommand{\kluczStop}{\newline} %koniec poprawnej odpowiedzi pytania testowego 
\newcommand{\wstawGrafike}[2]{\begin{figure}[h] \includegraphics[scale=#2] {#1} \end{figure}} %gdyby była potrzeba wstawienia obrazka, parametry: nazwa pliku, skala (jak nie wiesz co wpisać, to wpisz 1)

\begin{document}
\maketitle


\kategoria{Wikieł/Z1.106j}
\zadStart{Zadanie z Wikieł Z 1.106 j) moja wersja nr [nrWersji]}
%[y]:[2]
%[a]:[3,5,7,9,11,13,15,17,19]
%[c]=6*[a]
Rozwiązać równania.\\
 $4sin^3([a]x)+2=1-2sin^2([a]x)-2sin([a]x)$
\zadStop
\rozwStart{Katarzyna Filipowicz}{}
$$4sin^3([a]x)+2=1-2sin^2([a]x)-2sin([a]x)$$
$$
t=sin([a]x) \qquad D_t:[-1,1]
$$ $$
4t^3+2=1-2t^2-2t
$$ $$
2t^2(2t+1)+(2t+1)=0
$$ $$
(2t^2+1)(2t+1)=0
$$

1. $$
t^2=-\frac{1}{2} \quad \text{Brak rozwiązań}
$$ 

2. $$
t=-\frac{1}{2} \quad \Rightarrow \quad sin([a]x)=-\frac{1}{2}
$$ 

2a$$
[a]x=-\frac{\pi}{6}+2k\pi
$$ $$
x=-\frac{\pi}{[c]}+\frac{2k\pi}{[a]}
$$

2b$$
[a]x=\frac{7\pi}{6}+2k\pi
$$ $$
x=\frac{7\pi}{[c]}+\frac{2k\pi}{[a]}
$$
\rozwStop
\odpStart
$x=-\frac{\pi}{[c]}+\frac{2k\pi}{[a]} \vee x=\frac{7\pi}{[c]}+\frac{2k\pi}{[a]}$
\odpStop
\testStart
A.$x=-\frac{\pi}{[c]}+\frac{2k\pi}{[a]} \vee x=\frac{7\pi}{[c]}+\frac{2k\pi}{[a]}$\\
B.$x=0 \vee x=\frac{7\pi}{[c]}+\frac{2k\pi}{[a]}$\\
C.$x=-\frac{\pi}{[c]}+\frac{2k\pi}{[a]} \vee x=0+2k\pi$\\
D.$x=-\frac{\pi}{6}+2k\pi$\\
E.$x=\frac{\pi}{6}+2k\pi$\\
F.$x=-\frac{\pi}{6}+k\pi$\\
G.$x=-\frac{\pi}{[c]}+\frac{2k\pi}{[c]} \vee x=\frac{7\pi}{[c]}+\frac{2k\pi}{[c]}$\\
H.$x=-\frac{\pi}{[c]}+\frac{2k\pi}{[a]} \vee x=-\frac{7\pi}{[c]}+\frac{2k\pi}{[a]}$\\
I.$x=-\frac{\pi}{[c]}+\frac{2k\pi}{[a]} \vee x=\frac{7\pi}{[c]}+\frac{4k\pi}{[a]}$
\testStop
\kluczStart
A
\kluczStop



\end{document}