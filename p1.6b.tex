\documentclass[12pt, a4paper]{article}
\usepackage[utf8]{inputenc}
\usepackage{polski}

\usepackage{amsthm}  %pakiet do tworzenia twierdzeń itp.
\usepackage{amsmath} %pakiet do niektórych symboli matematycznych
\usepackage{amssymb} %pakiet do symboli mat., np. \nsubseteq
\usepackage{amsfonts}
\usepackage{graphicx} %obsługa plików graficznych z rozszerzeniem png, jpg
\theoremstyle{definition} %styl dla definicji
\newtheorem{zad}{} 
\title{Multizestaw zadań}
\author{Robert Fidytek}
%\date{\today}
\date{}
\newcounter{liczniksekcji}
\newcommand{\kategoria}[1]{\section{#1}} %olreślamy nazwę kateforii zadań
\newcommand{\zadStart}[1]{\begin{zad}#1\newline} %oznaczenie początku zadania
\newcommand{\zadStop}{\end{zad}}   %oznaczenie końca zadania
%Makra opcjonarne (nie muszą występować):
\newcommand{\rozwStart}[2]{\noindent \textbf{Rozwiązanie (autor #1 , recenzent #2): }\newline} %oznaczenie początku rozwiązania, opcjonarnie można wprowadzić informację o autorze rozwiązania zadania i recenzencie poprawności wykonania rozwiązania zadania
\newcommand{\rozwStop}{\newline}                                            %oznaczenie końca rozwiązania
\newcommand{\odpStart}{\noindent \textbf{Odpowiedź:}\newline}    %oznaczenie początku odpowiedzi końcowej (wypisanie wyniku)
\newcommand{\odpStop}{\newline}                                             %oznaczenie końca odpowiedzi końcowej (wypisanie wyniku)
\newcommand{\testStart}{\noindent \textbf{Test:}\newline} %ewentualne możliwe opcje odpowiedzi testowej: A. ? B. ? C. ? D. ? itd.
\newcommand{\testStop}{\newline} %koniec wprowadzania odpowiedzi testowych
\newcommand{\kluczStart}{\noindent \textbf{Test poprawna odpowiedź:}\newline} %klucz, poprawna odpowiedź pytania testowego (jedna literka): A lub B lub C lub D itd.
\newcommand{\kluczStop}{\newline} %koniec poprawnej odpowiedzi pytania testowego 
\newcommand{\wstawGrafike}[2]{\begin{figure}[h] \includegraphics[scale=#2] {#1} \end{figure}} %gdyby była potrzeba wstawienia obrazka, parametry: nazwa pliku, skala (jak nie wiesz co wpisać, to wpisz 1)

\begin{document}
\maketitle


\kategoria{Wikieł/P1.6b}
\zadStart{Zadanie z Wikieł P 1.6 b) moja wersja nr [nrWersji]}
%[p1]:[4,5,6,7,8,9,10,11,12]
%[p2]:[2,3,4,5,6,7,8,9,10,11,12]
%[p3]:[2,3,4,5,6,7,8,9,10,11,12]
%[a]=random.randint(2,10)
%[e]=random.randint(2,10)
%[c]=random.randint(1,10)
%[d]=random.randint(2,10)
%[b]=random.randint(2,10)
%[f]=random.randint(1,10)
%[p1p2m]=[p1]-[p2]
%[p1p3m]=[p1]-[p3]
%[p1]>[p2] and [p1]>[p3] and [p2]!=[p3] and math.gcd([a],[d])==1 and [p1p2m]>1 and [p1p3m]>1 and [a]<[b]
%[g]=[b]-[a]
Rozwiązać nierówność $|x+[a]|-|x|<[b]$.
\zadStop
\rozwStart{Pascal Nawrocki}{}
Z uwagi na znaki wyrażeń w modułach rozpatrujemy przypadki:
\begin{enumerate}
\item$x\in(-\infty,-[a])$
\item$x\in[-[a],0)$
\item$x\in[0,+\infty)$
\end{enumerate}
Rozwiążmy je teraz:
\begin{enumerate}
\item dla $x\in(-\infty,-[a])$
 mamy: 
$$ -x-[a]-(-x)<[b]$$
$$ -x-[a]+x<[b]$$
$$ -[a]<[b]$$
Co oznacza, że dla wszystkich $x\in(-\infty,-[a])$ dana nierówność jest spełniona.
\item dla $x\in[-[a],0)$
mamy:
$$x+[a]-(-x)<[b]$$
$$x+[a]+x<[b]$$
$$2x<[b]-[a]$$
$$2x<[c]$$
$$x<\frac{[g]}{2}$$
Stąd, uwzględniając założenie, dana nierówność jest spełniona dla $x\in[-[a],0)$
\item dla $x\in[0,+\infty)$
mamy:
$$x+[a]-x<[b]$$
$$[a]<[b]$$
Co oznacza, że dla wszystkich $x\in[0,+\infty)$ dana nierówność jest spełniona.
\end{enumerate}
Ostatecznie otrzymamy, że rozważana nierówność jest spełniona dla: 
$$x\in(-\infty,-[a])\cup x\in[-[a],0)\cup x\in[0,+\infty)  \text{, czyli dla } x\in \mathbb{R}$$
\rozwStop
\odpStart
$ x\in \mathbb{R}$
\odpStop
\testStart
A.$x\in \mathbb{R}$
B.$x\in(-\infty,-[a])$
C.$x\in[-[a],0)$
D.$x\in \emptyset$
\testStop
\kluczStart
A
\kluczStop
\end{document}