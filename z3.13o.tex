\documentclass[12pt, a4paper]{article}
\usepackage[utf8]{inputenc}
\usepackage{polski}

\usepackage{amsthm}  %pakiet do tworzenia twierdzeń itp.
\usepackage{amsmath} %pakiet do niektórych symboli matematycznych
\usepackage{amssymb} %pakiet do symboli mat., np. \nsubseteq
\usepackage{amsfonts}
\usepackage{graphicx} %obsługa plików graficznych z rozszerzeniem png, jpg
\theoremstyle{definition} %styl dla definicji
\newtheorem{zad}{} 
\title{Multizestaw zadań}
\author{Robert Fidytek}
%\date{\today}
\date{}
\newcounter{liczniksekcji}
\newcommand{\kategoria}[1]{\section{#1}} %olreślamy nazwę kateforii zadań
\newcommand{\zadStart}[1]{\begin{zad}#1\newline} %oznaczenie początku zadania
\newcommand{\zadStop}{\end{zad}}   %oznaczenie końca zadania
%Makra opcjonarne (nie muszą występować):
\newcommand{\rozwStart}[2]{\noindent \textbf{Rozwiązanie (autor #1 , recenzent #2): }\newline} %oznaczenie początku rozwiązania, opcjonarnie można wprowadzić informację o autorze rozwiązania zadania i recenzencie poprawności wykonania rozwiązania zadania
\newcommand{\rozwStop}{\newline}                                            %oznaczenie końca rozwiązania
\newcommand{\odpStart}{\noindent \textbf{Odpowiedź:}\newline}    %oznaczenie początku odpowiedzi końcowej (wypisanie wyniku)
\newcommand{\odpStop}{\newline}                                             %oznaczenie końca odpowiedzi końcowej (wypisanie wyniku)
\newcommand{\testStart}{\noindent \textbf{Test:}\newline} %ewentualne możliwe opcje odpowiedzi testowej: A. ? B. ? C. ? D. ? itd.
\newcommand{\testStop}{\newline} %koniec wprowadzania odpowiedzi testowych
\newcommand{\kluczStart}{\noindent \textbf{Test poprawna odpowiedź:}\newline} %klucz, poprawna odpowiedź pytania testowego (jedna literka): A lub B lub C lub D itd.
\newcommand{\kluczStop}{\newline} %koniec poprawnej odpowiedzi pytania testowego 
\newcommand{\wstawGrafike}[2]{\begin{figure}[h] \includegraphics[scale=#2] {#1} \end{figure}} %gdyby była potrzeba wstawienia obrazka, parametry: nazwa pliku, skala (jak nie wiesz co wpisać, to wpisz 1)

\begin{document}
\maketitle


\kategoria{Wikieł/Z3.13o}
\zadStart{Zadanie z Wikieł Z 3.13 o) moja wersja nr [nrWersji]}
%[a]:[2,3,4,5,6,7,8,9,10,11,12,13,14,15,16,17,18,19,20]
%[2a]=2*[a]
%
Obliczyć granicę ciągu 
$$a_n=\bigg(\frac{1}{[a]n^2}+\frac{2}{[a]n^2}+\dots+\frac{n-1}{[a]n^2}\bigg).$$
\zadStop
\rozwStart{Adrianna Stobiecka}{}
Zaczniemy od przekształcenia wyrazu ogólnego $a_n$.
$$a_n=\frac{1}{[a]n^2}+\frac{2}{[a]n^2}+\dots+\frac{n-1}{[a]n^2}=\frac{1+2+\dots+(n-1)}{[a]n^2}=(*)$$
W liczniku skorzystamy ze wzoru na sumę ciągu arytmetycznego. Wzór ten wygląda następująco:
$$S_n=\frac{a_1+a_n}{2}\cdot n$$ 
Mamy zatem:
$$(*)=\frac{\frac{1+(n-1)}{2}\cdot(n-1)}{[a]n^2}=\frac{n(n-1)}{2\cdot[a]n^2}=\frac{n-1}{[2a]n}$$
Przejdziemy teraz do obliczenia granicy.
$$\lim_{n\to\infty}a_n=\lim_{n\to\infty}\frac{n-1}{[2a]n}=\lim_{n\to\infty}\frac{n(1-\frac{1}{n})}{n\cdot[2a]}=\lim_{n\to\infty}\frac{1-\frac{1}{n}}{[2a]}=\frac{1}{[2a]}$$
\rozwStop
\odpStart
$\frac{1}{[2a]}$
\odpStop
\testStart
A.$[a]$
B.$-[2a]$
C.$[2a]$
D.$\infty$
E.$-\frac{1}{[2a]}$
F.$\frac{1}{[2a]}$
G.$-\infty$
H.$0$
I.$\frac{1}{[a]}$
\testStop
\kluczStart
F
\kluczStop



\end{document}
