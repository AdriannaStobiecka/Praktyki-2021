\documentclass[12pt, a4paper]{article}
\usepackage[utf8]{inputenc}
\usepackage{polski}

\usepackage{amsthm}  %pakiet do tworzenia twierdzeń itp.
\usepackage{amsmath} %pakiet do niektórych symboli matematycznych
\usepackage{amssymb} %pakiet do symboli mat., np. \nsubseteq
\usepackage{amsfonts}
\usepackage{graphicx} %obsługa plików graficznych z rozszerzeniem png, jpg
\theoremstyle{definition} %styl dla definicji
\newtheorem{zad}{} 
\title{Multizestaw zadań}
\author{Robert Fidytek}
%\date{\today}
\date{}
\newcounter{liczniksekcji}
\newcommand{\kategoria}[1]{\section{#1}} %olreślamy nazwę kateforii zadań
\newcommand{\zadStart}[1]{\begin{zad}#1\newline} %oznaczenie początku zadania
\newcommand{\zadStop}{\end{zad}}   %oznaczenie końca zadania
%Makra opcjonarne (nie muszą występować):
\newcommand{\rozwStart}[2]{\noindent \textbf{Rozwiązanie (autor #1 , recenzent #2): }\newline} %oznaczenie początku rozwiązania, opcjonarnie można wprowadzić informację o autorze rozwiązania zadania i recenzencie poprawności wykonania rozwiązania zadania
\newcommand{\rozwStop}{\newline}                                            %oznaczenie końca rozwiązania
\newcommand{\odpStart}{\noindent \textbf{Odpowiedź:}\newline}    %oznaczenie początku odpowiedzi końcowej (wypisanie wyniku)
\newcommand{\odpStop}{\newline}                                             %oznaczenie końca odpowiedzi końcowej (wypisanie wyniku)
\newcommand{\testStart}{\noindent \textbf{Test:}\newline} %ewentualne możliwe opcje odpowiedzi testowej: A. ? B. ? C. ? D. ? itd.
\newcommand{\testStop}{\newline} %koniec wprowadzania odpowiedzi testowych
\newcommand{\kluczStart}{\noindent \textbf{Test poprawna odpowiedź:}\newline} %klucz, poprawna odpowiedź pytania testowego (jedna literka): A lub B lub C lub D itd.
\newcommand{\kluczStop}{\newline} %koniec poprawnej odpowiedzi pytania testowego 
\newcommand{\wstawGrafike}[2]{\begin{figure}[h] \includegraphics[scale=#2] {#1} \end{figure}} %gdyby była potrzeba wstawienia obrazka, parametry: nazwa pliku, skala (jak nie wiesz co wpisać, to wpisz 1)

\begin{document}
\maketitle


\kategoria{Wikieł/Z1.15a}
\zadStart{Zadanie z Wikieł Z 1.15 a) moja wersja nr [nrWersji]}
%[a]:[2,3,4,5,6,7,8,9,10,11,12]
%[b]:[1,2,3,4,5,6,7,8,9,10,11,12]
%[c]=random.randint(1,10)
%[p1]=[c]-[b]
%[p2]=-[b]-[c]
%[b2]=-[b]
%math.gcd([a],[b])==1 and math.gcd([a],[c])==1
Rozwiąż równanie $|[a]x + [b]| \leq [c]$.
\zadStop
\rozwStart{Małgorzata Ugowska}{}
$$|[a]x + [b]| \leq [c] \Leftrightarrow |x + \frac{[b]}{[a]}| \leq \frac{[c]}{[a]}$$ 
Szukamy $x$ takich, że:
$$x + \frac{[b]}{[a]} \leq \frac{[c]}{[a]} \land -x - \frac{[b]}{[a]}\leq \frac{[c]}{[a]}$$
$$x  \leq \frac{[p1]}{[a]}  \land x \geq \frac{[p2]}{[a]}$$
$$x \in \big[\frac{[p2]}{[a]},\frac{[p1]}{[a]}\big]$$
Lub krócej:\\
Szukamy liczb odległych od $\frac{[b2]}{[a]}$ o mniej lub równo $\frac{[c]}{[a]}$. Szukany przez nas przedział to $\big[\frac{[p2]}{[a]},\frac{[p1]}{[a]}\big]$.
\rozwStop
\odpStart
$x \in \big[\frac{[p2]}{[a]},\frac{[p1]}{[a]}\big]$
\odpStop
\testStart
A. $\big(\frac{[p2]}{[a]},\frac{[p1]}{[a]}\big)$\\
B. $\big[\frac{[p2]}{[a]},\frac{[p1]}{[a]}\big]$\\
C. $\big(-\infty,\frac{[p2]}{[a]}\big] \cup \big[\frac{[p1]}{[a]},\infty\big)$\\
D. $\big(-\infty,\frac{[p2]}{[a]}\big) \cup \big(\frac{[p1]}{[a]},\infty\big)$
\testStop
\kluczStart
B
\kluczStop



\end{document}