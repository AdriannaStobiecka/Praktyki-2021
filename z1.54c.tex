\documentclass[12pt, a4paper]{article}
\usepackage[utf8]{inputenc}
\usepackage{polski}

\usepackage{amsthm}  %pakiet do tworzenia twierdzeń itp.
\usepackage{amsmath} %pakiet do niektórych symboli matematycznych
\usepackage{amssymb} %pakiet do symboli mat., np. \nsubseteq
\usepackage{amsfonts}
\usepackage{graphicx} %obsługa plików graficznych z rozszerzeniem png, jpg
\theoremstyle{definition} %styl dla definicji
\newtheorem{zad}{} 
\title{Multizestaw zadań}
\author{Robert Fidytek}
%\date{\today}
\date{}
\newcounter{liczniksekcji}
\newcommand{\kategoria}[1]{\section{#1}} %olreślamy nazwę kateforii zadań
\newcommand{\zadStart}[1]{\begin{zad}#1\newline} %oznaczenie początku zadania
\newcommand{\zadStop}{\end{zad}}   %oznaczenie końca zadania
%Makra opcjonarne (nie muszą występować):
\newcommand{\rozwStart}[2]{\noindent \textbf{Rozwiązanie (autor #1 , recenzent #2): }\newline} %oznaczenie początku rozwiązania, opcjonarnie można wprowadzić informację o autorze rozwiązania zadania i recenzencie poprawności wykonania rozwiązania zadania
\newcommand{\rozwStop}{\newline}                                            %oznaczenie końca rozwiązania
\newcommand{\odpStart}{\noindent \textbf{Odpowiedź:}\newline}    %oznaczenie początku odpowiedzi końcowej (wypisanie wyniku)
\newcommand{\odpStop}{\newline}                                             %oznaczenie końca odpowiedzi końcowej (wypisanie wyniku)
\newcommand{\testStart}{\noindent \textbf{Test:}\newline} %ewentualne możliwe opcje odpowiedzi testowej: A. ? B. ? C. ? D. ? itd.
\newcommand{\testStop}{\newline} %koniec wprowadzania odpowiedzi testowych
\newcommand{\kluczStart}{\noindent \textbf{Test poprawna odpowiedź:}\newline} %klucz, poprawna odpowiedź pytania testowego (jedna literka): A lub B lub C lub D itd.
\newcommand{\kluczStop}{\newline} %koniec poprawnej odpowiedzi pytania testowego 
\newcommand{\wstawGrafike}[2]{\begin{figure}[h] \includegraphics[scale=#2] {#1} \end{figure}} %gdyby była potrzeba wstawienia obrazka, parametry: nazwa pliku, skala (jak nie wiesz co wpisać, to wpisz 1)

\begin{document}
\maketitle
\kategoria{Wikieł/Z1.54c}
\zadStart{Zadanie z Wikieł Z 1.54 c) moja wersja nr [nrWersji]}
%[a]:[2,3,4]
%[b]:[2,3,4]
%[c]:[2,3,4]
%[d]:[2,3,4]
%[a]=random.randint(2,9)
%[b]=random.randint(2,9)
%[c]=random.randint(2,10)
%[d]=random.randint(2,10)
%[a2]=[a]*(-8)
%[b2]=[b]*4
%[c2]=[c]*(-2)
%[wynik]=[a2]+[b2]+[c2]+[d]
Nie wykonując dzielenia, znaleźć resztę z dzielenia wielomianów $W(x)=[a]x^3+[b]x^2-[c]x+[d]$ przez $P(x)=x+2$.
\zadStop
\rozwStart{Pascal Nawrocki}{}
Pamiętamy o tym, że nasza reszta będzie stopnia co najmniej o 1 niższego od stopnia naszego $P(x)$. Skorzystamy ze wzoru:
$$W(x)=P(x)Q(x)+R(x)$$
$$[a]x^3+[b]x^2-[c]x+[d]=(x+2)Q(x)+b$$
Gdzie R(x)=b i Q(x) jest pewnym wielomianem. Podstawiamy do tej równości miejsca zerowe $P(x)$
$$[a]\cdot(-2)^3+[b]\cdot(-2)^2-[c]\cdot(-2)+[d]=(-2+2)Q(x)+b$$
$$[a2]+[b2][c2]+[d]=b$$
Stąd otrzymujemy, że $b=[wynik]$ i jest to nasza reszta z dzielenia tych wielomianów.
\odpStart
$[wynik]$
\odpStop
\testStart
A.$[wynik]$
B.$[c]$
C.$[a]$
D.$[b]$
\testStop
\kluczStart
A
\kluczStop
\end{document}