\documentclass[12pt, a4paper]{article}
\usepackage[utf8]{inputenc}
\usepackage{polski}

\usepackage{amsthm}  %pakiet do tworzenia twierdzeń itp.
\usepackage{amsmath} %pakiet do niektórych symboli matematycznych
\usepackage{amssymb} %pakiet do symboli mat., np. \nsubseteq
\usepackage{amsfonts}
\usepackage{graphicx} %obsługa plików graficznych z rozszerzeniem png, jpg
\theoremstyle{definition} %styl dla definicji
\newtheorem{zad}{} 
\title{Multizestaw zadań}
\author{Robert Fidytek}
%\date{\today}
\date{}\documentclass[12pt, a4paper]{article}
\usepackage[utf8]{inputenc}
\usepackage{polski}

\usepackage{amsthm}  %pakiet do tworzenia twierdzeń itp.
\usepackage{amsmath} %pakiet do niektórych symboli matematycznych
\usepackage{amssymb} %pakiet do symboli mat., np. \nsubseteq
\usepackage{amsfonts}
\usepackage{graphicx} %obsługa plików graficznych z rozszerzeniem png, jpg
\theoremstyle{definition} %styl dla definicji
\newtheorem{zad}{} 
\title{Multizestaw zadań}
\author{Robert Fidytek}
%\date{\today}
\date{}
\newcounter{liczniksekcji}
\newcommand{\kategoria}[1]{\section{#1}} %olreślamy nazwę kateforii zadań
\newcommand{\zadStart}[1]{\begin{zad}#1\newline} %oznaczenie początku zadania
\newcommand{\zadStop}{\end{zad}}   %oznaczenie końca zadania
%Makra opcjonarne (nie muszą występować):
\newcommand{\rozwStart}[2]{\noindent \textbf{Rozwiązanie (autor #1 , recenzent #2): }\newline} %oznaczenie początku rozwiązania, opcjonarnie można wprowadzić informację o autorze rozwiązania zadania i recenzencie poprawności wykonania rozwiązania zadania
\newcommand{\rozwStop}{\newline}                                            %oznaczenie końca rozwiązania
\newcommand{\odpStart}{\noindent \textbf{Odpowiedź:}\newline}    %oznaczenie początku odpowiedzi końcowej (wypisanie wyniku)
\newcommand{\odpStop}{\newline}                                             %oznaczenie końca odpowiedzi końcowej (wypisanie wyniku)
\newcommand{\testStart}{\noindent \textbf{Test:}\newline} %ewentualne możliwe opcje odpowiedzi testowej: A. ? B. ? C. ? D. ? itd.
\newcommand{\testStop}{\newline} %koniec wprowadzania odpowiedzi testowych
\newcommand{\kluczStart}{\noindent \textbf{Test poprawna odpowiedź:}\newline} %klucz, poprawna odpowiedź pytania testowego (jedna literka): A lub B lub C lub D itd.
\newcommand{\kluczStop}{\newline} %koniec poprawnej odpowiedzi pytania testowego 
\newcommand{\wstawGrafike}[2]{\begin{figure}[h] \includegraphics[scale=#2] {#1} \end{figure}} %gdyby była potrzeba wstawienia obrazka, parametry: nazwa pliku, skala (jak nie wiesz co wpisać, to wpisz 1)

\begin{document}
\maketitle


\kategoria{Wikieł/Z1.129j}
\zadStart{Zadanie z Wikieł Z 1.129 j) moja wersja nr [nrWersji]}
%[p1]:[2,3,4,5,6,7,8,9,10]
%[p2]:[2,3,4,5,6,7,8,9,10]
%[p3]=random.randint(2,10)
%math.gcd([p1],[p2])==1

Wyznaczyć dziedzinę naturalną funkcji.
$$f(x)=[p3]\arccos \frac{[p1]}{[p2]\ln x}$$

\zadStop

\rozwStart{Maja Szabłowska}{}
$$-1\leq \frac{[p1]}{[p2]\ln x} \leq 1 $$
$$-1\leq \frac{[p1]}{[p2]\ln x} \quad \land \quad \frac{[p1]}{[p2]\ln x} \leq 1$$
$$-\ln^{2}x\leq \frac{[p1]}{[p2]}\ln x \quad \land \quad \frac{[p1]}{[p2]}\ln x \leq \ln^{2}x$$
Wykonujemy podstawienie $\ln x=t, \quad t\in\mathbb{R}.$
$$-t^{2}-\frac{[p1]}{[p2]}t\leq 0 \quad \land \quad \frac{[p1]}{[p2]}t-t^{2}\leq 0$$
$$-t\left(t+\frac{[p1]}{[p2]}\right)\leq 0 \quad \land \quad t\left(\frac{[p1]}{[p2]}-t \right)\leq 0$$
$$t\in\left(-\infty, -\frac{[p1]}{[p2]}\right]\cup[0,\infty) \quad \land \quad t\in(-\infty,0]\cup\left[\frac{[p1]}{[p2]},\infty\right)$$
$$t\in\left(-\infty,-\frac{[p1]}{[p2]}\right]\cup\left[\frac{[p1]}{[p2]},\infty\right)$$
$$\ln x_{1}=-\frac{[p1]}{[p2]} \Rightarrow x_{1}=e^{-\frac{[p1]}{[p2]}}, \quad \ln x_{2}=\frac{[p1]}{[p2]} \Rightarrow x_{2}=e^{\frac{[p1]}{[p2]}}$$
$$x\in\left(-\infty,e^{-\frac{[p1]}{[p2]}}\right]\cup\left[e^{\frac{[p1]}{[p2]}},\infty\right)$$
Dodatkowo:
$$\ln x \neq 0 \quad \land \quad x>0$$
Zatem ostatecznie:
$$x\in\left(0,e^{-\frac{[p1]}{[p2]}}\right]\cup\left[e^{\frac{[p1]}{[p2]}},\infty\right)$$

\rozwStop
\odpStart
$x\in\left(0,e^{-\frac{[p1]}{[p2]}}\right]\cup\left[e^{\frac{[p1]}{[p2]}},\infty\right)$
\odpStop
\testStart
A.$x\in\left(0,e^{-\frac{[p1]}{[p2]}}\right]\cup\left[e^{\frac{[p1]}{[p2]}},\infty\right)$
B.$x\in[e^{[p2]},\infty)$
C.$x\in(-\infty, 0)$
D.$x\in(-\infty, -[p2]] \cup [\ln[p1],\infty)$
E.$x\in[[p1],\infty)$
F.$x\in([p3],\infty)$
G.$x\in\emptyset$

\testStop
\kluczStart
A
\kluczStop



\end{document}
