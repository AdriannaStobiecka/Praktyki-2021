\documentclass[12pt, a4paper]{article}
\usepackage[utf8]{inputenc}
\usepackage{polski}

\usepackage{amsthm}  %pakiet do tworzenia twierdzeń itp.
\usepackage{amsmath} %pakiet do niektórych symboli matematycznych
\usepackage{amssymb} %pakiet do symboli mat., np. \nsubseteq
\usepackage{amsfonts}
\usepackage{graphicx} %obsługa plików graficznych z rozszerzeniem png, jpg
\theoremstyle{definition} %styl dla definicji
\newtheorem{zad}{} 
\title{Multizestaw zadań}
\author{Robert Fidytek}
%\date{\today}
\date{}\documentclass[12pt, a4paper]{article}
\usepackage[utf8]{inputenc}
\usepackage{polski}

\usepackage{amsthm}  %pakiet do tworzenia twierdzeń itp.
\usepackage{amsmath} %pakiet do niektórych symboli matematycznych
\usepackage{amssymb} %pakiet do symboli mat., np. \nsubseteq
\usepackage{amsfonts}
\usepackage{graphicx} %obsługa plików graficznych z rozszerzeniem png, jpg
\theoremstyle{definition} %styl dla definicji
\newtheorem{zad}{} 
\title{Multizestaw zadań}
\author{Robert Fidytek}
%\date{\today}
\date{}
\newcounter{liczniksekcji}
\newcommand{\kategoria}[1]{\section{#1}} %olreślamy nazwę kateforii zadań
\newcommand{\zadStart}[1]{\begin{zad}#1\newline} %oznaczenie początku zadania
\newcommand{\zadStop}{\end{zad}}   %oznaczenie końca zadania
%Makra opcjonarne (nie muszą występować):
\newcommand{\rozwStart}[2]{\noindent \textbf{Rozwiązanie (autor #1 , recenzent #2): }\newline} %oznaczenie początku rozwiązania, opcjonarnie można wprowadzić informację o autorze rozwiązania zadania i recenzencie poprawności wykonania rozwiązania zadania
\newcommand{\rozwStop}{\newline}                                            %oznaczenie końca rozwiązania
\newcommand{\odpStart}{\noindent \textbf{Odpowiedź:}\newline}    %oznaczenie początku odpowiedzi końcowej (wypisanie wyniku)
\newcommand{\odpStop}{\newline}                                             %oznaczenie końca odpowiedzi końcowej (wypisanie wyniku)
\newcommand{\testStart}{\noindent \textbf{Test:}\newline} %ewentualne możliwe opcje odpowiedzi testowej: A. ? B. ? C. ? D. ? itd.
\newcommand{\testStop}{\newline} %koniec wprowadzania odpowiedzi testowych
\newcommand{\kluczStart}{\noindent \textbf{Test poprawna odpowiedź:}\newline} %klucz, poprawna odpowiedź pytania testowego (jedna literka): A lub B lub C lub D itd.
\newcommand{\kluczStop}{\newline} %koniec poprawnej odpowiedzi pytania testowego 
\newcommand{\wstawGrafike}[2]{\begin{figure}[h] \includegraphics[scale=#2] {#1} \end{figure}} %gdyby była potrzeba wstawienia obrazka, parametry: nazwa pliku, skala (jak nie wiesz co wpisać, to wpisz 1)

\begin{document}
\maketitle


\kategoria{Wikieł/Z1.52c}
\zadStart{Zadanie z Wikieł Z 1.52c moja wersja nr [nrWersji]}
%[p1]:[2,3,4,5,6,7,8,9]
%[p2]:[2,3,4,5,6,7,8,9]
%[p3]=random.randint(2,10)
%[p4]=random.randint(2,10)
%[p5]=random.randint(1,10)
%[p6]=random.randint(1,10)
%[a]=[p6]*[p1]
%[b]=[p6]*[a]-[p2]
%[c]=[p6]*[b]-[p3]
%[d]=[p6]*[c]-[p4]
%[e]=[p6]*[d]-[p5]

Obliczyć iloraz wielomianu $[p1]x^{5}-[p2]x^{3}-[p3]x^{2}-[p4]x-[p5] : x - [p6].$
\zadStop

\rozwStart{Maja Szabłowska}{}
Dzielenie wykonujemy schematem Hornera.
\begin{table}[h!]
\begin{tabular}{|c|c|c|c|c|c|c|}
\hline
 &
  [p1] &
  0 &
  -[p2] &
  -[p3] &
  -[p4] &
  -[p5] \\ \hline
 &
   &
  [p6]\cdot[p1]-0 &
  [p6]\cdot[a]-[p2] &
  [p6]\cdot[b]-[p3] &
  [p6]\cdot[c]-[p4] &
  [p6]\cdot[d]-[p5] \\ \hline
[p6] &
  [p1] &
  [a] &
  [b] &
  [c] &
  [d] &
  [e] \\ \hline
\end{tabular}
\end{table}
$$[p1]x^{5}-[p2]x^{3}-[p3]x^{2}-[p4]x[p5] : x - [p6]= [p1]x^{4}+[a]x^{3}+[b]x^{2}+[c]x+[d] \quad \textrm{reszta:}\quad [e]$$
\rozwStop


\odpStart
$[p1]x^{4}+[a]x^{3}+[b]x^{2}+[c]x+[d] \quad \textrm{reszta:}\quad [e]$
\odpStop
\testStart
A.$[p1]x^{4}+[a]x^{3}+[b]x^{2}+[c]x+[d] \quad \textrm{reszta:}\quad [e]$
B.$[p1]x^{2}+[a]x+[b]$
C.$[p1]x^{2}+[p2]x+[p3]$
D.$[p3]x$
E.$[p4]x^{2}-[p3]$
F.nie da się wykonać takiego dzielenia


\testStop
\kluczStart
A
\kluczStop



\end{document}
