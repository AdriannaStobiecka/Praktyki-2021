\documentclass[12pt, a4paper]{article}
\usepackage[utf8]{inputenc}
\usepackage{polski}

\usepackage{amsthm}  %pakiet do tworzenia twierdzeń itp.
\usepackage{amsmath} %pakiet do niektórych symboli matematycznych
\usepackage{amssymb} %pakiet do symboli mat., np. \nsubseteq
\usepackage{amsfonts}
\usepackage{graphicx} %obsługa plików graficznych z rozszerzeniem png, jpg
\theoremstyle{definition} %styl dla definicji
\newtheorem{zad}{} 
\title{Multizestaw zadań}
\author{Robert Fidytek}
%\date{\today}
\date{}
\newcounter{liczniksekcji}
\newcommand{\kategoria}[1]{\section{#1}} %olreślamy nazwę kateforii zadań
\newcommand{\zadStart}[1]{\begin{zad}#1\newline} %oznaczenie początku zadania
\newcommand{\zadStop}{\end{zad}}   %oznaczenie końca zadania
%Makra opcjonarne (nie muszą występować):
\newcommand{\rozwStart}[2]{\noindent \textbf{Rozwiązanie (autor #1 , recenzent #2): }\newline} %oznaczenie początku rozwiązania, opcjonarnie można wprowadzić informację o autorze rozwiązania zadania i recenzencie poprawności wykonania rozwiązania zadania
\newcommand{\rozwStop}{\newline}                                            %oznaczenie końca rozwiązania
\newcommand{\odpStart}{\noindent \textbf{Odpowiedź:}\newline}    %oznaczenie początku odpowiedzi końcowej (wypisanie wyniku)
\newcommand{\odpStop}{\newline}                                             %oznaczenie końca odpowiedzi końcowej (wypisanie wyniku)
\newcommand{\testStart}{\noindent \textbf{Test:}\newline} %ewentualne możliwe opcje odpowiedzi testowej: A. ? B. ? C. ? D. ? itd.
\newcommand{\testStop}{\newline} %koniec wprowadzania odpowiedzi testowych
\newcommand{\kluczStart}{\noindent \textbf{Test poprawna odpowiedź:}\newline} %klucz, poprawna odpowiedź pytania testowego (jedna literka): A lub B lub C lub D itd.
\newcommand{\kluczStop}{\newline} %koniec poprawnej odpowiedzi pytania testowego 
\newcommand{\wstawGrafike}[2]{\begin{figure}[h] \includegraphics[scale=#2] {#1} \end{figure}} %gdyby była potrzeba wstawienia obrazka, parametry: nazwa pliku, skala (jak nie wiesz co wpisać, to wpisz 1)

\begin{document}
\maketitle


\kategoria{Wikieł/Z2.9}
\zadStart{Zadanie z Wikieł Z 2.9) moja wersja nr [nrWersji]}
%[a]:[2,3,4,5,6,7,8,9,10]
%[b]:[2,3,4,5,6,7,8,9,10]
%[c]:[1,2,3,4,5,6,7,8,9]
%[aa]=[a]*[a]
%[bb]=[b]*[b]
%[abp]=2*[a]*[b]
%[cc]=[c]*[c]
%[aacc]=[aa]*[cc]
%[bbcc]=[bb]*[cc]
%[abpcc]=[abp]*[cc]
%[d]=int([abpcc]/2)
%[w]=[aacc]-[d]+[bbcc]
%[a]!=[b] and math.gcd([w],4)==1 and math.gcd([w],9)==1 and math.gcd([w],16)==1 and math.gcd([w],25)==1 and math.gcd([w],36)==1 and math.gcd([w],49)==1
Obliczyć długość wektora $\overrightarrow{a}=[a]\overrightarrow{u}-[b]\overrightarrow{v}$, wiedząc, że $|\overrightarrow{u}|=|\overrightarrow{v}|=[c]$, $|\sphericalangle (\overrightarrow{u},\overrightarrow{v})|=60^{\circ}$.
\zadStop
\rozwStart{Justyna Chojecka}{}
Obliczamy kwadrat długości wektora $\overrightarrow{a}$
$$|\overrightarrow{a}|^{2}=(\overrightarrow{a})^{2}=\left([a]\overrightarrow{u}-[b]\overrightarrow{v}\right)^{2}=[aa]|\overrightarrow{u}|^{2}-[abp](\overrightarrow{u}\circ\overrightarrow{v})+[bb]|\overrightarrow{v}|^{2}$$$$=[aa]\cdot[cc]-[abp]\cdot [c] \cdot [c] \cdot cos(60^{\circ})+[bb]\cdot [cc]$$$$=[aacc]-[abpcc]\cdot \frac{1}{2}+[bbcc]=[aacc]-[d]+[bbcc]=[w].$$
Stąd 
$$|\overrightarrow{a}|=\sqrt{[w]}.$$
\rozwStop
\odpStart
$\sqrt{[w]}$
\odpStop
\testStart
A.$\sqrt{[w]}$
B.$\frac{[a]}{[b]}$
C.$\frac{[b]}{[a]}$
D.$[a]$
E.$[aa]$
F.$[b]$
G.$[bb]$
H.$\frac{[aacc]}{[bb]}$
I.$\frac{[a]}{[bb]}$
\testStop
\kluczStart
A
\kluczStop



\end{document}