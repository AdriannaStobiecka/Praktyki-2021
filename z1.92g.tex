\documentclass[12pt, a4paper]{article}
\usepackage[utf8]{inputenc}
\usepackage{polski}

\usepackage{amsthm}  %pakiet do tworzenia twierdzeń itp.
\usepackage{amsmath} %pakiet do niektórych symboli matematycznych
\usepackage{amssymb} %pakiet do symboli mat., np. \nsubseteq
\usepackage{amsfonts}
\usepackage{graphicx} %obsługa plików graficznych z rozszerzeniem png, jpg
\theoremstyle{definition} %styl dla definicji
\newtheorem{zad}{} 
\title{Multizestaw zadań}
\author{Robert Fidytek}
%\date{\today}
\date{}
\newcounter{liczniksekcji}
\newcommand{\kategoria}[1]{\section{#1}} %olreślamy nazwę kateforii zadań
\newcommand{\zadStart}[1]{\begin{zad}#1\newline} %oznaczenie początku zadania
\newcommand{\zadStop}{\end{zad}}   %oznaczenie końca zadania
%Makra opcjonarne (nie muszą występować):
\newcommand{\rozwStart}[2]{\noindent \textbf{Rozwiązanie (autor #1 , recenzent #2): }\newline} %oznaczenie początku rozwiązania, opcjonarnie można wprowadzić informację o autorze rozwiązania zadania i recenzencie poprawności wykonania rozwiązania zadania
\newcommand{\rozwStop}{\newline}                                            %oznaczenie końca rozwiązania
\newcommand{\odpStart}{\noindent \textbf{Odpowiedź:}\newline}    %oznaczenie początku odpowiedzi końcowej (wypisanie wyniku)
\newcommand{\odpStop}{\newline}                                             %oznaczenie końca odpowiedzi końcowej (wypisanie wyniku)
\newcommand{\testStart}{\noindent \textbf{Test:}\newline} %ewentualne możliwe opcje odpowiedzi testowej: A. ? B. ? C. ? D. ? itd.
\newcommand{\testStop}{\newline} %koniec wprowadzania odpowiedzi testowych
\newcommand{\kluczStart}{\noindent \textbf{Test poprawna odpowiedź:}\newline} %klucz, poprawna odpowiedź pytania testowego (jedna literka): A lub B lub C lub D itd.
\newcommand{\kluczStop}{\newline} %koniec poprawnej odpowiedzi pytania testowego 
\newcommand{\wstawGrafike}[2]{\begin{figure}[h] \includegraphics[scale=#2] {#1} \end{figure}} %gdyby była potrzeba wstawienia obrazka, parametry: nazwa pliku, skala (jak nie wiesz co wpisać, to wpisz 1)

\begin{document}
\maketitle


\kategoria{Wikieł/Z1.92g}
\zadStart{Zadanie z Wikieł Z 1.92 g) moja wersja nr [nrWersji]}
%[a]:[2,3,5,6,7,8,9,10,11,13,15,16]
%[b]:[1,2,3,4,5,6,7,8]
%[c]=[b]+1
%[p]=pow([c],1/2)
%[pp]=int([p].real)
%[p].is_integer()==True
Rozwiązać równanie $\log_{\frac{1}{[a]}}{(x^2-[b] \sqrt{[a]})}=-\frac{1}{2}$
\zadStop
\rozwStart{Małgorzata Ugowska}{}
Dziedzina:
$$x^2-[b] \sqrt{[a]} >0  \quad \Longrightarrow \quad D =  (-\infty, - \sqrt{[b] \sqrt{[a]}}) \cup (\sqrt{[b] \sqrt{[a]}}, \infty)$$
Rozwiązujemy równanie:
$$\log_{\frac{1}{[a]}}{(x^2-[b] \sqrt{[a]})}=-\frac{1}{2} \quad \Longleftrightarrow \quad x^2-[b] \sqrt{[a]}=\Big(\frac{1}{[a]}\Big)^{-\frac{1}{2}} \quad  \Longleftrightarrow \quad x^2-[b] \sqrt{[a]}= \sqrt{[a]}$$
$$\Longleftrightarrow \quad x^2-[c] \sqrt{[a]}= 0 \quad \Longleftrightarrow \quad (x- \sqrt{[c] \sqrt{[a]}})(x+ \sqrt{[c] \sqrt{[a]}}) =0 $$
$$ \Longleftrightarrow \quad x = [pp] \sqrt[4]{[a]} \in D \quad \vee \quad x = - [pp] \sqrt[4]{[a]} \in D$$
\rozwStop
\odpStart
$x \in \{- [pp] \sqrt[4]{[a]}, [pp] \sqrt[4]{[a]}\}$
\odpStop
\testStart
A. $x \in \{\sqrt{5}, \sqrt{6}\}$\\
B. $x \in \{- 1, 1\}$\\
C. $x \in \{- [c] \sqrt{[a]}, [c] \sqrt{[a]}\}$\\
D. $x \in \{- [pp] \sqrt[4]{[a]}, [pp] \sqrt[4]{[a]}\}$\\
E. $[c]$
\testStop
\kluczStart
D
\kluczStop



\end{document}