\documentclass[12pt, a4paper]{article}
\usepackage[utf8]{inputenc}
\usepackage{polski}

\usepackage{amsthm}  %pakiet do tworzenia twierdzeń itp.
\usepackage{amsmath} %pakiet do niektórych symboli matematycznych
\usepackage{amssymb} %pakiet do symboli mat., np. \nsubseteq
\usepackage{amsfonts}
\usepackage{graphicx} %obsługa plików graficznych z rozszerzeniem png, jpg
\theoremstyle{definition} %styl dla definicji
\newtheorem{zad}{} 
\title{Multizestaw zadań}
\author{Robert Fidytek}
%\date{\today}
\date{}
\newcounter{liczniksekcji}
\newcommand{\kategoria}[1]{\section{#1}} %olreślamy nazwę kateforii zadań
\newcommand{\zadStart}[1]{\begin{zad}#1\newline} %oznaczenie początku zadania
\newcommand{\zadStop}{\end{zad}}   %oznaczenie końca zadania
%Makra opcjonarne (nie muszą występować):
\newcommand{\rozwStart}[2]{\noindent \textbf{Rozwiązanie (autor #1 , recenzent #2): }\newline} %oznaczenie początku rozwiązania, opcjonarnie można wprowadzić informację o autorze rozwiązania zadania i recenzencie poprawności wykonania rozwiązania zadania
\newcommand{\rozwStop}{\newline}                                            %oznaczenie końca rozwiązania
\newcommand{\odpStart}{\noindent \textbf{Odpowiedź:}\newline}    %oznaczenie początku odpowiedzi końcowej (wypisanie wyniku)
\newcommand{\odpStop}{\newline}                                             %oznaczenie końca odpowiedzi końcowej (wypisanie wyniku)
\newcommand{\testStart}{\noindent \textbf{Test:}\newline} %ewentualne możliwe opcje odpowiedzi testowej: A. ? B. ? C. ? D. ? itd.
\newcommand{\testStop}{\newline} %koniec wprowadzania odpowiedzi testowych
\newcommand{\kluczStart}{\noindent \textbf{Test poprawna odpowiedź:}\newline} %klucz, poprawna odpowiedź pytania testowego (jedna literka): A lub B lub C lub D itd.
\newcommand{\kluczStop}{\newline} %koniec poprawnej odpowiedzi pytania testowego 
\newcommand{\wstawGrafike}[2]{\begin{figure}[h] \includegraphics[scale=#2] {#1} \end{figure}} %gdyby była potrzeba wstawienia obrazka, parametry: nazwa pliku, skala (jak nie wiesz co wpisać, to wpisz 1)

\begin{document}
\maketitle


\kategoria{Wikieł/Z1.84s}
\zadStart{Zadanie z Wikieł Z 1.84 s) moja wersja nr [nrWersji]}
%[a]:[2,3,4,5,6,7,8,9,10,11,12,13,14,15,16]
%[b]:[2,3,4,5,6,7,8,9]
%[z]:[2,3]
%[c]=[b]*[a]
%[d]:[2,6,12,20,30,42,56,72,90]
%[delta]=4*[d] +1
%[p]=int(math.sqrt([delta]))
%[t2]=int((1 + [p])/2)
%[t22]=[t2]*[t2]
%[t21]=int(pow([c],[z]))
%[z1]=[z]+1
%[z2]=[z]-1
%[z3]=[z]+2
%[z4]=[z]+4
%[z5]=[z]+3
%[p]!=1 and [t2]*[t2]-pow([c],[z])==0
Rozwiązać równanie $\sqrt{{[a]}^x} \cdot \sqrt{{[b]}^x}= {[c]}^x -[d]$.
\zadStop
\rozwStart{Barbara Bączek}{}
$$\sqrt{{[a]}^x} \cdot \sqrt{{[b]}^x}= {[c]}^x -[d]$$
$$\sqrt{{([a]\cdot [b])}^x} = {[c]}^x -[d]$$
$${[c]}^{\frac{x}{2}}={[c]}^x-[d]$$
$${[c]}^x-{[c]}^{\frac{x}{2}}-[d]=0$$
Niech $t:={[c]}^{\frac{x}{2}}$, skoro $[c]>0$ należy założyć, że $t>0$.
$$t^2-t-[d]=0$$
$$\Delta= 1+ 4\cdot[d]= [delta], \hspace{0.5 cm} \sqrt{\Delta}= [p]$$
$$t_1=\frac{1- [p]}{2} <0, \hspace{0.5 cm} t_2=\frac{1+[p]}{2}=[t2]$$
Zatem pozostaje nam rozwiązać: ${[c]}^{\frac{x}{2}}=[t2]$.
$${[c]}^{\frac{x}{2}}=[t2]$$
$${[c]}^x=[t22]$$
$$x=\log_{[c]}([t22])=[z]$$
\rozwStop
\odpStart
$[z]$
\odpStop
\testStart
A.$[z3]$
B.$[z1]$
C.$[z]$
D.$0$
E.$[z4]$
G.$[z5]$
H.$[z2]$
\testStop
\kluczStart
C
\kluczStop



\end{document}