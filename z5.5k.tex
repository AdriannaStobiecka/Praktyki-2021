\documentclass[12pt, a4paper]{article}
\usepackage[utf8]{inputenc}
\usepackage{polski}

\usepackage{amsthm}  %pakiet do tworzenia twierdzeń itp.
\usepackage{amsmath} %pakiet do niektórych symboli matematycznych
\usepackage{amssymb} %pakiet do symboli mat., np. \nsubseteq
\usepackage{amsfonts}
\usepackage{graphicx} %obsługa plików graficznych z rozszerzeniem png, jpg
\theoremstyle{definition} %styl dla definicji
\newtheorem{zad}{} 
\title{Multizestaw zadań}
\author{Robert Fidytek}
%\date{\today}
\date{}
\newcounter{liczniksekcji}
\newcommand{\kategoria}[1]{\section{#1}} %olreślamy nazwę kateforii zadań
\newcommand{\zadStart}[1]{\begin{zad}#1\newline} %oznaczenie początku zadania
\newcommand{\zadStop}{\end{zad}}   %oznaczenie końca zadania
%Makra opcjonarne (nie muszą występować):
\newcommand{\rozwStart}[2]{\noindent \textbf{Rozwiązanie (autor #1 , recenzent #2): }\newline} %oznaczenie początku rozwiązania, opcjonarnie można wprowadzić informację o autorze rozwiązania zadania i recenzencie poprawności wykonania rozwiązania zadania
\newcommand{\rozwStop}{\newline}                                            %oznaczenie końca rozwiązania
\newcommand{\odpStart}{\noindent \textbf{Odpowiedź:}\newline}    %oznaczenie początku odpowiedzi końcowej (wypisanie wyniku)
\newcommand{\odpStop}{\newline}                                             %oznaczenie końca odpowiedzi końcowej (wypisanie wyniku)
\newcommand{\testStart}{\noindent \textbf{Test:}\newline} %ewentualne możliwe opcje odpowiedzi testowej: A. ? B. ? C. ? D. ? itd.
\newcommand{\testStop}{\newline} %koniec wprowadzania odpowiedzi testowych
\newcommand{\kluczStart}{\noindent \textbf{Test poprawna odpowiedź:}\newline} %klucz, poprawna odpowiedź pytania testowego (jedna literka): A lub B lub C lub D itd.
\newcommand{\kluczStop}{\newline} %koniec poprawnej odpowiedzi pytania testowego 
\newcommand{\wstawGrafike}[2]{\begin{figure}[h] \includegraphics[scale=#2] {#1} \end{figure}} %gdyby była potrzeba wstawienia obrazka, parametry: nazwa pliku, skala (jak nie wiesz co wpisać, to wpisz 1)

\begin{document}
\maketitle


\kategoria{Wikieł/Z5.5k}
\zadStart{Zadanie z Wikieł Z 5.5 k) moja wersja nr [nrWersji]}
%[a]:[2,3,4,5,6,7,8,9]
%[b]:[2,3,4,5,6,7,8,9]
%[c]=[a]*[b]
%[d]=[b]*[b]
%math.gcd([c],2)==1 
Wyznacz pochodną funkcji \\ $f(x)=[a] \tg^{-1}\left( \sqrt{\frac{1-[b]x}{1+[b]x}} \right)$.
\zadStop
\rozwStart{Joanna Świerzbin}{}
$$f(x)=[a] \tg^{-1}\left( \sqrt{\frac{1-[b]x}{1+[b]x}} \right)$$
$$f'(x)= \left( [a] \tg^{-1}\left( \sqrt{\frac{1-[b]x}{1+[b]x}} \right) \right)'  = \frac{[a]}{\frac{1-[b]x}{1+[b]x}+1} \left( \sqrt{\frac{1-[b]x}{1+[b]x}} \right)' =$$
$$ =\frac{1}{2} \cdot \frac{[a]}{\frac{1-[b]x+1+[b]x}{1+[b]x}}\cdot \frac{1}{\sqrt{\frac{1-[b]x}{1+[b]x}}} \left( \frac{1-[b]x}{1+[b]x} \right)' =$$
$$ = \frac{[a]}{2} \cdot \frac{1}{\frac{2}{1+[b]x}} \cdot \frac{\sqrt{1+[b]x}}{\sqrt{1-[b]x}} \cdot \frac{-[b](1+[b]x)-[b](1-[b]x)}{(1+[b]x)^2} =$$
$$ = \frac{[a](1+[b]x)}{4} \cdot \frac{\sqrt{1+[b]x}}{\sqrt{1-[b]x}} \cdot \frac{-[b]-[b]^2x-[b]+[b]^2x}{(1+[b]x)^2} =$$
$$ = \frac{[a]}{4} \cdot \frac{\sqrt{1+[b]x}}{\sqrt{1-[b]x}} \cdot \frac{-2\cdot[b]}{1+[b]x} = - \frac{[a]\cdot[b]}{2(1+[b]x)} \cdot \frac{\sqrt{1+[b]x}}{\sqrt{1-[b]x}} =$$
$$ = - \frac{[c]}{2} \cdot \frac{1}{\sqrt{1+[b]x}\sqrt{1-[b]x}} = - \frac{[c]}{2} \cdot \frac{1}{\sqrt{(1+[b]x)(1-[b]x)}} =$$
$$ = - \frac{[c]}{2 (\sqrt{1-[b]^2x^2})} = - \frac{[c]}{2 \sqrt{1-[d]x^2}}$$
\rozwStop
\odpStart
$ f'(x) =  - \frac{[c]}{2 \sqrt{1-[d]x^2}} $
\odpStop
\testStart
A. $ f'(x) =  - \frac{[c]}{2 \sqrt{1-[d]x^2}} $\\
B. $ f'(x) =  - \frac{1}{2 \sqrt{1-[d]x^2}} $ \\
C. $ f'(x) =  - \frac{[c]}{\sqrt{1-[d]x^2}} $\\
D. $ f'(x) =  - \frac{[c]}{2 \sqrt{1-x^2}} $\\
E. $ f'(x) =  - \frac{[c]}{2 \sqrt{1-[c]x^2}} $\\
F. $ f'(x) =  - \frac{[c]}{2 \sqrt{1+[d]x^2}} $
\testStop
\kluczStart
A
\kluczStop



\end{document}