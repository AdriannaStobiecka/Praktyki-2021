\documentclass[12pt, a4paper]{article}
\usepackage[utf8]{inputenc}
\usepackage{polski}

\usepackage{amsthm}  %pakiet do tworzenia twierdzeń itp.
\usepackage{amsmath} %pakiet do niektórych symboli matematycznych
\usepackage{amssymb} %pakiet do symboli mat., np. \nsubseteq
\usepackage{amsfonts}
\usepackage{graphicx} %obsługa plików graficznych z rozszerzeniem png, jpg
\theoremstyle{definition} %styl dla definicji
\newtheorem{zad}{} 
\title{Multizestaw zadań}
\author{Robert Fidytek}
%\date{\today}
\date{}
\newcounter{liczniksekcji}
\newcommand{\kategoria}[1]{\section{#1}} %olreślamy nazwę kateforii zadań
\newcommand{\zadStart}[1]{\begin{zad}#1\newline} %oznaczenie początku zadania
\newcommand{\zadStop}{\end{zad}}   %oznaczenie końca zadania
%Makra opcjonarne (nie muszą występować):
\newcommand{\rozwStart}[2]{\noindent \textbf{Rozwiązanie (autor #1 , recenzent #2): }\newline} %oznaczenie początku rozwiązania, opcjonarnie można wprowadzić informację o autorze rozwiązania zadania i recenzencie poprawności wykonania rozwiązania zadania
\newcommand{\rozwStop}{\newline}                                            %oznaczenie końca rozwiązania
\newcommand{\odpStart}{\noindent \textbf{Odpowiedź:}\newline}    %oznaczenie początku odpowiedzi końcowej (wypisanie wyniku)
\newcommand{\odpStop}{\newline}                                             %oznaczenie końca odpowiedzi końcowej (wypisanie wyniku)
\newcommand{\testStart}{\noindent \textbf{Test:}\newline} %ewentualne możliwe opcje odpowiedzi testowej: A. ? B. ? C. ? D. ? itd.
\newcommand{\testStop}{\newline} %koniec wprowadzania odpowiedzi testowych
\newcommand{\kluczStart}{\noindent \textbf{Test poprawna odpowiedź:}\newline} %klucz, poprawna odpowiedź pytania testowego (jedna literka): A lub B lub C lub D itd.
\newcommand{\kluczStop}{\newline} %koniec poprawnej odpowiedzi pytania testowego 
\newcommand{\wstawGrafike}[2]{\begin{figure}[h] \includegraphics[scale=#2] {#1} \end{figure}} %gdyby była potrzeba wstawienia obrazka, parametry: nazwa pliku, skala (jak nie wiesz co wpisać, to wpisz 1)

\begin{document}
\maketitle


\kategoria{Wikieł/Z3.14g}
\zadStart{Zadanie z Wikieł Z 3.14 g) moja wersja nr [nrWersji]}
%[z1]:[1,2,3,4,5]
%[b]:[3,4,5,6,7,8,9,10,11,12,13,14,15,16,17,18,19,20,21,22]
%[b1]=[b]-1
%[a]=random.randint(2,[b1])
%[z]=[a]+[b]
Obliczyć granicę ciągu $a_n= \sqrt[n]{{[a]}^n+{[b]}^n +\cos^2(n)}$.
\zadStop
\rozwStart{Barbara Bączek}{}
$$\lim_{n \rightarrow \infty} a_n= \lim_{n \rightarrow \infty}  \sqrt[n]{{[a]}^n+{[b]}^n +\cos^2(n)} $$
Ograniczając $a_n$ z obu stron za pomocą $\cos^2(n)\in [0,1]$:
$$\sqrt[n]{{[a]}^n+{[b]}^n + 0} \leq \sqrt[n]{{[a]}^n+{[b]}^n +\cos^2(n)} \leq \sqrt[n]{{[a]}^n+{[b]}^n + 1} $$
$$\sqrt[n]{{[a]}^n+{[b]}^n} \leq \sqrt[n]{{[a]}^n+{[b]}^n +\cos^2(n)} \leq \sqrt[n]{{[a]}^n+{[b]}^n +1}$$
$$[b] \cdot \sqrt[n]{\Big{(}{\frac{[a]}{[b]}\Big{)}}^n+ 1} \leq \sqrt[n]{{[a]}^n+{[b]}^n +\cos^2(n)} \leq [b] \cdot \sqrt[n]{\Big{(}{\frac{[a]}{[b]}\Big{)}}^n+ 1 + \frac{1}{{[b]}^n}}$$
Zatem, gdy $n \rightarrow \infty$, otrzymujemy
$$[b] \leq  \lim_{n \rightarrow \infty} \sqrt[n]{{[a]}^n+{[b]}^n +\cos^2(n)} \leq [b] $$
Podsumowując, z twierdzenia o trzech ciągach:  $$\lim_{n \rightarrow \infty} a_n= [b]$$
\rozwStop
\odpStart
$[b]$
\odpStop
\testStart
A.$\infty$
B.$[a]$
C.$-\infty$
D.$0$
E.$[z]$
G.$[b]$
H.$-[b]$
\testStop
\kluczStart
G
\kluczStop



\end{document}