\documentclass[12pt, a4paper]{article}
\usepackage[utf8]{inputenc}
\usepackage{polski}

\usepackage{amsthm}  %pakiet do tworzenia twierdzeń itp.
\usepackage{amsmath} %pakiet do niektórych symboli matematycznych
\usepackage{amssymb} %pakiet do symboli mat., np. \nsubseteq
\usepackage{amsfonts}
\usepackage{graphicx} %obsługa plików graficznych z rozszerzeniem png, jpg
\theoremstyle{definition} %styl dla definicji
\newtheorem{zad}{} 
\title{Multizestaw zadań}
\author{Robert Fidytek}
%\date{\today}
\date{}
\newcounter{liczniksekcji}
\newcommand{\kategoria}[1]{\section{#1}} %olreślamy nazwę kateforii zadań
\newcommand{\zadStart}[1]{\begin{zad}#1\newline} %oznaczenie początku zadania
\newcommand{\zadStop}{\end{zad}}   %oznaczenie końca zadania
%Makra opcjonarne (nie muszą występować):
\newcommand{\rozwStart}[2]{\noindent \textbf{Rozwiązanie (autor #1 , recenzent #2): }\newline} %oznaczenie początku rozwiązania, opcjonarnie można wprowadzić informację o autorze rozwiązania zadania i recenzencie poprawności wykonania rozwiązania zadania
\newcommand{\rozwStop}{\newline}                                            %oznaczenie końca rozwiązania
\newcommand{\odpStart}{\noindent \textbf{Odpowiedź:}\newline}    %oznaczenie początku odpowiedzi końcowej (wypisanie wyniku)
\newcommand{\odpStop}{\newline}                                             %oznaczenie końca odpowiedzi końcowej (wypisanie wyniku)
\newcommand{\testStart}{\noindent \textbf{Test:}\newline} %ewentualne możliwe opcje odpowiedzi testowej: A. ? B. ? C. ? D. ? itd.
\newcommand{\testStop}{\newline} %koniec wprowadzania odpowiedzi testowych
\newcommand{\kluczStart}{\noindent \textbf{Test poprawna odpowiedź:}\newline} %klucz, poprawna odpowiedź pytania testowego (jedna literka): A lub B lub C lub D itd.
\newcommand{\kluczStop}{\newline} %koniec poprawnej odpowiedzi pytania testowego 
\newcommand{\wstawGrafike}[2]{\begin{figure}[h] \includegraphics[scale=#2] {#1} \end{figure}} %gdyby była potrzeba wstawienia obrazka, parametry: nazwa pliku, skala (jak nie wiesz co wpisać, to wpisz 1)

\begin{document}
\maketitle


\kategoria{Wikieł/Z3.15f}
\zadStart{Zadanie z Wikieł Z 3.15 f) moja wersja nr [nrWersji]}
%[a]:[8,9,10,11,12]
%[b]:[8,9,10,11,12]
%[c]:[8,9,10,11,12]
%[d]:[8,9,10,11,12]
%[e]:[8,9,10,11,12]
%[bc]=[b]-[c]
%[dbc]=[d]*[bc]
%[de]=[d]+[e]
%math.gcd([b],[a])==1 and math.gcd([c],[a])==1 and math.gcd([dbc],[a])==1 and [dbc]!=0 and [dbc]!=1  and [de]!=[dbc] and [de]!=1 and [de]!=0 and [b]!=[c] and math.gcd([b],[c])==1
Obliczyć granicę ciągu 
$$a_n=\bigg(\frac{[a]n+[b]}{[a]n+[c]}\bigg)^{[d]n+[e]}.$$
\zadStop
\rozwStart{Adrianna Stobiecka}{}
$$\lim_{n\to\infty}\bigg(\frac{[a]n+[b]}{[a]n+[c]}\bigg)^{[d]n+[e]}=\lim_{n\to\infty}\Bigg(\frac{[a]n\big(1+\frac{\frac{[b]}{[a]}}{n}\big)}{[a]n\big(1+\frac{\frac{[c]}{[a]}}{n}\big)}\Bigg)^{[d]n+[e]}=\lim_{n\to\infty}\Bigg(\frac{1+\frac{\frac{[b]}{[a]}}{n}}{1+\frac{\frac{[c]}{[a]}}{n}}\Bigg)^{[d]n+[e]}$$
$$=\lim_{n\to\infty}\frac{\big(1+\frac{\frac{[b]}{[a]}}{n}\big)^{[d]n}}{\big(1+\frac{\frac{[c]}{[a]}}{n}\big)^{[d]n}}\Bigg(\frac{1+\frac{\frac{[b]}{[a]}}{n}}{1+\frac{\frac{[c]}{[a]}}{n}}\Bigg)^{[e]}=\lim_{n\to\infty}\frac{\big[\big(1+\frac{\frac{[b]}{[a]}}{n}\big)^{\frac{n}{\frac{[b]}{[a]}}}\big]^{\frac{[b]}{[a]}\cdot[d]}}{\big[\big(1+\frac{\frac{[c]}{[a]}}{n}\big)^{\frac{n}{\frac{[c]}{[a]}}}\big]^{\frac{[c]}{[a]}\cdot[d]}}\Bigg(\frac{1+\frac{\frac{[b]}{[a]}}{n}}{1+\frac{\frac{[c]}{[a]}}{n}}\Bigg)^{[e]}$$
$$=\frac{e^{\frac{[b]}{[a]}\cdot[d]}}{e^{\frac{[c]}{[a]}\cdot[d]}}\cdot1^{[e]}=e^{\frac{[b]}{[a]}\cdot[d]-\frac{[c]}{[a]}\cdot[d]}\cdot1=e^{\frac{[d]}{[a]}([b]-[c])}=e^{\frac{[dbc]}{[a]}}$$
\rozwStop
\odpStart
$e^{\frac{[dbc]}{[a]}}$
\odpStop
\testStart
A.$0$
B.$e^{\frac{[b]}{[c]}}$
C.$\infty$
D.$\frac{[dbc]}{[a]}$
E.$1$
F.$-\infty$
G.$e^{\frac{[dbc]}{[a]}}$
H.$e^{[de]}$
I.$e^{[d]}$
\testStop
\kluczStart
G
\kluczStop



\end{document}
