\documentclass[12pt, a4paper]{article}
\usepackage[utf8]{inputenc}
\usepackage{polski}

\usepackage{amsthm}  %pakiet do tworzenia twierdzeń itp.
\usepackage{amsmath} %pakiet do niektórych symboli matematycznych
\usepackage{amssymb} %pakiet do symboli mat., np. \nsubseteq
\usepackage{amsfonts}
\usepackage{graphicx} %obsługa plików graficznych z rozszerzeniem png, jpg
\theoremstyle{definition} %styl dla definicji
\newtheorem{zad}{} 
\title{Multizestaw zadań}
\author{Robert Fidytek}
%\date{\today}
\date{}
\newcounter{liczniksekcji}
\newcommand{\kategoria}[1]{\section{#1}} %olreślamy nazwę kateforii zadań
\newcommand{\zadStart}[1]{\begin{zad}#1\newline} %oznaczenie początku zadania
\newcommand{\zadStop}{\end{zad}}   %oznaczenie końca zadania
%Makra opcjonarne (nie muszą występować):
\newcommand{\rozwStart}[2]{\noindent \textbf{Rozwiązanie (autor #1 , recenzent #2): }\newline} %oznaczenie początku rozwiązania, opcjonarnie można wprowadzić informację o autorze rozwiązania zadania i recenzencie poprawności wykonania rozwiązania zadania
\newcommand{\rozwStop}{\newline}                                            %oznaczenie końca rozwiązania
\newcommand{\odpStart}{\noindent \textbf{Odpowiedź:}\newline}    %oznaczenie początku odpowiedzi końcowej (wypisanie wyniku)
\newcommand{\odpStop}{\newline}                                             %oznaczenie końca odpowiedzi końcowej (wypisanie wyniku)
\newcommand{\testStart}{\noindent \textbf{Test:}\newline} %ewentualne możliwe opcje odpowiedzi testowej: A. ? B. ? C. ? D. ? itd.
\newcommand{\testStop}{\newline} %koniec wprowadzania odpowiedzi testowych
\newcommand{\kluczStart}{\noindent \textbf{Test poprawna odpowiedź:}\newline} %klucz, poprawna odpowiedź pytania testowego (jedna literka): A lub B lub C lub D itd.
\newcommand{\kluczStop}{\newline} %koniec poprawnej odpowiedzi pytania testowego 
\newcommand{\wstawGrafike}[2]{\begin{figure}[h] \includegraphics[scale=#2] {#1} \end{figure}} %gdyby była potrzeba wstawienia obrazka, parametry: nazwa pliku, skala (jak nie wiesz co wpisać, to wpisz 1)

\begin{document}
\maketitle


\kategoria{Wikieł/Z1.48}
\zadStart{Zadanie z Wikieł Z 1.48 moja wersja nr [nrWersji]}
%[p1]:[2,3,4,5,6,7,8,9]
%[p2]:[2,3,4,5,6,7,8,9]
%[p3]=random.randint(2,10)
%[p4]=random.randint(2,10)
%[p2k]=[p2]*[p2]
%[2p2p3]=2*[p2]*[p3]
%[p3k]=[p3]*[p3]
%[4p1]=4*[p1]
%[4p1p4]=4*[p1]*[p4]
%[m]=[2p2p3]-[4p1]
%[r]=[p3k]-[4p1p4]
%[del]=[m]*[m]-4*[p2k]*[r]
%[pdel]=round(math.sqrt(abs([del])),2)
%[2p2k]=2*[p2k]
%[m1]=round((-[m]-[pdel])/[2p2k],2)
%[m2]=round((-[m]+[pdel])/[2p2k],2)
%[p1k]=[p1]*[p1]
%[2p4]=2*[p4]
%[2p1]=2*[p1]
%[2p4p1]=[2p4]*[p1]
%[mr]=[2p2p3]+[2p1]
%[r2]=[p3k]-[2p4p1]
%[r3]=[r2]-[p1k]
%[2p2k]=2*[p2k]
%[del2]=[mr]*[mr]-4*[p2k]*[r3]
%[pdel2]=round(math.sqrt(abs([del2])),2)
%[m11]=round(([mr]-[pdel2])/[2p2k],2)
%[m22]=round(([mr]+[pdel2])/[2p2k],2)
%[del]>0 and [m1]<[m2] and [del2]>0 and [m11]<[m22] and [m1]<[m11] and [m2]<[m22] and [r]<0

Wyznaczyć wartość parametru $m$, dla których pierwiastki rzeczywiste $x_{1}$ i $x_{2}$ równania $[p1]x^{2}+([p2]m-[p3])x+m+[p4]=0$ spełniają warunek $x_{1}^{2}+x_{2}^{2}>1.$
\zadStop

\rozwStart{Maja Szabłowska}{}
Aby równanie posiadało dwa pierwiastki powinien być spełniony warunek $\Delta\geq0.$
$$\Delta=([p2]m+[p3])^{2}-4\cdot[p1]\cdot(m+[p4])=[p2k]m^{2}+[2p2p3]m+[p3k]-[4p1]m-[4p1p4]=$$
$$=[p2k]m^{2}+[m]m+([r])\geq0$$

$$[p2k]m^{2}+[m]m+([r])\geq0$$
$$\Delta=[m]^{2}-4\cdot[p2k]\cdot([r])=[del] \Rightarrow \sqrt{\Delta}=[pdel]$$
$$m_{1}=\frac{-[m]-[pdel]}{[2p2k]}=[m1], \quad m_{2}=\frac{-[m]+[pdel]}{[2p2k]}=[m2]$$
Zatem $m\in(-\infty,[m1]]\cup[[m2],\infty)$

Przejdźmy do warunku:
$$x_{1}^{2}+x_{2}^{2}>1$$
Korzystamy ze wzorów Viete'a:
$$(x_{1}+x_{2})^{2}=x_{1}^{2}+x_{2}^{2}+2x_{1}x_{2}$$
$$x_{1}^{2}+x_{2}^{2}=(x_{1}+x_{2})^{2}-2x_{1}x_{2}>1$$

$$\left(\frac{-[p2]m+[p3]}{[p1]}\right)^{2}-2\cdot \frac{m+[p4]}{[p1]}>1$$
$$\frac{[p2k]m^{2}-[2p2p3]m+[p3k]}{[p1k]}-\frac{2m+[2p4]}{[p1]}>1$$
$$\frac{[p2k]m^{2}-[2p2p3]m+[p3k]}{[p1k]}-\frac{[2p1]m+[2p4p1]}{[p1k]}>1$$
$$\frac{[p2k]m^{2}-[mr]m+([r2])}{[p1k]}>1$$
$$[p2k]m^{2}-[mr]m+([r2])>[p1k]$$
$$[p2k]m^{2}-[mr]m+([r3])>0$$
$$\Delta=(-[mr])^{2}-4\cdot[p2k]\cdot([r3])=[del2] \Rightarrow \sqrt{\Delta}=[pdel2]$$
$$m_{1}=\frac{[mr]-[pdel2]}{[2p2k]}=[m11], \quad m_{2}=\frac{[mr]+[pdel2]}{[2p2k]}=[m22]$$
$$m\in(-\infty, [m11])\cup([m22],\infty)$$
Zatem ostatecznie:
$$m\in(-\infty, [m11])\cup([m22],\infty)$$
\rozwStop


\odpStart
$m\in(-\infty, [m11])\cup([m22],\infty)$
\odpStop
\testStart
A.$m\in(-\infty, [m11])\cup([m22],\infty)$
B.$m\in([m2],[m1])$
C.$m\in(-\infty,[m22])$
D.$m\in([m1],[m2])$
E.$m\in([p2k],[p1])$
F.$m\in[[m1],\infty)$
\testStop
\kluczStart
A
\kluczStop



\end{document}
