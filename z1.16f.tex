\documentclass[12pt, a4paper]{article}
\usepackage[utf8]{inputenc}
\usepackage{polski}
\usepackage{amsthm}  %pakiet do tworzenia twierdzeń itp.
\usepackage{amsmath} %pakiet do niektórych symboli matematycznych
\usepackage{amssymb} %pakiet do symboli mat., np. \nsubseteq
\usepackage{amsfonts}
\usepackage{graphicx} %obsługa plików graficznych z rozszerzeniem png, jpg
\theoremstyle{definition} %styl dla definicji
\newtheorem{zad}{} 
\title{Multizestaw zadań}
\author{Patryk Wirkus}
%\date{\today}
\date{}
\newcommand{\kategoria}[1]{\section{#1}}
\newcommand{\zadStart}[1]{\begin{zad}#1\newline}
\newcommand{\zadStop}{\end{zad}}
\newcommand{\rozwStart}[2]{\noindent \textbf{Rozwiązanie (autor #1 , recenzent #2): }\newline}
\newcommand{\rozwStop}{\newline}                                           
\newcommand{\odpStart}{\noindent \textbf{Odpowiedź:}\newline}
\newcommand{\odpStop}{\newline}
\newcommand{\testStart}{\noindent \textbf{Test:}\newline}
\newcommand{\testStop}{\newline}
\newcommand{\kluczStart}{\noindent \textbf{Test poprawna odpowiedź:}\newline}
\newcommand{\kluczStop}{\newline}
\newcommand{\wstawGrafike}[2]{\begin{figure}[h] \includegraphics[scale=#2] {#1} \end{figure}}

\begin{document}
\maketitle

\kategoria{Wikieł/Z1.16f}


\zadStart{Zadanie z Wikieł Z 1.16 f) moja wersja nr 1}

Obliczyć symbol Newtona ${63 \choose 63-1}$.
\zadStop
\rozwStart{Patryk Wirkus}{Szymon Tokarski}
$${63 \choose 63-1} = \frac{63!}{1! \cdot (63-1)!} = \frac{63}{1!} = \frac{63}{1} = 63$$
\rozwStop
\odpStart
$63$
\odpStop
\testStart
A.$63$ B.$-63$ C.$0$ D.$1$ E.$-1$
\testStop
\kluczStart
A
\kluczStop



\zadStart{Zadanie z Wikieł Z 1.16 f) moja wersja nr 2}

Obliczyć symbol Newtona ${67 \choose 67-1}$.
\zadStop
\rozwStart{Patryk Wirkus}{Szymon Tokarski}
$${67 \choose 67-1} = \frac{67!}{1! \cdot (67-1)!} = \frac{67}{1!} = \frac{67}{1} = 67$$
\rozwStop
\odpStart
$67$
\odpStop
\testStart
A.$67$ B.$-67$ C.$0$ D.$1$ E.$-1$
\testStop
\kluczStart
A
\kluczStop



\zadStart{Zadanie z Wikieł Z 1.16 f) moja wersja nr 3}

Obliczyć symbol Newtona ${73 \choose 73-1}$.
\zadStop
\rozwStart{Patryk Wirkus}{Szymon Tokarski}
$${73 \choose 73-1} = \frac{73!}{1! \cdot (73-1)!} = \frac{73}{1!} = \frac{73}{1} = 73$$
\rozwStop
\odpStart
$73$
\odpStop
\testStart
A.$73$ B.$-73$ C.$0$ D.$1$ E.$-1$
\testStop
\kluczStart
A
\kluczStop



\zadStart{Zadanie z Wikieł Z 1.16 f) moja wersja nr 4}

Obliczyć symbol Newtona ${79 \choose 79-1}$.
\zadStop
\rozwStart{Patryk Wirkus}{Szymon Tokarski}
$${79 \choose 79-1} = \frac{79!}{1! \cdot (79-1)!} = \frac{79}{1!} = \frac{79}{1} = 79$$
\rozwStop
\odpStart
$79$
\odpStop
\testStart
A.$79$ B.$-79$ C.$0$ D.$1$ E.$-1$
\testStop
\kluczStart
A
\kluczStop



\zadStart{Zadanie z Wikieł Z 1.16 f) moja wersja nr 5}

Obliczyć symbol Newtona ${51 \choose 51-1}$.
\zadStop
\rozwStart{Patryk Wirkus}{Szymon Tokarski}
$${51 \choose 51-1} = \frac{51!}{1! \cdot (51-1)!} = \frac{51}{1!} = \frac{51}{1} = 51$$
\rozwStop
\odpStart
$51$
\odpStop
\testStart
A.$51$ B.$-51$ C.$0$ D.$1$ E.$-1$
\testStop
\kluczStart
A
\kluczStop



\zadStart{Zadanie z Wikieł Z 1.16 f) moja wersja nr 6}

Obliczyć symbol Newtona ${57 \choose 57-1}$.
\zadStop
\rozwStart{Patryk Wirkus}{Szymon Tokarski}
$${57 \choose 57-1} = \frac{57!}{1! \cdot (57-1)!} = \frac{57}{1!} = \frac{57}{1} = 57$$
\rozwStop
\odpStart
$57$
\odpStop
\testStart
A.$57$ B.$-57$ C.$0$ D.$1$ E.$-1$
\testStop
\kluczStart
A
\kluczStop



\zadStart{Zadanie z Wikieł Z 1.16 f) moja wersja nr 7}

Obliczyć symbol Newtona ${61 \choose 61-1}$.
\zadStop
\rozwStart{Patryk Wirkus}{Szymon Tokarski}
$${61 \choose 61-1} = \frac{61!}{1! \cdot (61-1)!} = \frac{61}{1!} = \frac{61}{1} = 61$$
\rozwStop
\odpStart
$61$
\odpStop
\testStart
A.$61$ B.$-61$ C.$0$ D.$1$ E.$-1$
\testStop
\kluczStart
A
\kluczStop



\zadStart{Zadanie z Wikieł Z 1.16 f) moja wersja nr 8}

Obliczyć symbol Newtona ${69 \choose 69-1}$.
\zadStop
\rozwStart{Patryk Wirkus}{Szymon Tokarski}
$${69 \choose 69-1} = \frac{69!}{1! \cdot (69-1)!} = \frac{69}{1!} = \frac{69}{1} = 69$$
\rozwStop
\odpStart
$69$
\odpStop
\testStart
A.$69$ B.$-69$ C.$0$ D.$1$ E.$-1$
\testStop
\kluczStart
A
\kluczStop



\zadStart{Zadanie z Wikieł Z 1.16 f) moja wersja nr 9}

Obliczyć symbol Newtona ${71 \choose 71-1}$.
\zadStop
\rozwStart{Patryk Wirkus}{Szymon Tokarski}
$${71 \choose 71-1} = \frac{71!}{1! \cdot (71-1)!} = \frac{71}{1!} = \frac{71}{1} = 71$$
\rozwStop
\odpStart
$71$
\odpStop
\testStart
A.$71$ B.$-71$ C.$0$ D.$1$ E.$-1$
\testStop
\kluczStart
A
\kluczStop



\zadStart{Zadanie z Wikieł Z 1.16 f) moja wersja nr 10}

Obliczyć symbol Newtona ${77 \choose 77-1}$.
\zadStop
\rozwStart{Patryk Wirkus}{Szymon Tokarski}
$${77 \choose 77-1} = \frac{77!}{1! \cdot (77-1)!} = \frac{77}{1!} = \frac{77}{1} = 77$$
\rozwStop
\odpStart
$77$
\odpStop
\testStart
A.$77$ B.$-77$ C.$0$ D.$1$ E.$-1$
\testStop
\kluczStart
A
\kluczStop





\end{document}
