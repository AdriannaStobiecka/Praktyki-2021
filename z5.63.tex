\documentclass[12pt, a4paper]{article}
\usepackage[utf8]{inputenc}
\usepackage{polski}

\usepackage{amsthm}  %pakiet do tworzenia twierdzeń itp.
\usepackage{amsmath} %pakiet do niektórych symboli matematycznych
\usepackage{amssymb} %pakiet do symboli mat., np. \nsubseteq
\usepackage{amsfonts}
\usepackage{graphicx} %obsługa plików graficznych z rozszerzeniem png, jpg
\theoremstyle{definition} %styl dla definicji
\newtheorem{zad}{} 
\title{Multizestaw zadań}
\author{Robert Fidytek}
%\date{\today}
\date{}
\newcounter{liczniksekcji}
\newcommand{\kategoria}[1]{\section{#1}} %olreślamy nazwę kateforii zadań
\newcommand{\zadStart}[1]{\begin{zad}#1\newline} %oznaczenie początku zadania
\newcommand{\zadStop}{\end{zad}}   %oznaczenie końca zadania
%Makra opcjonarne (nie muszą występować):
\newcommand{\rozwStart}[2]{\noindent \textbf{Rozwiązanie (autor #1 , recenzent #2): }\newline} %oznaczenie początku rozwiązania, opcjonarnie można wprowadzić informację o autorze rozwiązania zadania i recenzencie poprawności wykonania rozwiązania zadania
\newcommand{\rozwStop}{\newline}                                            %oznaczenie końca rozwiązania
\newcommand{\odpStart}{\noindent \textbf{Odpowiedź:}\newline}    %oznaczenie początku odpowiedzi końcowej (wypisanie wyniku)
\newcommand{\odpStop}{\newline}                                             %oznaczenie końca odpowiedzi końcowej (wypisanie wyniku)
\newcommand{\testStart}{\noindent \textbf{Test:}\newline} %ewentualne możliwe opcje odpowiedzi testowej: A. ? B. ? C. ? D. ? itd.
\newcommand{\testStop}{\newline} %koniec wprowadzania odpowiedzi testowych
\newcommand{\kluczStart}{\noindent \textbf{Test poprawna odpowiedź:}\newline} %klucz, poprawna odpowiedź pytania testowego (jedna literka): A lub B lub C lub D itd.
\newcommand{\kluczStop}{\newline} %koniec poprawnej odpowiedzi pytania testowego 
\newcommand{\wstawGrafike}[2]{\begin{figure}[h] \includegraphics[scale=#2] {#1} \end{figure}} %gdyby była potrzeba wstawienia obrazka, parametry: nazwa pliku, skala (jak nie wiesz co wpisać, to wpisz 1)

\begin{document}
\maketitle


\kategoria{Wikieł/Z5.63}
\zadStart{Zadanie z Wikieł Z 5.63 ) moja wersja nr [nrWersji]}
%[a]:[2,3,4,5,6,7,8,9,10,11,12,18,32]
%[b]:[2,3,4,5,6,7,8,9]
%[c]:[2,3,4,5,6,7,8,9]
%[d]:[2,3,4,5,6,7,8,9]
%[a2]=[a]*2
%[ak]=[a]*[a]
%[db]=[d]-[b]
%[2c]=2*[c]
%[2db]=[db]*2
%[ck]=[c]*[c]
%[2cdb]=2*[c]*[db]
%[dbk]=[db]*[db]
%[x2]=[2db]+[ck]+1
%[akdbk]=[dbk]+[ak]
%[6c]=[2c]*3
%[x22]=[x2]*2
%[6ck]=[6c]*[6c]
%[4ab]=16*[x22]
%[delta]=[6ck]-[4ab]
%[pierw]=round(math.sqrt(abs([delta])),2)
%[x_1]=round((-[6c]-[pierw])/8,2)
%[x_2]=round((-[6c]+[pierw])/8,2)
%[absx_1]=abs([x_1])
%[absx_2]=abs([x_2])
%[d0]=round(math.sqrt([a]*[a]+[db]*[db]),2)
%[d1]=round(math.sqrt(([x_1]-[a])**(2)+([x_1]**(2)-[c]*[x_1]+[db])**(2)),2)
%[d2]=round(math.sqrt(([x_2]-[a])**(2)+([x_2]**(2)-[c]*[x_2]+[db])**(2)),2)
%[y]=round([x_2]**(2)+[c]*[x_2]+[d],2)
%[d]>[b] and (2*[c]*([d]-[b]))==([a]*2) and [delta]>0 and [d0]<[d1] and [d0]<[d2]
Wyznaczyć współrzędne punktu A należącego do paraboli o równaniu $y=x^{2}+[c]x+[d]$ tak, aby jego odległość od punktu $P([a],[b])$ była najmniejsza.
\zadStop
\rozwStart{Wojciech Przybylski}{}
$$P=([a],[b]),\hspace{3mm} A=(x, x^{2}+[c]x+[d])$$
$$d=d(x)=\sqrt{(x-[a])^{2}+(x^{2}+[c]x+[d]-[b])^{2}}=$$
$$=\sqrt{x^{2}-[a2]x+[ak]+x^{4}+[2c]x^{3}+[2db]x^{2}+[ck]x^{2}+[2cdb]x+[dbk]}$$
$$d(x)=\sqrt{x^{4}+[2c]x^{3}+[x2]x^{2}+[akdbk]}$$
$$d'(x)=\frac{4x^{3}+[6c]x^{2}+[x22]x}{2\sqrt{x^{4}-[2c]x^{3}+[x2]x^{2}+[akdbk]}}$$
$$d'(x)=0 \Rightarrow 4x^{3}+[6c]x^{2}+[x22]x=0 \Rightarrow x(4x^{2}+[6c]x+[x22])=0$$
$$\Delta=[6ck]-[4ab]=[delta] \Rightarrow \sqrt{\Delta}=[pierw]$$
$$x_{1}=0\vee x_{2}=\frac{-[6c]-[pierw]}{8}=[x_1] \vee x_{3}=\frac{-[6c]+[pierw]}{8}=[x_2]$$
$$d(0)=\sqrt{(0-[a])^{2}+(0^{2}+[c]\cdot 0+[db])^{2}}=[d0]$$
$$d([x_1])=\sqrt{([x_1]-[a])^{2}+([x_1]^{2}-[c]\cdot [absx_1]+[db])^{2}}=[d1]$$
$$d([x_2])=\sqrt{([x_2]-[a])^{2}+([x_2]^{2}-[c]\cdot [absx_2]+[db])^{2}}=[d2]$$
Stąd wiemy, że A ma współrzędne $(0,[d])$.
\rozwStop
\odpStart
$A=(0,[d])$.
\odpStop
\testStart
A. $A=(0,[d])$.\\
B. $A=([x_1],[y])$.\\
C. $A=([x_2],[db])$.\\
D.$A=([x_1],[d2])$.\\
E. $A=(0,[db])$.\\
F. Nie istnieją taki punkt A.
\testStop
\kluczStart
A
\kluczStop



\end{document}