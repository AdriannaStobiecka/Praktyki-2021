\documentclass[12pt, a4paper]{article}
\usepackage[utf8]{inputenc}
\usepackage{polski}

\usepackage{amsthm}  %pakiet do tworzenia twierdzeń itp.
\usepackage{amsmath} %pakiet do niektórych symboli matematycznych
\usepackage{amssymb} %pakiet do symboli mat., np. \nsubseteq
\usepackage{amsfonts}
\usepackage{graphicx} %obsługa plików graficznych z rozszerzeniem png, jpg
\theoremstyle{definition} %styl dla definicji
\newtheorem{zad}{} 
\title{Multizestaw zadań}
\author{Radosław Grzyb}
%\date{\today}
\date{}
\newcounter{liczniksekcji}
\newcommand{\kategoria}[1]{\section{#1}} %olreślamy nazwę kateforii zadań
\newcommand{\zadStart}[1]{\begin{zad}#1\newline} %oznaczenie początku zadania
\newcommand{\zadStop}{\end{zad}}   %oznaczenie końca zadania
%Makra opcjonarne (nie muszą występować):
\newcommand{\rozwStart}[2]{\noindent \textbf{Rozwiązanie (autor #1 , recenzent #2): }\newline} %oznaczenie początku rozwiązania, opcjonarnie można wprowadzić informację o autorze rozwiązania zadania i recenzencie poprawności wykonania rozwiązania zadania
\newcommand{\rozwStop}{\newline}                                            %oznaczenie końca rozwiązania
\newcommand{\odpStart}{\noindent \textbf{Odpowiedź:}\newline}    %oznaczenie początku odpowiedzi końcowej (wypisanie wyniku)
\newcommand{\odpStop}{\newline}                                             %oznaczenie końca odpowiedzi końcowej (wypisanie wyniku)
\newcommand{\testStart}{\noindent \textbf{Test:}\newline} %ewentualne możliwe opcje odpowiedzi testowej: A. ? B. ? C. ? D. ? itd.
\newcommand{\testStop}{\newline} %koniec wprowadzania odpowiedzi testowych
\newcommand{\kluczStart}{\noindent \textbf{Test poprawna odpowiedź:}\newline} %klucz, poprawna odpowiedź pytania testowego (jedna literka): A lub B lub C lub D itd.
\newcommand{\kluczStop}{\newline} %koniec poprawnej odpowiedzi pytania testowego 
\newcommand{\wstawGrafike}[2]{\begin{figure}[h] \includegraphics[scale=#2] {#1} \end{figure}} %gdyby była potrzeba wstawienia obrazka, parametry: nazwa pliku, skala (jak nie wiesz co wpisać, to wpisz 1)

\begin{document}
\maketitle


\kategoria{Wikieł/Z5.21e}
\zadStart{Zadanie z Wikieł Z 5.21 e) moja wersja nr [nrWersji]}
%[a]:[2,3,4,5,6,7,8,9,10,11,12]
%[b]:[2,3,4,5,6,7,8,9,10,11,12]
%[c]=-[a]
%[d]=[c]/[b]
%math.gcd([c],[b])==1 and [d]<-1
Wyznaczyć przedziały monotoniczności funkcji f:
$$f(x)=[a]x+[b]sin(x)$$
\zadStop
\rozwStart{Klaudia Klejdysz}{}
Funkcja $f$ jest ciągła i różniczkowalna w całym zbiorze liczb rzeczywistych, przy czym
$$f'(x)=[a]+[b]cos(x)\text{.}$$
Ponieważ:
$$f'(x)>0 \Leftrightarrow [a]+[b]cos(x)>0\Leftrightarrow [b]cos(x)>[c]\Leftrightarrow cos(x)>\frac{[c]}{[b]}$$
Ponieważ $cos(x)$ przyjmuje wartości równe lub większe niż -1, nierówność ta jest prawdziwa w całym zbiorze liczb rzeczywistych. Jest to jednoznaczne z tym, że funkcja f jest rosnąca w każdym przedziale dziedziny.
\\
\rozwStop
\odpStart
rosnąca: $\mathbb{R}$
\odpStop
\testStart
A. rosnąca: $\mathbb{R}$\\
B. rosnąca: $([a],\infty)$, malejąca: $(-\infty,[b])$\\
C. rosnąca: $([c],\infty)$, malejąca: $(-\infty,[c])$\\
D. rosnąca: $(-\infty,[c])$, malejąca: $([c],\infty)$\\
E. malejąca: $\mathbb{R}$\\
F. rosnąca: $(-\infty,[a])$, malejąca: $([a],\infty)$
\testStop
\kluczStart
A
\kluczStop



\end{document}

