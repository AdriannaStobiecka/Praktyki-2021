\documentclass[12pt, a4paper]{article}
\usepackage[utf8]{inputenc}
\usepackage{polski}

\usepackage{amsthm}  %pakiet do tworzenia twierdzeń itp.
\usepackage{amsmath} %pakiet do niektórych symboli matematycznych
\usepackage{amssymb} %pakiet do symboli mat., np. \nsubseteq
\usepackage{amsfonts}
\usepackage{graphicx} %obsługa plików graficznych z rozszerzeniem png, jpg
\theoremstyle{definition} %styl dla definicji
\newtheorem{zad}{} 
\title{Multizestaw zadań}
\author{Robert Fidytek}
%\date{\today}
\date{}
\newcounter{liczniksekcji}
\newcommand{\kategoria}[1]{\section{#1}} %olreślamy nazwę kateforii zadań
\newcommand{\zadStart}[1]{\begin{zad}#1\newline} %oznaczenie początku zadania
\newcommand{\zadStop}{\end{zad}}   %oznaczenie końca zadania
%Makra opcjonarne (nie muszą występować):
\newcommand{\rozwStart}[2]{\noindent \textbf{Rozwiązanie (autor #1 , recenzent #2): }\newline} %oznaczenie początku rozwiązania, opcjonarnie można wprowadzić informację o autorze rozwiązania zadania i recenzencie poprawności wykonania rozwiązania zadania
\newcommand{\rozwStop}{\newline}                                            %oznaczenie końca rozwiązania
\newcommand{\odpStart}{\noindent \textbf{Odpowiedź:}\newline}    %oznaczenie początku odpowiedzi końcowej (wypisanie wyniku)
\newcommand{\odpStop}{\newline}                                             %oznaczenie końca odpowiedzi końcowej (wypisanie wyniku)
\newcommand{\testStart}{\noindent \textbf{Test:}\newline} %ewentualne możliwe opcje odpowiedzi testowej: A. ? B. ? C. ? D. ? itd.
\newcommand{\testStop}{\newline} %koniec wprowadzania odpowiedzi testowych
\newcommand{\kluczStart}{\noindent \textbf{Test poprawna odpowiedź:}\newline} %klucz, poprawna odpowiedź pytania testowego (jedna literka): A lub B lub C lub D itd.
\newcommand{\kluczStop}{\newline} %koniec poprawnej odpowiedzi pytania testowego 
\newcommand{\wstawGrafike}[2]{\begin{figure}[h] \includegraphics[scale=#2] {#1} \end{figure}} %gdyby była potrzeba wstawienia obrazka, parametry: nazwa pliku, skala (jak nie wiesz co wpisać, to wpisz 1)

\begin{document}
\maketitle


\kategoria{Wikieł/Z4.21o}
\zadStart{Zadanie z Wikieł Z 4.21o) moja wersja nr [nrWersji]}
%[a]:[2,3,4,5,6,7,8,9,10]
%[b]:[2,3,4,5,6,7,8,9,10]
%[c]:[2,3,4,5,6,7,8,9,10]
%[e]=[c]*[a]
%[e1]=int([e]/(math.gcd([e],[b])))
%[b1]=int([b]/(math.gcd([e],[b])))
%[a]!=[b] and [a]!=[c] and [b]!=[c] and [e1]!=1
Obliczyć granicę funkcji $$ \lim_{x \rightarrow 0} \frac{\ln([a]+[b]x)-\ln([a])}{[c]x}$$.
\zadStop
\rozwStart{Barbara Bączek}{}
W celu rozwiązania powyższego przykładu, skorzystamy z reguły de l'Hospitala.
$$ \lim_{x \rightarrow 0} \frac{\ln([a]+[b]x)-\ln([a])}{[c]x}= \lim_{x \rightarrow 0} \frac{\ln \big{(}\frac{[a]+[b]x}{[a]}\big{)}}{[c]x}= \Big{[} \frac{0}{0}\Big{]} \stackrel{[H]}{=} \lim_{x \rightarrow 0} \frac{\frac{[a]}{[a]+[b]x} \cdot \frac{[b]}{[a]}}{[c]}= \frac{[b]}{[a]} \cdot \frac{1}{[c]}= \frac{[b1]}{[e1]} $$
\rozwStop
\odpStart
$\frac{[b1]}{[e1]}$
\odpStop
\testStart
A.$\infty$
B.$\frac{[b1]}{[e1]}$
C.$-\infty$
D.$0$
E.$-\frac{[b1]}{[e1]}$
G.$[a]$
H.$-[a]$
\testStop
\kluczStart
B
\kluczStop



\end{document}