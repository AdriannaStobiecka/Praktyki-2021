\documentclass[12pt, a4paper]{article}
\usepackage[utf8]{inputenc}
\usepackage{polski}
\usepackage{amsthm}  %pakiet do tworzenia twierdzeń itp.
\usepackage{amsmath} %pakiet do niektórych symboli matematycznych
\usepackage{amssymb} %pakiet do symboli mat., np. \nsubseteq
\usepackage{amsfonts}
\usepackage{graphicx} %obsługa plików graficznych z rozszerzeniem png, jpg
\theoremstyle{definition} %styl dla definicji
\newtheorem{zad}{} 
\title{Multizestaw zadań}
\author{Patryk Wirkus}
%\date{\today}
\date{}
\newcommand{\kategoria}[1]{\section{#1}}
\newcommand{\zadStart}[1]{\begin{zad}#1\newline}
\newcommand{\zadStop}{\end{zad}}
\newcommand{\rozwStart}[2]{\noindent \textbf{Rozwiązanie (autor #1 , recenzent #2): }\newline}
\newcommand{\rozwStop}{\newline}                                           
\newcommand{\odpStart}{\noindent \textbf{Odpowiedź:}\newline}
\newcommand{\odpStop}{\newline}
\newcommand{\testStart}{\noindent \textbf{Test:}\newline}
\newcommand{\testStop}{\newline}
\newcommand{\kluczStart}{\noindent \textbf{Test poprawna odpowiedź:}\newline}
\newcommand{\kluczStop}{\newline}
\newcommand{\wstawGrafike}[2]{\begin{figure}[h] \includegraphics[scale=#2] {#1} \end{figure}}

\begin{document}
\maketitle

\kategoria{Wikieł/Z3.33a}


\zadStart{Zadanie z Wikieł Z 3.33 a) moja wersja nr 1}

Obliczyć granicę ciągu $a_{n}=\frac{{n+103\choose103}+{n+103+1\choose103}}{{n+103+1\choose103}+{n+103+2\choose103}}$.
\zadStop
\rozwStart{Patryk Wirkus}{}
$$\lim\limits_{n\to\ \infty}\frac{{n+103\choose103}+{n+103+1\choose103}}{{n+103+1\choose103}+{n+103+2\choose103}} = \lim\limits_{n\to\ \infty}\frac{\frac{(n+103)!}{103! \cdot n!}+\frac{(n+103+1)!}{103! \cdot (n+1)!}}{\frac{(n+103+1)!}{103! \cdot (n+1)!}+\frac{(n+103+2)!}{103! \cdot (n+2)!}} = 1$$
\rozwStop
\odpStart
$1$
\odpStop
\testStart
A.$1$ B.$-1$ C.$0$ D.$\infty$ E.$-\infty$
F.$103$ G.$-103$
H.$104$
I.$105$
\testStop
\kluczStart
A
\kluczStop



\zadStart{Zadanie z Wikieł Z 3.33 a) moja wersja nr 2}

Obliczyć granicę ciągu $a_{n}=\frac{{n+107\choose107}+{n+107+1\choose107}}{{n+107+1\choose107}+{n+107+2\choose107}}$.
\zadStop
\rozwStart{Patryk Wirkus}{}
$$\lim\limits_{n\to\ \infty}\frac{{n+107\choose107}+{n+107+1\choose107}}{{n+107+1\choose107}+{n+107+2\choose107}} = \lim\limits_{n\to\ \infty}\frac{\frac{(n+107)!}{107! \cdot n!}+\frac{(n+107+1)!}{107! \cdot (n+1)!}}{\frac{(n+107+1)!}{107! \cdot (n+1)!}+\frac{(n+107+2)!}{107! \cdot (n+2)!}} = 1$$
\rozwStop
\odpStart
$1$
\odpStop
\testStart
A.$1$ B.$-1$ C.$0$ D.$\infty$ E.$-\infty$
F.$107$ G.$-107$
H.$108$
I.$109$
\testStop
\kluczStart
A
\kluczStop



\zadStart{Zadanie z Wikieł Z 3.33 a) moja wersja nr 3}

Obliczyć granicę ciągu $a_{n}=\frac{{n+109\choose109}+{n+109+1\choose109}}{{n+109+1\choose109}+{n+109+2\choose109}}$.
\zadStop
\rozwStart{Patryk Wirkus}{}
$$\lim\limits_{n\to\ \infty}\frac{{n+109\choose109}+{n+109+1\choose109}}{{n+109+1\choose109}+{n+109+2\choose109}} = \lim\limits_{n\to\ \infty}\frac{\frac{(n+109)!}{109! \cdot n!}+\frac{(n+109+1)!}{109! \cdot (n+1)!}}{\frac{(n+109+1)!}{109! \cdot (n+1)!}+\frac{(n+109+2)!}{109! \cdot (n+2)!}} = 1$$
\rozwStop
\odpStart
$1$
\odpStop
\testStart
A.$1$ B.$-1$ C.$0$ D.$\infty$ E.$-\infty$
F.$109$ G.$-109$
H.$110$
I.$111$
\testStop
\kluczStart
A
\kluczStop



\zadStart{Zadanie z Wikieł Z 3.33 a) moja wersja nr 4}

Obliczyć granicę ciągu $a_{n}=\frac{{n+113\choose113}+{n+113+1\choose113}}{{n+113+1\choose113}+{n+113+2\choose113}}$.
\zadStop
\rozwStart{Patryk Wirkus}{}
$$\lim\limits_{n\to\ \infty}\frac{{n+113\choose113}+{n+113+1\choose113}}{{n+113+1\choose113}+{n+113+2\choose113}} = \lim\limits_{n\to\ \infty}\frac{\frac{(n+113)!}{113! \cdot n!}+\frac{(n+113+1)!}{113! \cdot (n+1)!}}{\frac{(n+113+1)!}{113! \cdot (n+1)!}+\frac{(n+113+2)!}{113! \cdot (n+2)!}} = 1$$
\rozwStop
\odpStart
$1$
\odpStop
\testStart
A.$1$ B.$-1$ C.$0$ D.$\infty$ E.$-\infty$
F.$113$ G.$-113$
H.$114$
I.$115$
\testStop
\kluczStart
A
\kluczStop



\zadStart{Zadanie z Wikieł Z 3.33 a) moja wersja nr 5}

Obliczyć granicę ciągu $a_{n}=\frac{{n+127\choose127}+{n+127+1\choose127}}{{n+127+1\choose127}+{n+127+2\choose127}}$.
\zadStop
\rozwStart{Patryk Wirkus}{}
$$\lim\limits_{n\to\ \infty}\frac{{n+127\choose127}+{n+127+1\choose127}}{{n+127+1\choose127}+{n+127+2\choose127}} = \lim\limits_{n\to\ \infty}\frac{\frac{(n+127)!}{127! \cdot n!}+\frac{(n+127+1)!}{127! \cdot (n+1)!}}{\frac{(n+127+1)!}{127! \cdot (n+1)!}+\frac{(n+127+2)!}{127! \cdot (n+2)!}} = 1$$
\rozwStop
\odpStart
$1$
\odpStop
\testStart
A.$1$ B.$-1$ C.$0$ D.$\infty$ E.$-\infty$
F.$127$ G.$-127$
H.$128$
I.$129$
\testStop
\kluczStart
A
\kluczStop



\zadStart{Zadanie z Wikieł Z 3.33 a) moja wersja nr 6}

Obliczyć granicę ciągu $a_{n}=\frac{{n+131\choose131}+{n+131+1\choose131}}{{n+131+1\choose131}+{n+131+2\choose131}}$.
\zadStop
\rozwStart{Patryk Wirkus}{}
$$\lim\limits_{n\to\ \infty}\frac{{n+131\choose131}+{n+131+1\choose131}}{{n+131+1\choose131}+{n+131+2\choose131}} = \lim\limits_{n\to\ \infty}\frac{\frac{(n+131)!}{131! \cdot n!}+\frac{(n+131+1)!}{131! \cdot (n+1)!}}{\frac{(n+131+1)!}{131! \cdot (n+1)!}+\frac{(n+131+2)!}{131! \cdot (n+2)!}} = 1$$
\rozwStop
\odpStart
$1$
\odpStop
\testStart
A.$1$ B.$-1$ C.$0$ D.$\infty$ E.$-\infty$
F.$131$ G.$-131$
H.$132$
I.$133$
\testStop
\kluczStart
A
\kluczStop



\zadStart{Zadanie z Wikieł Z 3.33 a) moja wersja nr 7}

Obliczyć granicę ciągu $a_{n}=\frac{{n+137\choose137}+{n+137+1\choose137}}{{n+137+1\choose137}+{n+137+2\choose137}}$.
\zadStop
\rozwStart{Patryk Wirkus}{}
$$\lim\limits_{n\to\ \infty}\frac{{n+137\choose137}+{n+137+1\choose137}}{{n+137+1\choose137}+{n+137+2\choose137}} = \lim\limits_{n\to\ \infty}\frac{\frac{(n+137)!}{137! \cdot n!}+\frac{(n+137+1)!}{137! \cdot (n+1)!}}{\frac{(n+137+1)!}{137! \cdot (n+1)!}+\frac{(n+137+2)!}{137! \cdot (n+2)!}} = 1$$
\rozwStop
\odpStart
$1$
\odpStop
\testStart
A.$1$ B.$-1$ C.$0$ D.$\infty$ E.$-\infty$
F.$137$ G.$-137$
H.$138$
I.$139$
\testStop
\kluczStart
A
\kluczStop



\zadStart{Zadanie z Wikieł Z 3.33 a) moja wersja nr 8}

Obliczyć granicę ciągu $a_{n}=\frac{{n+139\choose139}+{n+139+1\choose139}}{{n+139+1\choose139}+{n+139+2\choose139}}$.
\zadStop
\rozwStart{Patryk Wirkus}{}
$$\lim\limits_{n\to\ \infty}\frac{{n+139\choose139}+{n+139+1\choose139}}{{n+139+1\choose139}+{n+139+2\choose139}} = \lim\limits_{n\to\ \infty}\frac{\frac{(n+139)!}{139! \cdot n!}+\frac{(n+139+1)!}{139! \cdot (n+1)!}}{\frac{(n+139+1)!}{139! \cdot (n+1)!}+\frac{(n+139+2)!}{139! \cdot (n+2)!}} = 1$$
\rozwStop
\odpStart
$1$
\odpStop
\testStart
A.$1$ B.$-1$ C.$0$ D.$\infty$ E.$-\infty$
F.$139$ G.$-139$
H.$140$
I.$141$
\testStop
\kluczStart
A
\kluczStop



\zadStart{Zadanie z Wikieł Z 3.33 a) moja wersja nr 9}

Obliczyć granicę ciągu $a_{n}=\frac{{n+149\choose149}+{n+149+1\choose149}}{{n+149+1\choose149}+{n+149+2\choose149}}$.
\zadStop
\rozwStart{Patryk Wirkus}{}
$$\lim\limits_{n\to\ \infty}\frac{{n+149\choose149}+{n+149+1\choose149}}{{n+149+1\choose149}+{n+149+2\choose149}} = \lim\limits_{n\to\ \infty}\frac{\frac{(n+149)!}{149! \cdot n!}+\frac{(n+149+1)!}{149! \cdot (n+1)!}}{\frac{(n+149+1)!}{149! \cdot (n+1)!}+\frac{(n+149+2)!}{149! \cdot (n+2)!}} = 1$$
\rozwStop
\odpStart
$1$
\odpStop
\testStart
A.$1$ B.$-1$ C.$0$ D.$\infty$ E.$-\infty$
F.$149$ G.$-149$
H.$150$
I.$151$
\testStop
\kluczStart
A
\kluczStop



\zadStart{Zadanie z Wikieł Z 3.33 a) moja wersja nr 10}

Obliczyć granicę ciągu $a_{n}=\frac{{n+151\choose151}+{n+151+1\choose151}}{{n+151+1\choose151}+{n+151+2\choose151}}$.
\zadStop
\rozwStart{Patryk Wirkus}{}
$$\lim\limits_{n\to\ \infty}\frac{{n+151\choose151}+{n+151+1\choose151}}{{n+151+1\choose151}+{n+151+2\choose151}} = \lim\limits_{n\to\ \infty}\frac{\frac{(n+151)!}{151! \cdot n!}+\frac{(n+151+1)!}{151! \cdot (n+1)!}}{\frac{(n+151+1)!}{151! \cdot (n+1)!}+\frac{(n+151+2)!}{151! \cdot (n+2)!}} = 1$$
\rozwStop
\odpStart
$1$
\odpStop
\testStart
A.$1$ B.$-1$ C.$0$ D.$\infty$ E.$-\infty$
F.$151$ G.$-151$
H.$152$
I.$153$
\testStop
\kluczStart
A
\kluczStop



\zadStart{Zadanie z Wikieł Z 3.33 a) moja wersja nr 11}

Obliczyć granicę ciągu $a_{n}=\frac{{n+157\choose157}+{n+157+1\choose157}}{{n+157+1\choose157}+{n+157+2\choose157}}$.
\zadStop
\rozwStart{Patryk Wirkus}{}
$$\lim\limits_{n\to\ \infty}\frac{{n+157\choose157}+{n+157+1\choose157}}{{n+157+1\choose157}+{n+157+2\choose157}} = \lim\limits_{n\to\ \infty}\frac{\frac{(n+157)!}{157! \cdot n!}+\frac{(n+157+1)!}{157! \cdot (n+1)!}}{\frac{(n+157+1)!}{157! \cdot (n+1)!}+\frac{(n+157+2)!}{157! \cdot (n+2)!}} = 1$$
\rozwStop
\odpStart
$1$
\odpStop
\testStart
A.$1$ B.$-1$ C.$0$ D.$\infty$ E.$-\infty$
F.$157$ G.$-157$
H.$158$
I.$159$
\testStop
\kluczStart
A
\kluczStop



\zadStart{Zadanie z Wikieł Z 3.33 a) moja wersja nr 12}

Obliczyć granicę ciągu $a_{n}=\frac{{n+163\choose163}+{n+163+1\choose163}}{{n+163+1\choose163}+{n+163+2\choose163}}$.
\zadStop
\rozwStart{Patryk Wirkus}{}
$$\lim\limits_{n\to\ \infty}\frac{{n+163\choose163}+{n+163+1\choose163}}{{n+163+1\choose163}+{n+163+2\choose163}} = \lim\limits_{n\to\ \infty}\frac{\frac{(n+163)!}{163! \cdot n!}+\frac{(n+163+1)!}{163! \cdot (n+1)!}}{\frac{(n+163+1)!}{163! \cdot (n+1)!}+\frac{(n+163+2)!}{163! \cdot (n+2)!}} = 1$$
\rozwStop
\odpStart
$1$
\odpStop
\testStart
A.$1$ B.$-1$ C.$0$ D.$\infty$ E.$-\infty$
F.$163$ G.$-163$
H.$164$
I.$165$
\testStop
\kluczStart
A
\kluczStop



\zadStart{Zadanie z Wikieł Z 3.33 a) moja wersja nr 13}

Obliczyć granicę ciągu $a_{n}=\frac{{n+167\choose167}+{n+167+1\choose167}}{{n+167+1\choose167}+{n+167+2\choose167}}$.
\zadStop
\rozwStart{Patryk Wirkus}{}
$$\lim\limits_{n\to\ \infty}\frac{{n+167\choose167}+{n+167+1\choose167}}{{n+167+1\choose167}+{n+167+2\choose167}} = \lim\limits_{n\to\ \infty}\frac{\frac{(n+167)!}{167! \cdot n!}+\frac{(n+167+1)!}{167! \cdot (n+1)!}}{\frac{(n+167+1)!}{167! \cdot (n+1)!}+\frac{(n+167+2)!}{167! \cdot (n+2)!}} = 1$$
\rozwStop
\odpStart
$1$
\odpStop
\testStart
A.$1$ B.$-1$ C.$0$ D.$\infty$ E.$-\infty$
F.$167$ G.$-167$
H.$168$
I.$169$
\testStop
\kluczStart
A
\kluczStop



\zadStart{Zadanie z Wikieł Z 3.33 a) moja wersja nr 14}

Obliczyć granicę ciągu $a_{n}=\frac{{n+173\choose173}+{n+173+1\choose173}}{{n+173+1\choose173}+{n+173+2\choose173}}$.
\zadStop
\rozwStart{Patryk Wirkus}{}
$$\lim\limits_{n\to\ \infty}\frac{{n+173\choose173}+{n+173+1\choose173}}{{n+173+1\choose173}+{n+173+2\choose173}} = \lim\limits_{n\to\ \infty}\frac{\frac{(n+173)!}{173! \cdot n!}+\frac{(n+173+1)!}{173! \cdot (n+1)!}}{\frac{(n+173+1)!}{173! \cdot (n+1)!}+\frac{(n+173+2)!}{173! \cdot (n+2)!}} = 1$$
\rozwStop
\odpStart
$1$
\odpStop
\testStart
A.$1$ B.$-1$ C.$0$ D.$\infty$ E.$-\infty$
F.$173$ G.$-173$
H.$174$
I.$175$
\testStop
\kluczStart
A
\kluczStop



\zadStart{Zadanie z Wikieł Z 3.33 a) moja wersja nr 15}

Obliczyć granicę ciągu $a_{n}=\frac{{n+179\choose179}+{n+179+1\choose179}}{{n+179+1\choose179}+{n+179+2\choose179}}$.
\zadStop
\rozwStart{Patryk Wirkus}{}
$$\lim\limits_{n\to\ \infty}\frac{{n+179\choose179}+{n+179+1\choose179}}{{n+179+1\choose179}+{n+179+2\choose179}} = \lim\limits_{n\to\ \infty}\frac{\frac{(n+179)!}{179! \cdot n!}+\frac{(n+179+1)!}{179! \cdot (n+1)!}}{\frac{(n+179+1)!}{179! \cdot (n+1)!}+\frac{(n+179+2)!}{179! \cdot (n+2)!}} = 1$$
\rozwStop
\odpStart
$1$
\odpStop
\testStart
A.$1$ B.$-1$ C.$0$ D.$\infty$ E.$-\infty$
F.$179$ G.$-179$
H.$180$
I.$181$
\testStop
\kluczStart
A
\kluczStop



\zadStart{Zadanie z Wikieł Z 3.33 a) moja wersja nr 16}

Obliczyć granicę ciągu $a_{n}=\frac{{n+251\choose251}+{n+251+1\choose251}}{{n+251+1\choose251}+{n+251+2\choose251}}$.
\zadStop
\rozwStart{Patryk Wirkus}{}
$$\lim\limits_{n\to\ \infty}\frac{{n+251\choose251}+{n+251+1\choose251}}{{n+251+1\choose251}+{n+251+2\choose251}} = \lim\limits_{n\to\ \infty}\frac{\frac{(n+251)!}{251! \cdot n!}+\frac{(n+251+1)!}{251! \cdot (n+1)!}}{\frac{(n+251+1)!}{251! \cdot (n+1)!}+\frac{(n+251+2)!}{251! \cdot (n+2)!}} = 1$$
\rozwStop
\odpStart
$1$
\odpStop
\testStart
A.$1$ B.$-1$ C.$0$ D.$\infty$ E.$-\infty$
F.$251$ G.$-251$
H.$252$
I.$253$
\testStop
\kluczStart
A
\kluczStop



\zadStart{Zadanie z Wikieł Z 3.33 a) moja wersja nr 17}

Obliczyć granicę ciągu $a_{n}=\frac{{n+257\choose257}+{n+257+1\choose257}}{{n+257+1\choose257}+{n+257+2\choose257}}$.
\zadStop
\rozwStart{Patryk Wirkus}{}
$$\lim\limits_{n\to\ \infty}\frac{{n+257\choose257}+{n+257+1\choose257}}{{n+257+1\choose257}+{n+257+2\choose257}} = \lim\limits_{n\to\ \infty}\frac{\frac{(n+257)!}{257! \cdot n!}+\frac{(n+257+1)!}{257! \cdot (n+1)!}}{\frac{(n+257+1)!}{257! \cdot (n+1)!}+\frac{(n+257+2)!}{257! \cdot (n+2)!}} = 1$$
\rozwStop
\odpStart
$1$
\odpStop
\testStart
A.$1$ B.$-1$ C.$0$ D.$\infty$ E.$-\infty$
F.$257$ G.$-257$
H.$258$
I.$259$
\testStop
\kluczStart
A
\kluczStop



\zadStart{Zadanie z Wikieł Z 3.33 a) moja wersja nr 18}

Obliczyć granicę ciągu $a_{n}=\frac{{n+263\choose263}+{n+263+1\choose263}}{{n+263+1\choose263}+{n+263+2\choose263}}$.
\zadStop
\rozwStart{Patryk Wirkus}{}
$$\lim\limits_{n\to\ \infty}\frac{{n+263\choose263}+{n+263+1\choose263}}{{n+263+1\choose263}+{n+263+2\choose263}} = \lim\limits_{n\to\ \infty}\frac{\frac{(n+263)!}{263! \cdot n!}+\frac{(n+263+1)!}{263! \cdot (n+1)!}}{\frac{(n+263+1)!}{263! \cdot (n+1)!}+\frac{(n+263+2)!}{263! \cdot (n+2)!}} = 1$$
\rozwStop
\odpStart
$1$
\odpStop
\testStart
A.$1$ B.$-1$ C.$0$ D.$\infty$ E.$-\infty$
F.$263$ G.$-263$
H.$264$
I.$265$
\testStop
\kluczStart
A
\kluczStop



\zadStart{Zadanie z Wikieł Z 3.33 a) moja wersja nr 19}

Obliczyć granicę ciągu $a_{n}=\frac{{n+269\choose269}+{n+269+1\choose269}}{{n+269+1\choose269}+{n+269+2\choose269}}$.
\zadStop
\rozwStart{Patryk Wirkus}{}
$$\lim\limits_{n\to\ \infty}\frac{{n+269\choose269}+{n+269+1\choose269}}{{n+269+1\choose269}+{n+269+2\choose269}} = \lim\limits_{n\to\ \infty}\frac{\frac{(n+269)!}{269! \cdot n!}+\frac{(n+269+1)!}{269! \cdot (n+1)!}}{\frac{(n+269+1)!}{269! \cdot (n+1)!}+\frac{(n+269+2)!}{269! \cdot (n+2)!}} = 1$$
\rozwStop
\odpStart
$1$
\odpStop
\testStart
A.$1$ B.$-1$ C.$0$ D.$\infty$ E.$-\infty$
F.$269$ G.$-269$
H.$270$
I.$271$
\testStop
\kluczStart
A
\kluczStop



\zadStart{Zadanie z Wikieł Z 3.33 a) moja wersja nr 20}

Obliczyć granicę ciągu $a_{n}=\frac{{n+271\choose271}+{n+271+1\choose271}}{{n+271+1\choose271}+{n+271+2\choose271}}$.
\zadStop
\rozwStart{Patryk Wirkus}{}
$$\lim\limits_{n\to\ \infty}\frac{{n+271\choose271}+{n+271+1\choose271}}{{n+271+1\choose271}+{n+271+2\choose271}} = \lim\limits_{n\to\ \infty}\frac{\frac{(n+271)!}{271! \cdot n!}+\frac{(n+271+1)!}{271! \cdot (n+1)!}}{\frac{(n+271+1)!}{271! \cdot (n+1)!}+\frac{(n+271+2)!}{271! \cdot (n+2)!}} = 1$$
\rozwStop
\odpStart
$1$
\odpStop
\testStart
A.$1$ B.$-1$ C.$0$ D.$\infty$ E.$-\infty$
F.$271$ G.$-271$
H.$272$
I.$273$
\testStop
\kluczStart
A
\kluczStop



\zadStart{Zadanie z Wikieł Z 3.33 a) moja wersja nr 21}

Obliczyć granicę ciągu $a_{n}=\frac{{n+277\choose277}+{n+277+1\choose277}}{{n+277+1\choose277}+{n+277+2\choose277}}$.
\zadStop
\rozwStart{Patryk Wirkus}{}
$$\lim\limits_{n\to\ \infty}\frac{{n+277\choose277}+{n+277+1\choose277}}{{n+277+1\choose277}+{n+277+2\choose277}} = \lim\limits_{n\to\ \infty}\frac{\frac{(n+277)!}{277! \cdot n!}+\frac{(n+277+1)!}{277! \cdot (n+1)!}}{\frac{(n+277+1)!}{277! \cdot (n+1)!}+\frac{(n+277+2)!}{277! \cdot (n+2)!}} = 1$$
\rozwStop
\odpStart
$1$
\odpStop
\testStart
A.$1$ B.$-1$ C.$0$ D.$\infty$ E.$-\infty$
F.$277$ G.$-277$
H.$278$
I.$279$
\testStop
\kluczStart
A
\kluczStop





\end{document}
