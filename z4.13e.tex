\documentclass[12pt, a4paper]{article}
\usepackage[utf8]{inputenc}
\usepackage{polski}

\usepackage{amsthm}  %pakiet do tworzenia twierdzeń itp.
\usepackage{amsmath} %pakiet do niektórych symboli matematycznych
\usepackage{amssymb} %pakiet do symboli mat., np. \nsubseteq
\usepackage{amsfonts}
\usepackage{graphicx} %obsługa plików graficznych z rozszerzeniem png, jpg
\theoremstyle{definition} %styl dla definicji
\newtheorem{zad}{} 
\title{Multizestaw zadań}
\author{Robert Fidytek}
%\date{\today}
\date{}
\newcounter{liczniksekcji}
\newcommand{\kategoria}[1]{\section{#1}} %olreślamy nazwę kateforii zadań
\newcommand{\zadStart}[1]{\begin{zad}#1\newline} %oznaczenie początku zadania
\newcommand{\zadStop}{\end{zad}}   %oznaczenie końca zadania
%Makra opcjonarne (nie muszą występować):
\newcommand{\rozwStart}[2]{\noindent \textbf{Rozwiązanie (autor #1 , recenzent #2): }\newline} %oznaczenie początku rozwiązania, opcjonarnie można wprowadzić informację o autorze rozwiązania zadania i recenzencie poprawności wykonania rozwiązania zadania
\newcommand{\rozwStop}{\newline}                                            %oznaczenie końca rozwiązania
\newcommand{\odpStart}{\noindent \textbf{Odpowiedź:}\newline}    %oznaczenie początku odpowiedzi końcowej (wypisanie wyniku)
\newcommand{\odpStop}{\newline}                                             %oznaczenie końca odpowiedzi końcowej (wypisanie wyniku)
\newcommand{\testStart}{\noindent \textbf{Test:}\newline} %ewentualne możliwe opcje odpowiedzi testowej: A. ? B. ? C. ? D. ? itd.
\newcommand{\testStop}{\newline} %koniec wprowadzania odpowiedzi testowych
\newcommand{\kluczStart}{\noindent \textbf{Test poprawna odpowiedź:}\newline} %klucz, poprawna odpowiedź pytania testowego (jedna literka): A lub B lub C lub D itd.
\newcommand{\kluczStop}{\newline} %koniec poprawnej odpowiedzi pytania testowego 
\newcommand{\wstawGrafike}[2]{\begin{figure}[h] \includegraphics[scale=#2] {#1} \end{figure}} %gdyby była potrzeba wstawienia obrazka, parametry: nazwa pliku, skala (jak nie wiesz co wpisać, to wpisz 1)

\begin{document}
\maketitle


\kategoria{Wikieł/Z4.13e}
\zadStart{Zadanie z Wikieł Z 4.13 e) moja wersja nr [nrWersji]}
%[a]:[1,2,3,4,5]
%[b]:[1,2,3,4,5]
%[c]:[2,3,4,5,6]
%[x2]:[1,2,3,4]
%[x1]=-[x2]
%[x2x2]=[x2]*[x2]
%[b]=[c]*[x2x2]
%[x]=[b]-[c]*[x2x2]
Obliczyć granice jednostronne funkcji w podanych punktach. $$f(x)=e^\frac{[a]}{[b]-[c]x^2},x_{1}=[x1],x_{2}=[x2]$$.
\zadStop
\rozwStart{Aleksandra Pasińska}{}
1)$$f(x)=e^\frac{[a]}{[b]-[c]x^2},x_{1}=[x1]$$
$$\lim_{x\rightarrow [x1]^-}e^\frac{[a]}{[b]-[c]x^2}=e^{\biggl[\frac{[a]}{[x]^-}\biggr]}=e^{-\infty}=0$$ 
$$\lim_{x\rightarrow [x1]^+}e^\frac{[a]}{[b]-[c]x^2}=e^{\biggl[\frac{[a]}{[x]^+}\biggr]}=e^{\infty}=\infty$$ 
2)$$f(x)=e^\frac{[a]}{[b]-[c]x^2},x_{2}=[x2]$$
$$\lim_{x\rightarrow [x2]^-}e^\frac{[a]}{[b]-[c]x^2}=e^{\biggl[\frac{[a]}{[x]^+}\biggr]}=e^{\infty}=\infty$$ 
$$\lim_{x\rightarrow [x2]^+}e^\frac{[a]}{[b]-[c]x^2}=e^{\biggl[\frac{[a]}{[x]^-}\biggr]}=e^{-\infty}=0$$ 
\rozwStop
\odpStart
$L=\lim_{x\rightarrow x^-}f(x),P=\lim_{x\rightarrow x^+}f(x)$\\
$x_{1}:L=0, P=\infty$
$x_{2}:L=\infty, P=0$
\odpStop
\testStart
A.$x_{1}:L=0, P=\infty, x_{2}:L=\infty, P=0$
B.$x_{1}:L=-\infty, P=\infty, x_{2}:L=\infty, P=0$
C.$x_{1}:L=0, P=\infty, x_{2}:L=\infty, P=-\infty$
D.$x_{1}:L=0, P=0, x_{2}:L=\infty, P=0$
E.$x_{1}:L=\infty, P=\infty, x_{2}:L=\infty, P=0$
F.$x_{1}:L=0, P=\infty, x_{2}:L=\infty, P=\infty$
G.$x_{1}:L=0, P=\infty, x_{2}:L=0, P=0$
H.$x_{1}:L=1, P=\infty, x_{2}:L=\infty, P=0$
I.$x_{1}:L=0, P=\infty, x_{2}:L=\infty, P=1$
\testStop
\kluczStart
A
\kluczStop



\end{document}