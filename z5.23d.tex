\documentclass[12pt, a4paper]{article}
\usepackage[utf8]{inputenc}
\usepackage{polski}

\usepackage{amsthm}  %pakiet do tworzenia twierdzeń itp.
\usepackage{amsmath} %pakiet do niektórych symboli matematycznych
\usepackage{amssymb} %pakiet do symboli mat., np. \nsubseteq
\usepackage{amsfonts}
\usepackage{graphicx} %obsługa plików graficznych z rozszerzeniem png, jpg
\theoremstyle{definition} %styl dla definicji
\newtheorem{zad}{} 
\title{Multizestaw zadań}
\author{Laura Mieczkowska}
%\date{\today}
\date{}
\newcounter{liczniksekcji}
\newcommand{\kategoria}[1]{\section{#1}} %olreślamy nazwę kateforii zadań
\newcommand{\zadStart}[1]{\begin{zad}#1\newline} %oznaczenie początku zadania
\newcommand{\zadStop}{\end{zad}}   %oznaczenie końca zadania
%Makra opcjonarne (nie muszą występować):
\newcommand{\rozwStart}[2]{\noindent \textbf{Rozwiązanie (autor #1 , recenzent #2): }\newline} %oznaczenie początku rozwiązania, opcjonarnie można wprowadzić informację o autorze rozwiązania zadania i recenzencie poprawności wykonania rozwiązania zadania
\newcommand{\rozwStop}{\newline}                                            %oznaczenie końca rozwiązania
\newcommand{\odpStart}{\noindent \textbf{Odpowiedź:}\newline}    %oznaczenie początku odpowiedzi końcowej (wypisanie wyniku)
\newcommand{\odpStop}{\newline}                                             %oznaczenie końca odpowiedzi końcowej (wypisanie wyniku)
\newcommand{\testStart}{\noindent \textbf{Test:}\newline} %ewentualne możliwe opcje odpowiedzi testowej: A. ? B. ? C. ? D. ? itd.
\newcommand{\testStop}{\newline} %koniec wprowadzania odpowiedzi testowych
\newcommand{\kluczStart}{\noindent \textbf{Test poprawna odpowiedź:}\newline} %klucz, poprawna odpowiedź pytania testowego (jedna literka): A lub B lub C lub D itd.
\newcommand{\kluczStop}{\newline} %koniec poprawnej odpowiedzi pytania testowego 
\newcommand{\wstawGrafike}[2]{\begin{figure}[h] \includegraphics[scale=#2] {#1} \end{figure}} %gdyby była potrzeba wstawienia obrazka, parametry: nazwa pliku, skala (jak nie wiesz co wpisać, to wpisz 1)

\begin{document}
\maketitle


\kategoria{Wikieł/Z5.23d}
\zadStart{Zadanie z Wikieł Z 5.23 d) moja wersja nr [nrWersji]}
%[a]:[2,3,4,5,6,7]
%[b]=2*[a]
%[bkw]=[b]**2
%[ab]=abs([a]-[b])
%[nwd]=math.gcd([a],[bkw])
%[licz1]=[a]/[nwd]
%[licz]=int([licz1])
%[mian1]=[bkw]/[nwd]
%[mian]=int([mian1])
%[aa]=[a]-1
%[nmian]=2*[bkw]
%[pierw]=round(math.sqrt([nmian]),2)
%[u1]=round((2/[pierw]),2)
%[u2]=round((1/[u1]),2)
%[w1]=round([a]-[u2],2)
%[l]=2*[a]-1
%[licz1].is_integer()==True and [mian1].is_integer()==True
Znaleźć ekstrema lokalne funkcji $y=[a]x-\sqrt{x}$.
\zadStop
\rozwStart{Laura Mieczkowska}{}
$$y=[a]x-\sqrt{x}$$
$$y'=[a]-\frac{1}{2\sqrt{x}}=\frac{[b]\sqrt{x}-1}{2\sqrt{x}}$$
$$\frac{[b]\sqrt{x}-1}{2\sqrt{x}}=0 \Rightarrow [b]\sqrt{x}-1=0 \Rightarrow \sqrt{x}=\frac{1}{[b]} \Rightarrow x=\frac{1}{[bkw]}$$
Otrzymujemy punkt, w którym może znajdować się ekstremum. Ten punkt (wraz z dziedziną funkcji) wyznacza dwa przedziały, w których należy zbadać znak funkcji:
\\\\1. $\bigg[0;\frac{1}{[bkw]}\bigg)$
$$y'\bigg(\frac{1}{[nmian]}\bigg)=[a]-\frac{1}{2\sqrt{\frac{1}{[nmian]}}}=[a]-\frac{1}{[u1]}=[a]-[u2]=[w1]$$
[w1] ma ujemny znak, więc funkcja na tym przedziale jest malejąca.
\\\\2. $\bigg(\frac{1}{[bkw]};\infty\bigg)$
$$y'(1)=[a]-\frac{1}{2\sqrt{1}}=[a]-\frac{1}{2}=\frac{[l]}{2}$$
$\frac{[l]}{2}$ ma dodatni znak, więc funkcja na tym przedziale jest rosnąca.
\\Podsumowując, funkcja na przedziale $\bigg[0;\frac{1}{[bkw]}\bigg)$ maleje, a następnie rośnie na przedziale $\bigg(\frac{1}{[bkw]};\infty\bigg)$, wobec tego w punkcie $x=\frac{1}{[bkw]}$ istnieje minimum lokalne.
$$y\bigg(\frac{1}{[bkw]}\bigg)=[a]\cdot\frac{1}{[bkw]}-\sqrt{\frac{1}{[bkw]}}=\frac{[a]}{[bkw]}-\frac{1}{[b]}=-\frac{[ab]}{[bkw]}=-\frac{[licz]}{[mian]}$$
\odpStart
$y_{min}=y\bigg(\frac{1}{[bkw]}\bigg)=-\frac{[licz]}{[mian]}$
\odpStop
\testStart
A. $y_{max}=y\big(\frac{1}{[bkw]}\big)=\frac{[licz]}{[mian]}$\\
B. $y_{min}=y\big(\frac{1}{[bkw]}\big)=-\frac{[licz]}{[mian]}$ \\
C. $y_{min}=y\big(-\frac{1}{[bkw]}\big)=-\frac{[licz]}{[mian]}$, $y_{max}=y\big(\frac{1}{[bkw]}\big)=\frac{[licz]}{[mian]}$ \\
D. $y_{min}=y\big(-\frac{1}{[bkw]}\big)=-\frac{[licz]}{[mian]}$ 
\testStop
\kluczStart
B
\kluczStop



\end{document}