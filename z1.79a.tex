\documentclass[12pt, a4paper]{article}
\usepackage[utf8]{inputenc}
\usepackage{polski}

\usepackage{amsthm}  %pakiet do tworzenia twierdzeń itp.
\usepackage{amsmath} %pakiet do niektórych symboli matematycznych
\usepackage{amssymb} %pakiet do symboli mat., np. \nsubseteq
\usepackage{amsfonts}
\usepackage{graphicx} %obsługa plików graficznych z rozszerzeniem png, jpg
\theoremstyle{definition} %styl dla definicji
\newtheorem{zad}{} 
\title{Multizestaw zadań}
\author{Robert Fidytek}
%\date{\today}
\date{}
\newcounter{liczniksekcji}
\newcommand{\kategoria}[1]{\section{#1}} %olreślamy nazwę kateforii zadań
\newcommand{\zadStart}[1]{\begin{zad}#1\newline} %oznaczenie początku zadania
\newcommand{\zadStop}{\end{zad}}   %oznaczenie końca zadania
%Makra opcjonarne (nie muszą występować):
\newcommand{\rozwStart}[2]{\noindent \textbf{Rozwiązanie (autor #1 , recenzent #2): }\newline} %oznaczenie początku rozwiązania, opcjonarnie można wprowadzić informację o autorze rozwiązania zadania i recenzencie poprawności wykonania rozwiązania zadania
\newcommand{\rozwStop}{\newline}                                            %oznaczenie końca rozwiązania
\newcommand{\odpStart}{\noindent \textbf{Odpowiedź:}\newline}    %oznaczenie początku odpowiedzi końcowej (wypisanie wyniku)
\newcommand{\odpStop}{\newline}                                             %oznaczenie końca odpowiedzi końcowej (wypisanie wyniku)
\newcommand{\testStart}{\noindent \textbf{Test:}\newline} %ewentualne możliwe opcje odpowiedzi testowej: A. ? B. ? C. ? D. ? itd.
\newcommand{\testStop}{\newline} %koniec wprowadzania odpowiedzi testowych
\newcommand{\kluczStart}{\noindent \textbf{Test poprawna odpowiedź:}\newline} %klucz, poprawna odpowiedź pytania testowego (jedna literka): A lub B lub C lub D itd.
\newcommand{\kluczStop}{\newline} %koniec poprawnej odpowiedzi pytania testowego 
\newcommand{\wstawGrafike}[2]{\begin{figure}[h] \includegraphics[scale=#2] {#1} \end{figure}} %gdyby była potrzeba wstawienia obrazka, parametry: nazwa pliku, skala (jak nie wiesz co wpisać, to wpisz 1)

\begin{document}
\maketitle


\kategoria{Wikieł/Z1.79a}
\zadStart{Zadanie z Wikieł Z 1.79 a) moja wersja nr [nrWersji]}
%[a]:[2,3,4,5,6,7,8,9]
%[b]:[2,3,4,5,6,7,8,9]
%[b3]=[b]*[b]*[b]
%[ab]=[a]+[b3]
%
Rozwiązać nierówność
$$\sqrt[3]{[a]-x}<-[b].$$
\zadStop
\rozwStart{Adrianna Stobiecka}{}
Pierwiastek nieparzystego stopnia istnieje dla każdej liczby rzeczywistej, stąd założenie $x\in\mathbb{R}$.
$$\sqrt[3]{[a]-x}<-[b]\Leftrightarrow[a]-x<(-[b])^3\Leftrightarrow[a]-x<-[b3]$$
$$\Leftrightarrow[a]+[b3]<x\Leftrightarrow x>[ab]$$
Zatem otrzymujemy ostatecznie, że 
$$x\in([ab],\infty).$$
\rozwStop
\odpStart
$x\in([ab],\infty)$
\odpStop
\testStart
A.$x\in(-[ab],\infty)$
B.$x\in([ab],\infty)$
C.$x\in\mathbb{R}$
D.$x\in[[ab],\infty)$
E.$x\in(-\infty,[ab]]$
F.$x\in\mathbb{R}\setminus\{0\}$
G.$x\in(-\infty,[ab])$
H.$x\in\emptyset$
I.$x\in[-[ab],\infty)$
\testStop
\kluczStart
B
\kluczStop



\end{document}
