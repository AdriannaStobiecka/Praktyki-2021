\documentclass[12pt, a4paper]{article}
\usepackage[utf8]{inputenc}
\usepackage{polski}

\usepackage{amsthm}  %pakiet do tworzenia twierdzeń itp.
\usepackage{amsmath} %pakiet do niektórych symboli matematycznych
\usepackage{amssymb} %pakiet do symboli mat., np. \nsubseteq
\usepackage{amsfonts}
\usepackage{graphicx} %obsługa plików graficznych z rozszerzeniem png, jpg
\theoremstyle{definition} %styl dla definicji
\newtheorem{zad}{} 
\title{Multizestaw zadań}
\author{Robert Fidytek}
%\date{\today}
\date{}
\newcounter{liczniksekcji}
\newcommand{\kategoria}[1]{\section{#1}} %olreślamy nazwę kateforii zadań
\newcommand{\zadStart}[1]{\begin{zad}#1\newline} %oznaczenie początku zadania
\newcommand{\zadStop}{\end{zad}}   %oznaczenie końca zadania
%Makra opcjonarne (nie muszą występować):
\newcommand{\rozwStart}[2]{\noindent \textbf{Rozwiązanie (autor #1 , recenzent #2): }\newline} %oznaczenie początku rozwiązania, opcjonarnie można wprowadzić informację o autorze rozwiązania zadania i recenzencie poprawności wykonania rozwiązania zadania
\newcommand{\rozwStop}{\newline}                                            %oznaczenie końca rozwiązania
\newcommand{\odpStart}{\noindent \textbf{Odpowiedź:}\newline}    %oznaczenie początku odpowiedzi końcowej (wypisanie wyniku)
\newcommand{\odpStop}{\newline}                                             %oznaczenie końca odpowiedzi końcowej (wypisanie wyniku)
\newcommand{\testStart}{\noindent \textbf{Test:}\newline} %ewentualne możliwe opcje odpowiedzi testowej: A. ? B. ? C. ? D. ? itd.
\newcommand{\testStop}{\newline} %koniec wprowadzania odpowiedzi testowych
\newcommand{\kluczStart}{\noindent \textbf{Test poprawna odpowiedź:}\newline} %klucz, poprawna odpowiedź pytania testowego (jedna literka): A lub B lub C lub D itd.
\newcommand{\kluczStop}{\newline} %koniec poprawnej odpowiedzi pytania testowego 
\newcommand{\wstawGrafike}[2]{\begin{figure}[h] \includegraphics[scale=#2] {#1} \end{figure}} %gdyby była potrzeba wstawienia obrazka, parametry: nazwa pliku, skala (jak nie wiesz co wpisać, to wpisz 1)

\begin{document}
\maketitle


\kategoria{Wikieł/P1.25b}
\zadStart{Zadanie z Wikieł P 1.25 b)  moja wersja nr [nrWersji]}
%[p1]:[2,3,4,5,6,7,8,9,10,11,12]
%[p2]=random.randint(1,10)
%[p3]=random.randint(1,10)
%[p4]=random.randint(1,10)
%[p5]=random.randint(1,10)
%[p6]=random.randint(1,10)
%[p7]=random.randint(1,10)
%[a]=[p1]-[p2]+[p3]
%[b]=[p4]+[p5]-[p6]+[p7]
%[a3]=pow([a],3)
%[b2]=pow([b],2)
%[ab]=[a3]+[b2]

Dany jest wielomian $W(x)=([p1]x^{5}-[p2]x^{2}+[p3])^{3}\cdot([p4]x^{7}+[p5]x^{4}-[p6]x+[p7])^{2}.$ Obliczyć sumę współczynników wielomianu $W(x)$.
\zadStop
\rozwStart{Maja Szabłowska}{}
Zauważamy, że suma współczynników wielomianu to $W(1).$ Zatem:
$$W(1)=([p1]-[p2]+[p3])^{3}\cdot([p4]+[p5]-[p6]+[p7])^{2}=$$
$$=([a])^{3}\cdot([b])^{2}=[a3]+[b2]=[ab] $$
\rozwStop
\odpStart
$[ab]$
\odpStop
\testStart
A.$[ab]$
B.$[a]$
C.$[b]$
D.$[a3]$
E.$[b2]$
F.$0$
G.$[p1]$
H.$[p7]$

\testStop
\kluczStart
A
\kluczStop



\end{document}