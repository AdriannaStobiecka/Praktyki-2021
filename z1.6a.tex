\documentclass[12pt, a4paper]{article}
\usepackage[utf8]{inputenc}
\usepackage{polski}

\usepackage{amsthm}  %pakiet do tworzenia twierdzeń itp.
\usepackage{amsmath} %pakiet do niektórych symboli matematycznych
\usepackage{amssymb} %pakiet do symboli mat., np. \nsubseteq
\usepackage{amsfonts}
\usepackage{graphicx} %obsługa plików graficznych z rozszerzeniem png, jpg
\theoremstyle{definition} %styl dla definicji
\newtheorem{zad}{} 
\title{Multizestaw zadań}
\author{Robert Fidytek}
%\date{\today}
\date{}
\newcounter{liczniksekcji}
\newcommand{\kategoria}[1]{\section{#1}} %olreślamy nazwę kateforii zadań
\newcommand{\zadStart}[1]{\begin{zad}#1\newline} %oznaczenie początku zadania
\newcommand{\zadStop}{\end{zad}}   %oznaczenie końca zadania
%Makra opcjonarne (nie muszą występować):
\newcommand{\rozwStart}[2]{\noindent \textbf{Rozwiązanie (autor #1 , recenzent #2): }\newline} %oznaczenie początku rozwiązania, opcjonarnie można wprowadzić informację o autorze rozwiązania zadania i recenzencie poprawności wykonania rozwiązania zadania
\newcommand{\rozwStop}{\newline}                                            %oznaczenie końca rozwiązania
\newcommand{\odpStart}{\noindent \textbf{Odpowiedź:}\newline}    %oznaczenie początku odpowiedzi końcowej (wypisanie wyniku)
\newcommand{\odpStop}{\newline}                                             %oznaczenie końca odpowiedzi końcowej (wypisanie wyniku)
\newcommand{\testStart}{\noindent \textbf{Test:}\newline} %ewentualne możliwe opcje odpowiedzi testowej: A. ? B. ? C. ? D. ? itd.
\newcommand{\testStop}{\newline} %koniec wprowadzania odpowiedzi testowych
\newcommand{\kluczStart}{\noindent \textbf{Test poprawna odpowiedź:}\newline} %klucz, poprawna odpowiedź pytania testowego (jedna literka): A lub B lub C lub D itd.
\newcommand{\kluczStop}{\newline} %koniec poprawnej odpowiedzi pytania testowego 
\newcommand{\wstawGrafike}[2]{\begin{figure}[h] \includegraphics[scale=#2] {#1} \end{figure}} %gdyby była potrzeba wstawienia obrazka, parametry: nazwa pliku, skala (jak nie wiesz co wpisać, to wpisz 1)

\begin{document}
\maketitle


\kategoria{Wikieł/Z1.6a}
\zadStart{Zadanie z Wikieł Z 1.6 a) moja wersja nr [nrWersji]}
%[a]:[3,5,6,7,10,11,13,15,17]
%[c]:[3,5,6,7,10,11,13,15,17]
%[b]=[a]-1
%[d]=[c]-1
%[aa]=pow([a],2)
%[b]>[c]
Sprawdzić, która z liczb jest większa: $A=\sqrt{[a]} -\sqrt{[b]} $ czy $B=\sqrt{[c]}-\sqrt{[d]} $.
\zadStop
\rozwStart{Małgorzata Ugowska}{}
Przekształcamy liczby A i B korzystając ze wzorów skróconego mnożenia:
$$A = \sqrt{[a]} -\sqrt{[b]} = \frac{(\sqrt{[a]} -\sqrt{[b]})(\sqrt{[a]} +\sqrt{[b]})}{\sqrt{[a]} +\sqrt{[b]}} = \frac{[a]-[b]}{\sqrt{[a]} +\sqrt{[b]}} = \frac{1}{\sqrt{[a]} +\sqrt{[b]}}$$
$$B = \sqrt{[c]} -\sqrt{[d]} = \frac{(\sqrt{[c]} -\sqrt{[d]})(\sqrt{[c]} +\sqrt{[d]})}{\sqrt{[c]} +\sqrt{[d]}} = \frac{[c]-[d]}{\sqrt{[c]} +\sqrt{[d]}} = \frac{1}{\sqrt{[c]} +\sqrt{[d]}}$$
$$\sqrt{[a]} +\sqrt{[b]}>\sqrt{[c]} +\sqrt{[d]}$$
$$\frac{1}{\sqrt{[a]} +\sqrt{[b]}}<\frac{1}{\sqrt{[c]} +\sqrt{[d]}}$$
$$A<B$$
\rozwStop
\odpStart
B
\odpStop
\testStart
A. liczba A jest większa\\
B. liczba B jest większa\\
C. liczby A i B są sobie równe
\testStop
\kluczStart
B
\kluczStop



\end{document}