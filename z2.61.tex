\documentclass[12pt, a4paper]{article}
\usepackage[utf8]{inputenc}
\usepackage{polski}

\usepackage{amsthm}  %pakiet do tworzenia twierdzeń itp.
\usepackage{amsmath} %pakiet do niektórych symboli matematycznych
\usepackage{amssymb} %pakiet do symboli mat., np. \nsubseteq
\usepackage{amsfonts}
\usepackage{graphicx} %obsługa plików graficznych z rozszerzeniem png, jpg
\theoremstyle{definition} %styl dla definicji
\newtheorem{zad}{} 
\title{Multizestaw zadań}
\author{Robert Fidytek}
%\date{\today}
\date{}
\newcounter{liczniksekcji}
\newcommand{\kategoria}[1]{\section{#1}} %olreślamy nazwę kateforii zadań
\newcommand{\zadStart}[1]{\begin{zad}#1\newline} %oznaczenie początku zadania
\newcommand{\zadStop}{\end{zad}}   %oznaczenie końca zadania
%Makra opcjonarne (nie muszą występować):
\newcommand{\rozwStart}[2]{\noindent \textbf{Rozwiązanie (autor #1 , recenzent #2): }\newline} %oznaczenie początku rozwiązania, opcjonarnie można wprowadzić informację o autorze rozwiązania zadania i recenzencie poprawności wykonania rozwiązania zadania
\newcommand{\rozwStop}{\newline}                                            %oznaczenie końca rozwiązania
\newcommand{\odpStart}{\noindent \textbf{Odpowiedź:}\newline}    %oznaczenie początku odpowiedzi końcowej (wypisanie wyniku)
\newcommand{\odpStop}{\newline}                                             %oznaczenie końca odpowiedzi końcowej (wypisanie wyniku)
\newcommand{\testStart}{\noindent \textbf{Test:}\newline} %ewentualne możliwe opcje odpowiedzi testowej: A. ? B. ? C. ? D. ? itd.
\newcommand{\testStop}{\newline} %koniec wprowadzania odpowiedzi testowych
\newcommand{\kluczStart}{\noindent \textbf{Test poprawna odpowiedź:}\newline} %klucz, poprawna odpowiedź pytania testowego (jedna literka): A lub B lub C lub D itd.
\newcommand{\kluczStop}{\newline} %koniec poprawnej odpowiedzi pytania testowego 
\newcommand{\wstawGrafike}[2]{\begin{figure}[h] \includegraphics[scale=#2] {#1} \end{figure}} %gdyby była potrzeba wstawienia obrazka, parametry: nazwa pliku, skala (jak nie wiesz co wpisać, to wpisz 1)

\begin{document}
\maketitle


\kategoria{Wikieł/Z2.61}
\zadStart{Zadanie z Wikieł Z 2.61 moja wersja nr [nrWersji]}
%[a]:[2,3,4,5,6,7,8,9,10,11,12,13,14,15,16,17,18,19,20,21,22,23,24,25,26,27,28,29,30]
%[a1]:[1,2,3,5,6,7,9,10,11,13,14,15,16,17,19,20,21,22,23,24,25,26,27,28,29,30]
%[aa1]=[a1]*[a1]
%[2a1]=2*[a1]
%[k2a1]=[2a1]*[2a1]
%[22aa1]=[2a1]*[a]*2
%[aa]=[a]*[a]
%[4aa1]=4*[aa1]
%[x]=[22aa1]/[aa]
%[cx]=int([x])
%[x].is_integer()==True
Napisać równanie stycznej do paraboli $y^2=[a]x$ przechodzącej przez punkt A(0,[a1]). 
\zadStop
\rozwStart{Aleksandra Pasińska}{}
$$y=ax+b$$
$$b=[a1]$$
$$y=ax+[a1]$$
$$(ax+[a1])^2=[a]x$$
$$a^2x^2+[2a1]ax+[aa1]-[a]x=0$$
$$a^2x^2+x([2a1]a-[a])+[aa1]=0$$
$$\Delta=([2a1]a-[a])^2-4a^2\cdot [aa1]$$
$$[k2a1]a^2-[22aa1]a+[aa]-[4aa1]a^2=0$$
$$-[22aa1]a=-[aa]$$
$$a=\frac{1}{[cx]}$$
$$y=\frac{1}{[cx]}x+[a1]$$
\rozwStop
\odpStart
$y=\frac{1}{[cx]}x+[a1]$\\
\odpStop
\testStart
A.$y=\frac{1}{[cx]}x+[a1]$
B.$ y=-x-[a1], y=-x-[aa1]$
C.$ y=[aa1], y=[a1]$
D.$ y=x-[aa1], y=0$
E.$ y=0, y=-x+[aa1]$
F.$ y=-[aa1], y=-x+[a1]$
G.$ y=x, y=[a1]$
H.$ y=x-[a1], y=-x$
I.$ y=x, y=x+[aa1]$
\testStop
\kluczStart
A
\kluczStop



\end{document}