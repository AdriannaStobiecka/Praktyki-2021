\documentclass[12pt, a4paper]{article}
\usepackage[utf8]{inputenc}
\usepackage{polski}

\usepackage{amsthm}  %pakiet do tworzenia twierdzeń itp.
\usepackage{amsmath} %pakiet do niektórych symboli matematycznych
\usepackage{amssymb} %pakiet do symboli mat., np. \nsubseteq
\usepackage{amsfonts}
\usepackage{graphicx} %obsługa plików graficznych z rozszerzeniem png, jpg
\theoremstyle{definition} %styl dla definicji
\newtheorem{zad}{} 
\title{Multizestaw zadań}
\author{Robert Fidytek}
%\date{\today}
\date{}
\newcounter{liczniksekcji}
\newcommand{\kategoria}[1]{\section{#1}} %olreślamy nazwę kateforii zadań
\newcommand{\zadStart}[1]{\begin{zad}#1\newline} %oznaczenie początku zadania
\newcommand{\zadStop}{\end{zad}}   %oznaczenie końca zadania
%Makra opcjonarne (nie muszą występować):
\newcommand{\rozwStart}[2]{\noindent \textbf{Rozwiązanie (autor #1 , recenzent #2): }\newline} %oznaczenie początku rozwiązania, opcjonarnie można wprowadzić informację o autorze rozwiązania zadania i recenzencie poprawności wykonania rozwiązania zadania
\newcommand{\rozwStop}{\newline}                                            %oznaczenie końca rozwiązania
\newcommand{\odpStart}{\noindent \textbf{Odpowiedź:}\newline}    %oznaczenie początku odpowiedzi końcowej (wypisanie wyniku)
\newcommand{\odpStop}{\newline}                                             %oznaczenie końca odpowiedzi końcowej (wypisanie wyniku)
\newcommand{\testStart}{\noindent \textbf{Test:}\newline} %ewentualne możliwe opcje odpowiedzi testowej: A. ? B. ? C. ? D. ? itd.
\newcommand{\testStop}{\newline} %koniec wprowadzania odpowiedzi testowych
\newcommand{\kluczStart}{\noindent \textbf{Test poprawna odpowiedź:}\newline} %klucz, poprawna odpowiedź pytania testowego (jedna literka): A lub B lub C lub D itd.
\newcommand{\kluczStop}{\newline} %koniec poprawnej odpowiedzi pytania testowego 
\newcommand{\wstawGrafike}[2]{\begin{figure}[h] \includegraphics[scale=#2] {#1} \end{figure}} %gdyby była potrzeba wstawienia obrazka, parametry: nazwa pliku, skala (jak nie wiesz co wpisać, to wpisz 1)

\begin{document}
\maketitle


\kategoria{Wikieł/Z1.129v}
\zadStart{Zadanie z Wikieł Z 1.129 v) moja wersja nr [nrWersji]}
%[p1]:[2,3,4,5]
%[p2]:[2,3,4,5,6]
%[p3]:[2,3,4,5,6]
%[a]=pow([p1],[p2])
%[r]=-1+[p2]-[p3]
%[r1]=1+[p2]-[p3]
%[r12]=round([r1]/2,2)
%[p1r12]=round([p1]**([r12]),2)
%[c]=abs(-1-[p2]-[p3])
%[c2]=round([c]/2,2)
%[p1c2]=round([p1]**([c2]),2)
%[p1]<[a] and [r]<0  and [p1r12]>0 and [p1r12]<[p1] and [p1c2]>[a]


Wyznaczyć dziedzinę naturalną funkcji.
$$f(x)=\ln(|1-\log_{[p1]}x|+|\log_{[p1]}x-[p2]|-[p3])$$
\zadStop

\rozwStart{Maja Szabłowska}{Wojciech Przybylski}
$$|1-\log_{[p1]}x|+|\log_{[p1]}x-[p2]|-[p3]>0$$
$$x\in(0,\infty)$$
Na początku należy sprawdzić, dla jakich wartości $x$ zmienia się znak w wartości bezwzględnej.
$$1-\log_{[p1]}x=0 \Rightarrow \log_{[p1]}x=\log_{[p1]}[p1] \Rightarrow x=[p1]$$
$$\log_{[p1]}x-[p2]=0 \Rightarrow \log_{[p1]}x=\log_{[p1]}[a]\Rightarrow x=[a]$$
Kolejno, przechodzimy do rozwiązania nierówności.
\begin{enumerate}
    \item $x\in(0, [p1])$
    $$1-\log_{[p1]}x-\log_{[p1]}x+[p2]-[p3]>0$$
    $$-2\log_{[p1]}x+[r1]>0$$
    $$[r12]>\log_{[p1]}x$$
    $$[p1r12]>x$$
    $$x\in(0, [p1r12])$$
    
    \item $x\in[[p1],[a])$
    $$-1+\log_{[p1]}x-\log_{[p1]}x+[p2]-[p3]>0$$
    $$[r]>0, \quad \textit{Sprzeczność!}$$
    
    \item $x\in [[a],\infty)$
    $$-1+\log_{[p1]}x+\log_{[p1]}x-[p2]-[p3]>0$$
    $$2\log_{[p1]}x>[c]$$
    $$\log_{[p1]}x>[c2]$$
    $$x>[p1c2]$$
    $$x\in([p1c2],\infty)$$
\end{enumerate}
Ostatecznie otrzymujemy
$$x\in(0, [p1r12])\cup([p1c2],\infty)$$
\rozwStop
\odpStart
$x\in(0, [p1r12])\cup([p1c2],\infty)$
\odpStop
\testStart
A. $x\in(0, [p1r12])\cup([p1c2],\infty)$
B. $x\in([p1c2],\infty)$
C. $x\in(-\infty, 0)$\\
D. $x\in(-\infty, -[p2]] \cup [\ln[p1],\infty)$
E. $x\in[[p1],\infty)$
F. $x\in([p2],\infty)$
G. $x\in\emptyset$
\testStop
\kluczStart
A
\kluczStop



\end{document}
