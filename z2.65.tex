\documentclass[12pt, a4paper]{article}
\usepackage[utf8]{inputenc}
\usepackage{polski}

\usepackage{amsthm}  %pakiet do tworzenia twierdzeń itp.
\usepackage{amsmath} %pakiet do niektórych symboli matematycznych
\usepackage{amssymb} %pakiet do symboli mat., np. \nsubseteq
\usepackage{amsfonts}
\usepackage{graphicx} %obsługa plików graficznych z rozszerzeniem png, jpg
\theoremstyle{definition} %styl dla definicji
\newtheorem{zad}{} 
\title{Multizestaw zadań}
\author{Robert Fidytek}
%\date{\today}
\date{}
\newcounter{liczniksekcji}
\newcommand{\kategoria}[1]{\section{#1}} %olreślamy nazwę kateforii zadań
\newcommand{\zadStart}[1]{\begin{zad}#1\newline} %oznaczenie początku zadania
\newcommand{\zadStop}{\end{zad}}   %oznaczenie końca zadania
%Makra opcjonarne (nie muszą występować):
\newcommand{\rozwStart}[2]{\noindent \textbf{Rozwiązanie (autor #1 , recenzent #2): }\newline} %oznaczenie początku rozwiązania, opcjonarnie można wprowadzić informację o autorze rozwiązania zadania i recenzencie poprawności wykonania rozwiązania zadania
\newcommand{\rozwStop}{\newline}                                            %oznaczenie końca rozwiązania
\newcommand{\odpStart}{\noindent \textbf{Odpowiedź:}\newline}    %oznaczenie początku odpowiedzi końcowej (wypisanie wyniku)
\newcommand{\odpStop}{\newline}                                             %oznaczenie końca odpowiedzi końcowej (wypisanie wyniku)
\newcommand{\testStart}{\noindent \textbf{Test:}\newline} %ewentualne możliwe opcje odpowiedzi testowej: A. ? B. ? C. ? D. ? itd.
\newcommand{\testStop}{\newline} %koniec wprowadzania odpowiedzi testowych
\newcommand{\kluczStart}{\noindent \textbf{Test poprawna odpowiedź:}\newline} %klucz, poprawna odpowiedź pytania testowego (jedna literka): A lub B lub C lub D itd.
\newcommand{\kluczStop}{\newline} %koniec poprawnej odpowiedzi pytania testowego 
\newcommand{\wstawGrafike}[2]{\begin{figure}[h] \includegraphics[scale=#2] {#1} \end{figure}} %gdyby była potrzeba wstawienia obrazka, parametry: nazwa pliku, skala (jak nie wiesz co wpisać, to wpisz 1)

\begin{document}
\maketitle


\kategoria{Wikieł/Z2.65}
\zadStart{Zadanie z Wikieł Z 2.65 moja wersja nr [nrWersji]}
%[a]:[3,4,5,6,7,8,9,10,11,12,13,14,15,16,17,18,19,20,21,22,23,24,25,26,27,28,29,30,31,32,33,34,35,36,37,38,39,40,41,42,43,44,45,46,47,48,49,50]
%[b]:[2,3,4,5]
%[c]:[2,3,4,5,6,7,8,9,10,11]
%[2ba]=2*[b]*[a]
%[baa]=[b]*[a]*[a]
%[b1]=[b]-1
%[baac]=[baa]+[c]
%[k2ba]=[2ba]*[2ba]
%[4bb]=4*[b1]*[baac]
%[dd]=[k2ba]-[4bb]
%[d]=math.sqrt([dd])
%[cd]=int([d])
%[2bd1]=[2ba]+[cd]
%[2bd2]=[2ba]-[cd]
%[2b1]=2*[b1]
%[x1]=[2bd1]/[2b1]
%[x2]=[2bd2]/[2b1]
%[cx1]=int([x1])
%[cx2]=int([x2])
%[y1]=[cx1]-[a]
%[y2]=[cx2]-[a]
%[d].is_integer()==True and [x1].is_integer()==True and [x2].is_integer()==True  
Znaleźć punkty wspólne dla prostej $y=x-[a]$ i hiperboli $x^2-[b]y^2=[c]$.
\zadStop
\rozwStart{Aleksandra Pasińska}{}
$$x^2-[b](x-[a])^2=[c]$$
$$x^2-[b]x^2+[2ba]x-[baa]-[c]=0$$
$$-[b1]x^2+[2ba]x-[baac]=0$$
$$[b1]x^2-[2ba]x+[baac]=0$$
$$\Delta=[k2ba]-[4bb]$$
$$\sqrt{\Delta}=[cd]$$
$$x_1=\frac{[2ba]+[cd]}{2\cdot[b1]}=[cx1]$$
$$x_2=\frac{[2ba]-[cd]}{2\cdot[b1]}=[cx2]$$
$$y_1=[cx1]-[a]=[y1]$$
$$y_2=[cx2]-[a]=[y2]$$
$$([cx1],[y1]),([cx2],[y2])$$
\rozwStop
\odpStart
$([cx1],[y1]),([cx2],[y2])$\\
\odpStop
\testStart
A.$([cx1],[y1]),([cx2],[y2])$
B.$ ([cx1],[y2]),([cx2],[y2])$
C.$ ([cx1],[cx2]),([cx1],[y2])$
D.$([cx2],[y1]),([cx2],[y2])$
E.$([cx2],[y1]),([cx1],[y2])$
F.$([cx1],[y1]),([cx2],[y1])$
G.$([cx1],[y1]),(e,[cx2])$
H.$([cx1],e),([cx2],[cx1])$
I.$([cx1],[cx1]),([cx2],[y2])$
\testStop
\kluczStart
A
\kluczStop



\end{document}