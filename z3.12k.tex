\documentclass[12pt, a4paper]{article}
\usepackage[utf8]{inputenc}
\usepackage{polski}
\usepackage{amsthm}  %pakiet do tworzenia twierdzeń itp.
\usepackage{amsmath} %pakiet do niektórych symboli matematycznych
\usepackage{amssymb} %pakiet do symboli mat., np. \nsubseteq
\usepackage{amsfonts}
\usepackage{graphicx} %obsługa plików graficznych z rozszerzeniem png, jpg
\theoremstyle{definition} %styl dla definicji
\newtheorem{zad}{} 
\title{Multizestaw zadań}
\author{Patryk Wirkus}
%\date{\today}
\date{}
\newcommand{\kategoria}[1]{\section{#1}}
\newcommand{\zadStart}[1]{\begin{zad}#1\newline}
\newcommand{\zadStop}{\end{zad}}
\newcommand{\rozwStart}[2]{\noindent \textbf{Rozwiązanie (autor #1 , recenzent #2): }\newline}
\newcommand{\rozwStop}{\newline}                                           
\newcommand{\odpStart}{\noindent \textbf{Odpowiedź:}\newline}
\newcommand{\odpStop}{\newline}
\newcommand{\testStart}{\noindent \textbf{Test:}\newline}
\newcommand{\testStop}{\newline}
\newcommand{\kluczStart}{\noindent \textbf{Test poprawna odpowiedź:}\newline}
\newcommand{\kluczStop}{\newline}
\newcommand{\wstawGrafike}[2]{\begin{figure}[h] \includegraphics[scale=#2] {#1} \end{figure}}

\begin{document}
\maketitle

\kategoria{Wikieł/Z3.12k}


\zadStart{Zadanie z Wikieł Z 3.12 k) moja wersja nr 1}
Obliczyć granicę ciągu $a_{n}=\frac{5^{n}}{1+2^{n} + 3^{n}}$.
\zadStop
\rozwStart{Patryk Wirkus}{Wojciech Przybylski}
$$\lim\limits_{n\to\infty}\frac{5^{n}}{1+2^{n} + 3^{n}}=
\lim\limits_{n\to\infty}\frac{3^{n}(\frac{5^{n}}{3^{n}} )}{3^{n}(\frac{1}{3^{n}}+\frac{2^{n}}{3^{n}}+1)} = \infty$$
\rozwStop
\odpStart
$\infty$
\odpStop
\testStart
A.$\infty$
B.$-\infty$
C.$0$
D.$\frac{5}{2}$
E.$-\frac{5}{2}$
F.$\frac{2}{5}$
G.$-\frac{2}{5}$
H.$-5$
I.$5$
\testStop
\kluczStart
A
\kluczStop



\zadStart{Zadanie z Wikieł Z 3.12 k) moja wersja nr 2}
Obliczyć granicę ciągu $a_{n}=\frac{5^{n}}{1+2^{n} + 4^{n}}$.
\zadStop
\rozwStart{Patryk Wirkus}{Wojciech Przybylski}
$$\lim\limits_{n\to\infty}\frac{5^{n}}{1+2^{n} + 4^{n}}=
\lim\limits_{n\to\infty}\frac{4^{n}(\frac{5^{n}}{4^{n}} )}{4^{n}(\frac{1}{4^{n}}+\frac{2^{n}}{4^{n}}+1)} = \infty$$
\rozwStop
\odpStart
$\infty$
\odpStop
\testStart
A.$\infty$
B.$-\infty$
C.$0$
D.$\frac{5}{2}$
E.$-\frac{5}{2}$
F.$\frac{2}{5}$
G.$-\frac{2}{5}$
H.$-5$
I.$5$
\testStop
\kluczStart
A
\kluczStop



\zadStart{Zadanie z Wikieł Z 3.12 k) moja wersja nr 3}
Obliczyć granicę ciągu $a_{n}=\frac{5^{n}}{1+3^{n} + 4^{n}}$.
\zadStop
\rozwStart{Patryk Wirkus}{Wojciech Przybylski}
$$\lim\limits_{n\to\infty}\frac{5^{n}}{1+3^{n} + 4^{n}}=
\lim\limits_{n\to\infty}\frac{4^{n}(\frac{5^{n}}{4^{n}} )}{4^{n}(\frac{1}{4^{n}}+\frac{3^{n}}{4^{n}}+1)} = \infty$$
\rozwStop
\odpStart
$\infty$
\odpStop
\testStart
A.$\infty$
B.$-\infty$
C.$0$
D.$\frac{5}{3}$
E.$-\frac{5}{3}$
F.$\frac{3}{5}$
G.$-\frac{3}{5}$
H.$-5$
I.$5$
\testStop
\kluczStart
A
\kluczStop



\zadStart{Zadanie z Wikieł Z 3.12 k) moja wersja nr 4}
Obliczyć granicę ciągu $a_{n}=\frac{7^{n}}{1+2^{n} + 3^{n}}$.
\zadStop
\rozwStart{Patryk Wirkus}{Wojciech Przybylski}
$$\lim\limits_{n\to\infty}\frac{7^{n}}{1+2^{n} + 3^{n}}=
\lim\limits_{n\to\infty}\frac{3^{n}(\frac{7^{n}}{3^{n}} )}{3^{n}(\frac{1}{3^{n}}+\frac{2^{n}}{3^{n}}+1)} = \infty$$
\rozwStop
\odpStart
$\infty$
\odpStop
\testStart
A.$\infty$
B.$-\infty$
C.$0$
D.$\frac{7}{2}$
E.$-\frac{7}{2}$
F.$\frac{2}{7}$
G.$-\frac{2}{7}$
H.$-7$
I.$7$
\testStop
\kluczStart
A
\kluczStop



\zadStart{Zadanie z Wikieł Z 3.12 k) moja wersja nr 5}
Obliczyć granicę ciągu $a_{n}=\frac{7^{n}}{1+2^{n} + 4^{n}}$.
\zadStop
\rozwStart{Patryk Wirkus}{Wojciech Przybylski}
$$\lim\limits_{n\to\infty}\frac{7^{n}}{1+2^{n} + 4^{n}}=
\lim\limits_{n\to\infty}\frac{4^{n}(\frac{7^{n}}{4^{n}} )}{4^{n}(\frac{1}{4^{n}}+\frac{2^{n}}{4^{n}}+1)} = \infty$$
\rozwStop
\odpStart
$\infty$
\odpStop
\testStart
A.$\infty$
B.$-\infty$
C.$0$
D.$\frac{7}{2}$
E.$-\frac{7}{2}$
F.$\frac{2}{7}$
G.$-\frac{2}{7}$
H.$-7$
I.$7$
\testStop
\kluczStart
A
\kluczStop



\zadStart{Zadanie z Wikieł Z 3.12 k) moja wersja nr 6}
Obliczyć granicę ciągu $a_{n}=\frac{7^{n}}{1+2^{n} + 5^{n}}$.
\zadStop
\rozwStart{Patryk Wirkus}{Wojciech Przybylski}
$$\lim\limits_{n\to\infty}\frac{7^{n}}{1+2^{n} + 5^{n}}=
\lim\limits_{n\to\infty}\frac{5^{n}(\frac{7^{n}}{5^{n}} )}{5^{n}(\frac{1}{5^{n}}+\frac{2^{n}}{5^{n}}+1)} = \infty$$
\rozwStop
\odpStart
$\infty$
\odpStop
\testStart
A.$\infty$
B.$-\infty$
C.$0$
D.$\frac{7}{2}$
E.$-\frac{7}{2}$
F.$\frac{2}{7}$
G.$-\frac{2}{7}$
H.$-7$
I.$7$
\testStop
\kluczStart
A
\kluczStop



\zadStart{Zadanie z Wikieł Z 3.12 k) moja wersja nr 7}
Obliczyć granicę ciągu $a_{n}=\frac{7^{n}}{1+2^{n} + 6^{n}}$.
\zadStop
\rozwStart{Patryk Wirkus}{Wojciech Przybylski}
$$\lim\limits_{n\to\infty}\frac{7^{n}}{1+2^{n} + 6^{n}}=
\lim\limits_{n\to\infty}\frac{6^{n}(\frac{7^{n}}{6^{n}} )}{6^{n}(\frac{1}{6^{n}}+\frac{2^{n}}{6^{n}}+1)} = \infty$$
\rozwStop
\odpStart
$\infty$
\odpStop
\testStart
A.$\infty$
B.$-\infty$
C.$0$
D.$\frac{7}{2}$
E.$-\frac{7}{2}$
F.$\frac{2}{7}$
G.$-\frac{2}{7}$
H.$-7$
I.$7$
\testStop
\kluczStart
A
\kluczStop



\zadStart{Zadanie z Wikieł Z 3.12 k) moja wersja nr 8}
Obliczyć granicę ciągu $a_{n}=\frac{7^{n}}{1+3^{n} + 4^{n}}$.
\zadStop
\rozwStart{Patryk Wirkus}{Wojciech Przybylski}
$$\lim\limits_{n\to\infty}\frac{7^{n}}{1+3^{n} + 4^{n}}=
\lim\limits_{n\to\infty}\frac{4^{n}(\frac{7^{n}}{4^{n}} )}{4^{n}(\frac{1}{4^{n}}+\frac{3^{n}}{4^{n}}+1)} = \infty$$
\rozwStop
\odpStart
$\infty$
\odpStop
\testStart
A.$\infty$
B.$-\infty$
C.$0$
D.$\frac{7}{3}$
E.$-\frac{7}{3}$
F.$\frac{3}{7}$
G.$-\frac{3}{7}$
H.$-7$
I.$7$
\testStop
\kluczStart
A
\kluczStop



\zadStart{Zadanie z Wikieł Z 3.12 k) moja wersja nr 9}
Obliczyć granicę ciągu $a_{n}=\frac{7^{n}}{1+3^{n} + 5^{n}}$.
\zadStop
\rozwStart{Patryk Wirkus}{Wojciech Przybylski}
$$\lim\limits_{n\to\infty}\frac{7^{n}}{1+3^{n} + 5^{n}}=
\lim\limits_{n\to\infty}\frac{5^{n}(\frac{7^{n}}{5^{n}} )}{5^{n}(\frac{1}{5^{n}}+\frac{3^{n}}{5^{n}}+1)} = \infty$$
\rozwStop
\odpStart
$\infty$
\odpStop
\testStart
A.$\infty$
B.$-\infty$
C.$0$
D.$\frac{7}{3}$
E.$-\frac{7}{3}$
F.$\frac{3}{7}$
G.$-\frac{3}{7}$
H.$-7$
I.$7$
\testStop
\kluczStart
A
\kluczStop



\zadStart{Zadanie z Wikieł Z 3.12 k) moja wersja nr 10}
Obliczyć granicę ciągu $a_{n}=\frac{7^{n}}{1+3^{n} + 6^{n}}$.
\zadStop
\rozwStart{Patryk Wirkus}{Wojciech Przybylski}
$$\lim\limits_{n\to\infty}\frac{7^{n}}{1+3^{n} + 6^{n}}=
\lim\limits_{n\to\infty}\frac{6^{n}(\frac{7^{n}}{6^{n}} )}{6^{n}(\frac{1}{6^{n}}+\frac{3^{n}}{6^{n}}+1)} = \infty$$
\rozwStop
\odpStart
$\infty$
\odpStop
\testStart
A.$\infty$
B.$-\infty$
C.$0$
D.$\frac{7}{3}$
E.$-\frac{7}{3}$
F.$\frac{3}{7}$
G.$-\frac{3}{7}$
H.$-7$
I.$7$
\testStop
\kluczStart
A
\kluczStop



\zadStart{Zadanie z Wikieł Z 3.12 k) moja wersja nr 11}
Obliczyć granicę ciągu $a_{n}=\frac{7^{n}}{1+4^{n} + 5^{n}}$.
\zadStop
\rozwStart{Patryk Wirkus}{Wojciech Przybylski}
$$\lim\limits_{n\to\infty}\frac{7^{n}}{1+4^{n} + 5^{n}}=
\lim\limits_{n\to\infty}\frac{5^{n}(\frac{7^{n}}{5^{n}} )}{5^{n}(\frac{1}{5^{n}}+\frac{4^{n}}{5^{n}}+1)} = \infty$$
\rozwStop
\odpStart
$\infty$
\odpStop
\testStart
A.$\infty$
B.$-\infty$
C.$0$
D.$\frac{7}{4}$
E.$-\frac{7}{4}$
F.$\frac{4}{7}$
G.$-\frac{4}{7}$
H.$-7$
I.$7$
\testStop
\kluczStart
A
\kluczStop



\zadStart{Zadanie z Wikieł Z 3.12 k) moja wersja nr 12}
Obliczyć granicę ciągu $a_{n}=\frac{7^{n}}{1+4^{n} + 6^{n}}$.
\zadStop
\rozwStart{Patryk Wirkus}{Wojciech Przybylski}
$$\lim\limits_{n\to\infty}\frac{7^{n}}{1+4^{n} + 6^{n}}=
\lim\limits_{n\to\infty}\frac{6^{n}(\frac{7^{n}}{6^{n}} )}{6^{n}(\frac{1}{6^{n}}+\frac{4^{n}}{6^{n}}+1)} = \infty$$
\rozwStop
\odpStart
$\infty$
\odpStop
\testStart
A.$\infty$
B.$-\infty$
C.$0$
D.$\frac{7}{4}$
E.$-\frac{7}{4}$
F.$\frac{4}{7}$
G.$-\frac{4}{7}$
H.$-7$
I.$7$
\testStop
\kluczStart
A
\kluczStop



\zadStart{Zadanie z Wikieł Z 3.12 k) moja wersja nr 13}
Obliczyć granicę ciągu $a_{n}=\frac{7^{n}}{1+5^{n} + 6^{n}}$.
\zadStop
\rozwStart{Patryk Wirkus}{Wojciech Przybylski}
$$\lim\limits_{n\to\infty}\frac{7^{n}}{1+5^{n} + 6^{n}}=
\lim\limits_{n\to\infty}\frac{6^{n}(\frac{7^{n}}{6^{n}} )}{6^{n}(\frac{1}{6^{n}}+\frac{5^{n}}{6^{n}}+1)} = \infty$$
\rozwStop
\odpStart
$\infty$
\odpStop
\testStart
A.$\infty$
B.$-\infty$
C.$0$
D.$\frac{7}{5}$
E.$-\frac{7}{5}$
F.$\frac{5}{7}$
G.$-\frac{5}{7}$
H.$-7$
I.$7$
\testStop
\kluczStart
A
\kluczStop



\zadStart{Zadanie z Wikieł Z 3.12 k) moja wersja nr 14}
Obliczyć granicę ciągu $a_{n}=\frac{8^{n}}{1+3^{n} + 5^{n}}$.
\zadStop
\rozwStart{Patryk Wirkus}{Wojciech Przybylski}
$$\lim\limits_{n\to\infty}\frac{8^{n}}{1+3^{n} + 5^{n}}=
\lim\limits_{n\to\infty}\frac{5^{n}(\frac{8^{n}}{5^{n}} )}{5^{n}(\frac{1}{5^{n}}+\frac{3^{n}}{5^{n}}+1)} = \infty$$
\rozwStop
\odpStart
$\infty$
\odpStop
\testStart
A.$\infty$
B.$-\infty$
C.$0$
D.$\frac{8}{3}$
E.$-\frac{8}{3}$
F.$\frac{3}{8}$
G.$-\frac{3}{8}$
H.$-8$
I.$8$
\testStop
\kluczStart
A
\kluczStop



\zadStart{Zadanie z Wikieł Z 3.12 k) moja wersja nr 15}
Obliczyć granicę ciągu $a_{n}=\frac{8^{n}}{1+3^{n} + 7^{n}}$.
\zadStop
\rozwStart{Patryk Wirkus}{Wojciech Przybylski}
$$\lim\limits_{n\to\infty}\frac{8^{n}}{1+3^{n} + 7^{n}}=
\lim\limits_{n\to\infty}\frac{7^{n}(\frac{8^{n}}{7^{n}} )}{7^{n}(\frac{1}{7^{n}}+\frac{3^{n}}{7^{n}}+1)} = \infty$$
\rozwStop
\odpStart
$\infty$
\odpStop
\testStart
A.$\infty$
B.$-\infty$
C.$0$
D.$\frac{8}{3}$
E.$-\frac{8}{3}$
F.$\frac{3}{8}$
G.$-\frac{3}{8}$
H.$-8$
I.$8$
\testStop
\kluczStart
A
\kluczStop



\zadStart{Zadanie z Wikieł Z 3.12 k) moja wersja nr 16}
Obliczyć granicę ciągu $a_{n}=\frac{8^{n}}{1+5^{n} + 7^{n}}$.
\zadStop
\rozwStart{Patryk Wirkus}{Wojciech Przybylski}
$$\lim\limits_{n\to\infty}\frac{8^{n}}{1+5^{n} + 7^{n}}=
\lim\limits_{n\to\infty}\frac{7^{n}(\frac{8^{n}}{7^{n}} )}{7^{n}(\frac{1}{7^{n}}+\frac{5^{n}}{7^{n}}+1)} = \infty$$
\rozwStop
\odpStart
$\infty$
\odpStop
\testStart
A.$\infty$
B.$-\infty$
C.$0$
D.$\frac{8}{5}$
E.$-\frac{8}{5}$
F.$\frac{5}{8}$
G.$-\frac{5}{8}$
H.$-8$
I.$8$
\testStop
\kluczStart
A
\kluczStop



\zadStart{Zadanie z Wikieł Z 3.12 k) moja wersja nr 17}
Obliczyć granicę ciągu $a_{n}=\frac{9^{n}}{1+2^{n} + 4^{n}}$.
\zadStop
\rozwStart{Patryk Wirkus}{Wojciech Przybylski}
$$\lim\limits_{n\to\infty}\frac{9^{n}}{1+2^{n} + 4^{n}}=
\lim\limits_{n\to\infty}\frac{4^{n}(\frac{9^{n}}{4^{n}} )}{4^{n}(\frac{1}{4^{n}}+\frac{2^{n}}{4^{n}}+1)} = \infty$$
\rozwStop
\odpStart
$\infty$
\odpStop
\testStart
A.$\infty$
B.$-\infty$
C.$0$
D.$\frac{9}{2}$
E.$-\frac{9}{2}$
F.$\frac{2}{9}$
G.$-\frac{2}{9}$
H.$-9$
I.$9$
\testStop
\kluczStart
A
\kluczStop



\zadStart{Zadanie z Wikieł Z 3.12 k) moja wersja nr 18}
Obliczyć granicę ciągu $a_{n}=\frac{9^{n}}{1+2^{n} + 5^{n}}$.
\zadStop
\rozwStart{Patryk Wirkus}{Wojciech Przybylski}
$$\lim\limits_{n\to\infty}\frac{9^{n}}{1+2^{n} + 5^{n}}=
\lim\limits_{n\to\infty}\frac{5^{n}(\frac{9^{n}}{5^{n}} )}{5^{n}(\frac{1}{5^{n}}+\frac{2^{n}}{5^{n}}+1)} = \infty$$
\rozwStop
\odpStart
$\infty$
\odpStop
\testStart
A.$\infty$
B.$-\infty$
C.$0$
D.$\frac{9}{2}$
E.$-\frac{9}{2}$
F.$\frac{2}{9}$
G.$-\frac{2}{9}$
H.$-9$
I.$9$
\testStop
\kluczStart
A
\kluczStop



\zadStart{Zadanie z Wikieł Z 3.12 k) moja wersja nr 19}
Obliczyć granicę ciągu $a_{n}=\frac{9^{n}}{1+2^{n} + 7^{n}}$.
\zadStop
\rozwStart{Patryk Wirkus}{Wojciech Przybylski}
$$\lim\limits_{n\to\infty}\frac{9^{n}}{1+2^{n} + 7^{n}}=
\lim\limits_{n\to\infty}\frac{7^{n}(\frac{9^{n}}{7^{n}} )}{7^{n}(\frac{1}{7^{n}}+\frac{2^{n}}{7^{n}}+1)} = \infty$$
\rozwStop
\odpStart
$\infty$
\odpStop
\testStart
A.$\infty$
B.$-\infty$
C.$0$
D.$\frac{9}{2}$
E.$-\frac{9}{2}$
F.$\frac{2}{9}$
G.$-\frac{2}{9}$
H.$-9$
I.$9$
\testStop
\kluczStart
A
\kluczStop



\zadStart{Zadanie z Wikieł Z 3.12 k) moja wersja nr 20}
Obliczyć granicę ciągu $a_{n}=\frac{9^{n}}{1+2^{n} + 8^{n}}$.
\zadStop
\rozwStart{Patryk Wirkus}{Wojciech Przybylski}
$$\lim\limits_{n\to\infty}\frac{9^{n}}{1+2^{n} + 8^{n}}=
\lim\limits_{n\to\infty}\frac{8^{n}(\frac{9^{n}}{8^{n}} )}{8^{n}(\frac{1}{8^{n}}+\frac{2^{n}}{8^{n}}+1)} = \infty$$
\rozwStop
\odpStart
$\infty$
\odpStop
\testStart
A.$\infty$
B.$-\infty$
C.$0$
D.$\frac{9}{2}$
E.$-\frac{9}{2}$
F.$\frac{2}{9}$
G.$-\frac{2}{9}$
H.$-9$
I.$9$
\testStop
\kluczStart
A
\kluczStop



\zadStart{Zadanie z Wikieł Z 3.12 k) moja wersja nr 21}
Obliczyć granicę ciągu $a_{n}=\frac{9^{n}}{1+4^{n} + 5^{n}}$.
\zadStop
\rozwStart{Patryk Wirkus}{Wojciech Przybylski}
$$\lim\limits_{n\to\infty}\frac{9^{n}}{1+4^{n} + 5^{n}}=
\lim\limits_{n\to\infty}\frac{5^{n}(\frac{9^{n}}{5^{n}} )}{5^{n}(\frac{1}{5^{n}}+\frac{4^{n}}{5^{n}}+1)} = \infty$$
\rozwStop
\odpStart
$\infty$
\odpStop
\testStart
A.$\infty$
B.$-\infty$
C.$0$
D.$\frac{9}{4}$
E.$-\frac{9}{4}$
F.$\frac{4}{9}$
G.$-\frac{4}{9}$
H.$-9$
I.$9$
\testStop
\kluczStart
A
\kluczStop



\zadStart{Zadanie z Wikieł Z 3.12 k) moja wersja nr 22}
Obliczyć granicę ciągu $a_{n}=\frac{9^{n}}{1+4^{n} + 7^{n}}$.
\zadStop
\rozwStart{Patryk Wirkus}{Wojciech Przybylski}
$$\lim\limits_{n\to\infty}\frac{9^{n}}{1+4^{n} + 7^{n}}=
\lim\limits_{n\to\infty}\frac{7^{n}(\frac{9^{n}}{7^{n}} )}{7^{n}(\frac{1}{7^{n}}+\frac{4^{n}}{7^{n}}+1)} = \infty$$
\rozwStop
\odpStart
$\infty$
\odpStop
\testStart
A.$\infty$
B.$-\infty$
C.$0$
D.$\frac{9}{4}$
E.$-\frac{9}{4}$
F.$\frac{4}{9}$
G.$-\frac{4}{9}$
H.$-9$
I.$9$
\testStop
\kluczStart
A
\kluczStop



\zadStart{Zadanie z Wikieł Z 3.12 k) moja wersja nr 23}
Obliczyć granicę ciągu $a_{n}=\frac{9^{n}}{1+4^{n} + 8^{n}}$.
\zadStop
\rozwStart{Patryk Wirkus}{Wojciech Przybylski}
$$\lim\limits_{n\to\infty}\frac{9^{n}}{1+4^{n} + 8^{n}}=
\lim\limits_{n\to\infty}\frac{8^{n}(\frac{9^{n}}{8^{n}} )}{8^{n}(\frac{1}{8^{n}}+\frac{4^{n}}{8^{n}}+1)} = \infty$$
\rozwStop
\odpStart
$\infty$
\odpStop
\testStart
A.$\infty$
B.$-\infty$
C.$0$
D.$\frac{9}{4}$
E.$-\frac{9}{4}$
F.$\frac{4}{9}$
G.$-\frac{4}{9}$
H.$-9$
I.$9$
\testStop
\kluczStart
A
\kluczStop



\zadStart{Zadanie z Wikieł Z 3.12 k) moja wersja nr 24}
Obliczyć granicę ciągu $a_{n}=\frac{9^{n}}{1+5^{n} + 7^{n}}$.
\zadStop
\rozwStart{Patryk Wirkus}{Wojciech Przybylski}
$$\lim\limits_{n\to\infty}\frac{9^{n}}{1+5^{n} + 7^{n}}=
\lim\limits_{n\to\infty}\frac{7^{n}(\frac{9^{n}}{7^{n}} )}{7^{n}(\frac{1}{7^{n}}+\frac{5^{n}}{7^{n}}+1)} = \infty$$
\rozwStop
\odpStart
$\infty$
\odpStop
\testStart
A.$\infty$
B.$-\infty$
C.$0$
D.$\frac{9}{5}$
E.$-\frac{9}{5}$
F.$\frac{5}{9}$
G.$-\frac{5}{9}$
H.$-9$
I.$9$
\testStop
\kluczStart
A
\kluczStop



\zadStart{Zadanie z Wikieł Z 3.12 k) moja wersja nr 25}
Obliczyć granicę ciągu $a_{n}=\frac{9^{n}}{1+5^{n} + 8^{n}}$.
\zadStop
\rozwStart{Patryk Wirkus}{Wojciech Przybylski}
$$\lim\limits_{n\to\infty}\frac{9^{n}}{1+5^{n} + 8^{n}}=
\lim\limits_{n\to\infty}\frac{8^{n}(\frac{9^{n}}{8^{n}} )}{8^{n}(\frac{1}{8^{n}}+\frac{5^{n}}{8^{n}}+1)} = \infty$$
\rozwStop
\odpStart
$\infty$
\odpStop
\testStart
A.$\infty$
B.$-\infty$
C.$0$
D.$\frac{9}{5}$
E.$-\frac{9}{5}$
F.$\frac{5}{9}$
G.$-\frac{5}{9}$
H.$-9$
I.$9$
\testStop
\kluczStart
A
\kluczStop



\zadStart{Zadanie z Wikieł Z 3.12 k) moja wersja nr 26}
Obliczyć granicę ciągu $a_{n}=\frac{9^{n}}{1+7^{n} + 8^{n}}$.
\zadStop
\rozwStart{Patryk Wirkus}{Wojciech Przybylski}
$$\lim\limits_{n\to\infty}\frac{9^{n}}{1+7^{n} + 8^{n}}=
\lim\limits_{n\to\infty}\frac{8^{n}(\frac{9^{n}}{8^{n}} )}{8^{n}(\frac{1}{8^{n}}+\frac{7^{n}}{8^{n}}+1)} = \infty$$
\rozwStop
\odpStart
$\infty$
\odpStop
\testStart
A.$\infty$
B.$-\infty$
C.$0$
D.$\frac{9}{7}$
E.$-\frac{9}{7}$
F.$\frac{7}{9}$
G.$-\frac{7}{9}$
H.$-9$
I.$9$
\testStop
\kluczStart
A
\kluczStop



\zadStart{Zadanie z Wikieł Z 3.12 k) moja wersja nr 27}
Obliczyć granicę ciągu $a_{n}=\frac{10^{n}}{1+3^{n} + 7^{n}}$.
\zadStop
\rozwStart{Patryk Wirkus}{Wojciech Przybylski}
$$\lim\limits_{n\to\infty}\frac{10^{n}}{1+3^{n} + 7^{n}}=
\lim\limits_{n\to\infty}\frac{7^{n}(\frac{10^{n}}{7^{n}} )}{7^{n}(\frac{1}{7^{n}}+\frac{3^{n}}{7^{n}}+1)} = \infty$$
\rozwStop
\odpStart
$\infty$
\odpStop
\testStart
A.$\infty$
B.$-\infty$
C.$0$
D.$\frac{10}{3}$
E.$-\frac{10}{3}$
F.$\frac{3}{10}$
G.$-\frac{3}{10}$
H.$-10$
I.$10$
\testStop
\kluczStart
A
\kluczStop



\zadStart{Zadanie z Wikieł Z 3.12 k) moja wersja nr 28}
Obliczyć granicę ciągu $a_{n}=\frac{10^{n}}{1+3^{n} + 9^{n}}$.
\zadStop
\rozwStart{Patryk Wirkus}{Wojciech Przybylski}
$$\lim\limits_{n\to\infty}\frac{10^{n}}{1+3^{n} + 9^{n}}=
\lim\limits_{n\to\infty}\frac{9^{n}(\frac{10^{n}}{9^{n}} )}{9^{n}(\frac{1}{9^{n}}+\frac{3^{n}}{9^{n}}+1)} = \infty$$
\rozwStop
\odpStart
$\infty$
\odpStop
\testStart
A.$\infty$
B.$-\infty$
C.$0$
D.$\frac{10}{3}$
E.$-\frac{10}{3}$
F.$\frac{3}{10}$
G.$-\frac{3}{10}$
H.$-10$
I.$10$
\testStop
\kluczStart
A
\kluczStop



\zadStart{Zadanie z Wikieł Z 3.12 k) moja wersja nr 29}
Obliczyć granicę ciągu $a_{n}=\frac{10^{n}}{1+7^{n} + 9^{n}}$.
\zadStop
\rozwStart{Patryk Wirkus}{Wojciech Przybylski}
$$\lim\limits_{n\to\infty}\frac{10^{n}}{1+7^{n} + 9^{n}}=
\lim\limits_{n\to\infty}\frac{9^{n}(\frac{10^{n}}{9^{n}} )}{9^{n}(\frac{1}{9^{n}}+\frac{7^{n}}{9^{n}}+1)} = \infty$$
\rozwStop
\odpStart
$\infty$
\odpStop
\testStart
A.$\infty$
B.$-\infty$
C.$0$
D.$\frac{10}{7}$
E.$-\frac{10}{7}$
F.$\frac{7}{10}$
G.$-\frac{7}{10}$
H.$-10$
I.$10$
\testStop
\kluczStart
A
\kluczStop





\end{document}
