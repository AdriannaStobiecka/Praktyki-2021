\documentclass[12pt, a4paper]{article}
\usepackage[utf8]{inputenc}
\usepackage{polski}

\usepackage{amsthm}  %pakiet do tworzenia twierdzeń itp.
\usepackage{amsmath} %pakiet do niektórych symboli matematycznych
\usepackage{amssymb} %pakiet do symboli mat., np. \nsubseteq
\usepackage{amsfonts}
\usepackage{graphicx} %obsługa plików graficznych z rozszerzeniem png, jpg
\theoremstyle{definition} %styl dla definicji
\newtheorem{zad}{} 
\title{Multizestaw zadań}
\author{Jacek Jabłoński}
%\date{\today}
\date{}
\newcounter{liczniksekcji}
\newcommand{\kategoria}[1]{\section{#1}} %olreślamy nazwę kateforii zadań
\newcommand{\zadStart}[1]{\begin{zad}#1\newline} %oznaczenie początku zadania
\newcommand{\zadStop}{\end{zad}}   %oznaczenie końca zadania
%Makra opcjonarne (nie muszą występować):
\newcommand{\rozwStart}[2]{\noindent \textbf{Rozwiązanie (autor #1 , recenzent #2): }\newline} %oznaczenie początku rozwiązania, opcjonarnie można wprowadzić informację o autorze rozwiązania zadania i recenzencie poprawności wykonania rozwiązania zadania
\newcommand{\rozwStop}{\newline}                                            %oznaczenie końca rozwiązania
\newcommand{\odpStart}{\noindent \textbf{Odpowiedź:}\newline}    %oznaczenie początku odpowiedzi końcowej (wypisanie wyniku)
\newcommand{\odpStop}{\newline}                                             %oznaczenie końca odpowiedzi końcowej (wypisanie wyniku)
\newcommand{\testStart}{\noindent \textbf{Test:}\newline} %ewentualne możliwe opcje odpowiedzi testowej: A. ? B. ? C. ? D. ? itd.
\newcommand{\testStop}{\newline} %koniec wprowadzania odpowiedzi testowych
\newcommand{\kluczStart}{\noindent \textbf{Test poprawna odpowiedź:}\newline} %klucz, poprawna odpowiedź pytania testowego (jedna literka): A lub B lub C lub D itd.
\newcommand{\kluczStop}{\newline} %koniec poprawnej odpowiedzi pytania testowego 
\newcommand{\wstawGrafike}[2]{\begin{figure}[h] \includegraphics[scale=#2] {#1} \end{figure}} %gdyby była potrzeba wstawienia obrazka, parametry: nazwa pliku, skala (jak nie wiesz co wpisać, to wpisz 1)

\begin{document}
\maketitle


\kategoria{Wikieł/P5.2b}
\zadStart{Zadanie z Wikieł P 5.2b) moja wersja nr [nrWersji]}
%[p1]:[2,4,8,16,32,64,128,256,512,1024]
%[p2]:[2,4,8,16,32,64,128,256,512,1024]
%[a]=random.randint(2,8)
%[b]=random.randint(2,8)
%[c]=random.randint(2,8)
%[d]=random.randint(2,32)
%[w1]=2-[a]
%[w2]=1-[b]
%[w3]=1-[c]
%[w11]=[w1]*(-1)
%[w22]=[w2]*(-1)
%[w33]=[w3]*(-1)
%[a]!=[b] and [b]!=[c] and [c]!=[a] and [w1]<0 and [w2]<0 and [w3]<0
%[f44]=(1-[d])*(-1)
Korzystając z podanych wzorów i twierdzeń, wyznacz pochodną funkcji:
b) $\sqrt[[a]]{x^2} + \sqrt[[b]]{x} +\sqrt[[c]]{x} + \sqrt{[d]}$
\zadStop
\rozwStart{Jacek Jabłoński}{}
$$\sqrt[[a]]{x^2} + \sqrt[[b]]{x} +\sqrt[[c]]{x} + \sqrt{[d]} = (x^{\frac{2}{[a]}})' + (x^{\frac{1}{[b]}})' + (x^{\frac{1}{[c]}})' +([d]^{\frac{1}{2}})' =$$
$$= \frac{2}{[a]}x^{\frac{[w1]}{[a]}} + \frac{1}{[b]}x^{\frac{[w2]}{[b]}} + \frac{1}{[c]}x^{\frac{[w3]}{[c]}} + 0 =$$
$$= \frac{2}{[a] \sqrt[[a]]{x^{[w11]}}} + \frac{1}{[b] \sqrt[[b]]{x^{[w22]}}} + \frac{1}{[c] \sqrt[[c]]{x^{[w33]}}}$$
\rozwStop
\odpStart
$$\frac{2}{[a] \sqrt[[a]]{x^{[w11]}}} + \frac{1}{[b] \sqrt[[b]]{x^{[w22]}}} + \frac{1}{[c] \sqrt[[c]]{x^{[w33]}}}$$
\odpStop
\testStart
A. $$\frac{2}{[a] \sqrt[[a]]{x^{[w11]}}} + \frac{1}{[b] \sqrt[[b]]{x^{[w22]}}} + \frac{1}{[c] \sqrt[[c]]{x^{[w33]}}}$$
B. $$\frac{1}{[a] \sqrt[[a]]{x^{[w11]}}} - \frac{1}{[b] \sqrt[[b]]{x^{[w22]}}} + \frac{2}{[c] \sqrt[[c]]{x^{[w33]}}}$$
C. $$\frac{2}{[a] \sqrt[[a]]{x^{[w11]}}} + \frac{2}{[b] \sqrt[[b]]{x^{[w22]}}}$$
D. $$\frac{2}{[a] \sqrt[[a]]{x^{[w11]}}} + \frac{1}{[c] \sqrt[[c]]{x^{[w33]}}}$$
E. $$\frac{2}{[a] \sqrt[[a]]{x^{[w11]}}} + \frac{1}{[b] \sqrt[[b]]{x^{[w22]}}} + \frac{1}{[c] \sqrt[[c]]{x^{[w33]}}} + \frac{1}{[d] \sqrt[[d]]{x^{[f44]}}} $$
F. $$\frac{2}{[a] \sqrt[[a]]{x^{[w11]}}} + \frac{1}{[c] \sqrt[[c]]{x^{[w33]}}} - \frac{1}{[d] \sqrt[[d]]{x^{[f44]}}} $$
G. $$\frac{4}{[a] \sqrt[[a]]{x^{[w11]}}} - \frac{1}{[b] \sqrt[[b]]{x^{[w22]}}} + \frac{1}{[c] \sqrt[[c]]{x^{[w33]}}} + \frac{1}{[d] \sqrt[[d]]{x^{[f44]}}} $$
H. $$\frac{2}{[b] \sqrt[[c]]{x^{[w22]}}} + \frac{1}{[b] \sqrt[[b]]{x^{[w22]}}} + \frac{1}{[c] \sqrt[[c]]{x^{[w33]}}}$$
I. $$\frac{2}{[a] \sqrt[[a]]{x^{[w22]}}} + \frac{1}{[b] \sqrt[[b]]{x^{[w33]}}} + \frac{1}{[c] \sqrt[[c]]{x^{[w11]}}}$$
\testStop
\kluczStart
A
\kluczStop



\end{document}