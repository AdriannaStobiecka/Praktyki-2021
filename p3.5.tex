\documentclass[12pt, a4paper]{article}
\usepackage[utf8]{inputenc}
\usepackage{polski}

\usepackage{amsthm}  %pakiet do tworzenia twierdzeń itp.
\usepackage{amsmath} %pakiet do niektórych symboli matematycznych
\usepackage{amssymb} %pakiet do symboli mat., np. \nsubseteq
\usepackage{amsfonts}
\usepackage{graphicx} %obsługa plików graficznych z rozszerzeniem png, jpg
\theoremstyle{definition} %styl dla definicji
\newtheorem{zad}{} 
\title{Multizestaw zadań}
\author{Robert Fidytek}
%\date{\today}
\date{}
\newcounter{liczniksekcji}
\newcommand{\kategoria}[1]{\section{#1}} %olreślamy nazwę kateforii zadań
\newcommand{\zadStart}[1]{\begin{zad}#1\newline} %oznaczenie początku zadania
\newcommand{\zadStop}{\end{zad}}   %oznaczenie końca zadania
%Makra opcjonarne (nie muszą występować):
\newcommand{\rozwStart}[2]{\noindent \textbf{Rozwiązanie (autor #1 , recenzent #2): }\newline} %oznaczenie początku rozwiązania, opcjonarnie można wprowadzić informację o autorze rozwiązania zadania i recenzencie poprawności wykonania rozwiązania zadania
\newcommand{\rozwStop}{\newline}                                            %oznaczenie końca rozwiązania
\newcommand{\odpStart}{\noindent \textbf{Odpowiedź:}\newline}    %oznaczenie początku odpowiedzi końcowej (wypisanie wyniku)
\newcommand{\odpStop}{\newline}                                             %oznaczenie końca odpowiedzi końcowej (wypisanie wyniku)
\newcommand{\testStart}{\noindent \textbf{Test:}\newline} %ewentualne możliwe opcje odpowiedzi testowej: A. ? B. ? C. ? D. ? itd.
\newcommand{\testStop}{\newline} %koniec wprowadzania odpowiedzi testowych
\newcommand{\kluczStart}{\noindent \textbf{Test poprawna odpowiedź:}\newline} %klucz, poprawna odpowiedź pytania testowego (jedna literka): A lub B lub C lub D itd.
\newcommand{\kluczStop}{\newline} %koniec poprawnej odpowiedzi pytania testowego 
\newcommand{\wstawGrafike}[2]{\begin{figure}[h] \includegraphics[scale=#2] {#1} \end{figure}} %gdyby była potrzeba wstawienia obrazka, parametry: nazwa pliku, skala (jak nie wiesz co wpisać, to wpisz 1)

\begin{document}
\maketitle


\kategoria{Wikieł/P3.5}
\zadStart{Zadanie z Wikieł P 3.5 moja wersja nr [nrWersji]}
%[a]:[2,3,4,5,6,7,8,9]
%[b]:[2,3,4,5,6,7,8,9]
%[c]:[200,204,207,209,306,304,309,360,395,350,490,402,406,459,492,454,444,401,503,523,553,556,576,587,536,590,510,522,552,600,645,623,665,680,690,604,678,709,749,766,702,799]
%[a1]=[a]+[b]
%[a1b]=[a1]+[b]
%[c2]=2*[c]
%[c3]=4*[a]*[c2]
%[d]=[a1b]*[a1b]+[c3]
%[d1]=math.sqrt([d]) 
%[d1].is_integer()==True
%[nn]=[d1]-[a1b]
%[nm]=2*[a]
%[n]=[nn]/[nm] 
%[n].is_integer()==True 
%[p]=int([n])
%[d2]=int([d1])
Wyznaczyć liczbę początkowych wyrazów ciągu arytmetycznego $a_{n}=[a]n+[b]$, dających w sumie [c].
\zadStop
\rozwStart{Aleksandra Pasińska}{}
Szukamy takiego n, dla którego $S_{n}=[c]$. Z treści zadania wynika, że 
$$a_{1}=[a1], a_{n}=[a]n+[b]$$ Czyli 
$$\frac{(a_{1}+a_{n})n}{2}=[c]$$ 
$$n([a1]+[a]n+[b])=[c2]$$
$$[a]n^2+[a1b]n-[c2]=0$$
$$\sqrt{\Delta}=[d2]$$
$$n=\frac{-[a1b]+[d2]}{2\cdot [a]}=[p]$$
$$n=[p]$$
Zatem dla n$=[p]$ otrzymujemy $S_{14}=[c]$.
\rozwStop
\odpStart
$n=[p]$
\odpStop
\testStart
A.$n=[p]$
B.$n=2*[p]$
C.$n=0$
D.$n=-6$
E.$n=[p]*[p]$
F.$n=-9$
G.$n=1/2$
H.$n=\frac{1}{7}$
I.$n=-1$
\testStop
\kluczStart
A
\kluczStop



\end{document}