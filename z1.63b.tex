\documentclass[12pt, a4paper]{article}
\usepackage[utf8]{inputenc}
\usepackage{polski}
\usepackage{amsthm}  %pakiet do tworzenia twierdzeń itp.
\usepackage{amsmath} %pakiet do niektórych symboli matematycznych
\usepackage{amssymb} %pakiet do symboli mat., np. \nsubseteq
\usepackage{amsfonts}
\usepackage{graphicx} %obsługa plików graficznych z rozszerzeniem png, jpg
\theoremstyle{definition} %styl dla definicji
\newtheorem{zad}{} 
\title{Multizestaw zadań}
\author{Patryk Wirkus}
%\date{\today}
\date{}
\newcommand{\kategoria}[1]{\section{#1}}
\newcommand{\zadStart}[1]{\begin{zad}#1\newline}
\newcommand{\zadStop}{\end{zad}}
\newcommand{\rozwStart}[2]{\noindent \textbf{Rozwiązanie (autor #1 , recenzent #2): }\newline}
\newcommand{\rozwStop}{\newline}                                           
\newcommand{\odpStart}{\noindent \textbf{Odpowiedź:}\newline}
\newcommand{\odpStop}{\newline}
\newcommand{\testStart}{\noindent \textbf{Test:}\newline}
\newcommand{\testStop}{\newline}
\newcommand{\kluczStart}{\noindent \textbf{Test poprawna odpowiedź:}\newline}
\newcommand{\kluczStop}{\newline}
\newcommand{\wstawGrafike}[2]{\begin{figure}[h] \includegraphics[scale=#2] {#1} \end{figure}}

\begin{document}
\maketitle

\kategoria{Wikieł/Z1.63b}


\zadStart{Zadanie z Wikieł Z 1.63 b) moja wersja nr 1}

Rozwiązać nierówności $x^{64} \ge 1$.
\zadStop
\rozwStart{Patryk Wirkus}{}
$$x^{64} \ge 1 \iff x \in (-\infty,-1] \cup [1,\infty)$$
\rozwStop
\odpStart
$x \in (-\infty,-1] \cup [1.\infty)$
\odpStop
\testStart
A.$x \in (-\infty,-1] \cup [1.\infty)$\\ B.$x \in (-\infty,-1) \cup (1.\infty)$\\ C.$x = 1 \wedge x = -1$\\ D.$x \in (-\infty,-1) \cup [1.\infty)$\\ E.$x \in (-\infty,-1] \cup (1.\infty)$
\testStop
\kluczStart
A
\kluczStop



\zadStart{Zadanie z Wikieł Z 1.63 b) moja wersja nr 2}

Rozwiązać nierówności $x^{68} \ge 1$.
\zadStop
\rozwStart{Patryk Wirkus}{}
$$x^{68} \ge 1 \iff x \in (-\infty,-1] \cup [1,\infty)$$
\rozwStop
\odpStart
$x \in (-\infty,-1] \cup [1.\infty)$
\odpStop
\testStart
A.$x \in (-\infty,-1] \cup [1.\infty)$\\ B.$x \in (-\infty,-1) \cup (1.\infty)$\\ C.$x = 1 \wedge x = -1$\\ D.$x \in (-\infty,-1) \cup [1.\infty)$\\ E.$x \in (-\infty,-1] \cup (1.\infty)$
\testStop
\kluczStart
A
\kluczStop



\zadStart{Zadanie z Wikieł Z 1.63 b) moja wersja nr 3}

Rozwiązać nierówności $x^{72} \ge 1$.
\zadStop
\rozwStart{Patryk Wirkus}{}
$$x^{72} \ge 1 \iff x \in (-\infty,-1] \cup [1,\infty)$$
\rozwStop
\odpStart
$x \in (-\infty,-1] \cup [1.\infty)$
\odpStop
\testStart
A.$x \in (-\infty,-1] \cup [1.\infty)$\\ B.$x \in (-\infty,-1) \cup (1.\infty)$\\ C.$x = 1 \wedge x = -1$\\ D.$x \in (-\infty,-1) \cup [1.\infty)$\\ E.$x \in (-\infty,-1] \cup (1.\infty)$
\testStop
\kluczStart
A
\kluczStop



\zadStart{Zadanie z Wikieł Z 1.63 b) moja wersja nr 4}

Rozwiązać nierówności $x^{78} \ge 1$.
\zadStop
\rozwStart{Patryk Wirkus}{}
$$x^{78} \ge 1 \iff x \in (-\infty,-1] \cup [1,\infty)$$
\rozwStop
\odpStart
$x \in (-\infty,-1] \cup [1.\infty)$
\odpStop
\testStart
A.$x \in (-\infty,-1] \cup [1.\infty)$\\ B.$x \in (-\infty,-1) \cup (1.\infty)$\\ C.$x = 1 \wedge x = -1$\\ D.$x \in (-\infty,-1) \cup [1.\infty)$\\ E.$x \in (-\infty,-1] \cup (1.\infty)$
\testStop
\kluczStart
A
\kluczStop



\zadStart{Zadanie z Wikieł Z 1.63 b) moja wersja nr 5}

Rozwiązać nierówności $x^{52} \ge 1$.
\zadStop
\rozwStart{Patryk Wirkus}{}
$$x^{52} \ge 1 \iff x \in (-\infty,-1] \cup [1,\infty)$$
\rozwStop
\odpStart
$x \in (-\infty,-1] \cup [1.\infty)$
\odpStop
\testStart
A.$x \in (-\infty,-1] \cup [1.\infty)$\\ B.$x \in (-\infty,-1) \cup (1.\infty)$\\ C.$x = 1 \wedge x = -1$\\ D.$x \in (-\infty,-1) \cup [1.\infty)$\\ E.$x \in (-\infty,-1] \cup (1.\infty)$
\testStop
\kluczStart
A
\kluczStop



\zadStart{Zadanie z Wikieł Z 1.63 b) moja wersja nr 6}

Rozwiązać nierówności $x^{56} \ge 1$.
\zadStop
\rozwStart{Patryk Wirkus}{}
$$x^{56} \ge 1 \iff x \in (-\infty,-1] \cup [1,\infty)$$
\rozwStop
\odpStart
$x \in (-\infty,-1] \cup [1.\infty)$
\odpStop
\testStart
A.$x \in (-\infty,-1] \cup [1.\infty)$\\ B.$x \in (-\infty,-1) \cup (1.\infty)$\\ C.$x = 1 \wedge x = -1$\\ D.$x \in (-\infty,-1) \cup [1.\infty)$\\ E.$x \in (-\infty,-1] \cup (1.\infty)$
\testStop
\kluczStart
A
\kluczStop



\zadStart{Zadanie z Wikieł Z 1.63 b) moja wersja nr 7}

Rozwiązać nierówności $x^{62} \ge 1$.
\zadStop
\rozwStart{Patryk Wirkus}{}
$$x^{62} \ge 1 \iff x \in (-\infty,-1] \cup [1,\infty)$$
\rozwStop
\odpStart
$x \in (-\infty,-1] \cup [1.\infty)$
\odpStop
\testStart
A.$x \in (-\infty,-1] \cup [1.\infty)$\\ B.$x \in (-\infty,-1) \cup (1.\infty)$\\ C.$x = 1 \wedge x = -1$\\ D.$x \in (-\infty,-1) \cup [1.\infty)$\\ E.$x \in (-\infty,-1] \cup (1.\infty)$
\testStop
\kluczStart
A
\kluczStop



\zadStart{Zadanie z Wikieł Z 1.63 b) moja wersja nr 8}

Rozwiązać nierówności $x^{74} \ge 1$.
\zadStop
\rozwStart{Patryk Wirkus}{}
$$x^{74} \ge 1 \iff x \in (-\infty,-1] \cup [1,\infty)$$
\rozwStop
\odpStart
$x \in (-\infty,-1] \cup [1.\infty)$
\odpStop
\testStart
A.$x \in (-\infty,-1] \cup [1.\infty)$\\ B.$x \in (-\infty,-1) \cup (1.\infty)$\\ C.$x = 1 \wedge x = -1$\\ D.$x \in (-\infty,-1) \cup [1.\infty)$\\ E.$x \in (-\infty,-1] \cup (1.\infty)$
\testStop
\kluczStart
A
\kluczStop



\zadStart{Zadanie z Wikieł Z 1.63 b) moja wersja nr 9}

Rozwiązać nierówności $x^{76} \ge 1$.
\zadStop
\rozwStart{Patryk Wirkus}{}
$$x^{76} \ge 1 \iff x \in (-\infty,-1] \cup [1,\infty)$$
\rozwStop
\odpStart
$x \in (-\infty,-1] \cup [1.\infty)$
\odpStop
\testStart
A.$x \in (-\infty,-1] \cup [1.\infty)$\\ B.$x \in (-\infty,-1) \cup (1.\infty)$\\ C.$x = 1 \wedge x = -1$\\ D.$x \in (-\infty,-1) \cup [1.\infty)$\\ E.$x \in (-\infty,-1] \cup (1.\infty)$
\testStop
\kluczStart
A
\kluczStop



\zadStart{Zadanie z Wikieł Z 1.63 b) moja wersja nr 10}

Rozwiązać nierówności $x^{84} \ge 1$.
\zadStop
\rozwStart{Patryk Wirkus}{}
$$x^{84} \ge 1 \iff x \in (-\infty,-1] \cup [1,\infty)$$
\rozwStop
\odpStart
$x \in (-\infty,-1] \cup [1.\infty)$
\odpStop
\testStart
A.$x \in (-\infty,-1] \cup [1.\infty)$\\ B.$x \in (-\infty,-1) \cup (1.\infty)$\\ C.$x = 1 \wedge x = -1$\\ D.$x \in (-\infty,-1) \cup [1.\infty)$\\ E.$x \in (-\infty,-1] \cup (1.\infty)$
\testStop
\kluczStart
A
\kluczStop





\end{document}
