\documentclass[12pt, a4paper]{article}
\usepackage[utf8]{inputenc}
\usepackage{polski}
\usepackage{amsthm}  %pakiet do tworzenia twierdzeń itp.
\usepackage{amsmath} %pakiet do niektórych symboli matematycznych
\usepackage{amssymb} %pakiet do symboli mat., np. \nsubseteq
\usepackage{amsfonts}
\usepackage{graphicx} %obsługa plików graficznych z rozszerzeniem png, jpg
\theoremstyle{definition} %styl dla definicji
\newtheorem{zad}{} 
\title{Multizestaw zadań}
\author{Patryk Wirkus}
%\date{\today}
\date{}
\newcommand{\kategoria}[1]{\section{#1}}
\newcommand{\zadStart}[1]{\begin{zad}#1\newline}
\newcommand{\zadStop}{\end{zad}}
\newcommand{\rozwStart}[2]{\noindent \textbf{Rozwiązanie (autor #1 , recenzent #2): }\newline}
\newcommand{\rozwStop}{\newline}                                           
\newcommand{\odpStart}{\noindent \textbf{Odpowiedź:}\newline}
\newcommand{\odpStop}{\newline}
\newcommand{\testStart}{\noindent \textbf{Test:}\newline}
\newcommand{\testStop}{\newline}
\newcommand{\kluczStart}{\noindent \textbf{Test poprawna odpowiedź:}\newline}
\newcommand{\kluczStop}{\newline}
\newcommand{\wstawGrafike}[2]{\begin{figure}[h] \includegraphics[scale=#2] {#1} \end{figure}}

\begin{document}
\maketitle

\kategoria{Wikieł/Z3.18c}


\zadStart{Zadanie z Wikieł Z 3.18 c) moja wersja nr 1}

Zamień poniższe ułamki dziesiętne okresowe na ułamki zwykłe $0,3(10)$.
\zadStop
\rozwStart{Patryk Wirkus}{Martyna Czarnobaj}
$$0,3(10)=0,3101010=0,3+(0,010+0,00010+...)=0,3+\frac{0,010}{1-0,01}$$
$$=0,3+\frac{10}{990}=\frac{3\cdot99+10}{990}$$
\rozwStop
\odpStart
$\frac{3\cdot99+10}{990}$
\odpStop
\testStart
A.$\frac{3\cdot99+10}{9900}$\\ B.$-\frac{3\cdot99+10}{990}$\\ C.$0,3$\\ D.$\frac{10\cdot100}{9900}$
\testStop
\kluczStart
A
\kluczStop



\zadStart{Zadanie z Wikieł Z 3.18 c) moja wersja nr 2}

Zamień poniższe ułamki dziesiętne okresowe na ułamki zwykłe $0,3(20)$.
\zadStop
\rozwStart{Patryk Wirkus}{Martyna Czarnobaj}
$$0,3(20)=0,3202020=0,3+(0,020+0,00020+...)=0,3+\frac{0,020}{1-0,01}$$
$$=0,3+\frac{20}{990}=\frac{3\cdot99+20}{990}$$
\rozwStop
\odpStart
$\frac{3\cdot99+20}{990}$
\odpStop
\testStart
A.$\frac{3\cdot99+20}{9900}$\\ B.$-\frac{3\cdot99+20}{990}$\\ C.$0,3$\\ D.$\frac{20\cdot100}{9900}$
\testStop
\kluczStart
A
\kluczStop



\zadStart{Zadanie z Wikieł Z 3.18 c) moja wersja nr 3}

Zamień poniższe ułamki dziesiętne okresowe na ułamki zwykłe $0,3(21)$.
\zadStop
\rozwStart{Patryk Wirkus}{Martyna Czarnobaj}
$$0,3(21)=0,3212121=0,3+(0,021+0,00021+...)=0,3+\frac{0,021}{1-0,01}$$
$$=0,3+\frac{21}{990}=\frac{3\cdot99+21}{990}$$
\rozwStop
\odpStart
$\frac{3\cdot99+21}{990}$
\odpStop
\testStart
A.$\frac{3\cdot99+21}{9900}$\\ B.$-\frac{3\cdot99+21}{990}$\\ C.$0,3$\\ D.$\frac{21\cdot100}{9900}$
\testStop
\kluczStart
A
\kluczStop



\zadStart{Zadanie z Wikieł Z 3.18 c) moja wersja nr 4}

Zamień poniższe ułamki dziesiętne okresowe na ułamki zwykłe $0,5(10)$.
\zadStop
\rozwStart{Patryk Wirkus}{Martyna Czarnobaj}
$$0,5(10)=0,5101010=0,5+(0,010+0,00010+...)=0,5+\frac{0,010}{1-0,01}$$
$$=0,5+\frac{10}{990}=\frac{5\cdot99+10}{990}$$
\rozwStop
\odpStart
$\frac{5\cdot99+10}{990}$
\odpStop
\testStart
A.$\frac{5\cdot99+10}{9900}$\\ B.$-\frac{5\cdot99+10}{990}$\\ C.$0,5$\\ D.$\frac{10\cdot100}{9900}$
\testStop
\kluczStart
A
\kluczStop



\zadStart{Zadanie z Wikieł Z 3.18 c) moja wersja nr 5}

Zamień poniższe ułamki dziesiętne okresowe na ułamki zwykłe $0,5(20)$.
\zadStop
\rozwStart{Patryk Wirkus}{Martyna Czarnobaj}
$$0,5(20)=0,5202020=0,5+(0,020+0,00020+...)=0,5+\frac{0,020}{1-0,01}$$
$$=0,5+\frac{20}{990}=\frac{5\cdot99+20}{990}$$
\rozwStop
\odpStart
$\frac{5\cdot99+20}{990}$
\odpStop
\testStart
A.$\frac{5\cdot99+20}{9900}$\\ B.$-\frac{5\cdot99+20}{990}$\\ C.$0,5$\\ D.$\frac{20\cdot100}{9900}$
\testStop
\kluczStart
A
\kluczStop



\zadStart{Zadanie z Wikieł Z 3.18 c) moja wersja nr 6}

Zamień poniższe ułamki dziesiętne okresowe na ułamki zwykłe $0,5(21)$.
\zadStop
\rozwStart{Patryk Wirkus}{Martyna Czarnobaj}
$$0,5(21)=0,5212121=0,5+(0,021+0,00021+...)=0,5+\frac{0,021}{1-0,01}$$
$$=0,5+\frac{21}{990}=\frac{5\cdot99+21}{990}$$
\rozwStop
\odpStart
$\frac{5\cdot99+21}{990}$
\odpStop
\testStart
A.$\frac{5\cdot99+21}{9900}$\\ B.$-\frac{5\cdot99+21}{990}$\\ C.$0,5$\\ D.$\frac{21\cdot100}{9900}$
\testStop
\kluczStart
A
\kluczStop



\zadStart{Zadanie z Wikieł Z 3.18 c) moja wersja nr 7}

Zamień poniższe ułamki dziesiętne okresowe na ułamki zwykłe $0,5(30)$.
\zadStop
\rozwStart{Patryk Wirkus}{Martyna Czarnobaj}
$$0,5(30)=0,5303030=0,5+(0,030+0,00030+...)=0,5+\frac{0,030}{1-0,01}$$
$$=0,5+\frac{30}{990}=\frac{5\cdot99+30}{990}$$
\rozwStop
\odpStart
$\frac{5\cdot99+30}{990}$
\odpStop
\testStart
A.$\frac{5\cdot99+30}{9900}$\\ B.$-\frac{5\cdot99+30}{990}$\\ C.$0,5$\\ D.$\frac{30\cdot100}{9900}$
\testStop
\kluczStart
A
\kluczStop



\zadStart{Zadanie z Wikieł Z 3.18 c) moja wersja nr 8}

Zamień poniższe ułamki dziesiętne okresowe na ułamki zwykłe $0,5(31)$.
\zadStop
\rozwStart{Patryk Wirkus}{Martyna Czarnobaj}
$$0,5(31)=0,5313131=0,5+(0,031+0,00031+...)=0,5+\frac{0,031}{1-0,01}$$
$$=0,5+\frac{31}{990}=\frac{5\cdot99+31}{990}$$
\rozwStop
\odpStart
$\frac{5\cdot99+31}{990}$
\odpStop
\testStart
A.$\frac{5\cdot99+31}{9900}$\\ B.$-\frac{5\cdot99+31}{990}$\\ C.$0,5$\\ D.$\frac{31\cdot100}{9900}$
\testStop
\kluczStart
A
\kluczStop



\zadStart{Zadanie z Wikieł Z 3.18 c) moja wersja nr 9}

Zamień poniższe ułamki dziesiętne okresowe na ułamki zwykłe $0,5(32)$.
\zadStop
\rozwStart{Patryk Wirkus}{Martyna Czarnobaj}
$$0,5(32)=0,5323232=0,5+(0,032+0,00032+...)=0,5+\frac{0,032}{1-0,01}$$
$$=0,5+\frac{32}{990}=\frac{5\cdot99+32}{990}$$
\rozwStop
\odpStart
$\frac{5\cdot99+32}{990}$
\odpStop
\testStart
A.$\frac{5\cdot99+32}{9900}$\\ B.$-\frac{5\cdot99+32}{990}$\\ C.$0,5$\\ D.$\frac{32\cdot100}{9900}$
\testStop
\kluczStart
A
\kluczStop



\zadStart{Zadanie z Wikieł Z 3.18 c) moja wersja nr 10}

Zamień poniższe ułamki dziesiętne okresowe na ułamki zwykłe $0,5(40)$.
\zadStop
\rozwStart{Patryk Wirkus}{Martyna Czarnobaj}
$$0,5(40)=0,5404040=0,5+(0,040+0,00040+...)=0,5+\frac{0,040}{1-0,01}$$
$$=0,5+\frac{40}{990}=\frac{5\cdot99+40}{990}$$
\rozwStop
\odpStart
$\frac{5\cdot99+40}{990}$
\odpStop
\testStart
A.$\frac{5\cdot99+40}{9900}$\\ B.$-\frac{5\cdot99+40}{990}$\\ C.$0,5$\\ D.$\frac{40\cdot100}{9900}$
\testStop
\kluczStart
A
\kluczStop



\zadStart{Zadanie z Wikieł Z 3.18 c) moja wersja nr 11}

Zamień poniższe ułamki dziesiętne okresowe na ułamki zwykłe $0,5(41)$.
\zadStop
\rozwStart{Patryk Wirkus}{Martyna Czarnobaj}
$$0,5(41)=0,5414141=0,5+(0,041+0,00041+...)=0,5+\frac{0,041}{1-0,01}$$
$$=0,5+\frac{41}{990}=\frac{5\cdot99+41}{990}$$
\rozwStop
\odpStart
$\frac{5\cdot99+41}{990}$
\odpStop
\testStart
A.$\frac{5\cdot99+41}{9900}$\\ B.$-\frac{5\cdot99+41}{990}$\\ C.$0,5$\\ D.$\frac{41\cdot100}{9900}$
\testStop
\kluczStart
A
\kluczStop



\zadStart{Zadanie z Wikieł Z 3.18 c) moja wersja nr 12}

Zamień poniższe ułamki dziesiętne okresowe na ułamki zwykłe $0,5(42)$.
\zadStop
\rozwStart{Patryk Wirkus}{Martyna Czarnobaj}
$$0,5(42)=0,5424242=0,5+(0,042+0,00042+...)=0,5+\frac{0,042}{1-0,01}$$
$$=0,5+\frac{42}{990}=\frac{5\cdot99+42}{990}$$
\rozwStop
\odpStart
$\frac{5\cdot99+42}{990}$
\odpStop
\testStart
A.$\frac{5\cdot99+42}{9900}$\\ B.$-\frac{5\cdot99+42}{990}$\\ C.$0,5$\\ D.$\frac{42\cdot100}{9900}$
\testStop
\kluczStart
A
\kluczStop



\zadStart{Zadanie z Wikieł Z 3.18 c) moja wersja nr 13}

Zamień poniższe ułamki dziesiętne okresowe na ułamki zwykłe $0,5(43)$.
\zadStop
\rozwStart{Patryk Wirkus}{Martyna Czarnobaj}
$$0,5(43)=0,5434343=0,5+(0,043+0,00043+...)=0,5+\frac{0,043}{1-0,01}$$
$$=0,5+\frac{43}{990}=\frac{5\cdot99+43}{990}$$
\rozwStop
\odpStart
$\frac{5\cdot99+43}{990}$
\odpStop
\testStart
A.$\frac{5\cdot99+43}{9900}$\\ B.$-\frac{5\cdot99+43}{990}$\\ C.$0,5$\\ D.$\frac{43\cdot100}{9900}$
\testStop
\kluczStart
A
\kluczStop



\zadStart{Zadanie z Wikieł Z 3.18 c) moja wersja nr 14}

Zamień poniższe ułamki dziesiętne okresowe na ułamki zwykłe $0,7(10)$.
\zadStop
\rozwStart{Patryk Wirkus}{Martyna Czarnobaj}
$$0,7(10)=0,7101010=0,7+(0,010+0,00010+...)=0,7+\frac{0,010}{1-0,01}$$
$$=0,7+\frac{10}{990}=\frac{7\cdot99+10}{990}$$
\rozwStop
\odpStart
$\frac{7\cdot99+10}{990}$
\odpStop
\testStart
A.$\frac{7\cdot99+10}{9900}$\\ B.$-\frac{7\cdot99+10}{990}$\\ C.$0,7$\\ D.$\frac{10\cdot100}{9900}$
\testStop
\kluczStart
A
\kluczStop



\zadStart{Zadanie z Wikieł Z 3.18 c) moja wersja nr 15}

Zamień poniższe ułamki dziesiętne okresowe na ułamki zwykłe $0,7(20)$.
\zadStop
\rozwStart{Patryk Wirkus}{Martyna Czarnobaj}
$$0,7(20)=0,7202020=0,7+(0,020+0,00020+...)=0,7+\frac{0,020}{1-0,01}$$
$$=0,7+\frac{20}{990}=\frac{7\cdot99+20}{990}$$
\rozwStop
\odpStart
$\frac{7\cdot99+20}{990}$
\odpStop
\testStart
A.$\frac{7\cdot99+20}{9900}$\\ B.$-\frac{7\cdot99+20}{990}$\\ C.$0,7$\\ D.$\frac{20\cdot100}{9900}$
\testStop
\kluczStart
A
\kluczStop



\zadStart{Zadanie z Wikieł Z 3.18 c) moja wersja nr 16}

Zamień poniższe ułamki dziesiętne okresowe na ułamki zwykłe $0,7(21)$.
\zadStop
\rozwStart{Patryk Wirkus}{Martyna Czarnobaj}
$$0,7(21)=0,7212121=0,7+(0,021+0,00021+...)=0,7+\frac{0,021}{1-0,01}$$
$$=0,7+\frac{21}{990}=\frac{7\cdot99+21}{990}$$
\rozwStop
\odpStart
$\frac{7\cdot99+21}{990}$
\odpStop
\testStart
A.$\frac{7\cdot99+21}{9900}$\\ B.$-\frac{7\cdot99+21}{990}$\\ C.$0,7$\\ D.$\frac{21\cdot100}{9900}$
\testStop
\kluczStart
A
\kluczStop



\zadStart{Zadanie z Wikieł Z 3.18 c) moja wersja nr 17}

Zamień poniższe ułamki dziesiętne okresowe na ułamki zwykłe $0,7(30)$.
\zadStop
\rozwStart{Patryk Wirkus}{Martyna Czarnobaj}
$$0,7(30)=0,7303030=0,7+(0,030+0,00030+...)=0,7+\frac{0,030}{1-0,01}$$
$$=0,7+\frac{30}{990}=\frac{7\cdot99+30}{990}$$
\rozwStop
\odpStart
$\frac{7\cdot99+30}{990}$
\odpStop
\testStart
A.$\frac{7\cdot99+30}{9900}$\\ B.$-\frac{7\cdot99+30}{990}$\\ C.$0,7$\\ D.$\frac{30\cdot100}{9900}$
\testStop
\kluczStart
A
\kluczStop



\zadStart{Zadanie z Wikieł Z 3.18 c) moja wersja nr 18}

Zamień poniższe ułamki dziesiętne okresowe na ułamki zwykłe $0,7(31)$.
\zadStop
\rozwStart{Patryk Wirkus}{Martyna Czarnobaj}
$$0,7(31)=0,7313131=0,7+(0,031+0,00031+...)=0,7+\frac{0,031}{1-0,01}$$
$$=0,7+\frac{31}{990}=\frac{7\cdot99+31}{990}$$
\rozwStop
\odpStart
$\frac{7\cdot99+31}{990}$
\odpStop
\testStart
A.$\frac{7\cdot99+31}{9900}$\\ B.$-\frac{7\cdot99+31}{990}$\\ C.$0,7$\\ D.$\frac{31\cdot100}{9900}$
\testStop
\kluczStart
A
\kluczStop



\zadStart{Zadanie z Wikieł Z 3.18 c) moja wersja nr 19}

Zamień poniższe ułamki dziesiętne okresowe na ułamki zwykłe $0,7(32)$.
\zadStop
\rozwStart{Patryk Wirkus}{Martyna Czarnobaj}
$$0,7(32)=0,7323232=0,7+(0,032+0,00032+...)=0,7+\frac{0,032}{1-0,01}$$
$$=0,7+\frac{32}{990}=\frac{7\cdot99+32}{990}$$
\rozwStop
\odpStart
$\frac{7\cdot99+32}{990}$
\odpStop
\testStart
A.$\frac{7\cdot99+32}{9900}$\\ B.$-\frac{7\cdot99+32}{990}$\\ C.$0,7$\\ D.$\frac{32\cdot100}{9900}$
\testStop
\kluczStart
A
\kluczStop



\zadStart{Zadanie z Wikieł Z 3.18 c) moja wersja nr 20}

Zamień poniższe ułamki dziesiętne okresowe na ułamki zwykłe $0,7(40)$.
\zadStop
\rozwStart{Patryk Wirkus}{Martyna Czarnobaj}
$$0,7(40)=0,7404040=0,7+(0,040+0,00040+...)=0,7+\frac{0,040}{1-0,01}$$
$$=0,7+\frac{40}{990}=\frac{7\cdot99+40}{990}$$
\rozwStop
\odpStart
$\frac{7\cdot99+40}{990}$
\odpStop
\testStart
A.$\frac{7\cdot99+40}{9900}$\\ B.$-\frac{7\cdot99+40}{990}$\\ C.$0,7$\\ D.$\frac{40\cdot100}{9900}$
\testStop
\kluczStart
A
\kluczStop



\zadStart{Zadanie z Wikieł Z 3.18 c) moja wersja nr 21}

Zamień poniższe ułamki dziesiętne okresowe na ułamki zwykłe $0,7(41)$.
\zadStop
\rozwStart{Patryk Wirkus}{Martyna Czarnobaj}
$$0,7(41)=0,7414141=0,7+(0,041+0,00041+...)=0,7+\frac{0,041}{1-0,01}$$
$$=0,7+\frac{41}{990}=\frac{7\cdot99+41}{990}$$
\rozwStop
\odpStart
$\frac{7\cdot99+41}{990}$
\odpStop
\testStart
A.$\frac{7\cdot99+41}{9900}$\\ B.$-\frac{7\cdot99+41}{990}$\\ C.$0,7$\\ D.$\frac{41\cdot100}{9900}$
\testStop
\kluczStart
A
\kluczStop



\zadStart{Zadanie z Wikieł Z 3.18 c) moja wersja nr 22}

Zamień poniższe ułamki dziesiętne okresowe na ułamki zwykłe $0,7(42)$.
\zadStop
\rozwStart{Patryk Wirkus}{Martyna Czarnobaj}
$$0,7(42)=0,7424242=0,7+(0,042+0,00042+...)=0,7+\frac{0,042}{1-0,01}$$
$$=0,7+\frac{42}{990}=\frac{7\cdot99+42}{990}$$
\rozwStop
\odpStart
$\frac{7\cdot99+42}{990}$
\odpStop
\testStart
A.$\frac{7\cdot99+42}{9900}$\\ B.$-\frac{7\cdot99+42}{990}$\\ C.$0,7$\\ D.$\frac{42\cdot100}{9900}$
\testStop
\kluczStart
A
\kluczStop



\zadStart{Zadanie z Wikieł Z 3.18 c) moja wersja nr 23}

Zamień poniższe ułamki dziesiętne okresowe na ułamki zwykłe $0,7(43)$.
\zadStop
\rozwStart{Patryk Wirkus}{Martyna Czarnobaj}
$$0,7(43)=0,7434343=0,7+(0,043+0,00043+...)=0,7+\frac{0,043}{1-0,01}$$
$$=0,7+\frac{43}{990}=\frac{7\cdot99+43}{990}$$
\rozwStop
\odpStart
$\frac{7\cdot99+43}{990}$
\odpStop
\testStart
A.$\frac{7\cdot99+43}{9900}$\\ B.$-\frac{7\cdot99+43}{990}$\\ C.$0,7$\\ D.$\frac{43\cdot100}{9900}$
\testStop
\kluczStart
A
\kluczStop



\zadStart{Zadanie z Wikieł Z 3.18 c) moja wersja nr 24}

Zamień poniższe ułamki dziesiętne okresowe na ułamki zwykłe $0,7(50)$.
\zadStop
\rozwStart{Patryk Wirkus}{Martyna Czarnobaj}
$$0,7(50)=0,7505050=0,7+(0,050+0,00050+...)=0,7+\frac{0,050}{1-0,01}$$
$$=0,7+\frac{50}{990}=\frac{7\cdot99+50}{990}$$
\rozwStop
\odpStart
$\frac{7\cdot99+50}{990}$
\odpStop
\testStart
A.$\frac{7\cdot99+50}{9900}$\\ B.$-\frac{7\cdot99+50}{990}$\\ C.$0,7$\\ D.$\frac{50\cdot100}{9900}$
\testStop
\kluczStart
A
\kluczStop



\zadStart{Zadanie z Wikieł Z 3.18 c) moja wersja nr 25}

Zamień poniższe ułamki dziesiętne okresowe na ułamki zwykłe $0,7(51)$.
\zadStop
\rozwStart{Patryk Wirkus}{Martyna Czarnobaj}
$$0,7(51)=0,7515151=0,7+(0,051+0,00051+...)=0,7+\frac{0,051}{1-0,01}$$
$$=0,7+\frac{51}{990}=\frac{7\cdot99+51}{990}$$
\rozwStop
\odpStart
$\frac{7\cdot99+51}{990}$
\odpStop
\testStart
A.$\frac{7\cdot99+51}{9900}$\\ B.$-\frac{7\cdot99+51}{990}$\\ C.$0,7$\\ D.$\frac{51\cdot100}{9900}$
\testStop
\kluczStart
A
\kluczStop



\zadStart{Zadanie z Wikieł Z 3.18 c) moja wersja nr 26}

Zamień poniższe ułamki dziesiętne okresowe na ułamki zwykłe $0,7(52)$.
\zadStop
\rozwStart{Patryk Wirkus}{Martyna Czarnobaj}
$$0,7(52)=0,7525252=0,7+(0,052+0,00052+...)=0,7+\frac{0,052}{1-0,01}$$
$$=0,7+\frac{52}{990}=\frac{7\cdot99+52}{990}$$
\rozwStop
\odpStart
$\frac{7\cdot99+52}{990}$
\odpStop
\testStart
A.$\frac{7\cdot99+52}{9900}$\\ B.$-\frac{7\cdot99+52}{990}$\\ C.$0,7$\\ D.$\frac{52\cdot100}{9900}$
\testStop
\kluczStart
A
\kluczStop



\zadStart{Zadanie z Wikieł Z 3.18 c) moja wersja nr 27}

Zamień poniższe ułamki dziesiętne okresowe na ułamki zwykłe $0,7(53)$.
\zadStop
\rozwStart{Patryk Wirkus}{Martyna Czarnobaj}
$$0,7(53)=0,7535353=0,7+(0,053+0,00053+...)=0,7+\frac{0,053}{1-0,01}$$
$$=0,7+\frac{53}{990}=\frac{7\cdot99+53}{990}$$
\rozwStop
\odpStart
$\frac{7\cdot99+53}{990}$
\odpStop
\testStart
A.$\frac{7\cdot99+53}{9900}$\\ B.$-\frac{7\cdot99+53}{990}$\\ C.$0,7$\\ D.$\frac{53\cdot100}{9900}$
\testStop
\kluczStart
A
\kluczStop



\zadStart{Zadanie z Wikieł Z 3.18 c) moja wersja nr 28}

Zamień poniższe ułamki dziesiętne okresowe na ułamki zwykłe $0,7(54)$.
\zadStop
\rozwStart{Patryk Wirkus}{Martyna Czarnobaj}
$$0,7(54)=0,7545454=0,7+(0,054+0,00054+...)=0,7+\frac{0,054}{1-0,01}$$
$$=0,7+\frac{54}{990}=\frac{7\cdot99+54}{990}$$
\rozwStop
\odpStart
$\frac{7\cdot99+54}{990}$
\odpStop
\testStart
A.$\frac{7\cdot99+54}{9900}$\\ B.$-\frac{7\cdot99+54}{990}$\\ C.$0,7$\\ D.$\frac{54\cdot100}{9900}$
\testStop
\kluczStart
A
\kluczStop



\zadStart{Zadanie z Wikieł Z 3.18 c) moja wersja nr 29}

Zamień poniższe ułamki dziesiętne okresowe na ułamki zwykłe $0,7(60)$.
\zadStop
\rozwStart{Patryk Wirkus}{Martyna Czarnobaj}
$$0,7(60)=0,7606060=0,7+(0,060+0,00060+...)=0,7+\frac{0,060}{1-0,01}$$
$$=0,7+\frac{60}{990}=\frac{7\cdot99+60}{990}$$
\rozwStop
\odpStart
$\frac{7\cdot99+60}{990}$
\odpStop
\testStart
A.$\frac{7\cdot99+60}{9900}$\\ B.$-\frac{7\cdot99+60}{990}$\\ C.$0,7$\\ D.$\frac{60\cdot100}{9900}$
\testStop
\kluczStart
A
\kluczStop



\zadStart{Zadanie z Wikieł Z 3.18 c) moja wersja nr 30}

Zamień poniższe ułamki dziesiętne okresowe na ułamki zwykłe $0,7(61)$.
\zadStop
\rozwStart{Patryk Wirkus}{Martyna Czarnobaj}
$$0,7(61)=0,7616161=0,7+(0,061+0,00061+...)=0,7+\frac{0,061}{1-0,01}$$
$$=0,7+\frac{61}{990}=\frac{7\cdot99+61}{990}$$
\rozwStop
\odpStart
$\frac{7\cdot99+61}{990}$
\odpStop
\testStart
A.$\frac{7\cdot99+61}{9900}$\\ B.$-\frac{7\cdot99+61}{990}$\\ C.$0,7$\\ D.$\frac{61\cdot100}{9900}$
\testStop
\kluczStart
A
\kluczStop



\zadStart{Zadanie z Wikieł Z 3.18 c) moja wersja nr 31}

Zamień poniższe ułamki dziesiętne okresowe na ułamki zwykłe $0,7(62)$.
\zadStop
\rozwStart{Patryk Wirkus}{Martyna Czarnobaj}
$$0,7(62)=0,7626262=0,7+(0,062+0,00062+...)=0,7+\frac{0,062}{1-0,01}$$
$$=0,7+\frac{62}{990}=\frac{7\cdot99+62}{990}$$
\rozwStop
\odpStart
$\frac{7\cdot99+62}{990}$
\odpStop
\testStart
A.$\frac{7\cdot99+62}{9900}$\\ B.$-\frac{7\cdot99+62}{990}$\\ C.$0,7$\\ D.$\frac{62\cdot100}{9900}$
\testStop
\kluczStart
A
\kluczStop



\zadStart{Zadanie z Wikieł Z 3.18 c) moja wersja nr 32}

Zamień poniższe ułamki dziesiętne okresowe na ułamki zwykłe $0,7(63)$.
\zadStop
\rozwStart{Patryk Wirkus}{Martyna Czarnobaj}
$$0,7(63)=0,7636363=0,7+(0,063+0,00063+...)=0,7+\frac{0,063}{1-0,01}$$
$$=0,7+\frac{63}{990}=\frac{7\cdot99+63}{990}$$
\rozwStop
\odpStart
$\frac{7\cdot99+63}{990}$
\odpStop
\testStart
A.$\frac{7\cdot99+63}{9900}$\\ B.$-\frac{7\cdot99+63}{990}$\\ C.$0,7$\\ D.$\frac{63\cdot100}{9900}$
\testStop
\kluczStart
A
\kluczStop



\zadStart{Zadanie z Wikieł Z 3.18 c) moja wersja nr 33}

Zamień poniższe ułamki dziesiętne okresowe na ułamki zwykłe $0,7(64)$.
\zadStop
\rozwStart{Patryk Wirkus}{Martyna Czarnobaj}
$$0,7(64)=0,7646464=0,7+(0,064+0,00064+...)=0,7+\frac{0,064}{1-0,01}$$
$$=0,7+\frac{64}{990}=\frac{7\cdot99+64}{990}$$
\rozwStop
\odpStart
$\frac{7\cdot99+64}{990}$
\odpStop
\testStart
A.$\frac{7\cdot99+64}{9900}$\\ B.$-\frac{7\cdot99+64}{990}$\\ C.$0,7$\\ D.$\frac{64\cdot100}{9900}$
\testStop
\kluczStart
A
\kluczStop



\zadStart{Zadanie z Wikieł Z 3.18 c) moja wersja nr 34}

Zamień poniższe ułamki dziesiętne okresowe na ułamki zwykłe $0,7(65)$.
\zadStop
\rozwStart{Patryk Wirkus}{Martyna Czarnobaj}
$$0,7(65)=0,7656565=0,7+(0,065+0,00065+...)=0,7+\frac{0,065}{1-0,01}$$
$$=0,7+\frac{65}{990}=\frac{7\cdot99+65}{990}$$
\rozwStop
\odpStart
$\frac{7\cdot99+65}{990}$
\odpStop
\testStart
A.$\frac{7\cdot99+65}{9900}$\\ B.$-\frac{7\cdot99+65}{990}$\\ C.$0,7$\\ D.$\frac{65\cdot100}{9900}$
\testStop
\kluczStart
A
\kluczStop



\zadStart{Zadanie z Wikieł Z 3.18 c) moja wersja nr 35}

Zamień poniższe ułamki dziesiętne okresowe na ułamki zwykłe $0,9(10)$.
\zadStop
\rozwStart{Patryk Wirkus}{Martyna Czarnobaj}
$$0,9(10)=0,9101010=0,9+(0,010+0,00010+...)=0,9+\frac{0,010}{1-0,01}$$
$$=0,9+\frac{10}{990}=\frac{9\cdot99+10}{990}$$
\rozwStop
\odpStart
$\frac{9\cdot99+10}{990}$
\odpStop
\testStart
A.$\frac{9\cdot99+10}{9900}$\\ B.$-\frac{9\cdot99+10}{990}$\\ C.$0,9$\\ D.$\frac{10\cdot100}{9900}$
\testStop
\kluczStart
A
\kluczStop



\zadStart{Zadanie z Wikieł Z 3.18 c) moja wersja nr 36}

Zamień poniższe ułamki dziesiętne okresowe na ułamki zwykłe $0,9(20)$.
\zadStop
\rozwStart{Patryk Wirkus}{Martyna Czarnobaj}
$$0,9(20)=0,9202020=0,9+(0,020+0,00020+...)=0,9+\frac{0,020}{1-0,01}$$
$$=0,9+\frac{20}{990}=\frac{9\cdot99+20}{990}$$
\rozwStop
\odpStart
$\frac{9\cdot99+20}{990}$
\odpStop
\testStart
A.$\frac{9\cdot99+20}{9900}$\\ B.$-\frac{9\cdot99+20}{990}$\\ C.$0,9$\\ D.$\frac{20\cdot100}{9900}$
\testStop
\kluczStart
A
\kluczStop



\zadStart{Zadanie z Wikieł Z 3.18 c) moja wersja nr 37}

Zamień poniższe ułamki dziesiętne okresowe na ułamki zwykłe $0,9(21)$.
\zadStop
\rozwStart{Patryk Wirkus}{Martyna Czarnobaj}
$$0,9(21)=0,9212121=0,9+(0,021+0,00021+...)=0,9+\frac{0,021}{1-0,01}$$
$$=0,9+\frac{21}{990}=\frac{9\cdot99+21}{990}$$
\rozwStop
\odpStart
$\frac{9\cdot99+21}{990}$
\odpStop
\testStart
A.$\frac{9\cdot99+21}{9900}$\\ B.$-\frac{9\cdot99+21}{990}$\\ C.$0,9$\\ D.$\frac{21\cdot100}{9900}$
\testStop
\kluczStart
A
\kluczStop



\zadStart{Zadanie z Wikieł Z 3.18 c) moja wersja nr 38}

Zamień poniższe ułamki dziesiętne okresowe na ułamki zwykłe $0,9(30)$.
\zadStop
\rozwStart{Patryk Wirkus}{Martyna Czarnobaj}
$$0,9(30)=0,9303030=0,9+(0,030+0,00030+...)=0,9+\frac{0,030}{1-0,01}$$
$$=0,9+\frac{30}{990}=\frac{9\cdot99+30}{990}$$
\rozwStop
\odpStart
$\frac{9\cdot99+30}{990}$
\odpStop
\testStart
A.$\frac{9\cdot99+30}{9900}$\\ B.$-\frac{9\cdot99+30}{990}$\\ C.$0,9$\\ D.$\frac{30\cdot100}{9900}$
\testStop
\kluczStart
A
\kluczStop



\zadStart{Zadanie z Wikieł Z 3.18 c) moja wersja nr 39}

Zamień poniższe ułamki dziesiętne okresowe na ułamki zwykłe $0,9(31)$.
\zadStop
\rozwStart{Patryk Wirkus}{Martyna Czarnobaj}
$$0,9(31)=0,9313131=0,9+(0,031+0,00031+...)=0,9+\frac{0,031}{1-0,01}$$
$$=0,9+\frac{31}{990}=\frac{9\cdot99+31}{990}$$
\rozwStop
\odpStart
$\frac{9\cdot99+31}{990}$
\odpStop
\testStart
A.$\frac{9\cdot99+31}{9900}$\\ B.$-\frac{9\cdot99+31}{990}$\\ C.$0,9$\\ D.$\frac{31\cdot100}{9900}$
\testStop
\kluczStart
A
\kluczStop



\zadStart{Zadanie z Wikieł Z 3.18 c) moja wersja nr 40}

Zamień poniższe ułamki dziesiętne okresowe na ułamki zwykłe $0,9(32)$.
\zadStop
\rozwStart{Patryk Wirkus}{Martyna Czarnobaj}
$$0,9(32)=0,9323232=0,9+(0,032+0,00032+...)=0,9+\frac{0,032}{1-0,01}$$
$$=0,9+\frac{32}{990}=\frac{9\cdot99+32}{990}$$
\rozwStop
\odpStart
$\frac{9\cdot99+32}{990}$
\odpStop
\testStart
A.$\frac{9\cdot99+32}{9900}$\\ B.$-\frac{9\cdot99+32}{990}$\\ C.$0,9$\\ D.$\frac{32\cdot100}{9900}$
\testStop
\kluczStart
A
\kluczStop



\zadStart{Zadanie z Wikieł Z 3.18 c) moja wersja nr 41}

Zamień poniższe ułamki dziesiętne okresowe na ułamki zwykłe $0,9(40)$.
\zadStop
\rozwStart{Patryk Wirkus}{Martyna Czarnobaj}
$$0,9(40)=0,9404040=0,9+(0,040+0,00040+...)=0,9+\frac{0,040}{1-0,01}$$
$$=0,9+\frac{40}{990}=\frac{9\cdot99+40}{990}$$
\rozwStop
\odpStart
$\frac{9\cdot99+40}{990}$
\odpStop
\testStart
A.$\frac{9\cdot99+40}{9900}$\\ B.$-\frac{9\cdot99+40}{990}$\\ C.$0,9$\\ D.$\frac{40\cdot100}{9900}$
\testStop
\kluczStart
A
\kluczStop



\zadStart{Zadanie z Wikieł Z 3.18 c) moja wersja nr 42}

Zamień poniższe ułamki dziesiętne okresowe na ułamki zwykłe $0,9(41)$.
\zadStop
\rozwStart{Patryk Wirkus}{Martyna Czarnobaj}
$$0,9(41)=0,9414141=0,9+(0,041+0,00041+...)=0,9+\frac{0,041}{1-0,01}$$
$$=0,9+\frac{41}{990}=\frac{9\cdot99+41}{990}$$
\rozwStop
\odpStart
$\frac{9\cdot99+41}{990}$
\odpStop
\testStart
A.$\frac{9\cdot99+41}{9900}$\\ B.$-\frac{9\cdot99+41}{990}$\\ C.$0,9$\\ D.$\frac{41\cdot100}{9900}$
\testStop
\kluczStart
A
\kluczStop



\zadStart{Zadanie z Wikieł Z 3.18 c) moja wersja nr 43}

Zamień poniższe ułamki dziesiętne okresowe na ułamki zwykłe $0,9(42)$.
\zadStop
\rozwStart{Patryk Wirkus}{Martyna Czarnobaj}
$$0,9(42)=0,9424242=0,9+(0,042+0,00042+...)=0,9+\frac{0,042}{1-0,01}$$
$$=0,9+\frac{42}{990}=\frac{9\cdot99+42}{990}$$
\rozwStop
\odpStart
$\frac{9\cdot99+42}{990}$
\odpStop
\testStart
A.$\frac{9\cdot99+42}{9900}$\\ B.$-\frac{9\cdot99+42}{990}$\\ C.$0,9$\\ D.$\frac{42\cdot100}{9900}$
\testStop
\kluczStart
A
\kluczStop



\zadStart{Zadanie z Wikieł Z 3.18 c) moja wersja nr 44}

Zamień poniższe ułamki dziesiętne okresowe na ułamki zwykłe $0,9(43)$.
\zadStop
\rozwStart{Patryk Wirkus}{Martyna Czarnobaj}
$$0,9(43)=0,9434343=0,9+(0,043+0,00043+...)=0,9+\frac{0,043}{1-0,01}$$
$$=0,9+\frac{43}{990}=\frac{9\cdot99+43}{990}$$
\rozwStop
\odpStart
$\frac{9\cdot99+43}{990}$
\odpStop
\testStart
A.$\frac{9\cdot99+43}{9900}$\\ B.$-\frac{9\cdot99+43}{990}$\\ C.$0,9$\\ D.$\frac{43\cdot100}{9900}$
\testStop
\kluczStart
A
\kluczStop



\zadStart{Zadanie z Wikieł Z 3.18 c) moja wersja nr 45}

Zamień poniższe ułamki dziesiętne okresowe na ułamki zwykłe $0,9(50)$.
\zadStop
\rozwStart{Patryk Wirkus}{Martyna Czarnobaj}
$$0,9(50)=0,9505050=0,9+(0,050+0,00050+...)=0,9+\frac{0,050}{1-0,01}$$
$$=0,9+\frac{50}{990}=\frac{9\cdot99+50}{990}$$
\rozwStop
\odpStart
$\frac{9\cdot99+50}{990}$
\odpStop
\testStart
A.$\frac{9\cdot99+50}{9900}$\\ B.$-\frac{9\cdot99+50}{990}$\\ C.$0,9$\\ D.$\frac{50\cdot100}{9900}$
\testStop
\kluczStart
A
\kluczStop



\zadStart{Zadanie z Wikieł Z 3.18 c) moja wersja nr 46}

Zamień poniższe ułamki dziesiętne okresowe na ułamki zwykłe $0,9(51)$.
\zadStop
\rozwStart{Patryk Wirkus}{Martyna Czarnobaj}
$$0,9(51)=0,9515151=0,9+(0,051+0,00051+...)=0,9+\frac{0,051}{1-0,01}$$
$$=0,9+\frac{51}{990}=\frac{9\cdot99+51}{990}$$
\rozwStop
\odpStart
$\frac{9\cdot99+51}{990}$
\odpStop
\testStart
A.$\frac{9\cdot99+51}{9900}$\\ B.$-\frac{9\cdot99+51}{990}$\\ C.$0,9$\\ D.$\frac{51\cdot100}{9900}$
\testStop
\kluczStart
A
\kluczStop



\zadStart{Zadanie z Wikieł Z 3.18 c) moja wersja nr 47}

Zamień poniższe ułamki dziesiętne okresowe na ułamki zwykłe $0,9(52)$.
\zadStop
\rozwStart{Patryk Wirkus}{Martyna Czarnobaj}
$$0,9(52)=0,9525252=0,9+(0,052+0,00052+...)=0,9+\frac{0,052}{1-0,01}$$
$$=0,9+\frac{52}{990}=\frac{9\cdot99+52}{990}$$
\rozwStop
\odpStart
$\frac{9\cdot99+52}{990}$
\odpStop
\testStart
A.$\frac{9\cdot99+52}{9900}$\\ B.$-\frac{9\cdot99+52}{990}$\\ C.$0,9$\\ D.$\frac{52\cdot100}{9900}$
\testStop
\kluczStart
A
\kluczStop



\zadStart{Zadanie z Wikieł Z 3.18 c) moja wersja nr 48}

Zamień poniższe ułamki dziesiętne okresowe na ułamki zwykłe $0,9(53)$.
\zadStop
\rozwStart{Patryk Wirkus}{Martyna Czarnobaj}
$$0,9(53)=0,9535353=0,9+(0,053+0,00053+...)=0,9+\frac{0,053}{1-0,01}$$
$$=0,9+\frac{53}{990}=\frac{9\cdot99+53}{990}$$
\rozwStop
\odpStart
$\frac{9\cdot99+53}{990}$
\odpStop
\testStart
A.$\frac{9\cdot99+53}{9900}$\\ B.$-\frac{9\cdot99+53}{990}$\\ C.$0,9$\\ D.$\frac{53\cdot100}{9900}$
\testStop
\kluczStart
A
\kluczStop



\zadStart{Zadanie z Wikieł Z 3.18 c) moja wersja nr 49}

Zamień poniższe ułamki dziesiętne okresowe na ułamki zwykłe $0,9(54)$.
\zadStop
\rozwStart{Patryk Wirkus}{Martyna Czarnobaj}
$$0,9(54)=0,9545454=0,9+(0,054+0,00054+...)=0,9+\frac{0,054}{1-0,01}$$
$$=0,9+\frac{54}{990}=\frac{9\cdot99+54}{990}$$
\rozwStop
\odpStart
$\frac{9\cdot99+54}{990}$
\odpStop
\testStart
A.$\frac{9\cdot99+54}{9900}$\\ B.$-\frac{9\cdot99+54}{990}$\\ C.$0,9$\\ D.$\frac{54\cdot100}{9900}$
\testStop
\kluczStart
A
\kluczStop



\zadStart{Zadanie z Wikieł Z 3.18 c) moja wersja nr 50}

Zamień poniższe ułamki dziesiętne okresowe na ułamki zwykłe $0,9(60)$.
\zadStop
\rozwStart{Patryk Wirkus}{Martyna Czarnobaj}
$$0,9(60)=0,9606060=0,9+(0,060+0,00060+...)=0,9+\frac{0,060}{1-0,01}$$
$$=0,9+\frac{60}{990}=\frac{9\cdot99+60}{990}$$
\rozwStop
\odpStart
$\frac{9\cdot99+60}{990}$
\odpStop
\testStart
A.$\frac{9\cdot99+60}{9900}$\\ B.$-\frac{9\cdot99+60}{990}$\\ C.$0,9$\\ D.$\frac{60\cdot100}{9900}$
\testStop
\kluczStart
A
\kluczStop



\zadStart{Zadanie z Wikieł Z 3.18 c) moja wersja nr 51}

Zamień poniższe ułamki dziesiętne okresowe na ułamki zwykłe $0,9(61)$.
\zadStop
\rozwStart{Patryk Wirkus}{Martyna Czarnobaj}
$$0,9(61)=0,9616161=0,9+(0,061+0,00061+...)=0,9+\frac{0,061}{1-0,01}$$
$$=0,9+\frac{61}{990}=\frac{9\cdot99+61}{990}$$
\rozwStop
\odpStart
$\frac{9\cdot99+61}{990}$
\odpStop
\testStart
A.$\frac{9\cdot99+61}{9900}$\\ B.$-\frac{9\cdot99+61}{990}$\\ C.$0,9$\\ D.$\frac{61\cdot100}{9900}$
\testStop
\kluczStart
A
\kluczStop



\zadStart{Zadanie z Wikieł Z 3.18 c) moja wersja nr 52}

Zamień poniższe ułamki dziesiętne okresowe na ułamki zwykłe $0,9(62)$.
\zadStop
\rozwStart{Patryk Wirkus}{Martyna Czarnobaj}
$$0,9(62)=0,9626262=0,9+(0,062+0,00062+...)=0,9+\frac{0,062}{1-0,01}$$
$$=0,9+\frac{62}{990}=\frac{9\cdot99+62}{990}$$
\rozwStop
\odpStart
$\frac{9\cdot99+62}{990}$
\odpStop
\testStart
A.$\frac{9\cdot99+62}{9900}$\\ B.$-\frac{9\cdot99+62}{990}$\\ C.$0,9$\\ D.$\frac{62\cdot100}{9900}$
\testStop
\kluczStart
A
\kluczStop



\zadStart{Zadanie z Wikieł Z 3.18 c) moja wersja nr 53}

Zamień poniższe ułamki dziesiętne okresowe na ułamki zwykłe $0,9(63)$.
\zadStop
\rozwStart{Patryk Wirkus}{Martyna Czarnobaj}
$$0,9(63)=0,9636363=0,9+(0,063+0,00063+...)=0,9+\frac{0,063}{1-0,01}$$
$$=0,9+\frac{63}{990}=\frac{9\cdot99+63}{990}$$
\rozwStop
\odpStart
$\frac{9\cdot99+63}{990}$
\odpStop
\testStart
A.$\frac{9\cdot99+63}{9900}$\\ B.$-\frac{9\cdot99+63}{990}$\\ C.$0,9$\\ D.$\frac{63\cdot100}{9900}$
\testStop
\kluczStart
A
\kluczStop



\zadStart{Zadanie z Wikieł Z 3.18 c) moja wersja nr 54}

Zamień poniższe ułamki dziesiętne okresowe na ułamki zwykłe $0,9(64)$.
\zadStop
\rozwStart{Patryk Wirkus}{Martyna Czarnobaj}
$$0,9(64)=0,9646464=0,9+(0,064+0,00064+...)=0,9+\frac{0,064}{1-0,01}$$
$$=0,9+\frac{64}{990}=\frac{9\cdot99+64}{990}$$
\rozwStop
\odpStart
$\frac{9\cdot99+64}{990}$
\odpStop
\testStart
A.$\frac{9\cdot99+64}{9900}$\\ B.$-\frac{9\cdot99+64}{990}$\\ C.$0,9$\\ D.$\frac{64\cdot100}{9900}$
\testStop
\kluczStart
A
\kluczStop



\zadStart{Zadanie z Wikieł Z 3.18 c) moja wersja nr 55}

Zamień poniższe ułamki dziesiętne okresowe na ułamki zwykłe $0,9(65)$.
\zadStop
\rozwStart{Patryk Wirkus}{Martyna Czarnobaj}
$$0,9(65)=0,9656565=0,9+(0,065+0,00065+...)=0,9+\frac{0,065}{1-0,01}$$
$$=0,9+\frac{65}{990}=\frac{9\cdot99+65}{990}$$
\rozwStop
\odpStart
$\frac{9\cdot99+65}{990}$
\odpStop
\testStart
A.$\frac{9\cdot99+65}{9900}$\\ B.$-\frac{9\cdot99+65}{990}$\\ C.$0,9$\\ D.$\frac{65\cdot100}{9900}$
\testStop
\kluczStart
A
\kluczStop



\zadStart{Zadanie z Wikieł Z 3.18 c) moja wersja nr 56}

Zamień poniższe ułamki dziesiętne okresowe na ułamki zwykłe $0,9(70)$.
\zadStop
\rozwStart{Patryk Wirkus}{Martyna Czarnobaj}
$$0,9(70)=0,9707070=0,9+(0,070+0,00070+...)=0,9+\frac{0,070}{1-0,01}$$
$$=0,9+\frac{70}{990}=\frac{9\cdot99+70}{990}$$
\rozwStop
\odpStart
$\frac{9\cdot99+70}{990}$
\odpStop
\testStart
A.$\frac{9\cdot99+70}{9900}$\\ B.$-\frac{9\cdot99+70}{990}$\\ C.$0,9$\\ D.$\frac{70\cdot100}{9900}$
\testStop
\kluczStart
A
\kluczStop



\zadStart{Zadanie z Wikieł Z 3.18 c) moja wersja nr 57}

Zamień poniższe ułamki dziesiętne okresowe na ułamki zwykłe $0,9(71)$.
\zadStop
\rozwStart{Patryk Wirkus}{Martyna Czarnobaj}
$$0,9(71)=0,9717171=0,9+(0,071+0,00071+...)=0,9+\frac{0,071}{1-0,01}$$
$$=0,9+\frac{71}{990}=\frac{9\cdot99+71}{990}$$
\rozwStop
\odpStart
$\frac{9\cdot99+71}{990}$
\odpStop
\testStart
A.$\frac{9\cdot99+71}{9900}$\\ B.$-\frac{9\cdot99+71}{990}$\\ C.$0,9$\\ D.$\frac{71\cdot100}{9900}$
\testStop
\kluczStart
A
\kluczStop



\zadStart{Zadanie z Wikieł Z 3.18 c) moja wersja nr 58}

Zamień poniższe ułamki dziesiętne okresowe na ułamki zwykłe $0,9(72)$.
\zadStop
\rozwStart{Patryk Wirkus}{Martyna Czarnobaj}
$$0,9(72)=0,9727272=0,9+(0,072+0,00072+...)=0,9+\frac{0,072}{1-0,01}$$
$$=0,9+\frac{72}{990}=\frac{9\cdot99+72}{990}$$
\rozwStop
\odpStart
$\frac{9\cdot99+72}{990}$
\odpStop
\testStart
A.$\frac{9\cdot99+72}{9900}$\\ B.$-\frac{9\cdot99+72}{990}$\\ C.$0,9$\\ D.$\frac{72\cdot100}{9900}$
\testStop
\kluczStart
A
\kluczStop



\zadStart{Zadanie z Wikieł Z 3.18 c) moja wersja nr 59}

Zamień poniższe ułamki dziesiętne okresowe na ułamki zwykłe $0,9(73)$.
\zadStop
\rozwStart{Patryk Wirkus}{Martyna Czarnobaj}
$$0,9(73)=0,9737373=0,9+(0,073+0,00073+...)=0,9+\frac{0,073}{1-0,01}$$
$$=0,9+\frac{73}{990}=\frac{9\cdot99+73}{990}$$
\rozwStop
\odpStart
$\frac{9\cdot99+73}{990}$
\odpStop
\testStart
A.$\frac{9\cdot99+73}{9900}$\\ B.$-\frac{9\cdot99+73}{990}$\\ C.$0,9$\\ D.$\frac{73\cdot100}{9900}$
\testStop
\kluczStart
A
\kluczStop



\zadStart{Zadanie z Wikieł Z 3.18 c) moja wersja nr 60}

Zamień poniższe ułamki dziesiętne okresowe na ułamki zwykłe $0,9(74)$.
\zadStop
\rozwStart{Patryk Wirkus}{Martyna Czarnobaj}
$$0,9(74)=0,9747474=0,9+(0,074+0,00074+...)=0,9+\frac{0,074}{1-0,01}$$
$$=0,9+\frac{74}{990}=\frac{9\cdot99+74}{990}$$
\rozwStop
\odpStart
$\frac{9\cdot99+74}{990}$
\odpStop
\testStart
A.$\frac{9\cdot99+74}{9900}$\\ B.$-\frac{9\cdot99+74}{990}$\\ C.$0,9$\\ D.$\frac{74\cdot100}{9900}$
\testStop
\kluczStart
A
\kluczStop



\zadStart{Zadanie z Wikieł Z 3.18 c) moja wersja nr 61}

Zamień poniższe ułamki dziesiętne okresowe na ułamki zwykłe $0,9(75)$.
\zadStop
\rozwStart{Patryk Wirkus}{Martyna Czarnobaj}
$$0,9(75)=0,9757575=0,9+(0,075+0,00075+...)=0,9+\frac{0,075}{1-0,01}$$
$$=0,9+\frac{75}{990}=\frac{9\cdot99+75}{990}$$
\rozwStop
\odpStart
$\frac{9\cdot99+75}{990}$
\odpStop
\testStart
A.$\frac{9\cdot99+75}{9900}$\\ B.$-\frac{9\cdot99+75}{990}$\\ C.$0,9$\\ D.$\frac{75\cdot100}{9900}$
\testStop
\kluczStart
A
\kluczStop



\zadStart{Zadanie z Wikieł Z 3.18 c) moja wersja nr 62}

Zamień poniższe ułamki dziesiętne okresowe na ułamki zwykłe $0,9(76)$.
\zadStop
\rozwStart{Patryk Wirkus}{Martyna Czarnobaj}
$$0,9(76)=0,9767676=0,9+(0,076+0,00076+...)=0,9+\frac{0,076}{1-0,01}$$
$$=0,9+\frac{76}{990}=\frac{9\cdot99+76}{990}$$
\rozwStop
\odpStart
$\frac{9\cdot99+76}{990}$
\odpStop
\testStart
A.$\frac{9\cdot99+76}{9900}$\\ B.$-\frac{9\cdot99+76}{990}$\\ C.$0,9$\\ D.$\frac{76\cdot100}{9900}$
\testStop
\kluczStart
A
\kluczStop



\zadStart{Zadanie z Wikieł Z 3.18 c) moja wersja nr 63}

Zamień poniższe ułamki dziesiętne okresowe na ułamki zwykłe $0,9(80)$.
\zadStop
\rozwStart{Patryk Wirkus}{Martyna Czarnobaj}
$$0,9(80)=0,9808080=0,9+(0,080+0,00080+...)=0,9+\frac{0,080}{1-0,01}$$
$$=0,9+\frac{80}{990}=\frac{9\cdot99+80}{990}$$
\rozwStop
\odpStart
$\frac{9\cdot99+80}{990}$
\odpStop
\testStart
A.$\frac{9\cdot99+80}{9900}$\\ B.$-\frac{9\cdot99+80}{990}$\\ C.$0,9$\\ D.$\frac{80\cdot100}{9900}$
\testStop
\kluczStart
A
\kluczStop



\zadStart{Zadanie z Wikieł Z 3.18 c) moja wersja nr 64}

Zamień poniższe ułamki dziesiętne okresowe na ułamki zwykłe $0,9(81)$.
\zadStop
\rozwStart{Patryk Wirkus}{Martyna Czarnobaj}
$$0,9(81)=0,9818181=0,9+(0,081+0,00081+...)=0,9+\frac{0,081}{1-0,01}$$
$$=0,9+\frac{81}{990}=\frac{9\cdot99+81}{990}$$
\rozwStop
\odpStart
$\frac{9\cdot99+81}{990}$
\odpStop
\testStart
A.$\frac{9\cdot99+81}{9900}$\\ B.$-\frac{9\cdot99+81}{990}$\\ C.$0,9$\\ D.$\frac{81\cdot100}{9900}$
\testStop
\kluczStart
A
\kluczStop



\zadStart{Zadanie z Wikieł Z 3.18 c) moja wersja nr 65}

Zamień poniższe ułamki dziesiętne okresowe na ułamki zwykłe $0,9(82)$.
\zadStop
\rozwStart{Patryk Wirkus}{Martyna Czarnobaj}
$$0,9(82)=0,9828282=0,9+(0,082+0,00082+...)=0,9+\frac{0,082}{1-0,01}$$
$$=0,9+\frac{82}{990}=\frac{9\cdot99+82}{990}$$
\rozwStop
\odpStart
$\frac{9\cdot99+82}{990}$
\odpStop
\testStart
A.$\frac{9\cdot99+82}{9900}$\\ B.$-\frac{9\cdot99+82}{990}$\\ C.$0,9$\\ D.$\frac{82\cdot100}{9900}$
\testStop
\kluczStart
A
\kluczStop



\zadStart{Zadanie z Wikieł Z 3.18 c) moja wersja nr 66}

Zamień poniższe ułamki dziesiętne okresowe na ułamki zwykłe $0,9(83)$.
\zadStop
\rozwStart{Patryk Wirkus}{Martyna Czarnobaj}
$$0,9(83)=0,9838383=0,9+(0,083+0,00083+...)=0,9+\frac{0,083}{1-0,01}$$
$$=0,9+\frac{83}{990}=\frac{9\cdot99+83}{990}$$
\rozwStop
\odpStart
$\frac{9\cdot99+83}{990}$
\odpStop
\testStart
A.$\frac{9\cdot99+83}{9900}$\\ B.$-\frac{9\cdot99+83}{990}$\\ C.$0,9$\\ D.$\frac{83\cdot100}{9900}$
\testStop
\kluczStart
A
\kluczStop



\zadStart{Zadanie z Wikieł Z 3.18 c) moja wersja nr 67}

Zamień poniższe ułamki dziesiętne okresowe na ułamki zwykłe $0,9(84)$.
\zadStop
\rozwStart{Patryk Wirkus}{Martyna Czarnobaj}
$$0,9(84)=0,9848484=0,9+(0,084+0,00084+...)=0,9+\frac{0,084}{1-0,01}$$
$$=0,9+\frac{84}{990}=\frac{9\cdot99+84}{990}$$
\rozwStop
\odpStart
$\frac{9\cdot99+84}{990}$
\odpStop
\testStart
A.$\frac{9\cdot99+84}{9900}$\\ B.$-\frac{9\cdot99+84}{990}$\\ C.$0,9$\\ D.$\frac{84\cdot100}{9900}$
\testStop
\kluczStart
A
\kluczStop



\zadStart{Zadanie z Wikieł Z 3.18 c) moja wersja nr 68}

Zamień poniższe ułamki dziesiętne okresowe na ułamki zwykłe $0,9(85)$.
\zadStop
\rozwStart{Patryk Wirkus}{Martyna Czarnobaj}
$$0,9(85)=0,9858585=0,9+(0,085+0,00085+...)=0,9+\frac{0,085}{1-0,01}$$
$$=0,9+\frac{85}{990}=\frac{9\cdot99+85}{990}$$
\rozwStop
\odpStart
$\frac{9\cdot99+85}{990}$
\odpStop
\testStart
A.$\frac{9\cdot99+85}{9900}$\\ B.$-\frac{9\cdot99+85}{990}$\\ C.$0,9$\\ D.$\frac{85\cdot100}{9900}$
\testStop
\kluczStart
A
\kluczStop



\zadStart{Zadanie z Wikieł Z 3.18 c) moja wersja nr 69}

Zamień poniższe ułamki dziesiętne okresowe na ułamki zwykłe $0,9(86)$.
\zadStop
\rozwStart{Patryk Wirkus}{Martyna Czarnobaj}
$$0,9(86)=0,9868686=0,9+(0,086+0,00086+...)=0,9+\frac{0,086}{1-0,01}$$
$$=0,9+\frac{86}{990}=\frac{9\cdot99+86}{990}$$
\rozwStop
\odpStart
$\frac{9\cdot99+86}{990}$
\odpStop
\testStart
A.$\frac{9\cdot99+86}{9900}$\\ B.$-\frac{9\cdot99+86}{990}$\\ C.$0,9$\\ D.$\frac{86\cdot100}{9900}$
\testStop
\kluczStart
A
\kluczStop



\zadStart{Zadanie z Wikieł Z 3.18 c) moja wersja nr 70}

Zamień poniższe ułamki dziesiętne okresowe na ułamki zwykłe $0,9(87)$.
\zadStop
\rozwStart{Patryk Wirkus}{Martyna Czarnobaj}
$$0,9(87)=0,9878787=0,9+(0,087+0,00087+...)=0,9+\frac{0,087}{1-0,01}$$
$$=0,9+\frac{87}{990}=\frac{9\cdot99+87}{990}$$
\rozwStop
\odpStart
$\frac{9\cdot99+87}{990}$
\odpStop
\testStart
A.$\frac{9\cdot99+87}{9900}$\\ B.$-\frac{9\cdot99+87}{990}$\\ C.$0,9$\\ D.$\frac{87\cdot100}{9900}$
\testStop
\kluczStart
A
\kluczStop





\end{document}
