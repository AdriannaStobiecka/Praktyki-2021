\documentclass[12pt, a4paper]{article}
\usepackage[utf8]{inputenc}
\usepackage{polski}

\usepackage{amsthm}  %pakiet do tworzenia twierdzeń itp.
\usepackage{amsmath} %pakiet do niektórych symboli matematycznych
\usepackage{amssymb} %pakiet do symboli mat., np. \nsubseteq
\usepackage{amsfonts}
\usepackage{graphicx} %obsługa plików graficznych z rozszerzeniem png, jpg
\theoremstyle{definition} %styl dla definicji
\newtheorem{zad}{} 
\title{Multizestaw zadań}
\author{Robert Fidytek}
%\date{\today}
\date{}
\newcounter{liczniksekcji}
\newcommand{\kategoria}[1]{\section{#1}} %olreślamy nazwę kateforii zadań
\newcommand{\zadStart}[1]{\begin{zad}#1\newline} %oznaczenie początku zadania
\newcommand{\zadStop}{\end{zad}}   %oznaczenie końca zadania
%Makra opcjonarne (nie muszą występować):
\newcommand{\rozwStart}[2]{\noindent \textbf{Rozwiązanie (autor #1 , recenzent #2): }\newline} %oznaczenie początku rozwiązania, opcjonarnie można wprowadzić informację o autorze rozwiązania zadania i recenzencie poprawności wykonania rozwiązania zadania
\newcommand{\rozwStop}{\newline}                                            %oznaczenie końca rozwiązania
\newcommand{\odpStart}{\noindent \textbf{Odpowiedź:}\newline}    %oznaczenie początku odpowiedzi końcowej (wypisanie wyniku)
\newcommand{\odpStop}{\newline}                                             %oznaczenie końca odpowiedzi końcowej (wypisanie wyniku)
\newcommand{\testStart}{\noindent \textbf{Test:}\newline} %ewentualne możliwe opcje odpowiedzi testowej: A. ? B. ? C. ? D. ? itd.
\newcommand{\testStop}{\newline} %koniec wprowadzania odpowiedzi testowych
\newcommand{\kluczStart}{\noindent \textbf{Test poprawna odpowiedź:}\newline} %klucz, poprawna odpowiedź pytania testowego (jedna literka): A lub B lub C lub D itd.
\newcommand{\kluczStop}{\newline} %koniec poprawnej odpowiedzi pytania testowego 
\newcommand{\wstawGrafike}[2]{\begin{figure}[h] \includegraphics[scale=#2] {#1} \end{figure}} %gdyby była potrzeba wstawienia obrazka, parametry: nazwa pliku, skala (jak nie wiesz co wpisać, to wpisz 1)

\begin{document}
\maketitle


\kategoria{Wikieł/Z5.60}
\zadStart{Zadanie z Wikieł Z 5.60 ) moja wersja nr [nrWersji]}
%[a]:[10,11,12,13,14,15,16,17,18,19,20,21,22,23,24,25,26,27,28,29,30]
%[a2]=[a]*2
%[ak]=[a]*[a]
%[xmin]=int([a2]/4)
%[y]=[a]-[xmin]
%[minimum]=[xmin]*[xmin]+[y]*[y]
%[zx]=[xmin]-2
%[zy]=[y]+2
%math.gcd([a2],4)==4
Przedstawić liczbę $[a]$ jako sumę dwóch składników, których suma kwadratów jest najmniejsza.
\zadStop
\rozwStart{Wojciech Przybylski}{}
$$x+y=[a] \Rightarrow y=[a]-x$$
$$x^{2}+y^{2}=\mbox{minimum}$$
$$f(x)=x^{2}+([a]-x)^{2}=2x^{2}-[a2]x+[ak]$$
Powyższa parabola ma ramiona skierowane ku górze, więc jej wierzchołek jest jej minimum.
$$x_{min}=\frac{-b}{2a}=\frac{[a2]}{4}=[xmin]$$
$$x=[xmin], \hspace{3mm} y=[a]-[xmin]=[y]$$
$$x^{2}+y^{2}=\mbox{minimum}=[xmin]^{2}+[y]^{2}=[minimum]$$
\rozwStop
\odpStart
$x=[xmin], \hspace{3mm} y=[y].$
\odpStop
\testStart
A.$x=[xmin], \hspace{3mm} y=[y]$.\\
B. $x=[zx], \hspace{3mm} y=[y]$.\\
C. $x=[zx], \hspace{3mm} y=[zy]$.\\
D.$x=[xmin], \hspace{3mm} y=[zy]$.\\
E. $x=[ak], \hspace{3mm} y=[a2]$.\\
F. Nie istnieją takie składniki spełniające to zadanie.
\testStop
\kluczStart
A
\kluczStop



\end{document}