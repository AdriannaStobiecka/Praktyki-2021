\documentclass[12pt, a4paper]{article}
\usepackage[utf8]{inputenc}
\usepackage{polski}

\usepackage{amsthm}  %pakiet do tworzenia twierdzeń itp.
\usepackage{amsmath} %pakiet do niektórych symboli matematycznych
\usepackage{amssymb} %pakiet do symboli mat., np. \nsubseteq
\usepackage{amsfonts}
\usepackage{graphicx} %obsługa plików graficznych z rozszerzeniem png, jpg
\theoremstyle{definition} %styl dla definicji
\newtheorem{zad}{} 
\title{Multizestaw zadań}
\author{Robert Fidytek}
%\date{\today}
\date{}
\newcounter{liczniksekcji}
\newcommand{\kategoria}[1]{\section{#1}} %olreślamy nazwę kateforii zadań
\newcommand{\zadStart}[1]{\begin{zad}#1\newline} %oznaczenie początku zadania
\newcommand{\zadStop}{\end{zad}}   %oznaczenie końca zadania
%Makra opcjonarne (nie muszą występować):
\newcommand{\rozwStart}[2]{\noindent \textbf{Rozwiązanie (autor #1 , recenzent #2): }\newline} %oznaczenie początku rozwiązania, opcjonarnie można wprowadzić informację o autorze rozwiązania zadania i recenzencie poprawności wykonania rozwiązania zadania
\newcommand{\rozwStop}{\newline}                                            %oznaczenie końca rozwiązania
\newcommand{\odpStart}{\noindent \textbf{Odpowiedź:}\newline}    %oznaczenie początku odpowiedzi końcowej (wypisanie wyniku)
\newcommand{\odpStop}{\newline}                                             %oznaczenie końca odpowiedzi końcowej (wypisanie wyniku)
\newcommand{\testStart}{\noindent \textbf{Test:}\newline} %ewentualne możliwe opcje odpowiedzi testowej: A. ? B. ? C. ? D. ? itd.
\newcommand{\testStop}{\newline} %koniec wprowadzania odpowiedzi testowych
\newcommand{\kluczStart}{\noindent \textbf{Test poprawna odpowiedź:}\newline} %klucz, poprawna odpowiedź pytania testowego (jedna literka): A lub B lub C lub D itd.
\newcommand{\kluczStop}{\newline} %koniec poprawnej odpowiedzi pytania testowego 
\newcommand{\wstawGrafike}[2]{\begin{figure}[h] \includegraphics[scale=#2] {#1} \end{figure}} %gdyby była potrzeba wstawienia obrazka, parametry: nazwa pliku, skala (jak nie wiesz co wpisać, to wpisz 1)

\begin{document}
\maketitle


\kategoria{Wikieł/Z1.76a}
\zadStart{Zadanie z Wikieł Z 1.76 a) moja wersja nr [nrWersji]}
%[a]:[3,5,7,9]
%[b]:[3,5,7,9]
%[d]:[2,3,4,5,6,7,8,9]
%[d1]=[d]+1
%math.gcd([a],[b])==1 and [a]>[d1]
Wyznaczyć dziedzinę naturalną funkcji określonej podanym wzorem.
$$f(x)=x^2(x^2-[a]x)^{-\frac{[a]}{[b]}}+\sqrt{1-\frac{1}{x-[d]}}$$
\zadStop
\rozwStart{Adrianna Stobiecka}{}
Funkcję potęgową $f_1(x)=(x^2-[a]x)^{-\frac{[a]}{[b]}}$ możemy zapisać w postaci $f_1(x)=\big(\frac{1}{x^2-[a]x}\big)^{\frac{[a]}{[b]}}$. Zatem musi być spełniony warunek $x^2-[a]x\ne0$. Funkcja $f_2(x)=1-\frac{1}{x-[d]}$ znajdująca się pod pierwiastkiem musi spełniać nierówność $f_2(x)\geq0$. Mianownik $x-[d]$ musi być różny od $0$. Zatem dziedziną funkcji $f$ jest zbiór
$$D_{f}=\bigg\{x\in\mathbb{R}:x^2-[a]x\ne0 \land 1-\frac{1}{x-[d]}\geq0 \land x-[d]\ne0\bigg\}.$$
Zacznijmy od pierwszego warunku.
$$x^2-[a]x\ne0\qquad\Leftrightarrow\qquad x(x-[a])\ne0\qquad\Leftrightarrow\qquad x\ne0~~\land~~ x\ne[a]$$
Rozwiążemy teraz drugi warunek.
$$1-\frac{1}{x-[d]}\geq0\Leftrightarrow\frac{-1+x-[d]}{x-[d]}\geq0\Leftrightarrow\frac{x-[d1]}{x-[d]}\geq0\Leftrightarrow(x-[d1])(x-[d])\geq0$$
Lewa strona nierówności to parabola z ramionami skierowanymi do góry oraz pierwiastkami w $x_1=[d]$ i $x_2=[d1]$. Nierówność jest spełniona zatem dla $x\in(-\infty,[d]]\cup[[d1],\infty)$.
\\Z warunku $x-[d]\ne0$. Otrzymujemy, że $x$ musi być różne od $[d]$.
\\Łącząc wszystkie warunki otrzymujemy, że 
$$D_{f}=(-\infty,0)\cup(0,[d])\cup[[d1],[a])\cup([a],\infty).$$
\rozwStop
\odpStart
$D_{f}=(-\infty,0)\cup(0,[d])\cup[[d1],[a])\cup([a],\infty)$
\odpStop
\testStart
A.$D_{f}=(-\infty,0)\cup(0,[d])\cup([d],[a])\cup([a],\infty)$
B.$D_{f}=(-\infty,0)\cup(0,[d])\cup[[d1],[a])\cup([a],\infty)$
C.$D_{f}=(-\infty,0)\cup(0,[d1])\cup([d1],[a])\cup([a],\infty)$
D.$D_{f}=(-\infty,0)\cup(0,[d])\cup[[d1],\infty)$
E.$D_{f}=(-\infty,[d])\cup[[d1],\infty)$
F.$D_{f}=(-\infty,0)\cup(0,[a])\cup([a],\infty)$
G.$D_{f}=\mathbb{R}\setminus\{0\}$
H.$D_{f}=(-\infty,[d])\cup[[d1],[a])\cup([a],\infty)$
I.$D_{f}=[[d],[d1]]$
\testStop
\kluczStart
B
\kluczStop



\end{document}
