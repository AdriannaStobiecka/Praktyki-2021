\documentclass[12pt, a4paper]{article}
\usepackage[utf8]{inputenc}
\usepackage{polski}

\usepackage{amsthm}  %pakiet do tworzenia twierdzeń itp.
\usepackage{amsmath} %pakiet do niektórych symboli matematycznych
\usepackage{amssymb} %pakiet do symboli mat., np. \nsubseteq
\usepackage{amsfonts}
\usepackage{graphicx} %obsługa plików graficznych z rozszerzeniem png, jpg
\theoremstyle{definition} %styl dla definicji
\newtheorem{zad}{} 
\title{Multizestaw zadań}
\author{Robert Fidytek}
%\date{\today}
\date{}
\newcounter{liczniksekcji}
\newcommand{\kategoria}[1]{\section{#1}} %olreślamy nazwę kateforii zadań
\newcommand{\zadStart}[1]{\begin{zad}#1\newline} %oznaczenie początku zadania
\newcommand{\zadStop}{\end{zad}}   %oznaczenie końca zadania
%Makra opcjonarne (nie muszą występować):
\newcommand{\rozwStart}[2]{\noindent \textbf{Rozwiązanie (autor #1 , recenzent #2): }\newline} %oznaczenie początku rozwiązania, opcjonarnie można wprowadzić informację o autorze rozwiązania zadania i recenzencie poprawności wykonania rozwiązania zadania
\newcommand{\rozwStop}{\newline}                                            %oznaczenie końca rozwiązania
\newcommand{\odpStart}{\noindent \textbf{Odpowiedź:}\newline}    %oznaczenie początku odpowiedzi końcowej (wypisanie wyniku)
\newcommand{\odpStop}{\newline}                                             %oznaczenie końca odpowiedzi końcowej (wypisanie wyniku)
\newcommand{\testStart}{\noindent \textbf{Test:}\newline} %ewentualne możliwe opcje odpowiedzi testowej: A. ? B. ? C. ? D. ? itd.
\newcommand{\testStop}{\newline} %koniec wprowadzania odpowiedzi testowych
\newcommand{\kluczStart}{\noindent \textbf{Test poprawna odpowiedź:}\newline} %klucz, poprawna odpowiedź pytania testowego (jedna literka): A lub B lub C lub D itd.
\newcommand{\kluczStop}{\newline} %koniec poprawnej odpowiedzi pytania testowego 
\newcommand{\wstawGrafike}[2]{\begin{figure}[h] \includegraphics[scale=#2] {#1} \end{figure}} %gdyby była potrzeba wstawienia obrazka, parametry: nazwa pliku, skala (jak nie wiesz co wpisać, to wpisz 1)

\begin{document}
\maketitle

\kategoria{Wikieł/P1.13}

\zadStart{Zadanie z Wikieł P 1.13 moja wersja nr [nrWersji]}
%[a]:[7,11,13,17,19,23,29,31,37,41]
%[b]=[a]//6
%[c]=[a]%6
Obliczyć dla $n \in \mathbb{N}$:\\
$$\binom{[a]n - 1}{1} + \binom{[a]n - 1}{2} + \binom{[a]n}{3}$$
\zadStop

\rozwStart{Natalia Danieluk}{}
$$\binom{[a]n - 1}{1} + \binom{[a]n - 1}{2} + \binom{[a]n}{3} = 
\left \rVert 
\begin{split} 
\text{z drugiej własności} \\
\text{symbolu Newtona}
\end{split} \right \rVert 
= \binom{[a]n}{2} + \binom{[a]n}{3} = $$
$$= \left \rVert 
\begin{split} 
\text{z drugiej własności} \\
\text{symbolu Newtona}
\end{split} \right \rVert 
= \binom{[a]n + 1}{3} = \frac{([a]n + 1)!}{3!([a]n - 2)!} = 
\frac{([a]n - 2)!([a]n - 1)[a]n([a]n + 1)}{6([a]n - 2)!} = $$
$$= [b]\frac{[c]}{6}n([a]n - 1)([a]n + 1)$$
\rozwStop

\odpStart
$[b]\frac{[c]}{6}n([a]n - 1)([a]n + 1)$
\odpStop

\testStart
A. $([a]n)$
B. $([a]n - 1)$
C. $\emptyset$
D. $[b]\frac{[c]}{6}n([a]n - 1)([a]n + 1)$
\testStop

\kluczStart
D
\kluczStop

\end{document}