\documentclass[12pt, a4paper]{article}
\usepackage[utf8]{inputenc}
\usepackage{polski}

\usepackage{amsthm}  %pakiet do tworzenia twierdzeń itp.
\usepackage{amsmath} %pakiet do niektórych symboli matematycznych
\usepackage{amssymb} %pakiet do symboli mat., np. \nsubseteq
\usepackage{amsfonts}
\usepackage{graphicx} %obsługa plików graficznych z rozszerzeniem png, jpg
\theoremstyle{definition} %styl dla definicji
\newtheorem{zad}{} 
\title{Multizestaw zadań}
\author{Robert Fidytek}
%\date{\today}
\date{}
\newcounter{liczniksekcji}
\newcommand{\kategoria}[1]{\section{#1}} %olreślamy nazwę kateforii zadań
\newcommand{\zadStart}[1]{\begin{zad}#1\newline} %oznaczenie początku zadania
\newcommand{\zadStop}{\end{zad}}   %oznaczenie końca zadania
%Makra opcjonarne (nie muszą występować):
\newcommand{\rozwStart}[2]{\noindent \textbf{Rozwiązanie (autor #1 , recenzent #2): }\newline} %oznaczenie początku rozwiązania, opcjonarnie można wprowadzić informację o autorze rozwiązania zadania i recenzencie poprawności wykonania rozwiązania zadania
\newcommand{\rozwStop}{\newline}                                            %oznaczenie końca rozwiązania
\newcommand{\odpStart}{\noindent \textbf{Odpowiedź:}\newline}    %oznaczenie początku odpowiedzi końcowej (wypisanie wyniku)
\newcommand{\odpStop}{\newline}                                             %oznaczenie końca odpowiedzi końcowej (wypisanie wyniku)
\newcommand{\testStart}{\noindent \textbf{Test:}\newline} %ewentualne możliwe opcje odpowiedzi testowej: A. ? B. ? C. ? D. ? itd.
\newcommand{\testStop}{\newline} %koniec wprowadzania odpowiedzi testowych
\newcommand{\kluczStart}{\noindent \textbf{Test poprawna odpowiedź:}\newline} %klucz, poprawna odpowiedź pytania testowego (jedna literka): A lub B lub C lub D itd.
\newcommand{\kluczStop}{\newline} %koniec poprawnej odpowiedzi pytania testowego 
\newcommand{\wstawGrafike}[2]{\begin{figure}[h] \includegraphics[scale=#2] {#1} \end{figure}} %gdyby była potrzeba wstawienia obrazka, parametry: nazwa pliku, skala (jak nie wiesz co wpisać, to wpisz 1)

\begin{document}
\maketitle


\kategoria{Wikieł/Z3.17b}
\zadStart{Zadanie z Wikieł Z 3.17 b) moja wersja nr [nrWersji]}
%[a]:[3,5,6,7,10,11,13,14,15,17,19,21,22,23,27,29,30,31,33,34,35,38,39]
%[a1]=[a]-1
%[calosci]=[a]//[a1]
%[reszta]=[a]%[a1]
%math.gcd([a],[a1])==1
Obliczyć poniższą sumę
$$\sqrt{[a]}+\frac{1}{\sqrt{[a]}}+\frac{1}{[a]\sqrt{[a]}}+\cdots.$$
\zadStop
\rozwStart{Adrianna Stobiecka}{}
Zauważamy, że liczby $\sqrt{[a]}$, $\frac{1}{\sqrt{[a]}}$, $\frac{1}{[a]\sqrt{[a]}}$, $\dots$ są kolejnymi wyrazami ciągu geometrycznego o pierwszym wyrazie $a_1=\sqrt{[a]}$ oraz ilorazie $q=\frac{1}{[a]}$. Widzimy, że 
$$|q|=\bigg|\frac{1}{[a]}\bigg|=\frac{1}{[a]}<1.$$
Zatem powyższą sumę obliczymy jako sumę nieskończonego ciągu geometrycznego.
$$S=\frac{a_1}{1-q}=\frac{\sqrt{[a]}}{1-\frac{1}{[a]}}=\frac{\sqrt{[a]}}{\frac{[a1]}{[a]}}=\sqrt{[a]}\cdot\frac{[a]}{[a1]}=\frac{[a]\sqrt{[a]}}{[a1]}$$
Otrzymyjemy zatem, że suma jest równa $\frac{[a]\sqrt{[a]}}{[a1]}$.
\rozwStop
\odpStart
$\frac{[a]\sqrt{[a]}}{[a1]}$
\odpStop
\testStart
A.$[a]$
B.$\frac{[a]\sqrt{[a]}}{[a1]}$
C.$\sqrt{[a]}$
D.$[a1]$
E.$[calosci]\frac{[reszta]}{[a1]}$
F.$\frac{\sqrt{[a]}}{[a1]}$
G.$\frac{\sqrt{[a]}}{[a]}$
H.$\frac{[a1]}{[a]}$
I.$[a]\sqrt{[a]}$
\testStop
\kluczStart
B
\kluczStop



\end{document}
