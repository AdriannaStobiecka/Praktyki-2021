\documentclass[12pt, a4paper]{article}
\usepackage[utf8]{inputenc}
\usepackage{polski}

\usepackage{amsthm}  %pakiet do tworzenia twierdzeń itp.
\usepackage{amsmath} %pakiet do niektórych symboli matematycznych
\usepackage{amssymb} %pakiet do symboli mat., np. \nsubseteq
\usepackage{amsfonts}
\usepackage{graphicx} %obsługa plików graficznych z rozszerzeniem png, jpg
\theoremstyle{definition} %styl dla definicji
\newtheorem{zad}{} 
\title{Multizestaw zadań}
\author{Robert Fidytek}
%\date{\today}
\date{}\documentclass[12pt, a4paper]{article}
\usepackage[utf8]{inputenc}
\usepackage{polski}

\usepackage{amsthm}  %pakiet do tworzenia twierdzeń itp.
\usepackage{amsmath} %pakiet do niektórych symboli matematycznych
\usepackage{amssymb} %pakiet do symboli mat., np. \nsubseteq
\usepackage{amsfonts}
\usepackage{graphicx} %obsługa plików graficznych z rozszerzeniem png, jpg
\theoremstyle{definition} %styl dla definicji
\newtheorem{zad}{} 
\title{Multizestaw zadań}
\author{Robert Fidytek}
%\date{\today}
\date{}
\newcounter{liczniksekcji}
\newcommand{\kategoria}[1]{\section{#1}} %olreślamy nazwę kateforii zadań
\newcommand{\zadStart}[1]{\begin{zad}#1\newline} %oznaczenie początku zadania
\newcommand{\zadStop}{\end{zad}}   %oznaczenie końca zadania
%Makra opcjonarne (nie muszą występować):
\newcommand{\rozwStart}[2]{\noindent \textbf{Rozwiązanie (autor #1 , recenzent #2): }\newline} %oznaczenie początku rozwiązania, opcjonarnie można wprowadzić informację o autorze rozwiązania zadania i recenzencie poprawności wykonania rozwiązania zadania
\newcommand{\rozwStop}{\newline}                                            %oznaczenie końca rozwiązania
\newcommand{\odpStart}{\noindent \textbf{Odpowiedź:}\newline}    %oznaczenie początku odpowiedzi końcowej (wypisanie wyniku)
\newcommand{\odpStop}{\newline}                                             %oznaczenie końca odpowiedzi końcowej (wypisanie wyniku)
\newcommand{\testStart}{\noindent \textbf{Test:}\newline} %ewentualne możliwe opcje odpowiedzi testowej: A. ? B. ? C. ? D. ? itd.
\newcommand{\testStop}{\newline} %koniec wprowadzania odpowiedzi testowych
\newcommand{\kluczStart}{\noindent \textbf{Test poprawna odpowiedź:}\newline} %klucz, poprawna odpowiedź pytania testowego (jedna literka): A lub B lub C lub D itd.
\newcommand{\kluczStop}{\newline} %koniec poprawnej odpowiedzi pytania testowego 
\newcommand{\wstawGrafike}[2]{\begin{figure}[h] \includegraphics[scale=#2] {#1} \end{figure}} %gdyby była potrzeba wstawienia obrazka, parametry: nazwa pliku, skala (jak nie wiesz co wpisać, to wpisz 1)

\begin{document}
\maketitle


\kategoria{Wikieł/Z2.2}
\zadStart{Zadanie z Wikieł Z 2.32  moja wersja nr [nrWersji]}
%[p1]:[2,3,4,5,6,7,8,9,10]
%[p2]=random.randint(2,10)
%[p3]:[2,3,4,5,6,7,8,9,10]
%[p4]:[2,3,4,5]
%[p5]=random.randint(2,10)
%[p2p4]=[p2]-[p4]
%[p1p3]=[p1]-[p3]
%[a]=round([p2p4]/([p1p3]+0.0000001),2)
%[ab]=[p1]*[a]
%[b]=[p2]-[ab]
%[p5b]=[p5]-[b]
%[x]=round([a]/([p5b]+0.0000001),2)
%[p2p4]!=0 and [p1p3]!=0 and math.gcd([p2p4],[p1p3])==1 

Punkty $A([p1],[p2]), B([p3],[p4]), C(x,[p5])$ są współliniowe. Wyznaczyć $x$.
\zadStop

\rozwStart{Maja Szabłowska}{}
$$\left\{ \begin{array}{ll}
[p2]=[p1]a+b\\
[p4]=[p3]a+b
\end{array} \right.$$
$$[p2]-[p4]=[p1]a-[p3]a$$
$$[p2p4]=[p1p3]a$$
$$a=\frac{[p2p4]}{[p1p3]}=[a]$$

$$[p2]=[p1]\cdot\frac{[p2p4]}{[p1p3]}+b $$
$$b=[p2]-([ab])=[b]$$

Zatem $y=[a]x+[b].$ Punkt $C$ również należy do tej prostej, wówczas:
$$[p5]=[a]x+[b]$$
$$[p5b]=[a]x$$
$$x=\frac{[a]}{[p5b]}=[x]$$

\rozwStop


\odpStart
$x=[x]$
\odpStop
\testStart
A.$x=[x]$
B.$x=\frac{[p1]}{[p2]}$
D.$x=[b]$
E.$x=[ab]$
F.$x=[p2p4]$
G.$x=[p5b]$
H.$x=\frac{[p5]}{[p3]}$

\testStop
\kluczStart
A
\kluczStop



\end{document}
