\documentclass[12pt, a4paper]{article}
\usepackage[utf8]{inputenc}
\usepackage{polski}

\usepackage{amsthm}  %pakiet do tworzenia twierdzeń itp.
\usepackage{amsmath} %pakiet do niektórych symboli matematycznych
\usepackage{amssymb} %pakiet do symboli mat., np. \nsubseteq
\usepackage{amsfonts}
\usepackage{graphicx} %obsługa plików graficznych z rozszerzeniem png, jpg
\theoremstyle{definition} %styl dla definicji
\newtheorem{zad}{} 
\title{Multizestaw zadań}
\author{Robert Fidytek}
%\date{\today}
\date{}
\newcounter{liczniksekcji}
\newcommand{\kategoria}[1]{\section{#1}} %olreślamy nazwę kateforii zadań
\newcommand{\zadStart}[1]{\begin{zad}#1\newline} %oznaczenie początku zadania
\newcommand{\zadStop}{\end{zad}}   %oznaczenie końca zadania
%Makra opcjonarne (nie muszą występować):
\newcommand{\rozwStart}[2]{\noindent \textbf{Rozwiązanie (autor #1 , recenzent #2): }\newline} %oznaczenie początku rozwiązania, opcjonarnie można wprowadzić informację o autorze rozwiązania zadania i recenzencie poprawności wykonania rozwiązania zadania
\newcommand{\rozwStop}{\newline}                                            %oznaczenie końca rozwiązania
\newcommand{\odpStart}{\noindent \textbf{Odpowiedź:}\newline}    %oznaczenie początku odpowiedzi końcowej (wypisanie wyniku)
\newcommand{\odpStop}{\newline}                                             %oznaczenie końca odpowiedzi końcowej (wypisanie wyniku)
\newcommand{\testStart}{\noindent \textbf{Test:}\newline} %ewentualne możliwe opcje odpowiedzi testowej: A. ? B. ? C. ? D. ? itd.
\newcommand{\testStop}{\newline} %koniec wprowadzania odpowiedzi testowych
\newcommand{\kluczStart}{\noindent \textbf{Test poprawna odpowiedź:}\newline} %klucz, poprawna odpowiedź pytania testowego (jedna literka): A lub B lub C lub D itd.
\newcommand{\kluczStop}{\newline} %koniec poprawnej odpowiedzi pytania testowego 
\newcommand{\wstawGrafike}[2]{\begin{figure}[h] \includegraphics[scale=#2] {#1} \end{figure}} %gdyby była potrzeba wstawienia obrazka, parametry: nazwa pliku, skala (jak nie wiesz co wpisać, to wpisz 1)

\begin{document}
\maketitle


\kategoria{Wikieł/Z4.4v}
\zadStart{Zadanie z Wikieł Z 4.4v) moja wersja nr [nrWersji]}
%[p1]:[2,3,4,5,6,7,8,9,10]
%[g]=math.gcd([p1],4)
%[up]=int([p1]/[g])
%[down]=int(4/[g])
%[down]!=1 and [up]!=1
Obliczyć granicę funkcji $\lim_{x \to \frac{\Pi}{4}} \frac{[p1]x(1-\tg{x})}{\cos{2x}}$
\zadStop
\rozwStart{Jakub Janik}{}
Zaczniemy od udowodnienia tożsamości $\cos{2x}=\frac{1-\tg^2{x}}{1+\tg^2{x}}$
$$\cos{2x}=\cos^2{x}-\sin^2{x}=\frac{\cos^2{x}-\sin^2{x}}{\cos^2{x}+\sin^2{x}}=$$
$$\frac{\frac{\cos^2{x}-\sin^2{x}}{\cos^2{x}}}{\frac{\cos^2{x}+\sin^2{x}}{\cos^2{x}}}=\frac{1-\tg^2{x}}{1+\tg^2{x}}$$
Możemy przejść do zadania
$$\lim_{x \to \frac{\Pi}{4}} \frac{[p1]x(1-\tg{x})}{\cos{2x}}=\lim_{x \to \frac{\Pi}{4}} \frac{[p1]x(1-\tg{x})}{\frac{1-\tg^2{x}}{1+\tg^2{x}}}=$$
$$\lim_{x \to \frac{\Pi}{4}} \frac{[p1]x(1-\tg{x})}{\frac{(1-\tg{x})(1+\tg{x})}{1+\tg^2{x}}}=\lim_{x \to \frac{\Pi}{4}} \frac{[p1]x}{\frac{1+\tg{x}}{1+\tg^2{x}}}=$$
$$\lim_{x \to \frac{\Pi}{4}} \frac{[p1]x(1+\tg^2{x})}{1+\tg{x}}=\frac{[up]\Pi}{[down]}$$
\rozwStop
\odpStart
$\frac{[up]\Pi}{[down]}$
\odpStop
\testStart
A.$\frac{[up]\Pi}{[down]}$
B.$0$
C.$-\frac{[up]\Pi}{[down]}$
D.$\infty$
\testStop
\kluczStart
A
\kluczStop



\end{document}