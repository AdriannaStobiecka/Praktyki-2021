\documentclass[12pt, a4paper]{article}
\usepackage[utf8]{inputenc}
\usepackage{polski}

\usepackage{amsthm}  %pakiet do tworzenia twierdzeń itp.
\usepackage{amsmath} %pakiet do niektórych symboli matematycznych
\usepackage{amssymb} %pakiet do symboli mat., np. \nsubseteq
\usepackage{amsfonts}
\usepackage{graphicx} %obsługa plików graficznych z rozszerzeniem png, jpg
\theoremstyle{definition} %styl dla definicji
\newtheorem{zad}{} 
\title{Multizestaw zadań}
\author{Robert Fidytek}
%\date{\today}
\date{}
\newcounter{liczniksekcji}
\newcommand{\kategoria}[1]{\section{#1}} %olreślamy nazwę kateforii zadań
\newcommand{\zadStart}[1]{\begin{zad}#1\newline} %oznaczenie początku zadania
\newcommand{\zadStop}{\end{zad}}   %oznaczenie końca zadania
%Makra opcjonarne (nie muszą występować):
\newcommand{\rozwStart}[2]{\noindent \textbf{Rozwiązanie (autor #1 , recenzent #2): }\newline} %oznaczenie początku rozwiązania, opcjonarnie można wprowadzić informację o autorze rozwiązania zadania i recenzencie poprawności wykonania rozwiązania zadania
\newcommand{\rozwStop}{\newline}                                            %oznaczenie końca rozwiązania
\newcommand{\odpStart}{\noindent \textbf{Odpowiedź:}\newline}    %oznaczenie początku odpowiedzi końcowej (wypisanie wyniku)
\newcommand{\odpStop}{\newline}                                             %oznaczenie końca odpowiedzi końcowej (wypisanie wyniku)
\newcommand{\testStart}{\noindent \textbf{Test:}\newline} %ewentualne możliwe opcje odpowiedzi testowej: A. ? B. ? C. ? D. ? itd.
\newcommand{\testStop}{\newline} %koniec wprowadzania odpowiedzi testowych
\newcommand{\kluczStart}{\noindent \textbf{Test poprawna odpowiedź:}\newline} %klucz, poprawna odpowiedź pytania testowego (jedna literka): A lub B lub C lub D itd.
\newcommand{\kluczStop}{\newline} %koniec poprawnej odpowiedzi pytania testowego 
\newcommand{\wstawGrafike}[2]{\begin{figure}[h] \includegraphics[scale=#2] {#1} \end{figure}} %gdyby była potrzeba wstawienia obrazka, parametry: nazwa pliku, skala (jak nie wiesz co wpisać, to wpisz 1)

\begin{document}
\maketitle


\kategoria{Wikieł/Z1.97d}
\zadStart{Zadanie z Wikieł Z 1.97 d) moja wersja nr [nrWersji]}
%[a]:[2,3,4,5]
%[b]:[2,3,4,5,6,7,8,9,10]
%[c]:[1,2,3,4,5,6,7,8,9,10,11,12,13,14,15,16,17,18,19,20,21,22,23,24,25,26,27,28,29,30]
%[p]=random.randint(2,9)
%[c2]=[c]+1
%[d]=[b]**2-4*[a]*[c2]
%[pr2]=(pow([d],(1/2)))
%[pr1]=[pr2].real
%[pr]=int([pr1])
%[aa]=2*[a]
%[z1]=round(([b]-[pr])/[aa],2)
%[z2]=round(([b]+[pr])/[aa],2)
%[d]>0 and [pr2].is_integer()==True
Rozwiązać nierówno\'sć $[p]^{\log_{\frac{1}{[p]}}{([a]x^2-[b]x-[c])}}<1$
\zadStop
\rozwStart{Małgorzata Ugowska}{}
$$[p]^{\log_{\frac{1}{[p]}}{([a]x^2-[b]x-[c])}}<1 \quad \Longleftrightarrow \quad [p]^{\log_{[p]^{-1}}{([a]x^2-[b]x-[c])}}<1 $$
$$\Longleftrightarrow \quad [p]^{-1 \log_{[p]}{([a]x^2-[b]x-[c])}}<1 \quad \Longleftrightarrow \quad [p]^{\log_{[p]}{([a]x^2-[b]x-[c])^{-1}}}<1$$
$$\quad \Longleftrightarrow \quad \frac{1}{[a]x^2-[b]x-[c]}<1 \quad \Longleftrightarrow \quad [a]x^2-[b]x-[c]>1$$
$$\quad \Longleftrightarrow \quad [a]x^2-[b]x-[c2]>0$$
Szukamy miejsc zerowych funkcji $[a]x^2-[b]x-[c2]$:
$$ \bigtriangleup = [b]^2-4 \cdot [a] \cdot [c2] = [d] \Longrightarrow x_1=\frac{[b]-\sqrt{\bigtriangleup}}{[aa]} = [z1], \quad x_2=\frac{[b]+\sqrt{\bigtriangleup}}{[aa]} = [z2]$$
$$[a]x^2-[b]x-[c2]>0 \quad \Longleftrightarrow \quad (x-[z1])(x-[z2])>0 \quad \Longleftrightarrow \quad x \in (-\infty, [z1]) \cup ([z2],\infty)$$
\rozwStop
\odpStart
$x \in (-\infty, [z1]) \cup ([z2],\infty)$
\odpStop
\testStart
A. $x \in ([z1],[z2])$\\
B. $x \in (-\infty, [z1]) \cup ([z2],\infty)$\\
C. $x \in (-\infty, -4) \cup (3,\infty)$\\
D. $x \in (-4,3)$\\
E. $x \in \emptyset$
\testStop
\kluczStart
B
\kluczStop



\end{document}