\documentclass[12pt, a4paper]{article}
\usepackage[utf8]{inputenc}
\usepackage{polski}

\usepackage{amsthm}  %pakiet do tworzenia twierdzeń itp.
\usepackage{amsmath} %pakiet do niektórych symboli matematycznych
\usepackage{amssymb} %pakiet do symboli mat., np. \nsubseteq
\usepackage{amsfonts}
\usepackage{graphicx} %obsługa plików graficznych z rozszerzeniem png, jpg
\theoremstyle{definition} %styl dla definicji
\newtheorem{zad}{} 
\title{Multizestaw zadań}
\author{Laura Mieczkowska}
%\date{\today}
\date{}
\newcounter{liczniksekcji}
\newcommand{\kategoria}[1]{\section{#1}} %olreślamy nazwę kateforii zadań
\newcommand{\zadStart}[1]{\begin{zad}#1\newline} %oznaczenie początku zadania
\newcommand{\zadStop}{\end{zad}}   %oznaczenie końca zadania
%Makra opcjonarne (nie muszą występować):
\newcommand{\rozwStart}[2]{\noindent \textbf{Rozwiązanie (autor #1 , recenzent #2): }\newline} %oznaczenie początku rozwiązania, opcjonarnie można wprowadzić informację o autorze rozwiązania zadania i recenzencie poprawności wykonania rozwiązania zadania
\newcommand{\rozwStop}{\newline}                                            %oznaczenie końca rozwiązania
\newcommand{\odpStart}{\noindent \textbf{Odpowiedź:}\newline}    %oznaczenie początku odpowiedzi końcowej (wypisanie wyniku)
\newcommand{\odpStop}{\newline}                                             %oznaczenie końca odpowiedzi końcowej (wypisanie wyniku)
\newcommand{\testStart}{\noindent \textbf{Test:}\newline} %ewentualne możliwe opcje odpowiedzi testowej: A. ? B. ? C. ? D. ? itd.
\newcommand{\testStop}{\newline} %koniec wprowadzania odpowiedzi testowych
\newcommand{\kluczStart}{\noindent \textbf{Test poprawna odpowiedź:}\newline} %klucz, poprawna odpowiedź pytania testowego (jedna literka): A lub B lub C lub D itd.
\newcommand{\kluczStop}{\newline} %koniec poprawnej odpowiedzi pytania testowego 
\newcommand{\wstawGrafike}[2]{\begin{figure}[h] \includegraphics[scale=#2] {#1} \end{figure}} %gdyby była potrzeba wstawienia obrazka, parametry: nazwa pliku, skala (jak nie wiesz co wpisać, to wpisz 1)

\begin{document}
\maketitle


\kategoria{Wikieł/Z1.33b}
\zadStart{Zadanie z Wikieł Z 1.33 b) moja wersja nr [nrWersji]}
%[a]:[1,2,3,4,5,6,7,8,9,10,11,12,13,14,15]
%[b]:[2,3,4,5,6,7,8,9,10,11,12,13,14,15]
%[d]=2*[a]
%[akw]=[a]**2
%[mian]=abs([akw]-[b])
%[a]<[b] and [akw]<[b] and [mian]>1 and [mian]!=[d] and math.gcd([d],[mian])==1
Podać przykłąd trójmianu kwadratowego o współczynnikach całkowitych, którego pierwiastkami są pary liczb $\frac{1}{[a]-\sqrt{[b]}}$ i $\frac{1}{[a]+\sqrt{[b]}}$.
\zadStop
\rozwStart{Laura Mieczkowska}{}
Dane są pierwiastki równania kwadratowego $x_1=\frac{1}{[a]-\sqrt{[b]}}$ i $x_2=\frac{1}{[a]+\sqrt{[b]}}$.
\\\\Ponieważ
$$(x-x_1)(x-x_2)=0 \Leftrightarrow x^2-(x_1+x_2)x+x_1x_2=0$$
więc obliczamy:
$$x_1+x_2=\frac{1}{[a]-\sqrt{[b]}}+\frac{1}{[a]+\sqrt{[b]}}=\frac{[a]+\sqrt{[b]}+[a]-\sqrt{[b]}}{([a]-\sqrt{[b]})([a]+\sqrt{[b]})}=\frac{[d]}{[akw]-[b]}=-\frac{[d]}{[mian]}$$
$$x_1\cdot x_2=\bigg(\frac{1}{[a]-\sqrt{[b]}}\bigg)\bigg(\frac{1}{[a]+\sqrt{[b]}}\bigg)=\frac{1}{[akw]-[b]}=-\frac{1}{[mian]}$$
\\Ostatecznie
$$x^2+\frac{[d]}{[mian]}x-\frac{1}{[mian]}$$
\odpStart
$x^2+\frac{[d]}{[mian]}x-\frac{1}{[mian]}$
\odpStop
\testStart
A. $x^2+\frac{[d]}{[mian]}x+\frac{1}{[mian]}$ \\
B. $-x^2+\frac{[d]}{[mian]}x-\frac{1}{[mian]}$ \\
C. $x^2+\frac{[d]}{[mian]}x-\frac{1}{[mian]}$ \\
D. $x^2-\frac{[d]}{[mian]}x-\frac{1}{[mian]}$ 
\testStop
\kluczStart
C
\kluczStop



\end{document}