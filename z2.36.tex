\documentclass[12pt, a4paper]{article}
\usepackage[utf8]{inputenc}
\usepackage{polski}

\usepackage{amsthm}  %pakiet do tworzenia twierdzeń itp.
\usepackage{amsmath} %pakiet do niektórych symboli matematycznych
\usepackage{amssymb} %pakiet do symboli mat., np. \nsubseteq
\usepackage{amsfonts}
\usepackage{graphicx} %obsługa plików graficznych z rozszerzeniem png, jpg
\theoremstyle{definition} %styl dla definicji
\newtheorem{zad}{} 
\title{Multizestaw zadań}
\author{Robert Fidytek}
%\date{\today}
\date{}\documentclass[12pt, a4paper]{article}
\usepackage[utf8]{inputenc}
\usepackage{polski}

\usepackage{amsthm}  %pakiet do tworzenia twierdzeń itp.
\usepackage{amsmath} %pakiet do niektórych symboli matematycznych
\usepackage{amssymb} %pakiet do symboli mat., np. \nsubseteq
\usepackage{amsfonts}
\usepackage{graphicx} %obsługa plików graficznych z rozszerzeniem png, jpg
\theoremstyle{definition} %styl dla definicji
\newtheorem{zad}{} 
\title{Multizestaw zadań}
\author{Robert Fidytek}
%\date{\today}
\date{}
\newcounter{liczniksekcji}
\newcommand{\kategoria}[1]{\section{#1}} %olreślamy nazwę kateforii zadań
\newcommand{\zadStart}[1]{\begin{zad}#1\newline} %oznaczenie początku zadania
\newcommand{\zadStop}{\end{zad}}   %oznaczenie końca zadania
%Makra opcjonarne (nie muszą występować):
\newcommand{\rozwStart}[2]{\noindent \textbf{Rozwiązanie (autor #1 , recenzent #2): }\newline} %oznaczenie początku rozwiązania, opcjonarnie można wprowadzić informację o autorze rozwiązania zadania i recenzencie poprawności wykonania rozwiązania zadania
\newcommand{\rozwStop}{\newline}                                            %oznaczenie końca rozwiązania
\newcommand{\odpStart}{\noindent \textbf{Odpowiedź:}\newline}    %oznaczenie początku odpowiedzi końcowej (wypisanie wyniku)
\newcommand{\odpStop}{\newline}                                             %oznaczenie końca odpowiedzi końcowej (wypisanie wyniku)
\newcommand{\testStart}{\noindent \textbf{Test:}\newline} %ewentualne możliwe opcje odpowiedzi testowej: A. ? B. ? C. ? D. ? itd.
\newcommand{\testStop}{\newline} %koniec wprowadzania odpowiedzi testowych
\newcommand{\kluczStart}{\noindent \textbf{Test poprawna odpowiedź:}\newline} %klucz, poprawna odpowiedź pytania testowego (jedna literka): A lub B lub C lub D itd.
\newcommand{\kluczStop}{\newline} %koniec poprawnej odpowiedzi pytania testowego 
\newcommand{\wstawGrafike}[2]{\begin{figure}[h] \includegraphics[scale=#2] {#1} \end{figure}} %gdyby była potrzeba wstawienia obrazka, parametry: nazwa pliku, skala (jak nie wiesz co wpisać, to wpisz 1)

\begin{document}
\maketitle


\kategoria{Wikieł/Z2.36}
\zadStart{Zadanie z Wikieł Z 2.36 moja wersja nr [nrWersji]}
%[p1]:[1,2,3,4,5]
%[p2]:[2,3,4,5,6,7,8,9,10]
%[p3]:[1,2,3,4,5]
%[p4]:[2,3,4,5,6,7,8,9,10]
%[p5]:[1,2,3,4,5]
%[p6]=random.randint(2,10)
%[p7]=random.randint(1,10)
%[p8]=random.randint(2,10)
%[p3p2]=[p3]*[p2]
%[a]=[p3p2]+[p1]
%[d]=math.gcd([a],[p2])
%[np2]=int([p2]/[d])
%[na]=int([a]/[d])
%[p7p6]=[p7]*[p6]
%[b]=[p7p6]+[p5]
%[d2]=math.gcd([b],[p6])
%[nb]=int([b]/[d2])
%[nd]=int([p6]/[d2])
%[nbp6]=[nb]*[p6]
%[bb]=[nbp6]/[nd]
%([p8]/[p6])==([p4]/[p2]) and [nb]==[na] and [np2]==[nd] 

Wykazać, że proste w postaci parametrycznej $l:\left\{ \begin{array}{ll}
x=[p1]+[p2]t \\
y=[p3]+[p4]t  
\end{array} \right  t\in\mathbb{R},
k:\left\{ \begin{array}{ll}
x=[p5]+[p6]s \\
y=[p7]-[p8]s  
\end{array} \right  s\in\mathbb{R}.$ Opisują jedną prostą. Napisać w najprostszej postaci równanie ogólne tej prostej.

\zadStop

\rozwStart{Maja Szabłowska}{}
$$x=[p1]+[p2]t \iff x-[p1]=[p2]t \iff t=\frac{x-[p1]}{[p2]}$$
$$ y=[p3]-[p4]\cdot\frac{x-[p1]}{[p2]}=-\frac{[p4]}{[p2]}x+[p3]+\frac{[p1]}{[p2]}=-\frac{[p4]}{[p2]}x+\frac{[p3p2]+[p1]}{[p2]}$$
$$l: y=-\frac{[p4]}{[p2]}x+\frac{[na]}{[np2]}$$

$$x=[p5]+[p6]s \iff x-[p5]=[p6]s \iff s=\frac{x-[p5]}{[p6]}$$
$$y=[p7]-[p8]\cdot\frac{x-[p5]}{[p6]}=-\frac{[p8]}{[p6]}x+[p7]+\frac{[p5]}{[p6]}=-\frac{[p8]}{[p6]}x+\frac{[p7p6]+[p5]}{[p6]}$$
$$k: y=-\frac{[p8]}{[p6]}x+\frac{[nb]}{[nd]}$$

Równanie ogólne:
$$[p6]y=-[p8]x+\frac{[nb]\cdot[p6]}{[nd]}$$
$$[p6]y+[p8]x-[bb]=0$$
\rozwStop


\odpStart
$l:\left\{ \begin{array}{ll}
x=\frac{[p4]}{[p2]}t\\
y=t+\frac{[b]}{[np4]}
\end{array} \right  t\in\mathbb{R}$
\odpStop
\testStart
A.$l:\left\{ \begin{array}{ll}
x=\frac{[p4]}{[p2]}t\\
y=t+\frac{[b]}{[np4]}
\end{array} \right  t\in\mathbb{R}.$
B.$l:\left\{ \begin{array}{ll}
x=\frac{[p4]}{[p2]}\\
y=t
\end{array} \right  t\in\mathbb{R}.$
D.$l:\left\{ \begin{array}{ll}
x=\frac{[p4]}{[p3]}t\\
y=\frac{[b]}{[np4]}
\end{array} \right  t\in\mathbb{R}.$
E.$l:\left\{ \begin{array}{ll}
x=\frac{[p3]}{[p2]}t\\
y=t-\frac{[b]}{[np4]}
\end{array} \right  t\in\mathbb{R}.$
F.$l:\left\{ \begin{array}{ll}
x=[p4]t\\
y=-t-\frac{[b]}{[np4]}
\end{array} \right  t\in\mathbb{R}.$
G.$l:\left\{ \begin{array}{ll}
x=\frac{[p4]}{[p2]}\\
y=t-\frac{[b]}{[np4]}
\end{array} \right  t\in\mathbb{R}.$
H.$l:\left\{ \begin{array}{ll}
x=[p1]t+[p2]\\
y=t+\frac{[b]}{[np4]}
\end{array} \right  t\in\mathbb{R}.$
\testStop
\kluczStart
A
\kluczStop



\end{document}
