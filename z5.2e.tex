\documentclass[12pt, a4paper]{article}
\usepackage[utf8]{inputenc}
\usepackage{polski}

\usepackage{amsthm}  %pakiet do tworzenia twierdzeń itp.
\usepackage{amsmath} %pakiet do niektórych symboli matematycznych
\usepackage{amssymb} %pakiet do symboli mat., np. \nsubseteq
\usepackage{amsfonts}
\usepackage{graphicx} %obsługa plików graficznych z rozszerzeniem png, jpg
\theoremstyle{definition} %styl dla definicji
\newtheorem{zad}{} 
\title{Multizestaw zadań}
\author{Robert Fidytek}
%\date{\today}
\date{}
\newcounter{liczniksekcji}
\newcommand{\kategoria}[1]{\section{#1}} %olreślamy nazwę kateforii zadań
\newcommand{\zadStart}[1]{\begin{zad}#1\newline} %oznaczenie początku zadania
\newcommand{\zadStop}{\end{zad}}   %oznaczenie końca zadania
%Makra opcjonarne (nie muszą występować):
\newcommand{\rozwStart}[2]{\noindent \textbf{Rozwiązanie (autor #1 , recenzent #2): }\newline} %oznaczenie początku rozwiązania, opcjonarnie można wprowadzić informację o autorze rozwiązania zadania i recenzencie poprawności wykonania rozwiązania zadania
\newcommand{\rozwStop}{\newline}                                            %oznaczenie końca rozwiązania
\newcommand{\odpStart}{\noindent \textbf{Odpowiedź:}\newline}    %oznaczenie początku odpowiedzi końcowej (wypisanie wyniku)
\newcommand{\odpStop}{\newline}                                             %oznaczenie końca odpowiedzi końcowej (wypisanie wyniku)
\newcommand{\testStart}{\noindent \textbf{Test:}\newline} %ewentualne możliwe opcje odpowiedzi testowej: A. ? B. ? C. ? D. ? itd.
\newcommand{\testStop}{\newline} %koniec wprowadzania odpowiedzi testowych
\newcommand{\kluczStart}{\noindent \textbf{Test poprawna odpowiedź:}\newline} %klucz, poprawna odpowiedź pytania testowego (jedna literka): A lub B lub C lub D itd.
\newcommand{\kluczStop}{\newline} %koniec poprawnej odpowiedzi pytania testowego 
\newcommand{\wstawGrafike}[2]{\begin{figure}[h] \includegraphics[scale=#2] {#1} \end{figure}} %gdyby była potrzeba wstawienia obrazka, parametry: nazwa pliku, skala (jak nie wiesz co wpisać, to wpisz 1)

\begin{document}
\maketitle


\kategoria{Wikieł/Z5.2e}
\zadStart{Zadanie z Wikieł Z 5.2 e) moja wersja nr [nrWersji]}
%[a]:[2,3,4,5,6,7,8,9]
%[b]:[2,3,4,5,6,7,8,9]
%math.gcd([b],[a])==1
Korzystając z definicji pochodnej , oblicz wartość wskazanej pochodnej \\ $f(x)=\sqrt{[b]x}, \ \ \ f'([a])$.
\zadStop
\rozwStart{Joanna Świerzbin}{}
$$f(x)=\sqrt{[b]x}$$
$$f'(x)=\lim_{\Delta x \rightarrow 0} \frac{f(x+\Delta x)-f(x)}{\Delta x} = $$ 
$$ =\lim_{\Delta x \rightarrow 0} \frac{\sqrt{[b](x+\Delta x)}-\sqrt{[b]x}}{\Delta x} = $$
$$ =\lim_{\Delta x \rightarrow 0} \frac{\sqrt{[b]}(\sqrt{x+\Delta x}-\sqrt{x})}{\Delta x} = $$
$$ =\lim_{\Delta x \rightarrow 0} \frac{\sqrt{[b]}(\sqrt{x+\Delta x}-\sqrt{x})(\sqrt{x+\Delta x}+\sqrt{x})}{\Delta x(\sqrt{x+\Delta x}+\sqrt{x})} = $$
$$ =\lim_{\Delta x \rightarrow 0} \frac{\sqrt{[b]}(x+\Delta x-x)}{\Delta x(\sqrt{x+\Delta x}+\sqrt{x})} = $$
$$ =\lim_{\Delta x \rightarrow 0} \frac{\sqrt{[b]}\Delta x}{\Delta x (\sqrt{x+\Delta x}+\sqrt{x})} = $$
$$ =\lim_{\Delta x \rightarrow 0} \frac{\sqrt{[b]}}{\sqrt{x+\Delta x}+\sqrt{x}}  = \frac{\sqrt{[b]}}{2\sqrt{x}} $$
$$f'([a])=\frac{\sqrt{[b]}}{2\sqrt{[a]}} $$
\rozwStop
\odpStart
$f'([a])=\frac{\sqrt{[b]}}{2\sqrt{[a]}} $
\odpStop
\testStart
A.$f'([a])=\frac{\sqrt{[b]}}{2\sqrt{[a]}} $\\
B. $f'([a])=1$ \\
C. $f'([a])=[a]$  \\
D. $f'([a])=0$\\
E. $f'([a])=[a]\sqrt{[b]}$\\
F. $f'([a])=[a]x^2$
\testStop
\kluczStart
A
\kluczStop



\end{document}