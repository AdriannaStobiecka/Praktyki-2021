\documentclass[12pt, a4paper]{article}
\usepackage[utf8]{inputenc}
\usepackage{polski}
\usepackage{amsthm}  %pakiet do tworzenia twierdzeń itp.
\usepackage{amsmath} %pakiet do niektórych symboli matematycznych
\usepackage{amssymb} %pakiet do symboli mat., np. \nsubseteq
\usepackage{amsfonts}
\usepackage{graphicx} %obsługa plików graficznych z rozszerzeniem png, jpg
\theoremstyle{definition} %styl dla definicji
\newtheorem{zad}{} 
\title{Multizestaw zadań}
\author{Patryk Wirkus}
%\date{\today}
\date{}
\newcommand{\kategoria}[1]{\section{#1}}
\newcommand{\zadStart}[1]{\begin{zad}#1\newline}
\newcommand{\zadStop}{\end{zad}}
\newcommand{\rozwStart}[2]{\noindent \textbf{Rozwiązanie (autor #1 , recenzent #2): }\newline}
\newcommand{\rozwStop}{\newline}                                           
\newcommand{\odpStart}{\noindent \textbf{Odpowiedź:}\newline}
\newcommand{\odpStop}{\newline}
\newcommand{\testStart}{\noindent \textbf{Test:}\newline}
\newcommand{\testStop}{\newline}
\newcommand{\kluczStart}{\noindent \textbf{Test poprawna odpowiedź:}\newline}
\newcommand{\kluczStop}{\newline}
\newcommand{\wstawGrafike}[2]{\begin{figure}[h] \includegraphics[scale=#2] {#1} \end{figure}}

\begin{document}
\maketitle

\kategoria{Wikieł/Z1.16b}


\zadStart{Zadanie z Wikieł Z 1.16 b) moja wersja nr 1}

Obliczyć symbol Newtona ${103 \choose 63}$.
\zadStop
\rozwStart{Patryk Wirkus}{Szymon Tokarski}
$${103 \choose 63} = \frac{103!}{(103-63)! \cdot 63!} = \frac{103!}{40! \cdot 63!}$$
\rozwStop
\odpStart
$\frac{103!}{40! \cdot 63!}$
\odpStop
\testStart
A.$\frac{103!}{40! \cdot 63!}$ B.$-\frac{103!}{40! \cdot 63!}$ C.$0$ D.$\frac{103}{63}$ E.$\frac{63}{103}$
F.$\frac{103}{2520}$ G.$\frac{2520}{103}$
H.$-\frac{103}{63}$
I.$-\frac{103}{2520}$
\testStop
\kluczStart
A
\kluczStop



\zadStart{Zadanie z Wikieł Z 1.16 b) moja wersja nr 2}

Obliczyć symbol Newtona ${107 \choose 63}$.
\zadStop
\rozwStart{Patryk Wirkus}{Szymon Tokarski}
$${107 \choose 63} = \frac{107!}{(107-63)! \cdot 63!} = \frac{107!}{44! \cdot 63!}$$
\rozwStop
\odpStart
$\frac{107!}{44! \cdot 63!}$
\odpStop
\testStart
A.$\frac{107!}{44! \cdot 63!}$ B.$-\frac{107!}{44! \cdot 63!}$ C.$0$ D.$\frac{107}{63}$ E.$\frac{63}{107}$
F.$\frac{107}{2772}$ G.$\frac{2772}{107}$
H.$-\frac{107}{63}$
I.$-\frac{107}{2772}$
\testStop
\kluczStart
A
\kluczStop



\zadStart{Zadanie z Wikieł Z 1.16 b) moja wersja nr 3}

Obliczyć symbol Newtona ${109 \choose 63}$.
\zadStop
\rozwStart{Patryk Wirkus}{Szymon Tokarski}
$${109 \choose 63} = \frac{109!}{(109-63)! \cdot 63!} = \frac{109!}{46! \cdot 63!}$$
\rozwStop
\odpStart
$\frac{109!}{46! \cdot 63!}$
\odpStop
\testStart
A.$\frac{109!}{46! \cdot 63!}$ B.$-\frac{109!}{46! \cdot 63!}$ C.$0$ D.$\frac{109}{63}$ E.$\frac{63}{109}$
F.$\frac{109}{2898}$ G.$\frac{2898}{109}$
H.$-\frac{109}{63}$
I.$-\frac{109}{2898}$
\testStop
\kluczStart
A
\kluczStop



\zadStart{Zadanie z Wikieł Z 1.16 b) moja wersja nr 4}

Obliczyć symbol Newtona ${113 \choose 63}$.
\zadStop
\rozwStart{Patryk Wirkus}{Szymon Tokarski}
$${113 \choose 63} = \frac{113!}{(113-63)! \cdot 63!} = \frac{113!}{50! \cdot 63!}$$
\rozwStop
\odpStart
$\frac{113!}{50! \cdot 63!}$
\odpStop
\testStart
A.$\frac{113!}{50! \cdot 63!}$ B.$-\frac{113!}{50! \cdot 63!}$ C.$0$ D.$\frac{113}{63}$ E.$\frac{63}{113}$
F.$\frac{113}{3150}$ G.$\frac{3150}{113}$
H.$-\frac{113}{63}$
I.$-\frac{113}{3150}$
\testStop
\kluczStart
A
\kluczStop



\zadStart{Zadanie z Wikieł Z 1.16 b) moja wersja nr 5}

Obliczyć symbol Newtona ${103 \choose 67}$.
\zadStop
\rozwStart{Patryk Wirkus}{Szymon Tokarski}
$${103 \choose 67} = \frac{103!}{(103-67)! \cdot 67!} = \frac{103!}{36! \cdot 67!}$$
\rozwStop
\odpStart
$\frac{103!}{36! \cdot 67!}$
\odpStop
\testStart
A.$\frac{103!}{36! \cdot 67!}$ B.$-\frac{103!}{36! \cdot 67!}$ C.$0$ D.$\frac{103}{67}$ E.$\frac{67}{103}$
F.$\frac{103}{2412}$ G.$\frac{2412}{103}$
H.$-\frac{103}{67}$
I.$-\frac{103}{2412}$
\testStop
\kluczStart
A
\kluczStop



\zadStart{Zadanie z Wikieł Z 1.16 b) moja wersja nr 6}

Obliczyć symbol Newtona ${107 \choose 67}$.
\zadStop
\rozwStart{Patryk Wirkus}{Szymon Tokarski}
$${107 \choose 67} = \frac{107!}{(107-67)! \cdot 67!} = \frac{107!}{40! \cdot 67!}$$
\rozwStop
\odpStart
$\frac{107!}{40! \cdot 67!}$
\odpStop
\testStart
A.$\frac{107!}{40! \cdot 67!}$ B.$-\frac{107!}{40! \cdot 67!}$ C.$0$ D.$\frac{107}{67}$ E.$\frac{67}{107}$
F.$\frac{107}{2680}$ G.$\frac{2680}{107}$
H.$-\frac{107}{67}$
I.$-\frac{107}{2680}$
\testStop
\kluczStart
A
\kluczStop



\zadStart{Zadanie z Wikieł Z 1.16 b) moja wersja nr 7}

Obliczyć symbol Newtona ${109 \choose 67}$.
\zadStop
\rozwStart{Patryk Wirkus}{Szymon Tokarski}
$${109 \choose 67} = \frac{109!}{(109-67)! \cdot 67!} = \frac{109!}{42! \cdot 67!}$$
\rozwStop
\odpStart
$\frac{109!}{42! \cdot 67!}$
\odpStop
\testStart
A.$\frac{109!}{42! \cdot 67!}$ B.$-\frac{109!}{42! \cdot 67!}$ C.$0$ D.$\frac{109}{67}$ E.$\frac{67}{109}$
F.$\frac{109}{2814}$ G.$\frac{2814}{109}$
H.$-\frac{109}{67}$
I.$-\frac{109}{2814}$
\testStop
\kluczStart
A
\kluczStop



\zadStart{Zadanie z Wikieł Z 1.16 b) moja wersja nr 8}

Obliczyć symbol Newtona ${113 \choose 67}$.
\zadStop
\rozwStart{Patryk Wirkus}{Szymon Tokarski}
$${113 \choose 67} = \frac{113!}{(113-67)! \cdot 67!} = \frac{113!}{46! \cdot 67!}$$
\rozwStop
\odpStart
$\frac{113!}{46! \cdot 67!}$
\odpStop
\testStart
A.$\frac{113!}{46! \cdot 67!}$ B.$-\frac{113!}{46! \cdot 67!}$ C.$0$ D.$\frac{113}{67}$ E.$\frac{67}{113}$
F.$\frac{113}{3082}$ G.$\frac{3082}{113}$
H.$-\frac{113}{67}$
I.$-\frac{113}{3082}$
\testStop
\kluczStart
A
\kluczStop



\zadStart{Zadanie z Wikieł Z 1.16 b) moja wersja nr 9}

Obliczyć symbol Newtona ${103 \choose 73}$.
\zadStop
\rozwStart{Patryk Wirkus}{Szymon Tokarski}
$${103 \choose 73} = \frac{103!}{(103-73)! \cdot 73!} = \frac{103!}{30! \cdot 73!}$$
\rozwStop
\odpStart
$\frac{103!}{30! \cdot 73!}$
\odpStop
\testStart
A.$\frac{103!}{30! \cdot 73!}$ B.$-\frac{103!}{30! \cdot 73!}$ C.$0$ D.$\frac{103}{73}$ E.$\frac{73}{103}$
F.$\frac{103}{2190}$ G.$\frac{2190}{103}$
H.$-\frac{103}{73}$
I.$-\frac{103}{2190}$
\testStop
\kluczStart
A
\kluczStop



\zadStart{Zadanie z Wikieł Z 1.16 b) moja wersja nr 10}

Obliczyć symbol Newtona ${107 \choose 73}$.
\zadStop
\rozwStart{Patryk Wirkus}{Szymon Tokarski}
$${107 \choose 73} = \frac{107!}{(107-73)! \cdot 73!} = \frac{107!}{34! \cdot 73!}$$
\rozwStop
\odpStart
$\frac{107!}{34! \cdot 73!}$
\odpStop
\testStart
A.$\frac{107!}{34! \cdot 73!}$ B.$-\frac{107!}{34! \cdot 73!}$ C.$0$ D.$\frac{107}{73}$ E.$\frac{73}{107}$
F.$\frac{107}{2482}$ G.$\frac{2482}{107}$
H.$-\frac{107}{73}$
I.$-\frac{107}{2482}$
\testStop
\kluczStart
A
\kluczStop



\zadStart{Zadanie z Wikieł Z 1.16 b) moja wersja nr 11}

Obliczyć symbol Newtona ${109 \choose 73}$.
\zadStop
\rozwStart{Patryk Wirkus}{Szymon Tokarski}
$${109 \choose 73} = \frac{109!}{(109-73)! \cdot 73!} = \frac{109!}{36! \cdot 73!}$$
\rozwStop
\odpStart
$\frac{109!}{36! \cdot 73!}$
\odpStop
\testStart
A.$\frac{109!}{36! \cdot 73!}$ B.$-\frac{109!}{36! \cdot 73!}$ C.$0$ D.$\frac{109}{73}$ E.$\frac{73}{109}$
F.$\frac{109}{2628}$ G.$\frac{2628}{109}$
H.$-\frac{109}{73}$
I.$-\frac{109}{2628}$
\testStop
\kluczStart
A
\kluczStop



\zadStart{Zadanie z Wikieł Z 1.16 b) moja wersja nr 12}

Obliczyć symbol Newtona ${113 \choose 73}$.
\zadStop
\rozwStart{Patryk Wirkus}{Szymon Tokarski}
$${113 \choose 73} = \frac{113!}{(113-73)! \cdot 73!} = \frac{113!}{40! \cdot 73!}$$
\rozwStop
\odpStart
$\frac{113!}{40! \cdot 73!}$
\odpStop
\testStart
A.$\frac{113!}{40! \cdot 73!}$ B.$-\frac{113!}{40! \cdot 73!}$ C.$0$ D.$\frac{113}{73}$ E.$\frac{73}{113}$
F.$\frac{113}{2920}$ G.$\frac{2920}{113}$
H.$-\frac{113}{73}$
I.$-\frac{113}{2920}$
\testStop
\kluczStart
A
\kluczStop



\zadStart{Zadanie z Wikieł Z 1.16 b) moja wersja nr 13}

Obliczyć symbol Newtona ${103 \choose 79}$.
\zadStop
\rozwStart{Patryk Wirkus}{Szymon Tokarski}
$${103 \choose 79} = \frac{103!}{(103-79)! \cdot 79!} = \frac{103!}{24! \cdot 79!}$$
\rozwStop
\odpStart
$\frac{103!}{24! \cdot 79!}$
\odpStop
\testStart
A.$\frac{103!}{24! \cdot 79!}$ B.$-\frac{103!}{24! \cdot 79!}$ C.$0$ D.$\frac{103}{79}$ E.$\frac{79}{103}$
F.$\frac{103}{1896}$ G.$\frac{1896}{103}$
H.$-\frac{103}{79}$
I.$-\frac{103}{1896}$
\testStop
\kluczStart
A
\kluczStop



\zadStart{Zadanie z Wikieł Z 1.16 b) moja wersja nr 14}

Obliczyć symbol Newtona ${107 \choose 79}$.
\zadStop
\rozwStart{Patryk Wirkus}{Szymon Tokarski}
$${107 \choose 79} = \frac{107!}{(107-79)! \cdot 79!} = \frac{107!}{28! \cdot 79!}$$
\rozwStop
\odpStart
$\frac{107!}{28! \cdot 79!}$
\odpStop
\testStart
A.$\frac{107!}{28! \cdot 79!}$ B.$-\frac{107!}{28! \cdot 79!}$ C.$0$ D.$\frac{107}{79}$ E.$\frac{79}{107}$
F.$\frac{107}{2212}$ G.$\frac{2212}{107}$
H.$-\frac{107}{79}$
I.$-\frac{107}{2212}$
\testStop
\kluczStart
A
\kluczStop



\zadStart{Zadanie z Wikieł Z 1.16 b) moja wersja nr 15}

Obliczyć symbol Newtona ${109 \choose 79}$.
\zadStop
\rozwStart{Patryk Wirkus}{Szymon Tokarski}
$${109 \choose 79} = \frac{109!}{(109-79)! \cdot 79!} = \frac{109!}{30! \cdot 79!}$$
\rozwStop
\odpStart
$\frac{109!}{30! \cdot 79!}$
\odpStop
\testStart
A.$\frac{109!}{30! \cdot 79!}$ B.$-\frac{109!}{30! \cdot 79!}$ C.$0$ D.$\frac{109}{79}$ E.$\frac{79}{109}$
F.$\frac{109}{2370}$ G.$\frac{2370}{109}$
H.$-\frac{109}{79}$
I.$-\frac{109}{2370}$
\testStop
\kluczStart
A
\kluczStop



\zadStart{Zadanie z Wikieł Z 1.16 b) moja wersja nr 16}

Obliczyć symbol Newtona ${113 \choose 79}$.
\zadStop
\rozwStart{Patryk Wirkus}{Szymon Tokarski}
$${113 \choose 79} = \frac{113!}{(113-79)! \cdot 79!} = \frac{113!}{34! \cdot 79!}$$
\rozwStop
\odpStart
$\frac{113!}{34! \cdot 79!}$
\odpStop
\testStart
A.$\frac{113!}{34! \cdot 79!}$ B.$-\frac{113!}{34! \cdot 79!}$ C.$0$ D.$\frac{113}{79}$ E.$\frac{79}{113}$
F.$\frac{113}{2686}$ G.$\frac{2686}{113}$
H.$-\frac{113}{79}$
I.$-\frac{113}{2686}$
\testStop
\kluczStart
A
\kluczStop



\zadStart{Zadanie z Wikieł Z 1.16 b) moja wersja nr 17}

Obliczyć symbol Newtona ${103 \choose 51}$.
\zadStop
\rozwStart{Patryk Wirkus}{Szymon Tokarski}
$${103 \choose 51} = \frac{103!}{(103-51)! \cdot 51!} = \frac{103!}{52! \cdot 51!}$$
\rozwStop
\odpStart
$\frac{103!}{52! \cdot 51!}$
\odpStop
\testStart
A.$\frac{103!}{52! \cdot 51!}$ B.$-\frac{103!}{52! \cdot 51!}$ C.$0$ D.$\frac{103}{51}$ E.$\frac{51}{103}$
F.$\frac{103}{2652}$ G.$\frac{2652}{103}$
H.$-\frac{103}{51}$
I.$-\frac{103}{2652}$
\testStop
\kluczStart
A
\kluczStop



\zadStart{Zadanie z Wikieł Z 1.16 b) moja wersja nr 18}

Obliczyć symbol Newtona ${107 \choose 51}$.
\zadStop
\rozwStart{Patryk Wirkus}{Szymon Tokarski}
$${107 \choose 51} = \frac{107!}{(107-51)! \cdot 51!} = \frac{107!}{56! \cdot 51!}$$
\rozwStop
\odpStart
$\frac{107!}{56! \cdot 51!}$
\odpStop
\testStart
A.$\frac{107!}{56! \cdot 51!}$ B.$-\frac{107!}{56! \cdot 51!}$ C.$0$ D.$\frac{107}{51}$ E.$\frac{51}{107}$
F.$\frac{107}{2856}$ G.$\frac{2856}{107}$
H.$-\frac{107}{51}$
I.$-\frac{107}{2856}$
\testStop
\kluczStart
A
\kluczStop



\zadStart{Zadanie z Wikieł Z 1.16 b) moja wersja nr 19}

Obliczyć symbol Newtona ${109 \choose 51}$.
\zadStop
\rozwStart{Patryk Wirkus}{Szymon Tokarski}
$${109 \choose 51} = \frac{109!}{(109-51)! \cdot 51!} = \frac{109!}{58! \cdot 51!}$$
\rozwStop
\odpStart
$\frac{109!}{58! \cdot 51!}$
\odpStop
\testStart
A.$\frac{109!}{58! \cdot 51!}$ B.$-\frac{109!}{58! \cdot 51!}$ C.$0$ D.$\frac{109}{51}$ E.$\frac{51}{109}$
F.$\frac{109}{2958}$ G.$\frac{2958}{109}$
H.$-\frac{109}{51}$
I.$-\frac{109}{2958}$
\testStop
\kluczStart
A
\kluczStop



\zadStart{Zadanie z Wikieł Z 1.16 b) moja wersja nr 20}

Obliczyć symbol Newtona ${113 \choose 51}$.
\zadStop
\rozwStart{Patryk Wirkus}{Szymon Tokarski}
$${113 \choose 51} = \frac{113!}{(113-51)! \cdot 51!} = \frac{113!}{62! \cdot 51!}$$
\rozwStop
\odpStart
$\frac{113!}{62! \cdot 51!}$
\odpStop
\testStart
A.$\frac{113!}{62! \cdot 51!}$ B.$-\frac{113!}{62! \cdot 51!}$ C.$0$ D.$\frac{113}{51}$ E.$\frac{51}{113}$
F.$\frac{113}{3162}$ G.$\frac{3162}{113}$
H.$-\frac{113}{51}$
I.$-\frac{113}{3162}$
\testStop
\kluczStart
A
\kluczStop



\zadStart{Zadanie z Wikieł Z 1.16 b) moja wersja nr 21}

Obliczyć symbol Newtona ${103 \choose 57}$.
\zadStop
\rozwStart{Patryk Wirkus}{Szymon Tokarski}
$${103 \choose 57} = \frac{103!}{(103-57)! \cdot 57!} = \frac{103!}{46! \cdot 57!}$$
\rozwStop
\odpStart
$\frac{103!}{46! \cdot 57!}$
\odpStop
\testStart
A.$\frac{103!}{46! \cdot 57!}$ B.$-\frac{103!}{46! \cdot 57!}$ C.$0$ D.$\frac{103}{57}$ E.$\frac{57}{103}$
F.$\frac{103}{2622}$ G.$\frac{2622}{103}$
H.$-\frac{103}{57}$
I.$-\frac{103}{2622}$
\testStop
\kluczStart
A
\kluczStop



\zadStart{Zadanie z Wikieł Z 1.16 b) moja wersja nr 22}

Obliczyć symbol Newtona ${107 \choose 57}$.
\zadStop
\rozwStart{Patryk Wirkus}{Szymon Tokarski}
$${107 \choose 57} = \frac{107!}{(107-57)! \cdot 57!} = \frac{107!}{50! \cdot 57!}$$
\rozwStop
\odpStart
$\frac{107!}{50! \cdot 57!}$
\odpStop
\testStart
A.$\frac{107!}{50! \cdot 57!}$ B.$-\frac{107!}{50! \cdot 57!}$ C.$0$ D.$\frac{107}{57}$ E.$\frac{57}{107}$
F.$\frac{107}{2850}$ G.$\frac{2850}{107}$
H.$-\frac{107}{57}$
I.$-\frac{107}{2850}$
\testStop
\kluczStart
A
\kluczStop



\zadStart{Zadanie z Wikieł Z 1.16 b) moja wersja nr 23}

Obliczyć symbol Newtona ${109 \choose 57}$.
\zadStop
\rozwStart{Patryk Wirkus}{Szymon Tokarski}
$${109 \choose 57} = \frac{109!}{(109-57)! \cdot 57!} = \frac{109!}{52! \cdot 57!}$$
\rozwStop
\odpStart
$\frac{109!}{52! \cdot 57!}$
\odpStop
\testStart
A.$\frac{109!}{52! \cdot 57!}$ B.$-\frac{109!}{52! \cdot 57!}$ C.$0$ D.$\frac{109}{57}$ E.$\frac{57}{109}$
F.$\frac{109}{2964}$ G.$\frac{2964}{109}$
H.$-\frac{109}{57}$
I.$-\frac{109}{2964}$
\testStop
\kluczStart
A
\kluczStop



\zadStart{Zadanie z Wikieł Z 1.16 b) moja wersja nr 24}

Obliczyć symbol Newtona ${113 \choose 57}$.
\zadStop
\rozwStart{Patryk Wirkus}{Szymon Tokarski}
$${113 \choose 57} = \frac{113!}{(113-57)! \cdot 57!} = \frac{113!}{56! \cdot 57!}$$
\rozwStop
\odpStart
$\frac{113!}{56! \cdot 57!}$
\odpStop
\testStart
A.$\frac{113!}{56! \cdot 57!}$ B.$-\frac{113!}{56! \cdot 57!}$ C.$0$ D.$\frac{113}{57}$ E.$\frac{57}{113}$
F.$\frac{113}{3192}$ G.$\frac{3192}{113}$
H.$-\frac{113}{57}$
I.$-\frac{113}{3192}$
\testStop
\kluczStart
A
\kluczStop



\zadStart{Zadanie z Wikieł Z 1.16 b) moja wersja nr 25}

Obliczyć symbol Newtona ${103 \choose 61}$.
\zadStop
\rozwStart{Patryk Wirkus}{Szymon Tokarski}
$${103 \choose 61} = \frac{103!}{(103-61)! \cdot 61!} = \frac{103!}{42! \cdot 61!}$$
\rozwStop
\odpStart
$\frac{103!}{42! \cdot 61!}$
\odpStop
\testStart
A.$\frac{103!}{42! \cdot 61!}$ B.$-\frac{103!}{42! \cdot 61!}$ C.$0$ D.$\frac{103}{61}$ E.$\frac{61}{103}$
F.$\frac{103}{2562}$ G.$\frac{2562}{103}$
H.$-\frac{103}{61}$
I.$-\frac{103}{2562}$
\testStop
\kluczStart
A
\kluczStop



\zadStart{Zadanie z Wikieł Z 1.16 b) moja wersja nr 26}

Obliczyć symbol Newtona ${107 \choose 61}$.
\zadStop
\rozwStart{Patryk Wirkus}{Szymon Tokarski}
$${107 \choose 61} = \frac{107!}{(107-61)! \cdot 61!} = \frac{107!}{46! \cdot 61!}$$
\rozwStop
\odpStart
$\frac{107!}{46! \cdot 61!}$
\odpStop
\testStart
A.$\frac{107!}{46! \cdot 61!}$ B.$-\frac{107!}{46! \cdot 61!}$ C.$0$ D.$\frac{107}{61}$ E.$\frac{61}{107}$
F.$\frac{107}{2806}$ G.$\frac{2806}{107}$
H.$-\frac{107}{61}$
I.$-\frac{107}{2806}$
\testStop
\kluczStart
A
\kluczStop



\zadStart{Zadanie z Wikieł Z 1.16 b) moja wersja nr 27}

Obliczyć symbol Newtona ${109 \choose 61}$.
\zadStop
\rozwStart{Patryk Wirkus}{Szymon Tokarski}
$${109 \choose 61} = \frac{109!}{(109-61)! \cdot 61!} = \frac{109!}{48! \cdot 61!}$$
\rozwStop
\odpStart
$\frac{109!}{48! \cdot 61!}$
\odpStop
\testStart
A.$\frac{109!}{48! \cdot 61!}$ B.$-\frac{109!}{48! \cdot 61!}$ C.$0$ D.$\frac{109}{61}$ E.$\frac{61}{109}$
F.$\frac{109}{2928}$ G.$\frac{2928}{109}$
H.$-\frac{109}{61}$
I.$-\frac{109}{2928}$
\testStop
\kluczStart
A
\kluczStop



\zadStart{Zadanie z Wikieł Z 1.16 b) moja wersja nr 28}

Obliczyć symbol Newtona ${113 \choose 61}$.
\zadStop
\rozwStart{Patryk Wirkus}{Szymon Tokarski}
$${113 \choose 61} = \frac{113!}{(113-61)! \cdot 61!} = \frac{113!}{52! \cdot 61!}$$
\rozwStop
\odpStart
$\frac{113!}{52! \cdot 61!}$
\odpStop
\testStart
A.$\frac{113!}{52! \cdot 61!}$ B.$-\frac{113!}{52! \cdot 61!}$ C.$0$ D.$\frac{113}{61}$ E.$\frac{61}{113}$
F.$\frac{113}{3172}$ G.$\frac{3172}{113}$
H.$-\frac{113}{61}$
I.$-\frac{113}{3172}$
\testStop
\kluczStart
A
\kluczStop



\zadStart{Zadanie z Wikieł Z 1.16 b) moja wersja nr 29}

Obliczyć symbol Newtona ${103 \choose 69}$.
\zadStop
\rozwStart{Patryk Wirkus}{Szymon Tokarski}
$${103 \choose 69} = \frac{103!}{(103-69)! \cdot 69!} = \frac{103!}{34! \cdot 69!}$$
\rozwStop
\odpStart
$\frac{103!}{34! \cdot 69!}$
\odpStop
\testStart
A.$\frac{103!}{34! \cdot 69!}$ B.$-\frac{103!}{34! \cdot 69!}$ C.$0$ D.$\frac{103}{69}$ E.$\frac{69}{103}$
F.$\frac{103}{2346}$ G.$\frac{2346}{103}$
H.$-\frac{103}{69}$
I.$-\frac{103}{2346}$
\testStop
\kluczStart
A
\kluczStop



\zadStart{Zadanie z Wikieł Z 1.16 b) moja wersja nr 30}

Obliczyć symbol Newtona ${107 \choose 69}$.
\zadStop
\rozwStart{Patryk Wirkus}{Szymon Tokarski}
$${107 \choose 69} = \frac{107!}{(107-69)! \cdot 69!} = \frac{107!}{38! \cdot 69!}$$
\rozwStop
\odpStart
$\frac{107!}{38! \cdot 69!}$
\odpStop
\testStart
A.$\frac{107!}{38! \cdot 69!}$ B.$-\frac{107!}{38! \cdot 69!}$ C.$0$ D.$\frac{107}{69}$ E.$\frac{69}{107}$
F.$\frac{107}{2622}$ G.$\frac{2622}{107}$
H.$-\frac{107}{69}$
I.$-\frac{107}{2622}$
\testStop
\kluczStart
A
\kluczStop



\zadStart{Zadanie z Wikieł Z 1.16 b) moja wersja nr 31}

Obliczyć symbol Newtona ${109 \choose 69}$.
\zadStop
\rozwStart{Patryk Wirkus}{Szymon Tokarski}
$${109 \choose 69} = \frac{109!}{(109-69)! \cdot 69!} = \frac{109!}{40! \cdot 69!}$$
\rozwStop
\odpStart
$\frac{109!}{40! \cdot 69!}$
\odpStop
\testStart
A.$\frac{109!}{40! \cdot 69!}$ B.$-\frac{109!}{40! \cdot 69!}$ C.$0$ D.$\frac{109}{69}$ E.$\frac{69}{109}$
F.$\frac{109}{2760}$ G.$\frac{2760}{109}$
H.$-\frac{109}{69}$
I.$-\frac{109}{2760}$
\testStop
\kluczStart
A
\kluczStop



\zadStart{Zadanie z Wikieł Z 1.16 b) moja wersja nr 32}

Obliczyć symbol Newtona ${113 \choose 69}$.
\zadStop
\rozwStart{Patryk Wirkus}{Szymon Tokarski}
$${113 \choose 69} = \frac{113!}{(113-69)! \cdot 69!} = \frac{113!}{44! \cdot 69!}$$
\rozwStop
\odpStart
$\frac{113!}{44! \cdot 69!}$
\odpStop
\testStart
A.$\frac{113!}{44! \cdot 69!}$ B.$-\frac{113!}{44! \cdot 69!}$ C.$0$ D.$\frac{113}{69}$ E.$\frac{69}{113}$
F.$\frac{113}{3036}$ G.$\frac{3036}{113}$
H.$-\frac{113}{69}$
I.$-\frac{113}{3036}$
\testStop
\kluczStart
A
\kluczStop



\zadStart{Zadanie z Wikieł Z 1.16 b) moja wersja nr 33}

Obliczyć symbol Newtona ${103 \choose 71}$.
\zadStop
\rozwStart{Patryk Wirkus}{Szymon Tokarski}
$${103 \choose 71} = \frac{103!}{(103-71)! \cdot 71!} = \frac{103!}{32! \cdot 71!}$$
\rozwStop
\odpStart
$\frac{103!}{32! \cdot 71!}$
\odpStop
\testStart
A.$\frac{103!}{32! \cdot 71!}$ B.$-\frac{103!}{32! \cdot 71!}$ C.$0$ D.$\frac{103}{71}$ E.$\frac{71}{103}$
F.$\frac{103}{2272}$ G.$\frac{2272}{103}$
H.$-\frac{103}{71}$
I.$-\frac{103}{2272}$
\testStop
\kluczStart
A
\kluczStop



\zadStart{Zadanie z Wikieł Z 1.16 b) moja wersja nr 34}

Obliczyć symbol Newtona ${107 \choose 71}$.
\zadStop
\rozwStart{Patryk Wirkus}{Szymon Tokarski}
$${107 \choose 71} = \frac{107!}{(107-71)! \cdot 71!} = \frac{107!}{36! \cdot 71!}$$
\rozwStop
\odpStart
$\frac{107!}{36! \cdot 71!}$
\odpStop
\testStart
A.$\frac{107!}{36! \cdot 71!}$ B.$-\frac{107!}{36! \cdot 71!}$ C.$0$ D.$\frac{107}{71}$ E.$\frac{71}{107}$
F.$\frac{107}{2556}$ G.$\frac{2556}{107}$
H.$-\frac{107}{71}$
I.$-\frac{107}{2556}$
\testStop
\kluczStart
A
\kluczStop



\zadStart{Zadanie z Wikieł Z 1.16 b) moja wersja nr 35}

Obliczyć symbol Newtona ${109 \choose 71}$.
\zadStop
\rozwStart{Patryk Wirkus}{Szymon Tokarski}
$${109 \choose 71} = \frac{109!}{(109-71)! \cdot 71!} = \frac{109!}{38! \cdot 71!}$$
\rozwStop
\odpStart
$\frac{109!}{38! \cdot 71!}$
\odpStop
\testStart
A.$\frac{109!}{38! \cdot 71!}$ B.$-\frac{109!}{38! \cdot 71!}$ C.$0$ D.$\frac{109}{71}$ E.$\frac{71}{109}$
F.$\frac{109}{2698}$ G.$\frac{2698}{109}$
H.$-\frac{109}{71}$
I.$-\frac{109}{2698}$
\testStop
\kluczStart
A
\kluczStop



\zadStart{Zadanie z Wikieł Z 1.16 b) moja wersja nr 36}

Obliczyć symbol Newtona ${113 \choose 71}$.
\zadStop
\rozwStart{Patryk Wirkus}{Szymon Tokarski}
$${113 \choose 71} = \frac{113!}{(113-71)! \cdot 71!} = \frac{113!}{42! \cdot 71!}$$
\rozwStop
\odpStart
$\frac{113!}{42! \cdot 71!}$
\odpStop
\testStart
A.$\frac{113!}{42! \cdot 71!}$ B.$-\frac{113!}{42! \cdot 71!}$ C.$0$ D.$\frac{113}{71}$ E.$\frac{71}{113}$
F.$\frac{113}{2982}$ G.$\frac{2982}{113}$
H.$-\frac{113}{71}$
I.$-\frac{113}{2982}$
\testStop
\kluczStart
A
\kluczStop



\zadStart{Zadanie z Wikieł Z 1.16 b) moja wersja nr 37}

Obliczyć symbol Newtona ${103 \choose 77}$.
\zadStop
\rozwStart{Patryk Wirkus}{Szymon Tokarski}
$${103 \choose 77} = \frac{103!}{(103-77)! \cdot 77!} = \frac{103!}{26! \cdot 77!}$$
\rozwStop
\odpStart
$\frac{103!}{26! \cdot 77!}$
\odpStop
\testStart
A.$\frac{103!}{26! \cdot 77!}$ B.$-\frac{103!}{26! \cdot 77!}$ C.$0$ D.$\frac{103}{77}$ E.$\frac{77}{103}$
F.$\frac{103}{2002}$ G.$\frac{2002}{103}$
H.$-\frac{103}{77}$
I.$-\frac{103}{2002}$
\testStop
\kluczStart
A
\kluczStop



\zadStart{Zadanie z Wikieł Z 1.16 b) moja wersja nr 38}

Obliczyć symbol Newtona ${107 \choose 77}$.
\zadStop
\rozwStart{Patryk Wirkus}{Szymon Tokarski}
$${107 \choose 77} = \frac{107!}{(107-77)! \cdot 77!} = \frac{107!}{30! \cdot 77!}$$
\rozwStop
\odpStart
$\frac{107!}{30! \cdot 77!}$
\odpStop
\testStart
A.$\frac{107!}{30! \cdot 77!}$ B.$-\frac{107!}{30! \cdot 77!}$ C.$0$ D.$\frac{107}{77}$ E.$\frac{77}{107}$
F.$\frac{107}{2310}$ G.$\frac{2310}{107}$
H.$-\frac{107}{77}$
I.$-\frac{107}{2310}$
\testStop
\kluczStart
A
\kluczStop



\zadStart{Zadanie z Wikieł Z 1.16 b) moja wersja nr 39}

Obliczyć symbol Newtona ${109 \choose 77}$.
\zadStop
\rozwStart{Patryk Wirkus}{Szymon Tokarski}
$${109 \choose 77} = \frac{109!}{(109-77)! \cdot 77!} = \frac{109!}{32! \cdot 77!}$$
\rozwStop
\odpStart
$\frac{109!}{32! \cdot 77!}$
\odpStop
\testStart
A.$\frac{109!}{32! \cdot 77!}$ B.$-\frac{109!}{32! \cdot 77!}$ C.$0$ D.$\frac{109}{77}$ E.$\frac{77}{109}$
F.$\frac{109}{2464}$ G.$\frac{2464}{109}$
H.$-\frac{109}{77}$
I.$-\frac{109}{2464}$
\testStop
\kluczStart
A
\kluczStop



\zadStart{Zadanie z Wikieł Z 1.16 b) moja wersja nr 40}

Obliczyć symbol Newtona ${113 \choose 77}$.
\zadStop
\rozwStart{Patryk Wirkus}{Szymon Tokarski}
$${113 \choose 77} = \frac{113!}{(113-77)! \cdot 77!} = \frac{113!}{36! \cdot 77!}$$
\rozwStop
\odpStart
$\frac{113!}{36! \cdot 77!}$
\odpStop
\testStart
A.$\frac{113!}{36! \cdot 77!}$ B.$-\frac{113!}{36! \cdot 77!}$ C.$0$ D.$\frac{113}{77}$ E.$\frac{77}{113}$
F.$\frac{113}{2772}$ G.$\frac{2772}{113}$
H.$-\frac{113}{77}$
I.$-\frac{113}{2772}$
\testStop
\kluczStart
A
\kluczStop





\end{document}
