\documentclass[12pt, a4paper]{article}
\usepackage[utf8]{inputenc}
\usepackage{polski}

\usepackage{amsthm}  %pakiet do tworzenia twierdzeń itp.
\usepackage{amsmath} %pakiet do niektórych symboli matematycznych
\usepackage{amssymb} %pakiet do symboli mat., np. \nsubseteq
\usepackage{amsfonts}
\usepackage{graphicx} %obsługa plików graficznych z rozszerzeniem png, jpg
\theoremstyle{definition} %styl dla definicji
\newtheorem{zad}{} 
\title{Multizestaw zadań}
\author{Robert Fidytek}
%\date{\today}
\date{}
\newcounter{liczniksekcji}
\newcommand{\kategoria}[1]{\section{#1}} %olreślamy nazwę kateforii zadań
\newcommand{\zadStart}[1]{\begin{zad}#1\newline} %oznaczenie początku zadania
\newcommand{\zadStop}{\end{zad}}   %oznaczenie końca zadania
%Makra opcjonarne (nie muszą występować):
\newcommand{\rozwStart}[2]{\noindent \textbf{Rozwiązanie (autor #1 , recenzent #2): }\newline} %oznaczenie początku rozwiązania, opcjonarnie można wprowadzić informację o autorze rozwiązania zadania i recenzencie poprawności wykonania rozwiązania zadania
\newcommand{\rozwStop}{\newline}                                            %oznaczenie końca rozwiązania
\newcommand{\odpStart}{\noindent \textbf{Odpowiedź:}\newline}    %oznaczenie początku odpowiedzi końcowej (wypisanie wyniku)
\newcommand{\odpStop}{\newline}                                             %oznaczenie końca odpowiedzi końcowej (wypisanie wyniku)
\newcommand{\testStart}{\noindent \textbf{Test:}\newline} %ewentualne możliwe opcje odpowiedzi testowej: A. ? B. ? C. ? D. ? itd.
\newcommand{\testStop}{\newline} %koniec wprowadzania odpowiedzi testowych
\newcommand{\kluczStart}{\noindent \textbf{Test poprawna odpowiedź:}\newline} %klucz, poprawna odpowiedź pytania testowego (jedna literka): A lub B lub C lub D itd.
\newcommand{\kluczStop}{\newline} %koniec poprawnej odpowiedzi pytania testowego 
\newcommand{\wstawGrafike}[2]{\begin{figure}[h] \includegraphics[scale=#2] {#1} \end{figure}} %gdyby była potrzeba wstawienia obrazka, parametry: nazwa pliku, skala (jak nie wiesz co wpisać, to wpisz 1)

\begin{document}
\maketitle


\kategoria{Wikieł/Z3.14a}
\zadStart{Zadanie z Wikieł Z 3.14 a) moja wersja nr [nrWersji]}
%[f]:[1,2,3,4,5,9,10,11,12,13,14,15,16,17,18,19,20,21]
%[z]:[1,2,3,4,5,9,10,11,12,13,14]
%[b]=random.randint(3,70)
%[a]=random.randint(2,[b]-1)
%[c]=random.randint(1,90)
%[a]!=[b] and math.gcd([a],[b])==1
Obliczyć granicę ciągu $a_n= \frac{[a]n + {(-1)}^n}{[b]n+[c]}$.
\zadStop
\rozwStart{Barbara Bączek}{}
$$\lim_{n \rightarrow \infty} a_n= \lim_{n \rightarrow \infty} \frac{[a]n + {(-1)}^n}{[b]n+[c]}=  \lim_{n \rightarrow \infty} \frac{n([a] + \frac{{(-1)}^n}{n})}{n([b]+\frac{[c]}{n})} = (*)$$
Dla $n$ parzystych, czy nieparzystych $$ \lim_{n \rightarrow \infty} \frac{{(-1)}^n}{n} = 0, $$
a zatem mamy $(*)= \frac{[a]}{[b]}$. 
\rozwStop
\odpStart
$\frac{[a]}{[b]}$
\odpStop
\testStart
A.$\infty$
B.$[a]$
C.$-\infty$
D.$0$
E.$\frac{[a]}{[b]}$
G.$[b]$
H.$\frac{[b]}{[a]}$
\testStop
\kluczStart
E
\kluczStop



\end{document}