\documentclass[12pt, a4paper]{article}
\usepackage[utf8]{inputenc}
\usepackage{polski}

\usepackage{amsthm}  %pakiet do tworzenia twierdzeń itp.
\usepackage{amsmath} %pakiet do niektórych symboli matematycznych
\usepackage{amssymb} %pakiet do symboli mat., np. \nsubseteq
\usepackage{amsfonts}
\usepackage{graphicx} %obsługa plików graficznych z rozszerzeniem png, jpg
\theoremstyle{definition} %styl dla definicji
\newtheorem{zad}{} 
\title{Multizestaw zadań}
\author{Robert Fidytek}
%\date{\today}
\date{}
\newcounter{liczniksekcji}
\newcommand{\kategoria}[1]{\section{#1}} %olreślamy nazwę kateforii zadań
\newcommand{\zadStart}[1]{\begin{zad}#1\newline} %oznaczenie początku zadania
\newcommand{\zadStop}{\end{zad}}   %oznaczenie końca zadania
%Makra opcjonarne (nie muszą występować):
\newcommand{\rozwStart}[2]{\noindent \textbf{Rozwiązanie (autor #1 , recenzent #2): }\newline} %oznaczenie początku rozwiązania, opcjonarnie można wprowadzić informację o autorze rozwiązania zadania i recenzencie poprawności wykonania rozwiązania zadania
\newcommand{\rozwStop}{\newline}                                            %oznaczenie końca rozwiązania
\newcommand{\odpStart}{\noindent \textbf{Odpowiedź:}\newline}    %oznaczenie początku odpowiedzi końcowej (wypisanie wyniku)
\newcommand{\odpStop}{\newline}                                             %oznaczenie końca odpowiedzi końcowej (wypisanie wyniku)
\newcommand{\testStart}{\noindent \textbf{Test:}\newline} %ewentualne możliwe opcje odpowiedzi testowej: A. ? B. ? C. ? D. ? itd.
\newcommand{\testStop}{\newline} %koniec wprowadzania odpowiedzi testowych
\newcommand{\kluczStart}{\noindent \textbf{Test poprawna odpowiedź:}\newline} %klucz, poprawna odpowiedź pytania testowego (jedna literka): A lub B lub C lub D itd.
\newcommand{\kluczStop}{\newline} %koniec poprawnej odpowiedzi pytania testowego 
\newcommand{\wstawGrafike}[2]{\begin{figure}[h] \includegraphics[scale=#2] {#1} \end{figure}} %gdyby była potrzeba wstawienia obrazka, parametry: nazwa pliku, skala (jak nie wiesz co wpisać, to wpisz 1)

\begin{document}
\maketitle


\kategoria{Wikieł/Z3.13h}
\zadStart{Zadanie z Wikieł Z 3.13 h) moja wersja nr [nrWersji]}
%[a]:[2,3,4,5,6]
%[b]:[2,3,4,5,6,7,8]
%[f]=random.randint(10,15)
%[d]=random.randint(1,10)
%[e]:[2,3,4,5,6,7,8,9,10,11,12,13,14,15]
%[g]=random.randint(1,10)
%[c]=[a]*[a]*[a]
%[h]=[a]*[a]
%[h1]=int([h]/(math.gcd([h],[e])))
%[e1]=int([e]/(math.gcd([h],[e])))
%[a]!=[e] and [e1]!=1
Obliczyć granicę ciągu $a_n= \frac{([a]n+[b])\sqrt[3]{[c]n^3+[d]}}{([e]n+[f])\sqrt{n^2+[g]}}$.
\zadStop
\rozwStart{Barbara Bączek}{}
$$\lim_{n \rightarrow \infty} a_n= \lim_{n \rightarrow \infty} \frac{([a]n+[b])\sqrt[3]{[c]n^3+[d]}}{([e]n+[f])\sqrt{n^2+[g]}}= $$
$$\lim_{n \rightarrow \infty} \frac{n^2\Big{(}\big{(}[a]+\frac{[b]}{n}\big{)}\sqrt[3]{[c]+\frac{[d]}{n^3}}\Big{)}}{n^2\Big{(}\big{(}[e]+\frac{[f]}{n}\big{)}\sqrt{1+\frac{[g]}{n^2}}\Big{)}}=\lim_{n \rightarrow \infty} \frac{[a] \sqrt[3]{[c]}}{[e]}=\frac{[h1]}{[e1]} $$
\rozwStop
\odpStart
$\frac{[h1]}{[e1]}$
\odpStop
\testStart
A.$\infty$
B.$-\frac{[h1]}{[e1]}$
C.$-\infty$
D.$0$
E.$\frac{[h1]}{[e1]}$
G.$[a]$
H.$\frac{[e1]}{[h1]}$
\testStop
\kluczStart
E
\kluczStop



\end{document}