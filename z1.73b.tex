\documentclass[12pt, a4paper]{article}
\usepackage[utf8]{inputenc}
\usepackage{polski}

\usepackage{amsthm}  %pakiet do tworzenia twierdzeń itp.
\usepackage{amsmath} %pakiet do niektórych symboli matematycznych
\usepackage{amssymb} %pakiet do symboli mat., np. \nsubseteq
\usepackage{amsfonts}
\usepackage{graphicx} %obsługa plików graficznych z rozszerzeniem png, jpg
\theoremstyle{definition} %styl dla definicji
\newtheorem{zad}{} 
\title{Multizestaw zadań}
\author{Robert Fidytek}
%\date{\today}
\date{}
\newcounter{liczniksekcji}
\newcommand{\kategoria}[1]{\section{#1}} %olreślamy nazwę kateforii zadań
\newcommand{\zadStart}[1]{\begin{zad}#1\newline} %oznaczenie początku zadania
\newcommand{\zadStop}{\end{zad}}   %oznaczenie końca zadania
%Makra opcjonarne (nie muszą występować):
\newcommand{\rozwStart}[2]{\noindent \textbf{Rozwiązanie (autor #1 , recenzent #2): }\newline} %oznaczenie początku rozwiązania, opcjonarnie można wprowadzić informację o autorze rozwiązania zadania i recenzencie poprawności wykonania rozwiązania zadania
\newcommand{\rozwStop}{\newline}                                            %oznaczenie końca rozwiązania
\newcommand{\odpStart}{\noindent \textbf{Odpowiedź:}\newline}    %oznaczenie początku odpowiedzi końcowej (wypisanie wyniku)
\newcommand{\odpStop}{\newline}                                             %oznaczenie końca odpowiedzi końcowej (wypisanie wyniku)
\newcommand{\testStart}{\noindent \textbf{Test:}\newline} %ewentualne możliwe opcje odpowiedzi testowej: A. ? B. ? C. ? D. ? itd.
\newcommand{\testStop}{\newline} %koniec wprowadzania odpowiedzi testowych
\newcommand{\kluczStart}{\noindent \textbf{Test poprawna odpowiedź:}\newline} %klucz, poprawna odpowiedź pytania testowego (jedna literka): A lub B lub C lub D itd.
\newcommand{\kluczStop}{\newline} %koniec poprawnej odpowiedzi pytania testowego 
\newcommand{\wstawGrafike}[2]{\begin{figure}[h] \includegraphics[scale=#2] {#1} \end{figure}} %gdyby była potrzeba wstawienia obrazka, parametry: nazwa pliku, skala (jak nie wiesz co wpisać, to wpisz 1)

\begin{document}
\maketitle


\kategoria{Wikieł/Z1.73b}
\zadStart{Zadanie z Wikieł Z 1.73 b) moja wersja nr [nrWersji]}
%[a]:[2,3,4,5,6]
%[b]:[2,3,4,5,6]
%[a]=random.randint(1,10)
%[b]=random.randint(1,10)
%[ab]=[a]-[b]
%[apb]=[a]+[b]
%[b2]=round([b]/2,2)
%[ab]>0 and [ab]<[b]
Rozwiązać nierówność: $\frac{|x-[a]|}{x+[b]}<1$
\zadStop
\rozwStart{Pascal Nawrocki}{}
Zaczynamy od dziedziny: $x\in\mathbb{R}\symbol{92}\{-[b]\}$
Rozwiązujemy:
$$\frac{|x-[a]|}{x+[b]}<1$$
$$\frac{|x-[a]|}{x+[b]}-1<0$$
$$\frac{|x-[a]|-(x+[b])}{x+[b]}<0$$
Porządkujemy i mnożymy przez mianownik podniesiony do kwadratu:
$$(|x-[a]|-x-[b])(x+[b])<0$$
\begin{enumerate}
\item Przypadek gdy: $x\in(-\infty,[a])$
$$(|x-[a]|-x-[b])(x+[b])<0$$
$$(-x+[a]-x-[b])(x+[b])<0$$
$$(-2x+[ab])(x+[b])<0$$
$$x\in(-\infty,-[b])\cup[[b2],\infty)$$
Jeszcze patrzymy w jakim przedziale działamy i mamy:
$$x\in(-\infty,-[b])\cup[[b2],[a])$$
\item Przypadek gdy: $x\in[[a],\infty)$
$$(|x-[a]|-x-[b])(x+[b])<0$$
$$(x-[a]-x-[b])(x+[b])<0$$
$$-[apb](x+[b])<0$$
$$(x+[b])>0$$
$$x>-[b]$$
Patrzymy na przedział i zapisujemy rozwiązanie:
$$x\in[[a],\infty)$$
\end{enumerate}
Sumujemy pamiętając o założeniu i mamy:
$$x\in(-\infty,-[b])\cup[[b2],\infty)$$
\odpStop
\testStart
A. $x\in(-\infty,-[a])\cup([2a],\infty)$
B.$x\in([2a],\infty)$
C.$x\in\emptyset$
D.$x\in(-\infty,-[a])$
\testStop
\kluczStart
A
\kluczStop
\end{document}