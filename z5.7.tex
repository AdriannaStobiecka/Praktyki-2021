\documentclass[12pt, a4paper]{article}
\usepackage[utf8]{inputenc}
\usepackage{polski}

\usepackage{amsthm}  %pakiet do tworzenia twierdzeń itp.
\usepackage{amsmath} %pakiet do niektórych symboli matematycznych
\usepackage{amssymb} %pakiet do symboli mat., np. \nsubseteq
\usepackage{amsfonts}
\usepackage{graphicx} %obsługa plików graficznych z rozszerzeniem png, jpg
\theoremstyle{definition} %styl dla definicji
\newtheorem{zad}{} 
\title{Multizestaw zadań}
\author{Robert Fidytek}
%\date{\today}
\date{}
\newcounter{liczniksekcji}
\newcommand{\kategoria}[1]{\section{#1}} %olreślamy nazwę kateforii zadań
\newcommand{\zadStart}[1]{\begin{zad}#1\newline} %oznaczenie początku zadania
\newcommand{\zadStop}{\end{zad}}   %oznaczenie końca zadania
%Makra opcjonarne (nie muszą występować):
\newcommand{\rozwStart}[2]{\noindent \textbf{Rozwiązanie (autor #1 , recenzent #2): }\newline} %oznaczenie początku rozwiązania, opcjonarnie można wprowadzić informację o autorze rozwiązania zadania i recenzencie poprawności wykonania rozwiązania zadania
\newcommand{\rozwStop}{\newline}                                            %oznaczenie końca rozwiązania
\newcommand{\odpStart}{\noindent \textbf{Odpowiedź:}\newline}    %oznaczenie początku odpowiedzi końcowej (wypisanie wyniku)
\newcommand{\odpStop}{\newline}                                             %oznaczenie końca odpowiedzi końcowej (wypisanie wyniku)
\newcommand{\testStart}{\noindent \textbf{Test:}\newline} %ewentualne możliwe opcje odpowiedzi testowej: A. ? B. ? C. ? D. ? itd.
\newcommand{\testStop}{\newline} %koniec wprowadzania odpowiedzi testowych
\newcommand{\kluczStart}{\noindent \textbf{Test poprawna odpowiedź:}\newline} %klucz, poprawna odpowiedź pytania testowego (jedna literka): A lub B lub C lub D itd.
\newcommand{\kluczStop}{\newline} %koniec poprawnej odpowiedzi pytania testowego 
\newcommand{\wstawGrafike}[2]{\begin{figure}[h] \includegraphics[scale=#2] {#1} \end{figure}} %gdyby była potrzeba wstawienia obrazka, parametry: nazwa pliku, skala (jak nie wiesz co wpisać, to wpisz 1)

\begin{document}
\maketitle


\kategoria{Wikieł/Z5.7}
\zadStart{Zadanie z Wikieł Z 5.7 moja wersja nr [nrWersji]}
%[c]:[2,3,4,5,6,7,8,9]
%[d]:[3,4,5,6,7,8,9]
%[c1]=[c]-1
%[dc]=[d]*[c]
%[d1]=[d]-1
%[c]!=0 and [c1]-[dc]+ (1/2) !=0
Wyznacz wartość parametrów $a,b \in \mathbb{R}$, dla których funkcja f jest ciągła i różniczkowalna.
$$
f(x) = \left\{ \begin{array}{ll}
[c]x^{[d]}-ax & \textrm{gdy $x<1$}\\
\sqrt{x}+bx & \textrm{gdy $ x\geq 1$}
\end{array} \right.
$$
\zadStop
\rozwStart{Joanna Świerzbin}{}
\begin{enumerate}
\item $$ [c]x^{[d]}-ax = \sqrt{x}+bx $$
 $$ [c]-a =1+b $$
 $$ a =[c]-1-b $$
 $$ a =[c1]-b $$
\item 
$$
f'(x) = \left\{ \begin{array}{ll}
[d]\cdot[c]x^{[d]-1}-a & \textrm{gdy $x<1$}\\
\frac{1}{2\sqrt{x}}+b & \textrm{gdy $ x\geq 1$}
\end{array} \right.
$$
$$[dc]x^{[d1]}-a = \frac{1}{2\sqrt{x}}+b $$
$$[dc]-a = \frac{1}{2}+b $$
$$a =[dc]- \frac{1}{2}-b $$
\end{enumerate}
$$
\left\{ \begin{array}{ll}
a =[c1]-b\\
a =[dc]- \frac{1}{2}-b
\end{array} \right.
$$
$$ [c1]-b =[dc]- \frac{1}{2}-b $$
$$ [c1]-[dc]+ \frac{1}{2} \neq 0 $$
Więc nie istnieją takie parametry
\rozwStop
\odpStart
nie istnieją takie parametry
\odpStop
\testStart
A. nie istnieją takie parametry\\
B. $a=1, b=2$\\
C. $a=1, b=1$\\
D. $a=1, b=-1$\\
E. $a=-1, b=2$\\
F. $a=-1, b=-2$
\testStop
\kluczStart
A
\kluczStop



\end{document}