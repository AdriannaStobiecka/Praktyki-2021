\documentclass[12pt, a4paper]{article}
\usepackage[utf8]{inputenc}
\usepackage{polski}

\usepackage{amsthm}  %pakiet do tworzenia twierdzeń itp.
\usepackage{amsmath} %pakiet do niektórych symboli matematycznych
\usepackage{amssymb} %pakiet do symboli mat., np. \nsubseteq
\usepackage{amsfonts}
\usepackage{graphicx} %obsługa plików graficznych z rozszerzeniem png, jpg
\theoremstyle{definition} %styl dla definicji
\newtheorem{zad}{} 
\title{Multizestaw zadań}
\author{Robert Fidytek}
%\date{\today}
\date{}
\newcounter{liczniksekcji}
\newcommand{\kategoria}[1]{\section{#1}} %olreślamy nazwę kateforii zadań
\newcommand{\zadStart}[1]{\begin{zad}#1\newline} %oznaczenie początku zadania
\newcommand{\zadStop}{\end{zad}}   %oznaczenie końca zadania
%Makra opcjonarne (nie muszą występować):
\newcommand{\rozwStart}[2]{\noindent \textbf{Rozwiązanie (autor #1 , recenzent #2): }\newline} %oznaczenie początku rozwiązania, opcjonarnie można wprowadzić informację o autorze rozwiązania zadania i recenzencie poprawności wykonania rozwiązania zadania
\newcommand{\rozwStop}{\newline}                                            %oznaczenie końca rozwiązania
\newcommand{\odpStart}{\noindent \textbf{Odpowiedź:}\newline}    %oznaczenie początku odpowiedzi końcowej (wypisanie wyniku)
\newcommand{\odpStop}{\newline}                                             %oznaczenie końca odpowiedzi końcowej (wypisanie wyniku)
\newcommand{\testStart}{\noindent \textbf{Test:}\newline} %ewentualne możliwe opcje odpowiedzi testowej: A. ? B. ? C. ? D. ? itd.
\newcommand{\testStop}{\newline} %koniec wprowadzania odpowiedzi testowych
\newcommand{\kluczStart}{\noindent \textbf{Test poprawna odpowiedź:}\newline} %klucz, poprawna odpowiedź pytania testowego (jedna literka): A lub B lub C lub D itd.
\newcommand{\kluczStop}{\newline} %koniec poprawnej odpowiedzi pytania testowego 
\newcommand{\wstawGrafike}[2]{\begin{figure}[h] \includegraphics[scale=#2] {#1} \end{figure}} %gdyby była potrzeba wstawienia obrazka, parametry: nazwa pliku, skala (jak nie wiesz co wpisać, to wpisz 1)

\begin{document}
\maketitle


\kategoria{Wikieł/Z3.13i}
\zadStart{Zadanie z Wikieł Z 3.13 i) moja wersja nr [nrWersji]}
%[a]:[3,4,5,6,7,8,9,10]
%[b]=random.randint(1,30)
%[c]:[1,2,3,4,5,6,7,8,9,10,11,12,13,14,15,16,17,18,19,20,21,22,23,24,25,26,27,28,29,30]
%[d]=random.randint(1,30)
%[a]!=[b] and [a]!=[c] and [a]!=[d]
Obliczyć granicę ciągu $a_n= \frac{\sqrt{n^{[a]}}+[b]}{\sqrt{n^{[a]} +[c]}+[d]}$.
\zadStop
\rozwStart{Barbara Bączek}{}
$$\lim_{n \rightarrow \infty} a_n= \lim_{n \rightarrow \infty}  \frac{\sqrt{n^{[a]}}+[b]}{\sqrt{n^{[a]} +[c]}+[d]}= \lim_{n \rightarrow \infty} \frac{\sqrt{n^{[a]}}\big{(}1+\frac{[b]}{\sqrt{n^{[a]}}}\big{)}}{\sqrt{n^{[a]}}\big{(}\sqrt{1+\frac{[c]}{n^{[a]}}}+\frac{[d]}{\sqrt{n^{[a]}}}\big{)}} =1$$
\rozwStop
\odpStart
$1$
\odpStop
\testStart
A.$\infty$
B.$[a]$
C.$-\infty$
D.$0$
E.$-[a]$
G.$1$
H.$-1$
\testStop
\kluczStart
G
\kluczStop



\end{document}