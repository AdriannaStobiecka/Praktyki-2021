\documentclass[12pt, a4paper]{article}
\usepackage[utf8]{inputenc}
\usepackage{polski}

\usepackage{amsthm}  %pakiet do tworzenia twierdzeń itp.
\usepackage{amsmath} %pakiet do niektórych symboli matematycznych
\usepackage{amssymb} %pakiet do symboli mat., np. \nsubseteq
\usepackage{amsfonts}
\usepackage{graphicx} %obsługa plików graficznych z rozszerzeniem png, jpg
\theoremstyle{definition} %styl dla definicji
\newtheorem{zad}{} 
\title{Multizestaw zadań}
\author{Robert Fidytek}
%\date{\today}
\date{}
\newcounter{liczniksekcji}
\newcommand{\kategoria}[1]{\section{#1}} %olreślamy nazwę kateforii zadań
\newcommand{\zadStart}[1]{\begin{zad}#1\newline} %oznaczenie początku zadania
\newcommand{\zadStop}{\end{zad}}   %oznaczenie końca zadania
%Makra opcjonarne (nie muszą występować):
\newcommand{\rozwStart}[2]{\noindent \textbf{Rozwiązanie (autor #1 , recenzent #2): }\newline} %oznaczenie początku rozwiązania, opcjonarnie można wprowadzić informację o autorze rozwiązania zadania i recenzencie poprawności wykonania rozwiązania zadania
\newcommand{\rozwStop}{\newline}                                            %oznaczenie końca rozwiązania
\newcommand{\odpStart}{\noindent \textbf{Odpowiedź:}\newline}    %oznaczenie początku odpowiedzi końcowej (wypisanie wyniku)
\newcommand{\odpStop}{\newline}                                             %oznaczenie końca odpowiedzi końcowej (wypisanie wyniku)
\newcommand{\testStart}{\noindent \textbf{Test:}\newline} %ewentualne możliwe opcje odpowiedzi testowej: A. ? B. ? C. ? D. ? itd.
\newcommand{\testStop}{\newline} %koniec wprowadzania odpowiedzi testowych
\newcommand{\kluczStart}{\noindent \textbf{Test poprawna odpowiedź:}\newline} %klucz, poprawna odpowiedź pytania testowego (jedna literka): A lub B lub C lub D itd.
\newcommand{\kluczStop}{\newline} %koniec poprawnej odpowiedzi pytania testowego 
\newcommand{\wstawGrafike}[2]{\begin{figure}[h] \includegraphics[scale=#2] {#1} \end{figure}} %gdyby była potrzeba wstawienia obrazka, parametry: nazwa pliku, skala (jak nie wiesz co wpisać, to wpisz 1)

\begin{document}
\maketitle


\kategoria{Wikieł/Z2.37}
\zadStart{Zadanie z Wikieł Z 2.37  moja wersja nr [nrWersji]}
%[p1]=random.randint(-10,-2)
%[p2]:[2,3,4,5,6,7,8,9,10]
%[p4]:[2,3,4,5,6,7,8,9,10]
%[p5]=random.randint(1,10)
%[p11]=-[p1]
%[m]=round([p11]/[p4],2)


Podać wartości parametru $m$, dla których układ równań
$$\left\{\begin{array}{rcl}
mx [p1]y&=&[p2]\\
 x+[p4]y&=&[p5]
\end{array} \right.$$
ma rozwiązanie.
\zadStop
\rozwStart{Maja Szabłowska}{}
$$W=\left| \begin{array}{lccr} m & [p1]  \\ 1 & [p4] \end{array}\right| = m\cdot [p4] [p1]\cdot 1=[p4]m [p1] \neq 0$$

$$[p4]m [p1] \neq 0 \Rightarrow [p4]m\neq [p11] \Rightarrow m\neq [m]$$
\rozwStop
\odpStart
$m\neq [m]$
\odpStop
\testStart
A.$m\neq [p1]$
B.$m\neq [p2]$
C.$m\neq [p4]$
D.$m\neq [p5]$
E.$m\neq [p11]$
F.$m\neq -[p2]$
G.$m\neq 0$
H.$m\neq -[p4]$
I.$m\neq -[p5]$
\testStop
\kluczStart
A
\kluczStop



\end{document}