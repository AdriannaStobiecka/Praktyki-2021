\documentclass[12pt, a4paper]{article}
\usepackage[utf8]{inputenc}
\usepackage{polski}

\usepackage{amsthm}  %pakiet do tworzenia twierdzeń itp.
\usepackage{amsmath} %pakiet do niektórych symboli matematycznych
\usepackage{amssymb} %pakiet do symboli mat., np. \nsubseteq
\usepackage{amsfonts}
\usepackage{graphicx} %obsługa plików graficznych z rozszerzeniem png, jpg
\theoremstyle{definition} %styl dla definicji
\newtheorem{zad}{} 
\title{Multizestaw zadań}
\author{Robert Fidytek}
%\date{\today}
\date{}
\newcounter{liczniksekcji}
\newcommand{\kategoria}[1]{\section{#1}} %olreślamy nazwę kateforii zadań
\newcommand{\zadStart}[1]{\begin{zad}#1\newline} %oznaczenie początku zadania
\newcommand{\zadStop}{\end{zad}}   %oznaczenie końca zadania
%Makra opcjonarne (nie muszą występować):
\newcommand{\rozwStart}[2]{\noindent \textbf{Rozwiązanie (autor #1 , recenzent #2): }\newline} %oznaczenie początku rozwiązania, opcjonarnie można wprowadzić informację o autorze rozwiązania zadania i recenzencie poprawności wykonania rozwiązania zadania
\newcommand{\rozwStop}{\newline}                                            %oznaczenie końca rozwiązania
\newcommand{\odpStart}{\noindent \textbf{Odpowiedź:}\newline}    %oznaczenie początku odpowiedzi końcowej (wypisanie wyniku)
\newcommand{\odpStop}{\newline}                                             %oznaczenie końca odpowiedzi końcowej (wypisanie wyniku)
\newcommand{\testStart}{\noindent \textbf{Test:}\newline} %ewentualne możliwe opcje odpowiedzi testowej: A. ? B. ? C. ? D. ? itd.
\newcommand{\testStop}{\newline} %koniec wprowadzania odpowiedzi testowych
\newcommand{\kluczStart}{\noindent \textbf{Test poprawna odpowiedź:}\newline} %klucz, poprawna odpowiedź pytania testowego (jedna literka): A lub B lub C lub D itd.
\newcommand{\kluczStop}{\newline} %koniec poprawnej odpowiedzi pytania testowego 
\newcommand{\wstawGrafike}[2]{\begin{figure}[h] \includegraphics[scale=#2] {#1} \end{figure}} %gdyby była potrzeba wstawienia obrazka, parametry: nazwa pliku, skala (jak nie wiesz co wpisać, to wpisz 1)

\begin{document}
\maketitle


\kategoria{Wikieł/Z1.79n}
\zadStart{Zadanie z Wikieł Z 1.79 n) moja wersja nr [nrWersji]}
%[b]:[2,9,4,3,8,5,16,12,18,49,32,25]
%[m1]:[8,36,32,18,48,50,80,108,144,12,392,320,150]
%[d1]=int([m1]/[b])
%[de]=[b]*[b] + 4*[m1]
%[dep]=int(math.sqrt([de]))
%[x1]=int((-[b]-[dep])/(-2))
%[x2]=int((-[b]+[dep])/(-2))
%[d1]- ([m1]/[b])==0 and [dep]-(math.sqrt([de]))==0 and [dep]>[b] and [x1]-((-[b]-[dep])/(-2))==0 and [x2]-((-[b]+[dep])/(-2))==0
Rozwiązać nierówność $\frac{\sqrt{[b]x+[m1]}}{x}<1$
\zadStop
\rozwStart{Barbara Bączek}{}
Zaczniemy od wyznaczenia dziedziny.
$$D:[b]x+[m1] \geq 0 \hspace{0.2 cm} \wedge \hspace{0.2 cm} x \ne 0$$
$$D: x \in [ -[d1], 0) \cup (0, \infty)$$
\begin{enumerate}
\item Niech $x \in [-[d1], 0)$, wtedy:
$$\frac{\sqrt{[b]x+[m1]}}{x}<1$$
$$\sqrt{[b]x+[m1]}>x$$
Nierówność jest tożsamościowa dla $x \in [-[d1], 0)$.
\item  Niech $x \in (0, \infty)$, wtedy:
$$\frac{\sqrt{[b]x+[m1]}}{x}<1$$
$$\sqrt{[b]x+[m1]}<x$$
Obie strony są dodatnie.
$$-x^2 + [b]x+ [m1]<0$$
$$\Delta = [de] \hspace{0.2 cm} \wedge \hspace{0.2 cm} \sqrt{\Delta}= [dep]$$
$$x_1 = \frac{-[b]-[dep]}{-2}=[x1]  \hspace{0.2 cm} \wedge \hspace{0.2 cm} x_2 = \frac{-[b]+[dep]}{-2}=[x2]$$
Rozwiązaniem $-x^2+[b]x+20<0$ w zbiorze $(0, \infty)$ jest więc $([x1], \infty)$.
\end{enumerate}
3. Podsumowując: $x \in [-[d1],0) \cup ([x1], \infty)$
\rozwStop
\odpStart
$x \in [-[d1],0) \cup ([x1], \infty)$
\odpStop
\testStart
A.$x \in [-[d1],0)$
B.$x \in (0,[x1])$
C.$x \in ([x1], \infty)$
D.$x \in [-[d1],0) \cup ([x1], \infty)$
E.$x \in [-[d1],[x1])$
G.$x \in [-[d1],0) \cup [[x1], \infty)$
H.$x \in (-[d1],0) \cup ([x1], \infty)$
\testStop
\kluczStart
D
\kluczStop



\end{document}