\documentclass[12pt, a4paper]{article}
\usepackage[utf8]{inputenc}
\usepackage{polski}

\usepackage{amsthm}  %pakiet do tworzenia twierdzeń itp.
\usepackage{amsmath} %pakiet do niektórych symboli matematycznych
\usepackage{amssymb} %pakiet do symboli mat., np. \nsubseteq
\usepackage{amsfonts}
\usepackage{graphicx} %obsługa plików graficznych z rozszerzeniem png, jpg
\theoremstyle{definition} %styl dla definicji
\newtheorem{zad}{} 
\title{Multizestaw zadań}
\author{Robert Fidytek}
%\date{\today}
\date{}
\newcounter{liczniksekcji}
\newcommand{\kategoria}[1]{\section{#1}} %olreślamy nazwę kateforii zadań
\newcommand{\zadStart}[1]{\begin{zad}#1\newline} %oznaczenie początku zadania
\newcommand{\zadStop}{\end{zad}}   %oznaczenie końca zadania
%Makra opcjonarne (nie muszą występować):
\newcommand{\rozwStart}[2]{\noindent \textbf{Rozwiązanie (autor #1 , recenzent #2): }\newline} %oznaczenie początku rozwiązania, opcjonarnie można wprowadzić informację o autorze rozwiązania zadania i recenzencie poprawności wykonania rozwiązania zadania
\newcommand{\rozwStop}{\newline}                                            %oznaczenie końca rozwiązania
\newcommand{\odpStart}{\noindent \textbf{Odpowiedź:}\newline}    %oznaczenie początku odpowiedzi końcowej (wypisanie wyniku)
\newcommand{\odpStop}{\newline}                                             %oznaczenie końca odpowiedzi końcowej (wypisanie wyniku)
\newcommand{\testStart}{\noindent \textbf{Test:}\newline} %ewentualne możliwe opcje odpowiedzi testowej: A. ? B. ? C. ? D. ? itd.
\newcommand{\testStop}{\newline} %koniec wprowadzania odpowiedzi testowych
\newcommand{\kluczStart}{\noindent \textbf{Test poprawna odpowiedź:}\newline} %klucz, poprawna odpowiedź pytania testowego (jedna literka): A lub B lub C lub D itd.
\newcommand{\kluczStop}{\newline} %koniec poprawnej odpowiedzi pytania testowego 
\newcommand{\wstawGrafike}[2]{\begin{figure}[h] \includegraphics[scale=#2] {#1} \end{figure}} %gdyby była potrzeba wstawienia obrazka, parametry: nazwa pliku, skala (jak nie wiesz co wpisać, to wpisz 1)

\begin{document}
\maketitle


\kategoria{Wikieł/Z1.79k}
\zadStart{Zadanie z Wikieł Z 1.79 k) moja wersja nr [nrWersji]}
%[z]:[1,2,3,4,5,6,7,8,9,10,11]
%[z1]:[12,13,14,15,16,17,18,19,20]
%[z2]:[12,13,14,15,16,17,18,19,20]
%[b]=random.randint(2,20)
%[a]=random.randint([b]*[b],400)
%[b2]=2*[b]-1
%[b3]=-[b]*[b]+[a]
%[d]=[b2]*[b2]+4*[b3]
%[dep]=int(math.sqrt([d]))
%[x1]=int((-[b2]-[dep])/(-2))
%[x2]=int((-[b2]+[dep])/(-2))
%[b3]>0 and [d]>0 and [dep]-(math.sqrt([d]))==0 and [x2]<[b] and [x1]<[a]
Rozwiązać nierówność $\sqrt{[a]-x}>x-[b]$
\zadStop
\rozwStart{Barbara Bączek}{}
Zaczniemy od wyznaczenia dziedziny.
$$D:[a]-x \geq 0 $$
$$D: x \in (-\infty, [a]]$$
\begin{enumerate}
\item Rozpatrzmy przypadek, gdy $x-[b]<0$, wtedy 
$$x \in (-\infty,[b]) \hspace{0.1 cm} \wedge \hspace{0.1 cm} x \in (-\infty, [a]]$$
$$x \in (-\infty,[b])$$
Nierówność jest tożsamościowa w zbiorze $ x \in (-\infty, [b])$.
\item Teraz, niech $(x-[b] \geq 0) \hspace{0.1 cm} \wedge \hspace{0.1 cm} (x \in (-\infty, [a]]$, czyli
$$x \in [[b], [a]]$$
W powyższym zbiorze obie strony nierówności są nieujemne.
$$[a]-x>x^2-2[b]+{[b]}^2$$
$$-x^2 +[b2]x + [b3]>0$$
$$\Delta = [d] \hspace{0.1 cm} \wedge \hspace{0.1 cm} \sqrt{\Delta}=[dep]$$
$$x_1=[x1] \hspace{0.1 cm} \wedge \hspace{0.1cm}  x_2=[x2]$$
Rozwiązaniem $\sqrt{[a]-x}>x-[b]$ w zbiorze $[[b], [a]]$ jest więc $[[b], [x1])$.
\end{enumerate}
3. Podsumowując: $x \in (-\infty, [x1])$
\rozwStop
\odpStart
$x \in (-\infty, [a]]$
\odpStop
\testStart
A.$x \in [[b],[a])$
B.$x \in (-\infty, [x1]]$
C.$x \in ([b], [x1])$
D.$x \in (-\infty,[b]) \cup ([a], \infty)$
E.$x \in (-\infty, [x1])$
G.$x \in (-\infty,[b]] \cup [[a], \infty)$
H.$x \in [[x1], \infty)$
\testStop
\kluczStart
E
\kluczStop



\end{document}