\documentclass[12pt, a4paper]{article}
\usepackage[utf8]{inputenc}
\usepackage{polski}

\usepackage{amsthm}  %pakiet do tworzenia twierdzeń itp.
\usepackage{amsmath} %pakiet do niektórych symboli matematycznych
\usepackage{amssymb} %pakiet do symboli mat., np. \nsubseteq
\usepackage{amsfonts}
\usepackage{graphicx} %obsługa plików graficznych z rozszerzeniem png, jpg
\theoremstyle{definition} %styl dla definicji
\newtheorem{zad}{} 
\title{Multizestaw zadań}
\author{Robert Fidytek}
%\date{\today}
\date{}
\newcounter{liczniksekcji}
\newcommand{\kategoria}[1]{\section{#1}} %olreślamy nazwę kateforii zadań
\newcommand{\zadStart}[1]{\begin{zad}#1\newline} %oznaczenie początku zadania
\newcommand{\zadStop}{\end{zad}}   %oznaczenie końca zadania
%Makra opcjonarne (nie muszą występować):
\newcommand{\rozwStart}[2]{\noindent \textbf{Rozwiązanie (autor #1 , recenzent #2): }\newline} %oznaczenie początku rozwiązania, opcjonarnie można wprowadzić informację o autorze rozwiązania zadania i recenzencie poprawności wykonania rozwiązania zadania
\newcommand{\rozwStop}{\newline}                                            %oznaczenie końca rozwiązania
\newcommand{\odpStart}{\noindent \textbf{Odpowiedź:}\newline}    %oznaczenie początku odpowiedzi końcowej (wypisanie wyniku)
\newcommand{\odpStop}{\newline}                                             %oznaczenie końca odpowiedzi końcowej (wypisanie wyniku)
\newcommand{\testStart}{\noindent \textbf{Test:}\newline} %ewentualne możliwe opcje odpowiedzi testowej: A. ? B. ? C. ? D. ? itd.
\newcommand{\testStop}{\newline} %koniec wprowadzania odpowiedzi testowych
\newcommand{\kluczStart}{\noindent \textbf{Test poprawna odpowiedź:}\newline} %klucz, poprawna odpowiedź pytania testowego (jedna literka): A lub B lub C lub D itd.
\newcommand{\kluczStop}{\newline} %koniec poprawnej odpowiedzi pytania testowego 
\newcommand{\wstawGrafike}[2]{\begin{figure}[h] \includegraphics[scale=#2] {#1} \end{figure}} %gdyby była potrzeba wstawienia obrazka, parametry: nazwa pliku, skala (jak nie wiesz co wpisać, to wpisz 1)

\begin{document}
\maketitle


\kategoria{Wikieł/P1.22}
\zadStart{Zadanie z Wikieł P 1.22) moja wersja nr [nrWersji]}
%[a]:[2,3,4,5,6,7,8,9,10,11,12,13,14,15]
%[b]:[3,4,5,6,7,8,9,10,11,12,13,14,15,16,17,18]
%[abm]=[b]-[a]
%[ap]=[a]+[a]
%[ak]=[a]*[a]
%[akbm]=[ak]-[b]
%[b]>[a] and [abm]!=0 and [akbm]!=0 and [ak]>[b] and [ap]!=[abm]
Niech funkcje $f:\mathbb{R}\rightarrow\mathbb{R}$ i $g:\mathbb{R}\rightarrow\mathbb{R}$ określone będą wzorami: $f(x)=x+[a]$ oraz $g(x)=x^{2}-[b]$. Wyznaczyć złożenia $f\circ g$ i $g\circ f$.
\zadStop
\rozwStart{Justyna Chojecka}{}
Dla $x\in \mathbb{R}$ mamy
$$(f\circ g)(x)=f(g(x))=f(x^{2}-[b])=(x^{2}-[b])+[a]=x^2-[abm]$$
$$(g\circ f)(x)=g(f(x))=g(x+[a])=(x+[a])^{2}-[b]=x^{2}+[ap]x+[ak]-[b]$$$$=x^{2}+[ap]x+[akbm]$$
\rozwStop
\odpStart
$(f\circ g)(x)=x^2-[abm], (g\circ f)(x)=x^{2}+[ap]x+[akbm]$
\odpStop
\testStart
A.$(f\circ g)(x)=x^2-[abm], (g\circ f)(x)=x^{2}+[ap]x+[akbm]$\\
B.$(f\circ g)(x)=x^2+[abm], (g\circ f)(x)=x^{2}+[ap]x+[akbm]$\\
C.$(f\circ g)(x)=x^2-[abm], (g\circ f)(x)=x^{2}-[ap]x+[akbm]$\\
D.$(f\circ g)(x)=x^2-[ap], (g\circ f)(x)=x^{2}+[ap]x+[abm]$\\
E.$(f\circ g)(x)=x^2+[ap], (g\circ f)(x)=x^{2}-[ap]x+[akbm]$\\
F.$(f\circ g)(x)=x^2+[ap], (g\circ f)(x)=x^{2}+[ap]x-[abm]$\\
G.$(f\circ g)(x)=x^2-[ap], (g\circ f)(x)=x^{2}+[ap]x+[akbm]$\\
H$(f\circ g)(x)=x^2+[akbm], (g\circ f)(x)=x^{2}+[ap]x+[abm]$\\
I.$(f\circ g)(x)=x^2-[ap], (g\circ f)(x)=x^{2}+[akbm]x+[abm]$
\testStop
\kluczStart
A
\kluczStop



\end{document}