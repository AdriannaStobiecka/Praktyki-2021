\documentclass[12pt, a4paper]{article}
\usepackage[utf8]{inputenc}
\usepackage{polski}

\usepackage{amsthm}  %pakiet do tworzenia twierdzeń itp.
\usepackage{amsmath} %pakiet do niektórych symboli matematycznych
\usepackage{amssymb} %pakiet do symboli mat., np. \nsubseteq
\usepackage{amsfonts}
\usepackage{graphicx} %obsługa plików graficznych z rozszerzeniem png, jpg
\theoremstyle{definition} %styl dla definicji
\newtheorem{zad}{} 
\title{Multizestaw zadań}
\author{Robert Fidytek}
%\date{\today}
\date{}
\newcounter{liczniksekcji}
\newcommand{\kategoria}[1]{\section{#1}} %olreślamy nazwę kateforii zadań
\newcommand{\zadStart}[1]{\begin{zad}#1\newline} %oznaczenie początku zadania
\newcommand{\zadStop}{\end{zad}}   %oznaczenie końca zadania
%Makra opcjonarne (nie muszą występować):
\newcommand{\rozwStart}[2]{\noindent \textbf{Rozwiązanie (autor #1 , recenzent #2): }\newline} %oznaczenie początku rozwiązania, opcjonarnie można wprowadzić informację o autorze rozwiązania zadania i recenzencie poprawności wykonania rozwiązania zadania
\newcommand{\rozwStop}{\newline}                                            %oznaczenie końca rozwiązania
\newcommand{\odpStart}{\noindent \textbf{Odpowiedź:}\newline}    %oznaczenie początku odpowiedzi końcowej (wypisanie wyniku)
\newcommand{\odpStop}{\newline}                                             %oznaczenie końca odpowiedzi końcowej (wypisanie wyniku)
\newcommand{\testStart}{\noindent \textbf{Test:}\newline} %ewentualne możliwe opcje odpowiedzi testowej: A. ? B. ? C. ? D. ? itd.
\newcommand{\testStop}{\newline} %koniec wprowadzania odpowiedzi testowych
\newcommand{\kluczStart}{\noindent \textbf{Test poprawna odpowiedź:}\newline} %klucz, poprawna odpowiedź pytania testowego (jedna literka): A lub B lub C lub D itd.
\newcommand{\kluczStop}{\newline} %koniec poprawnej odpowiedzi pytania testowego 
\newcommand{\wstawGrafike}[2]{\begin{figure}[h] \includegraphics[scale=#2] {#1} \end{figure}} %gdyby była potrzeba wstawienia obrazka, parametry: nazwa pliku, skala (jak nie wiesz co wpisać, to wpisz 1)

\begin{document}
\maketitle


\kategoria{Wikieł/Z1.72}
\zadStart{Zadanie z Wikieł Z 1.72 moja wersja nr [nrWersji]}
%[a]:[2,3,4,5,6]
%[b]:[2,3,4,5,6]
%[c]:[2,3,4,5,6]
%[d]:[2,3,4,5,6]
%[e]:[2,3,4,5,6]
%[a]=random.randint(2,8)
%[b]=random.randint(2,8)
%[c]=random.randint(2,8)
%[d]=random.randint(2,8)
%[e]=random.randint(2,8)
%[4a]=4*[a]
%[c4a]=[4a]-[c]
%[zerowe]=round((-1)*[c4a]/[b],2)
%[2a]=[a]+[a]
%[mc4a]=[c]+[4a]
%[delta]=([b]*[b])+(4*[2a]*[mc4a])
%[pierw]=math.sqrt([delta])
%[pierw1]=int([pierw])
%[x1]=round((-[b]+[pierw1])/(2*[2a]),2)
%[x2]=round((-[b]-[pierw1])/(2*[2a]),2)
%[mx2]=-[x2]
%[pierw].is_integer() and [zerowe]>-2 and [zerowe]<2 and [x1]<2 and [x2]<-2
Rozwiązać nierówność podwójną: $-[a]\leq\frac{[a]x^2+[b]x-[c]}{4-x^2}\leq [a]$ (w zadaniu zaokrąglamy do 2 miejsc po przecinku).
\zadStop
\rozwStart{Pascal Nawrocki}{Jakub Ulrych}
Pamiętamy o wyznaczeniu dziedziny:  $x\in\mathbb{R}\symbol{92}\{-2,2\}$ (wtedy mianownik się zeruje, a jak pamiętamy, nie dzielimy przez 0).
Następnie taką nierówność podwójną możemy sobie rozbić na dwie nierówności tj. $-[a]\leq\frac{[a]x^2+[b]x-[c]}{4-x^2}$ oraz $\frac{[a]x^2+[b]x-[c]}{4-x^2}\leq [a]$ i weźmiemy część wspólną z obu rozwiązań. Aby znaleźć miejsca zerowe tych wielomianów, które otrzymamy później stosujemy metodę rysowania wykresu wielomianu.
\begin{enumerate}
\item
$$-[a]\leq\frac{[a]x^2+[b]x-[c]}{4-x^2}$$
$$\frac{[a]x^2+[b]x-[c]}{4-x^2}+[a]\geq0$$
$$\frac{[a]x^2+[b]x-[c]+[4a]-[a]x^2}{4-x^2}\geq0$$
$$\frac{[b]x+[c4a]}{4-x^2}\geq0$$
$$([b]x+[c4a])(4-x^2)\geq0$$
$$([b]x+[c4a])(2-x)(2+x)\geq0$$
$$x\in(-\infty,-2)\cup[[zerowe],2)$$
\item
$$\frac{[a]x^2+[b]x-[c]}{4-x^2}\leq [a]$$
$$\frac{[a]x^2+[b]x-[c]}{4-x^2}-[a]\leq0$$
$$\frac{[a]x^2+[b]x-[c]-[4a]+[a]x^2}{4-x^2}\leq0$$
$$\frac{[2a]x^2+[b]x-[mc4a]}{4-x^2}\leq0$$
$$([2a]x^2+[b]x-[mc4a])(2-x)(2+x)\leq0$$
$$\Delta=[delta]\Rightarrow \sqrt{\Delta}=\sqrt{[delta]}=[pierw1]$$
$$[2a](x-[x1])(x+[mx2])(2-x)(2+x)\leq0$$
$$x\in(-\infty,[x2]]\cup(-2,[x1]]\cup(2,\infty)$$
\end{enumerate}
Zatem rozwiązaniem jest część wspólna obu rozwiązań, czyli przedział: $$x\in(-\infty,[x2]]\cup[[zerowe],[x1]]$$
\rozwStop
\odpStart
$x\in(-\infty,[x2]]\cup[[zerowe],[x1]]$
\odpStop
\testStart
A. $x\in(-\infty,[x2]]\cup[[zerowe],[x1]]$
B.$x\in([b],\infty)$
C.$x\in\emptyset$
D.$x\in(-\infty,-[a])$
\testStop
\kluczStart
A
\kluczStop
\end{document}