\documentclass[12pt, a4paper]{article}
\usepackage[utf8]{inputenc}
\usepackage{polski}

\usepackage{amsthm}  %pakiet do tworzenia twierdzeń itp.
\usepackage{amsmath} %pakiet do niektórych symboli matematycznych
\usepackage{amssymb} %pakiet do symboli mat., np. \nsubseteq
\usepackage{amsfonts}
\usepackage{graphicx} %obsługa plików graficznych z rozszerzeniem png, jpg
\theoremstyle{definition} %styl dla definicji
\newtheorem{zad}{} 
\title{Multizestaw zadań}
\author{Robert Fidytek}
%\date{\today}
\date{}\documentclass[12pt, a4paper]{article}
\usepackage[utf8]{inputenc}
\usepackage{polski}

\usepackage{amsthm}  %pakiet do tworzenia twierdzeń itp.
\usepackage{amsmath} %pakiet do niektórych symboli matematycznych
\usepackage{amssymb} %pakiet do symboli mat., np. \nsubseteq
\usepackage{amsfonts}
\usepackage{graphicx} %obsługa plików graficznych z rozszerzeniem png, jpg
\theoremstyle{definition} %styl dla definicji
\newtheorem{zad}{} 
\title{Multizestaw zadań}
\author{Robert Fidytek}
%\date{\today}
\date{}
\newcounter{liczniksekcji}
\newcommand{\kategoria}[1]{\section{#1}} %olreślamy nazwę kateforii zadań
\newcommand{\zadStart}[1]{\begin{zad}#1\newline} %oznaczenie początku zadania
\newcommand{\zadStop}{\end{zad}}   %oznaczenie końca zadania
%Makra opcjonarne (nie muszą występować):
\newcommand{\rozwStart}[2]{\noindent \textbf{Rozwiązanie (autor #1 , recenzent #2): }\newline} %oznaczenie początku rozwiązania, opcjonarnie można wprowadzić informację o autorze rozwiązania zadania i recenzencie poprawności wykonania rozwiązania zadania
\newcommand{\rozwStop}{\newline}                                            %oznaczenie końca rozwiązania
\newcommand{\odpStart}{\noindent \textbf{Odpowiedź:}\newline}    %oznaczenie początku odpowiedzi końcowej (wypisanie wyniku)
\newcommand{\odpStop}{\newline}                                             %oznaczenie końca odpowiedzi końcowej (wypisanie wyniku)
\newcommand{\testStart}{\noindent \textbf{Test:}\newline} %ewentualne możliwe opcje odpowiedzi testowej: A. ? B. ? C. ? D. ? itd.
\newcommand{\testStop}{\newline} %koniec wprowadzania odpowiedzi testowych
\newcommand{\kluczStart}{\noindent \textbf{Test poprawna odpowiedź:}\newline} %klucz, poprawna odpowiedź pytania testowego (jedna literka): A lub B lub C lub D itd.
\newcommand{\kluczStop}{\newline} %koniec poprawnej odpowiedzi pytania testowego 
\newcommand{\wstawGrafike}[2]{\begin{figure}[h] \includegraphics[scale=#2] {#1} \end{figure}} %gdyby była potrzeba wstawienia obrazka, parametry: nazwa pliku, skala (jak nie wiesz co wpisać, to wpisz 1)

\begin{document}
\maketitle


\kategoria{Wikieł/Z2.35a}
\zadStart{Zadanie z Wikieł Z 2.35a moja wersja nr [nrWersji]}
%[p1]:[2,3,4,5,6,7,8,9,10]
%[p2]:[2,3,4,5,6,7,8,9,10]
%[p3]=random.randint(2,10)
%[p4]:[2,3,4,5,6,7,8,9,10]
%[p5]=random.randint(1,10)
%[p6]=random.randint(1,10)
%[p2p5]=[p2]*[p5]
%[d]=math.gcd([p2p5],[p4])
%[np4]=int([p4]/[d])
%[np2p5]=int([p2p5]/[d])
%[p6np4]=[p6]*[np4]
%[b]=[p6np4]-[np2p5]
%math.gcd([p4],[p2])==1 and math.gcd([p1],[p2])==1 and [b]>0 and [np4]!=1

Dana jest prosta $l:\left\{ \begin{array}{ll}
x=[p1]+[p2]t \\
y=[p3]-[p4]t  
\end{array} \right  t\in\mathbb{R}.$ Napisać w postaci parametrycznej równanie prostej przechodzącej przez $A([p5],[p6])$ i prostopadłej do prostej $l.$

\zadStop

\rozwStart{Maja Szabłowska}{}
$$x=[p1]+[p2]t \iff x-[p1]=[p2]t \iff t=\frac{x-[p1]}{[p2]}$$
$$ y=[p3]-[p4]\cdot\frac{x-[p1]}{[p2]}=-\frac{[p4]}{[p2]}x+[p3]-\frac{[p1]}{[p2]}$$
Szukamy prostej prostopadłej do $l$, zatem:
$$-\frac{[p4]}{[p2]}\cdot a_{2}=-1$$
$$a_{2}=\frac{[p2]}{[p4]}$$
Szukana prosta $k: y=\frac{[p2]}{[p4]}x+b$ powinna przechodzić przez punkt $A([p5],[p6])$, zatem:
$$[p6]=\frac{[p2]}{[p4]}\cdot[p5]+b$$
$$b=[p6]-\frac{[np2p5]}{[np4]}=\frac{[p6np4]-[np2p5]}{[np4]}=\frac{[b]}{[np4]}$$

$$k: y=\frac{[p2]}{[p4]}x+\frac{[b]}{[np4]}$
Postać parametryczna:
$$t=y-\frac{[b]}{[np4]}=\frac{[p2]}{[p4]}x$$
$$l:\left\{ \begin{array}{ll}
x=\frac{[p4]}{[p2]}t\\
y=t+\frac{[b]}{[np4]}
\end{array} \right  t\in\mathbb{R}.$$

\rozwStop


\odpStart
$l:\left\{ \begin{array}{ll}
x=\frac{[p4]}{[p2]}t\\
y=t+\frac{[b]}{[np4]}
\end{array} \right  t\in\mathbb{R}$
\odpStop
\testStart
A.$l:\left\{ \begin{array}{ll}
x=\frac{[p4]}{[p2]}t\\
y=t+\frac{[b]}{[np4]}
\end{array} \right  t\in\mathbb{R}.$
B.$l:\left\{ \begin{array}{ll}
x=\frac{[p4]}{[p2]}\\
y=t
\end{array} \right  t\in\mathbb{R}.$
D.$l:\left\{ \begin{array}{ll}
x=\frac{[p4]}{[p3]}t\\
y=\frac{[b]}{[np4]}
\end{array} \right  t\in\mathbb{R}.$
E.$l:\left\{ \begin{array}{ll}
x=\frac{[p3]}{[p2]}t\\
y=t-\frac{[b]}{[np4]}
\end{array} \right  t\in\mathbb{R}.$
F.$l:\left\{ \begin{array}{ll}
x=[p4]t\\
y=-t-\frac{[b]}{[np4]}
\end{array} \right  t\in\mathbb{R}.$
G.$l:\left\{ \begin{array}{ll}
x=\frac{[p4]}{[p2]}\\
y=t-\frac{[b]}{[np4]}
\end{array} \right  t\in\mathbb{R}.$
H.$l:\left\{ \begin{array}{ll}
x=[p1]t+[p2]\\
y=t+\frac{[b]}{[np4]}
\end{array} \right  t\in\mathbb{R}.$
\testStop
\kluczStart
A
\kluczStop



\end{document}
