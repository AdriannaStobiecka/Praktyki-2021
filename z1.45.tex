\documentclass[12pt, a4paper]{article}
\usepackage[utf8]{inputenc}
\usepackage{polski}

\usepackage{amsthm}  %pakiet do tworzenia twierdzeń itp.
\usepackage{amsmath} %pakiet do niektórych symboli matematycznych
\usepackage{amssymb} %pakiet do symboli mat., np. \nsubseteq
\usepackage{amsfonts}
\usepackage{graphicx} %obsługa plików graficznych z rozszerzeniem png, jpg
\theoremstyle{definition} %styl dla definicji
\newtheorem{zad}{} 
\title{Multizestaw zadań}
\author{Laura Mieczkowska}
%\date{\today}
\date{}
\newcounter{liczniksekcji}
\newcommand{\kategoria}[1]{\section{#1}} %olreślamy nazwę kateforii zadań
\newcommand{\zadStart}[1]{\begin{zad}#1\newline} %oznaczenie początku zadania
\newcommand{\zadStop}{\end{zad}}   %oznaczenie końca zadania
%Makra opcjonarne (nie muszą występować):
\newcommand{\rozwStart}[2]{\noindent \textbf{Rozwiązanie (autor #1 , recenzent #2): }\newline} %oznaczenie początku rozwiązania, opcjonarnie można wprowadzić informację o autorze rozwiązania zadania i recenzencie poprawności wykonania rozwiązania zadania
\newcommand{\rozwStop}{\newline}                                            %oznaczenie końca rozwiązania
\newcommand{\odpStart}{\noindent \textbf{Odpowiedź:}\newline}    %oznaczenie początku odpowiedzi końcowej (wypisanie wyniku)
\newcommand{\odpStop}{\newline}                                             %oznaczenie końca odpowiedzi końcowej (wypisanie wyniku)
\newcommand{\testStart}{\noindent \textbf{Test:}\newline} %ewentualne możliwe opcje odpowiedzi testowej: A. ? B. ? C. ? D. ? itd.
\newcommand{\testStop}{\newline} %koniec wprowadzania odpowiedzi testowych
\newcommand{\kluczStart}{\noindent \textbf{Test poprawna odpowiedź:}\newline} %klucz, poprawna odpowiedź pytania testowego (jedna literka): A lub B lub C lub D itd.
\newcommand{\kluczStop}{\newline} %koniec poprawnej odpowiedzi pytania testowego 
\newcommand{\wstawGrafike}[2]{\begin{figure}[h] \includegraphics[scale=#2] {#1} \end{figure}} %gdyby była potrzeba wstawienia obrazka, parametry: nazwa pliku, skala (jak nie wiesz co wpisać, to wpisz 1)

\begin{document}
\maketitle


\kategoria{Wikieł/Z1.45}
\zadStart{Zadanie z Wikieł Z 1.45) moja wersja nr [nrWersji]}
%[a]:[2,3,4,5,6,7,8]
%[b]:[2,3,4,5,6,7,8,9,10]
%[c]:[4,9,16,25,36,49,64]
%[cp2]=math.sqrt([c])
%[cp1]=[cp2].real
%[cp]=int([cp1])
%[d]=[b]**2
%[e]=4*[a]
%[f]=4*[a]*[c]
%[g]=abs([d]-[e])
%[delta]=4*[g]*[f]
%[pierw2]=math.sqrt([delta])
%[pierw1]=[pierw2].real
%[pierw]=int([pierw1])
%[mian2]=abs(-2*[g])
%[dziel]=math.gcd([pierw],[mian2])
%[licz1]=[pierw]/([dziel]+0.001)
%[l]=round([licz1],1)
%[licz]=int([l])
%[mian1]=[mian2]/([dziel]+0.001)
%[m]=round([mian1],1)
%[mian]=int([m])
%[ulamek1]=[licz]/([mian]+0.001)
%[ulamek]=int(round([ulamek1],1))
%[d]<[e] and [pierw2].is_integer()==True and [cp2].is_integer()==True
Wyznaczyć wartości parametru $m$, dla których oba pierwiastki równania $[a]x^2+[b]mx+m^2-[c]=0$ są ujemne.
\zadStop
\rozwStart{Laura Mieczkowska}{}
$$[a]x^2+[b]mx+m^2-[c]=0$$
$$\Delta=([b]m)^2-4\cdot[a]\cdot(m^2-[c])=[d]m^2-[e]m^2+[f]=-[g]m^2+[f]>0$$
$$\Delta_m=0-4\cdot(-[g])\cdot[f]=[delta] \Rightarrow \sqrt{\Delta}=\sqrt{[delta]}=[pierw]$$
$$m_1=\frac{-[pierw]}{-2\cdot[g]} \vee m_2=\frac{[pierw]}{-2\cdot[g]}$$
$$m_1=[ulamek] \vee m_2=-[ulamek]$$
$$m\in (-[ulamek];[ulamek])$$
$$x_1+x_2<0 \Rightarrow -\frac{b}{a}$$
$$\frac{-[b]m}{[a]}<0 \Rightarrow -[b]m<0 \Rightarrow m>0$$
$$x_1x_2>0 \Rightarrow \frac{c}{a}$$
$$\frac{m^2-[c]}{[a]}>0 \Rightarrow m^2-[c]>0 \Rightarrow (m-[cp])(m+[cp])>0$$
$$\Rightarrow m\in(-\infty;-[cp])\cup([cp];\infty)$$
\\Część wspólna warunków $m\in (-[ulamek];[ulamek])$, $m>0$, $m\in(-\infty;-[cp])\cup([cp];\infty)$
to $m\in([cp];[ulamek])$ 
$$$$
\odpStart
$m\in([cp];[ulamek])$
\odpStop
\testStart
A. $m\in(-[ulamek];[ulamek]))$ \\
B. $m\in([cp];[ulamek])$ \\
C. $m\in(-\infty;-[cp])\cup([cp];\infty)$ \\
D. $m\in(-[cp];[ulamek])$ 
\testStop
\kluczStart
B
\kluczStop



\end{document}