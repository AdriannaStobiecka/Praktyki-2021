\documentclass[12pt, a4paper]{article}
\usepackage[utf8]{inputenc}
\usepackage{polski}

\usepackage{amsthm}  %pakiet do tworzenia twierdzeń itp.
\usepackage{amsmath} %pakiet do niektórych symboli matematycznych
\usepackage{amssymb} %pakiet do symboli mat., np. \nsubseteq
\usepackage{amsfonts}
\usepackage{graphicx} %obsługa plików graficznych z rozszerzeniem png, jpg
\theoremstyle{definition} %styl dla definicji
\newtheorem{zad}{} 
\title{Multizestaw zadań}
\author{Robert Fidytek}
%\date{\today}
\date{}
\newcounter{liczniksekcji}
\newcommand{\kategoria}[1]{\section{#1}} %olreślamy nazwę kateforii zadań
\newcommand{\zadStart}[1]{\begin{zad}#1\newline} %oznaczenie początku zadania
\newcommand{\zadStop}{\end{zad}}   %oznaczenie końca zadania
%Makra opcjonarne (nie muszą występować):
\newcommand{\rozwStart}[2]{\noindent \textbf{Rozwiązanie (autor #1 , recenzent #2): }\newline} %oznaczenie początku rozwiązania, opcjonarnie można wprowadzić informację o autorze rozwiązania zadania i recenzencie poprawności wykonania rozwiązania zadania
\newcommand{\rozwStop}{\newline}                                            %oznaczenie końca rozwiązania
\newcommand{\odpStart}{\noindent \textbf{Odpowiedź:}\newline}    %oznaczenie początku odpowiedzi końcowej (wypisanie wyniku)
\newcommand{\odpStop}{\newline}                                             %oznaczenie końca odpowiedzi końcowej (wypisanie wyniku)
\newcommand{\testStart}{\noindent \textbf{Test:}\newline} %ewentualne możliwe opcje odpowiedzi testowej: A. ? B. ? C. ? D. ? itd.
\newcommand{\testStop}{\newline} %koniec wprowadzania odpowiedzi testowych
\newcommand{\kluczStart}{\noindent \textbf{Test poprawna odpowiedź:}\newline} %klucz, poprawna odpowiedź pytania testowego (jedna literka): A lub B lub C lub D itd.
\newcommand{\kluczStop}{\newline} %koniec poprawnej odpowiedzi pytania testowego 
\newcommand{\wstawGrafike}[2]{\begin{figure}[h] \includegraphics[scale=#2] {#1} \end{figure}} %gdyby była potrzeba wstawienia obrazka, parametry: nazwa pliku, skala (jak nie wiesz co wpisać, to wpisz 1)

\begin{document}
\maketitle


\kategoria{Wikieł/Z3.13b}
\zadStart{Zadanie z Wikieł Z 3.13 b) moja wersja nr [nrWersji]}
%[b]:[3,5,7,9,11,13,15,17]
%[c]:[2,3,4,5,6,7,8,9]
%math.gcd([b],2)==1
Obliczyć granicę ciągu 
$$a_n=\big(\sqrt{n^2+[c]}-\sqrt{n^2+[b]n+[c]}\big).$$
\zadStop
\rozwStart{Adrianna Stobiecka}{}
$$\lim_{n\to\infty}\big(\sqrt{n^2+[c]}-\sqrt{n^2+[b]n+[c]}\big)$$
$$=\lim_{n\to\infty}\frac{\big(\sqrt{n^2+[c]}-\sqrt{n^2+[b]n+[c]}\big)\big(\sqrt{n^2+[c]}+\sqrt{n^2+[b]n+[c]}\big)}{\sqrt{n^2+[c]}+\sqrt{n^2+[b]n+[c]}}=(*)$$
W liczniku skorzystamy ze wzoru $(a-b)(a+b)=a^2-b^2$.
$$(*)=\lim_{n\to\infty}\frac{n^2+[c]-n^2-[b]n-[c]}{\sqrt{n^2+[c]}+\sqrt{n^2+[b]n+[c]}}=\lim_{n\to\infty}\frac{-[b]n}{\sqrt{n^2+[c]}+\sqrt{n^2+[b]n+[c]}}$$
$$=\lim_{n\to\infty}\frac{-[b]n}{\sqrt{n^2(1+\frac{[c]}{n^2})}+\sqrt{n^2(1+\frac{[b]}{n}+\frac{[c]}{n^2})}}=\lim_{n\to\infty}\frac{n\cdot(-[b])}{n\sqrt{1+\frac{[c]}{n^2}}+n\sqrt{1+\frac{[b]}{n}+\frac{[c]}{n^2}}}$$
$$\lim_{n\to\infty}\frac{-[b]}{\sqrt{1+\frac{[c]}{n^2}}+\sqrt{1+\frac{[b]}{n}+\frac{[c]}{n^2}}}=(**)$$
Wiemy, że $$\lim_{n\to\infty}\frac{[c]}{n^2}=0$$
oraz 
$$\lim_{n\to\infty}\frac{[b]}{n}=0.$$
Mamy zatem:
$$(**)=\frac{-[b]}{1+1}=-\frac{[b]}{2}$$
\rozwStop
\odpStart
$-\frac{[b]}{2}$
\odpStop
\testStart
A.$-\frac{[b]}{2}$
B.$-[b]$
C.$e^{-[b]}$
D.$0$
E.$-\infty$
F.$e^{[b]}$
G.$1$
H.$\infty$
I.$-1$
\testStop
\kluczStart
A
\kluczStop



\end{document}
