\documentclass[12pt, a4paper]{article}
\usepackage[utf8]{inputenc}
\usepackage{polski}

\usepackage{amsthm}  %pakiet do tworzenia twierdzeń itp.
\usepackage{amsmath} %pakiet do niektórych symboli matematycznych
\usepackage{amssymb} %pakiet do symboli mat., np. \nsubseteq
\usepackage{amsfonts}
\usepackage{graphicx} %obsługa plików graficznych z rozszerzeniem png, jpg
\theoremstyle{definition} %styl dla definicji
\newtheorem{zad}{} 
\title{Multizestaw zadań}
\author{Robert Fidytek}
%\date{\today}
\date{}
\newcounter{liczniksekcji}
\newcommand{\kategoria}[1]{\section{#1}} %olreślamy nazwę kateforii zadań
\newcommand{\zadStart}[1]{\begin{zad}#1\newline} %oznaczenie początku zadania
\newcommand{\zadStop}{\end{zad}}   %oznaczenie końca zadania
%Makra opcjonarne (nie muszą występować):
\newcommand{\rozwStart}[2]{\noindent \textbf{Rozwiązanie (autor #1 , recenzent #2): }\newline} %oznaczenie początku rozwiązania, opcjonarnie można wprowadzić informację o autorze rozwiązania zadania i recenzencie poprawności wykonania rozwiązania zadania
\newcommand{\rozwStop}{\newline}                                            %oznaczenie końca rozwiązania
\newcommand{\odpStart}{\noindent \textbf{Odpowiedź:}\newline}    %oznaczenie początku odpowiedzi końcowej (wypisanie wyniku)
\newcommand{\odpStop}{\newline}                                             %oznaczenie końca odpowiedzi końcowej (wypisanie wyniku)
\newcommand{\testStart}{\noindent \textbf{Test:}\newline} %ewentualne możliwe opcje odpowiedzi testowej: A. ? B. ? C. ? D. ? itd.
\newcommand{\testStop}{\newline} %koniec wprowadzania odpowiedzi testowych
\newcommand{\kluczStart}{\noindent \textbf{Test poprawna odpowiedź:}\newline} %klucz, poprawna odpowiedź pytania testowego (jedna literka): A lub B lub C lub D itd.
\newcommand{\kluczStop}{\newline} %koniec poprawnej odpowiedzi pytania testowego 
\newcommand{\wstawGrafike}[2]{\begin{figure}[h] \includegraphics[scale=#2] {#1} \end{figure}} %gdyby była potrzeba wstawienia obrazka, parametry: nazwa pliku, skala (jak nie wiesz co wpisać, to wpisz 1)

\begin{document}
\maketitle


\kategoria{Wikieł/Z1.93h}
\zadStart{Zadanie z Wikieł Z 1.93 h) moja wersja nr [nrWersji]}
%[a]:[2,3,4,5,6]
%[b]:[2,3,4,5,6]
%[c]:[1,2,3,4,5,6,7,8,9,10,11,12,13,14,15,16,17,18,19,20]
%[t]=random.randint(2,9)
%[bb]=(pow([b],2))
%[bc]=(2*[b])*[c]
%[cc]=(pow([c],2))
%[2bc]=[bc]+[a]
%[j]=([c]+1)
%[h]=math.gcd([j],[b])
%[j2]=int([j]/[h])
%[b2]=int([b]/[h])
%[del]=([2bc]**2)
%[ta]=((4*[bb])*[cc])
%[delta]=([del]-[ta])
%[p]=(pow([delta],1/2))
%[pp]=int([p].real)
%[z1]=[2bc]-[pp]
%[z2]=[2bc]+[pp]
%[mianownik]=(2*[bb])
%[k1]=math.gcd([mianownik],[z1])
%[k2]=math.gcd([mianownik],[z2])
%[e1]=int([z1]/[k1])
%[f1]=int([mianownik]/[k1])
%[e2]=int([z2]/[k2])
%[f2]=int([mianownik]/[k2])
%math.gcd([b],[c])==1 and ([j]/[b]).is_integer()==False and [delta]>0 and [p].is_integer()==True and ([z1]/[mianownik]).is_integer()==False and ([z2]/[mianownik]).is_integer()==False and ([e1]/[f1])<([c]/[b]) and ([e2]/[f2])>([j2]/[b2])
Rozwiązać równanie $\frac{\log_{[t]}{[a]x}}{\log_{[t]}{([b]x-[c])}}= 2$
\zadStop
\rozwStart{Małgorzata Ugowska}{}
Zacznijmy od dziedziny:
$$[b]x-[c]>0 \quad \Longrightarrow \quad x>\frac{[c]}{[b]}$$
$$[a]x>0 \quad \Longrightarrow \quad x>0$$
$$\log_{[t]}{([b]x-[c])} \ne 0 \quad \Longrightarrow \quad [b]x-[c] \ne 1 \quad \Longrightarrow \quad x \ne \frac{[j2]}{[b2]}$$
$$D = \Big(\frac{[c]}{[b]}, \frac{[j2]}{[b2]}\Big) \cup \Big(\frac{[j2]}{[b2]}, \infty) $$
Teraz przechodzimy do rozwiązywania równania.
$$ \frac{\log_{[t]}{[a]x}}{\log_{[t]}{([b]x-[c])}}= 2 \quad \Longleftrightarrow \quad \log_{([b]x-[c])}{[a]x}= 2 \quad \Longleftrightarrow \quad ([b]x-[c])^2=[a]x$$
$$\Longleftrightarrow \quad [bb] x^2 - [bc]x + [cc] = [a] x \quad \Longleftrightarrow \quad [bb] x^2 - [2bc]x + [cc] = 0 $$
$$ \bigtriangleup = [2bc]^2 - 4 \cdot [bb] \cdot [cc] = [delta], \qquad \sqrt{\bigtriangleup} = [pp]$$
$$ x_1=\frac{[2bc] -\sqrt{\bigtriangleup}}{2 \cdot [bb]} = \frac{[z1]}{[mianownik]} = \frac{[e1]}{[f1]} \notin D$$
$$ x_2=\frac{[2bc]+\sqrt{\bigtriangleup}}{2 \cdot [bb]} = \frac{[z2]}{[mianownik]} = \frac{[e2]}{[f2]} \in D$$
\rozwStop
\odpStart
$x = \frac{[e2]}{[f2]}$
\odpStop
\testStart
A. $x = \frac{[e1]}{[f1]}$\\
B. $x \in \{0, 1\}$\\
C. $x \in \{\frac{1}{4}, \frac{1}{11}\}$\\
D. $x \in \{\frac{[e1]}{[f1]}, \frac{[e2]}{[f2]}\}$\\
E. $x = \frac{[e2]}{[f2]}$
\testStop
\kluczStart
E
\kluczStop



\end{document}