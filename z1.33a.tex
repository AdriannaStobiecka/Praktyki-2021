\documentclass[12pt, a4paper]{article}
\usepackage[utf8]{inputenc}
\usepackage{polski}

\usepackage{amsthm}  %pakiet do tworzenia twierdzeń itp.
\usepackage{amsmath} %pakiet do niektórych symboli matematycznych
\usepackage{amssymb} %pakiet do symboli mat., np. \nsubseteq
\usepackage{amsfonts}
\usepackage{graphicx} %obsługa plików graficznych z rozszerzeniem png, jpg
\theoremstyle{definition} %styl dla definicji
\newtheorem{zad}{} 
\title{Multizestaw zadań}
\author{Laura Mieczkowska}
%\date{\today}
\date{}
\newcounter{liczniksekcji}
\newcommand{\kategoria}[1]{\section{#1}} %olreślamy nazwę kateforii zadań
\newcommand{\zadStart}[1]{\begin{zad}#1\newline} %oznaczenie początku zadania
\newcommand{\zadStop}{\end{zad}}   %oznaczenie końca zadania
%Makra opcjonarne (nie muszą występować):
\newcommand{\rozwStart}[2]{\noindent \textbf{Rozwiązanie (autor #1 , recenzent #2): }\newline} %oznaczenie początku rozwiązania, opcjonarnie można wprowadzić informację o autorze rozwiązania zadania i recenzencie poprawności wykonania rozwiązania zadania
\newcommand{\rozwStop}{\newline}                                            %oznaczenie końca rozwiązania
\newcommand{\odpStart}{\noindent \textbf{Odpowiedź:}\newline}    %oznaczenie początku odpowiedzi końcowej (wypisanie wyniku)
\newcommand{\odpStop}{\newline}                                             %oznaczenie końca odpowiedzi końcowej (wypisanie wyniku)
\newcommand{\testStart}{\noindent \textbf{Test:}\newline} %ewentualne możliwe opcje odpowiedzi testowej: A. ? B. ? C. ? D. ? itd.
\newcommand{\testStop}{\newline} %koniec wprowadzania odpowiedzi testowych
\newcommand{\kluczStart}{\noindent \textbf{Test poprawna odpowiedź:}\newline} %klucz, poprawna odpowiedź pytania testowego (jedna literka): A lub B lub C lub D itd.
\newcommand{\kluczStop}{\newline} %koniec poprawnej odpowiedzi pytania testowego 
\newcommand{\wstawGrafike}[2]{\begin{figure}[h] \includegraphics[scale=#2] {#1} \end{figure}} %gdyby była potrzeba wstawienia obrazka, parametry: nazwa pliku, skala (jak nie wiesz co wpisać, to wpisz 1)

\begin{document}
\maketitle


\kategoria{Wikieł/Z1.33a}
\zadStart{Zadanie z Wikieł Z 1.33 a) moja wersja nr [nrWersji]}
%[a]:[2,3,4,5,6,7,8,9,10]
%[b]:[2,3,4,5,6,7,8,9,10]
%[c]:[2,3,4,5,6,7,8,9,10]
%[d]=2*[a]
%[akw]=[a]**2
%[ab]=[a]*[b]
%[pkw]=[b]*[b]*[c]
%[w]=[akw]-[pkw]
%[a]>[b] and [a]>[c] and [akw]>[pkw]
Podać przykłąd trójmianu kwadratowego o współczynnikach całkowitych, którego pierwiastkami są pary liczb $[a]-[b]\sqrt{[c]}$ i $[a]+[b]\sqrt{[c]}$.
\zadStop
\rozwStart{Laura Mieczkowska}{}
Dane są pierwiastki równania kwadratowego $x_1=[a]-[b]\sqrt{[c]}$ i $x_2=[a]+[b]\sqrt{[c]}$.
\\\\Ponieważ
$$(x-x_1)(x-x_2)=0 \Leftrightarrow x^2-(x_1+x_2)x+x_1x_2=0$$
więc obliczamy:
$$x_1+x_2=[a]-[b]\sqrt{[c]}+[a]+[b]\sqrt{[c]}=[d]$$
$$x_1\cdot x_2=([a]-[b]\sqrt{[c]})([a]+[b]\sqrt{[c]})=[akw]+[ab]\sqrt{[c]}-[ab]\sqrt{[c]}-[pkw]=[w]$$
\\Ostatecznie
$$x^2-[d]x+[w]$$
\odpStart
$x^2-[d]x+[w]$
\odpStop
\testStart
A. $x^2+[d]x+[w]$ \\
B. $x^2-[d]x-[w]$ \\
C. $x^2-[d]x+[w]$ \\
D. $-x^2-[d]x+[w]$ 
\testStop
\kluczStart
C
\kluczStop



\end{document}