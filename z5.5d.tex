\documentclass[12pt, a4paper]{article}
\usepackage[utf8]{inputenc}
\usepackage{polski}

\usepackage{amsthm}  %pakiet do tworzenia twierdzeń itp.
\usepackage{amsmath} %pakiet do niektórych symboli matematycznych
\usepackage{amssymb} %pakiet do symboli mat., np. \nsubseteq
\usepackage{amsfonts}
\usepackage{graphicx} %obsługa plików graficznych z rozszerzeniem png, jpg
\theoremstyle{definition} %styl dla definicji
\newtheorem{zad}{} 
\title{Multizestaw zadań}
\author{Robert Fidytek}
%\date{\today}
\date{}
\newcounter{liczniksekcji}
\newcommand{\kategoria}[1]{\section{#1}} %olreślamy nazwę kateforii zadań
\newcommand{\zadStart}[1]{\begin{zad}#1\newline} %oznaczenie początku zadania
\newcommand{\zadStop}{\end{zad}}   %oznaczenie końca zadania
%Makra opcjonarne (nie muszą występować):
\newcommand{\rozwStart}[2]{\noindent \textbf{Rozwiązanie (autor #1 , recenzent #2): }\newline} %oznaczenie początku rozwiązania, opcjonarnie można wprowadzić informację o autorze rozwiązania zadania i recenzencie poprawności wykonania rozwiązania zadania
\newcommand{\rozwStop}{\newline}                                            %oznaczenie końca rozwiązania
\newcommand{\odpStart}{\noindent \textbf{Odpowiedź:}\newline}    %oznaczenie początku odpowiedzi końcowej (wypisanie wyniku)
\newcommand{\odpStop}{\newline}                                             %oznaczenie końca odpowiedzi końcowej (wypisanie wyniku)
\newcommand{\testStart}{\noindent \textbf{Test:}\newline} %ewentualne możliwe opcje odpowiedzi testowej: A. ? B. ? C. ? D. ? itd.
\newcommand{\testStop}{\newline} %koniec wprowadzania odpowiedzi testowych
\newcommand{\kluczStart}{\noindent \textbf{Test poprawna odpowiedź:}\newline} %klucz, poprawna odpowiedź pytania testowego (jedna literka): A lub B lub C lub D itd.
\newcommand{\kluczStop}{\newline} %koniec poprawnej odpowiedzi pytania testowego 
\newcommand{\wstawGrafike}[2]{\begin{figure}[h] \includegraphics[scale=#2] {#1} \end{figure}} %gdyby była potrzeba wstawienia obrazka, parametry: nazwa pliku, skala (jak nie wiesz co wpisać, to wpisz 1)

\begin{document}
\maketitle


\kategoria{Wikieł/Z5.5d}
\zadStart{Zadanie z Wikieł Z 5.5 d) moja wersja nr [nrWersji]}
%[a]=random.randint(2,10)
%[b]:[2,3,4,5,6,7,8,9]
%[b1]=2*[b]
%[c]:[2,3,4,5,6,7,8,9]
%[d]=random.randint(2,10)
%
%math.gcd(3,[a])==1
Wyznacz pochodną funkcji \\ $f(x)=\frac{1}{[a]}x^3-[b]x^2+[c]x-[d]$.
\zadStop
\rozwStart{Joanna Świerzbin}{}
$$f(x)=\frac{1}{[a]}x^3-[b]x^2+[c]x-[d]$$
$$f'(x)=\left(\frac{1}{[a]}x^3-[b]x^2+[c]x-[d]\right)' = $$
$$ = \frac{3}{[a]}x^2-2\cdot[b]x+[c] = $$
$$ =  \frac{3}{[a]}x^2-[b1]x+[c] $$
\rozwStop
\odpStart
$ f'(x)=  \frac{3}{[a]}x^2-[b1]x+[c] $
\odpStop
\testStart
A.$ f'(x)= [a]x^2 - [b1]x+ [c]x $\\
B. $ f'(x)=  \frac{3}{[a]}x^2-[b1]x+[c] $ \\
C. $ f'(x)= x^2- [c]x $ \\
D. $ f'(x)=  \frac{1}{[a]}x^2-[b1]x+[c] $\\
E. $ f'(x)=  \frac{3}{[a]}x^2-[b]x+[c] $\\
F. $ f'(x)=  \frac{3}{[a]}x^2-[b1]x $
\testStop
\kluczStart
B
\kluczStop



\end{document}