\documentclass[12pt, a4paper]{article}
\usepackage[utf8]{inputenc}
\usepackage{polski}

\usepackage{amsthm}  %pakiet do tworzenia twierdzeń itp.
\usepackage{amsmath} %pakiet do niektórych symboli matematycznych
\usepackage{amssymb} %pakiet do symboli mat., np. \nsubseteq
\usepackage{amsfonts}
\usepackage{graphicx} %obsługa plików graficznych z rozszerzeniem png, jpg
\theoremstyle{definition} %styl dla definicji
\newtheorem{zad}{} 
\title{Multizestaw zadań}
\author{Robert Fidytek}
%\date{\today}
\date{}
\newcounter{liczniksekcji}
\newcommand{\kategoria}[1]{\section{#1}} %olreślamy nazwę kateforii zadań
\newcommand{\zadStart}[1]{\begin{zad}#1\newline} %oznaczenie początku zadania
\newcommand{\zadStop}{\end{zad}}   %oznaczenie końca zadania
%Makra opcjonarne (nie muszą występować):
\newcommand{\rozwStart}[2]{\noindent \textbf{Rozwiązanie (autor #1 , recenzent #2): }\newline} %oznaczenie początku rozwiązania, opcjonarnie można wprowadzić informację o autorze rozwiązania zadania i recenzencie poprawności wykonania rozwiązania zadania
\newcommand{\rozwStop}{\newline}                                            %oznaczenie końca rozwiązania
\newcommand{\odpStart}{\noindent \textbf{Odpowiedź:}\newline}    %oznaczenie początku odpowiedzi końcowej (wypisanie wyniku)
\newcommand{\odpStop}{\newline}                                             %oznaczenie końca odpowiedzi końcowej (wypisanie wyniku)
\newcommand{\testStart}{\noindent \textbf{Test:}\newline} %ewentualne możliwe opcje odpowiedzi testowej: A. ? B. ? C. ? D. ? itd.
\newcommand{\testStop}{\newline} %koniec wprowadzania odpowiedzi testowych
\newcommand{\kluczStart}{\noindent \textbf{Test poprawna odpowiedź:}\newline} %klucz, poprawna odpowiedź pytania testowego (jedna literka): A lub B lub C lub D itd.
\newcommand{\kluczStop}{\newline} %koniec poprawnej odpowiedzi pytania testowego 
\newcommand{\wstawGrafike}[2]{\begin{figure}[h] \includegraphics[scale=#2] {#1} \end{figure}} %gdyby była potrzeba wstawienia obrazka, parametry: nazwa pliku, skala (jak nie wiesz co wpisać, to wpisz 1)

\begin{document}
\maketitle


\kategoria{Wikieł/Z3.1e}
\zadStart{Zadanie z Wikieł Z 3.1 e) moja wersja nr 1}
%[a]:[1, 2, 3, 4, 5]
%[b]:[1, 2, 3, 4, 5]
%[c]:[1, 2, 3, 4, 5]
%[d]:[1, 2, 3, 4, 5]
%[a]=random.randint(2,20)
%[b]=random.randint(2,20)
%[c]=random.randint(2,20)
%[d]=random.randint(2,20)
%[apb]=[a]+[b]
%[cpd]=[c]+[d]
%[ad]=[a]*[d]
%[apbc]=[apb]*[c]
%[cpda]=[cpd]*[a]
%[bc]=[b]*[c]
%[apbd]=[apb]*[d]
%[cpdb]=[cpd]*[b]
%[licznik]=[apbd]-[cpdb]
%[ad]+[apbc]-[cpda]-[bc]=0
%[licznik]>0
Zbadać monotoniczność ciągu $a_{n}$.\\ $a_{n}=\frac{[a]n+[b]}{[c]n+[d]}$
\zadStop
\rozwStart{Jakub Ulrych}{}
$$a_{n+1}=\frac{[a](n+1)+[b]}{[c](n+1)+[d]}=\frac{[a]n+[apb]}{[c]n+[cpd]}$$
$$a_{n+1}-a_{n}=\frac{[a]n+[apb]}{[c]n+[cpd]}-\frac{[a]n+[b]}{[c]n+[d]}$$
$$\frac{([a]n+[apb])([c]n+[d])-([a]n+[b])([c]n+[cpd])}{([c]n+[cpd])([c]n+[d])}$$
$$\frac{[licznik]}{([c]n+[cpd])([c]n+[d])}$$
dla każdego $n\in\mathbb{N}$ zachodzi
$$\frac{[licznik]}{([c]n+[cpd])([c]n+[d])}>0\Rightarrow \text{Ciąg jest rosnący}$$
\rozwStop
\odpStart
$$\text{Ciąg jest rosnący}$$
\odpStop
\testStart
A.$\text{Ciąg jest rosnący}$
B.$\text{Ciąg jest malejący}$
\testStop
\kluczStart
A
\kluczStop



\end{document}