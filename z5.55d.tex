\documentclass[12pt, a4paper]{article}
\usepackage[utf8]{inputenc}
\usepackage{polski}

\usepackage{amsthm}  %pakiet do tworzenia twierdzeń itp.
\usepackage{amsmath} %pakiet do niektórych symboli matematycznych
\usepackage{amssymb} %pakiet do symboli mat., np. \nsubseteq
\usepackage{amsfonts}
\usepackage{graphicx} %obsługa plików graficznych z rozszerzeniem png, jpg
\theoremstyle{definition} %styl dla definicji
\newtheorem{zad}{} 
\title{Multizestaw zadań}
\author{Robert Fidytek}
%\date{\today}
\date{}
\newcounter{liczniksekcji}
\newcommand{\kategoria}[1]{\section{#1}} %olreślamy nazwę kateforii zadań
\newcommand{\zadStart}[1]{\begin{zad}#1\newline} %oznaczenie początku zadania
\newcommand{\zadStop}{\end{zad}}   %oznaczenie końca zadania
%Makra opcjonarne (nie muszą występować):
\newcommand{\rozwStart}[2]{\noindent \textbf{Rozwiązanie (autor #1 , recenzent #2): }\newline} %oznaczenie początku rozwiązania, opcjonarnie można wprowadzić informację o autorze rozwiązania zadania i recenzencie poprawności wykonania rozwiązania zadania
\newcommand{\rozwStop}{\newline}                                            %oznaczenie końca rozwiązania
\newcommand{\odpStart}{\noindent \textbf{Odpowiedź:}\newline}    %oznaczenie początku odpowiedzi końcowej (wypisanie wyniku)
\newcommand{\odpStop}{\newline}                                             %oznaczenie końca odpowiedzi końcowej (wypisanie wyniku)
\newcommand{\testStart}{\noindent \textbf{Test:}\newline} %ewentualne możliwe opcje odpowiedzi testowej: A. ? B. ? C. ? D. ? itd.
\newcommand{\testStop}{\newline} %koniec wprowadzania odpowiedzi testowych
\newcommand{\kluczStart}{\noindent \textbf{Test poprawna odpowiedź:}\newline} %klucz, poprawna odpowiedź pytania testowego (jedna literka): A lub B lub C lub D itd.
\newcommand{\kluczStop}{\newline} %koniec poprawnej odpowiedzi pytania testowego 
\newcommand{\wstawGrafike}[2]{\begin{figure}[h] \includegraphics[scale=#2] {#1} \end{figure}} %gdyby była potrzeba wstawienia obrazka, parametry: nazwa pliku, skala (jak nie wiesz co wpisać, to wpisz 1)

\begin{document}
\maketitle


\kategoria{Wikieł/Z5.55d}
\zadStart{Zadanie z Wikieł Z 5.55d) moja wersja nr [nrWersji]}
%[a]=random.randint(0,10)
%[b]:[1,2,3,4,5,6,7]
%[c]:[-7,-6,-5,-4,-3,-2]
%[d]:[-5,-4,-3,-2]
%[e]:[1,2,3,4]
%[be]=[b]*[e]
%[cd]=[c]*[d]
%[cc]=[c]*[c]
%[l]=[cd]-[be]
%math.gcd([l],[cc])==1  and math.gcd([cc],[a])==1 and [a]!=[l] and [a]!=1 and [cd]>[be] and [l]!=1
Na podstawie podanych wartości $f'([a])=[b],$ $f([a])=[c],$ $g'([a])=[d],$ $g([a])=[e]$ obliczyć wartość następującej pochodnej $\frac{d}{dx}\left[\frac{g(x)}{f(x)}\right]\big|_{x=[a]}$.
\zadStop
\rozwStart{Justyna Chojecka}{}
Zauważmy, że 
$$\left(\frac{g(x)}{f(x)}\right)'=\frac{g'(x)\cdot f(x)-g(x)\cdot f'(x)}{\left[f(x)\right]^{2}}.$$
Obliczamy wartość pochodnej $\left(\frac{f(x)}{g(x)}\right)'$ dla $x=[a]$.
$$\left(\frac{g([a])}{f([a])}\right)'=\frac{g'([a])\cdot f([a])-g([a])\cdot f'([a])}{\left[f([a])\right]^{2}}=\frac{[d]\cdot ([c]) - [e]\cdot [b]}{([c])^{2}}=\frac{[cd]-[be]}{[cc]}=\frac{[l]}{[cc]}$$
\rozwStop
\odpStart
$\frac{[l]}{[cc]}$
\odpStop
\testStart
A.$\frac{[l]}{[cc]}$
B.$\frac{[a]}{[cc]}$
C.$-\frac{[cc]}{[l]}$
D.$\frac{[cc]}{[a]}$
E.$-[a]$
F.$-\frac{[l]}{[cc]}$
G.$-\frac{[cc]}{[a]}$
H.$-\frac{[a]}{[cc]}$
I.$\frac{[cc]}{[l]}$
\testStop
\kluczStart
A
\kluczStop



\end{document}