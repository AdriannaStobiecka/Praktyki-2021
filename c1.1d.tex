\documentclass[12pt, a4paper]{article}
\usepackage[utf8]{inputenc}
\usepackage{polski}

\usepackage{amsthm}  %pakiet do tworzenia twierdzeń itp.
\usepackage{amsmath} %pakiet do niektórych symboli matematycznych
\usepackage{amssymb} %pakiet do symboli mat., np. \nsubseteq
\usepackage{amsfonts}
\usepackage{graphicx} %obsługa plików graficznych z rozszerzeniem png, jpg
\theoremstyle{definition} %styl dla definicji
\newtheorem{zad}{} 
\title{Multizestaw zadań}
\author{Robert Fidytek}
%\date{\today}
\date{}
\newcounter{liczniksekcji}
\newcommand{\kategoria}[1]{\section{#1}} %olreślamy nazwę kateforii zadań
\newcommand{\zadStart}[1]{\begin{zad}#1\newline} %oznaczenie początku zadania
\newcommand{\zadStop}{\end{zad}}   %oznaczenie końca zadania
%Makra opcjonarne (nie muszą występować):
\newcommand{\rozwStart}[2]{\noindent \textbf{Rozwiązanie (autor #1 , recenzent #2): }\newline} %oznaczenie początku rozwiązania, opcjonarnie można wprowadzić informację o autorze rozwiązania zadania i recenzencie poprawności wykonania rozwiązania zadania
\newcommand{\rozwStop}{\newline}                                            %oznaczenie końca rozwiązania
\newcommand{\odpStart}{\noindent \textbf{Odpowiedź:}\newline}    %oznaczenie początku odpowiedzi końcowej (wypisanie wyniku)
\newcommand{\odpStop}{\newline}                                             %oznaczenie końca odpowiedzi końcowej (wypisanie wyniku)
\newcommand{\testStart}{\noindent \textbf{Test:}\newline} %ewentualne możliwe opcje odpowiedzi testowej: A. ? B. ? C. ? D. ? itd.
\newcommand{\testStop}{\newline} %koniec wprowadzania odpowiedzi testowych
\newcommand{\kluczStart}{\noindent \textbf{Test poprawna odpowiedź:}\newline} %klucz, poprawna odpowiedź pytania testowego (jedna literka): A lub B lub C lub D itd.
\newcommand{\kluczStop}{\newline} %koniec poprawnej odpowiedzi pytania testowego 
\newcommand{\wstawGrafike}[2]{\begin{figure}[h] \includegraphics[scale=#2] {#1} \end{figure}} %gdyby była potrzeba wstawienia obrazka, parametry: nazwa pliku, skala (jak nie wiesz co wpisać, to wpisz 1)

\begin{document}
\maketitle


\kategoria{Wikieł/C1.1d}
\zadStart{Zadanie z Wikieł C 1.1d moja wersja nr [nrWersji]}
%[a]:[2,3,4,5,6,7,8,9,10,11,12,13,14,15,16,17,18,19,20,21,22,23,24,25,26,27,28,29,30,31,32,33,34,35,36,37,38,39,40,41,42,43,44,45,46,47,48,49,50]
%[aaa]=[a]*[a]*[a]
%[3aa]=[a]*[a]*3
%[3a]=[a]*3
%[3a2]=[3aa]*2
%[3aa2]=[3a2]/3
%[c3aa2]=int([3aa2])
%[3aa2].is_integer()==True and math.gcd([3a],2)==1
Oblicz całkę $$\int ([a]+\sqrt{x})^3dx.$$
\zadStop
\rozwStart{Aleksandra Pasińska}{}
$$\int ([a]+\sqrt{x})^3dx=\int [aaa]+[3aa]\sqrt{x}+[3a]x+x\sqrt{x}dx=$$
$$=\int [aaa]+[3aa]x^{\frac{1}{2}}+[3a]x+x^{\frac{3}{2}}dx=$$ 
$$=[aaa]\int 1dx+[3aa]\int x^\frac{1}{2}dx+[3a]\int xdx+\int x^{\frac{3}{2}}dx=$$
$$=[aaa]x+[3aa]\cdot \frac{2x^{\frac{3}{2}}}{3}+[3a]\cdot \frac{x^2}{2}+\frac{2x^{\frac{5}{2}}}{5}+C=$$
$$=[aaa]x+[c3aa2]x\sqrt{x}+\frac{[3a]x^2}{2}+\frac{2x^2\sqrt{x}}{5}+C$$
\rozwStop
\odpStart
$[aaa]x+[c3aa2]x\sqrt{x}+\frac{[3a]x^2}{2}+\frac{2x^2\sqrt{x}}{5}+C$\\
\odpStop
\testStart
A.$[aaa]x+[c3aa2]x\sqrt{x}+\frac{[3a]x^2}{2}+\frac{2x^2\sqrt{x}}{5}+C$
B.$-\frac{2x^5\sqrt{x}}{11}+C$
C.$\frac{[aaa]x\sqrt{x}}{4}-2+C$
D.$\frac{[aaa]x\sqrt{x}}{4}+C$
E.$\frac{2x^5\sqrt{x}}{11}+C$
F.$\frac{[c3aa2]x^3\sqrt{x}}{7}+\frac{2x^5\sqrt{x}}{11}+C$
G.$-\frac{[c3aa2]x^3\sqrt{x}}{7}+\frac{2x^5\sqrt{x}}{11}+C$
H.$\frac{[3aa]x\sqrt{x}}{2}+\frac{2x^5\sqrt{x}}{11}+C$
I.$\frac{[3aa]x\sqrt{x}}{2}-\frac{[aaa]x^3\sqrt{x}}{7}+C$
\testStop
\kluczStart
A
\kluczStop



\end{document}