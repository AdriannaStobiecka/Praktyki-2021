\documentclass[12pt, a4paper]{article}
\usepackage[utf8]{inputenc}
\usepackage{polski}

\usepackage{amsthm}  %pakiet do tworzenia twierdzeń itp.
\usepackage{amsmath} %pakiet do niektórych symboli matematycznych
\usepackage{amssymb} %pakiet do symboli mat., np. \nsubseteq
\usepackage{amsfonts}
\usepackage{graphicx} %obsługa plików graficznych z rozszerzeniem png, jpg
\theoremstyle{definition} %styl dla definicji
\newtheorem{zad}{} 
\title{Multizestaw zadań}
\author{Robert Fidytek}
%\date{\today}
\date{}
\newcounter{liczniksekcji}
\newcommand{\kategoria}[1]{\section{#1}} %olreślamy nazwę kateforii zadań
\newcommand{\zadStart}[1]{\begin{zad}#1\newline} %oznaczenie początku zadania
\newcommand{\zadStop}{\end{zad}}   %oznaczenie końca zadania
%Makra opcjonarne (nie muszą występować):
\newcommand{\rozwStart}[2]{\noindent \textbf{Rozwiązanie (autor #1 , recenzent #2): }\newline} %oznaczenie początku rozwiązania, opcjonarnie można wprowadzić informację o autorze rozwiązania zadania i recenzencie poprawności wykonania rozwiązania zadania
\newcommand{\rozwStop}{\newline}                                            %oznaczenie końca rozwiązania
\newcommand{\odpStart}{\noindent \textbf{Odpowiedź:}\newline}    %oznaczenie początku odpowiedzi końcowej (wypisanie wyniku)
\newcommand{\odpStop}{\newline}                                             %oznaczenie końca odpowiedzi końcowej (wypisanie wyniku)
\newcommand{\testStart}{\noindent \textbf{Test:}\newline} %ewentualne możliwe opcje odpowiedzi testowej: A. ? B. ? C. ? D. ? itd.
\newcommand{\testStop}{\newline} %koniec wprowadzania odpowiedzi testowych
\newcommand{\kluczStart}{\noindent \textbf{Test poprawna odpowiedź:}\newline} %klucz, poprawna odpowiedź pytania testowego (jedna literka): A lub B lub C lub D itd.
\newcommand{\kluczStop}{\newline} %koniec poprawnej odpowiedzi pytania testowego 
\newcommand{\wstawGrafike}[2]{\begin{figure}[h] \includegraphics[scale=#2] {#1} \end{figure}} %gdyby była potrzeba wstawienia obrazka, parametry: nazwa pliku, skala (jak nie wiesz co wpisać, to wpisz 1)

\begin{document}
\maketitle


\kategoria{Wikieł/Z1.138g}
\zadStart{Zadanie z Wikieł Z 1.138 g) moja wersja nr [nrWersji]}
%[a]:[2,3,4,5,9,10]
%[b]:[2,3,6,7,8,9,10]
%[c]:[2,3,4,5,6,7,8]
%[d]:[2,3,4,5,6,7,8,9]
%[bc]=[b]*[c]
%math.gcd([a],[c])==1 and math.gcd([b],[d])==1
Wyznaczyć funckję odwrotną do danej funkcji $f$ określonej na zbiorze $\mathcal{D}_{f}$.\\
g) $f(x)=\frac{[a]e^{x}-[b]e^{-x}}{[c]e^{x}+[d]e^{-x}}\hspace{5mm}\mathcal{D}_{f}=\mathbb{R}$
\zadStop
\rozwStart{Wojciech Przybylski}{}
$$f(x)=\frac{[a]e^{x}-[b]e^{-x}}{[c]e^{x}+[d]e^{-x}}\hspace{5mm} \mathcal{D}_{f}=\mathbb{R}, f(\mathcal{D}_{f})=(-\frac{[b]}{[d]},\frac{[a]}{[c]})$$
$$y=\frac{[a]e^{x}-[b]e^{-x}}{[c]e^{x}+[d]e^{-x}}\Rightarrow [a]e^{x}-[b]e^{-x}=y\cdot([c]e^{x}+[d]e^{-x}) \Rightarrow $$
$$\Rightarrow [a]e^{x}-[c]ye^{x}-[b]e^{-x}-[d]ye^{-x}=0 \Rightarrow e^{x}([a]-[c]y)-e^{-x}([b]+[d]y)=0 \Rightarrow$$
$$\Rightarrow e^{2x}=\frac{[b]+[d]y}{[a]-[c]y} \Rightarrow e^{x}=\sqrt{\frac{[b]+[d]y}{[a]-[c]y}} \Rightarrow x=ln(\frac{\sqrt{[b]+[d]y}}{\sqrt{[a]-[c]y}})$$
$$y=f^{-1}(x)=ln(\frac{\sqrt{[b]+[d]x}}{\sqrt{[a]-[c]x}}) \mbox{ dla } x\in (-\frac{[b]}{[d]},\frac{[a]}{[c]})$$
\rozwStop
\odpStart
Funkcja odwrotna jest postaci $y=ln(\frac{\sqrt{[b]+[d]x}}{\sqrt{[a]-[c]x}})  \mbox{ dla }x\in (-\frac{[b]}{[d]},\frac{[a]}{[c]})$
\odpStop
\testStart
A. Funkcja odwrotna jest postaci $y=ln(\frac{\sqrt{[b]+[d]x}}{\sqrt{[a]-[c]x}})   \mbox{ dla }x\in(-\frac{[b]}{[d]},\frac{[a]}{[c]})$\\
B. Funkcja odwrotna jest postaci $y=\frac{\sqrt{[b]+[d]x}}{\sqrt{[a]-[c]x}}  \mbox{ dla }x\in(-\frac{[b]}{[d]},\frac{[a]}{[c]})$\\
C. Funkcja odwrotna jest postaci $y=\sqrt{[b]+[d]x} \mbox{ dla }x\in(-\infty,\frac{[a]}{[c]})$\\
D. Funkcja odwrotna jest postaci $y=ln(\frac{[b]+[d]x}{[a]-[c]x})  \mbox{ dla }x\in[0,\infty)$\\
E. Funkcja odwrotna jest postaci $y=ln(\frac{[b]+[d]x}{[a]-[c]x})  \mbox{ dla }x\in \mathbb{R}$\\
F. Funkcja odwrotna jest postaci $y=[a]-[c]x\mbox{ dla }x\in[\frac{[a]}{[c]},\infty)$
\testStop
\kluczStart
A
\kluczStop



\end{document}