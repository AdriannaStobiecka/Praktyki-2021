\documentclass[12pt, a4paper]{article}
\usepackage[utf8]{inputenc}
\usepackage{polski}

\usepackage{amsthm}  %pakiet do tworzenia twierdzeń itp.
\usepackage{amsmath} %pakiet do niektórych symboli matematycznych
\usepackage{amssymb} %pakiet do symboli mat., np. \nsubseteq
\usepackage{amsfonts}
\usepackage{graphicx} %obsługa plików graficznych z rozszerzeniem png, jpg
\theoremstyle{definition} %styl dla definicji
\newtheorem{zad}{} 
\title{Multizestaw zadań}
\author{Robert Fidytek}
%\date{\today}
\date{}
\newcounter{liczniksekcji}
\newcommand{\kategoria}[1]{\section{#1}} %olreślamy nazwę kateforii zadań
\newcommand{\zadStart}[1]{\begin{zad}#1\newline} %oznaczenie początku zadania
\newcommand{\zadStop}{\end{zad}}   %oznaczenie końca zadania
%Makra opcjonarne (nie muszą występować):
\newcommand{\rozwStart}[2]{\noindent \textbf{Rozwiązanie (autor #1 , recenzent #2): }\newline} %oznaczenie początku rozwiązania, opcjonarnie można wprowadzić informację o autorze rozwiązania zadania i recenzencie poprawności wykonania rozwiązania zadania
\newcommand{\rozwStop}{\newline}                                            %oznaczenie końca rozwiązania
\newcommand{\odpStart}{\noindent \textbf{Odpowiedź:}\newline}    %oznaczenie początku odpowiedzi końcowej (wypisanie wyniku)
\newcommand{\odpStop}{\newline}                                             %oznaczenie końca odpowiedzi końcowej (wypisanie wyniku)
\newcommand{\testStart}{\noindent \textbf{Test:}\newline} %ewentualne możliwe opcje odpowiedzi testowej: A. ? B. ? C. ? D. ? itd.
\newcommand{\testStop}{\newline} %koniec wprowadzania odpowiedzi testowych
\newcommand{\kluczStart}{\noindent \textbf{Test poprawna odpowiedź:}\newline} %klucz, poprawna odpowiedź pytania testowego (jedna literka): A lub B lub C lub D itd.
\newcommand{\kluczStop}{\newline} %koniec poprawnej odpowiedzi pytania testowego 
\newcommand{\wstawGrafike}[2]{\begin{figure}[h] \includegraphics[scale=#2] {#1} \end{figure}} %gdyby była potrzeba wstawienia obrazka, parametry: nazwa pliku, skala (jak nie wiesz co wpisać, to wpisz 1)

\begin{document}
\maketitle


\kategoria{Wikieł/Z1.151a}
\zadStart{Zadanie z Wikieł Z 1.151 a) moja wersja nr [nrWersji]}
%[a]:[2,3,4,5,6]
%[b]:[2,3,4,5,6]
%[a2]=[a]*[a]
%[a4]=[a]**(4)
%[b2]=[b]*[b]
%[c]=[a2]+1
%[d]=[a]*[b]+1
%[d1]=[a]*[b]
%math.gcd([a],[b])==1
Rozwiązać równanie: $\sqrt[x]{[a2]}+[c]\sqrt[x]{[b2]}=[d]\cdot[a4]^{\frac{1}{2x}}+(\frac{1}{[b]})^{-\frac{2}{x}}$
\zadStop
\rozwStart{Wojciech Przybylski}{}
$$\sqrt[x]{[a2]}+[c]\sqrt[x]{[b2]}=[d]\cdot[a4]^{\frac{1}{2x}}+(\frac{1}{[b]})^{-\frac{2}{x}}$$
$${[a2]}^{\frac{1}{x}}+[c]\cdot{[b2]}^{\frac{1}{x}}=[d]\cdot[a]^{\frac{2}{x}}+[b]^{\frac{2}{x}}$$
$$[a]^{\frac{2}{x}}+[c]\cdot{[b]}^{\frac{2}{x}}=[d]\cdot[a]^{\frac{2}{x}}+[b]^{\frac{2}{x}}$$
$$[a2]\cdot{[b]}^{\frac{2}{x}}=[d1]\cdot[a]^{\frac{2}{x}}\Rightarrow[a2]\cdot{[b]}^{\frac{2}{x}}=[a]\cdot[b]\cdot[a]^{\frac{2}{x}}$$
$$[b]^{\frac{2}{x}-1}=[a]^{\frac{2}{x}-1}$$
$$(\frac{[b]}{[a]})^{\frac{2}{x}-1}=1 \Rightarrow (\frac{[b]}{[a]})^{\frac{2}{x}-1}=(\frac{[b]}{[a]})^{0}$$
$$\frac{2}{x}-1=0 \Rightarrow x=2$$
\rozwStop
\odpStart
$x=2$
\odpStop
\testStart
A. $x=2$
B. $x=1$
C. $x=0$
D. $x=-2$
E. $x=-1$
F. $x=3$
\testStop
\kluczStart
A
\kluczStop



\end{document}