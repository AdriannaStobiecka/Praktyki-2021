\documentclass[12pt, a4paper]{article}
\usepackage[utf8]{inputenc}
\usepackage{polski}

\usepackage{amsthm}  %pakiet do tworzenia twierdzeń itp.
\usepackage{amsmath} %pakiet do niektórych symboli matematycznych
\usepackage{amssymb} %pakiet do symboli mat., np. \nsubseteq
\usepackage{amsfonts}
\usepackage{graphicx} %obsługa plików graficznych z rozszerzeniem png, jpg
\theoremstyle{definition} %styl dla definicji
\newtheorem{zad}{} 
\title{Multizestaw zadań}
\author{Robert Fidytek}
%\date{\today}
\date{}
\newcounter{liczniksekcji}
\newcommand{\kategoria}[1]{\section{#1}} %olreślamy nazwę kateforii zadań
\newcommand{\zadStart}[1]{\begin{zad}#1\newline} %oznaczenie początku zadania
\newcommand{\zadStop}{\end{zad}}   %oznaczenie końca zadania
%Makra opcjonarne (nie muszą występować):
\newcommand{\rozwStart}[2]{\noindent \textbf{Rozwiązanie (autor #1 , recenzent #2): }\newline} %oznaczenie początku rozwiązania, opcjonarnie można wprowadzić informację o autorze rozwiązania zadania i recenzencie poprawności wykonania rozwiązania zadania
\newcommand{\rozwStop}{\newline}                                            %oznaczenie końca rozwiązania
\newcommand{\odpStart}{\noindent \textbf{Odpowiedź:}\newline}    %oznaczenie początku odpowiedzi końcowej (wypisanie wyniku)
\newcommand{\odpStop}{\newline}                                             %oznaczenie końca odpowiedzi końcowej (wypisanie wyniku)
\newcommand{\testStart}{\noindent \textbf{Test:}\newline} %ewentualne możliwe opcje odpowiedzi testowej: A. ? B. ? C. ? D. ? itd.
\newcommand{\testStop}{\newline} %koniec wprowadzania odpowiedzi testowych
\newcommand{\kluczStart}{\noindent \textbf{Test poprawna odpowiedź:}\newline} %klucz, poprawna odpowiedź pytania testowego (jedna literka): A lub B lub C lub D itd.
\newcommand{\kluczStop}{\newline} %koniec poprawnej odpowiedzi pytania testowego 
\newcommand{\wstawGrafike}[2]{\begin{figure}[h] \includegraphics[scale=#2] {#1} \end{figure}} %gdyby była potrzeba wstawienia obrazka, parametry: nazwa pliku, skala (jak nie wiesz co wpisać, to wpisz 1)

\begin{document}
\maketitle


\kategoria{Wikieł/Z1.79m}
\zadStart{Zadanie z Wikieł Z 1.79 m) moja wersja nr 1}
%[a]:[1, 2, 3, 4, 5, 6]
%[b]:[1, 2, 3, 4, 5, 6]
%[c]:[1, 2, 3, 4, 5, 6]
%[d]:[1, 2, 3, 4, 5, 6]
%[a]=random.randint(2,20)
%[b]=random.randint(2,20)
%[c]=random.randint(2,20)
%[d]=random.randint(2,20)
%[c2]=[c]*[c]
%[d2]=[d]*[d]
%[2cd]=2*[c]*[d]
%[-1-c]=-1-[c2]
%[ba2cd]=[b]+[a]-[2cd]
%[ab]=[a]*[b]
%[resz]=[ab]+[d2]
%[ba2cd2]=[ba2cd]*[ba2cd]
%[-resz]=(-1)*[resz]
%[delta]=[ba2cd2]-4*[-1-c]*[-resz]
%[b]<[a] and [delta]<0 and [-1-c]<0
Rozwiązać równanie. $\sqrt{(x-[a])([b]-x)}<[c]x+[d]$
\zadStop
\rozwStart{Jakub Ulrych}{}
założenie: $$(x-[a])([b]-x)\geq0$$
dziedzina:
$$x\in[[b],[a]]$$
rozwiązanie:\\podnosimy obustronnie do kwadratu
$$(x-[a])([b]-x)<[c2]x^{2}+[2cd]x+[d2]$$
$$[-1-c]x^{2}+([ba2cd])x-[resz]<0$$
$$\Delta<0\Rightarrow x\in\mathbb{R}$$
$$x\in\mathbb{R} \land \text{dziedzina}$$
$$x\in[[b],[a]]$$
\rozwStop
\odpStart
$$x\in[[b],[a]]$$
\odpStop
\testStart
A.$x\in[[b],[a]]$
B.$x\in\mathbb{R}$
C.$x\in[0,[a]]$
D.$x\in[0,1]$
\testStop
\kluczStart
A
\kluczStop



\end{document}