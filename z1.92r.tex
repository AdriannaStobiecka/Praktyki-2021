\documentclass[12pt, a4paper]{article}
\usepackage[utf8]{inputenc}
\usepackage{polski}

\usepackage{amsthm}  %pakiet do tworzenia twierdzeń itp.
\usepackage{amsmath} %pakiet do niektórych symboli matematycznych
\usepackage{amssymb} %pakiet do symboli mat., np. \nsubseteq
\usepackage{amsfonts}
\usepackage{graphicx} %obsługa plików graficznych z rozszerzeniem png, jpg
\theoremstyle{definition} %styl dla definicji
\newtheorem{zad}{} 
\title{Multizestaw zadań}
\author{Robert Fidytek}
%\date{\today}
\date{}
\newcounter{liczniksekcji}
\newcommand{\kategoria}[1]{\section{#1}} %olreślamy nazwę kateforii zadań
\newcommand{\zadStart}[1]{\begin{zad}#1\newline} %oznaczenie początku zadania
\newcommand{\zadStop}{\end{zad}}   %oznaczenie końca zadania
%Makra opcjonarne (nie muszą występować):
\newcommand{\rozwStart}[2]{\noindent \textbf{Rozwiązanie (autor #1 , recenzent #2): }\newline} %oznaczenie początku rozwiązania, opcjonarnie można wprowadzić informację o autorze rozwiązania zadania i recenzencie poprawności wykonania rozwiązania zadania
\newcommand{\rozwStop}{\newline}                                            %oznaczenie końca rozwiązania
\newcommand{\odpStart}{\noindent \textbf{Odpowiedź:}\newline}    %oznaczenie początku odpowiedzi końcowej (wypisanie wyniku)
\newcommand{\odpStop}{\newline}                                             %oznaczenie końca odpowiedzi końcowej (wypisanie wyniku)
\newcommand{\testStart}{\noindent \textbf{Test:}\newline} %ewentualne możliwe opcje odpowiedzi testowej: A. ? B. ? C. ? D. ? itd.
\newcommand{\testStop}{\newline} %koniec wprowadzania odpowiedzi testowych
\newcommand{\kluczStart}{\noindent \textbf{Test poprawna odpowiedź:}\newline} %klucz, poprawna odpowiedź pytania testowego (jedna literka): A lub B lub C lub D itd.
\newcommand{\kluczStop}{\newline} %koniec poprawnej odpowiedzi pytania testowego 
\newcommand{\wstawGrafike}[2]{\begin{figure}[h] \includegraphics[scale=#2] {#1} \end{figure}} %gdyby była potrzeba wstawienia obrazka, parametry: nazwa pliku, skala (jak nie wiesz co wpisać, to wpisz 1)

\begin{document}
\maketitle


\kategoria{Wikieł/Z1.92r}
\zadStart{Zadanie z Wikieł Z 1.92 r) moja wersja nr [nrWersji]}
%[a]:[4,9]
%[a1]=int(math.sqrt([a]))
%[b]=2
%[b1]=pow([b],[a1])
%[b2]=int([b1]/2)
%[c]:[2,3,4]
%[c1]=pow([c],[b2])
%[d]:[2,3,4,5,6,7,8,9]
%[e]:[2,3,4,5,6,7,8,9]
%[f]:[11,12,13,14,15,16,17,18,19]
%[f1]=[f]-[c1]
%[delta]= ([e]*[e])-(4*[f1]*[d])
%[delta2]=abs([delta])
%[delta1]= math.sqrt([delta2])
%[delta3]=int([delta1])
%[l1]=-[e]-[delta3]
%[l2]=-[e]+[delta3]
%[m]=2*[d]
%[dziel1]=math.gcd([l1],[m])
%[dziel2]=math.gcd([l2],[m])
%[ll1]=int([l1]/[dziel1])
%[ll2]=int([l2]/[dziel2])
%[m1]=int([m]/[dziel1])
%[m2]=int([m]/[dziel2])
%[a]!=0 and [delta]>0 and [delta1].is_integer()==True and [m1]!=1 and [m2]!=1 and [f1]>0
Rozwiązać równanie: 
$ \log_{\frac{1}{[a]}}\left(\log_{[b]} \left( \log_{\sqrt{[c]}} \left([d]x^2+[e]x+[f] \right) \right) \right)=-\frac{1}{2}$
\zadStop
\rozwStart{Joanna Świerzbin}{}
$$ \log_{\frac{1}{[a]}}\left(\log_{[b]} \left( \log_{\sqrt{[c]}} \left([d]x^2+[e]x+[f] \right) \right) \right)=-\frac{1}{2}$$
$$ \log_{[b]} \left( \log_{\sqrt{[c]}} \left([d]x^2+[e]x+[f] \right) \right)=\sqrt{[a]}$$ 
$$ \log_{[b]} \left( \log_{\sqrt{[c]}} \left([d]x^2+[e]x+[f] \right) \right)=[a1]$$
$$ \log_{\sqrt{[c]}} \left([d]x^2+[e]x+[f] \right)=[b]^{[a1]}$$ 
$$ \log_{\sqrt{[c]}} \left([d]x^2+[e]x+[f] \right)=[b1]$$ 
$$ [d]x^2+[e]x+[f] =\left(\sqrt{[c]}\right)^{[b1]}$$ 
$$ [d]x^2+[e]x+[f] =[c]^{[b2]}$$ 
$$ [d]x^2+[e]x+[f] =[c1]$$
$$ [d]x^2+[e]x+[f1]=0$$
$$\Delta = ([e])^2-4\cdot[f1]\cdot[d]$$
$$\Delta = [delta] $$ 
$$\sqrt{\Delta} = [delta3] $$ 
$$x_1=\frac{-[e]-[delta3]}{2\cdot[d]} \ \ \ \ \ \ \ \ x_2=\frac{-[e]+[delta3]}{2\cdot[d]}$$
$$x_1=\frac{[l1]}{[m]} \ \ \ \ \ \ \ \ x_2=\frac{[l2]}{[m]}$$
$$x_1=\frac{[ll1]}{[m1]} \ \ \ \ \ \ \ \ x_2=\frac{[ll2]}{[m2]}$$
$$x \in \left\{ \frac{[ll1]}{[m1]},\frac{[ll2]}{[m2]}\right\}$$
\rozwStop
\odpStart
$x \in \left\{ \frac{[ll1]}{[m1]},\frac{[ll2]}{[m2]}\right\}$\\
\odpStop
\testStart
A. $x \in \left\{ \frac{[ll1]}{[m1]},\frac{[ll2]}{[m2]}\right\}$\\
B. $x \in \left\{ \frac{[ll1]}{[m1]},0\right\}$\\
C. $x \in \left\{ \frac{[ll1]}{[m1]},1\right\}$\\
D. $x = \frac{[ll1]}{[m1]}$\\
E. $x =\frac{[ll2]}{[m2]}$\\
F. $x =1$
\testStop
\kluczStart
A
\kluczStop



\end{document}