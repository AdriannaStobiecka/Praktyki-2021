\documentclass[12pt, a4paper]{article}
\usepackage[utf8]{inputenc}
\usepackage{polski}

\usepackage{amsthm}  %pakiet do tworzenia twierdzeń itp.
\usepackage{amsmath} %pakiet do niektórych symboli matematycznych
\usepackage{amssymb} %pakiet do symboli mat., np. \nsubseteq
\usepackage{amsfonts}
\usepackage{graphicx} %obsługa plików graficznych z rozszerzeniem png, jpg
\theoremstyle{definition} %styl dla definicji
\newtheorem{zad}{} 
\title{Multizestaw zadań}
\author{Robert Fidytek}
%\date{\today}
\date{}
\newcounter{liczniksekcji}
\newcommand{\kategoria}[1]{\section{#1}} %olreślamy nazwę kateforii zadań
\newcommand{\zadStart}[1]{\begin{zad}#1\newline} %oznaczenie początku zadania
\newcommand{\zadStop}{\end{zad}}   %oznaczenie końca zadania
%Makra opcjonarne (nie muszą występować):
\newcommand{\rozwStart}[2]{\noindent \textbf{Rozwiązanie (autor #1 , recenzent #2): }\newline} %oznaczenie początku rozwiązania, opcjonarnie można wprowadzić informację o autorze rozwiązania zadania i recenzencie poprawności wykonania rozwiązania zadania
\newcommand{\rozwStop}{\newline}                                            %oznaczenie końca rozwiązania
\newcommand{\odpStart}{\noindent \textbf{Odpowiedź:}\newline}    %oznaczenie początku odpowiedzi końcowej (wypisanie wyniku)
\newcommand{\odpStop}{\newline}                                             %oznaczenie końca odpowiedzi końcowej (wypisanie wyniku)
\newcommand{\testStart}{\noindent \textbf{Test:}\newline} %ewentualne możliwe opcje odpowiedzi testowej: A. ? B. ? C. ? D. ? itd.
\newcommand{\testStop}{\newline} %koniec wprowadzania odpowiedzi testowych
\newcommand{\kluczStart}{\noindent \textbf{Test poprawna odpowiedź:}\newline} %klucz, poprawna odpowiedź pytania testowego (jedna literka): A lub B lub C lub D itd.
\newcommand{\kluczStop}{\newline} %koniec poprawnej odpowiedzi pytania testowego 
\newcommand{\wstawGrafike}[2]{\begin{figure}[h] \includegraphics[scale=#2] {#1} \end{figure}} %gdyby była potrzeba wstawienia obrazka, parametry: nazwa pliku, skala (jak nie wiesz co wpisać, to wpisz 1)

\begin{document}
\maketitle


\kategoria{Wikieł/Z1.106a}
\zadStart{Zadanie z Wikieł Z 1.106 a) moja wersja nr [nrWersji]}
%[y]:[2,49,64,81,100,121,144,1,2,3,4,5,6,7]
%[x]:[4,9,16,25,36,49,64,81,100,121,144]
%[z]:[4,9,16,25,36,49,64,81,100,121,144]
%[a]=random.randint(2,20)
%[b]=random.randint(2,20)
%[c]=[a]+[b]
%[f]=([a]-[b])/2
%[d]=int([f])
%[e]=[c]/2
%[r]=int([e])
%[e].is_integer()==True and [d]>2 and [f].is_integer()==True and math.gcd(2,[d])==1
Rozwiązać równania.\\
 $sin([a]x)-sin([b]x)=0$
\zadStop
\rozwStart{Katarzyna Filipowicz}{}
$$
sin([a]x)-sin([b]x)=0
$$
$$
2cos\left(\frac{[a]x+[b]x}{2}\right)sin\left(\frac{[a]x-[b]x}{2}\right)=0
$$ $$
cos([r]x)sin([d]x)=0
$$
1.
 $$
[r]x=\frac{\pi}{2}+k\pi \qquad \Rightarrow \qquad x=\frac{\pi}{[c]}+\frac{k\pi}{[r]}
$$
2.
$$
[d]x=0+k\pi  \qquad \Rightarrow \qquad x=\frac{2k\pi}{[d]}
$$
\rozwStop
\odpStart
$x=\frac{\pi}{[c]}+\frac{k\pi}{[r]}  \vee x=\frac{2k\pi}{[d]}$
\odpStop
\testStart
A.$x=\frac{\pi}{[c]}+\frac{k\pi}{[r]}  \vee x=\frac{2k\pi}{[d]}$\\
B.$x=\frac{\pi}{[c]}  \vee x=0$\\
C.$x=\frac{\pi}{[c]}+\frac{k\pi}{[r]}  \vee x=0$\\
D.$x=0  \vee x=\frac{2k\pi}{[d]}$\\
E.$x=0+\frac{k\pi}{[r]}  \vee x=\frac{2k\pi}{[d]}$\\
F.$x=\frac{\pi}{[c]}-\frac{k\pi}{[r]}  \vee x=\frac{2k\pi}{[d]}$\\
G.$x=\frac{\pi}{[c]}+\frac{k\pi}{[r]} $\\
H.$x=\frac{\pi}{[c]}+\frac{k\pi}{[a]}  \vee x=\frac{2k\pi}{[d]}$\\
I.$x=\frac{\pi}{[r]}+\frac{k\pi}{[r]}  \vee x=\frac{2k\pi}{[d]}$
\testStop
\kluczStart
A
\kluczStop



\end{document}