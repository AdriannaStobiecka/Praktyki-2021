\documentclass[12pt, a4paper]{article}
\usepackage[utf8]{inputenc}
\usepackage{polski}

\usepackage{amsthm}  %pakiet do tworzenia twierdzeń itp.
\usepackage{amsmath} %pakiet do niektórych symboli matematycznych
\usepackage{amssymb} %pakiet do symboli mat., np. \nsubseteq
\usepackage{amsfonts}
\usepackage{graphicx} %obsługa plików graficznych z rozszerzeniem png, jpg
\theoremstyle{definition} %styl dla definicji
\newtheorem{zad}{} 
\title{Multizestaw zadań}
\author{Robert Fidytek}
%\date{\today}
\date{}
\newcounter{liczniksekcji}
\newcommand{\kategoria}[1]{\section{#1}} %olreślamy nazwę kateforii zadań
\newcommand{\zadStart}[1]{\begin{zad}#1\newline} %oznaczenie początku zadania
\newcommand{\zadStop}{\end{zad}}   %oznaczenie końca zadania
%Makra opcjonarne (nie muszą występować):
\newcommand{\rozwStart}[2]{\noindent \textbf{Rozwiązanie (autor #1 , recenzent #2): }\newline} %oznaczenie początku rozwiązania, opcjonarnie można wprowadzić informację o autorze rozwiązania zadania i recenzencie poprawności wykonania rozwiązania zadania
\newcommand{\rozwStop}{\newline}                                            %oznaczenie końca rozwiązania
\newcommand{\odpStart}{\noindent \textbf{Odpowiedź:}\newline}    %oznaczenie początku odpowiedzi końcowej (wypisanie wyniku)
\newcommand{\odpStop}{\newline}                                             %oznaczenie końca odpowiedzi końcowej (wypisanie wyniku)
\newcommand{\testStart}{\noindent \textbf{Test:}\newline} %ewentualne możliwe opcje odpowiedzi testowej: A. ? B. ? C. ? D. ? itd.
\newcommand{\testStop}{\newline} %koniec wprowadzania odpowiedzi testowych
\newcommand{\kluczStart}{\noindent \textbf{Test poprawna odpowiedź:}\newline} %klucz, poprawna odpowiedź pytania testowego (jedna literka): A lub B lub C lub D itd.
\newcommand{\kluczStop}{\newline} %koniec poprawnej odpowiedzi pytania testowego 
\newcommand{\wstawGrafike}[2]{\begin{figure}[h] \includegraphics[scale=#2] {#1} \end{figure}} %gdyby była potrzeba wstawienia obrazka, parametry: nazwa pliku, skala (jak nie wiesz co wpisać, to wpisz 1)

\begin{document}
\maketitle


\kategoria{Wikieł/Z5.5w}
\zadStart{Zadanie z Wikieł Z 5.5 w) moja wersja nr [nrWersji]}
%[x]:[2,3,4,5,6,7,8,9,10,11,12,13]
%[y]:[2,3,4,5,6,7,8,9,10,11,12,13]
%[z]:[2,3,4,5,6,7,8]
%[a]=random.randint(2,10)
%[b]=random.randint(2,10)
%[c]=random.randint(2,10)
%[d]=random.randint(2,10)
%[e]=random.randint(2,10)
%[f]=random.randint(2,10)
%[m]=3*[a]*[d]
%[n]=4*[a]*[e]
%[u]=5*[a]*[f]
%[p]=[d]*[b]
%[r]=2*[c]*[d]
%[s]=([b]*[f])-([c]*[e])
%[t]=5*[a]*[d]
%[w]=5*[a]*[e]
%[h]=[b]*[e]
%[l]=[b]*[f]
%[k1]=2*[a]*[d]
%[k2]=2*[d]*[b]
%[k3]=[e]*[a]
%[k4]=[e]*[c]
%[s]>0
Korzystając z podstawowych twierdzeń i wzorów, wyznaczyć pochodną funkcji (bez określania zakresu zmienności $x$).\\ 
$f(x)=\frac{[a]x^5-[b]x+[c]}{[d]x^2+[e]x-[f]}$.
\zadStop
\rozwStart{Katarzyna Filipowicz}{}
$$f(x)=\frac{[a]x^5-[b]x+[c]}{[d]x^2+[e]x-[f]}$$
$$f'(x)=\left(\frac{[a]x^5-[b]x+[c]}{[d]x^2+[e]x-[f]}\right)' = $$
$$ =\frac{(5\cdot [a]\cdot x^4-[b])([d]\cdot x^2+[e]\cdot x-[f])-([a]\cdot x^5-[b]\cdot x+[c])(2\cdot [d]\cdot x+[e])}{([d]\cdot x^2+[e]\cdot x-[f])^2}=
$$ $$
=\frac{5\cdot [a]\cdot [d]\cdot x^6+5\cdot [a]\cdot [e]\cdot x^5-5\cdot [a]\cdot [f]\cdot x^4-[b]\cdot [d]\cdot x^2-[b]\cdot [e]\cdot x+[b]\cdot [f]}{([d]\cdot x^2+[e]\cdot x-[f])^2}-
$$ $$
-\frac{([a]\cdot 2\cdot [d]\cdot x^6-2\cdot[d]\cdot [b]\cdot x^2+[c]\cdot 2\cdot [d]\cdot x+[e]\cdot [a]\cdot x^5-[e]\cdot [b]\cdot x+[e]\cdot [c])}{([d]\cdot x^2+[e]\cdot x-[f])^2}=
$$ $$
=\frac{[t] x^6+[w] x^5-[u] x^4-[p] x^2-[h] x+[l]-[k1] x^6+[k2] x^2-[r] x-[k3] x^5+[h] x-[k4]}{([d]x^2+[e]x-[f])^2}=
$$ $$
=\frac{[m]x^6+[n]x^5-[u]x^4+[p]x^2-[r]x+[s]}{([d]x^2+[e]x-[f])^2}
$$
\rozwStop
\odpStart
$ f'(x)=\frac{[m]x^6+[n]x^5-[u]x^4+[p]x^2-[r]x+[s]}{([d]x^2+[e]x-[f])^2}$
\odpStop
\testStart
A. $ f'(x)=\frac{[m]x^6+[n]x^5-[u]x^4+[p]x^2-[r]x+[s]}{([d]x^2+[e]x-[f])^2}$\\
B. $ f'(x)=\frac{[m]x^6+[n]x^5-[u]x^4+[p]x^2-[r]x+[s]}{[d]x^2+[e]x-[f]}$\\
C. $ f'(x)=\frac{[a]x^6+[n]x^5-[u]x^4+[p]x^2-[r]x+[s]}{([d]x^2+[e]x-[f])^2}$ \\
D. $ f'(x)=\frac{[m]x^6+[n]x^5-[u]x^4+[p]x^2-[r]x-[s]}{([d]x^2+[e]x-[f])^2}$\\
E. $ f'(x)=\frac{[m]x^6+[n]x^5-[u]x^4+[k1]x^2-[r]x+[s]}{([d]x^2+[e]x-[f])^2}$\\
F. $ f'(x)=\frac{[m]x^6+[k3]x^5-[u]x^4+[p]x^2-[r]x+[s]}{([d]x^2+[e]x-[f])^2}$
\testStop
\kluczStart
A
\kluczStop



\end{document}