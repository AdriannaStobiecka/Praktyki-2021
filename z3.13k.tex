\documentclass[12pt, a4paper]{article}
\usepackage[utf8]{inputenc}
\usepackage{polski}

\usepackage{amsthm}  %pakiet do tworzenia twierdzeń itp.
\usepackage{amsmath} %pakiet do niektórych symboli matematycznych
\usepackage{amssymb} %pakiet do symboli mat., np. \nsubseteq
\usepackage{amsfonts}
\usepackage{graphicx} %obsługa plików graficznych z rozszerzeniem png, jpg
\theoremstyle{definition} %styl dla definicji
\newtheorem{zad}{} 
\title{Multizestaw zadań}
\author{Robert Fidytek}
%\date{\today}
\date{}
\newcounter{liczniksekcji}
\newcommand{\kategoria}[1]{\section{#1}} %olreślamy nazwę kateforii zadań
\newcommand{\zadStart}[1]{\begin{zad}#1\newline} %oznaczenie początku zadania
\newcommand{\zadStop}{\end{zad}}   %oznaczenie końca zadania
%Makra opcjonarne (nie muszą występować):
\newcommand{\rozwStart}[2]{\noindent \textbf{Rozwiązanie (autor #1 , recenzent #2): }\newline} %oznaczenie początku rozwiązania, opcjonarnie można wprowadzić informację o autorze rozwiązania zadania i recenzencie poprawności wykonania rozwiązania zadania
\newcommand{\rozwStop}{\newline}                                            %oznaczenie końca rozwiązania
\newcommand{\odpStart}{\noindent \textbf{Odpowiedź:}\newline}    %oznaczenie początku odpowiedzi końcowej (wypisanie wyniku)
\newcommand{\odpStop}{\newline}                                             %oznaczenie końca odpowiedzi końcowej (wypisanie wyniku)
\newcommand{\testStart}{\noindent \textbf{Test:}\newline} %ewentualne możliwe opcje odpowiedzi testowej: A. ? B. ? C. ? D. ? itd.
\newcommand{\testStop}{\newline} %koniec wprowadzania odpowiedzi testowych
\newcommand{\kluczStart}{\noindent \textbf{Test poprawna odpowiedź:}\newline} %klucz, poprawna odpowiedź pytania testowego (jedna literka): A lub B lub C lub D itd.
\newcommand{\kluczStop}{\newline} %koniec poprawnej odpowiedzi pytania testowego 
\newcommand{\wstawGrafike}[2]{\begin{figure}[h] \includegraphics[scale=#2] {#1} \end{figure}} %gdyby była potrzeba wstawienia obrazka, parametry: nazwa pliku, skala (jak nie wiesz co wpisać, to wpisz 1)

\begin{document}
\maketitle


\kategoria{Wikieł/Z3.13k}
\zadStart{Zadanie z Wikieł Z 3.13 k) moja wersja nr [nrWersji]}
%[a]:[2,3,4,5,6,7,8,9]
%[b]:[2,3,4,5,6,7,8,9]
%[c]:[2,3,4,5,6,7,8,9]
%[a2]=[a]**2
%[aa]=2*[a]
%[d]=math.gcd([b],[aa])
%[bd]=int([b]/[d])
%[aad]=int([aa]/[d])
%[b]<[aa]
Obliczyć granicę ciągu 
$$a_n=(\sqrt{[a2]n^2+[b]n+[c]}-[a]n).$$
\zadStop
\rozwStart{Adrianna Stobiecka}{}
$$\lim_{n\to\infty}(\sqrt{[a2]n^2+[b]n+[c]}-[a]n)$$
$$=\lim_{n\to\infty}\frac{(\sqrt{[a2]n^2+[b]n+[c]}-[a]n)(\sqrt{[a2]n^2+[b]n+[c]}+[a]n)}{\sqrt{[a2]n^2+[b]n+[c]}+[a]n}$$
$$=\lim_{n\to\infty}\frac{[a2]n^2+[b]n+[c]-[a2]n^2}{\sqrt{[a2]n^2+[b]n+[c]}+[a]n}=\lim_{n\to\infty}\frac{[b]n+[c]}{\sqrt{[a2]n^2+[b]n+[c]}+[a]n}$$
$$=\lim_{n\to\infty}\frac{n([b]+\frac{[c]}{n})}{n\sqrt{[a2]+\frac{[b]}{n}+\frac{[c]}{n^2}}+n\cdot[a]}=\lim_{n\to\infty}\frac{[b]+\frac{[c]}{n}}{\sqrt{[a2]+\frac{[b]}{n}+\frac{[c]}{n^2}}+[a]}$$
$$=\frac{[b]}{\sqrt{[a2]}+[a]}=\frac{[b]}{[a]+[a]}=\frac{[bd]}{[aad]}$$
\rozwStop
\odpStart
$\frac{[bd]}{[aad]}$
\odpStop
\testStart
A.$\infty$
B.$0$
C.$\frac{[bd]}{[aad]}$
D.$[a]$
E.$1$
F.$-\infty$
G.$-\frac{[bd]}{[aad]}$
H.$-1$
I.$-[a]$
\testStop
\kluczStart
C
\kluczStop



\end{document}
