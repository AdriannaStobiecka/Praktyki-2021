\documentclass[12pt, a4paper]{article}
\usepackage[utf8]{inputenc}
\usepackage{polski}

\usepackage{amsthm}  %pakiet do tworzenia twierdzeń itp.
\usepackage{amsmath} %pakiet do niektórych symboli matematycznych
\usepackage{amssymb} %pakiet do symboli mat., np. \nsubseteq
\usepackage{amsfonts}
\usepackage{graphicx} %obsługa plików graficznych z rozszerzeniem png, jpg
\theoremstyle{definition} %styl dla definicji
\newtheorem{zad}{} 
\title{Multizestaw zadań}
\author{Robert Fidytek}
%\date{\today}
\date{}
\newcounter{liczniksekcji}
\newcommand{\kategoria}[1]{\section{#1}} %olreślamy nazwę kateforii zadań
\newcommand{\zadStart}[1]{\begin{zad}#1\newline} %oznaczenie początku zadania
\newcommand{\zadStop}{\end{zad}}   %oznaczenie końca zadania
%Makra opcjonarne (nie muszą występować):
\newcommand{\rozwStart}[2]{\noindent \textbf{Rozwiązanie (autor #1 , recenzent #2): }\newline} %oznaczenie początku rozwiązania, opcjonarnie można wprowadzić informację o autorze rozwiązania zadania i recenzencie poprawności wykonania rozwiązania zadania
\newcommand{\rozwStop}{\newline}                                            %oznaczenie końca rozwiązania
\newcommand{\odpStart}{\noindent \textbf{Odpowiedź:}\newline}    %oznaczenie początku odpowiedzi końcowej (wypisanie wyniku)
\newcommand{\odpStop}{\newline}                                             %oznaczenie końca odpowiedzi końcowej (wypisanie wyniku)
\newcommand{\testStart}{\noindent \textbf{Test:}\newline} %ewentualne możliwe opcje odpowiedzi testowej: A. ? B. ? C. ? D. ? itd.
\newcommand{\testStop}{\newline} %koniec wprowadzania odpowiedzi testowych
\newcommand{\kluczStart}{\noindent \textbf{Test poprawna odpowiedź:}\newline} %klucz, poprawna odpowiedź pytania testowego (jedna literka): A lub B lub C lub D itd.
\newcommand{\kluczStop}{\newline} %koniec poprawnej odpowiedzi pytania testowego 
\newcommand{\wstawGrafike}[2]{\begin{figure}[h] \includegraphics[scale=#2] {#1} \end{figure}} %gdyby była potrzeba wstawienia obrazka, parametry: nazwa pliku, skala (jak nie wiesz co wpisać, to wpisz 1)

\begin{document}
\maketitle


\kategoria{Wikieł/Z3.22}
\zadStart{Zadanie z Wikieł Z 3.22 moja wersja nr [nrWersji]}
%[a]:[2,3,4,5,6,7,8,9,10,11,12,13,14,15]
%[a2]=[a]**2
%[a3]=[a]**3
%[a4]=[a]**4
%
Rozwiązać nierówność $[a2]x^2+[a3]x^3+[a4]x^4+\cdots>-1-[a]x$, której lewa strona jest sumą nieskończonego ciągu geometrycznego.
\zadStop
\rozwStart{Adrianna Stobiecka}{}
Powyższą nierówność możemy przekształcić do następującej postaci:
$$1+[a]x+[a2]x^2+[a3]x^3+[a4]x^4+\cdots>0$$
Zauważmy, że lewa strona nierówności to suma nieskończonego ciągu geometrycznego o pierwszym wyrazie $a_1=1$ oraz ilorazie $q=[a]x$. Zatem:
$$|q|<1\Leftrightarrow|[a]x|<1\Leftrightarrow[a]x<1\land[a]x>-1\Leftrightarrow x<\frac{1}{[a]}\land x>-\frac{1}{[a]}\Leftrightarrow x\in\bigg(-\frac{1}{[a]},\frac{1}{[a]}\bigg)$$
Otrzymujemy założenie $x\in(-\frac{1}{[a]},\frac{1}{[a]})$.
\\Przechodzimy do obliczenia sumy.
$$S=\frac{a_1}{1-q}=\frac{1}{1-[a]x}$$
Otrzymaną wynik wstawiamy do nierówności.
$$\frac{1}{1-[a]x}>0\Leftrightarrow 1-[a]x>0\Leftrightarrow[a]x<1\Leftrightarrow x<\frac{1}{[a]}\Leftrightarrow x\in\bigg(-\infty,\frac{1}{[a]}\bigg)$$
Mamy więc:
$$x\in\bigg(-\frac{1}{[a]},\frac{1}{[a]}\bigg)\qquad\land\qquad x\in\bigg(-\infty,\frac{1}{[a]}\bigg)$$
Ostatecznie otrzymujemy $x\in(-\frac{1}{[a]},\frac{1}{[a]})$.
\odpStart
 $x\in(-\frac{1}{[a]},\frac{1}{[a]})$
\odpStop
\testStart
A.$x\in[-\frac{1}{[a]},\frac{1}{[a]}]$
B.$x\in(-[a],[a])$
C.$x\in(-\frac{1}{[a]},\frac{1}{[a]})$
D.$x\in[-[a],[a]]$
E.$x\in\mathbb{R}$
F.$x\in[-\frac{1}{[a]},\frac{1}{[a]})$
G.$x\in(-\frac{1}{[a]},\frac{1}{[a]}]$
H.$x\in[-[a],[a])$
I.$x\in\emptyset$
\testStop
\kluczStart
C
\kluczStop



\end{document}
