\documentclass[12pt, a4paper]{article}
\usepackage[utf8]{inputenc}
\usepackage{polski}

\usepackage{amsthm}  %pakiet do tworzenia twierdzeń itp.
\usepackage{amsmath} %pakiet do niektórych symboli matematycznych
\usepackage{amssymb} %pakiet do symboli mat., np. \nsubseteq
\usepackage{amsfonts}
\usepackage{graphicx} %obsługa plików graficznych z rozszerzeniem png, jpg
\theoremstyle{definition} %styl dla definicji
\newtheorem{zad}{} 
\title{Multizestaw zadań}
\author{Robert Fidytek}
%\date{\today}
\date{}
\newcounter{liczniksekcji}
\newcommand{\kategoria}[1]{\section{#1}} %olreślamy nazwę kateforii zadań
\newcommand{\zadStart}[1]{\begin{zad}#1\newline} %oznaczenie początku zadania
\newcommand{\zadStop}{\end{zad}}   %oznaczenie końca zadania
%Makra opcjonarne (nie muszą występować):
\newcommand{\rozwStart}[2]{\noindent \textbf{Rozwiązanie (autor #1 , recenzent #2): }\newline} %oznaczenie początku rozwiązania, opcjonarnie można wprowadzić informację o autorze rozwiązania zadania i recenzencie poprawności wykonania rozwiązania zadania
\newcommand{\rozwStop}{\newline}                                            %oznaczenie końca rozwiązania
\newcommand{\odpStart}{\noindent \textbf{Odpowiedź:}\newline}    %oznaczenie początku odpowiedzi końcowej (wypisanie wyniku)
\newcommand{\odpStop}{\newline}                                             %oznaczenie końca odpowiedzi końcowej (wypisanie wyniku)
\newcommand{\testStart}{\noindent \textbf{Test:}\newline} %ewentualne możliwe opcje odpowiedzi testowej: A. ? B. ? C. ? D. ? itd.
\newcommand{\testStop}{\newline} %koniec wprowadzania odpowiedzi testowych
\newcommand{\kluczStart}{\noindent \textbf{Test poprawna odpowiedź:}\newline} %klucz, poprawna odpowiedź pytania testowego (jedna literka): A lub B lub C lub D itd.
\newcommand{\kluczStop}{\newline} %koniec poprawnej odpowiedzi pytania testowego 
\newcommand{\wstawGrafike}[2]{\begin{figure}[h] \includegraphics[scale=#2] {#1} \end{figure}} %gdyby była potrzeba wstawienia obrazka, parametry: nazwa pliku, skala (jak nie wiesz co wpisać, to wpisz 1)

\begin{document}
\maketitle


\kategoria{Wikieł/Z1.79o}
\zadStart{Zadanie z Wikieł Z 1.79 o) moja wersja nr [nrWersji]}
%[z]:[5,6,7,8,9,10,11,12,13]
%[z1]:[1,2,3,4,5,6,7,8,9,10]
%[a]=random.randint(2,40)
%[b]=random.randint(int([a]+1),50)
%[b1]=[b]*[b]
%[b2]=2*[b]
%[b3]=[b2]-[a]
%[b4]=[b1]/[b3]
%math.gcd([b1],[b3])==1 and [b4]<[b]
Rozwiązać nierówność $\sqrt{x^2-[a]x}>[b]-x$
\zadStop
\rozwStart{Barbara Bączek}{}
Zaczniemy od wyznaczenia dziedziny.
$$D:x^2 - [a]x \geq 0$$
$$D:x(x-[a]) \geq 0$$
$$D: x \in (-\infty,0] \cup [[a], \infty)$$
Powróćmy do nierówności:
$$\sqrt{x^2-[a]x}>[b]-x$$
\begin{enumerate}
\item Niech $x \in ([b], \infty)$, wtedy nierówność wyjściowa jest tożsamościowa, bo w zbiorze liczb rzeczywistych pierwiastek kwadratowy jest nieujemny. Zatem dla $x \in ([b], \infty)$ nierówność zachodzi.
\item  Niech $x \in (-\infty, 0] \cup [[a],[b]]$, wtedy obie strony nierówności są nieujemne.
$$\sqrt{x^2-[a]x}>[b]-x$$
$$x^2 -[a]x>x^2 -[b2]x +[b1] $$
$$[b3]x>[b1]$$
$$x>\frac{[b1]}{[b3]}$$
Zatem uwzględniając założenie $x \in (-\infty, 0] \cup [[a],[b]]$ otrzymujemy, że dla $x \in (\frac{[b1]}{[b3]},[b]]$ nierówność zachodzi.
\end{enumerate}
3. Podsumowując: $x \in (\frac{[b1]}{[b3]}, \infty)$
\rozwStop
\odpStart
$x \in (\frac{[b1]}{[b3]}, \infty)$
\odpStop
\testStart
A.$x \in [-\frac{[b1]}{[b3]}, \infty)$
B.$x \in [\frac{[b1]}{[b3]}, \infty)$
C.$x \in (-\frac{[b1]}{[b3]}, \infty)$
D.$x \in (\frac{[b1]}{[b3]}, \infty)$
E.$x \in (\frac{[b1]}{[b3]}, [b])$
G.$x \in (\frac{[b3]}{[b1]}, \infty)$
H.$x \in [\frac{[b3]}{[b1]}, \infty)$
\testStop
\kluczStart
D
\kluczStop



\end{document}