\documentclass[12pt, a4paper]{article}
\usepackage[utf8]{inputenc}
\usepackage{polski}

\usepackage{amsthm}  %pakiet do tworzenia twierdzeń itp.
\usepackage{amsmath} %pakiet do niektórych symboli matematycznych
\usepackage{amssymb} %pakiet do symboli mat., np. \nsubseteq
\usepackage{amsfonts}
\usepackage{graphicx} %obsługa plików graficznych z rozszerzeniem png, jpg
\theoremstyle{definition} %styl dla definicji
\newtheorem{zad}{} 
\title{Multizestaw zadań}
\author{Robert Fidytek}
%\date{\today}
\date{}
\newcounter{liczniksekcji}
\newcommand{\kategoria}[1]{\section{#1}} %olreślamy nazwę kateforii zadań
\newcommand{\zadStart}[1]{\begin{zad}#1\newline} %oznaczenie początku zadania
\newcommand{\zadStop}{\end{zad}}   %oznaczenie końca zadania
%Makra opcjonarne (nie muszą występować):
\newcommand{\rozwStart}[2]{\noindent \textbf{Rozwiązanie (autor #1 , recenzent #2): }\newline} %oznaczenie początku rozwiązania, opcjonarnie można wprowadzić informację o autorze rozwiązania zadania i recenzencie poprawności wykonania rozwiązania zadania
\newcommand{\rozwStop}{\newline}                                            %oznaczenie końca rozwiązania
\newcommand{\odpStart}{\noindent \textbf{Odpowiedź:}\newline}    %oznaczenie początku odpowiedzi końcowej (wypisanie wyniku)
\newcommand{\odpStop}{\newline}                                             %oznaczenie końca odpowiedzi końcowej (wypisanie wyniku)
\newcommand{\testStart}{\noindent \textbf{Test:}\newline} %ewentualne możliwe opcje odpowiedzi testowej: A. ? B. ? C. ? D. ? itd.
\newcommand{\testStop}{\newline} %koniec wprowadzania odpowiedzi testowych
\newcommand{\kluczStart}{\noindent \textbf{Test poprawna odpowiedź:}\newline} %klucz, poprawna odpowiedź pytania testowego (jedna literka): A lub B lub C lub D itd.
\newcommand{\kluczStop}{\newline} %koniec poprawnej odpowiedzi pytania testowego 
\newcommand{\wstawGrafike}[2]{\begin{figure}[h] \includegraphics[scale=#2] {#1} \end{figure}} %gdyby była potrzeba wstawienia obrazka, parametry: nazwa pliku, skala (jak nie wiesz co wpisać, to wpisz 1)

\begin{document}
\maketitle


\kategoria{Wikieł/Z3.13d}
\zadStart{Zadanie z Wikieł Z 3.13 d) moja wersja nr [nrWersji]}
%[a]:[2,3,4,5,6,7,8,9]
%[b]:[2,3,4,5,6,7,8,9]
%[2a]=2*[a]
%[a]!=[b] and math.gcd([a],[b])==1 and  math.gcd([2a],[b])==1
Obliczyć granicę ciągu 
$$a_n=\frac{e^{\sqrt{[a]n+[b]}}}{e^{\sqrt{[a]n}}}.$$
\zadStop
\rozwStart{Adrianna Stobiecka}{}
$$\lim_{n\to\infty}\frac{e^{\sqrt{[a]n+[b]}}}{e^{\sqrt{[a]n}}}=\lim_{n\to\infty}e^{\sqrt{[a]n+[b]}-\sqrt{[a]n}}=(*)$$
Zapiszemy potęgę w inny sposób:
$$\sqrt{[a]n+[b]}-\sqrt{[a]n}=\frac{\big(\sqrt{[a]n+[b]}-\sqrt{[a]n}\big)\big(\sqrt{[a]n+[b]}+\sqrt{[a]n}\big)}{\sqrt{[a]n+[b]}+\sqrt{[a]n}}=(**)$$
Skorzystamy teraz w liczniku ze wzoru $(a-b)(a+b)=a^2-b^2$. Mamy więc:
$$(**)=\frac{[a]n+[b]-[a]n}{\sqrt{[a]n+[b]}+\sqrt{[a]n}}=\frac{[b]}{\sqrt{[a]n+[b]}+\sqrt{[a]n}}$$
Wrócimy teraz do $(*)$. Za $\sqrt{[a]n+[b]}-\sqrt{[a]n}$ podstawiamy wyliczone $\frac{[b]}{\sqrt{[a]n+[b]}+\sqrt{[a]n}}$. Otrzymujemy zatem:
$$(*)=\lim_{n\to\infty}e^{\frac{[b]}{\sqrt{[a]n+[b]}+\sqrt{[a]n}}}=e^{0}=1$$
\rozwStop
\odpStart
$1$
\odpStop
\testStart
A.$1$
B.$e^{\frac{[b]}{[2a]}}$
C.$[b]$
D.$e^{\frac{[b]}{[a]}}$
E.$0$
F.$e^{\frac{[a]}{[b]}}$
G.$[a]$
H.$\frac{[a]}{[b]}$
I.$\frac{[b]}{[a]}$
\testStop
\kluczStart
A
\kluczStop



\end{document}
