\documentclass[12pt, a4paper]{article}
\usepackage[utf8]{inputenc}
\usepackage{polski}

\usepackage{amsthm}  %pakiet do tworzenia twierdzeń itp.
\usepackage{amsmath} %pakiet do niektórych symboli matematycznych
\usepackage{amssymb} %pakiet do symboli mat., np. \nsubseteq
\usepackage{amsfonts}
\usepackage{graphicx} %obsługa plików graficznych z rozszerzeniem png, jpg
\theoremstyle{definition} %styl dla definicji
\newtheorem{zad}{} 
\title{Multizestaw zadań}
\author{Robert Fidytek}
%\date{\today}
\date{}
\newcounter{liczniksekcji}
\newcommand{\kategoria}[1]{\section{#1}} %olreślamy nazwę kateforii zadań
\newcommand{\zadStart}[1]{\begin{zad}#1\newline} %oznaczenie początku zadania
\newcommand{\zadStop}{\end{zad}}   %oznaczenie końca zadania
%Makra opcjonarne (nie muszą występować):
\newcommand{\rozwStart}[2]{\noindent \textbf{Rozwiązanie (autor #1 , recenzent #2): }\newline} %oznaczenie początku rozwiązania, opcjonarnie można wprowadzić informację o autorze rozwiązania zadania i recenzencie poprawności wykonania rozwiązania zadania
\newcommand{\rozwStop}{\newline}                                            %oznaczenie końca rozwiązania
\newcommand{\odpStart}{\noindent \textbf{Odpowiedź:}\newline}    %oznaczenie początku odpowiedzi końcowej (wypisanie wyniku)
\newcommand{\odpStop}{\newline}                                             %oznaczenie końca odpowiedzi końcowej (wypisanie wyniku)
\newcommand{\testStart}{\noindent \textbf{Test:}\newline} %ewentualne możliwe opcje odpowiedzi testowej: A. ? B. ? C. ? D. ? itd.
\newcommand{\testStop}{\newline} %koniec wprowadzania odpowiedzi testowych
\newcommand{\kluczStart}{\noindent \textbf{Test poprawna odpowiedź:}\newline} %klucz, poprawna odpowiedź pytania testowego (jedna literka): A lub B lub C lub D itd.
\newcommand{\kluczStop}{\newline} %koniec poprawnej odpowiedzi pytania testowego 
\newcommand{\wstawGrafike}[2]{\begin{figure}[h] \includegraphics[scale=#2] {#1} \end{figure}} %gdyby była potrzeba wstawienia obrazka, parametry: nazwa pliku, skala (jak nie wiesz co wpisać, to wpisz 1)

\begin{document}
\maketitle


\kategoria{Wikieł/P1.5b}
\zadStart{Zadanie z Wikieł P 1.5 b) moja wersja nr [nrWersji]}
%[a]:[1,2,3,4,5,6,7,8,9]
%[b]:[2,3,4,5,6,7,8,9]
%[c]:[2,3,4,5,6,7,8,9]
%[d]:[1,2,3,4,5,6,7,8,9]
%[bcm]=-[b]-[c]
%[adm]=[d]-[a]
%[bcp]=-[b]+[c]
%[adp]=-[d]-[a]
%[bcm]!=0 and [bcp]!=0 and math.gcd([adm],[bcm])==1 and math.gcd([adp],[bcp])==1 and [c]>[d] and [adm]!=0 and [adp]!=0 and [bcm]!=1 and [bcp]!=1


Rozwiązać równanie  $|{[a]-[b]x}| = |{[c]x+[d]}|$.
\zadStop
\rozwStart{Joanna Świerzbin}{}
$$|{[a]-[b]x}| = |{[c]x+[d]}| \Leftrightarrow {[a]-[b]x} = {[c]x+[d]} \vee {[a]-[b]x} = {-[c]x-[d]}  \Leftrightarrow$$
$$\Leftrightarrow {[bcm]x} = {[adm]} \vee {[bcp]x} = {[adp]}  \Leftrightarrow$$
$$\Leftrightarrow {x} = {\frac{[adm]}{[bcm]}} \vee {x} = {\frac{[adp]}{[bcp]}} $$
Zatem $ {x} = {\frac{[adm]}{[bcm]}} $ i ${x} = {\frac{[adp]}{[bcp]}} $ jest rozwiązaniem rozważanego równania.
\rozwStop
\odpStart
$ {x} = {\frac{[adm]}{[bcm]}} $ i ${x} = {\frac{[adp]}{[bcp]}} $
\odpStop
\testStart
A.$ {x} = {\frac{[adm]}{[bcm]}} $ i ${x} = {\frac{[adp]}{[bcp]}} $\\
B.$ {x} = {\frac{[adm]}{[bcm]}} $ \\
C.$ {x} = {\frac{[adp]}{[bcp]}} $ \\
D.$ {x} = {\frac{[bcm]}{[adm]}} $ i ${x} = {\frac{[adp]}{[bcp]}} $\\
E.$ {x} = {\frac{[adm]}{[bcm]}} $ i ${x} = {\frac{[bcp]}{[adp]}} $\\
F.$ {x} = {\frac{[adm]}{[bcm]}} $ i ${x} = {\frac{[adm]}{[bcp]}} $ \\
G.$ {x} = {\frac{[adp]}{[bcm]}} $ i ${x} = {\frac{[adp]}{[bcp]}} $\\
H.$ {x} = {\frac{[adm]}{[bcp]}} $ i ${x} = {\frac{[adp]}{[bcp]}} $
\testStop
\kluczStart
A
\kluczStop



\end{document}