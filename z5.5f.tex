\documentclass[12pt, a4paper]{article}
\usepackage[utf8]{inputenc}
\usepackage{polski}

\usepackage{amsthm}  %pakiet do tworzenia twierdzeń itp.
\usepackage{amsmath} %pakiet do niektórych symboli matematycznych
\usepackage{amssymb} %pakiet do symboli mat., np. \nsubseteq
\usepackage{amsfonts}
\usepackage{graphicx} %obsługa plików graficznych z rozszerzeniem png, jpg
\theoremstyle{definition} %styl dla definicji
\newtheorem{zad}{} 
\title{Multizestaw zadań}
\author{Robert Fidytek}
%\date{\today}
\date{}
\newcounter{liczniksekcji}
\newcommand{\kategoria}[1]{\section{#1}} %olreślamy nazwę kateforii zadań
\newcommand{\zadStart}[1]{\begin{zad}#1\newline} %oznaczenie początku zadania
\newcommand{\zadStop}{\end{zad}}   %oznaczenie końca zadania
%Makra opcjonarne (nie muszą występować):
\newcommand{\rozwStart}[2]{\noindent \textbf{Rozwiązanie (autor #1 , recenzent #2): }\newline} %oznaczenie początku rozwiązania, opcjonarnie można wprowadzić informację o autorze rozwiązania zadania i recenzencie poprawności wykonania rozwiązania zadania
\newcommand{\rozwStop}{\newline}                                            %oznaczenie końca rozwiązania
\newcommand{\odpStart}{\noindent \textbf{Odpowiedź:}\newline}    %oznaczenie początku odpowiedzi końcowej (wypisanie wyniku)
\newcommand{\odpStop}{\newline}                                             %oznaczenie końca odpowiedzi końcowej (wypisanie wyniku)
\newcommand{\testStart}{\noindent \textbf{Test:}\newline} %ewentualne możliwe opcje odpowiedzi testowej: A. ? B. ? C. ? D. ? itd.
\newcommand{\testStop}{\newline} %koniec wprowadzania odpowiedzi testowych
\newcommand{\kluczStart}{\noindent \textbf{Test poprawna odpowiedź:}\newline} %klucz, poprawna odpowiedź pytania testowego (jedna literka): A lub B lub C lub D itd.
\newcommand{\kluczStop}{\newline} %koniec poprawnej odpowiedzi pytania testowego 
\newcommand{\wstawGrafike}[2]{\begin{figure}[h] \includegraphics[scale=#2] {#1} \end{figure}} %gdyby była potrzeba wstawienia obrazka, parametry: nazwa pliku, skala (jak nie wiesz co wpisać, to wpisz 1)

\begin{document}
\maketitle


\kategoria{Wikieł/Z5.5f}
\zadStart{Zadanie z Wikieł Z 5.5 f) moja wersja nr [nrWersji]}
%[a]:[2,3,4,5,6,7,8,9]
%[a1]=2*[a]
%[b]:[2,3,4,5,6,7,8,9]
%[b1]=2*[b]
%[ab]=[a1]+[b1]
%[ab1]=[a]+[b]
%[ab2]=3*[ab1]
%[c]:[2,3,4,5,6,7,8,9]
%[c1]=[c]*[c]
%[c2]=2*[c]
%math.gcd([ab2],[c2])==1 and [ab2]!=0
Wyznacz pochodną funkcji \\ $f(x)=\frac{[a]x^2+[b]x\sqrt{x^2}}{[c]\sqrt{x}}$.
\zadStop
\rozwStart{Joanna Świerzbin}{}
$$f(x)=\frac{[a]x^2+[b]x\sqrt{x^2}}{[c]\sqrt{x}}$$
$$f'(x)=\left(\frac{[a]x^2+[b]x\sqrt{x^2}}{[c]\sqrt{x}}\right)' = $$
$$ = \frac{\left([a]x^2+[b]x\sqrt{x^2}\right)'([c]\sqrt{x})- \left([a]x^2+[b]x\sqrt{x^2}\right) ([c]\sqrt{x})'}{([c]\sqrt{x})^2} = $$
$$  = \frac{\left(2\cdot[a]x+2\cdot[b]x\right)([c]\sqrt{x})- \left([a]x^2+[b]x\sqrt{x^2}\right) (\frac{[c]}{2\sqrt{x}})}{[c]\cdot[c]x} =  $$
$$  = \frac{\left([a1]x+[b1]x\right)([c]\sqrt{x})- \left([a]x^2+[b]x\sqrt{x^2}\right) (\frac{[c]}{2\sqrt{x}})}{[c1]x} =  $$
$$  = \frac{\left([ab]x\right)([c]\sqrt{x})- \left([ab1]x^2\right) (\frac{[c]}{2\sqrt{x}})}{[c1]x} =  $$
$$  = \frac{\left([ab]\sqrt{x}\right)}{[c]}-\frac{\left([ab1]x \frac{1}{2\sqrt{x}}\right)}{[c]} = \frac{\sqrt{x}}{[c]} \left(\frac{3\cdot[ab1]}{2}\right) =  \frac{[ab2]\sqrt{x}}{[c2]}  $$
\rozwStop
\odpStart
$ f'(x) = \frac{[ab2]\sqrt{x}}{[c2]}  $
\odpStop
\testStart
A. $ f'(x) = \frac{[ab2]\sqrt{x}}{[c2]}  $\\
B. $ f'(x) = \frac{[ab2]x}{[c2]}  $ \\
C. $ f'(x) = \frac{[ab2]\sqrt{x}}{[c]}  $ \\
D. $ f'(x) = \frac{[ab1]\sqrt{x}}{[c2]}  $\\
E. $ f'(x) = \frac{[ab2]}{[c2]}  $\\
F. $ f'(x) = \frac{[ab2]x}{[c]}  $
\testStop
\kluczStart
A
\kluczStop



\end{document}