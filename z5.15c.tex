\documentclass[12pt, a4paper]{article}
\usepackage[utf8]{inputenc}
\usepackage{polski}

\usepackage{amsthm}  %pakiet do tworzenia twierdzeń itp.
\usepackage{amsmath} %pakiet do niektórych symboli matematycznych
\usepackage{amssymb} %pakiet do symboli mat., np. \nsubseteq
\usepackage{amsfonts}
\usepackage{graphicx} %obsługa plików graficznych z rozszerzeniem png, jpg
\theoremstyle{definition} %styl dla definicji
\newtheorem{zad}{} 
\title{Multizestaw zadań}
\author{Robert Fidytek}
%\date{\today}
\date{}
\newcounter{liczniksekcji}
\newcommand{\kategoria}[1]{\section{#1}} %olreślamy nazwę kateforii zadań
\newcommand{\zadStart}[1]{\begin{zad}#1\newline} %oznaczenie początku zadania
\newcommand{\zadStop}{\end{zad}}   %oznaczenie końca zadania
%Makra opcjonarne (nie muszą występować):
\newcommand{\rozwStart}[2]{\noindent \textbf{Rozwiązanie (autor #1 , recenzent #2): }\newline} %oznaczenie początku rozwiązania, opcjonarnie można wprowadzić informację o autorze rozwiązania zadania i recenzencie poprawności wykonania rozwiązania zadania
\newcommand{\rozwStop}{\newline}                                            %oznaczenie końca rozwiązania
\newcommand{\odpStart}{\noindent \textbf{Odpowiedź:}\newline}    %oznaczenie początku odpowiedzi końcowej (wypisanie wyniku)
\newcommand{\odpStop}{\newline}                                             %oznaczenie końca odpowiedzi końcowej (wypisanie wyniku)
\newcommand{\testStart}{\noindent \textbf{Test:}\newline} %ewentualne możliwe opcje odpowiedzi testowej: A. ? B. ? C. ? D. ? itd.
\newcommand{\testStop}{\newline} %koniec wprowadzania odpowiedzi testowych
\newcommand{\kluczStart}{\noindent \textbf{Test poprawna odpowiedź:}\newline} %klucz, poprawna odpowiedź pytania testowego (jedna literka): A lub B lub C lub D itd.
\newcommand{\kluczStop}{\newline} %koniec poprawnej odpowiedzi pytania testowego 
\newcommand{\wstawGrafike}[2]{\begin{figure}[h] \includegraphics[scale=#2] {#1} \end{figure}} %gdyby była potrzeba wstawienia obrazka, parametry: nazwa pliku, skala (jak nie wiesz co wpisać, to wpisz 1)

\begin{document}
\maketitle


\kategoria{Wikieł/Z5.15 c}
\zadStart{Zadanie z Wikieł Z 5.15 c) moja wersja nr [nrWersji]}
%[a]:[2,3,4,5,6,7,8,9]
%[b]:[2,3,4,5,6,7,8,9]
%[c]=2*[a]*[b]
%[d]=[c]*[b]
%[a]!=0
Obliczyć pochodną rzędu drugiego funkcji $f(x)=\frac{[a]x^2}{x-[b]}$.
\zadStop
\rozwStart{Joanna Świerzbin}{}
$$f(x)=\frac{[a]x^2}{x-[b]}$$
$$f'(x)= \left( \frac{[a]x^2}{x-[b]} \right)' = \frac{([a]x^2)'(x-[b])-([a]x^2)(x-[b])'}{(x-[b])^2} =$$
$$= \frac{(2\cdot[a]x)(x-[b])-([a]x^2)}{(x-[b])^2} = \frac{2\cdot[a]x^2-2\cdot[a]\cdot[b]x-[a]x^2}{(x-[b])^2} = \frac{[a]x^2-[c]x}{(x-[b])^2} $$
$$f''(x)= \left( \frac{[a]x^2-[c]x}{(x-[b])^2} \right)'= \frac{([a]x^2-[c]x)'(x-[b])^2-([a]x^2-[c]x)((x-[b])^2)'}{(x-[b])^4} = $$
$$= \frac{(2\cdot[a]x-[c])(x-[b])^2-2([a]x^2-[c]x)(x-[b])}{(x-[b])^4} = $$
$$= \frac{(2\cdot[a]x-[c])(x-[b])-2([a]x^2-[c]x)}{(x-[b])^3} = $$
$$= \frac{2\cdot[a]x^2-[c]x -2\cdot[a]\cdot[b]x+[c]\cdot[b]-2\cdot[a]x^2+2\cdot[c]x)}{(x-[b])^3} = $$
$$= \frac{[c]\cdot[b]}{(x-[b])^3} = \frac{[d]}{(x-[b])^3} $$
\rozwStop
\odpStart
$f''(x) = \frac{[d]}{(x-[b])^3} $
\odpStop
\testStart
A. $f''(x) = \frac{[d]}{(x-[b])^3} $\\
B. $f''(x) = \frac{1}{(x-[b])^3} $ \\
C. $f''(x) = \frac{[d]}{(x)^3} $\\
D. $f''(x) = \frac{[d]}{(x-[b])^2} $\\
E. $f''(x) = -\frac{[d]}{(x-[b])^3} $\\
F. $f''(x) = \frac{[d]}{(x-[a])^3} $
\testStop
\kluczStart
A
\kluczStop



\end{document}