\documentclass[12pt, a4paper]{article}
\usepackage[utf8]{inputenc}
\usepackage{polski}

\usepackage{amsthm}  %pakiet do tworzenia twierdzeń itp.
\usepackage{amsmath} %pakiet do niektórych symboli matematycznych
\usepackage{amssymb} %pakiet do symboli mat., np. \nsubseteq
\usepackage{amsfonts}
\usepackage{graphicx} %obsługa plików graficznych z rozszerzeniem png, jpg
\theoremstyle{definition} %styl dla definicji
\newtheorem{zad}{} 
\title{Multizestaw zadań}
\author{Robert Fidytek}
%\date{\today}
\date{}
\newcounter{liczniksekcji}
\newcommand{\kategoria}[1]{\section{#1}} %olreślamy nazwę kateforii zadań
\newcommand{\zadStart}[1]{\begin{zad}#1\newline} %oznaczenie początku zadania
\newcommand{\zadStop}{\end{zad}}   %oznaczenie końca zadania
%Makra opcjonarne (nie muszą występować):
\newcommand{\rozwStart}[2]{\noindent \textbf{Rozwiązanie (autor #1 , recenzent #2): }\newline} %oznaczenie początku rozwiązania, opcjonarnie można wprowadzić informację o autorze rozwiązania zadania i recenzencie poprawności wykonania rozwiązania zadania
\newcommand{\rozwStop}{\newline}                                            %oznaczenie końca rozwiązania
\newcommand{\odpStart}{\noindent \textbf{Odpowiedź:}\newline}    %oznaczenie początku odpowiedzi końcowej (wypisanie wyniku)
\newcommand{\odpStop}{\newline}                                             %oznaczenie końca odpowiedzi końcowej (wypisanie wyniku)
\newcommand{\testStart}{\noindent \textbf{Test:}\newline} %ewentualne możliwe opcje odpowiedzi testowej: A. ? B. ? C. ? D. ? itd.
\newcommand{\testStop}{\newline} %koniec wprowadzania odpowiedzi testowych
\newcommand{\kluczStart}{\noindent \textbf{Test poprawna odpowiedź:}\newline} %klucz, poprawna odpowiedź pytania testowego (jedna literka): A lub B lub C lub D itd.
\newcommand{\kluczStop}{\newline} %koniec poprawnej odpowiedzi pytania testowego 
\newcommand{\wstawGrafike}[2]{\begin{figure}[h] \includegraphics[scale=#2] {#1} \end{figure}} %gdyby była potrzeba wstawienia obrazka, parametry: nazwa pliku, skala (jak nie wiesz co wpisać, to wpisz 1)

\begin{document}
\maketitle

\kategoria{Wikieł/Z5.37h}

\zadStart{Zadanie z Wikieł Z 5.37 h) moja wersja nr [nrWersji]}
%[a]:[2,3,4,5,6,7,8,9,10,11]
%[b]:[2,3,4,5,6,7,8,9,10,11]
%[c]=math.gcd([a],[b])
%[d]=int([a]/[c])
%[e]=int([b]/[c])
%[a]!=[b]
Wyznaczyć współrzędne punktów przegięcia wykresu podanej funkcji.
$$y = [a]x\sin (\ln([b]x))$$
\zadStop

\rozwStart{Natalia Danieluk}{}
Dziedzina funkcji: $\quad \mathcal{D}_f=\mathbb{R_+}$. \\
Postępujemy według schematu:
\begin{enumerate}
\item Obliczamy pochodne: 
$$f'(x) = [a]x'\sin (\ln([b]x)) + \Big(\sin (\ln([b]x)) \Big)[a]x = $$
$$\quad\quad\quad= [a]\sin (\ln([b]x)) + \Big(\cos (\ln([b]x)) \cdot \frac{1}{x}\Big)[a]x = $$
$$= [a]\big(\sin (\ln([b]x)) + \cos (\ln([b]x)) \big),$$ 
$$f''(x) = [a]\big(\sin (\ln([b]x)) + \cos (\ln([b]x)) \big)' = $$
$$\quad\quad\quad= [a]\big(\cos (\ln([b]x)) \cdot \frac{1}{x} - \sin (\ln([b]x)) \cdot \frac{1}{x} \big) = $$
$$= [a]\Bigg(\frac{\cos (\ln([b]x)) - \sin (\ln([b]x))}{x}\Bigg)$$
i określamy ich dziedziny: $\quad \mathcal{D}_{f'}=\mathcal{D}_{f''}=\mathbb{R_+}$. \\
\item Znajdujemy miejsca zerowe $f''$: \\
Zauważmy, że dla każdego $x \in \mathcal{D}_f$ mamy $\frac{[a]}{x} > 0$. \\
Wystarczy zatem zbadać znak czynnika $\cos (\ln([b]x)) - \sin (\ln([b]x))$. \\
$$f''(x)=0 \Leftrightarrow \cos (\ln([b]x)) = \sin (\ln([b]x)) \Leftrightarrow \ln([b]x) = \frac{\pi}{4} + k\pi, \ k\in\mathbb{Z} \Leftrightarrow $$
$$\Leftrightarrow x=\frac{1}{[b]}\exp(\frac{\pi}{4} + k\pi), \ k\in\mathbb{Z}$$
\item Badamy znak $f''$ po obu stronach miejsc zerowych. \\
	\begin{enumerate}
	\item $f''(x) > 0 \Leftrightarrow x \in \big(\frac{1}{[b]}\exp(\frac{\pi}{4}+ 2k\pi),\frac{1}{[b]}\exp(\frac{5\pi}{4}+ 2k\pi)\big), \ k\in\mathbb{Z}$\\
	\item $f''(x) < 0 \Leftrightarrow x \in \big(0,\frac{1}{[b]}\exp(\frac{\pi}{4})\big)\cup\big(\frac{1}{[b]}\exp(\frac{5\pi}{4} + 2k\pi),\frac{1}{[b]}\exp(\frac{9\pi}{4} + 2k\pi)\big), \ k\in\mathbb{Z}$
	\end{enumerate}
\end{enumerate}
Tym samym w sąsiedztwie punktów $x=\frac{1}{[b]}\exp(\frac{\pi}{4} + k\pi), \ k\in\mathbb{Z}$ druga pochodna zmienia znak, a więc wykres funkcji ma punkty przegięcia w punktach o współrzędnych $(x,f(x)) = (\frac{1}{[b]}\exp(\frac{\pi}{4} + k\pi),(-1)^k \cdot \frac{[d]}{[e]} \cdot \frac{\sqrt{2}}{2} \exp(\frac{\pi}{4} + k\pi)), \ k\in\mathbb{Z}$.
\rozwStop

\odpStart
Współrzędne punktów przegięcia to: $(\frac{1}{[b]}\exp(\frac{\pi}{4} + k\pi),(-1)^k \cdot \frac{[d]}{[e]} \cdot \frac{\sqrt{2}}{2} \exp(\frac{\pi}{4} + k\pi)), \ k\in\mathbb{Z}$.
\odpStop

\testStart
A. Funkcja nie ma punktów przegięcia.
B. Współrzędne punktów przegięcia to: $(0,0)$.
C. Współrzędne punktów przegięcia to:  $(\frac{1}{[b]}\exp(\frac{\pi}{4}),\frac{[d]}{[e]} \cdot \frac{\sqrt{2}}{2} \exp(\frac{\pi}{4}))$.
D. Współrzędne punktów przegięcia to:  $(\frac{1}{[b]}\exp(\frac{\pi}{4} + k\pi),(-1)^k \cdot \frac{[d]}{[e]} \cdot \frac{\sqrt{2}}{2} \exp(\frac{\pi}{4} + k\pi)), \ k\in\mathbb{Z}$.
\testStop

\kluczStart
D
\kluczStop

\end{document}