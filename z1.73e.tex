\documentclass[12pt, a4paper]{article}
\usepackage[utf8]{inputenc}
\usepackage{polski}

\usepackage{amsthm}  %pakiet do tworzenia twierdzeń itp.
\usepackage{amsmath} %pakiet do niektórych symboli matematycznych
\usepackage{amssymb} %pakiet do symboli mat., np. \nsubseteq
\usepackage{amsfonts}
\usepackage{graphicx} %obsługa plików graficznych z rozszerzeniem png, jpg
\theoremstyle{definition} %styl dla definicji
\newtheorem{zad}{} 
\title{Multizestaw zadań}
\author{Robert Fidytek}
%\date{\today}
\date{}
\newcounter{liczniksekcji}
\newcommand{\kategoria}[1]{\section{#1}} %olreślamy nazwę kateforii zadań
\newcommand{\zadStart}[1]{\begin{zad}#1\newline} %oznaczenie początku zadania
\newcommand{\zadStop}{\end{zad}}   %oznaczenie końca zadania
%Makra opcjonarne (nie muszą występować):
\newcommand{\rozwStart}[2]{\noindent \textbf{Rozwiązanie (autor #1 , recenzent #2): }\newline} %oznaczenie początku rozwiązania, opcjonarnie można wprowadzić informację o autorze rozwiązania zadania i recenzencie poprawności wykonania rozwiązania zadania
\newcommand{\rozwStop}{\newline}                                            %oznaczenie końca rozwiązania
\newcommand{\odpStart}{\noindent \textbf{Odpowiedź:}\newline}    %oznaczenie początku odpowiedzi końcowej (wypisanie wyniku)
\newcommand{\odpStop}{\newline}                                             %oznaczenie końca odpowiedzi końcowej (wypisanie wyniku)
\newcommand{\testStart}{\noindent \textbf{Test:}\newline} %ewentualne możliwe opcje odpowiedzi testowej: A. ? B. ? C. ? D. ? itd.
\newcommand{\testStop}{\newline} %koniec wprowadzania odpowiedzi testowych
\newcommand{\kluczStart}{\noindent \textbf{Test poprawna odpowiedź:}\newline} %klucz, poprawna odpowiedź pytania testowego (jedna literka): A lub B lub C lub D itd.
\newcommand{\kluczStop}{\newline} %koniec poprawnej odpowiedzi pytania testowego 
\newcommand{\wstawGrafike}[2]{\begin{figure}[h] \includegraphics[scale=#2] {#1} \end{figure}} %gdyby była potrzeba wstawienia obrazka, parametry: nazwa pliku, skala (jak nie wiesz co wpisać, to wpisz 1)

\begin{document}
\maketitle


\kategoria{Wikieł/Z1.73e}
\zadStart{Zadanie z Wikieł Z 1.73 e) moja wersja nr [nrWersji]}
%[a]:[2,3,4,5,6]
%[b]:[2,3,4,5,6]
%[c]:[2,3,4,5,6]
%[d]:[2,3,4,5,6]
%[e]:[2,3,4,5,6]
%[a]=random.randint(2,4)
%[b]=random.randint(2,8)
%[c]=random.randint(2,8)
%[d]=random.randint(6,9)
%[e]=random.randint(2,8)
%[dma]=[d]-[a]
%[bpd]=[b]+[d]
%[cpd]=[c]+[d]
%[delta]=([bpd]*[bpd])+(4*(-[dma])*[cpd])
%[pierw]=math.sqrt(abs([delta]))
%[pierw1]=int([pierw])
%[x1]=round(([bpd]-[pierw1])/(-2*[dma]),2)
%[x2]=round(([bpd]+[pierw1])/(-2*[dma]),2)
%[apd]=[a]+[d]
%[mbpd]=-[b]+[d]
%[mcpd]=-[c]+[d]
%[delta]>0 and [pierw].is_integer() and [mbpd]>1 and [mcpd]>0

Rozwiązać nierówność: $\big|\frac{[a]x^2-[b]x-[c]}{x^2+x+1}\big|\leq[d]$
\zadStop
\rozwStart{Pascal Nawrocki}{Jakub Ulrych}
Korzystamy z własności wartości bezwzględnej:
$$\bigg|\frac{[a]x^2-[b]x-[c]}{x^2+x+1}\bigg|=\frac{|[a]x^2-[b]x-[c]|}{|x^2+x+1|}$$
Zauważmy teraz, że mianownik jest zawsze dodatni, zatem możemy opuścić wartość bezwględną bez żadnych konsekwencji:
$$\frac{|[a]x^2-[b]x-[c]|}{|x^2+x+1|}=\frac{|[a]x^2-[b]x-[c]|}{x^2+x+1}$$
Po uproszczeniu ułamka, możemy wziąć się za rozwiązywanie nierówności w postaci:
$$\frac{|[a]x^2-[b]x-[c]|}{x^2+x+1}\leq[d]$$
Jako, że występuje tylko jedna wartość bezwględna, to zastosujemy metodę, aby mieć po jednej stronie wartość bezwględną a po drugiej całą resztę. (Mnożymy obustronnie przez mianownik, jako że nie ma wpływu on na zmianę dziubka nierówności).
$$\frac{|[a]x^2-[b]x-[c]|}{x^2+x+1}\leq[d]\Leftrightarrow|[a]x^2-[b]x-[c]|\leq[d](x^2+x+1)$$
Rozpatrujemy przypadki z własności:
$$|[a]x^2-[b]x-[c]|\leq[d](x^2+x+1) \wedge |[a]x^2-[b]x-[c]|\geq-[d](x^2+x+1)$$
Rozwiązujemy oba przypadki i bierzemy część wspólną:
\begin{enumerate}
\item$$|[a]x^2-[b]x-[c]|\leq[d](x^2+x+1)$$
$$-[dma]x^2-[bpd]x-[cpd]\leq0$$
$$\Delta=[delta]\Rightarrow\sqrt{\Delta}=\sqrt{[delta]}=[pierw1]$$
$$x_1=[x1] \vee x_2=[x2]\Leftrightarrow x\in(-\infty,[x2]]\cup[[x1],\infty)$$
\item$$|[a]x^2-[b]x-[c]|\geq-[d](x^2+x+1)$$
$$[apd]x^2+[mbpd]x+[mcpd]\geq0\Leftrightarrow x\in\mathbb{R} \text{ (bo $\Delta<0$)}$$
\end{enumerate}
A więc $x\in(-\infty,[x2]]\cup[[x1],\infty) \wedge x\in\mathbb{R} \Rightarrow  x\in(-\infty,[x2]]\cup[[x1],\infty)$
\rozwStop
\odpStart
$x\in(-\infty,[x2])\cup([x1],\infty)$
\odpStop
\testStart
A. $x\in(-\infty,[x2])\cup([x1],\infty)$
B.$x\in([b],\infty)$
C.$x\in\emptyset$
D.$x\in(-\infty,-[a])$
\testStop
\kluczStart
A
\kluczStop
\end{document}