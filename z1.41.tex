\documentclass[12pt, a4paper]{article}
\usepackage[utf8]{inputenc}
\usepackage{polski}

\usepackage{amsthm}  %pakiet do tworzenia twierdzeń itp.
\usepackage{amsmath} %pakiet do niektórych symboli matematycznych
\usepackage{amssymb} %pakiet do symboli mat., np. \nsubseteq
\usepackage{amsfonts}
\usepackage{graphicx} %obsługa plików graficznych z rozszerzeniem png, jpg
\theoremstyle{definition} %styl dla definicji
\newtheorem{zad}{} 
\title{Multizestaw zadań}
\author{Laura Mieczkowska}
%\date{\today}
\date{}
\newcounter{liczniksekcji}
\newcommand{\kategoria}[1]{\section{#1}} %olreślamy nazwę kateforii zadań
\newcommand{\zadStart}[1]{\begin{zad}#1\newline} %oznaczenie początku zadania
\newcommand{\zadStop}{\end{zad}}   %oznaczenie końca zadania
%Makra opcjonarne (nie muszą występować):
\newcommand{\rozwStart}[2]{\noindent \textbf{Rozwiązanie (autor #1 , recenzent #2): }\newline} %oznaczenie początku rozwiązania, opcjonarnie można wprowadzić informację o autorze rozwiązania zadania i recenzencie poprawności wykonania rozwiązania zadania
\newcommand{\rozwStop}{\newline}                                            %oznaczenie końca rozwiązania
\newcommand{\odpStart}{\noindent \textbf{Odpowiedź:}\newline}    %oznaczenie początku odpowiedzi końcowej (wypisanie wyniku)
\newcommand{\odpStop}{\newline}                                             %oznaczenie końca odpowiedzi końcowej (wypisanie wyniku)
\newcommand{\testStart}{\noindent \textbf{Test:}\newline} %ewentualne możliwe opcje odpowiedzi testowej: A. ? B. ? C. ? D. ? itd.
\newcommand{\testStop}{\newline} %koniec wprowadzania odpowiedzi testowych
\newcommand{\kluczStart}{\noindent \textbf{Test poprawna odpowiedź:}\newline} %klucz, poprawna odpowiedź pytania testowego (jedna literka): A lub B lub C lub D itd.
\newcommand{\kluczStop}{\newline} %koniec poprawnej odpowiedzi pytania testowego 
\newcommand{\wstawGrafike}[2]{\begin{figure}[h] \includegraphics[scale=#2] {#1} \end{figure}} %gdyby była potrzeba wstawienia obrazka, parametry: nazwa pliku, skala (jak nie wiesz co wpisać, to wpisz 1)

\begin{document}
\maketitle


\kategoria{Wikieł/Z1.41}
\zadStart{Zadanie z Wikieł Z 1.41) moja wersja nr [nrWersji]}
%[a]:[2,3,4,5,6,7,8,9,10]
%[b]:[2,3,4,5]
%[c]:[3,4,5,6,7,8,9,10]
%[e]:[6,7,8,9,10]
%[d]=[e]-[b]
%[e]>[b]
%[delta]=[c]**2-4*[a]*[d]
%[pierw2]=(pow([delta],1/2))
%[pierw1]=[pierw2].real
%[pierw]=int([pierw1])
%[m]=2*[d]
%[t]=-[c]-[pierw]
%[t]!=0
%[v]=-[c]+[pierw]
%[v]!=0
%[u1]=round([t]/[m],2)
%[u2]=round([v]/[m],2)
%[m]!=0 and [c]**2>4*[a]*[d] and [pierw1].is_integer()==True
Znaleźć dodatnie wartości $x$, dla których wartości funkcji $f(x)=[e]x^2+[a]$ są mniejsze od wartości funkcji $g(x)=[b]x^2-[c]x$
\zadStop
\rozwStart{Laura Mieczkowska}{}
$$f(x)<g(x)$$
$$[e]x^2+[a]<[b]x^2-[c]x$$ 
$$[d]x^2+[c]x+[a]<0$$
$$\triangle =[c]^2-4\cdot[a]\cdot[d]=[delta]\rightarrow \sqrt{\triangle}=[pierw]$$
$$x=\frac{-[c]-[pierw]}{[m]} \vee x=\frac{-[c]+[pierw]}{[m]}$$
$$x=[u1] \vee x=[u2]$$

\odpStart
$x\in\emptyset$
\odpStop
\testStart
A. $x=[u1] \vee x=[u1]$ \\
B. $x=[u1] \vee x=0$ \\
C. $x\in\emptyset$ \\
D. $x=[a] \vee x=[u2]$ 
\testStop
\kluczStart
C
\kluczStop



\end{document}