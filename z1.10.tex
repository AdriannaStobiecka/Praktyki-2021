\documentclass[12pt, a4paper]{article}
\usepackage[utf8]{inputenc}
\usepackage{polski}

\usepackage{amsthm}  %pakiet do tworzenia twierdzeń itp.
\usepackage{amsmath} %pakiet do niektórych symboli matematycznych
\usepackage{amssymb} %pakiet do symboli mat., np. \nsubseteq
\usepackage{amsfonts}
\usepackage{graphicx} %obsługa plików graficznych z rozszerzeniem png, jpg
\theoremstyle{definition} %styl dla definicji
\newtheorem{zad}{} 
\title{Multizestaw zadań}
\author{Robert Fidytek}
%\date{\today}
\date{}
\newcounter{liczniksekcji}
\newcommand{\kategoria}[1]{\section{#1}} %olreślamy nazwę kateforii zadań
\newcommand{\zadStart}[1]{\begin{zad}#1\newline} %oznaczenie początku zadania
\newcommand{\zadStop}{\end{zad}}   %oznaczenie końca zadania
%Makra opcjonarne (nie muszą występować):
\newcommand{\rozwStart}[2]{\noindent \textbf{Rozwiązanie (autor #1 , recenzent #2): }\newline} %oznaczenie początku rozwiązania, opcjonarnie można wprowadzić informację o autorze rozwiązania zadania i recenzencie poprawności wykonania rozwiązania zadania
\newcommand{\rozwStop}{\newline}                                            %oznaczenie końca rozwiązania
\newcommand{\odpStart}{\noindent \textbf{Odpowiedź:}\newline}    %oznaczenie początku odpowiedzi końcowej (wypisanie wyniku)
\newcommand{\odpStop}{\newline}                                             %oznaczenie końca odpowiedzi końcowej (wypisanie wyniku)
\newcommand{\testStart}{\noindent \textbf{Test:}\newline} %ewentualne możliwe opcje odpowiedzi testowej: A. ? B. ? C. ? D. ? itd.
\newcommand{\testStop}{\newline} %koniec wprowadzania odpowiedzi testowych
\newcommand{\kluczStart}{\noindent \textbf{Test poprawna odpowiedź:}\newline} %klucz, poprawna odpowiedź pytania testowego (jedna literka): A lub B lub C lub D itd.
\newcommand{\kluczStop}{\newline} %koniec poprawnej odpowiedzi pytania testowego 
\newcommand{\wstawGrafike}[2]{\begin{figure}[h] \includegraphics[scale=#2] {#1} \end{figure}} %gdyby była potrzeba wstawienia obrazka, parametry: nazwa pliku, skala (jak nie wiesz co wpisać, to wpisz 1)

\begin{document}
\maketitle


\kategoria{Wikieł/Z1.10}
\zadStart{Zadanie z Wikieł Z 1.10  moja wersja nr [nrWersji]}
%[p1]:[1,2,3,4,5,6,7,8,9,10]
%[p2]:[1,2,3,4,5,6,7,8,9,10]
%[p1p2]=[p1]*[p2]
%[p1p2k]=[p1]*[p2]*[p2]
%[m]=[p2]-[p1]
%[a]=round([m]/[p1p2],2)
%[b]=round([p1p2k]/9,2)
%[c]=round([b]**(1./3.),2)
%[w]=round([a]*[c],2)
%[m]!=0

Uprościć wyrażenie
$$\frac{x^{\frac{2}{3}}+\sqrt[3]{xy}+y^{\frac{2}{3}}}{x^{\frac{2}{3}}+\sqrt[3]{xy}}\cdot \frac{x^{\frac{2}{3}}-y^{\frac{2}{3}}}{\sqrt[3]{9y^{2}}}$$
a następnie obliczyć wartość tego wyrażenia dla $x=\frac{1}{[p1]}$ i $y=\frac{1}{[p2]}.$
\zadStop

\rozwStart{Maja Szabłowska}{}
$$\frac{x^{\frac{2}{3}}+\sqrt[3]{xy}+y^{\frac{2}{3}}}{x^{\frac{2}{3}}+\sqrt[3]{xy}}\cdot \frac{x^{\frac{2}{3}}-y^{\frac{2}{3}}}{\sqrt[3]{9y^{2}}}=$$
$$=\frac{(x^{\frac{2}{3}}+\sqrt[3]{xy}+y^{\frac{2}{3}})(x^{\frac{1}{3}}-y^{\frac{1}{3}})(x^{\frac{1}{3}}+y^{\frac{1}{3}})}{x^{\frac{1}{3}}(x^{\frac{1}{3}}+y^{\frac{1}{3}})\cdot \sqrt[3]{9y^{2}}}=$$
$$=\frac{(x^{\frac{2}{3}}+\sqrt[3]{xy}+y^{\frac{2}{3}})(x^{\frac{1}{3}}-y^{\frac{1}{3}})}{\sqrt[3]{9xy^{2}}}=$$
$$=\frac{x+x^{\frac{2}{3}}y^{\frac{1}{3}}+x^{\frac{1}{3}}y^{\frac{2}{3}}-x^{\frac{2}{3}}y^{\frac{1}{3}}-x^{\frac{1}{3}}y^{\frac{2}{3}}-y}{\sqrt[3]{9xy^{2}}}=$$
$$=\frac{x-y}{\sqrt[3]{9xy^{2}}}$$
Obliczenie wartości wyrażenia dla $x=\frac{1}{[p1]}$ i $y=\frac{1}{[p2]}$:
$$\frac{\frac{1}{[p1]}-\frac{1}{[p2]}}{\sqrt[3]{9\frac{1}{[p1]}\frac{1}{[p2]^{2}}}}=\frac{\frac{[p2]-[p1]}{[p1p2]}}{\sqrt[3]{\frac{9}{[p1p2k]}}}=\frac{[m]}{[p1p2]}\cdot\sqrt[3]{\frac{[p1p2k]}{9}}=[a]\cdot\sqrt[3]{[b]}=[a]\cdot[c]=[w]$$
\rozwStop


\odpStart
$[w]$
\odpStop
\testStart
A.$[w]$
B.$[a]$
D.$[b]$
E.$[c]$
F.$[p1p2]$
G.$[m]$
H.$0$
\testStop
\kluczStart
A
\kluczStop



\end{document}
