\documentclass[12pt, a4paper]{article}
\usepackage[utf8]{inputenc}
\usepackage{polski}

\usepackage{amsthm}  %pakiet do tworzenia twierdzeń itp.
\usepackage{amsmath} %pakiet do niektórych symboli matematycznych
\usepackage{amssymb} %pakiet do symboli mat., np. \nsubseteq
\usepackage{amsfonts}
\usepackage{graphicx} %obsługa plików graficznych z rozszerzeniem png, jpg
\theoremstyle{definition} %styl dla definicji
\newtheorem{zad}{} 
\title{Multizestaw zadań}
\author{Robert Fidytek}
%\date{\today}
\date{}
\newcounter{liczniksekcji}
\newcommand{\kategoria}[1]{\section{#1}} %olreślamy nazwę kateforii zadań
\newcommand{\zadStart}[1]{\begin{zad}#1\newline} %oznaczenie początku zadania
\newcommand{\zadStop}{\end{zad}}   %oznaczenie końca zadania
%Makra opcjonarne (nie muszą występować):
\newcommand{\rozwStart}[2]{\noindent \textbf{Rozwiązanie (autor #1 , recenzent #2): }\newline} %oznaczenie początku rozwiązania, opcjonarnie można wprowadzić informację o autorze rozwiązania zadania i recenzencie poprawności wykonania rozwiązania zadania
\newcommand{\rozwStop}{\newline}                                            %oznaczenie końca rozwiązania
\newcommand{\odpStart}{\noindent \textbf{Odpowiedź:}\newline}    %oznaczenie początku odpowiedzi końcowej (wypisanie wyniku)
\newcommand{\odpStop}{\newline}                                             %oznaczenie końca odpowiedzi końcowej (wypisanie wyniku)
\newcommand{\testStart}{\noindent \textbf{Test:}\newline} %ewentualne możliwe opcje odpowiedzi testowej: A. ? B. ? C. ? D. ? itd.
\newcommand{\testStop}{\newline} %koniec wprowadzania odpowiedzi testowych
\newcommand{\kluczStart}{\noindent \textbf{Test poprawna odpowiedź:}\newline} %klucz, poprawna odpowiedź pytania testowego (jedna literka): A lub B lub C lub D itd.
\newcommand{\kluczStop}{\newline} %koniec poprawnej odpowiedzi pytania testowego 
\newcommand{\wstawGrafike}[2]{\begin{figure}[h] \includegraphics[scale=#2] {#1} \end{figure}} %gdyby była potrzeba wstawienia obrazka, parametry: nazwa pliku, skala (jak nie wiesz co wpisać, to wpisz 1)

\begin{document}
\maketitle


\kategoria{Wikieł/Z5.9a}
\zadStart{Zadanie z Wikieł Z 5.9 a) moja wersja nr [nrWersji]}
%[x]:[2,3,4,5,6,7,8,9,10,11,12,13]
%[y]:[2,3,4,5,6,7,8,9,10,11,12,13]
%[z]:[2,3,4,5,6,7,8,9,10,11,12,13]
%[a]=random.randint(2,20)
%[b]=random.randint(2,20)
%[c]=random.randint(2,20)
%[a1]=random.randint(2,20)
%[c]=random.randint(2,20)
%[c1]=random.randint(2,20)
%[m]=2*[a]
%[n]=(-1)*[a1]+[b]
%[r]=[n]/[m]
%[p]=int([r])
%[s]=([a]*[p]*[p])-([b]*[p])+[c]
%[t]=([a1]*[p])+[s]
%[r].is_integer()==True and [r]!=0 and [t]<0
Wyznaczyć równanie stycznej do krzywej $y=f(x)$, wiedząc, że styczna ta jest równoległa do prostej $[a1]x+y-[c1]=0$, jeżeli $f$ wyraża się wzorem:\\
$f(x)=[a]x^2-[b]x+[c] $
\zadStop
\rozwStart{Katarzyna Filipowicz}{}
$$
[a1]x+y-[c1]=0 \quad \Rightarrow \quad y=-[a1]x+[c1]
$$
Równanie stycznej do wykresu funkcji $y=f(x)$ w punkcie $P(x_0,f(x_0))$:
$$
y-f(x_0)=f'(x_0)(x-x_0)
$$
Ponieważ proste są równoległe, to ich współczynnik kierunkowy jest taki sam, a więc: $f'(x_0)=-[a1]$
$$
y=-[a1]x+[a1]\cdot x_0+f(x_0)
$$ $$
f'(x)=2\cdot [a]x-[b] \quad \Rightarrow \quad f'(x_0)=[m]x_0-[b]=-[a1]
$$ $$
[m]x_0=[n]
$$ $$
x_0=\frac{[n]}{[m]}=[p]
$$ $$
f([p])=[a]\cdot([p])^2-[b]\cdot([p])+[c]=[s]
$$ $$
y=-[a1]x+[a1]\cdot[p]+([s])=-[a1]x [t]
$$
\rozwStop
\odpStart
$f(x)=-[a1]x [t]$
\odpStop
\testStart
A. $f(x)=-[a1]x [t]$\\
B. $f(x)=[a1]x [t]$\\
C. $f(x)=-[a1]x+([b])$\\
D. $f(x)=-[a]x [t]$\\
E. $f(x)=[a1]x+([c])$\\
F. $f(x)=-[a1]x+([s])$
\testStop
\kluczStart
A
\kluczStop



\end{document}