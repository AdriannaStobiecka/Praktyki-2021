\documentclass[12pt, a4paper]{article}
\usepackage[utf8]{inputenc}
\usepackage{polski}

\usepackage{amsthm}  %pakiet do tworzenia twierdzeń itp.
\usepackage{amsmath} %pakiet do niektórych symboli matematycznych
\usepackage{amssymb} %pakiet do symboli mat., np. \nsubseteq
\usepackage{amsfonts}
\usepackage{graphicx} %obsługa plików graficznych z rozszerzeniem png, jpg
\theoremstyle{definition} %styl dla definicji
\newtheorem{zad}{} 
\title{Multizestaw zadań}
\author{Robert Fidytek}
%\date{\today}
\date{}
\newcounter{liczniksekcji}
\newcommand{\kategoria}[1]{\section{#1}} %olreślamy nazwę kateforii zadań
\newcommand{\zadStart}[1]{\begin{zad}#1\newline} %oznaczenie początku zadania
\newcommand{\zadStop}{\end{zad}}   %oznaczenie końca zadania
%Makra opcjonarne (nie muszą występować):
\newcommand{\rozwStart}[2]{\noindent \textbf{Rozwiązanie (autor #1 , recenzent #2): }\newline} %oznaczenie początku rozwiązania, opcjonarnie można wprowadzić informację o autorze rozwiązania zadania i recenzencie poprawności wykonania rozwiązania zadania
\newcommand{\rozwStop}{\newline}                                            %oznaczenie końca rozwiązania
\newcommand{\odpStart}{\noindent \textbf{Odpowiedź:}\newline}    %oznaczenie początku odpowiedzi końcowej (wypisanie wyniku)
\newcommand{\odpStop}{\newline}                                             %oznaczenie końca odpowiedzi końcowej (wypisanie wyniku)
\newcommand{\testStart}{\noindent \textbf{Test:}\newline} %ewentualne możliwe opcje odpowiedzi testowej: A. ? B. ? C. ? D. ? itd.
\newcommand{\testStop}{\newline} %koniec wprowadzania odpowiedzi testowych
\newcommand{\kluczStart}{\noindent \textbf{Test poprawna odpowiedź:}\newline} %klucz, poprawna odpowiedź pytania testowego (jedna literka): A lub B lub C lub D itd.
\newcommand{\kluczStop}{\newline} %koniec poprawnej odpowiedzi pytania testowego 
\newcommand{\wstawGrafike}[2]{\begin{figure}[h] \includegraphics[scale=#2] {#1} \end{figure}} %gdyby była potrzeba wstawienia obrazka, parametry: nazwa pliku, skala (jak nie wiesz co wpisać, to wpisz 1)

\begin{document}
\maketitle


\kategoria{Wikieł/Z1.129o}
\zadStart{Zadanie z Wikieł Z 1.129 o) moja wersja nr [nrWersji]}
%[p1]:[2,3,4,5,6,7,8,9,10]
%[p2]:[2,3,4,5,6,7,8,9,10]
%[p3]=random.randint(2,10)
%[p4]=random.randint(2,10)
%[p5]=random.randint(2,10)
%[p6]:[2,3,4,5,6,7,8,9,10]
%[p7]=random.randint(2,10)
%[p2p4]=[p2]*[p4]
%[p2p5]=[p2]*[p5]
%[p3p4]=[p3]*[p4]
%[p3p5]=[p3]*[p5]
%[p4k]=[p4]*[p4]
%[2p4p5]=2*[p4]*[p5]
%[p5k]=[p5]*[p5]
%[xk1]=[p2p4]-[p1]*[p4k]
%[x1]=[p2p5]-[p3p4]-[p1]*[2p4p5]
%[x1p]=-[x1]
%[r1]=-[p3p5]-[p1]*[p5k]
%[xk2]=[p2p4]+[p1]*[p4k]
%[x2]=[p2p5]-[p3p4]+[p1]*[2p4p5]
%[r2]=-[p3p5]+[p1]*[p5k]
%[del1]=[x1]*[x1]-4*[xk1]*[r1]
%[pdel1]=math.sqrt(abs([del1]))
%[del2]=[x2]*[x2]-4*[xk2]*[r2]
%[pdel2]=math.sqrt(abs([del2]))
%[2a1]=2*[xk1]
%[2a2]=2*[xk2]
%[x11]=round(([x1p]-[pdel1])/([2a1]+0.0000001),2)
%[x12]=round(([x1p]+[pdel1])/([2a1]+0.0000001),2)
%[x21]=round((-[x2]-[pdel2])/([2a2]+0.0000001),2)
%[x22]=round((-[x2]+[pdel2])/([2a2]+0.0000001),2)
%[w]=round([p6]/[p7],2)
%[a]=round(-[p5]/[p4],2)
%math.gcd([p6],[p7])==1 and [del1]>0 and [del2]>0 and [x1]<0 and [r1]<0 and [x12]<[x11] and [x21]<[x22] and [x11]==[x21] and math.gcd([p5],[p4])==1

Wyznaczyć dziedzinę naturalną funkcji.
$$f(x)=\sqrt{[p1]-\left|\frac{[p2]x-[p3]}{[p4]x+[p5]}\right|}-\ln([p6]-[p7]x)$$
\zadStop

\rozwStart{Maja Szabłowska}{}
$$[p1]-\left|\frac{[p2]x-[p3]}{[p4]x+[p5]}\right|\geq 0 \quad \land\quad [p6]-[p7]x>0 \quad \land\quad [p4]x+[p5]\neq0$$
$$\left|\frac{[p2]x-[p3]}{[p4]x+[p5]}\right|\leq [p1] \quad \land \quad [p7]x<[p6] \quad \land\quad x\neq-\frac{[p5]}{[p4]}=[a]$$
$$\frac{[p2]x-[p3]}{[p4]x+[p5]}\leq [p1] \quad \land \quad \frac{[p2]x-[p3]}{[p4]x+[p5]}\geq -[p1] \quad \land \quad x<\frac{[p6]}{[p7]}$$

$$([p2]x-[p3])([p4]x+[p5])\leq [p1]([p4]x+[p5])^{2} \quad \land \quad ([p2]x-[p3])([p4]x+[p5])\geq -[p1]([p4]x+[p5])^{2}$$

$$[p2p4]x^{2}+[p2p5]x-[p3p4]x-[p3p5]\leq [p1]([p4k]x^{2}+[2p4p5]x+[p5k])$$
$$\land$$
$$[p2p4]x^{2}+[p2p5]x-[p3p4]x-[p3p5]\geq -[p1]([p4k]x^{2}+[2p4p5]x+[p5k])$$

$$[xk1]x^{2}[x1]x[r1]\leq0 \quad \land \quad [xk2]x^{2}+[x2]x+[r2]\geq 0$$

$$\Delta_{1}=([x1])^{2}-4\cdot([xk1])\cdot([r1])=[del1] \Rightarrow \sqrt{\Delta_{1}}=[pdel1]$$

$$\Delta_{2}=([x2])^{2}-4\cdot[xk2]\cdot[r2]=[del2] \Rightarrow \sqrt{\Delta_{2}}=[pdel2]$$

$$x_{11}=\frac{[x1p]-[pdel1]}{[2a1]}=[x11], \quad x_{12}=\frac{[x1p]+[pdel1]}{[2a1]}=[x12]$$

$$x_{21}=\frac{-[x2]-[pdel2]}{[2a2]}=[x21], \quad x_{22}=\frac{-[x2]+[pdel2]}{[2a2]}=[x22]$$

$$x\in(-\infty,[x12]]\cup[[x11],\infty) \land x\in(-\infty,[x21]]\cup[[x22],\infty) \land x<[w]$$

$$x\in(-\infty,[w])\setminus\{[a]\}$$
\rozwStop
\odpStart
$x\in(-\infty,[w])\setminus\{[a]\}$
\odpStop
\testStart
A.$x\in(-\infty,[w])\setminus\{[a]\}$
B.$x\in[e^{[p2]},\infty)$
C.$x\in(-\infty, 0)$
D.$x\in(-\infty, -[p2]] \cup [\ln[p1],\infty)$
E.$x\in[[p1],\infty)$
F.$x\in(0,\infty)$
G.$x\in\emptyset$
\testStop
\kluczStart
A
\kluczStop



\end{document}
