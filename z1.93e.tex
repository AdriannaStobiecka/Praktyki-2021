\documentclass[12pt, a4paper]{article}
\usepackage[utf8]{inputenc}
\usepackage{polski}

\usepackage{amsthm}  %pakiet do tworzenia twierdzeń itp.
\usepackage{amsmath} %pakiet do niektórych symboli matematycznych
\usepackage{amssymb} %pakiet do symboli mat., np. \nsubseteq
\usepackage{amsfonts}
\usepackage{graphicx} %obsługa plików graficznych z rozszerzeniem png, jpg
\theoremstyle{definition} %styl dla definicji
\newtheorem{zad}{} 
\title{Multizestaw zadań}
\author{Robert Fidytek}
%\date{\today}
\date{}
\newcounter{liczniksekcji}
\newcommand{\kategoria}[1]{\section{#1}} %olreślamy nazwę kateforii zadań
\newcommand{\zadStart}[1]{\begin{zad}#1\newline} %oznaczenie początku zadania
\newcommand{\zadStop}{\end{zad}}   %oznaczenie końca zadania
%Makra opcjonarne (nie muszą występować):
\newcommand{\rozwStart}[2]{\noindent \textbf{Rozwiązanie (autor #1 , recenzent #2): }\newline} %oznaczenie początku rozwiązania, opcjonarnie można wprowadzić informację o autorze rozwiązania zadania i recenzencie poprawności wykonania rozwiązania zadania
\newcommand{\rozwStop}{\newline}                                            %oznaczenie końca rozwiązania
\newcommand{\odpStart}{\noindent \textbf{Odpowiedź:}\newline}    %oznaczenie początku odpowiedzi końcowej (wypisanie wyniku)
\newcommand{\odpStop}{\newline}                                             %oznaczenie końca odpowiedzi końcowej (wypisanie wyniku)
\newcommand{\testStart}{\noindent \textbf{Test:}\newline} %ewentualne możliwe opcje odpowiedzi testowej: A. ? B. ? C. ? D. ? itd.
\newcommand{\testStop}{\newline} %koniec wprowadzania odpowiedzi testowych
\newcommand{\kluczStart}{\noindent \textbf{Test poprawna odpowiedź:}\newline} %klucz, poprawna odpowiedź pytania testowego (jedna literka): A lub B lub C lub D itd.
\newcommand{\kluczStop}{\newline} %koniec poprawnej odpowiedzi pytania testowego 
\newcommand{\wstawGrafike}[2]{\begin{figure}[h] \includegraphics[scale=#2] {#1} \end{figure}} %gdyby była potrzeba wstawienia obrazka, parametry: nazwa pliku, skala (jak nie wiesz co wpisać, to wpisz 1)

\begin{document}
\maketitle


\kategoria{Wikieł/Z1.93e}
\zadStart{Zadanie z Wikieł Z 1.93 e) moja wersja nr [nrWersji]}
%[a]:[2,3,4,5,6,7]
%[b]:[2,4,5,8,10,20]
%[d]:[1,2,3,4,5,6,7,8,9,11,12,13,14,15,16,17,18,19,21,22,23,24,25,26,27,28,29,31]
%[c]=[d]/[b]
%[r]=math.gcd([d],[b])
%[r1]=int([d]/[r])
%[r2]=int([b]/[r])
%[h]=int([c]*[b])
%[ar1]=[a]*[r1]
%[i]=math.gcd([ar1],[r2])
%[ar]=int([ar1]/[i])
%[rr]=int([r2]/[i])
%[br]=[b]*[rr]
%[eh]=[ar]-[br]
%[ah]=[a]*[rr]
%[oh]=([h]-1)*[rr]
%[delta]=[eh]**2+4*([ah]*[oh])
%[p]=(pow([delta],1/2))
%[pp]=int([p].real)
%[z1]=([eh]-[pp])
%[z2]=([eh]+[pp])
%[m]=2*[ah]
%[k1]=math.gcd([m],[z1])
%[k2]=math.gcd([m],[z2])
%[e1]=int([z1]/[k1])
%[f1]=int([m]/[k1])
%[e2]=int([z2]/[k2])
%[f2]=int([m]/[k2])
%math.gcd([a],[b])==1 and ([d]/[b]).is_integer()==False and ([ar]/[rr]).is_integer()==False and [eh]>0 and [delta]>0 and [p].is_integer()==True and ([e1]/[f1])>(-([b]/[a])) and ([e2]/[f2])>(-([b]/[a])) and ([e1]/[f1])<[c] and ([e2]/[f2])<[c] and [e1]!=0 and [e2]!=0
Rozwiązać równanie $\log{([a]x+[b])} + \log{([c]-x)} = 0$
\zadStop
\rozwStart{Małgorzata Ugowska}{}
Dziedzina:
$$[a]x+[b] > 0 \quad \land \quad [c]-x > 0 \quad \Longrightarrow \quad x > -\frac{[b]}{[a]} \quad \land \quad x<[c]  \quad \Longrightarrow \quad x \in (-\frac{[b]}{[a]}, [c])$$
Rozwiązujemy równo\'sć:
$$\log{([a]x+[b])} + \log{([c]-x)} = 0 \quad \Longleftrightarrow \quad \log{([a]x+[b])} + \log{(\frac{[r1]}{[r2]}-x)} = 0 $$
$$ \Longleftrightarrow \quad \log{([a]x+[b])(\frac{[r1]}{[r2]}-x)} = \log{1} \quad \Longleftrightarrow \quad ([a]x+[b])(\frac{[r1]}{[r2]}-x) = 1 $$
$$ \Longleftrightarrow \quad ([a]x+[b])(\frac{[r1]}{[r2]}-x) = 1 \quad \Longleftrightarrow \quad -[a] x^2 -[b] x + [a] \cdot \frac{[r1]}{[r2]} x + [c] \cdot [b] = 1$$
$$ \Longleftrightarrow \quad -[a] x^2 -\frac{[br]}{[rr]} x + \frac{[ar]}{[rr]} x + [h] -1 =0 \quad \Longleftrightarrow \quad [ah] x^2 - [eh] x - [oh]=0$$
$$ \bigtriangleup =[ah]^2+4 \cdot [ah] \cdot [oh] = [delta] \qquad \sqrt{\bigtriangleup} = [pp]$$
$$ x_1 = \frac{[eh]-[pp]}{2 \cdot [ah]} = \frac{[z1]}{[m]} = \frac{[e1]}{[f1]} \in D$$
$$ x_2 = \frac{[eh]+[pp]}{2 \cdot [ah]} = \frac{[z1]}{[m]} = \frac{[e2]}{[f2]} \in D$$
\rozwStop
\odpStart
$x \in \{\frac{[e1]}{[f1]}, \frac{[e2]}{[f2]}\}$
\odpStop
\testStart
A. $x \in \{-\frac{1}{8}, \frac{1}{8}\}$\\
B. $x \in \{0, 1\}$\\
C. $x \in \{\frac{1}{4}, \frac{1}{11}\}$\\
D. $x \in \{\frac{[e1]}{[f1]}, \frac{[e2]}{[f2]}\}$\\
E. $x \in \{-\frac{[b]}{[a]}, \frac{[e2]}{[f2]}\}$
\testStop
\kluczStart
D
\kluczStop



\end{document}