\documentclass[12pt, a4paper]{article}
\usepackage[utf8]{inputenc}
\usepackage{polski}

\usepackage{amsthm}  %pakiet do tworzenia twierdzeń itp.
\usepackage{amsmath} %pakiet do niektórych symboli matematycznych
\usepackage{amssymb} %pakiet do symboli mat., np. \nsubseteq
\usepackage{amsfonts}
\usepackage{graphicx} %obsługa plików graficznych z rozszerzeniem png, jpg
\theoremstyle{definition} %styl dla definicji
\newtheorem{zad}{} 
\title{Multizestaw zadań}
\author{Robert Fidytek}
%\date{\today}
\date{}
\newcounter{liczniksekcji}
\newcommand{\kategoria}[1]{\section{#1}} %olreślamy nazwę kateforii zadań
\newcommand{\zadStart}[1]{\begin{zad}#1\newline} %oznaczenie początku zadania
\newcommand{\zadStop}{\end{zad}}   %oznaczenie końca zadania
%Makra opcjonarne (nie muszą występować):
\newcommand{\rozwStart}[2]{\noindent \textbf{Rozwiązanie (autor #1 , recenzent #2): }\newline} %oznaczenie początku rozwiązania, opcjonarnie można wprowadzić informację o autorze rozwiązania zadania i recenzencie poprawności wykonania rozwiązania zadania
\newcommand{\rozwStop}{\newline}                                            %oznaczenie końca rozwiązania
\newcommand{\odpStart}{\noindent \textbf{Odpowiedź:}\newline}    %oznaczenie początku odpowiedzi końcowej (wypisanie wyniku)
\newcommand{\odpStop}{\newline}                                             %oznaczenie końca odpowiedzi końcowej (wypisanie wyniku)
\newcommand{\testStart}{\noindent \textbf{Test:}\newline} %ewentualne możliwe opcje odpowiedzi testowej: A. ? B. ? C. ? D. ? itd.
\newcommand{\testStop}{\newline} %koniec wprowadzania odpowiedzi testowych
\newcommand{\kluczStart}{\noindent \textbf{Test poprawna odpowiedź:}\newline} %klucz, poprawna odpowiedź pytania testowego (jedna literka): A lub B lub C lub D itd.
\newcommand{\kluczStop}{\newline} %koniec poprawnej odpowiedzi pytania testowego 
\newcommand{\wstawGrafike}[2]{\begin{figure}[h] \includegraphics[scale=#2] {#1} \end{figure}} %gdyby była potrzeba wstawienia obrazka, parametry: nazwa pliku, skala (jak nie wiesz co wpisać, to wpisz 1)

\begin{document}
\maketitle


\kategoria{Wikieł/P1.36m}
\zadStart{Zadanie z Wikieł P 1.36m) moja wersja nr [nrWersji]}
%[a]:[2,3,4,5,6,7,8,9]
%[b]:[2,3,4,5,6,7,8,9]
%math.gcd([a],[b])==1  

Rozwiąż nierówność \\ $[b]x^2-[a]x>0$.
\zadStop
\rozwStart{Joanna Świerzbin}{}
$$[b]x^2-[a]x>0$$
$$x ([b]x-[a])>0$$
Szukamy miejsc zerowych:
$$x=0 \vee [b]x-[a]=0$$
$$x=0 \vee [b]x=[a]$$
$$x=0 \vee x=\frac{[a]}{[b]}$$
Funkcja ta to parabola, która w przedziale $\left( -\infty,0\right)\cup \left(\frac{[a]}{[b]} , \infty \right)$ znajduje się ponad osią $y=0$, a w przedziale $\left( 0, \frac{[a]}{[b]} \right)$ poniżej tej osi, a więc 
$$x \in \left( -\infty,0\right)\cup \left(\frac{[a]}{[b]} , \infty \right)$$
\rozwStop
\odpStart
$x \in \left( -\infty,0\right)\cup \left(\frac{[a]}{[b]} , \infty \right)$
\odpStop
\testStart
A. $x \in \left( -\infty,0\right)\cup \left(\frac{[a]}{[b]} , \infty \right)$\\
B. $x \in \left( -\infty,0\right)$ \\
C. $x \in \left(\frac{[a]}{[b]} , \infty \right)$ \\
D. $x \in \left( -\infty,\frac{[a]}{[b]}\right)\cup \left(\frac{[a]}{[b]} , \infty \right)$\\
E. $x \in \left( -\infty,\frac{[b]}{[a]}\right)\cup \left(\frac{[b]}{[a]} , \infty \right)$\\
F. $ x \in \emptyset $
\testStop
\kluczStart
A
\kluczStop



\end{document}