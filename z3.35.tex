\documentclass[12pt, a4paper]{article}
\usepackage[utf8]{inputenc}
\usepackage{polski}

\usepackage{amsthm}  %pakiet do tworzenia twierdzeń itp.
\usepackage{amsmath} %pakiet do niektórych symboli matematycznych
\usepackage{amssymb} %pakiet do symboli mat., np. \nsubseteq
\usepackage{amsfonts}
\usepackage{graphicx} %obsługa plików graficznych z rozszerzeniem png, jpg
\theoremstyle{definition} %styl dla definicji
\newtheorem{zad}{} 
\title{Multizestaw zadań}
\author{Robert Fidytek}
%\date{\today}
\date{}
\newcounter{liczniksekcji}
\newcommand{\kategoria}[1]{\section{#1}} %olreślamy nazwę kateforii zadań
\newcommand{\zadStart}[1]{\begin{zad}#1\newline} %oznaczenie początku zadania
\newcommand{\zadStop}{\end{zad}}   %oznaczenie końca zadania
%Makra opcjonarne (nie muszą występować):
\newcommand{\rozwStart}[2]{\noindent \textbf{Rozwiązanie (autor #1 , recenzent #2): }\newline} %oznaczenie początku rozwiązania, opcjonarnie można wprowadzić informację o autorze rozwiązania zadania i recenzencie poprawności wykonania rozwiązania zadania
\newcommand{\rozwStop}{\newline}                                            %oznaczenie końca rozwiązania
\newcommand{\odpStart}{\noindent \textbf{Odpowiedź:}\newline}    %oznaczenie początku odpowiedzi końcowej (wypisanie wyniku)
\newcommand{\odpStop}{\newline}                                             %oznaczenie końca odpowiedzi końcowej (wypisanie wyniku)
\newcommand{\testStart}{\noindent \textbf{Test:}\newline} %ewentualne możliwe opcje odpowiedzi testowej: A. ? B. ? C. ? D. ? itd.
\newcommand{\testStop}{\newline} %koniec wprowadzania odpowiedzi testowych
\newcommand{\kluczStart}{\noindent \textbf{Test poprawna odpowiedź:}\newline} %klucz, poprawna odpowiedź pytania testowego (jedna literka): A lub B lub C lub D itd.
\newcommand{\kluczStop}{\newline} %koniec poprawnej odpowiedzi pytania testowego 
\newcommand{\wstawGrafike}[2]{\begin{figure}[h] \includegraphics[scale=#2] {#1} \end{figure}} %gdyby była potrzeba wstawienia obrazka, parametry: nazwa pliku, skala (jak nie wiesz co wpisać, to wpisz 1)

\begin{document}
\maketitle


\kategoria{Wikieł/Z3.35}
\zadStart{Zadanie z Wikieł Z 3.35 moja wersja nr [nrWersji]}
%[a]:[2,4,6,8,9]
%[b]:[2,3,4,5,6]
%[c]:[2,3,4,5,6]
%[d]:[2,3,4,5,6]
%[e]:[2,3,4,6]
%[f]=int([a]/[e])
%[delta1]=[f]**2
%[delta2]=4*[f]
%[delta]=[delta1]+[delta2]
%[x1]=(-[f]-math.sqrt([delta]))/2
%[x2]=(-[f]+math.sqrt([delta]))/2
%[x11]=round([x1], 2)
%[x22]=round([x2], 2)
%[f2]=int([f]/2)
%math.gcd([a],[e])==[e] and [x1]<-1 and [x2]>-1 and [x2]<1 and [delta]==12 and math.gcd([f],2)==2 and [b]!=[c] and [b]!=[d] and [c]!=[d] 
Rozwiązać równanie 
$$x+1+\frac{1}{x}+\frac{1}{x^2}+\dots=\lim_{n\to\infty}\frac{[a]n^2-[b]n+[c]}{[d]+n-[e]n^2},$$ 
którego lewa strona jest sumą nieskończonego ciągu geometrycznego.
\zadStop
\rozwStart{Adrianna Stobiecka}{}
Na początku zajmiemy się lewą stroną równania. Jest to suma nieskończonego ciągu geometrycznego o pierwszym wyrazie $a_1=x$ oraz $q=\frac{1}{x}$. Zakładamy $|q|<1$. Zatem:
$$|q|<1\qquad\Leftrightarrow\qquad\bigg|\frac{1}{x}\bigg|<1\qquad\leftrightarrow\qquad\frac{1}{x}<1~~\land~~\frac{1}{x}>-1$$
$$\Leftrightarrow\qquad\frac{1-x}{x}<0~~\land~~\frac{1+x}{x}>0\qquad\Leftrightarrow\qquad(1-x)x<0~~\land~~(1+x)x>0$$
Pierwsza z nierówności jest spełniona dla $x\in(-\infty,0)\cup(1,\infty)$, a druga dla $x\in(-\infty,-1)\cup(0,\infty)$. Otrzymujemy założenie, że $x\in(-\infty,-1)\cup(1,\infty)$.
\\Obliczamy lewą stronę równania:
$$x+1+\frac{1}{x}+\frac{1}{x^2}+\dots=\frac{x}{1-\frac{1}{x}}=\frac{x}{\frac{x-1}{x}}=\frac{x^2}{x-1}$$
Mianownik ułamka nie może być równy $0$, a zatem zakładamy, że $x\ne1$.
\\Przejdziemy teraz do obliczenia prawej strony równania.
$$\lim_{n\to\infty}\frac{[a]n^2-[b]n+[c]}{[d]+n-[e]n^2}=\lim_{n\to\infty}\frac{n^2([a]-\frac{[b]}{n}+\frac{[c]}{n^2})}{n^2(\frac{[d]}{n^2}+\frac{1}{n}-[e])}\lim_{n\to\infty}\frac{[a]-\frac{[b]}{n}+\frac{[c]}{n^2}}{\frac{[d]}{n^2}+\frac{1}{n}-[e]}=(*)$$
Wiemy, że:
$$\lim_{n\to\infty}\frac{[b]}{n}=0,\qquad\lim_{n\to\infty}\frac{[c]}{n^2}=0,\qquad\lim_{n\to\infty}\frac{[d]}{n^2}=0\qquad\lim_{n\to\infty}\frac{1}{n}=0$$
Mamy zatem:
$$(*)=-\frac{[a]}{[e]}=-[f]$$
Wstawiamy otrzymane wyniki do równania.
$$\frac{x^2}{x-1}=-[f]\Leftrightarrow\frac{x^2+[f]x-[f]}{x-1}=0\Leftrightarrow x^2+[f]x-[f]=0$$
Otrzymaliśmy równanie kwadratowe. Obliczamy jego pierwiastki.
$$\Delta=[f]^2-4\cdot(-[f])=[delta1]+[delta2]=[delta]\Rightarrow\sqrt{\Delta}=\sqrt{[delta]}=2\sqrt{3}$$
$$x_1=\frac{-[f]-2\sqrt{3}}{2}=-[f2]-\sqrt{3}\approx[x11],\qquad x_2=\frac{-[f]+2\sqrt{3}}{2}=-[f2]+\sqrt{3}=\approx[x22]$$
Zauważamy, że $x_2$ nie spełnia założenia, natomiast $x_1$ spełnia założenie. Zatem rozwiązaniem jest $x=-[f2]-\sqrt{3}$.
\rozwStop
\odpStart
$x=-[f2]-\sqrt{3}$
\odpStop
\testStart
A.$x=[f2]-\sqrt{3}$
B.$x=-[f2]+\sqrt{3}$
C.$x=-[f2]-\sqrt{3}$
D.$x=[f2]$
E.$x=-\sqrt{3}$
F.$x=[f2]+\sqrt{3}$
G.$x=-[f2]$
H.$x=0$
I.$x=\sqrt{3}$
\testStop
\kluczStart
C
\kluczStop



\end{document}
