\documentclass[12pt, a4paper]{article}
\usepackage[utf8]{inputenc}
\usepackage{polski}

\usepackage{amsthm}  %pakiet do tworzenia twierdzeń itp.
\usepackage{amsmath} %pakiet do niektórych symboli matematycznych
\usepackage{amssymb} %pakiet do symboli mat., np. \nsubseteq
\usepackage{amsfonts}
\usepackage{graphicx} %obsługa plików graficznych z rozszerzeniem png, jpg
\theoremstyle{definition} %styl dla definicji
\newtheorem{zad}{} 
\title{Multizestaw zadań}
\author{Robert Fidytek}
%\date{\today}
\date{}
\newcounter{liczniksekcji}
\newcommand{\kategoria}[1]{\section{#1}} %olreślamy nazwę kateforii zadań
\newcommand{\zadStart}[1]{\begin{zad}#1\newline} %oznaczenie początku zadania
\newcommand{\zadStop}{\end{zad}}   %oznaczenie końca zadania
%Makra opcjonarne (nie muszą występować):
\newcommand{\rozwStart}[2]{\noindent \textbf{Rozwiązanie (autor #1 , recenzent #2): }\newline} %oznaczenie początku rozwiązania, opcjonarnie można wprowadzić informację o autorze rozwiązania zadania i recenzencie poprawności wykonania rozwiązania zadania
\newcommand{\rozwStop}{\newline}                                            %oznaczenie końca rozwiązania
\newcommand{\odpStart}{\noindent \textbf{Odpowiedź:}\newline}    %oznaczenie początku odpowiedzi końcowej (wypisanie wyniku)
\newcommand{\odpStop}{\newline}                                             %oznaczenie końca odpowiedzi końcowej (wypisanie wyniku)
\newcommand{\testStart}{\noindent \textbf{Test:}\newline} %ewentualne możliwe opcje odpowiedzi testowej: A. ? B. ? C. ? D. ? itd.
\newcommand{\testStop}{\newline} %koniec wprowadzania odpowiedzi testowych
\newcommand{\kluczStart}{\noindent \textbf{Test poprawna odpowiedź:}\newline} %klucz, poprawna odpowiedź pytania testowego (jedna literka): A lub B lub C lub D itd.
\newcommand{\kluczStop}{\newline} %koniec poprawnej odpowiedzi pytania testowego 
\newcommand{\wstawGrafike}[2]{\begin{figure}[h] \includegraphics[scale=#2] {#1} \end{figure}} %gdyby była potrzeba wstawienia obrazka, parametry: nazwa pliku, skala (jak nie wiesz co wpisać, to wpisz 1)

\begin{document}
\maketitle


\kategoria{Wikieł/Z5.59b}
\zadStart{Zadanie z Wikieł Z 5.59 b) moja wersja nr [nrWersji]}
%[a]:[1,2,3,4,5,6,7,8,9,10]
Zbadać wypukłość funkcji:
b) $ f(x)=\sqrt{[a]+x^{2}}$
\zadStop
\rozwStart{Wojciech Przybylski}{}
$$ f(x)=\sqrt{[a]+x^{2}}\hspace{5mm} \mathcal{D}_{f}=\mathbb{R}$$
$$ f'(x)=2x\cdot\frac{1}{2}\cdot\frac{1}{\sqrt{[a]+x^{2}}}=\frac{x}{\sqrt{[a]+x^{2}}}$$
$$ f''(x)=\frac{\sqrt{[a]+x^{2}}-x\cdot\frac{x}{\sqrt{[a]+x^{2}}}}{\bigg(\sqrt{[a]+x^{2}}\bigg)^{2}}=\frac{\frac{[a]+x^{2}-x^{2}}{\sqrt{[a]+x^{2}}}}{[a]+x^{2}}=\frac{[a]}{([a]+x^{2})^{\frac{3}{2}}}$$
$\mbox{Sprawdzamy warunek konieczny: }f''(x_{0})=0 $\\
Przez to, że licznik jest liczbą dodatnią to $f''(x_{0})$ nie posiada punktów przeciążenia.\\
\wstawGrafike{z5.59b_wykres.png}{0.15}\\
Druga pochodna funkcji $f(x)$ prezentuje się niemal, jak na powyższym wykresie, więc f(x) jest wypukła na $\mathbb{R}$.
\rozwStop
\odpStart
$f(x)$ jest wypukła na $\mathbb{R}$.
\odpStop
\testStart
A. $f(x)$ jest wypukła na $\mathbb{R}$.\\
B. $f(x)$ jest wypukła na $\mathbb{Z}$.\\
C. $f(x)$ jest wypukła na $(-\infty,[a])$.\\
D. $f(x)$ jest wypukła na $([a],\infty)$.\\
E. $f(x)$ jest wypukła na $([a],\infty)$, wklęsła na $(-\infty,[a])$.\\
F. $f(x)$ nie jest wypukła.
\testStop
\kluczStart
A
\kluczStop



\end{document}