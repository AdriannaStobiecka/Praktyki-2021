\documentclass[12pt, a4paper]{article}
\usepackage[utf8]{inputenc}
\usepackage{polski}

\usepackage{amsthm}  %pakiet do tworzenia twierdzeń itp.
\usepackage{amsmath} %pakiet do niektórych symboli matematycznych
\usepackage{amssymb} %pakiet do symboli mat., np. \nsubseteq
\usepackage{amsfonts}
\usepackage{graphicx} %obsługa plików graficznych z rozszerzeniem png, jpg
\theoremstyle{definition} %styl dla definicji
\newtheorem{zad}{} 
\title{Multizestaw zadań}
\author{Robert Fidytek}
%\date{\today}
\date{}
\newcounter{liczniksekcji}
\newcommand{\kategoria}[1]{\section{#1}} %olreślamy nazwę kateforii zadań
\newcommand{\zadStart}[1]{\begin{zad}#1\newline} %oznaczenie początku zadania
\newcommand{\zadStop}{\end{zad}}   %oznaczenie końca zadania
%Makra opcjonarne (nie muszą występować):
\newcommand{\rozwStart}[2]{\noindent \textbf{Rozwiązanie (autor #1 , recenzent #2): }\newline} %oznaczenie początku rozwiązania, opcjonarnie można wprowadzić informację o autorze rozwiązania zadania i recenzencie poprawności wykonania rozwiązania zadania
\newcommand{\rozwStop}{\newline}                                            %oznaczenie końca rozwiązania
\newcommand{\odpStart}{\noindent \textbf{Odpowiedź:}\newline}    %oznaczenie początku odpowiedzi końcowej (wypisanie wyniku)
\newcommand{\odpStop}{\newline}                                             %oznaczenie końca odpowiedzi końcowej (wypisanie wyniku)
\newcommand{\testStart}{\noindent \textbf{Test:}\newline} %ewentualne możliwe opcje odpowiedzi testowej: A. ? B. ? C. ? D. ? itd.
\newcommand{\testStop}{\newline} %koniec wprowadzania odpowiedzi testowych
\newcommand{\kluczStart}{\noindent \textbf{Test poprawna odpowiedź:}\newline} %klucz, poprawna odpowiedź pytania testowego (jedna literka): A lub B lub C lub D itd.
\newcommand{\kluczStop}{\newline} %koniec poprawnej odpowiedzi pytania testowego 
\newcommand{\wstawGrafike}[2]{\begin{figure}[h] \includegraphics[scale=#2] {#1} \end{figure}} %gdyby była potrzeba wstawienia obrazka, parametry: nazwa pliku, skala (jak nie wiesz co wpisać, to wpisz 1)

\begin{document}
\maketitle


\kategoria{Wikieł/Z4.17c}
\zadStart{Zadanie z Wikieł Z 4.17 c) moja wersja nr [nrWersji]}
%[a]:[2,3,4,5,6,7,8,9,10,11,12,13,14,15,16,17,18,19,20,21,22,23,24,25,26,27,28,29,30]
Zbadać ciągłość funkcji $$f(x)=\frac{[a]x^2-x^3}{|x-[a]|}.$$
\zadStop
\rozwStart{Aleksandra Pasińska}{}
$$D:x-[a]\neq 0$$
$$x\neq[a]$$
$$D:x\in \mathbb{R}\backslash \{[a]\}$$
$$f(x)= \left\{ \begin{array}{ll}
\frac{[a]x^2-x^3}{[a]-x} & \textrm{dla $x<[a]$}\\
\frac{[a]x^2-x^3}{x-[a]} & \textrm{dla $x>[a]$} 
\end{array} \right.$$
\rozwStop
\odpStart
Funkcja jest ciągła w swojej dziedzinie naturalnej D$:x\in \mathbb{R}\backslash \{[a]\}$.\\
\odpStop
\testStart
A.D$:x\in \mathbb{R}\backslash \{[a]\}$
B.D$:x\in \mathbb{R}\backslash \{0,[a]\}$
C.D$:x\in \mathbb{R}\backslash \{-[a],0\}$
D.D$:x\in \mathbb{R}\backslash \{0\}$
E.D$:x\in \mathbb{R}\backslash \{90,[a]\}$
F.D$:x\in \mathbb{R}\backslash \{-[a]\}$
G.D$:x\in \mathbb{R}\backslash \{99,0,[a]\}$
H.D$:x\in \mathbb{R}\backslash \{-1\}$
I.D$:x\in \mathbb{R}\backslash \{-56\}$
\testStop
\kluczStart
A
\kluczStop



\end{document}