\documentclass[12pt, a4paper]{article}
\usepackage[utf8]{inputenc}
\usepackage{polski}

\usepackage{amsthm}  %pakiet do tworzenia twierdzeń itp.
\usepackage{amsmath} %pakiet do niektórych symboli matematycznych
\usepackage{amssymb} %pakiet do symboli mat., np. \nsubseteq
\usepackage{amsfonts}
\usepackage{graphicx} %obsługa plików graficznych z rozszerzeniem png, jpg
\theoremstyle{definition} %styl dla definicji
\newtheorem{zad}{} 
\title{Multizestaw zadań}
\author{Robert Fidytek}
%\date{\today}
\date{}
\newcounter{liczniksekcji}
\newcommand{\kategoria}[1]{\section{#1}} %olreślamy nazwę kateforii zadań
\newcommand{\zadStart}[1]{\begin{zad}#1\newline} %oznaczenie początku zadania
\newcommand{\zadStop}{\end{zad}}   %oznaczenie końca zadania
%Makra opcjonarne (nie muszą występować):
\newcommand{\rozwStart}[2]{\noindent \textbf{Rozwiązanie (autor #1 , recenzent #2): }\newline} %oznaczenie początku rozwiązania, opcjonarnie można wprowadzić informację o autorze rozwiązania zadania i recenzencie poprawności wykonania rozwiązania zadania
\newcommand{\rozwStop}{\newline}                                            %oznaczenie końca rozwiązania
\newcommand{\odpStart}{\noindent \textbf{Odpowiedź:}\newline}    %oznaczenie początku odpowiedzi końcowej (wypisanie wyniku)
\newcommand{\odpStop}{\newline}                                             %oznaczenie końca odpowiedzi końcowej (wypisanie wyniku)
\newcommand{\testStart}{\noindent \textbf{Test:}\newline} %ewentualne możliwe opcje odpowiedzi testowej: A. ? B. ? C. ? D. ? itd.
\newcommand{\testStop}{\newline} %koniec wprowadzania odpowiedzi testowych
\newcommand{\kluczStart}{\noindent \textbf{Test poprawna odpowiedź:}\newline} %klucz, poprawna odpowiedź pytania testowego (jedna literka): A lub B lub C lub D itd.
\newcommand{\kluczStop}{\newline} %koniec poprawnej odpowiedzi pytania testowego 
\newcommand{\wstawGrafike}[2]{\begin{figure}[h] \includegraphics[scale=#2] {#1} \end{figure}} %gdyby była potrzeba wstawienia obrazka, parametry: nazwa pliku, skala (jak nie wiesz co wpisać, to wpisz 1)

\begin{document}
\maketitle


\kategoria{Wikieł/Z5.15a}
\zadStart{Zadanie z Wikieł Z 5.15 a)moja wersja nr [nrWersji]}
%[x]:[2,3,4,5,6,7,8,9,10,11,12,13]
%[y]:[2,3,4,5,6,7,8,9,10,11,12,13]
%[a]=random.randint(2,10)
%[b]=random.randint(2,10)
%[d]=random.randint(4,10)
%[f]=random.randint(3,5)
%[e]=[d]-1
%[p]=[e]-1
%[m]=[a]*[d]*[f]
%[n]=[m]*[e]*[f]
%math.gcd([f],[b])==1
Obliczyć  pochodną rzędu drugiego funkcji:\\
$f(x)=[a]([f]x+[b])^{[d]}$
\zadStop
\rozwStart{Katarzyna Filipowicz}{}
$$
f'(x)=[a]\cdot [d]([f]x+[b])^{[d]-1}\cdot [f]=[m]([f]x+[b])^{[e]}
$$ $$
f''(x)=[m]\cdot [e]([f]x+[b])^{[e]-1}\cdot [f]=[n]([f]x+[b])^{[p]}
$$
\rozwStop
\odpStart
$f''(x)=[n]([f]x+[b])^{[p]}$
\odpStop
\testStart
A. $f''(x)=[n]([f]x+[b])^{[p]}$\\
B. $f''(x)=[a]([f]x+[b])^{[p]}$\\
C. $f''(x)=[n]([f]x+[b])^{-[p]}$\\
D. $f''(x)=[n]([f]x+[b])^{[e]}$\\
E. $f''(x)=[n](x+[b])^{[p]}$\\
F. $f''(x)=[m]([f]x+[b])^{[e]}$
\testStop
\kluczStart
A
\kluczStop



\end{document}