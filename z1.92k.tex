\documentclass[12pt, a4paper]{article}
\usepackage[utf8]{inputenc}
\usepackage{polski}

\usepackage{amsthm}  %pakiet do tworzenia twierdzeń itp.
\usepackage{amsmath} %pakiet do niektórych symboli matematycznych
\usepackage{amssymb} %pakiet do symboli mat., np. \nsubseteq
\usepackage{amsfonts}
\usepackage{graphicx} %obsługa plików graficznych z rozszerzeniem png, jpg
\theoremstyle{definition} %styl dla definicji
\newtheorem{zad}{} 
\title{Multizestaw zadań}
\author{Robert Fidytek}
%\date{\today}
\date{}
\newcounter{liczniksekcji}
\newcommand{\kategoria}[1]{\section{#1}} %olreślamy nazwę kateforii zadań
\newcommand{\zadStart}[1]{\begin{zad}#1\newline} %oznaczenie początku zadania
\newcommand{\zadStop}{\end{zad}}   %oznaczenie końca zadania
%Makra opcjonarne (nie muszą występować):
\newcommand{\rozwStart}[2]{\noindent \textbf{Rozwiązanie (autor #1 , recenzent #2): }\newline} %oznaczenie początku rozwiązania, opcjonarnie można wprowadzić informację o autorze rozwiązania zadania i recenzencie poprawności wykonania rozwiązania zadania
\newcommand{\rozwStop}{\newline}                                            %oznaczenie końca rozwiązania
\newcommand{\odpStart}{\noindent \textbf{Odpowiedź:}\newline}    %oznaczenie początku odpowiedzi końcowej (wypisanie wyniku)
\newcommand{\odpStop}{\newline}                                             %oznaczenie końca odpowiedzi końcowej (wypisanie wyniku)
\newcommand{\testStart}{\noindent \textbf{Test:}\newline} %ewentualne możliwe opcje odpowiedzi testowej: A. ? B. ? C. ? D. ? itd.
\newcommand{\testStop}{\newline} %koniec wprowadzania odpowiedzi testowych
\newcommand{\kluczStart}{\noindent \textbf{Test poprawna odpowiedź:}\newline} %klucz, poprawna odpowiedź pytania testowego (jedna literka): A lub B lub C lub D itd.
\newcommand{\kluczStop}{\newline} %koniec poprawnej odpowiedzi pytania testowego 
\newcommand{\wstawGrafike}[2]{\begin{figure}[h] \includegraphics[scale=#2] {#1} \end{figure}} %gdyby była potrzeba wstawienia obrazka, parametry: nazwa pliku, skala (jak nie wiesz co wpisać, to wpisz 1)

\begin{document}
\maketitle


\kategoria{Wikieł/Z1.92k}
\zadStart{Zadanie z Wikieł Z 1.92 k) moja wersja nr [nrWersji]}
%[b]:[1,2,3,4,5,6,7,8,9,10,11,12,13,14]
%[c]:[2,3,4,5,6,7,8,9,10,11,12]
%[d]:[1,2,3,4,5,6,7,8,9,10,11,12,13,14]
%[p]:[2,3,4,5,6,7,8,9,10]
%[e]:[2,3]
%[pp]=pow([p],[e])
%[m]=math.gcd([c],[d])
%[c2]=int([c]/[m])
%[d2]=int([d]/[m])
%[f]=[pp]*[c]
%[g]=[pp]*[d]
%[f2]=[f]-1
%[g2]=[g]+[b]
%[x]=[g2]/[f2]
%[xx]=int([x])
%[pp]<130 and ([d]/[c]).is_integer()==False and [f]!=1 and [x].is_integer()==True
Rozwiązać równanie $\log_{\frac{1}{[p]}}{\frac{x-[b]}{[c]x+[d]}}=-[e]$
\zadStop
\rozwStart{Małgorzata Ugowska}{}
Dziedzina:
$$[c]x+[d] \ne 0 \quad \wedge \quad \frac{x-[b]}{[c]x+[d]}>0$$
$$(x-[b])([c]x+[d])>0$$
$$D=(-\infty, -\frac{[d2]}{[c2]}) \cup ([b], \infty)$$
Rozwiązujemy równanie:
$$\log_{\frac{1}{[p]}}{\frac{x-[b]}{[c]x+[d]}}=-[e]$$
$$\Big({\frac{1}{[p]}}\Big)^{-[e]}= \frac{x-[b]}{[c]x+[d]}$$
$$[pp] = \frac{x-[b]}{[c]x+[d]}$$
$$x -[b]= [pp]([c]x+[d])$$
$$x -[b]= [f]x+[g]$$
$$[f2]x+[g2]=0$$
$$x=-[xx]$$
\rozwStop
\odpStart
$x =-[xx]$
\odpStop
\testStart
A. $3$
B. $-12$
C. $0$
D. $7$
E. $-[xx]$
\testStop
\kluczStart
E
\kluczStop



\end{document}