\documentclass[12pt, a4paper]{article}
\usepackage[utf8]{inputenc}
\usepackage{polski}
\usepackage{amsthm}  %pakiet do tworzenia twierdzeń itp.
\usepackage{amsmath} %pakiet do niektórych symboli matematycznych
\usepackage{amssymb} %pakiet do symboli mat., np. \nsubseteq
\usepackage{amsfonts}
\usepackage{graphicx} %obsługa plików graficznych z rozszerzeniem png, jpg
\theoremstyle{definition} %styl dla definicji
\newtheorem{zad}{} 
\title{Multizestaw zadań}
\author{Patryk Wirkus}
%\date{\today}
\date{}
\newcommand{\kategoria}[1]{\section{#1}}
\newcommand{\zadStart}[1]{\begin{zad}#1\newline}
\newcommand{\zadStop}{\end{zad}}
\newcommand{\rozwStart}[2]{\noindent \textbf{Rozwiązanie (autor #1 , recenzent #2): }\newline}
\newcommand{\rozwStop}{\newline}                                           
\newcommand{\odpStart}{\noindent \textbf{Odpowiedź:}\newline}
\newcommand{\odpStop}{\newline}
\newcommand{\testStart}{\noindent \textbf{Test:}\newline}
\newcommand{\testStop}{\newline}
\newcommand{\kluczStart}{\noindent \textbf{Test poprawna odpowiedź:}\newline}
\newcommand{\kluczStop}{\newline}
\newcommand{\wstawGrafike}[2]{\begin{figure}[h] \includegraphics[scale=#2] {#1} \end{figure}}

\begin{document}
\maketitle

\kategoria{Wikieł/1.62d}


\zadStart{Zadanie z Wikieł Z 1.62 d) moja wersja nr 1}

Rozwiązać nierówności $(x^{2}-9)^{2}(x+1)(x^{2}-2x-3)(x-0)\le0$.
\zadStop
\rozwStart{Patryk Wirkus}{}
Wejściową nierówność możemy zapisać w innej postaci.\\
Zauważmy, że wyrażenie $(x^{2}-2x-3)$ ma następujące miejsca zerowe: -1 i 3.\\
Pierwszy człon z kolei, $(x^{2}-9)^{2}$ ma 2 dwukrotne pierwiastki: -3 i 3.\\
Ostatecznie, nierówność do rozwiązania wygląda następująco:\\ $(x-3)^{3}(x+3)^{2}(x+1)^{2}(x-0)\le0$.\\
Miejsca zerowe naszego wielomianu to: $-3,-1,0,3$, przy czym niektóre z nich są wielokrotne.\\
Wielomian jest stopnia parzystego, ponadto znak współczynnika przy\linebreak najwyższej potędze x jest dodatni.\\ W związku z tym wykres wielomianu zaczyna się od lewej strony powyżej osi OX.\\
Ponadto w punktach -3 i -1 wykres odbija się od osi poziomej.\\
A więc $$x \in \{-3,-1\} \cup [0,3].$$
\rozwStop
\odpStart
$x \in \{-3,-1\} \cup [0,3]$
\odpStop
\testStart
A.$x \in \{-3,-1\} \cup [0,3]$\\
B.$x \in \{-3,-1\} \cup [0,3)$\\
C.$x \in \{-3,-1\} \cup (0,3]$\\
D.$x \in \{-3,-1\} \cup (0,3)$\\
E.$x \in \{0,3\} \cup [-3,-1]$\\
F.$x \in \{0,3\} \cup [-3,-1)$\\
G.$x \in \{0,3\} \cup (-3,-1]$\\
H.$x \in \{0,3\} \cup (-3,-1)$
\testStop
\kluczStart
A
\kluczStop



\zadStart{Zadanie z Wikieł Z 1.62 d) moja wersja nr 2}

Rozwiązać nierówności $(x^{2}-9)^{2}(x+1)(x^{2}-2x-3)(x-1)\le0$.
\zadStop
\rozwStart{Patryk Wirkus}{}
Wejściową nierówność możemy zapisać w innej postaci.\\
Zauważmy, że wyrażenie $(x^{2}-2x-3)$ ma następujące miejsca zerowe: -1 i 3.\\
Pierwszy człon z kolei, $(x^{2}-9)^{2}$ ma 2 dwukrotne pierwiastki: -3 i 3.\\
Ostatecznie, nierówność do rozwiązania wygląda następująco:\\ $(x-3)^{3}(x+3)^{2}(x+1)^{2}(x-1)\le0$.\\
Miejsca zerowe naszego wielomianu to: $-3,-1,1,3$, przy czym niektóre z nich są wielokrotne.\\
Wielomian jest stopnia parzystego, ponadto znak współczynnika przy\linebreak najwyższej potędze x jest dodatni.\\ W związku z tym wykres wielomianu zaczyna się od lewej strony powyżej osi OX.\\
Ponadto w punktach -3 i -1 wykres odbija się od osi poziomej.\\
A więc $$x \in \{-3,-1\} \cup [1,3].$$
\rozwStop
\odpStart
$x \in \{-3,-1\} \cup [1,3]$
\odpStop
\testStart
A.$x \in \{-3,-1\} \cup [1,3]$\\
B.$x \in \{-3,-1\} \cup [1,3)$\\
C.$x \in \{-3,-1\} \cup (1,3]$\\
D.$x \in \{-3,-1\} \cup (1,3)$\\
E.$x \in \{1,3\} \cup [-3,-1]$\\
F.$x \in \{1,3\} \cup [-3,-1)$\\
G.$x \in \{1,3\} \cup (-3,-1]$\\
H.$x \in \{1,3\} \cup (-3,-1)$
\testStop
\kluczStart
A
\kluczStop



\zadStart{Zadanie z Wikieł Z 1.62 d) moja wersja nr 3}

Rozwiązać nierówności $(x^{2}-9)^{2}(x+1)(x^{2}-2x-3)(x-2)\le0$.
\zadStop
\rozwStart{Patryk Wirkus}{}
Wejściową nierówność możemy zapisać w innej postaci.\\
Zauważmy, że wyrażenie $(x^{2}-2x-3)$ ma następujące miejsca zerowe: -1 i 3.\\
Pierwszy człon z kolei, $(x^{2}-9)^{2}$ ma 2 dwukrotne pierwiastki: -3 i 3.\\
Ostatecznie, nierówność do rozwiązania wygląda następująco:\\ $(x-3)^{3}(x+3)^{2}(x+1)^{2}(x-2)\le0$.\\
Miejsca zerowe naszego wielomianu to: $-3,-1,2,3$, przy czym niektóre z nich są wielokrotne.\\
Wielomian jest stopnia parzystego, ponadto znak współczynnika przy\linebreak najwyższej potędze x jest dodatni.\\ W związku z tym wykres wielomianu zaczyna się od lewej strony powyżej osi OX.\\
Ponadto w punktach -3 i -1 wykres odbija się od osi poziomej.\\
A więc $$x \in \{-3,-1\} \cup [2,3].$$
\rozwStop
\odpStart
$x \in \{-3,-1\} \cup [2,3]$
\odpStop
\testStart
A.$x \in \{-3,-1\} \cup [2,3]$\\
B.$x \in \{-3,-1\} \cup [2,3)$\\
C.$x \in \{-3,-1\} \cup (2,3]$\\
D.$x \in \{-3,-1\} \cup (2,3)$\\
E.$x \in \{2,3\} \cup [-3,-1]$\\
F.$x \in \{2,3\} \cup [-3,-1)$\\
G.$x \in \{2,3\} \cup (-3,-1]$\\
H.$x \in \{2,3\} \cup (-3,-1)$
\testStop
\kluczStart
A
\kluczStop





\end{document}
