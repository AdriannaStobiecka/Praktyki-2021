\documentclass[12pt, a4paper]{article}
\usepackage[utf8]{inputenc}
\usepackage{polski}

\usepackage{amsthm}  %pakiet do tworzenia twierdzeń itp.
\usepackage{amsmath} %pakiet do niektórych symboli matematycznych
\usepackage{amssymb} %pakiet do symboli mat., np. \nsubseteq
\usepackage{amsfonts}
\usepackage{graphicx} %obsługa plików graficznych z rozszerzeniem png, jpg
\theoremstyle{definition} %styl dla definicji
\newtheorem{zad}{} 
\title{Multizestaw zadań}
\author{Robert Fidytek}
%\date{\today}
\date{}
\newcounter{liczniksekcji}
\newcommand{\kategoria}[1]{\section{#1}} %olreślamy nazwę kateforii zadań
\newcommand{\zadStart}[1]{\begin{zad}#1\newline} %oznaczenie początku zadania
\newcommand{\zadStop}{\end{zad}}   %oznaczenie końca zadania
%Makra opcjonarne (nie muszą występować):
\newcommand{\rozwStart}[2]{\noindent \textbf{Rozwiązanie (autor #1 , recenzent #2): }\newline} %oznaczenie początku rozwiązania, opcjonarnie można wprowadzić informację o autorze rozwiązania zadania i recenzencie poprawności wykonania rozwiązania zadania
\newcommand{\rozwStop}{\newline}                                            %oznaczenie końca rozwiązania
\newcommand{\odpStart}{\noindent \textbf{Odpowiedź:}\newline}    %oznaczenie początku odpowiedzi końcowej (wypisanie wyniku)
\newcommand{\odpStop}{\newline}                                             %oznaczenie końca odpowiedzi końcowej (wypisanie wyniku)
\newcommand{\testStart}{\noindent \textbf{Test:}\newline} %ewentualne możliwe opcje odpowiedzi testowej: A. ? B. ? C. ? D. ? itd.
\newcommand{\testStop}{\newline} %koniec wprowadzania odpowiedzi testowych
\newcommand{\kluczStart}{\noindent \textbf{Test poprawna odpowiedź:}\newline} %klucz, poprawna odpowiedź pytania testowego (jedna literka): A lub B lub C lub D itd.
\newcommand{\kluczStop}{\newline} %koniec poprawnej odpowiedzi pytania testowego 
\newcommand{\wstawGrafike}[2]{\begin{figure}[h] \includegraphics[scale=#2] {#1} \end{figure}} %gdyby była potrzeba wstawienia obrazka, parametry: nazwa pliku, skala (jak nie wiesz co wpisać, to wpisz 1)

\begin{document}
\maketitle


\kategoria{Wikieł/Z2.58}
\zadStart{Zadanie z Wikieł Z 2.58 moja wersja nr [nrWersji]}
%[a]:[2,3,4,5,6,7,8,9,10,11,12,13,14,15,16,17,18,19,20,21,22,23,24,25,26,27,28,29,30,31,32,33,34,35,36,37,38,39,40]
%[b]:[2,3,4,5,6,7,8,9,10,11,12,13,14,15,16,7,18,19,20,21,22,23,24,25,26,27,28,29,30,31,32,33,34,35,36,37,38,39,40]
%[a1]:[2,3,4,5,6,7,8,9,10,11,12,13,14,15,16,17,18,19,20,21,22,23,24,25,26,27,28,29,30,31,32,33,34,35,36,37,38,39,40]
%[c]=[a]
%[ab]=[a1]*2
%[2ba1]=[ab]*[b]
%[aab]=[a1]*[a1]
%[aab1]=[b]*[aab]
%[k2ba1]=[2ba1]*[2ba1]
%[4aab1]=4*[a]*[aab1]
%[4ac]=4*[a]*[c]
%[4baa1]=4*[b]*[aab1]
%[4bc]=4*[b]*[c]
%[k]=[k2ba1]-[4baa1]
%[4a]=[4aab1]-[4bc]
%[ka]=[4ac]/[4a]
%[cka]=int([ka])
%[g]=math.sqrt([cka])
%[cg]=int([g])
%[bb1]=[a1]*[cg]
%[ka].is_integer()==True and [g].is_integer()==True 
Napisać równania stycznych do elipsy $[a]x^2+[b]y^2=[c]$ przechodzących przez punkt A([a1],0).
\zadStop
\rozwStart{Aleksandra Pasińska}{}
$$y=ax+b$$
$$0=[a1]a+b$$
$$b=-[a1]a$$
$$y=ax-[a1]a$$
$$[a]x^2+[b](ax-[a1]a)^2=[c]$$
$$[a]x^2+[b]a^2x^2-[2ba1]a^2x+[aab1]a^2-[c]=0$$
$$x^2([a]+[b]a^2)-[2ba1]a^2x+[aab1]a^2-[c]=0$$
$$\Delta=[k2ba1]a^4-4([a]+[b]a^2)([aab1]a^2-[c])$$
$$[k2ba1]a^4-[4aab1]a^2+[4ac]-[4baa1]a^4+[4bc]a^2=0$$
$$[k]a^4-[4a]a^2+[4ac]=0$$
$$-[4a]a^2=-[4ac]$$
$$a^2=[cka]$$
$$a=\pm \sqrt{[cka]}$$
$$a=\pm [cg]$$
$$b_1=-[bb1],b_2=[bb1]$$
$$ y=x-[bb1], y=-x+[bb1]$$
\rozwStop
\odpStart
$ y=x-[bb1], y=-x+[bb1]$\\
\odpStop
\testStart
A.$ y=x-[bb1], y=-x+[bb1]$
B.$ y=-x-[bb1], y=-x-[bb1]$
C.$ y=[bb1], y=[bb1]$
D.$ y=x-[bb1], y=0$
E.$ y=0, y=-x+[bb1]$
F.$ y=-[bb1], y=-x+[bb1]$
G.$ y=x, y=[bb1]$
H.$ y=x-[bb1], y=-x$
I.$ y=x, y=x+[bb1]$
\testStop
\kluczStart
A
\kluczStop



\end{document}