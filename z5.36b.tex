\documentclass[12pt, a4paper]{article}
\usepackage[utf8]{inputenc}
\usepackage{polski}

\usepackage{amsthm}  %pakiet do tworzenia twierdzeń itp.
\usepackage{amsmath} %pakiet do niektórych symboli matematycznych
\usepackage{amssymb} %pakiet do symboli mat., np. \nsubseteq
\usepackage{amsfonts}
\usepackage{graphicx} %obsługa plików graficznych z rozszerzeniem png, jpg
\usepackage{pgfplots}
\theoremstyle{definition} %styl dla definicji
\newtheorem{zad}{} 
\title{Multizestaw zadań}
\author{Robert Fidytek}
%\date{\today}
\date{}
\newcounter{liczniksekcji}
\newcommand{\kategoria}[1]{\section{#1}} %olreślamy nazwę kateforii zadań
\newcommand{\zadStart}[1]{\begin{zad}#1\newline} %oznaczenie początku zadania
\newcommand{\zadStop}{\end{zad}}   %oznaczenie końca zadania
%Makra opcjonarne (nie muszą występować):
\newcommand{\rozwStart}[2]{\noindent \textbf{Rozwiązanie (autor #1 , recenzent #2): }\newline} %oznaczenie początku rozwiązania, opcjonarnie można wprowadzić informację o autorze rozwiązania zadania i recenzencie poprawności wykonania rozwiązania zadania
\newcommand{\rozwStop}{\newline}                                            %oznaczenie końca rozwiązania
\newcommand{\odpStart}{\noindent \textbf{Odpowiedź:}\newline}    %oznaczenie początku odpowiedzi końcowej (wypisanie wyniku)
\newcommand{\odpStop}{\newline}                                             %oznaczenie końca odpowiedzi końcowej (wypisanie wyniku)
\newcommand{\testStart}{\noindent \textbf{Test:}\newline} %ewentualne możliwe opcje odpowiedzi testowej: A. ? B. ? C. ? D. ? itd.
\newcommand{\testStop}{\newline} %koniec wprowadzania odpowiedzi testowych
\newcommand{\kluczStart}{\noindent \textbf{Test poprawna odpowiedź:}\newline} %klucz, poprawna odpowiedź pytania testowego (jedna literka): A lub B lub C lub D itd.
\newcommand{\kluczStop}{\newline} %koniec poprawnej odpowiedzi pytania testowego 
\newcommand{\wstawGrafike}[2]{\begin{figure}[h] \centering \includegraphics[scale=#2] {#1} \end{figure}} %gdyby była potrzeba wstawienia obrazka, parametry: nazwa pliku, skala (jak nie wiesz co wpisać, to wpisz 1)

\begin{document}
\maketitle

\kategoria{Wikieł/Z5.36b}

\zadStart{Zadanie z Wikieł Z 5.36 b) moja wersja nr [nrWersji]}
%[a]:[2,3,4,5,6,7,8,9,10,11]
%[b]=[a]+1
%[c]=2*[a]
Wyznaczyć przedziały wypukłości i wklęsłości podanej funkcji.
$$y = \frac{x^2-[a]x+[b]}{x^2+1}$$
\zadStop

\rozwStart{Natalia Danieluk}{}
Postępujemy następująco:
\begin{enumerate}
\item Określamy dziedzinę funkcji: $\quad \mathcal{D}_f=\mathbb{R}$. \\
\item Obliczamy pochodne: 
$$\quad f'(x) = \frac{(x^2-[a]x+[b])'(x^2+1)-(x^2+1)'(x^2-[a]x+[b])}{(x^2+1)^2} = $$
$$=\frac{[a](x^2-2x-1)}{(x^2+1)^2},$$
$$\quad f''(x) = \frac{[a](x^2-2x-1)'(x^2+1)^2-((x^2+1)^2)'[a](x^2-2x-1)}{(x^2+1)^4} = $$
$$=\frac{[c](-x^3+3x^2+3x-1)}{(x^2+1)^3}$$
i określamy ich dziedziny: $\quad \mathcal{D}_{f'}=\mathcal{D}_{f''}=\mathbb{R}$. \\
\item Badamy znak $f''$. \\
Zauważmy, że dla każdego $x \in \mathcal{D}_f$ mamy $\frac{[c]}{(x^2+1)^3} > 0$. \\
Wystarczy zatem zbadać znak czynnika $(-x^3+3x^2+3x-1)$. \\
Sprawdzamy dzielniki wyrazu wolnego, tj. $1$ oraz $-1$ i zauważamy, że po podstawieniu $-1$ otrzymujemy 0. Zatem możemy wykorzystać schemat Hornera do podzielenia naszego wielomianu przez $(x+1)$.\\
\begin{center}
\begin{tabular}{ c|c c c c } 
 & -1 & 3 & 3 & -1\\ 
-1 & & 1 & -4 & 1 \\ 
 \hline
 & -1 & 4 & -1 & \textbf{0}
\end{tabular}
\end{center}
Reszta wyniosła 0, mamy więc następującą postać wielomianu:
$$(-x^2+4x-1)(x+1)$$
$$\Delta = 12, \quad \sqrt{\Delta} = 2\sqrt{3}, \quad x_1 = 2+\sqrt{3}, \quad x_2 = 2-\sqrt{3}, \quad x_0 = -1$$
\wstawGrafike{wykres_z5_36b.png}{0.75}
	\begin{enumerate}
	\item $f''(x) > 0 \Leftrightarrow x \in (-\infty,-1)\cup(2-\sqrt{3},2+\sqrt{3})$ i w tym przedziale wykres funkcji $f$ jest wypukły (wypukły w dół) $ \smile $ \\
	\item $f''(x) < 0 \Leftrightarrow x \in (-1,2-\sqrt{3})\cup(2+\sqrt{3},\infty)$ i w tym przedziale wykres funkcji $f$ jest wklęsły (wypukły w górę) $ \frown $
	\end{enumerate}
\end{enumerate}
.
\rozwStop

\odpStart
Funkcja jest wypukła w $(-\infty,-1)\cup(2-\sqrt{3},2+\sqrt{3})$ i wklęsła w $(-1,2-\sqrt{3})\cup(2+\sqrt{3},\infty)$.
\odpStop

\testStart
A. Funkcja jest wypukła w całej dziedzinie.
B. Funkcja jest wklęsła w całej dziedzinie.
C. Funkcja nie jest ani wypukła, ani wklęsła.
D. Funkcja jest wypukła w $(-1,2-\sqrt{3})\cup(2+\sqrt{3},\infty)$ i wklęsła w $(-\infty,-1)\cup(2-\sqrt{3},2+\sqrt{3})$.
E. Funkcja jest wypukła w $(-\infty,-1)\cup(2-\sqrt{3},2+\sqrt{3})$ i wklęsła w $(-1,2-\sqrt{3})\cup(2+\sqrt{3},\infty)$.
F. Funkcja jest wypukła w $(0,\infty)$ i wklęsła w $(-\infty,0)$.
\testStop

\kluczStart
E
\kluczStop

\end{document}