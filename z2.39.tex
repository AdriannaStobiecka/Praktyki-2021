\documentclass[12pt, a4paper]{article}
\usepackage[utf8]{inputenc}
\usepackage{polski}

\usepackage{amsthm}  %pakiet do tworzenia twierdzeń itp.
\usepackage{amsmath} %pakiet do niektórych symboli matematycznych
\usepackage{amssymb} %pakiet do symboli mat., np. \nsubseteq
\usepackage{amsfonts}
\usepackage{graphicx} %obsługa plików graficznych z rozszerzeniem png, jpg
\theoremstyle{definition} %styl dla definicji
\newtheorem{zad}{} 
\title{Multizestaw zadań}
\author{Robert Fidytek}
%\date{\today}
\date{}
\newcounter{liczniksekcji}
\newcommand{\kategoria}[1]{\section{#1}} %olreślamy nazwę kateforii zadań
\newcommand{\zadStart}[1]{\begin{zad}#1\newline} %oznaczenie początku zadania
\newcommand{\zadStop}{\end{zad}}   %oznaczenie końca zadania
%Makra opcjonarne (nie muszą występować):
\newcommand{\rozwStart}[2]{\noindent \textbf{Rozwiązanie (autor #1 , recenzent #2): }\newline} %oznaczenie początku rozwiązania, opcjonarnie można wprowadzić informację o autorze rozwiązania zadania i recenzencie poprawności wykonania rozwiązania zadania
\newcommand{\rozwStop}{\newline}                                            %oznaczenie końca rozwiązania
\newcommand{\odpStart}{\noindent \textbf{Odpowiedź:}\newline}    %oznaczenie początku odpowiedzi końcowej (wypisanie wyniku)
\newcommand{\odpStop}{\newline}                                             %oznaczenie końca odpowiedzi końcowej (wypisanie wyniku)
\newcommand{\testStart}{\noindent \textbf{Test:}\newline} %ewentualne możliwe opcje odpowiedzi testowej: A. ? B. ? C. ? D. ? itd.
\newcommand{\testStop}{\newline} %koniec wprowadzania odpowiedzi testowych
\newcommand{\kluczStart}{\noindent \textbf{Test poprawna odpowiedź:}\newline} %klucz, poprawna odpowiedź pytania testowego (jedna literka): A lub B lub C lub D itd.
\newcommand{\kluczStop}{\newline} %koniec poprawnej odpowiedzi pytania testowego 
\newcommand{\wstawGrafike}[2]{\begin{figure}[h] \includegraphics[scale=#2] {#1} \end{figure}} %gdyby była potrzeba wstawienia obrazka, parametry: nazwa pliku, skala (jak nie wiesz co wpisać, to wpisz 1)

\begin{document}
\maketitle


\kategoria{Wikieł/Z2.39}
\zadStart{Zadanie z Wikieł Z 2.39  moja wersja nr [nrWersji]}
%[p1]:[2,3,4,5,6,7,8,9,10]
%[p2]:[2,3,4,5,6,7,8,9,10]
%[p3]:[2,3,4,5,6,7,8,9,10]
%[a1]=random.randint(2,10)
%[a2]=random.randint(2,10)
%[a3]=random.randint(2,10)
%[w1]=[p2]*2
%[w2]=[p3]*2
%[p1w2]=[p1]-[w2]
%[p2p1]=[p2]*[p1]
%[a1pw1]=[a1]*[p1w2]
%[a2pp12]=[a2]*[p2p1]
%[a3w1]=[a3]*[w1]
%[a3p2]=[a3]*[p2]
%[m2]=[a1pw1]+[a2]-[a3]
%[m1]=-[a1]-[a2pp12]+[a3w1]
%[ma3p2]=-[a3p2]
%[del0]=pow([w1],2)-4*[p2]
%[pdel0]=round(math.sqrt(abs([del0])),2)
%[n1]=round(([w1]-[pdel0])/2,2)
%[n2]=round(([w1]+[pdel0])/2,2)
%[del]=pow([m1],2)-4*[m2]*[ma3p2]
%[pdel]=round(math.sqrt(abs([del])),2)
%[m22]=pow([m2],2)
%[odp1]=round((-[m1]-[pdel])/([m22]+0.0000001),2)
%[odp2]=round((-[m1]+[pdel])/([m22]+0.0000001),2)
%[del]>0 and [m2]!=0 and [del0]>0 and [odp1]!=[n1] and [odp1]!=[n2] and [odp2]!=[n1] and [odp2]!=[n2]



Rozwiązać układ
$$\left\{\begin{array}{rcl}
mx+(2m-1)y&=&[p1]m\\
\ [p2]x+my&=&[p3]m
\end{array} \right.$$
i podać wartości $m$, dla których punkt przecięcia się prostych danych równaniami układu należy do prostej $[a1]x+[a2]y-[a3]=0$.
\zadStop
\rozwStart{Maja Szabłowska}{}
$$W=\left| \begin{array}{lccr} m & 2m-1 \\ \ [p2] & m \end{array}\right| = m\cdot m - [p2]\cdot (2m-1)=m^{2}-[w1]m+[p2] \neq 0$$

$$m^{2}-[w1]m+[p2] \neq 0 $$

$$\Delta=(-[w1])^{2}-4\cdot 1 \cdot [p2]=[del0] \Rightarrow \sqrt{\Delta}=[pdel0]$$

$$m\neq \frac{[w1]-[pdel0]}{2}\neq [n1] \land m\neq \frac{[w1]+[pdel0]}{2} \neq [n2]$$

$$W_{x}=\left| \begin{array}{lccr} [p1]m & 2m-1 \\ \ [p3]m & m \end{array}\right| = [p1]m\cdot m -[p3]m\cdot(2m-1)=[p1]m^{2}-[w2]m^{2}-m=[p1w2]m^{2}-m$$

$$W_{y}=\left| \begin{array}{lccr} m & [p1]m \\ \ [p2] & m \end{array}\right| = m\cdot m - [p2]\cdot[p1]m=m^{2}-[p2p1]m$$

$$x=\frac{W_{x}}{W}=\frac{[p1w2]m^{2}-m}{m^{2}-[w1]m+[p2]}$$

$$y=\frac{W_{y}}{W}=\frac{m^{2}-[p2p1]m}{m^{2}-[w1]m+[p2]}$$

$$[a1]\frac{[p1w2]m^{2}-m}{m^{2}-[w1]m+[p2]}+[a2]\frac{m^{2}-[p2p1]m}{m^{2}-[w1]m+[p2]}-[a3]=0$$

$$\frac{[a1pw1]m^{2}-[a1]m}{m^{2}-[w1]m+[p2]}+\frac{[a2]m^{2}-[a2pp12]m}{m^{2}-[w1]m+[p2]}-\frac{[a3]m^{2}-[a3w1]m+[a3p2]}{m^{2}-[w1]m+[p2]}=0$$

$$\frac{[m2]m^{2}+([m1])m-[a3p2]}{m^{2}-[w1]m+[p2]}=0$$
$$[m2]m^{2}+([m1])m-[a3p2]=0$$
$$\Delta=([m1])^{2}-4\cdot([m2])\cdot(-[a3p2])=[del] \Rightarrow \sqrt{\Delta}=[pdel]$$
$$m_{1}=\frac{-([m1])-[pdel]}{[m22]}=[odp1]\quad \lor \quad m_{2}=\frac{-([m1])+[pdel]}{[m22]}=[odp2]$$



\rozwStop
\odpStart
$m_{1}=[odp1], m_{2}=[odp2]$
\odpStop
\testStart
A.$m_{1}=[n1], m_{2}=[odp2]$
B.$m_{1}=[odp1], m_{2}=[n2]$
C.$m=[a3p2]$
D.$m_{1}=[m2], m_{2}=[odp2]$
E.$m_{1}=[odp1], m_{2}=[del]$
F.$m_{1}=[p1w2], m_{2}=[odp2]$
G.$m_{1}=[p1], m_{2}=[a3p2]$
H.$m_{1}=[p2], m_{2}=[p3]$
I.$m=[a1]$
\testStop
\kluczStart
A
\kluczStop
\end{document}
