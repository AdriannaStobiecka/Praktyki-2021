\documentclass[12pt, a4paper]{article}
\usepackage[utf8]{inputenc}
\usepackage{polski}

\usepackage{amsthm}  %pakiet do tworzenia twierdzeń itp.
\usepackage{amsmath} %pakiet do niektórych symboli matematycznych
\usepackage{amssymb} %pakiet do symboli mat., np. \nsubseteq
\usepackage{amsfonts}
\usepackage{graphicx} %obsługa plików graficznych z rozszerzeniem png, jpg
\theoremstyle{definition} %styl dla definicji
\newtheorem{zad}{} 
\title{Multizestaw zadań}
\author{Robert Fidytek}
%\date{\today}
\date{}
\newcounter{liczniksekcji}
\newcommand{\kategoria}[1]{\section{#1}} %olreślamy nazwę kateforii zadań
\newcommand{\zadStart}[1]{\begin{zad}#1\newline} %oznaczenie początku zadania
\newcommand{\zadStop}{\end{zad}}   %oznaczenie końca zadania
%Makra opcjonarne (nie muszą występować):
\newcommand{\rozwStart}[2]{\noindent \textbf{Rozwiązanie (autor #1 , recenzent #2): }\newline} %oznaczenie początku rozwiązania, opcjonarnie można wprowadzić informację o autorze rozwiązania zadania i recenzencie poprawności wykonania rozwiązania zadania
\newcommand{\rozwStop}{\newline}                                            %oznaczenie końca rozwiązania
\newcommand{\odpStart}{\noindent \textbf{Odpowiedź:}\newline}    %oznaczenie początku odpowiedzi końcowej (wypisanie wyniku)
\newcommand{\odpStop}{\newline}                                             %oznaczenie końca odpowiedzi końcowej (wypisanie wyniku)
\newcommand{\testStart}{\noindent \textbf{Test:}\newline} %ewentualne możliwe opcje odpowiedzi testowej: A. ? B. ? C. ? D. ? itd.
\newcommand{\testStop}{\newline} %koniec wprowadzania odpowiedzi testowych
\newcommand{\kluczStart}{\noindent \textbf{Test poprawna odpowiedź:}\newline} %klucz, poprawna odpowiedź pytania testowego (jedna literka): A lub B lub C lub D itd.
\newcommand{\kluczStop}{\newline} %koniec poprawnej odpowiedzi pytania testowego 
\newcommand{\wstawGrafike}[2]{\begin{figure}[h] \includegraphics[scale=#2] {#1} \end{figure}} %gdyby była potrzeba wstawienia obrazka, parametry: nazwa pliku, skala (jak nie wiesz co wpisać, to wpisz 1)

\begin{document}
\maketitle


\kategoria{Wikieł/Z1.92i}
\zadStart{Zadanie z Wikieł Z 1.92 i) moja wersja nr [nrWersji]}
%[a]:[2,3,4,5,6,7,8,9,10,11,12]
%[b]:[1,2,3,4,5,6,7,8,9,10,11,12,13,14,15]
%[c]:[1,2,3,4,5,6,7]
%[d]=random.randint(2,25)
%[a2]=[a]*[c]
%[b2]=[b]*[c]
%[e]=[d]-[b2]
%(([d]-[c]*[b])/[c])+[b]>0 and (([d]-[c]*[b])/[c])+[b] != 1 and math.gcd([c],[d])==1 and math.gcd([e],[a2])==1
Rozwiązać równanie $\log_{[a]x+[b]}{\frac{[c]}{[d]}}=-1$
\zadStop
\rozwStart{Małgorzata Ugowska}{}
$$\log_{[a]x+[b]}{\frac{[c]}{[d]}}=-1 \quad \Longleftrightarrow \quad {([a]x+[b])}^{-1}=\frac{[c]}{[d]} \quad \Longleftrightarrow \quad \frac{1}{[a]x+[b]}=\frac{[c]}{[d]}$$
$$\Longleftrightarrow \quad ([a]x+[b])[c]=[d] \quad \Longleftrightarrow \quad [a2]x+[b2]=[d] \quad \Longleftrightarrow \quad [a2]x=[e] \quad \Longleftrightarrow \quad x=\frac{[e]}{[a2]}$$
\rozwStop
\odpStart
$x=\frac{[e]}{[a2]}$
\odpStop
\testStart
A. $\frac{[e]}{[a2]}$
B. $\frac{[a]}{[a2]}$
C. $\frac{[b]}{[d]}$
D. $[b2]$
E. $[e]$
\testStop
\kluczStart
A
\kluczStop



\end{document}