\documentclass[12pt, a4paper]{article}
\usepackage[utf8]{inputenc}
\usepackage{polski}

\usepackage{amsthm}  %pakiet do tworzenia twierdzeń itp.
\usepackage{amsmath} %pakiet do niektórych symboli matematycznych
\usepackage{amssymb} %pakiet do symboli mat., np. \nsubseteq
\usepackage{amsfonts}
\usepackage{graphicx} %obsługa plików graficznych z rozszerzeniem png, jpg
\theoremstyle{definition} %styl dla definicji
\newtheorem{zad}{} 
\title{Multizestaw zadań}
\author{Robert Fidytek}
%\date{\today}
\date{}
\newcounter{liczniksekcji}
\newcommand{\kategoria}[1]{\section{#1}} %olreślamy nazwę kateforii zadań
\newcommand{\zadStart}[1]{\begin{zad}#1\newline} %oznaczenie początku zadania
\newcommand{\zadStop}{\end{zad}}   %oznaczenie końca zadania
%Makra opcjonarne (nie muszą występować):
\newcommand{\rozwStart}[2]{\noindent \textbf{Rozwiązanie (autor #1 , recenzent #2): }\newline} %oznaczenie początku rozwiązania, opcjonarnie można wprowadzić informację o autorze rozwiązania zadania i recenzencie poprawności wykonania rozwiązania zadania
\newcommand{\rozwStop}{\newline}                                            %oznaczenie końca rozwiązania
\newcommand{\odpStart}{\noindent \textbf{Odpowiedź:}\newline}    %oznaczenie początku odpowiedzi końcowej (wypisanie wyniku)
\newcommand{\odpStop}{\newline}                                             %oznaczenie końca odpowiedzi końcowej (wypisanie wyniku)
\newcommand{\testStart}{\noindent \textbf{Test:}\newline} %ewentualne możliwe opcje odpowiedzi testowej: A. ? B. ? C. ? D. ? itd.
\newcommand{\testStop}{\newline} %koniec wprowadzania odpowiedzi testowych
\newcommand{\kluczStart}{\noindent \textbf{Test poprawna odpowiedź:}\newline} %klucz, poprawna odpowiedź pytania testowego (jedna literka): A lub B lub C lub D itd.
\newcommand{\kluczStop}{\newline} %koniec poprawnej odpowiedzi pytania testowego 
\newcommand{\wstawGrafike}[2]{\begin{figure}[h] \includegraphics[scale=#2] {#1} \end{figure}} %gdyby była potrzeba wstawienia obrazka, parametry: nazwa pliku, skala (jak nie wiesz co wpisać, to wpisz 1)

\begin{document}
\maketitle


\kategoria{Wikieł/Z3.12m}
\zadStart{Zadanie z Wikieł Z 3.12 m) moja wersja nr [nrWersji]}
%[f]:[1,2,9,10,11,12]
%[a1]:[3,4,5,6,7,8]
%[b]=random.randint(1,40)
%[a]=random.randint(1,40)
%[d]=random.randint(1,40)
%[c]=random.randint(1,40)
%[e]:[3,4,5,6,7]
%[m]=[e]-2
%[m3]=2*[e]
%[m1]=int([m]/(math.gcd([m],[m3])))
%[m2]=int([m3]/(math.gcd([m],[m3])))
%[m]!=[b]
Obliczyć granicę ciągu $a_n= \frac{\sqrt{n+[a]}-[b]}{\sqrt[{[e]}]{n-[c]}+[d]}$.
\zadStop
\rozwStart{Barbara Bączek}{}
$$\lim_{n \rightarrow \infty} a_n= \lim_{n \rightarrow \infty} \frac{\sqrt{n+[a]}-[b]}{\sqrt[{[e]}]{n-[c]}+[d]}= \lim_{n \rightarrow \infty} \frac{\sqrt{n}\big{(}\sqrt{1+\frac{[a]}{n}}-\frac{[b]}{\sqrt{n}}\big{)}}{\sqrt[{[e]}]{n}\big{(}\sqrt[{[e]}]{1-\frac{[c]}{n}}+\frac{[d]}{\sqrt[{[e]}]{n}}\big{)}}=  \lim_{n \rightarrow \infty} \frac{\sqrt{n}}{\sqrt[{[e]}]{n}}= \lim_{n \rightarrow \infty} n^{\frac{[m1]}{[m2]}}= \infty $$ 
\rozwStop
\odpStart
$\infty$
\odpStop
\testStart
A.$\infty$
B.$\frac{[m1]}{[m2]}$
C.$-\infty$
D.$0$
E.$\frac{[m2]}{[m1]}$
G.$[m]$
H.$[b]$
\testStop
\kluczStart
A
\kluczStop



\end{document}