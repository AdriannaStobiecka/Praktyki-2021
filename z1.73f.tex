\documentclass[12pt, a4paper]{article}
\usepackage[utf8]{inputenc}
\usepackage{polski}

\usepackage{amsthm}  %pakiet do tworzenia twierdzeń itp.
\usepackage{amsmath} %pakiet do niektórych symboli matematycznych
\usepackage{amssymb} %pakiet do symboli mat., np. \nsubseteq
\usepackage{amsfonts}
\usepackage{graphicx} %obsługa plików graficznych z rozszerzeniem png, jpg
\theoremstyle{definition} %styl dla definicji
\newtheorem{zad}{} 
\title{Multizestaw zadań}
\author{Robert Fidytek}
%\date{\today}
\date{}
\newcounter{liczniksekcji}
\newcommand{\kategoria}[1]{\section{#1}} %olreślamy nazwę kateforii zadań
\newcommand{\zadStart}[1]{\begin{zad}#1\newline} %oznaczenie początku zadania
\newcommand{\zadStop}{\end{zad}}   %oznaczenie końca zadania
%Makra opcjonarne (nie muszą występować):
\newcommand{\rozwStart}[2]{\noindent \textbf{Rozwiązanie (autor #1 , recenzent #2): }\newline} %oznaczenie początku rozwiązania, opcjonarnie można wprowadzić informację o autorze rozwiązania zadania i recenzencie poprawności wykonania rozwiązania zadania
\newcommand{\rozwStop}{\newline}                                            %oznaczenie końca rozwiązania
\newcommand{\odpStart}{\noindent \textbf{Odpowiedź:}\newline}    %oznaczenie początku odpowiedzi końcowej (wypisanie wyniku)
\newcommand{\odpStop}{\newline}                                             %oznaczenie końca odpowiedzi końcowej (wypisanie wyniku)
\newcommand{\testStart}{\noindent \textbf{Test:}\newline} %ewentualne możliwe opcje odpowiedzi testowej: A. ? B. ? C. ? D. ? itd.
\newcommand{\testStop}{\newline} %koniec wprowadzania odpowiedzi testowych
\newcommand{\kluczStart}{\noindent \textbf{Test poprawna odpowiedź:}\newline} %klucz, poprawna odpowiedź pytania testowego (jedna literka): A lub B lub C lub D itd.
\newcommand{\kluczStop}{\newline} %koniec poprawnej odpowiedzi pytania testowego 
\newcommand{\wstawGrafike}[2]{\begin{figure}[h] \includegraphics[scale=#2] {#1} \end{figure}} %gdyby była potrzeba wstawienia obrazka, parametry: nazwa pliku, skala (jak nie wiesz co wpisać, to wpisz 1)

\begin{document}
\maketitle


\kategoria{Wikieł/Z1.73f}
\zadStart{Zadanie z Wikieł Z 1.73 f) moja wersja nr [nrWersji]}
%[a]:[2,3,4,5,6,7,8,9]
%[b]:[2,3,4,5,6,7,8,9]
%[a]=random.randint(2,17)
%[b]=random.randint(2,17)
%[a2]=[a]*[a]
Rozwiązać nierówność.$\frac{x^{2}-[a]|x|}{x^{2}+[b]}<0$
\zadStop
\rozwStart{Jakub Ulrych}{}
$$\frac{x^{2}-[a]|x|}{x^{2}+[b]}<0$$
Mianownik jest zawsze większy od zera więc
$$x^{2}-[a]|x|<0$$
$$[a]|x|>x^{2}$$
$$\textbf{1)}[a]x>x^{2}\vee\textbf{2)}[a]x<-x^{2}$$
$$\textbf{1)}-x^{2}+[a]x>0$$
$$\Delta_{1}=[a2]\Rightarrow \sqrt{\Delta_{1}}=[a]$$
$$x_{1}=\frac{-[a]-[a]}{-2}=[a],x_{2}=\frac{-[a]+[a]}{-2}=0$$
$$x\in(0,[a])$$
$$\textbf{2)}x^{2}+[a]x<0$$
$$\Delta_{2}=[a2]\Rightarrow \sqrt{\Delta_{2}}=[a]$$
$$x_{3}=\frac{-[a]-[a]}{2}=-[a],x_{4}=\frac{-[a]+[a]}{2}=0$$
$$x\in(-[a],0)$$
$$\textbf{1)}\vee\textbf{2)}\Rightarrow x\in(-[a],0)\cup(0,[a])$$
\rozwStop
\odpStart
$$x\in(-[a],0)\cup(0,[a])$$
\odpStop
\testStart
A.$$x\in(-[a],0)\cup(0,[a])$$
B.$$x\in(-[a],[a])$$
C.$$x\in(-\infty,[a])$$
D.$$x\in(-[a],\infty)$$
\testStop
\kluczStart
A
\kluczStop
\end{document}