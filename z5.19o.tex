\documentclass[12pt, a4paper]{article}
\usepackage[utf8]{inputenc}
\usepackage{polski}

\usepackage{amsthm}  %pakiet do tworzenia twierdzeń itp.
\usepackage{amsmath} %pakiet do niektórych symboli matematycznych
\usepackage{amssymb} %pakiet do symboli mat., np. \nsubseteq
\usepackage{amsfonts}
\usepackage{graphicx} %obsługa plików graficznych z rozszerzeniem png, jpg
\theoremstyle{definition} %styl dla definicji
\newtheorem{zad}{} 
\title{Multizestaw zadań}
\author{Robert Fidytek}
%\date{\today}
\date{}
\newcounter{liczniksekcji}
\newcommand{\kategoria}[1]{\section{#1}} %olreślamy nazwę kateforii zadań
\newcommand{\zadStart}[1]{\begin{zad}#1\newline} %oznaczenie początku zadania
\newcommand{\zadStop}{\end{zad}}   %oznaczenie końca zadania
%Makra opcjonarne (nie muszą występować):
\newcommand{\rozwStart}[2]{\noindent \textbf{Rozwiązanie (autor #1 , recenzent #2): }\newline} %oznaczenie początku rozwiązania, opcjonarnie można wprowadzić informację o autorze rozwiązania zadania i recenzencie poprawności wykonania rozwiązania zadania
\newcommand{\rozwStop}{\newline}                                            %oznaczenie końca rozwiązania
\newcommand{\odpStart}{\noindent \textbf{Odpowiedź:}\newline}    %oznaczenie początku odpowiedzi końcowej (wypisanie wyniku)
\newcommand{\odpStop}{\newline}                                             %oznaczenie końca odpowiedzi końcowej (wypisanie wyniku)
\newcommand{\testStart}{\noindent \textbf{Test:}\newline} %ewentualne możliwe opcje odpowiedzi testowej: A. ? B. ? C. ? D. ? itd.
\newcommand{\testStop}{\newline} %koniec wprowadzania odpowiedzi testowych
\newcommand{\kluczStart}{\noindent \textbf{Test poprawna odpowiedź:}\newline} %klucz, poprawna odpowiedź pytania testowego (jedna literka): A lub B lub C lub D itd.
\newcommand{\kluczStop}{\newline} %koniec poprawnej odpowiedzi pytania testowego 
\newcommand{\wstawGrafike}[2]{\begin{figure}[h] \includegraphics[scale=#2] {#1} \end{figure}} %gdyby była potrzeba wstawienia obrazka, parametry: nazwa pliku, skala (jak nie wiesz co wpisać, to wpisz 1)

\begin{document}
\maketitle


\kategoria{Wikieł/Z5.19 o}
\zadStart{Zadanie z Wikieł Z 5.19 o) moja wersja nr [nrWersji]}
%[a]:[2,3,4,5,6,7,8,9]
%[b]:[2,3,4,5,6,7,8,9]
%[a0]=2*[a]
%[b0]=6*[b]
%[dziel]=math.gcd([a0],[b0])
%[a1]=int([a0]/[dziel])
%[b1]=int([b0]/[dziel])
%math.gcd([a],[b])==1
Oblicz granicę $\lim_{x \rightarrow 0} \frac{[a]e^x\sin(x)-[a]x(1+x)}{[b]x^3}$.
\zadStop
\rozwStart{Joanna Świerzbin}{}
$$ \lim_{x \rightarrow 0} \frac{[a]e^x\sin(x)-[a]x(1+x)}{[b]x^3}$$
Otrzymujemy $ \left[ \frac{0}{0} \right] $ więc możemy skorzystać z twierdzenia de l'Hospitala.
$$ \lim_{x \rightarrow 0} \frac{\left([a]e^x\sin(x)-[a]x(1+x)\right)'}{\left([b]x^3\right)'}=$$
$$= \lim_{x \rightarrow 0} \frac{[a]e^x\sin(x)+[a]e^x\cos(x)-[a](1+x)-[a]x}{3\cdot[b]x^2}=$$
$$= \lim_{x \rightarrow 0} \frac{[a]e^x\sin(x)+[a]e^x\cos(x)-[a]-2\cdot[a]x}{3\cdot[b]x^2}$$
Otrzymujemy $ \left[ \frac{0}{0} \right] $ więc możemy skorzystać z twierdzenia de l'Hospitala.
$$\lim_{x \rightarrow 0} \frac{\left([a]e^x\sin(x)+[a]e^x\cos(x)-[a]-2\cdot[a]x\right)'}{\left(3\cdot[b]x^2\right)'}=$$
$$=\lim_{x \rightarrow 0} \frac{[a]e^x\sin(x)+[a]e^x\cos(x)+[a]e^x\cos(x)-[a]e^x\sin(x)-2\cdot[a]}{6\cdot[b]x}=$$
$$=\lim_{x \rightarrow 0} \frac{2\cdot[a]e^x\cos(x)-2\cdot[a]}{6\cdot[b]x}$$
Otrzymujemy $ \left[ \frac{0}{0} \right] $ więc możemy skorzystać z twierdzenia de l'Hospitala.
$$\lim_{x \rightarrow 0} \frac{\left(2\cdot[a]e^x\cos(x)-2\cdot[a]\right)'}{\left(6\cdot[b]x\right)'}=$$
$$=\lim_{x \rightarrow 0} \frac{2\cdot[a]e^x\cos(x)-2\cdot[a]e^x\sin(x)}{6\cdot[b]}=\frac{2\cdot[a]}{6\cdot[b]} = \frac{[a1]}{[b1]} $$
\rozwStop
\odpStart
$\lim_{x \rightarrow 0} \frac{[a]e^x\sin(x)-[a]x(1+x)}{[b]x^3} =  \frac{[a1]}{[b1]} $
\odpStop
\testStart
A. $\lim_{x \rightarrow 0} \frac{[a]e^x\sin(x)-[a]x(1+x)}{[b]x^3} =  \frac{[a1]}{[b1]} $\\
B. $\lim_{x \rightarrow 0} \frac{[a]e^x\sin(x)-[a]x(1+x)}{[b]x^3} =  \infty $\\
C. $\lim_{x \rightarrow 0} \frac{[a]e^x\sin(x)-[a]x(1+x)}{[b]x^3} =  \frac{1}{[b1]} $\\
D. $\lim_{x \rightarrow 0} \frac{[a]e^x\sin(x)-[a]x(1+x)}{[b]x^3} =  1 $\\
E. $\lim_{x \rightarrow 0} \frac{[a]e^x\sin(x)-[a]x(1+x)}{[b]x^3} =  e $\\
F. $\lim_{x \rightarrow 0} \frac{[a]e^x\sin(x)-[a]x(1+x)}{[b]x^3} =  0 $
\testStop
\kluczStart
A
\kluczStop



\end{document}