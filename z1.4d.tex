\documentclass[12pt, a4paper]{article}
\usepackage[utf8]{inputenc}
\usepackage{polski}
\usepackage{amsthm}  %pakiet do tworzenia twierdzeń itp.
\usepackage{amsmath} %pakiet do niektórych symboli matematycznych
\usepackage{amssymb} %pakiet do symboli mat., np. \nsubseteq
\usepackage{amsfonts}
\usepackage{graphicx} %obsługa plików graficznych z rozszerzeniem png, jpg
\theoremstyle{definition} %styl dla definicji
\newtheorem{zad}{} 
\title{Multizestaw zadań}
\author{Patryk Wirkus}
%\date{\today}
\date{}
\newcommand{\kategoria}[1]{\section{#1}}
\newcommand{\zadStart}[1]{\begin{zad}#1\newline}
\newcommand{\zadStop}{\end{zad}}
\newcommand{\rozwStart}[2]{\noindent \textbf{Rozwiązanie (autor #1 , recenzent #2): }\newline}
\newcommand{\rozwStop}{\newline}                                           
\newcommand{\odpStart}{\noindent \textbf{Odpowiedź:}\newline}
\newcommand{\odpStop}{\newline}
\newcommand{\testStart}{\noindent \textbf{Test:}\newline}
\newcommand{\testStop}{\newline}
\newcommand{\kluczStart}{\noindent \textbf{Test poprawna odpowiedź:}\newline}
\newcommand{\kluczStop}{\newline}
\newcommand{\wstawGrafike}[2]{\begin{figure}[h] \includegraphics[scale=#2] {#1} \end{figure}}

\begin{document}
\maketitle

\kategoria{Wikieł/Z1.4d}


\zadStart{Zadanie z Wikieł Z 1.4 d) moja wersja nr 1}

Obliczyć wartość wyrażenia $\frac{6^{\frac{3}{4}}\cdot (\frac{4}{3})^{\frac{3}{4}}\cdot 2^{-\frac{3}{4}}}{3\sqrt{2}}$.
\zadStop
\rozwStart{Patryk Wirkus}{}
$$\frac{6^{\frac{3}{4}}\cdot (\frac{4}{3})^{\frac{3}{4}}\cdot 2^{-\frac{3}{4}}}{3\sqrt{2}} = \frac{4^{\frac{3}{4}}}{3\sqrt{2}} = \frac{\sqrt{8}}{3\sqrt{2}} = \frac{4}{3\cdot 2} = \frac{2}{3}$$
\rozwStop
\odpStart
$\frac{2}{3}$
\odpStop
\testStart
A.$\frac{2}{3}$\\ B.$-\frac{2}{3}$\\ C.$0$\\ D.$\frac{3}{2}$\\ E.$-\frac{3}{2}$
\testStop
\kluczStart
A
\kluczStop



\zadStart{Zadanie z Wikieł Z 1.4 d) moja wersja nr 2}

Obliczyć wartość wyrażenia $\frac{6^{\frac{3}{4}}\cdot (\frac{4}{3})^{\frac{3}{4}}\cdot 2^{-\frac{3}{4}}}{5\sqrt{2}}$.
\zadStop
\rozwStart{Patryk Wirkus}{}
$$\frac{6^{\frac{3}{4}}\cdot (\frac{4}{3})^{\frac{3}{4}}\cdot 2^{-\frac{3}{4}}}{5\sqrt{2}} = \frac{4^{\frac{3}{4}}}{5\sqrt{2}} = \frac{\sqrt{8}}{5\sqrt{2}} = \frac{4}{5\cdot 2} = \frac{2}{5}$$
\rozwStop
\odpStart
$\frac{2}{5}$
\odpStop
\testStart
A.$\frac{2}{5}$\\ B.$-\frac{2}{5}$\\ C.$0$\\ D.$\frac{5}{2}$\\ E.$-\frac{5}{2}$
\testStop
\kluczStart
A
\kluczStop



\zadStart{Zadanie z Wikieł Z 1.4 d) moja wersja nr 3}

Obliczyć wartość wyrażenia $\frac{6^{\frac{3}{4}}\cdot (\frac{4}{3})^{\frac{3}{4}}\cdot 2^{-\frac{3}{4}}}{7\sqrt{2}}$.
\zadStop
\rozwStart{Patryk Wirkus}{}
$$\frac{6^{\frac{3}{4}}\cdot (\frac{4}{3})^{\frac{3}{4}}\cdot 2^{-\frac{3}{4}}}{7\sqrt{2}} = \frac{4^{\frac{3}{4}}}{7\sqrt{2}} = \frac{\sqrt{8}}{7\sqrt{2}} = \frac{4}{7\cdot 2} = \frac{2}{7}$$
\rozwStop
\odpStart
$\frac{2}{7}$
\odpStop
\testStart
A.$\frac{2}{7}$\\ B.$-\frac{2}{7}$\\ C.$0$\\ D.$\frac{7}{2}$\\ E.$-\frac{7}{2}$
\testStop
\kluczStart
A
\kluczStop



\zadStart{Zadanie z Wikieł Z 1.4 d) moja wersja nr 4}

Obliczyć wartość wyrażenia $\frac{6^{\frac{3}{4}}\cdot (\frac{4}{3})^{\frac{3}{4}}\cdot 2^{-\frac{3}{4}}}{9\sqrt{2}}$.
\zadStop
\rozwStart{Patryk Wirkus}{}
$$\frac{6^{\frac{3}{4}}\cdot (\frac{4}{3})^{\frac{3}{4}}\cdot 2^{-\frac{3}{4}}}{9\sqrt{2}} = \frac{4^{\frac{3}{4}}}{9\sqrt{2}} = \frac{\sqrt{8}}{9\sqrt{2}} = \frac{4}{9\cdot 2} = \frac{2}{9}$$
\rozwStop
\odpStart
$\frac{2}{9}$
\odpStop
\testStart
A.$\frac{2}{9}$\\ B.$-\frac{2}{9}$\\ C.$0$\\ D.$\frac{9}{2}$\\ E.$-\frac{9}{2}$
\testStop
\kluczStart
A
\kluczStop



\zadStart{Zadanie z Wikieł Z 1.4 d) moja wersja nr 5}

Obliczyć wartość wyrażenia $\frac{6^{\frac{3}{4}}\cdot (\frac{4}{3})^{\frac{3}{4}}\cdot 2^{-\frac{3}{4}}}{11\sqrt{2}}$.
\zadStop
\rozwStart{Patryk Wirkus}{}
$$\frac{6^{\frac{3}{4}}\cdot (\frac{4}{3})^{\frac{3}{4}}\cdot 2^{-\frac{3}{4}}}{11\sqrt{2}} = \frac{4^{\frac{3}{4}}}{11\sqrt{2}} = \frac{\sqrt{8}}{11\sqrt{2}} = \frac{4}{11\cdot 2} = \frac{2}{11}$$
\rozwStop
\odpStart
$\frac{2}{11}$
\odpStop
\testStart
A.$\frac{2}{11}$\\ B.$-\frac{2}{11}$\\ C.$0$\\ D.$\frac{11}{2}$\\ E.$-\frac{11}{2}$
\testStop
\kluczStart
A
\kluczStop



\zadStart{Zadanie z Wikieł Z 1.4 d) moja wersja nr 6}

Obliczyć wartość wyrażenia $\frac{6^{\frac{3}{4}}\cdot (\frac{4}{3})^{\frac{3}{4}}\cdot 2^{-\frac{3}{4}}}{13\sqrt{2}}$.
\zadStop
\rozwStart{Patryk Wirkus}{}
$$\frac{6^{\frac{3}{4}}\cdot (\frac{4}{3})^{\frac{3}{4}}\cdot 2^{-\frac{3}{4}}}{13\sqrt{2}} = \frac{4^{\frac{3}{4}}}{13\sqrt{2}} = \frac{\sqrt{8}}{13\sqrt{2}} = \frac{4}{13\cdot 2} = \frac{2}{13}$$
\rozwStop
\odpStart
$\frac{2}{13}$
\odpStop
\testStart
A.$\frac{2}{13}$\\ B.$-\frac{2}{13}$\\ C.$0$\\ D.$\frac{13}{2}$\\ E.$-\frac{13}{2}$
\testStop
\kluczStart
A
\kluczStop



\zadStart{Zadanie z Wikieł Z 1.4 d) moja wersja nr 7}

Obliczyć wartość wyrażenia $\frac{6^{\frac{3}{4}}\cdot (\frac{4}{3})^{\frac{3}{4}}\cdot 2^{-\frac{3}{4}}}{15\sqrt{2}}$.
\zadStop
\rozwStart{Patryk Wirkus}{}
$$\frac{6^{\frac{3}{4}}\cdot (\frac{4}{3})^{\frac{3}{4}}\cdot 2^{-\frac{3}{4}}}{15\sqrt{2}} = \frac{4^{\frac{3}{4}}}{15\sqrt{2}} = \frac{\sqrt{8}}{15\sqrt{2}} = \frac{4}{15\cdot 2} = \frac{2}{15}$$
\rozwStop
\odpStart
$\frac{2}{15}$
\odpStop
\testStart
A.$\frac{2}{15}$\\ B.$-\frac{2}{15}$\\ C.$0$\\ D.$\frac{15}{2}$\\ E.$-\frac{15}{2}$
\testStop
\kluczStart
A
\kluczStop



\zadStart{Zadanie z Wikieł Z 1.4 d) moja wersja nr 8}

Obliczyć wartość wyrażenia $\frac{6^{\frac{3}{4}}\cdot (\frac{4}{3})^{\frac{3}{4}}\cdot 2^{-\frac{3}{4}}}{17\sqrt{2}}$.
\zadStop
\rozwStart{Patryk Wirkus}{}
$$\frac{6^{\frac{3}{4}}\cdot (\frac{4}{3})^{\frac{3}{4}}\cdot 2^{-\frac{3}{4}}}{17\sqrt{2}} = \frac{4^{\frac{3}{4}}}{17\sqrt{2}} = \frac{\sqrt{8}}{17\sqrt{2}} = \frac{4}{17\cdot 2} = \frac{2}{17}$$
\rozwStop
\odpStart
$\frac{2}{17}$
\odpStop
\testStart
A.$\frac{2}{17}$\\ B.$-\frac{2}{17}$\\ C.$0$\\ D.$\frac{17}{2}$\\ E.$-\frac{17}{2}$
\testStop
\kluczStart
A
\kluczStop



\zadStart{Zadanie z Wikieł Z 1.4 d) moja wersja nr 9}

Obliczyć wartość wyrażenia $\frac{6^{\frac{3}{4}}\cdot (\frac{4}{3})^{\frac{3}{4}}\cdot 2^{-\frac{3}{4}}}{19\sqrt{2}}$.
\zadStop
\rozwStart{Patryk Wirkus}{}
$$\frac{6^{\frac{3}{4}}\cdot (\frac{4}{3})^{\frac{3}{4}}\cdot 2^{-\frac{3}{4}}}{19\sqrt{2}} = \frac{4^{\frac{3}{4}}}{19\sqrt{2}} = \frac{\sqrt{8}}{19\sqrt{2}} = \frac{4}{19\cdot 2} = \frac{2}{19}$$
\rozwStop
\odpStart
$\frac{2}{19}$
\odpStop
\testStart
A.$\frac{2}{19}$\\ B.$-\frac{2}{19}$\\ C.$0$\\ D.$\frac{19}{2}$\\ E.$-\frac{19}{2}$
\testStop
\kluczStart
A
\kluczStop



\zadStart{Zadanie z Wikieł Z 1.4 d) moja wersja nr 10}

Obliczyć wartość wyrażenia $\frac{6^{\frac{3}{4}}\cdot (\frac{4}{3})^{\frac{3}{4}}\cdot 2^{-\frac{3}{4}}}{21\sqrt{2}}$.
\zadStop
\rozwStart{Patryk Wirkus}{}
$$\frac{6^{\frac{3}{4}}\cdot (\frac{4}{3})^{\frac{3}{4}}\cdot 2^{-\frac{3}{4}}}{21\sqrt{2}} = \frac{4^{\frac{3}{4}}}{21\sqrt{2}} = \frac{\sqrt{8}}{21\sqrt{2}} = \frac{4}{21\cdot 2} = \frac{2}{21}$$
\rozwStop
\odpStart
$\frac{2}{21}$
\odpStop
\testStart
A.$\frac{2}{21}$\\ B.$-\frac{2}{21}$\\ C.$0$\\ D.$\frac{21}{2}$\\ E.$-\frac{21}{2}$
\testStop
\kluczStart
A
\kluczStop



\zadStart{Zadanie z Wikieł Z 1.4 d) moja wersja nr 11}

Obliczyć wartość wyrażenia $\frac{6^{\frac{3}{4}}\cdot (\frac{4}{3})^{\frac{3}{4}}\cdot 2^{-\frac{3}{4}}}{23\sqrt{2}}$.
\zadStop
\rozwStart{Patryk Wirkus}{}
$$\frac{6^{\frac{3}{4}}\cdot (\frac{4}{3})^{\frac{3}{4}}\cdot 2^{-\frac{3}{4}}}{23\sqrt{2}} = \frac{4^{\frac{3}{4}}}{23\sqrt{2}} = \frac{\sqrt{8}}{23\sqrt{2}} = \frac{4}{23\cdot 2} = \frac{2}{23}$$
\rozwStop
\odpStart
$\frac{2}{23}$
\odpStop
\testStart
A.$\frac{2}{23}$\\ B.$-\frac{2}{23}$\\ C.$0$\\ D.$\frac{23}{2}$\\ E.$-\frac{23}{2}$
\testStop
\kluczStart
A
\kluczStop



\zadStart{Zadanie z Wikieł Z 1.4 d) moja wersja nr 12}

Obliczyć wartość wyrażenia $\frac{6^{\frac{3}{4}}\cdot (\frac{4}{3})^{\frac{3}{4}}\cdot 2^{-\frac{3}{4}}}{25\sqrt{2}}$.
\zadStop
\rozwStart{Patryk Wirkus}{}
$$\frac{6^{\frac{3}{4}}\cdot (\frac{4}{3})^{\frac{3}{4}}\cdot 2^{-\frac{3}{4}}}{25\sqrt{2}} = \frac{4^{\frac{3}{4}}}{25\sqrt{2}} = \frac{\sqrt{8}}{25\sqrt{2}} = \frac{4}{25\cdot 2} = \frac{2}{25}$$
\rozwStop
\odpStart
$\frac{2}{25}$
\odpStop
\testStart
A.$\frac{2}{25}$\\ B.$-\frac{2}{25}$\\ C.$0$\\ D.$\frac{25}{2}$\\ E.$-\frac{25}{2}$
\testStop
\kluczStart
A
\kluczStop



\zadStart{Zadanie z Wikieł Z 1.4 d) moja wersja nr 13}

Obliczyć wartość wyrażenia $\frac{6^{\frac{3}{4}}\cdot (\frac{4}{3})^{\frac{3}{4}}\cdot 2^{-\frac{3}{4}}}{27\sqrt{2}}$.
\zadStop
\rozwStart{Patryk Wirkus}{}
$$\frac{6^{\frac{3}{4}}\cdot (\frac{4}{3})^{\frac{3}{4}}\cdot 2^{-\frac{3}{4}}}{27\sqrt{2}} = \frac{4^{\frac{3}{4}}}{27\sqrt{2}} = \frac{\sqrt{8}}{27\sqrt{2}} = \frac{4}{27\cdot 2} = \frac{2}{27}$$
\rozwStop
\odpStart
$\frac{2}{27}$
\odpStop
\testStart
A.$\frac{2}{27}$\\ B.$-\frac{2}{27}$\\ C.$0$\\ D.$\frac{27}{2}$\\ E.$-\frac{27}{2}$
\testStop
\kluczStart
A
\kluczStop



\zadStart{Zadanie z Wikieł Z 1.4 d) moja wersja nr 14}

Obliczyć wartość wyrażenia $\frac{6^{\frac{3}{4}}\cdot (\frac{4}{3})^{\frac{3}{4}}\cdot 2^{-\frac{3}{4}}}{29\sqrt{2}}$.
\zadStop
\rozwStart{Patryk Wirkus}{}
$$\frac{6^{\frac{3}{4}}\cdot (\frac{4}{3})^{\frac{3}{4}}\cdot 2^{-\frac{3}{4}}}{29\sqrt{2}} = \frac{4^{\frac{3}{4}}}{29\sqrt{2}} = \frac{\sqrt{8}}{29\sqrt{2}} = \frac{4}{29\cdot 2} = \frac{2}{29}$$
\rozwStop
\odpStart
$\frac{2}{29}$
\odpStop
\testStart
A.$\frac{2}{29}$\\ B.$-\frac{2}{29}$\\ C.$0$\\ D.$\frac{29}{2}$\\ E.$-\frac{29}{2}$
\testStop
\kluczStart
A
\kluczStop





\end{document}
