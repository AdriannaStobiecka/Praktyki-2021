\documentclass[12pt, a4paper]{article}
\usepackage[utf8]{inputenc}
\usepackage{polski}

\usepackage{amsthm}  %pakiet do tworzenia twierdzeń itp.
\usepackage{amsmath} %pakiet do niektórych symboli matematycznych
\usepackage{amssymb} %pakiet do symboli mat., np. \nsubseteq
\usepackage{amsfonts}
\usepackage{graphicx} %obsługa plików graficznych z rozszerzeniem png, jpg
\theoremstyle{definition} %styl dla definicji
\newtheorem{zad}{} 
\title{Multizestaw zadań}
\author{Robert Fidytek}
%\date{\today}
\date{}
\newcounter{liczniksekcji}
\newcommand{\kategoria}[1]{\section{#1}} %olreślamy nazwę kateforii zadań
\newcommand{\zadStart}[1]{\begin{zad}#1\newline} %oznaczenie początku zadania
\newcommand{\zadStop}{\end{zad}}   %oznaczenie końca zadania
%Makra opcjonarne (nie muszą występować):
\newcommand{\rozwStart}[2]{\noindent \textbf{Rozwiązanie (autor #1 , recenzent #2): }\newline} %oznaczenie początku rozwiązania, opcjonarnie można wprowadzić informację o autorze rozwiązania zadania i recenzencie poprawności wykonania rozwiązania zadania
\newcommand{\rozwStop}{\newline}                                            %oznaczenie końca rozwiązania
\newcommand{\odpStart}{\noindent \textbf{Odpowiedź:}\newline}    %oznaczenie początku odpowiedzi końcowej (wypisanie wyniku)
\newcommand{\odpStop}{\newline}                                             %oznaczenie końca odpowiedzi końcowej (wypisanie wyniku)
\newcommand{\testStart}{\noindent \textbf{Test:}\newline} %ewentualne możliwe opcje odpowiedzi testowej: A. ? B. ? C. ? D. ? itd.
\newcommand{\testStop}{\newline} %koniec wprowadzania odpowiedzi testowych
\newcommand{\kluczStart}{\noindent \textbf{Test poprawna odpowiedź:}\newline} %klucz, poprawna odpowiedź pytania testowego (jedna literka): A lub B lub C lub D itd.
\newcommand{\kluczStop}{\newline} %koniec poprawnej odpowiedzi pytania testowego 
\newcommand{\wstawGrafike}[2]{\begin{figure}[h] \includegraphics[scale=#2] {#1} \end{figure}} %gdyby była potrzeba wstawienia obrazka, parametry: nazwa pliku, skala (jak nie wiesz co wpisać, to wpisz 1)

\begin{document}
\maketitle



\kategoria{Wikieł/P1.20a}
\zadStart{Zadanie z Wikieł P 1.20 a) moja wersja nr [nrWersji]}
%[p1]:[2,3,4,5,6,7,8]
%[p2]:[2,3,4,5,6]
%[p3]:[4,5,6,7,8,9,10,11]
%[p4]:[2,3,4,5,6]
%[a]=random.randint(2,10)
%[e]=random.randint(2,10)
%[c]=random.randint(1,10)
%[d]=random.randint(2,10)
%[b]=random.randint(2,10)
%[f]=random.randint(1,10)
%[g]=2*[a]
%[p3p1m]=[p3]-[p1]
%[p3p2m]=[p3]-[p2]
%[p3p4m]=[p3]-[p4]
%[p3]>[p1] and [p3]>[p4] and [p1]>[p2] and math.gcd([a],[d])==1 and [p3p1m]>1 and [p3p2m]>1 and [p3p4m]>1
Zbadać parzystość (nieparzystość) funkcji $f(x)=\sqrt[3]{(x-[a])^2}-\sqrt[3]{(x+[a])^2}$.
\zadStop
\rozwStart{Pascal Nawrocki}{}
Funkcja jest parzysta jeżeli w całej swojej dziedznie spełnia warunek, że $f(x)=-f(x)$. Natomiast nieparzysta jeżeli spełnia warunek, że $f(-x)=-f(x)$.
Sprawdzamy dziedzinę, jako że pod pierwiastkami występują wyrażenia kwadratowe, to oznacza, że dziedzina naszej funkcji równa jest $ \mathbb{R}$.
Sprawdźmy:
$$f(-x)=\sqrt[3]{(-x-[a])^2}-\sqrt[3]{(x+[a])^2}$$
Korzystamy z własności potęgi:
$$(-x-[a])^2=[-(x+[a])]^2=(x+[a])^2$$
$$(-x+[a])^2=[-(x-[a])]^2=(x-[a])^2$$
Dokańczamy:
\begin{equation}
\begin{split}
f(-x)=&\sqrt[3]{(-x-[a])^2}-\sqrt[3]{(x+[a])^2}=\sqrt[3]{(x+[a])^2}-\sqrt[3]{(x-[a])^2}=\\
&=-[\sqrt[3]{(x-[a])^2}-\sqrt[3]{(x+[a])^2}]=-f(x)
\end{split}
\end{equation}
Zatem funkcja jest nieparzysta, gdyż w całej swojej dziedznie spełnia warunek, że $f(-x)=-f(x)$.
\rozwStop
\odpStart
 Funkcja jest nieparzysta
\odpStop
\testStart
A.Funkcja jest nieparzysta
B.Funkcja jest parzysta
C.Funkcja nie jest parzysta ani nieparzysta
\testStop
\kluczStart
A
\kluczStop


\end{document}