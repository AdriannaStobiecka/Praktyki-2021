\documentclass[12pt, a4paper]{article}
\usepackage[utf8]{inputenc}
\usepackage{polski}

\usepackage{amsthm}  %pakiet do tworzenia twierdzeń itp.
\usepackage{amsmath} %pakiet do niektórych symboli matematycznych
\usepackage{amssymb} %pakiet do symboli mat., np. \nsubseteq
\usepackage{amsfonts}
\usepackage{graphicx} %obsługa plików graficznych z rozszerzeniem png, jpg
\theoremstyle{definition} %styl dla definicji
\newtheorem{zad}{} 
\title{Multizestaw zadań}
\author{Robert Fidytek}
%\date{\today}
\date{}
\newcounter{liczniksekcji}
\newcommand{\kategoria}[1]{\section{#1}} %olreślamy nazwę kateforii zadań
\newcommand{\zadStart}[1]{\begin{zad}#1\newline} %oznaczenie początku zadania
\newcommand{\zadStop}{\end{zad}}   %oznaczenie końca zadania
%Makra opcjonarne (nie muszą występować):
\newcommand{\rozwStart}[2]{\noindent \textbf{Rozwiązanie (autor #1 , recenzent #2): }\newline} %oznaczenie początku rozwiązania, opcjonarnie można wprowadzić informację o autorze rozwiązania zadania i recenzencie poprawności wykonania rozwiązania zadania
\newcommand{\rozwStop}{\newline}                                            %oznaczenie końca rozwiązania
\newcommand{\odpStart}{\noindent \textbf{Odpowiedź:}\newline}    %oznaczenie początku odpowiedzi końcowej (wypisanie wyniku)
\newcommand{\odpStop}{\newline}                                             %oznaczenie końca odpowiedzi końcowej (wypisanie wyniku)
\newcommand{\testStart}{\noindent \textbf{Test:}\newline} %ewentualne możliwe opcje odpowiedzi testowej: A. ? B. ? C. ? D. ? itd.
\newcommand{\testStop}{\newline} %koniec wprowadzania odpowiedzi testowych
\newcommand{\kluczStart}{\noindent \textbf{Test poprawna odpowiedź:}\newline} %klucz, poprawna odpowiedź pytania testowego (jedna literka): A lub B lub C lub D itd.
\newcommand{\kluczStop}{\newline} %koniec poprawnej odpowiedzi pytania testowego 
\newcommand{\wstawGrafike}[2]{\begin{figure}[h] \includegraphics[scale=#2] {#1} \end{figure}} %gdyby była potrzeba wstawienia obrazka, parametry: nazwa pliku, skala (jak nie wiesz co wpisać, to wpisz 1)

\begin{document}
\maketitle


\kategoria{Wikieł/Z1.70k}
\zadStart{Zadanie z Wikieł Z 1.70 k)  moja wersja nr [nrWersji]}
%[p1]:[2,3,4,5,6,7,8,9,10]
%[p2]:[2,3,4,5,6,7,8,9,10]
%[p3]:[2,3,4,5,6,7,8,9,10]
%[p4]=random.randint(2,10)
%[a]=[p4]*[p4]
%[del1]=[p1]*[p1]+4*[p4]
%[pdel1]=round(math.sqrt([del1]),2)
%[x1]=round((-[p1]-[pdel1])/2,2)
%[x2]=round((-[p1]+[pdel1])/2,2)
%[del2]=[p1]*[p1]-4*[p4]
%[del1]>0 and [del2]<0 and math.gcd([p3],[p2])==1 and [x1]<-[p1] and [p1]>([p3]/[p2]) 


Rozwiązać nierówność $$(x^{2}+[p1]x)([p2]x+[p3])-[a]\frac{[p2]x+[p3]}{x^{2}+[p1]x}\geq0.$$

\zadStop

\rozwStart{Maja Szabłowska}{}
$$(x^{2}+[p1]x)([p2]x+[p3])-[a]\frac{[p2]x+[p3]}{x^{2}+[p1]x})\geq0$$
$$(x^{2}+[p1]x)^{3}([p2]x+[p3])-[a]([p2]x+[p3])(x^{2}+[p1]x)\geq 0$$
$$(x^{2}+[p1]x)[(x^{2}+[p1]x)^{2}([p2]x+[p3])-[a]([p2]x+[p3])]\geq 0$$
$$(x^{2}+[p1]x)[([p2]x+[p3])((x^{2}+[p1]x)^{2}-[a])]\geq 0$$
$$(x^{2}+[p1]x)([p2]x+[p3])(x^{2}+[p1]x-[p4])(x^{2}+[p1]x+[p4])\geq 0$$
$$x(x+[p1])([p2]x+[p3])(x^{2}+[p1]x-[p4])(x^{2}+[p1]x+[p4])\geq 0$$
$$\Delta_{1}=[p1]^{2}-4\cdot1\cdot(-[p4])=[del1] \Rightarrow \sqrt{\Delta_{1}}=[pdel1]$$
$$x_{1}=\frac{-[p1]-[pdel1]}{2}=[x1], \quad x_{2}=\frac{-[p1]+[pdel1]}{2}=[x2]$$
$$\Delta_{2}=[p1]^{2}-4\cdot 1 \cdot [p4] =[del2]<0$$
Miejsca zerowe w nierówności to: $ 0,-[p1], -\frac{[p3]}{[p2]}, [x1], [x2].$
Rozwiązaniem nierówności zatem jest zbiór:
$$x\in[[x1],-[p1]]\cup\left[-\frac{[p3]}{[p2]},0\right]\cup[[x2],\infty)$$
\rozwStop


\odpStart
$x\in[[x1],-[p1]]\cup\left[-\frac{[p3]}{[p2]},0\right]\cup[[x2],\infty)$
\odpStop
\testStart
A.$x\in[[x1],-[p1]]\cup\left[-\frac{[p3]}{[p2]},0\right]\cup[[x2],\infty)$
B.$x\in[[x1],-[p1]]\cup[[x2],\infty)$
D.$x\in[[x2],\infty)$
E.$x\in[[x1],-[p1]]$
F.$x\in[[del2],-[p2]]$
G.$x\in\mathbb{R}$
H.$x\in\emptyset$
\testStop
\kluczStart
A
\kluczStop



\end{document}
