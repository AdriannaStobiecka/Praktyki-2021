\documentclass[12pt, a4paper]{article}
\usepackage[utf8]{inputenc}
\usepackage{polski}

\usepackage{amsthm}  %pakiet do tworzenia twierdzeń itp.
\usepackage{amsmath} %pakiet do niektórych symboli matematycznych
\usepackage{amssymb} %pakiet do symboli mat., np. \nsubseteq
\usepackage{amsfonts}
\usepackage{graphicx} %obsługa plików graficznych z rozszerzeniem png, jpg
\theoremstyle{definition} %styl dla definicji
\newtheorem{zad}{} 
\title{Multizestaw zadań}
\author{Robert Fidytek}
%\date{\today}
\date{}
\newcounter{liczniksekcji}
\newcommand{\kategoria}[1]{\section{#1}} %olreślamy nazwę kateforii zadań
\newcommand{\zadStart}[1]{\begin{zad}#1\newline} %oznaczenie początku zadania
\newcommand{\zadStop}{\end{zad}}   %oznaczenie końca zadania
%Makra opcjonarne (nie muszą występować):
\newcommand{\rozwStart}[2]{\noindent \textbf{Rozwiązanie (autor #1 , recenzent #2): }\newline} %oznaczenie początku rozwiązania, opcjonarnie można wprowadzić informację o autorze rozwiązania zadania i recenzencie poprawności wykonania rozwiązania zadania
\newcommand{\rozwStop}{\newline}                                            %oznaczenie końca rozwiązania
\newcommand{\odpStart}{\noindent \textbf{Odpowiedź:}\newline}    %oznaczenie początku odpowiedzi końcowej (wypisanie wyniku)
\newcommand{\odpStop}{\newline}                                             %oznaczenie końca odpowiedzi końcowej (wypisanie wyniku)
\newcommand{\testStart}{\noindent \textbf{Test:}\newline} %ewentualne możliwe opcje odpowiedzi testowej: A. ? B. ? C. ? D. ? itd.
\newcommand{\testStop}{\newline} %koniec wprowadzania odpowiedzi testowych
\newcommand{\kluczStart}{\noindent \textbf{Test poprawna odpowiedź:}\newline} %klucz, poprawna odpowiedź pytania testowego (jedna literka): A lub B lub C lub D itd.
\newcommand{\kluczStop}{\newline} %koniec poprawnej odpowiedzi pytania testowego 
\newcommand{\wstawGrafike}[2]{\begin{figure}[h] \includegraphics[scale=#2] {#1} \end{figure}} %gdyby była potrzeba wstawienia obrazka, parametry: nazwa pliku, skala (jak nie wiesz co wpisać, to wpisz 1)

\begin{document}
\maketitle


\kategoria{Wikieł/Z3.14e}
\zadStart{Zadanie z Wikieł Z 3.14 e) moja wersja nr [nrWersji]}
%[f]:[14,15,16,17,18,19,20,21]
%[z]:[1,2,3,4,5,9,10,11,12,13,14,15,16,17]
%[a]=random.randint(6,200)
%[b]=[a]-1
%[c]=[b]-1
%[d]=[c]-1
%[e]=[d]-1
Obliczyć granicę ciągu $a_n= {([a]^n + [b]^n + [c]^n + [d]^n +[e]^n)}^{\frac{1}{n}}$.
\zadStop
\rozwStart{Barbara Bączek}{}
$$\lim_{n \rightarrow \infty} a_n= \lim_{n \rightarrow \infty} {([a]^n + [b]^n + [c]^n + [d]^n +[e]^n)}^{\frac{1}{n}}=$$
$$ \lim_{n \rightarrow \infty} {\Big{(}[a]^n\Big{(} 1 + {\Big{(}\frac{[b]}{[a]}\Big{)}}^n +  {\Big{(}\frac{[c]}{[a]}\Big{)}}^n +  {\Big{(}\frac{[d]}{[a]}\Big{)}}^n + {\Big{(}\frac{[e]}{[a]}\Big{)}}^n \Big{)}\Big{)}}^{\frac{1}{n}}= $$

$$ \lim_{n \rightarrow \infty} \Big{(}[a] \cdot {\Big{(} 1 + {\Big{(}\frac{[b]}{[a]}\Big{)}}^n +  {\Big{(}\frac{[c]}{[a]}\Big{)}}^n +  {\Big{(}\frac{[d]}{[a]}\Big{)}}^n + {\Big{(}\frac{[e]}{[a]}\Big{)}}^n \Big{)}}^{\frac{1}{n}} \Big{)}=$$
$$ [a] \cdot \lim_{n \rightarrow \infty} {\Big{(} 1 + {\Big{(}\frac{[b]}{[a]}\Big{)}}^n +  {\Big{(}\frac{[c]}{[a]}\Big{)}}^n +  {\Big{(}\frac{[d]}{[a]}\Big{)}}^n + {\Big{(}\frac{[e]}{[a]}\Big{)}}^n \Big{)}}^{\frac{1}{n}}= [a]$$

Podsumowując: $$\lim_{n \rightarrow \infty} {([a]^n + [b]^n + [c]^n + [d]^n +[e]^n)}^{\frac{1}{n}} = [a] $$
\rozwStop
\odpStart
$[a]$
\odpStop
\testStart
A.$\infty$
B.$[a]$
C.$-\infty$
D.$0$
E.$[d]$
G.$[b]$
H.$[c]$
\testStop
\kluczStart
B
\kluczStop



\end{document}