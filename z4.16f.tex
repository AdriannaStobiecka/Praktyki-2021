\documentclass[12pt, a4paper]{article}
\usepackage[utf8]{inputenc}
\usepackage{polski}

\usepackage{amsthm}  %pakiet do tworzenia twierdzeń itp.
\usepackage{amsmath} %pakiet do niektórych symboli matematycznych
\usepackage{amssymb} %pakiet do symboli mat., np. \nsubseteq
\usepackage{amsfonts}
\usepackage{graphicx} %obsługa plików graficznych z rozszerzeniem png, jpg
\theoremstyle{definition} %styl dla definicji
\newtheorem{zad}{} 
\title{Multizestaw zadań}
\author{Robert Fidytek}
%\date{\today}
\date{}
\newcounter{liczniksekcji}
\newcommand{\kategoria}[1]{\section{#1}} %olreślamy nazwę kateforii zadań
\newcommand{\zadStart}[1]{\begin{zad}#1\newline} %oznaczenie początku zadania
\newcommand{\zadStop}{\end{zad}}   %oznaczenie końca zadania
%Makra opcjonarne (nie muszą występować):
\newcommand{\rozwStart}[2]{\noindent \textbf{Rozwiązanie (autor #1 , recenzent #2): }\newline} %oznaczenie początku rozwiązania, opcjonarnie można wprowadzić informację o autorze rozwiązania zadania i recenzencie poprawności wykonania rozwiązania zadania
\newcommand{\rozwStop}{\newline}                                            %oznaczenie końca rozwiązania
\newcommand{\odpStart}{\noindent \textbf{Odpowiedź:}\newline}    %oznaczenie początku odpowiedzi końcowej (wypisanie wyniku)
\newcommand{\odpStop}{\newline}                                             %oznaczenie końca odpowiedzi końcowej (wypisanie wyniku)
\newcommand{\testStart}{\noindent \textbf{Test:}\newline} %ewentualne możliwe opcje odpowiedzi testowej: A. ? B. ? C. ? D. ? itd.
\newcommand{\testStop}{\newline} %koniec wprowadzania odpowiedzi testowych
\newcommand{\kluczStart}{\noindent \textbf{Test poprawna odpowiedź:}\newline} %klucz, poprawna odpowiedź pytania testowego (jedna literka): A lub B lub C lub D itd.
\newcommand{\kluczStop}{\newline} %koniec poprawnej odpowiedzi pytania testowego 
\newcommand{\wstawGrafike}[2]{\begin{figure}[h] \includegraphics[scale=#2] {#1} \end{figure}} %gdyby była potrzeba wstawienia obrazka, parametry: nazwa pliku, skala (jak nie wiesz co wpisać, to wpisz 1)

\begin{document}
\maketitle


\kategoria{Wikieł/Z4.16f}
\zadStart{Zadanie z Wikieł Z 4.16 f) moja wersja nr [nrWersji]}
%[b]:[2,3,4,5,6,7,8,9,10,11,12,13,14,15,16,17,18,19,20,21,22,23,24,25,26,27,28,29,30,31,32,33,34,35,36,37,38,39,40,41,42,43,44,45]
%[a]=0
%[bb]=2*[b]
%[c]=[bb]-1
%[cd]=[c]+[bb]
%[dc]=1
%[i]=[cd]+1 
%[bb1]=[i]/2
%[bb3]=int([bb1])
Zbadać, czy istnieje $\lim_{x\rightarrow x_{0}}f(x)$. Jeśli tak, to obliczyć tę granicę $$x_{0}=[a],f(x)= \left\{ \begin{array}{ll}
\frac{1-cos[bb]x}{[b]x^2} & \textrm{dla $x<[a]$}\\
\frac{cos[c]x-cos[bb]x}{x^2}+\frac{1}{2} & \textrm{dla $x>[a]$}
\end{array} \right.$$
\zadStop
\rozwStart{Aleksandra Pasińska}{}
Korzystamy z twierdzenia de l'Hospitala:\\
$$\lim_{x\rightarrow [a]^-}\frac{1-cos[bb]x}{[b]x^2}=\lim_{x\rightarrow [a]^-}\frac{\frac{d}{dx}(1-cos[bb]x)}{\frac{d}{dx}([b]x^2)}=$$
$$=\lim_{x\rightarrow [a]^-}\frac{[bb]sin([bb]x)}{[bb]x}=\lim_{x\rightarrow [a]^-}\frac{\frac{d}{dx}(sin([bb]x))}{\frac{d}{dx}(x)}=$$
$$=\lim_{x\rightarrow [a]^-}[bb]cos([bb]x)=[bb]cos([bb]\cdot [a])=[bb]$$
$$\lim_{x\rightarrow [a]^+}\biggl(\frac{cos([c]x)-cos([bb]x)}{x^2}+\frac{1}{2}\biggr)=(1)\lim_{x\rightarrow [a]^+}\frac{cos([c]x)-cos([bb]x)}{x^2}+(2)\lim_{x\rightarrow [a]^+}\frac{1}{2}$$
$$(1)\lim_{x\rightarrow [a]^+}\biggl(\frac{-2sin(\frac{[cd]x}{2})sin(-\frac{[dc]x}{2})}{x^2}\biggr)=\lim_{x\rightarrow [a]^+}\biggl(\frac{-2sin(\frac{[cd]x}{2})}{x}\biggr)\cdot \lim_{x\rightarrow [a]^+}\biggl(\frac{\frac{d}{dx}sin(-\frac{[dc]x}{2})}{\frac{d}{dx}x}\biggr)=$$
$$=-2\cdot \lim_{x\rightarrow [a]^+}\biggl(sin(\frac{[cd]x}{2})\cdot \frac{1}{x}\biggr)\cdot \lim_{x\rightarrow [a]^+}\biggl(-\frac{cos(\frac{x}{2})}{2}\biggr)=-2\cdot \lim_{x\rightarrow [a]^+}\biggl(\frac{\frac{d}{dx}sin(\frac{[cd]x}{2})}{\frac{d}{dx}\frac{1}{\frac{1}{x}}}\biggr)\cdot \biggl(-\frac{cos(\frac{[a]}{2})}{2}\biggr)=$$ $$=-2\cdot \lim_{x\rightarrow [a]^+}\biggl(\frac{[cd]cos(\frac{[cd]x}{2})}{2}\biggr)\cdot \biggl(-\frac{1}{2}\biggr)=\frac{[cd]}{2}$$
$$\lim_{x\rightarrow [a]^+}\biggl(\frac{cos([c]x)-cos([bb]x)}{x^2}+\frac{1}{2}\biggr)=(1)+(2)=\frac{[cd]}{2}+\frac{1}{2}=[bb3]$$
\rozwStop
\odpStart
[bb]\\
\odpStop
\testStart
A.$[bb]$
B.$\infty$
C.$-\infty$
D.$1$
E.$9$
F.$e$
G.$4$
H.$7$
I.$-7$
\testStop
\kluczStart
A
\kluczStop



\end{document}