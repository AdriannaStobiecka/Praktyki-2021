\documentclass[12pt, a4paper]{article}
\usepackage[utf8]{inputenc}
\usepackage{polski}

\usepackage{amsthm}  %pakiet do tworzenia twierdzeń itp.
\usepackage{amsmath} %pakiet do niektórych symboli matematycznych
\usepackage{amssymb} %pakiet do symboli mat., np. \nsubseteq
\usepackage{amsfonts}
\usepackage{graphicx} %obsługa plików graficznych z rozszerzeniem png, jpg
\theoremstyle{definition} %styl dla definicji
\newtheorem{zad}{} 
\title{Multizestaw zadań}
\author{Robert Fidytek}
%\date{\today}
\date{}
\newcounter{liczniksekcji}
\newcommand{\kategoria}[1]{\section{#1}} %olreślamy nazwę kateforii zadań
\newcommand{\zadStart}[1]{\begin{zad}#1\newline} %oznaczenie początku zadania
\newcommand{\zadStop}{\end{zad}}   %oznaczenie końca zadania
%Makra opcjonarne (nie muszą występować):
\newcommand{\rozwStart}[2]{\noindent \textbf{Rozwiązanie (autor #1 , recenzent #2): }\newline} %oznaczenie początku rozwiązania, opcjonarnie można wprowadzić informację o autorze rozwiązania zadania i recenzencie poprawności wykonania rozwiązania zadania
\newcommand{\rozwStop}{\newline}                                            %oznaczenie końca rozwiązania
\newcommand{\odpStart}{\noindent \textbf{Odpowiedź:}\newline}    %oznaczenie początku odpowiedzi końcowej (wypisanie wyniku)
\newcommand{\odpStop}{\newline}                                             %oznaczenie końca odpowiedzi końcowej (wypisanie wyniku)
\newcommand{\testStart}{\noindent \textbf{Test:}\newline} %ewentualne możliwe opcje odpowiedzi testowej: A. ? B. ? C. ? D. ? itd.
\newcommand{\testStop}{\newline} %koniec wprowadzania odpowiedzi testowych
\newcommand{\kluczStart}{\noindent \textbf{Test poprawna odpowiedź:}\newline} %klucz, poprawna odpowiedź pytania testowego (jedna literka): A lub B lub C lub D itd.
\newcommand{\kluczStop}{\newline} %koniec poprawnej odpowiedzi pytania testowego 
\newcommand{\wstawGrafike}[2]{\begin{figure}[h] \includegraphics[scale=#2] {#1} \end{figure}} %gdyby była potrzeba wstawienia obrazka, parametry: nazwa pliku, skala (jak nie wiesz co wpisać, to wpisz 1)

\begin{document}
\maketitle


\kategoria{Wikieł/P1.34a}
\zadStart{Zadanie z Wikieł P 1.34a) moja wersja nr [nrWersji]}
%[p1]:[2,3,4,5,6,7,8,9,10,11,12,13,14,15,16,17,18,19,20]
%[p2]:[2,3,4,5,6,7,8,9,10,11,12,13,14,15,16,17,18,19,20]
%[p12]=[p1]*[p1]
%[p22]=[p2]*[p2]
%math.gcd([p1],[p2])==1
Rozwiązać nierówności $[p12]x^2-[p22]\leq 0$
\zadStop
\rozwStart{Jakub Janik}{}
$$[p12]x^2-[p22]\leq 0 \Leftrightarrow x^2 \leq \frac{[p22]}{[p12]}$$
$$|x| \leq \frac{[p2]}{[p1]} \Leftrightarrow -\frac{[p2]}{[p1]} \leq x \leq \frac{[p2]}{[p1]}$$
\rozwStop
\odpStart
$-\frac{[p2]}{[p1]} \leq x \leq \frac{[p2]}{[p1]}$
\odpStop
\testStart
A.$-\frac{[p2]}{[p1]} \leq x \leq \frac{[p2]}{[p1]}$
B.$-\frac{[p1]}{[p2]} \leq x \leq \frac{[p1]}{[p2]}$
C.$-\frac{[p22]}{[p12]} \leq x \leq \frac{[p22]}{[p12]}$
D.$-\frac{[p12]}{[p22]} \leq x \leq \frac{[p12]}{[p22]}$
\testStop
\kluczStart
A
\kluczStop



\end{document}