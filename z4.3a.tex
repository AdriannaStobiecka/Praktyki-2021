\documentclass[12pt, a4paper]{article}
\usepackage[utf8]{inputenc}
\usepackage{polski}
\usepackage{amsthm}  %pakiet do tworzenia twierdzeń itp.
\usepackage{amsmath} %pakiet do niektórych symboli matematycznych
\usepackage{amssymb} %pakiet do symboli mat., np. \nsubseteq
\usepackage{amsfonts}
\usepackage{graphicx} %obsługa plików graficznych z rozszerzeniem png, jpg
\theoremstyle{definition} %styl dla definicji
\newtheorem{zad}{} 
\title{Multizestaw zadań}
\author{Robert Fidytek}
%\date{\today}
\date{}
\newcommand{\kategoria}[1]{\section{#1}}
\newcommand{\zadStart}[1]{\begin{zad}#1\newline}
\newcommand{\zadStop}{\end{zad}}
\newcommand{\rozwStart}[2]{\noindent \textbf{Rozwiązanie (autor #1 , recenzent #2): }\newline}
\newcommand{\rozwStop}{\newline}                                           
\newcommand{\odpStart}{\noindent \textbf{Odpowiedź:}\newline}
\newcommand{\odpStop}{\newline}
\newcommand{\testStart}{\noindent \textbf{Test:}\newline}
\newcommand{\testStop}{\newline}
\newcommand{\kluczStart}{\noindent \textbf{Test poprawna odpowiedź:}\newline}
\newcommand{\kluczStop}{\newline}
\newcommand{\wstawGrafike}[2]{\begin{figure}[h] \includegraphics[scale=#2] {#1} \end{figure}}

\begin{document}
\maketitle

\kategoria{Wikieł/Z4.3a}


\zadStart{Zadanie z Wikieł Z 4.3 a) moja wersja nr 1}
Obliczyć granicę funkcji $f(x)=\frac{\sqrt{x+1}-1}{x}$.
\zadStop
\rozwStart{Patryk Wirkus}{}
$$\frac{\sqrt{x+1}-1}{x}=\frac{(\sqrt{x+1}-1)(\sqrt{x+1}+1)}{x(\sqrt{x+1}+1)}=\frac{1}{x(\sqrt{x+1}+1)}$$
\\
$$\lim\limits_{x\to0}\frac{\sqrt{x+1}-1}{x}=[\frac{0}{0}]=
\lim\limits_{x\to0}\frac{1}{x(\sqrt{x+1}+1)} = \frac{1}{\sqrt{1}+1}$$
\rozwStop
\odpStart
$\frac{1}{\sqrt{1}+1}$
\odpStop
\testStart
A.$\frac{1}{\sqrt{1}+1}$
B.$\frac{1}{\sqrt{1}+1}$
C.$0$
D.$\sqrt{1}+1$
E.$\infty$
F.$-\infty$
G.$\sqrt{1}-1$
H.$-1$
I.$1$
\testStop
\kluczStart
A
\kluczStop



\zadStart{Zadanie z Wikieł Z 4.3 a) moja wersja nr 2}
Obliczyć granicę funkcji $f(x)=\frac{\sqrt{x+4}-4}{x}$.
\zadStop
\rozwStart{Patryk Wirkus}{}
$$\frac{\sqrt{x+4}-4}{x}=\frac{(\sqrt{x+4}-4)(\sqrt{x+4}+4)}{x(\sqrt{x+4}+4)}=\frac{1}{x(\sqrt{x+4}+4)}$$
\\
$$\lim\limits_{x\to0}\frac{\sqrt{x+4}-4}{x}=[\frac{0}{0}]=
\lim\limits_{x\to0}\frac{1}{x(\sqrt{x+4}+4)} = \frac{1}{\sqrt{4}+4}$$
\rozwStop
\odpStart
$\frac{1}{\sqrt{4}+4}$
\odpStop
\testStart
A.$\frac{1}{\sqrt{4}+4}$
B.$\frac{4}{\sqrt{4}+4}$
C.$0$
D.$\sqrt{4}+4$
E.$\infty$
F.$-\infty$
G.$\sqrt{4}-4$
H.$-4$
I.$4$
\testStop
\kluczStart
A
\kluczStop



\zadStart{Zadanie z Wikieł Z 4.3 a) moja wersja nr 3}
Obliczyć granicę funkcji $f(x)=\frac{\sqrt{x+6}-6}{x}$.
\zadStop
\rozwStart{Patryk Wirkus}{}
$$\frac{\sqrt{x+6}-6}{x}=\frac{(\sqrt{x+6}-6)(\sqrt{x+6}+6)}{x(\sqrt{x+6}+6)}=\frac{1}{x(\sqrt{x+6}+6)}$$
\\
$$\lim\limits_{x\to0}\frac{\sqrt{x+6}-6}{x}=[\frac{0}{0}]=
\lim\limits_{x\to0}\frac{1}{x(\sqrt{x+6}+6)} = \frac{1}{\sqrt{6}+6}$$
\rozwStop
\odpStart
$\frac{1}{\sqrt{6}+6}$
\odpStop
\testStart
A.$\frac{1}{\sqrt{6}+6}$
B.$\frac{6}{\sqrt{6}+6}$
C.$0$
D.$\sqrt{6}+6$
E.$\infty$
F.$-\infty$
G.$\sqrt{6}-6$
H.$-6$
I.$6$
\testStop
\kluczStart
A
\kluczStop



\zadStart{Zadanie z Wikieł Z 4.3 a) moja wersja nr 4}
Obliczyć granicę funkcji $f(x)=\frac{\sqrt{x+9}-9}{x}$.
\zadStop
\rozwStart{Patryk Wirkus}{}
$$\frac{\sqrt{x+9}-9}{x}=\frac{(\sqrt{x+9}-9)(\sqrt{x+9}+9)}{x(\sqrt{x+9}+9)}=\frac{1}{x(\sqrt{x+9}+9)}$$
\\
$$\lim\limits_{x\to0}\frac{\sqrt{x+9}-9}{x}=[\frac{0}{0}]=
\lim\limits_{x\to0}\frac{1}{x(\sqrt{x+9}+9)} = \frac{1}{\sqrt{9}+9}$$
\rozwStop
\odpStart
$\frac{1}{\sqrt{9}+9}$
\odpStop
\testStart
A.$\frac{1}{\sqrt{9}+9}$
B.$\frac{9}{\sqrt{9}+9}$
C.$0$
D.$\sqrt{9}+9$
E.$\infty$
F.$-\infty$
G.$\sqrt{9}-9$
H.$-9$
I.$9$
\testStop
\kluczStart
A
\kluczStop



\zadStart{Zadanie z Wikieł Z 4.3 a) moja wersja nr 5}
Obliczyć granicę funkcji $f(x)=\frac{\sqrt{x+16}-16}{x}$.
\zadStop
\rozwStart{Patryk Wirkus}{}
$$\frac{\sqrt{x+16}-16}{x}=\frac{(\sqrt{x+16}-16)(\sqrt{x+16}+16)}{x(\sqrt{x+16}+16)}=\frac{1}{x(\sqrt{x+16}+16)}$$
\\
$$\lim\limits_{x\to0}\frac{\sqrt{x+16}-16}{x}=[\frac{0}{0}]=
\lim\limits_{x\to0}\frac{1}{x(\sqrt{x+16}+16)} = \frac{1}{\sqrt{16}+16}$$
\rozwStop
\odpStart
$\frac{1}{\sqrt{16}+16}$
\odpStop
\testStart
A.$\frac{1}{\sqrt{16}+16}$
B.$\frac{16}{\sqrt{16}+16}$
C.$0$
D.$\sqrt{16}+16$
E.$\infty$
F.$-\infty$
G.$\sqrt{16}-16$
H.$-16$
I.$16$
\testStop
\kluczStart
A
\kluczStop



\zadStart{Zadanie z Wikieł Z 4.3 a) moja wersja nr 6}
Obliczyć granicę funkcji $f(x)=\frac{\sqrt{x+25}-25}{x}$.
\zadStop
\rozwStart{Patryk Wirkus}{}
$$\frac{\sqrt{x+25}-25}{x}=\frac{(\sqrt{x+25}-25)(\sqrt{x+25}+25)}{x(\sqrt{x+25}+25)}=\frac{1}{x(\sqrt{x+25}+25)}$$
\\
$$\lim\limits_{x\to0}\frac{\sqrt{x+25}-25}{x}=[\frac{0}{0}]=
\lim\limits_{x\to0}\frac{1}{x(\sqrt{x+25}+25)} = \frac{1}{\sqrt{25}+25}$$
\rozwStop
\odpStart
$\frac{1}{\sqrt{25}+25}$
\odpStop
\testStart
A.$\frac{1}{\sqrt{25}+25}$
B.$\frac{25}{\sqrt{25}+25}$
C.$0$
D.$\sqrt{25}+25$
E.$\infty$
F.$-\infty$
G.$\sqrt{25}-25$
H.$-25$
I.$25$
\testStop
\kluczStart
A
\kluczStop



\zadStart{Zadanie z Wikieł Z 4.3 a) moja wersja nr 7}
Obliczyć granicę funkcji $f(x)=\frac{\sqrt{x+36}-36}{x}$.
\zadStop
\rozwStart{Patryk Wirkus}{}
$$\frac{\sqrt{x+36}-36}{x}=\frac{(\sqrt{x+36}-36)(\sqrt{x+36}+36)}{x(\sqrt{x+36}+36)}=\frac{1}{x(\sqrt{x+36}+36)}$$
\\
$$\lim\limits_{x\to0}\frac{\sqrt{x+36}-36}{x}=[\frac{0}{0}]=
\lim\limits_{x\to0}\frac{1}{x(\sqrt{x+36}+36)} = \frac{1}{\sqrt{36}+36}$$
\rozwStop
\odpStart
$\frac{1}{\sqrt{36}+36}$
\odpStop
\testStart
A.$\frac{1}{\sqrt{36}+36}$
B.$\frac{36}{\sqrt{36}+36}$
C.$0$
D.$\sqrt{36}+36$
E.$\infty$
F.$-\infty$
G.$\sqrt{36}-36$
H.$-36$
I.$36$
\testStop
\kluczStart
A
\kluczStop



\zadStart{Zadanie z Wikieł Z 4.3 a) moja wersja nr 8}
Obliczyć granicę funkcji $f(x)=\frac{\sqrt{x+49}-49}{x}$.
\zadStop
\rozwStart{Patryk Wirkus}{}
$$\frac{\sqrt{x+49}-49}{x}=\frac{(\sqrt{x+49}-49)(\sqrt{x+49}+49)}{x(\sqrt{x+49}+49)}=\frac{1}{x(\sqrt{x+49}+49)}$$
\\
$$\lim\limits_{x\to0}\frac{\sqrt{x+49}-49}{x}=[\frac{0}{0}]=
\lim\limits_{x\to0}\frac{1}{x(\sqrt{x+49}+49)} = \frac{1}{\sqrt{49}+49}$$
\rozwStop
\odpStart
$\frac{1}{\sqrt{49}+49}$
\odpStop
\testStart
A.$\frac{1}{\sqrt{49}+49}$
B.$\frac{49}{\sqrt{49}+49}$
C.$0$
D.$\sqrt{49}+49$
E.$\infty$
F.$-\infty$
G.$\sqrt{49}-49$
H.$-49$
I.$49$
\testStop
\kluczStart
A
\kluczStop



\zadStart{Zadanie z Wikieł Z 4.3 a) moja wersja nr 9}
Obliczyć granicę funkcji $f(x)=\frac{\sqrt{x+64}-64}{x}$.
\zadStop
\rozwStart{Patryk Wirkus}{}
$$\frac{\sqrt{x+64}-64}{x}=\frac{(\sqrt{x+64}-64)(\sqrt{x+64}+64)}{x(\sqrt{x+64}+64)}=\frac{1}{x(\sqrt{x+64}+64)}$$
\\
$$\lim\limits_{x\to0}\frac{\sqrt{x+64}-64}{x}=[\frac{0}{0}]=
\lim\limits_{x\to0}\frac{1}{x(\sqrt{x+64}+64)} = \frac{1}{\sqrt{64}+64}$$
\rozwStop
\odpStart
$\frac{1}{\sqrt{64}+64}$
\odpStop
\testStart
A.$\frac{1}{\sqrt{64}+64}$
B.$\frac{64}{\sqrt{64}+64}$
C.$0$
D.$\sqrt{64}+64$
E.$\infty$
F.$-\infty$
G.$\sqrt{64}-64$
H.$-64$
I.$64$
\testStop
\kluczStart
A
\kluczStop



\zadStart{Zadanie z Wikieł Z 4.3 a) moja wersja nr 10}
Obliczyć granicę funkcji $f(x)=\frac{\sqrt{x+81}-81}{x}$.
\zadStop
\rozwStart{Patryk Wirkus}{}
$$\frac{\sqrt{x+81}-81}{x}=\frac{(\sqrt{x+81}-81)(\sqrt{x+81}+81)}{x(\sqrt{x+81}+81)}=\frac{1}{x(\sqrt{x+81}+81)}$$
\\
$$\lim\limits_{x\to0}\frac{\sqrt{x+81}-81}{x}=[\frac{0}{0}]=
\lim\limits_{x\to0}\frac{1}{x(\sqrt{x+81}+81)} = \frac{1}{\sqrt{81}+81}$$
\rozwStop
\odpStart
$\frac{1}{\sqrt{81}+81}$
\odpStop
\testStart
A.$\frac{1}{\sqrt{81}+81}$
B.$\frac{81}{\sqrt{81}+81}$
C.$0$
D.$\sqrt{81}+81$
E.$\infty$
F.$-\infty$
G.$\sqrt{81}-81$
H.$-81$
I.$81$
\testStop
\kluczStart
A
\kluczStop



\zadStart{Zadanie z Wikieł Z 4.3 a) moja wersja nr 11}
Obliczyć granicę funkcji $f(x)=\frac{\sqrt{x+100}-100}{x}$.
\zadStop
\rozwStart{Patryk Wirkus}{}
$$\frac{\sqrt{x+100}-100}{x}=\frac{(\sqrt{x+100}-100)(\sqrt{x+100}+100)}{x(\sqrt{x+100}+100)}=\frac{1}{x(\sqrt{x+100}+100)}$$
\\
$$\lim\limits_{x\to0}\frac{\sqrt{x+100}-100}{x}=[\frac{0}{0}]=
\lim\limits_{x\to0}\frac{1}{x(\sqrt{x+100}+100)} = \frac{1}{\sqrt{100}+100}$$
\rozwStop
\odpStart
$\frac{1}{\sqrt{100}+100}$
\odpStop
\testStart
A.$\frac{1}{\sqrt{100}+100}$
B.$\frac{100}{\sqrt{100}+100}$
C.$0$
D.$\sqrt{100}+100$
E.$\infty$
F.$-\infty$
G.$\sqrt{100}-100$
H.$-100$
I.$100$
\testStop
\kluczStart
A
\kluczStop



\zadStart{Zadanie z Wikieł Z 4.3 a) moja wersja nr 12}
Obliczyć granicę funkcji $f(x)=\frac{\sqrt{x+121}-121}{x}$.
\zadStop
\rozwStart{Patryk Wirkus}{}
$$\frac{\sqrt{x+121}-121}{x}=\frac{(\sqrt{x+121}-121)(\sqrt{x+121}+121)}{x(\sqrt{x+121}+121)}=\frac{1}{x(\sqrt{x+121}+121)}$$
\\
$$\lim\limits_{x\to0}\frac{\sqrt{x+121}-121}{x}=[\frac{0}{0}]=
\lim\limits_{x\to0}\frac{1}{x(\sqrt{x+121}+121)} = \frac{1}{\sqrt{121}+121}$$
\rozwStop
\odpStart
$\frac{1}{\sqrt{121}+121}$
\odpStop
\testStart
A.$\frac{1}{\sqrt{121}+121}$
B.$\frac{121}{\sqrt{121}+121}$
C.$0$
D.$\sqrt{121}+121$
E.$\infty$
F.$-\infty$
G.$\sqrt{121}-121$
H.$-121$
I.$121$
\testStop
\kluczStart
A
\kluczStop



\zadStart{Zadanie z Wikieł Z 4.3 a) moja wersja nr 13}
Obliczyć granicę funkcji $f(x)=\frac{\sqrt{x+144}-144}{x}$.
\zadStop
\rozwStart{Patryk Wirkus}{}
$$\frac{\sqrt{x+144}-144}{x}=\frac{(\sqrt{x+144}-144)(\sqrt{x+144}+144)}{x(\sqrt{x+144}+144)}=\frac{1}{x(\sqrt{x+144}+144)}$$
\\
$$\lim\limits_{x\to0}\frac{\sqrt{x+144}-144}{x}=[\frac{0}{0}]=
\lim\limits_{x\to0}\frac{1}{x(\sqrt{x+144}+144)} = \frac{1}{\sqrt{144}+144}$$
\rozwStop
\odpStart
$\frac{1}{\sqrt{144}+144}$
\odpStop
\testStart
A.$\frac{1}{\sqrt{144}+144}$
B.$\frac{144}{\sqrt{144}+144}$
C.$0$
D.$\sqrt{144}+144$
E.$\infty$
F.$-\infty$
G.$\sqrt{144}-144$
H.$-144$
I.$144$
\testStop
\kluczStart
A
\kluczStop



\zadStart{Zadanie z Wikieł Z 4.3 a) moja wersja nr 14}
Obliczyć granicę funkcji $f(x)=\frac{\sqrt{x+169}-169}{x}$.
\zadStop
\rozwStart{Patryk Wirkus}{}
$$\frac{\sqrt{x+169}-169}{x}=\frac{(\sqrt{x+169}-169)(\sqrt{x+169}+169)}{x(\sqrt{x+169}+169)}=\frac{1}{x(\sqrt{x+169}+169)}$$
\\
$$\lim\limits_{x\to0}\frac{\sqrt{x+169}-169}{x}=[\frac{0}{0}]=
\lim\limits_{x\to0}\frac{1}{x(\sqrt{x+169}+169)} = \frac{1}{\sqrt{169}+169}$$
\rozwStop
\odpStart
$\frac{1}{\sqrt{169}+169}$
\odpStop
\testStart
A.$\frac{1}{\sqrt{169}+169}$
B.$\frac{169}{\sqrt{169}+169}$
C.$0$
D.$\sqrt{169}+169$
E.$\infty$
F.$-\infty$
G.$\sqrt{169}-169$
H.$-169$
I.$169$
\testStop
\kluczStart
A
\kluczStop



\zadStart{Zadanie z Wikieł Z 4.3 a) moja wersja nr 15}
Obliczyć granicę funkcji $f(x)=\frac{\sqrt{x+196}-196}{x}$.
\zadStop
\rozwStart{Patryk Wirkus}{}
$$\frac{\sqrt{x+196}-196}{x}=\frac{(\sqrt{x+196}-196)(\sqrt{x+196}+196)}{x(\sqrt{x+196}+196)}=\frac{1}{x(\sqrt{x+196}+196)}$$
\\
$$\lim\limits_{x\to0}\frac{\sqrt{x+196}-196}{x}=[\frac{0}{0}]=
\lim\limits_{x\to0}\frac{1}{x(\sqrt{x+196}+196)} = \frac{1}{\sqrt{196}+196}$$
\rozwStop
\odpStart
$\frac{1}{\sqrt{196}+196}$
\odpStop
\testStart
A.$\frac{1}{\sqrt{196}+196}$
B.$\frac{196}{\sqrt{196}+196}$
C.$0$
D.$\sqrt{196}+196$
E.$\infty$
F.$-\infty$
G.$\sqrt{196}-196$
H.$-196$
I.$196$
\testStop
\kluczStart
A
\kluczStop



\zadStart{Zadanie z Wikieł Z 4.3 a) moja wersja nr 16}
Obliczyć granicę funkcji $f(x)=\frac{\sqrt{x+225}-225}{x}$.
\zadStop
\rozwStart{Patryk Wirkus}{}
$$\frac{\sqrt{x+225}-225}{x}=\frac{(\sqrt{x+225}-225)(\sqrt{x+225}+225)}{x(\sqrt{x+225}+225)}=\frac{1}{x(\sqrt{x+225}+225)}$$
\\
$$\lim\limits_{x\to0}\frac{\sqrt{x+225}-225}{x}=[\frac{0}{0}]=
\lim\limits_{x\to0}\frac{1}{x(\sqrt{x+225}+225)} = \frac{1}{\sqrt{225}+225}$$
\rozwStop
\odpStart
$\frac{1}{\sqrt{225}+225}$
\odpStop
\testStart
A.$\frac{1}{\sqrt{225}+225}$
B.$\frac{225}{\sqrt{225}+225}$
C.$0$
D.$\sqrt{225}+225$
E.$\infty$
F.$-\infty$
G.$\sqrt{225}-225$
H.$-225$
I.$225$
\testStop
\kluczStart
A
\kluczStop



\zadStart{Zadanie z Wikieł Z 4.3 a) moja wersja nr 17}
Obliczyć granicę funkcji $f(x)=\frac{\sqrt{x+256}-256}{x}$.
\zadStop
\rozwStart{Patryk Wirkus}{}
$$\frac{\sqrt{x+256}-256}{x}=\frac{(\sqrt{x+256}-256)(\sqrt{x+256}+256)}{x(\sqrt{x+256}+256)}=\frac{1}{x(\sqrt{x+256}+256)}$$
\\
$$\lim\limits_{x\to0}\frac{\sqrt{x+256}-256}{x}=[\frac{0}{0}]=
\lim\limits_{x\to0}\frac{1}{x(\sqrt{x+256}+256)} = \frac{1}{\sqrt{256}+256}$$
\rozwStop
\odpStart
$\frac{1}{\sqrt{256}+256}$
\odpStop
\testStart
A.$\frac{1}{\sqrt{256}+256}$
B.$\frac{256}{\sqrt{256}+256}$
C.$0$
D.$\sqrt{256}+256$
E.$\infty$
F.$-\infty$
G.$\sqrt{256}-256$
H.$-256$
I.$256$
\testStop
\kluczStart
A
\kluczStop



\zadStart{Zadanie z Wikieł Z 4.3 a) moja wersja nr 18}
Obliczyć granicę funkcji $f(x)=\frac{\sqrt{x+289}-289}{x}$.
\zadStop
\rozwStart{Patryk Wirkus}{}
$$\frac{\sqrt{x+289}-289}{x}=\frac{(\sqrt{x+289}-289)(\sqrt{x+289}+289)}{x(\sqrt{x+289}+289)}=\frac{1}{x(\sqrt{x+289}+289)}$$
\\
$$\lim\limits_{x\to0}\frac{\sqrt{x+289}-289}{x}=[\frac{0}{0}]=
\lim\limits_{x\to0}\frac{1}{x(\sqrt{x+289}+289)} = \frac{1}{\sqrt{289}+289}$$
\rozwStop
\odpStart
$\frac{1}{\sqrt{289}+289}$
\odpStop
\testStart
A.$\frac{1}{\sqrt{289}+289}$
B.$\frac{289}{\sqrt{289}+289}$
C.$0$
D.$\sqrt{289}+289$
E.$\infty$
F.$-\infty$
G.$\sqrt{289}-289$
H.$-289$
I.$289$
\testStop
\kluczStart
A
\kluczStop



\zadStart{Zadanie z Wikieł Z 4.3 a) moja wersja nr 19}
Obliczyć granicę funkcji $f(x)=\frac{\sqrt{x+324}-324}{x}$.
\zadStop
\rozwStart{Patryk Wirkus}{}
$$\frac{\sqrt{x+324}-324}{x}=\frac{(\sqrt{x+324}-324)(\sqrt{x+324}+324)}{x(\sqrt{x+324}+324)}=\frac{1}{x(\sqrt{x+324}+324)}$$
\\
$$\lim\limits_{x\to0}\frac{\sqrt{x+324}-324}{x}=[\frac{0}{0}]=
\lim\limits_{x\to0}\frac{1}{x(\sqrt{x+324}+324)} = \frac{1}{\sqrt{324}+324}$$
\rozwStop
\odpStart
$\frac{1}{\sqrt{324}+324}$
\odpStop
\testStart
A.$\frac{1}{\sqrt{324}+324}$
B.$\frac{324}{\sqrt{324}+324}$
C.$0$
D.$\sqrt{324}+324$
E.$\infty$
F.$-\infty$
G.$\sqrt{324}-324$
H.$-324$
I.$324$
\testStop
\kluczStart
A
\kluczStop



\zadStart{Zadanie z Wikieł Z 4.3 a) moja wersja nr 20}
Obliczyć granicę funkcji $f(x)=\frac{\sqrt{x+361}-361}{x}$.
\zadStop
\rozwStart{Patryk Wirkus}{}
$$\frac{\sqrt{x+361}-361}{x}=\frac{(\sqrt{x+361}-361)(\sqrt{x+361}+361)}{x(\sqrt{x+361}+361)}=\frac{1}{x(\sqrt{x+361}+361)}$$
\\
$$\lim\limits_{x\to0}\frac{\sqrt{x+361}-361}{x}=[\frac{0}{0}]=
\lim\limits_{x\to0}\frac{1}{x(\sqrt{x+361}+361)} = \frac{1}{\sqrt{361}+361}$$
\rozwStop
\odpStart
$\frac{1}{\sqrt{361}+361}$
\odpStop
\testStart
A.$\frac{1}{\sqrt{361}+361}$
B.$\frac{361}{\sqrt{361}+361}$
C.$0$
D.$\sqrt{361}+361$
E.$\infty$
F.$-\infty$
G.$\sqrt{361}-361$
H.$-361$
I.$361$
\testStop
\kluczStart
A
\kluczStop



\zadStart{Zadanie z Wikieł Z 4.3 a) moja wersja nr 21}
Obliczyć granicę funkcji $f(x)=\frac{\sqrt{x+400}-400}{x}$.
\zadStop
\rozwStart{Patryk Wirkus}{}
$$\frac{\sqrt{x+400}-400}{x}=\frac{(\sqrt{x+400}-400)(\sqrt{x+400}+400)}{x(\sqrt{x+400}+400)}=\frac{1}{x(\sqrt{x+400}+400)}$$
\\
$$\lim\limits_{x\to0}\frac{\sqrt{x+400}-400}{x}=[\frac{0}{0}]=
\lim\limits_{x\to0}\frac{1}{x(\sqrt{x+400}+400)} = \frac{1}{\sqrt{400}+400}$$
\rozwStop
\odpStart
$\frac{1}{\sqrt{400}+400}$
\odpStop
\testStart
A.$\frac{1}{\sqrt{400}+400}$
B.$\frac{400}{\sqrt{400}+400}$
C.$0$
D.$\sqrt{400}+400$
E.$\infty$
F.$-\infty$
G.$\sqrt{400}-400$
H.$-400$
I.$400$
\testStop
\kluczStart
A
\kluczStop



\zadStart{Zadanie z Wikieł Z 4.3 a) moja wersja nr 22}
Obliczyć granicę funkcji $f(x)=\frac{\sqrt{x+625}-625}{x}$.
\zadStop
\rozwStart{Patryk Wirkus}{}
$$\frac{\sqrt{x+625}-625}{x}=\frac{(\sqrt{x+625}-625)(\sqrt{x+625}+625)}{x(\sqrt{x+625}+625)}=\frac{1}{x(\sqrt{x+625}+625)}$$
\\
$$\lim\limits_{x\to0}\frac{\sqrt{x+625}-625}{x}=[\frac{0}{0}]=
\lim\limits_{x\to0}\frac{1}{x(\sqrt{x+625}+625)} = \frac{1}{\sqrt{625}+625}$$
\rozwStop
\odpStart
$\frac{1}{\sqrt{625}+625}$
\odpStop
\testStart
A.$\frac{1}{\sqrt{625}+625}$
B.$\frac{625}{\sqrt{625}+625}$
C.$0$
D.$\sqrt{625}+625$
E.$\infty$
F.$-\infty$
G.$\sqrt{625}-625$
H.$-625$
I.$625$
\testStop
\kluczStart
A
\kluczStop



\zadStart{Zadanie z Wikieł Z 4.3 a) moja wersja nr 23}
Obliczyć granicę funkcji $f(x)=\frac{\sqrt{x+900}-900}{x}$.
\zadStop
\rozwStart{Patryk Wirkus}{}
$$\frac{\sqrt{x+900}-900}{x}=\frac{(\sqrt{x+900}-900)(\sqrt{x+900}+900)}{x(\sqrt{x+900}+900)}=\frac{1}{x(\sqrt{x+900}+900)}$$
\\
$$\lim\limits_{x\to0}\frac{\sqrt{x+900}-900}{x}=[\frac{0}{0}]=
\lim\limits_{x\to0}\frac{1}{x(\sqrt{x+900}+900)} = \frac{1}{\sqrt{900}+900}$$
\rozwStop
\odpStart
$\frac{1}{\sqrt{900}+900}$
\odpStop
\testStart
A.$\frac{1}{\sqrt{900}+900}$
B.$\frac{900}{\sqrt{900}+900}$
C.$0$
D.$\sqrt{900}+900$
E.$\infty$
F.$-\infty$
G.$\sqrt{900}-900$
H.$-900$
I.$900$
\testStop
\kluczStart
A
\kluczStop



\zadStart{Zadanie z Wikieł Z 4.3 a) moja wersja nr 24}
Obliczyć granicę funkcji $f(x)=\frac{\sqrt{x+1600}-1600}{x}$.
\zadStop
\rozwStart{Patryk Wirkus}{}
$$\frac{\sqrt{x+1600}-1600}{x}=\frac{(\sqrt{x+1600}-1600)(\sqrt{x+1600}+1600)}{x(\sqrt{x+1600}+1600)}=\frac{1}{x(\sqrt{x+1600}+1600)}$$
\\
$$\lim\limits_{x\to0}\frac{\sqrt{x+1600}-1600}{x}=[\frac{0}{0}]=
\lim\limits_{x\to0}\frac{1}{x(\sqrt{x+1600}+1600)} = \frac{1}{\sqrt{1600}+1600}$$
\rozwStop
\odpStart
$\frac{1}{\sqrt{1600}+1600}$
\odpStop
\testStart
A.$\frac{1}{\sqrt{1600}+1600}$
B.$\frac{1600}{\sqrt{1600}+1600}$
C.$0$
D.$\sqrt{1600}+1600$
E.$\infty$
F.$-\infty$
G.$\sqrt{1600}-1600$
H.$-1600$
I.$1600$
\testStop
\kluczStart
A
\kluczStop



\zadStart{Zadanie z Wikieł Z 4.3 a) moja wersja nr 25}
Obliczyć granicę funkcji $f(x)=\frac{\sqrt{x+2500}-2500}{x}$.
\zadStop
\rozwStart{Patryk Wirkus}{}
$$\frac{\sqrt{x+2500}-2500}{x}=\frac{(\sqrt{x+2500}-2500)(\sqrt{x+2500}+2500)}{x(\sqrt{x+2500}+2500)}=\frac{1}{x(\sqrt{x+2500}+2500)}$$
\\
$$\lim\limits_{x\to0}\frac{\sqrt{x+2500}-2500}{x}=[\frac{0}{0}]=
\lim\limits_{x\to0}\frac{1}{x(\sqrt{x+2500}+2500)} = \frac{1}{\sqrt{2500}+2500}$$
\rozwStop
\odpStart
$\frac{1}{\sqrt{2500}+2500}$
\odpStop
\testStart
A.$\frac{1}{\sqrt{2500}+2500}$
B.$\frac{2500}{\sqrt{2500}+2500}$
C.$0$
D.$\sqrt{2500}+2500$
E.$\infty$
F.$-\infty$
G.$\sqrt{2500}-2500$
H.$-2500$
I.$2500$
\testStop
\kluczStart
A
\kluczStop



\zadStart{Zadanie z Wikieł Z 4.3 a) moja wersja nr 26}
Obliczyć granicę funkcji $f(x)=\frac{\sqrt{x+3600}-3600}{x}$.
\zadStop
\rozwStart{Patryk Wirkus}{}
$$\frac{\sqrt{x+3600}-3600}{x}=\frac{(\sqrt{x+3600}-3600)(\sqrt{x+3600}+3600)}{x(\sqrt{x+3600}+3600)}=\frac{1}{x(\sqrt{x+3600}+3600)}$$
\\
$$\lim\limits_{x\to0}\frac{\sqrt{x+3600}-3600}{x}=[\frac{0}{0}]=
\lim\limits_{x\to0}\frac{1}{x(\sqrt{x+3600}+3600)} = \frac{1}{\sqrt{3600}+3600}$$
\rozwStop
\odpStart
$\frac{1}{\sqrt{3600}+3600}$
\odpStop
\testStart
A.$\frac{1}{\sqrt{3600}+3600}$
B.$\frac{3600}{\sqrt{3600}+3600}$
C.$0$
D.$\sqrt{3600}+3600$
E.$\infty$
F.$-\infty$
G.$\sqrt{3600}-3600$
H.$-3600$
I.$3600$
\testStop
\kluczStart
A
\kluczStop



\zadStart{Zadanie z Wikieł Z 4.3 a) moja wersja nr 27}
Obliczyć granicę funkcji $f(x)=\frac{\sqrt{x+4900}-4900}{x}$.
\zadStop
\rozwStart{Patryk Wirkus}{}
$$\frac{\sqrt{x+4900}-4900}{x}=\frac{(\sqrt{x+4900}-4900)(\sqrt{x+4900}+4900)}{x(\sqrt{x+4900}+4900)}=\frac{1}{x(\sqrt{x+4900}+4900)}$$
\\
$$\lim\limits_{x\to0}\frac{\sqrt{x+4900}-4900}{x}=[\frac{0}{0}]=
\lim\limits_{x\to0}\frac{1}{x(\sqrt{x+4900}+4900)} = \frac{1}{\sqrt{4900}+4900}$$
\rozwStop
\odpStart
$\frac{1}{\sqrt{4900}+4900}$
\odpStop
\testStart
A.$\frac{1}{\sqrt{4900}+4900}$
B.$\frac{4900}{\sqrt{4900}+4900}$
C.$0$
D.$\sqrt{4900}+4900$
E.$\infty$
F.$-\infty$
G.$\sqrt{4900}-4900$
H.$-4900$
I.$4900$
\testStop
\kluczStart
A
\kluczStop



\zadStart{Zadanie z Wikieł Z 4.3 a) moja wersja nr 28}
Obliczyć granicę funkcji $f(x)=\frac{\sqrt{x+6400}-6400}{x}$.
\zadStop
\rozwStart{Patryk Wirkus}{}
$$\frac{\sqrt{x+6400}-6400}{x}=\frac{(\sqrt{x+6400}-6400)(\sqrt{x+6400}+6400)}{x(\sqrt{x+6400}+6400)}=\frac{1}{x(\sqrt{x+6400}+6400)}$$
\\
$$\lim\limits_{x\to0}\frac{\sqrt{x+6400}-6400}{x}=[\frac{0}{0}]=
\lim\limits_{x\to0}\frac{1}{x(\sqrt{x+6400}+6400)} = \frac{1}{\sqrt{6400}+6400}$$
\rozwStop
\odpStart
$\frac{1}{\sqrt{6400}+6400}$
\odpStop
\testStart
A.$\frac{1}{\sqrt{6400}+6400}$
B.$\frac{6400}{\sqrt{6400}+6400}$
C.$0$
D.$\sqrt{6400}+6400$
E.$\infty$
F.$-\infty$
G.$\sqrt{6400}-6400$
H.$-6400$
I.$6400$
\testStop
\kluczStart
A
\kluczStop



\zadStart{Zadanie z Wikieł Z 4.3 a) moja wersja nr 29}
Obliczyć granicę funkcji $f(x)=\frac{\sqrt{x+8100}-8100}{x}$.
\zadStop
\rozwStart{Patryk Wirkus}{}
$$\frac{\sqrt{x+8100}-8100}{x}=\frac{(\sqrt{x+8100}-8100)(\sqrt{x+8100}+8100)}{x(\sqrt{x+8100}+8100)}=\frac{1}{x(\sqrt{x+8100}+8100)}$$
\\
$$\lim\limits_{x\to0}\frac{\sqrt{x+8100}-8100}{x}=[\frac{0}{0}]=
\lim\limits_{x\to0}\frac{1}{x(\sqrt{x+8100}+8100)} = \frac{1}{\sqrt{8100}+8100}$$
\rozwStop
\odpStart
$\frac{1}{\sqrt{8100}+8100}$
\odpStop
\testStart
A.$\frac{1}{\sqrt{8100}+8100}$
B.$\frac{8100}{\sqrt{8100}+8100}$
C.$0$
D.$\sqrt{8100}+8100$
E.$\infty$
F.$-\infty$
G.$\sqrt{8100}-8100$
H.$-8100$
I.$8100$
\testStop
\kluczStart
A
\kluczStop



\zadStart{Zadanie z Wikieł Z 4.3 a) moja wersja nr 30}
Obliczyć granicę funkcji $f(x)=\frac{\sqrt{x+10000}-10000}{x}$.
\zadStop
\rozwStart{Patryk Wirkus}{}
$$\frac{\sqrt{x+10000}-10000}{x}=\frac{(\sqrt{x+10000}-10000)(\sqrt{x+10000}+10000)}{x(\sqrt{x+10000}+10000)}=\frac{1}{x(\sqrt{x+10000}+10000)}$$
\\
$$\lim\limits_{x\to0}\frac{\sqrt{x+10000}-10000}{x}=[\frac{0}{0}]=
\lim\limits_{x\to0}\frac{1}{x(\sqrt{x+10000}+10000)} = \frac{1}{\sqrt{10000}+10000}$$
\rozwStop
\odpStart
$\frac{1}{\sqrt{10000}+10000}$
\odpStop
\testStart
A.$\frac{1}{\sqrt{10000}+10000}$
B.$\frac{10000}{\sqrt{10000}+10000}$
C.$0$
D.$\sqrt{10000}+10000$
E.$\infty$
F.$-\infty$
G.$\sqrt{10000}-10000$
H.$-10000$
I.$10000$
\testStop
\kluczStart
A
\kluczStop



\zadStart{Zadanie z Wikieł Z 4.3 a) moja wersja nr 31}
Obliczyć granicę funkcji $f(x)=\frac{\sqrt{x+2}-2}{x}$.
\zadStop
\rozwStart{Patryk Wirkus}{}
$$\frac{\sqrt{x+2}-2}{x}=\frac{(\sqrt{x+2}-2)(\sqrt{x+2}+2)}{x(\sqrt{x+2}+2)}=\frac{1}{x(\sqrt{x+2}+2)}$$
\\
$$\lim\limits_{x\to0}\frac{\sqrt{x+2}-2}{x}=[\frac{0}{0}]=
\lim\limits_{x\to0}\frac{1}{x(\sqrt{x+2}+2)} = \frac{1}{\sqrt{2}+2}$$
\rozwStop
\odpStart
$\frac{1}{\sqrt{2}+2}$
\odpStop
\testStart
A.$\frac{1}{\sqrt{2}+2}$
B.$\frac{2}{\sqrt{2}+2}$
C.$0$
D.$\sqrt{2}+2$
E.$\infty$
F.$-\infty$
G.$\sqrt{2}-2$
H.$-2$
I.$2$
\testStop
\kluczStart
A
\kluczStop



\zadStart{Zadanie z Wikieł Z 4.3 a) moja wersja nr 32}
Obliczyć granicę funkcji $f(x)=\frac{\sqrt{x+3}-3}{x}$.
\zadStop
\rozwStart{Patryk Wirkus}{}
$$\frac{\sqrt{x+3}-3}{x}=\frac{(\sqrt{x+3}-3)(\sqrt{x+3}+3)}{x(\sqrt{x+3}+3)}=\frac{1}{x(\sqrt{x+3}+3)}$$
\\
$$\lim\limits_{x\to0}\frac{\sqrt{x+3}-3}{x}=[\frac{0}{0}]=
\lim\limits_{x\to0}\frac{1}{x(\sqrt{x+3}+3)} = \frac{1}{\sqrt{3}+3}$$
\rozwStop
\odpStart
$\frac{1}{\sqrt{3}+3}$
\odpStop
\testStart
A.$\frac{1}{\sqrt{3}+3}$
B.$\frac{3}{\sqrt{3}+3}$
C.$0$
D.$\sqrt{3}+3$
E.$\infty$
F.$-\infty$
G.$\sqrt{3}-3$
H.$-3$
I.$3$
\testStop
\kluczStart
A
\kluczStop



\zadStart{Zadanie z Wikieł Z 4.3 a) moja wersja nr 33}
Obliczyć granicę funkcji $f(x)=\frac{\sqrt{x+5}-5}{x}$.
\zadStop
\rozwStart{Patryk Wirkus}{}
$$\frac{\sqrt{x+5}-5}{x}=\frac{(\sqrt{x+5}-5)(\sqrt{x+5}+5)}{x(\sqrt{x+5}+5)}=\frac{1}{x(\sqrt{x+5}+5)}$$
\\
$$\lim\limits_{x\to0}\frac{\sqrt{x+5}-5}{x}=[\frac{0}{0}]=
\lim\limits_{x\to0}\frac{1}{x(\sqrt{x+5}+5)} = \frac{1}{\sqrt{5}+5}$$
\rozwStop
\odpStart
$\frac{1}{\sqrt{5}+5}$
\odpStop
\testStart
A.$\frac{1}{\sqrt{5}+5}$
B.$\frac{5}{\sqrt{5}+5}$
C.$0$
D.$\sqrt{5}+5$
E.$\infty$
F.$-\infty$
G.$\sqrt{5}-5$
H.$-5$
I.$5$
\testStop
\kluczStart
A
\kluczStop



\zadStart{Zadanie z Wikieł Z 4.3 a) moja wersja nr 34}
Obliczyć granicę funkcji $f(x)=\frac{\sqrt{x+7}-7}{x}$.
\zadStop
\rozwStart{Patryk Wirkus}{}
$$\frac{\sqrt{x+7}-7}{x}=\frac{(\sqrt{x+7}-7)(\sqrt{x+7}+7)}{x(\sqrt{x+7}+7)}=\frac{1}{x(\sqrt{x+7}+7)}$$
\\
$$\lim\limits_{x\to0}\frac{\sqrt{x+7}-7}{x}=[\frac{0}{0}]=
\lim\limits_{x\to0}\frac{1}{x(\sqrt{x+7}+7)} = \frac{1}{\sqrt{7}+7}$$
\rozwStop
\odpStart
$\frac{1}{\sqrt{7}+7}$
\odpStop
\testStart
A.$\frac{1}{\sqrt{7}+7}$
B.$\frac{7}{\sqrt{7}+7}$
C.$0$
D.$\sqrt{7}+7$
E.$\infty$
F.$-\infty$
G.$\sqrt{7}-7$
H.$-7$
I.$7$
\testStop
\kluczStart
A
\kluczStop



\zadStart{Zadanie z Wikieł Z 4.3 a) moja wersja nr 35}
Obliczyć granicę funkcji $f(x)=\frac{\sqrt{x+11}-11}{x}$.
\zadStop
\rozwStart{Patryk Wirkus}{}
$$\frac{\sqrt{x+11}-11}{x}=\frac{(\sqrt{x+11}-11)(\sqrt{x+11}+11)}{x(\sqrt{x+11}+11)}=\frac{1}{x(\sqrt{x+11}+11)}$$
\\
$$\lim\limits_{x\to0}\frac{\sqrt{x+11}-11}{x}=[\frac{0}{0}]=
\lim\limits_{x\to0}\frac{1}{x(\sqrt{x+11}+11)} = \frac{1}{\sqrt{11}+11}$$
\rozwStop
\odpStart
$\frac{1}{\sqrt{11}+11}$
\odpStop
\testStart
A.$\frac{1}{\sqrt{11}+11}$
B.$\frac{11}{\sqrt{11}+11}$
C.$0$
D.$\sqrt{11}+11$
E.$\infty$
F.$-\infty$
G.$\sqrt{11}-11$
H.$-11$
I.$11$
\testStop
\kluczStart
A
\kluczStop



\zadStart{Zadanie z Wikieł Z 4.3 a) moja wersja nr 36}
Obliczyć granicę funkcji $f(x)=\frac{\sqrt{x+13}-13}{x}$.
\zadStop
\rozwStart{Patryk Wirkus}{}
$$\frac{\sqrt{x+13}-13}{x}=\frac{(\sqrt{x+13}-13)(\sqrt{x+13}+13)}{x(\sqrt{x+13}+13)}=\frac{1}{x(\sqrt{x+13}+13)}$$
\\
$$\lim\limits_{x\to0}\frac{\sqrt{x+13}-13}{x}=[\frac{0}{0}]=
\lim\limits_{x\to0}\frac{1}{x(\sqrt{x+13}+13)} = \frac{1}{\sqrt{13}+13}$$
\rozwStop
\odpStart
$\frac{1}{\sqrt{13}+13}$
\odpStop
\testStart
A.$\frac{1}{\sqrt{13}+13}$
B.$\frac{13}{\sqrt{13}+13}$
C.$0$
D.$\sqrt{13}+13$
E.$\infty$
F.$-\infty$
G.$\sqrt{13}-13$
H.$-13$
I.$13$
\testStop
\kluczStart
A
\kluczStop



\zadStart{Zadanie z Wikieł Z 4.3 a) moja wersja nr 37}
Obliczyć granicę funkcji $f(x)=\frac{\sqrt{x+17}-17}{x}$.
\zadStop
\rozwStart{Patryk Wirkus}{}
$$\frac{\sqrt{x+17}-17}{x}=\frac{(\sqrt{x+17}-17)(\sqrt{x+17}+17)}{x(\sqrt{x+17}+17)}=\frac{1}{x(\sqrt{x+17}+17)}$$
\\
$$\lim\limits_{x\to0}\frac{\sqrt{x+17}-17}{x}=[\frac{0}{0}]=
\lim\limits_{x\to0}\frac{1}{x(\sqrt{x+17}+17)} = \frac{1}{\sqrt{17}+17}$$
\rozwStop
\odpStart
$\frac{1}{\sqrt{17}+17}$
\odpStop
\testStart
A.$\frac{1}{\sqrt{17}+17}$
B.$\frac{17}{\sqrt{17}+17}$
C.$0$
D.$\sqrt{17}+17$
E.$\infty$
F.$-\infty$
G.$\sqrt{17}-17$
H.$-17$
I.$17$
\testStop
\kluczStart
A
\kluczStop



\zadStart{Zadanie z Wikieł Z 4.3 a) moja wersja nr 38}
Obliczyć granicę funkcji $f(x)=\frac{\sqrt{x+19}-19}{x}$.
\zadStop
\rozwStart{Patryk Wirkus}{}
$$\frac{\sqrt{x+19}-19}{x}=\frac{(\sqrt{x+19}-19)(\sqrt{x+19}+19)}{x(\sqrt{x+19}+19)}=\frac{1}{x(\sqrt{x+19}+19)}$$
\\
$$\lim\limits_{x\to0}\frac{\sqrt{x+19}-19}{x}=[\frac{0}{0}]=
\lim\limits_{x\to0}\frac{1}{x(\sqrt{x+19}+19)} = \frac{1}{\sqrt{19}+19}$$
\rozwStop
\odpStart
$\frac{1}{\sqrt{19}+19}$
\odpStop
\testStart
A.$\frac{1}{\sqrt{19}+19}$
B.$\frac{19}{\sqrt{19}+19}$
C.$0$
D.$\sqrt{19}+19$
E.$\infty$
F.$-\infty$
G.$\sqrt{19}-19$
H.$-19$
I.$19$
\testStop
\kluczStart
A
\kluczStop



\zadStart{Zadanie z Wikieł Z 4.3 a) moja wersja nr 39}
Obliczyć granicę funkcji $f(x)=\frac{\sqrt{x+23}-23}{x}$.
\zadStop
\rozwStart{Patryk Wirkus}{}
$$\frac{\sqrt{x+23}-23}{x}=\frac{(\sqrt{x+23}-23)(\sqrt{x+23}+23)}{x(\sqrt{x+23}+23)}=\frac{1}{x(\sqrt{x+23}+23)}$$
\\
$$\lim\limits_{x\to0}\frac{\sqrt{x+23}-23}{x}=[\frac{0}{0}]=
\lim\limits_{x\to0}\frac{1}{x(\sqrt{x+23}+23)} = \frac{1}{\sqrt{23}+23}$$
\rozwStop
\odpStart
$\frac{1}{\sqrt{23}+23}$
\odpStop
\testStart
A.$\frac{1}{\sqrt{23}+23}$
B.$\frac{23}{\sqrt{23}+23}$
C.$0$
D.$\sqrt{23}+23$
E.$\infty$
F.$-\infty$
G.$\sqrt{23}-23$
H.$-23$
I.$23$
\testStop
\kluczStart
A
\kluczStop



\zadStart{Zadanie z Wikieł Z 4.3 a) moja wersja nr 40}
Obliczyć granicę funkcji $f(x)=\frac{\sqrt{x+29}-29}{x}$.
\zadStop
\rozwStart{Patryk Wirkus}{}
$$\frac{\sqrt{x+29}-29}{x}=\frac{(\sqrt{x+29}-29)(\sqrt{x+29}+29)}{x(\sqrt{x+29}+29)}=\frac{1}{x(\sqrt{x+29}+29)}$$
\\
$$\lim\limits_{x\to0}\frac{\sqrt{x+29}-29}{x}=[\frac{0}{0}]=
\lim\limits_{x\to0}\frac{1}{x(\sqrt{x+29}+29)} = \frac{1}{\sqrt{29}+29}$$
\rozwStop
\odpStart
$\frac{1}{\sqrt{29}+29}$
\odpStop
\testStart
A.$\frac{1}{\sqrt{29}+29}$
B.$\frac{29}{\sqrt{29}+29}$
C.$0$
D.$\sqrt{29}+29$
E.$\infty$
F.$-\infty$
G.$\sqrt{29}-29$
H.$-29$
I.$29$
\testStop
\kluczStart
A
\kluczStop



\zadStart{Zadanie z Wikieł Z 4.3 a) moja wersja nr 41}
Obliczyć granicę funkcji $f(x)=\frac{\sqrt{x+31}-31}{x}$.
\zadStop
\rozwStart{Patryk Wirkus}{}
$$\frac{\sqrt{x+31}-31}{x}=\frac{(\sqrt{x+31}-31)(\sqrt{x+31}+31)}{x(\sqrt{x+31}+31)}=\frac{1}{x(\sqrt{x+31}+31)}$$
\\
$$\lim\limits_{x\to0}\frac{\sqrt{x+31}-31}{x}=[\frac{0}{0}]=
\lim\limits_{x\to0}\frac{1}{x(\sqrt{x+31}+31)} = \frac{1}{\sqrt{31}+31}$$
\rozwStop
\odpStart
$\frac{1}{\sqrt{31}+31}$
\odpStop
\testStart
A.$\frac{1}{\sqrt{31}+31}$
B.$\frac{31}{\sqrt{31}+31}$
C.$0$
D.$\sqrt{31}+31$
E.$\infty$
F.$-\infty$
G.$\sqrt{31}-31$
H.$-31$
I.$31$
\testStop
\kluczStart
A
\kluczStop



\zadStart{Zadanie z Wikieł Z 4.3 a) moja wersja nr 42}
Obliczyć granicę funkcji $f(x)=\frac{\sqrt{x+37}-37}{x}$.
\zadStop
\rozwStart{Patryk Wirkus}{}
$$\frac{\sqrt{x+37}-37}{x}=\frac{(\sqrt{x+37}-37)(\sqrt{x+37}+37)}{x(\sqrt{x+37}+37)}=\frac{1}{x(\sqrt{x+37}+37)}$$
\\
$$\lim\limits_{x\to0}\frac{\sqrt{x+37}-37}{x}=[\frac{0}{0}]=
\lim\limits_{x\to0}\frac{1}{x(\sqrt{x+37}+37)} = \frac{1}{\sqrt{37}+37}$$
\rozwStop
\odpStart
$\frac{1}{\sqrt{37}+37}$
\odpStop
\testStart
A.$\frac{1}{\sqrt{37}+37}$
B.$\frac{37}{\sqrt{37}+37}$
C.$0$
D.$\sqrt{37}+37$
E.$\infty$
F.$-\infty$
G.$\sqrt{37}-37$
H.$-37$
I.$37$
\testStop
\kluczStart
A
\kluczStop



\zadStart{Zadanie z Wikieł Z 4.3 a) moja wersja nr 43}
Obliczyć granicę funkcji $f(x)=\frac{\sqrt{x+41}-41}{x}$.
\zadStop
\rozwStart{Patryk Wirkus}{}
$$\frac{\sqrt{x+41}-41}{x}=\frac{(\sqrt{x+41}-41)(\sqrt{x+41}+41)}{x(\sqrt{x+41}+41)}=\frac{1}{x(\sqrt{x+41}+41)}$$
\\
$$\lim\limits_{x\to0}\frac{\sqrt{x+41}-41}{x}=[\frac{0}{0}]=
\lim\limits_{x\to0}\frac{1}{x(\sqrt{x+41}+41)} = \frac{1}{\sqrt{41}+41}$$
\rozwStop
\odpStart
$\frac{1}{\sqrt{41}+41}$
\odpStop
\testStart
A.$\frac{1}{\sqrt{41}+41}$
B.$\frac{41}{\sqrt{41}+41}$
C.$0$
D.$\sqrt{41}+41$
E.$\infty$
F.$-\infty$
G.$\sqrt{41}-41$
H.$-41$
I.$41$
\testStop
\kluczStart
A
\kluczStop



\zadStart{Zadanie z Wikieł Z 4.3 a) moja wersja nr 44}
Obliczyć granicę funkcji $f(x)=\frac{\sqrt{x+43}-43}{x}$.
\zadStop
\rozwStart{Patryk Wirkus}{}
$$\frac{\sqrt{x+43}-43}{x}=\frac{(\sqrt{x+43}-43)(\sqrt{x+43}+43)}{x(\sqrt{x+43}+43)}=\frac{1}{x(\sqrt{x+43}+43)}$$
\\
$$\lim\limits_{x\to0}\frac{\sqrt{x+43}-43}{x}=[\frac{0}{0}]=
\lim\limits_{x\to0}\frac{1}{x(\sqrt{x+43}+43)} = \frac{1}{\sqrt{43}+43}$$
\rozwStop
\odpStart
$\frac{1}{\sqrt{43}+43}$
\odpStop
\testStart
A.$\frac{1}{\sqrt{43}+43}$
B.$\frac{43}{\sqrt{43}+43}$
C.$0$
D.$\sqrt{43}+43$
E.$\infty$
F.$-\infty$
G.$\sqrt{43}-43$
H.$-43$
I.$43$
\testStop
\kluczStart
A
\kluczStop



\zadStart{Zadanie z Wikieł Z 4.3 a) moja wersja nr 45}
Obliczyć granicę funkcji $f(x)=\frac{\sqrt{x+47}-47}{x}$.
\zadStop
\rozwStart{Patryk Wirkus}{}
$$\frac{\sqrt{x+47}-47}{x}=\frac{(\sqrt{x+47}-47)(\sqrt{x+47}+47)}{x(\sqrt{x+47}+47)}=\frac{1}{x(\sqrt{x+47}+47)}$$
\\
$$\lim\limits_{x\to0}\frac{\sqrt{x+47}-47}{x}=[\frac{0}{0}]=
\lim\limits_{x\to0}\frac{1}{x(\sqrt{x+47}+47)} = \frac{1}{\sqrt{47}+47}$$
\rozwStop
\odpStart
$\frac{1}{\sqrt{47}+47}$
\odpStop
\testStart
A.$\frac{1}{\sqrt{47}+47}$
B.$\frac{47}{\sqrt{47}+47}$
C.$0$
D.$\sqrt{47}+47$
E.$\infty$
F.$-\infty$
G.$\sqrt{47}-47$
H.$-47$
I.$47$
\testStop
\kluczStart
A
\kluczStop



\zadStart{Zadanie z Wikieł Z 4.3 a) moja wersja nr 46}
Obliczyć granicę funkcji $f(x)=\frac{\sqrt{x+53}-53}{x}$.
\zadStop
\rozwStart{Patryk Wirkus}{}
$$\frac{\sqrt{x+53}-53}{x}=\frac{(\sqrt{x+53}-53)(\sqrt{x+53}+53)}{x(\sqrt{x+53}+53)}=\frac{1}{x(\sqrt{x+53}+53)}$$
\\
$$\lim\limits_{x\to0}\frac{\sqrt{x+53}-53}{x}=[\frac{0}{0}]=
\lim\limits_{x\to0}\frac{1}{x(\sqrt{x+53}+53)} = \frac{1}{\sqrt{53}+53}$$
\rozwStop
\odpStart
$\frac{1}{\sqrt{53}+53}$
\odpStop
\testStart
A.$\frac{1}{\sqrt{53}+53}$
B.$\frac{53}{\sqrt{53}+53}$
C.$0$
D.$\sqrt{53}+53$
E.$\infty$
F.$-\infty$
G.$\sqrt{53}-53$
H.$-53$
I.$53$
\testStop
\kluczStart
A
\kluczStop



\zadStart{Zadanie z Wikieł Z 4.3 a) moja wersja nr 47}
Obliczyć granicę funkcji $f(x)=\frac{\sqrt{x+59}-59}{x}$.
\zadStop
\rozwStart{Patryk Wirkus}{}
$$\frac{\sqrt{x+59}-59}{x}=\frac{(\sqrt{x+59}-59)(\sqrt{x+59}+59)}{x(\sqrt{x+59}+59)}=\frac{1}{x(\sqrt{x+59}+59)}$$
\\
$$\lim\limits_{x\to0}\frac{\sqrt{x+59}-59}{x}=[\frac{0}{0}]=
\lim\limits_{x\to0}\frac{1}{x(\sqrt{x+59}+59)} = \frac{1}{\sqrt{59}+59}$$
\rozwStop
\odpStart
$\frac{1}{\sqrt{59}+59}$
\odpStop
\testStart
A.$\frac{1}{\sqrt{59}+59}$
B.$\frac{59}{\sqrt{59}+59}$
C.$0$
D.$\sqrt{59}+59$
E.$\infty$
F.$-\infty$
G.$\sqrt{59}-59$
H.$-59$
I.$59$
\testStop
\kluczStart
A
\kluczStop



\zadStart{Zadanie z Wikieł Z 4.3 a) moja wersja nr 48}
Obliczyć granicę funkcji $f(x)=\frac{\sqrt{x+61}-61}{x}$.
\zadStop
\rozwStart{Patryk Wirkus}{}
$$\frac{\sqrt{x+61}-61}{x}=\frac{(\sqrt{x+61}-61)(\sqrt{x+61}+61)}{x(\sqrt{x+61}+61)}=\frac{1}{x(\sqrt{x+61}+61)}$$
\\
$$\lim\limits_{x\to0}\frac{\sqrt{x+61}-61}{x}=[\frac{0}{0}]=
\lim\limits_{x\to0}\frac{1}{x(\sqrt{x+61}+61)} = \frac{1}{\sqrt{61}+61}$$
\rozwStop
\odpStart
$\frac{1}{\sqrt{61}+61}$
\odpStop
\testStart
A.$\frac{1}{\sqrt{61}+61}$
B.$\frac{61}{\sqrt{61}+61}$
C.$0$
D.$\sqrt{61}+61$
E.$\infty$
F.$-\infty$
G.$\sqrt{61}-61$
H.$-61$
I.$61$
\testStop
\kluczStart
A
\kluczStop



\zadStart{Zadanie z Wikieł Z 4.3 a) moja wersja nr 49}
Obliczyć granicę funkcji $f(x)=\frac{\sqrt{x+67}-67}{x}$.
\zadStop
\rozwStart{Patryk Wirkus}{}
$$\frac{\sqrt{x+67}-67}{x}=\frac{(\sqrt{x+67}-67)(\sqrt{x+67}+67)}{x(\sqrt{x+67}+67)}=\frac{1}{x(\sqrt{x+67}+67)}$$
\\
$$\lim\limits_{x\to0}\frac{\sqrt{x+67}-67}{x}=[\frac{0}{0}]=
\lim\limits_{x\to0}\frac{1}{x(\sqrt{x+67}+67)} = \frac{1}{\sqrt{67}+67}$$
\rozwStop
\odpStart
$\frac{1}{\sqrt{67}+67}$
\odpStop
\testStart
A.$\frac{1}{\sqrt{67}+67}$
B.$\frac{67}{\sqrt{67}+67}$
C.$0$
D.$\sqrt{67}+67$
E.$\infty$
F.$-\infty$
G.$\sqrt{67}-67$
H.$-67$
I.$67$
\testStop
\kluczStart
A
\kluczStop



\zadStart{Zadanie z Wikieł Z 4.3 a) moja wersja nr 50}
Obliczyć granicę funkcji $f(x)=\frac{\sqrt{x+71}-71}{x}$.
\zadStop
\rozwStart{Patryk Wirkus}{}
$$\frac{\sqrt{x+71}-71}{x}=\frac{(\sqrt{x+71}-71)(\sqrt{x+71}+71)}{x(\sqrt{x+71}+71)}=\frac{1}{x(\sqrt{x+71}+71)}$$
\\
$$\lim\limits_{x\to0}\frac{\sqrt{x+71}-71}{x}=[\frac{0}{0}]=
\lim\limits_{x\to0}\frac{1}{x(\sqrt{x+71}+71)} = \frac{1}{\sqrt{71}+71}$$
\rozwStop
\odpStart
$\frac{1}{\sqrt{71}+71}$
\odpStop
\testStart
A.$\frac{1}{\sqrt{71}+71}$
B.$\frac{71}{\sqrt{71}+71}$
C.$0$
D.$\sqrt{71}+71$
E.$\infty$
F.$-\infty$
G.$\sqrt{71}-71$
H.$-71$
I.$71$
\testStop
\kluczStart
A
\kluczStop



\zadStart{Zadanie z Wikieł Z 4.3 a) moja wersja nr 51}
Obliczyć granicę funkcji $f(x)=\frac{\sqrt{x+73}-73}{x}$.
\zadStop
\rozwStart{Patryk Wirkus}{}
$$\frac{\sqrt{x+73}-73}{x}=\frac{(\sqrt{x+73}-73)(\sqrt{x+73}+73)}{x(\sqrt{x+73}+73)}=\frac{1}{x(\sqrt{x+73}+73)}$$
\\
$$\lim\limits_{x\to0}\frac{\sqrt{x+73}-73}{x}=[\frac{0}{0}]=
\lim\limits_{x\to0}\frac{1}{x(\sqrt{x+73}+73)} = \frac{1}{\sqrt{73}+73}$$
\rozwStop
\odpStart
$\frac{1}{\sqrt{73}+73}$
\odpStop
\testStart
A.$\frac{1}{\sqrt{73}+73}$
B.$\frac{73}{\sqrt{73}+73}$
C.$0$
D.$\sqrt{73}+73$
E.$\infty$
F.$-\infty$
G.$\sqrt{73}-73$
H.$-73$
I.$73$
\testStop
\kluczStart
A
\kluczStop



\zadStart{Zadanie z Wikieł Z 4.3 a) moja wersja nr 52}
Obliczyć granicę funkcji $f(x)=\frac{\sqrt{x+79}-79}{x}$.
\zadStop
\rozwStart{Patryk Wirkus}{}
$$\frac{\sqrt{x+79}-79}{x}=\frac{(\sqrt{x+79}-79)(\sqrt{x+79}+79)}{x(\sqrt{x+79}+79)}=\frac{1}{x(\sqrt{x+79}+79)}$$
\\
$$\lim\limits_{x\to0}\frac{\sqrt{x+79}-79}{x}=[\frac{0}{0}]=
\lim\limits_{x\to0}\frac{1}{x(\sqrt{x+79}+79)} = \frac{1}{\sqrt{79}+79}$$
\rozwStop
\odpStart
$\frac{1}{\sqrt{79}+79}$
\odpStop
\testStart
A.$\frac{1}{\sqrt{79}+79}$
B.$\frac{79}{\sqrt{79}+79}$
C.$0$
D.$\sqrt{79}+79$
E.$\infty$
F.$-\infty$
G.$\sqrt{79}-79$
H.$-79$
I.$79$
\testStop
\kluczStart
A
\kluczStop



\zadStart{Zadanie z Wikieł Z 4.3 a) moja wersja nr 53}
Obliczyć granicę funkcji $f(x)=\frac{\sqrt{x+83}-83}{x}$.
\zadStop
\rozwStart{Patryk Wirkus}{}
$$\frac{\sqrt{x+83}-83}{x}=\frac{(\sqrt{x+83}-83)(\sqrt{x+83}+83)}{x(\sqrt{x+83}+83)}=\frac{1}{x(\sqrt{x+83}+83)}$$
\\
$$\lim\limits_{x\to0}\frac{\sqrt{x+83}-83}{x}=[\frac{0}{0}]=
\lim\limits_{x\to0}\frac{1}{x(\sqrt{x+83}+83)} = \frac{1}{\sqrt{83}+83}$$
\rozwStop
\odpStart
$\frac{1}{\sqrt{83}+83}$
\odpStop
\testStart
A.$\frac{1}{\sqrt{83}+83}$
B.$\frac{83}{\sqrt{83}+83}$
C.$0$
D.$\sqrt{83}+83$
E.$\infty$
F.$-\infty$
G.$\sqrt{83}-83$
H.$-83$
I.$83$
\testStop
\kluczStart
A
\kluczStop



\zadStart{Zadanie z Wikieł Z 4.3 a) moja wersja nr 54}
Obliczyć granicę funkcji $f(x)=\frac{\sqrt{x+89}-89}{x}$.
\zadStop
\rozwStart{Patryk Wirkus}{}
$$\frac{\sqrt{x+89}-89}{x}=\frac{(\sqrt{x+89}-89)(\sqrt{x+89}+89)}{x(\sqrt{x+89}+89)}=\frac{1}{x(\sqrt{x+89}+89)}$$
\\
$$\lim\limits_{x\to0}\frac{\sqrt{x+89}-89}{x}=[\frac{0}{0}]=
\lim\limits_{x\to0}\frac{1}{x(\sqrt{x+89}+89)} = \frac{1}{\sqrt{89}+89}$$
\rozwStop
\odpStart
$\frac{1}{\sqrt{89}+89}$
\odpStop
\testStart
A.$\frac{1}{\sqrt{89}+89}$
B.$\frac{89}{\sqrt{89}+89}$
C.$0$
D.$\sqrt{89}+89$
E.$\infty$
F.$-\infty$
G.$\sqrt{89}-89$
H.$-89$
I.$89$
\testStop
\kluczStart
A
\kluczStop



\zadStart{Zadanie z Wikieł Z 4.3 a) moja wersja nr 55}
Obliczyć granicę funkcji $f(x)=\frac{\sqrt{x+97}-97}{x}$.
\zadStop
\rozwStart{Patryk Wirkus}{}
$$\frac{\sqrt{x+97}-97}{x}=\frac{(\sqrt{x+97}-97)(\sqrt{x+97}+97)}{x(\sqrt{x+97}+97)}=\frac{1}{x(\sqrt{x+97}+97)}$$
\\
$$\lim\limits_{x\to0}\frac{\sqrt{x+97}-97}{x}=[\frac{0}{0}]=
\lim\limits_{x\to0}\frac{1}{x(\sqrt{x+97}+97)} = \frac{1}{\sqrt{97}+97}$$
\rozwStop
\odpStart
$\frac{1}{\sqrt{97}+97}$
\odpStop
\testStart
A.$\frac{1}{\sqrt{97}+97}$
B.$\frac{97}{\sqrt{97}+97}$
C.$0$
D.$\sqrt{97}+97$
E.$\infty$
F.$-\infty$
G.$\sqrt{97}-97$
H.$-97$
I.$97$
\testStop
\kluczStart
A
\kluczStop



\zadStart{Zadanie z Wikieł Z 4.3 a) moja wersja nr 56}
Obliczyć granicę funkcji $f(x)=\frac{\sqrt{x+101}-101}{x}$.
\zadStop
\rozwStart{Patryk Wirkus}{}
$$\frac{\sqrt{x+101}-101}{x}=\frac{(\sqrt{x+101}-101)(\sqrt{x+101}+101)}{x(\sqrt{x+101}+101)}=\frac{1}{x(\sqrt{x+101}+101)}$$
\\
$$\lim\limits_{x\to0}\frac{\sqrt{x+101}-101}{x}=[\frac{0}{0}]=
\lim\limits_{x\to0}\frac{1}{x(\sqrt{x+101}+101)} = \frac{1}{\sqrt{101}+101}$$
\rozwStop
\odpStart
$\frac{1}{\sqrt{101}+101}$
\odpStop
\testStart
A.$\frac{1}{\sqrt{101}+101}$
B.$\frac{101}{\sqrt{101}+101}$
C.$0$
D.$\sqrt{101}+101$
E.$\infty$
F.$-\infty$
G.$\sqrt{101}-101$
H.$-101$
I.$101$
\testStop
\kluczStart
A
\kluczStop



\zadStart{Zadanie z Wikieł Z 4.3 a) moja wersja nr 57}
Obliczyć granicę funkcji $f(x)=\frac{\sqrt{x+103}-103}{x}$.
\zadStop
\rozwStart{Patryk Wirkus}{}
$$\frac{\sqrt{x+103}-103}{x}=\frac{(\sqrt{x+103}-103)(\sqrt{x+103}+103)}{x(\sqrt{x+103}+103)}=\frac{1}{x(\sqrt{x+103}+103)}$$
\\
$$\lim\limits_{x\to0}\frac{\sqrt{x+103}-103}{x}=[\frac{0}{0}]=
\lim\limits_{x\to0}\frac{1}{x(\sqrt{x+103}+103)} = \frac{1}{\sqrt{103}+103}$$
\rozwStop
\odpStart
$\frac{1}{\sqrt{103}+103}$
\odpStop
\testStart
A.$\frac{1}{\sqrt{103}+103}$
B.$\frac{103}{\sqrt{103}+103}$
C.$0$
D.$\sqrt{103}+103$
E.$\infty$
F.$-\infty$
G.$\sqrt{103}-103$
H.$-103$
I.$103$
\testStop
\kluczStart
A
\kluczStop



\zadStart{Zadanie z Wikieł Z 4.3 a) moja wersja nr 58}
Obliczyć granicę funkcji $f(x)=\frac{\sqrt{x+107}-107}{x}$.
\zadStop
\rozwStart{Patryk Wirkus}{}
$$\frac{\sqrt{x+107}-107}{x}=\frac{(\sqrt{x+107}-107)(\sqrt{x+107}+107)}{x(\sqrt{x+107}+107)}=\frac{1}{x(\sqrt{x+107}+107)}$$
\\
$$\lim\limits_{x\to0}\frac{\sqrt{x+107}-107}{x}=[\frac{0}{0}]=
\lim\limits_{x\to0}\frac{1}{x(\sqrt{x+107}+107)} = \frac{1}{\sqrt{107}+107}$$
\rozwStop
\odpStart
$\frac{1}{\sqrt{107}+107}$
\odpStop
\testStart
A.$\frac{1}{\sqrt{107}+107}$
B.$\frac{107}{\sqrt{107}+107}$
C.$0$
D.$\sqrt{107}+107$
E.$\infty$
F.$-\infty$
G.$\sqrt{107}-107$
H.$-107$
I.$107$
\testStop
\kluczStart
A
\kluczStop



\zadStart{Zadanie z Wikieł Z 4.3 a) moja wersja nr 59}
Obliczyć granicę funkcji $f(x)=\frac{\sqrt{x+109}-109}{x}$.
\zadStop
\rozwStart{Patryk Wirkus}{}
$$\frac{\sqrt{x+109}-109}{x}=\frac{(\sqrt{x+109}-109)(\sqrt{x+109}+109)}{x(\sqrt{x+109}+109)}=\frac{1}{x(\sqrt{x+109}+109)}$$
\\
$$\lim\limits_{x\to0}\frac{\sqrt{x+109}-109}{x}=[\frac{0}{0}]=
\lim\limits_{x\to0}\frac{1}{x(\sqrt{x+109}+109)} = \frac{1}{\sqrt{109}+109}$$
\rozwStop
\odpStart
$\frac{1}{\sqrt{109}+109}$
\odpStop
\testStart
A.$\frac{1}{\sqrt{109}+109}$
B.$\frac{109}{\sqrt{109}+109}$
C.$0$
D.$\sqrt{109}+109$
E.$\infty$
F.$-\infty$
G.$\sqrt{109}-109$
H.$-109$
I.$109$
\testStop
\kluczStart
A
\kluczStop



\zadStart{Zadanie z Wikieł Z 4.3 a) moja wersja nr 60}
Obliczyć granicę funkcji $f(x)=\frac{\sqrt{x+113}-113}{x}$.
\zadStop
\rozwStart{Patryk Wirkus}{}
$$\frac{\sqrt{x+113}-113}{x}=\frac{(\sqrt{x+113}-113)(\sqrt{x+113}+113)}{x(\sqrt{x+113}+113)}=\frac{1}{x(\sqrt{x+113}+113)}$$
\\
$$\lim\limits_{x\to0}\frac{\sqrt{x+113}-113}{x}=[\frac{0}{0}]=
\lim\limits_{x\to0}\frac{1}{x(\sqrt{x+113}+113)} = \frac{1}{\sqrt{113}+113}$$
\rozwStop
\odpStart
$\frac{1}{\sqrt{113}+113}$
\odpStop
\testStart
A.$\frac{1}{\sqrt{113}+113}$
B.$\frac{113}{\sqrt{113}+113}$
C.$0$
D.$\sqrt{113}+113$
E.$\infty$
F.$-\infty$
G.$\sqrt{113}-113$
H.$-113$
I.$113$
\testStop
\kluczStart
A
\kluczStop



\zadStart{Zadanie z Wikieł Z 4.3 a) moja wersja nr 61}
Obliczyć granicę funkcji $f(x)=\frac{\sqrt{x+127}-127}{x}$.
\zadStop
\rozwStart{Patryk Wirkus}{}
$$\frac{\sqrt{x+127}-127}{x}=\frac{(\sqrt{x+127}-127)(\sqrt{x+127}+127)}{x(\sqrt{x+127}+127)}=\frac{1}{x(\sqrt{x+127}+127)}$$
\\
$$\lim\limits_{x\to0}\frac{\sqrt{x+127}-127}{x}=[\frac{0}{0}]=
\lim\limits_{x\to0}\frac{1}{x(\sqrt{x+127}+127)} = \frac{1}{\sqrt{127}+127}$$
\rozwStop
\odpStart
$\frac{1}{\sqrt{127}+127}$
\odpStop
\testStart
A.$\frac{1}{\sqrt{127}+127}$
B.$\frac{127}{\sqrt{127}+127}$
C.$0$
D.$\sqrt{127}+127$
E.$\infty$
F.$-\infty$
G.$\sqrt{127}-127$
H.$-127$
I.$127$
\testStop
\kluczStart
A
\kluczStop



\zadStart{Zadanie z Wikieł Z 4.3 a) moja wersja nr 62}
Obliczyć granicę funkcji $f(x)=\frac{\sqrt{x+131}-131}{x}$.
\zadStop
\rozwStart{Patryk Wirkus}{}
$$\frac{\sqrt{x+131}-131}{x}=\frac{(\sqrt{x+131}-131)(\sqrt{x+131}+131)}{x(\sqrt{x+131}+131)}=\frac{1}{x(\sqrt{x+131}+131)}$$
\\
$$\lim\limits_{x\to0}\frac{\sqrt{x+131}-131}{x}=[\frac{0}{0}]=
\lim\limits_{x\to0}\frac{1}{x(\sqrt{x+131}+131)} = \frac{1}{\sqrt{131}+131}$$
\rozwStop
\odpStart
$\frac{1}{\sqrt{131}+131}$
\odpStop
\testStart
A.$\frac{1}{\sqrt{131}+131}$
B.$\frac{131}{\sqrt{131}+131}$
C.$0$
D.$\sqrt{131}+131$
E.$\infty$
F.$-\infty$
G.$\sqrt{131}-131$
H.$-131$
I.$131$
\testStop
\kluczStart
A
\kluczStop



\zadStart{Zadanie z Wikieł Z 4.3 a) moja wersja nr 63}
Obliczyć granicę funkcji $f(x)=\frac{\sqrt{x+137}-137}{x}$.
\zadStop
\rozwStart{Patryk Wirkus}{}
$$\frac{\sqrt{x+137}-137}{x}=\frac{(\sqrt{x+137}-137)(\sqrt{x+137}+137)}{x(\sqrt{x+137}+137)}=\frac{1}{x(\sqrt{x+137}+137)}$$
\\
$$\lim\limits_{x\to0}\frac{\sqrt{x+137}-137}{x}=[\frac{0}{0}]=
\lim\limits_{x\to0}\frac{1}{x(\sqrt{x+137}+137)} = \frac{1}{\sqrt{137}+137}$$
\rozwStop
\odpStart
$\frac{1}{\sqrt{137}+137}$
\odpStop
\testStart
A.$\frac{1}{\sqrt{137}+137}$
B.$\frac{137}{\sqrt{137}+137}$
C.$0$
D.$\sqrt{137}+137$
E.$\infty$
F.$-\infty$
G.$\sqrt{137}-137$
H.$-137$
I.$137$
\testStop
\kluczStart
A
\kluczStop



\zadStart{Zadanie z Wikieł Z 4.3 a) moja wersja nr 64}
Obliczyć granicę funkcji $f(x)=\frac{\sqrt{x+139}-139}{x}$.
\zadStop
\rozwStart{Patryk Wirkus}{}
$$\frac{\sqrt{x+139}-139}{x}=\frac{(\sqrt{x+139}-139)(\sqrt{x+139}+139)}{x(\sqrt{x+139}+139)}=\frac{1}{x(\sqrt{x+139}+139)}$$
\\
$$\lim\limits_{x\to0}\frac{\sqrt{x+139}-139}{x}=[\frac{0}{0}]=
\lim\limits_{x\to0}\frac{1}{x(\sqrt{x+139}+139)} = \frac{1}{\sqrt{139}+139}$$
\rozwStop
\odpStart
$\frac{1}{\sqrt{139}+139}$
\odpStop
\testStart
A.$\frac{1}{\sqrt{139}+139}$
B.$\frac{139}{\sqrt{139}+139}$
C.$0$
D.$\sqrt{139}+139$
E.$\infty$
F.$-\infty$
G.$\sqrt{139}-139$
H.$-139$
I.$139$
\testStop
\kluczStart
A
\kluczStop



\zadStart{Zadanie z Wikieł Z 4.3 a) moja wersja nr 65}
Obliczyć granicę funkcji $f(x)=\frac{\sqrt{x+149}-149}{x}$.
\zadStop
\rozwStart{Patryk Wirkus}{}
$$\frac{\sqrt{x+149}-149}{x}=\frac{(\sqrt{x+149}-149)(\sqrt{x+149}+149)}{x(\sqrt{x+149}+149)}=\frac{1}{x(\sqrt{x+149}+149)}$$
\\
$$\lim\limits_{x\to0}\frac{\sqrt{x+149}-149}{x}=[\frac{0}{0}]=
\lim\limits_{x\to0}\frac{1}{x(\sqrt{x+149}+149)} = \frac{1}{\sqrt{149}+149}$$
\rozwStop
\odpStart
$\frac{1}{\sqrt{149}+149}$
\odpStop
\testStart
A.$\frac{1}{\sqrt{149}+149}$
B.$\frac{149}{\sqrt{149}+149}$
C.$0$
D.$\sqrt{149}+149$
E.$\infty$
F.$-\infty$
G.$\sqrt{149}-149$
H.$-149$
I.$149$
\testStop
\kluczStart
A
\kluczStop



\zadStart{Zadanie z Wikieł Z 4.3 a) moja wersja nr 66}
Obliczyć granicę funkcji $f(x)=\frac{\sqrt{x+151}-151}{x}$.
\zadStop
\rozwStart{Patryk Wirkus}{}
$$\frac{\sqrt{x+151}-151}{x}=\frac{(\sqrt{x+151}-151)(\sqrt{x+151}+151)}{x(\sqrt{x+151}+151)}=\frac{1}{x(\sqrt{x+151}+151)}$$
\\
$$\lim\limits_{x\to0}\frac{\sqrt{x+151}-151}{x}=[\frac{0}{0}]=
\lim\limits_{x\to0}\frac{1}{x(\sqrt{x+151}+151)} = \frac{1}{\sqrt{151}+151}$$
\rozwStop
\odpStart
$\frac{1}{\sqrt{151}+151}$
\odpStop
\testStart
A.$\frac{1}{\sqrt{151}+151}$
B.$\frac{151}{\sqrt{151}+151}$
C.$0$
D.$\sqrt{151}+151$
E.$\infty$
F.$-\infty$
G.$\sqrt{151}-151$
H.$-151$
I.$151$
\testStop
\kluczStart
A
\kluczStop



\zadStart{Zadanie z Wikieł Z 4.3 a) moja wersja nr 67}
Obliczyć granicę funkcji $f(x)=\frac{\sqrt{x+157}-157}{x}$.
\zadStop
\rozwStart{Patryk Wirkus}{}
$$\frac{\sqrt{x+157}-157}{x}=\frac{(\sqrt{x+157}-157)(\sqrt{x+157}+157)}{x(\sqrt{x+157}+157)}=\frac{1}{x(\sqrt{x+157}+157)}$$
\\
$$\lim\limits_{x\to0}\frac{\sqrt{x+157}-157}{x}=[\frac{0}{0}]=
\lim\limits_{x\to0}\frac{1}{x(\sqrt{x+157}+157)} = \frac{1}{\sqrt{157}+157}$$
\rozwStop
\odpStart
$\frac{1}{\sqrt{157}+157}$
\odpStop
\testStart
A.$\frac{1}{\sqrt{157}+157}$
B.$\frac{157}{\sqrt{157}+157}$
C.$0$
D.$\sqrt{157}+157$
E.$\infty$
F.$-\infty$
G.$\sqrt{157}-157$
H.$-157$
I.$157$
\testStop
\kluczStart
A
\kluczStop



\zadStart{Zadanie z Wikieł Z 4.3 a) moja wersja nr 68}
Obliczyć granicę funkcji $f(x)=\frac{\sqrt{x+163}-163}{x}$.
\zadStop
\rozwStart{Patryk Wirkus}{}
$$\frac{\sqrt{x+163}-163}{x}=\frac{(\sqrt{x+163}-163)(\sqrt{x+163}+163)}{x(\sqrt{x+163}+163)}=\frac{1}{x(\sqrt{x+163}+163)}$$
\\
$$\lim\limits_{x\to0}\frac{\sqrt{x+163}-163}{x}=[\frac{0}{0}]=
\lim\limits_{x\to0}\frac{1}{x(\sqrt{x+163}+163)} = \frac{1}{\sqrt{163}+163}$$
\rozwStop
\odpStart
$\frac{1}{\sqrt{163}+163}$
\odpStop
\testStart
A.$\frac{1}{\sqrt{163}+163}$
B.$\frac{163}{\sqrt{163}+163}$
C.$0$
D.$\sqrt{163}+163$
E.$\infty$
F.$-\infty$
G.$\sqrt{163}-163$
H.$-163$
I.$163$
\testStop
\kluczStart
A
\kluczStop



\zadStart{Zadanie z Wikieł Z 4.3 a) moja wersja nr 69}
Obliczyć granicę funkcji $f(x)=\frac{\sqrt{x+167}-167}{x}$.
\zadStop
\rozwStart{Patryk Wirkus}{}
$$\frac{\sqrt{x+167}-167}{x}=\frac{(\sqrt{x+167}-167)(\sqrt{x+167}+167)}{x(\sqrt{x+167}+167)}=\frac{1}{x(\sqrt{x+167}+167)}$$
\\
$$\lim\limits_{x\to0}\frac{\sqrt{x+167}-167}{x}=[\frac{0}{0}]=
\lim\limits_{x\to0}\frac{1}{x(\sqrt{x+167}+167)} = \frac{1}{\sqrt{167}+167}$$
\rozwStop
\odpStart
$\frac{1}{\sqrt{167}+167}$
\odpStop
\testStart
A.$\frac{1}{\sqrt{167}+167}$
B.$\frac{167}{\sqrt{167}+167}$
C.$0$
D.$\sqrt{167}+167$
E.$\infty$
F.$-\infty$
G.$\sqrt{167}-167$
H.$-167$
I.$167$
\testStop
\kluczStart
A
\kluczStop



\zadStart{Zadanie z Wikieł Z 4.3 a) moja wersja nr 70}
Obliczyć granicę funkcji $f(x)=\frac{\sqrt{x+173}-173}{x}$.
\zadStop
\rozwStart{Patryk Wirkus}{}
$$\frac{\sqrt{x+173}-173}{x}=\frac{(\sqrt{x+173}-173)(\sqrt{x+173}+173)}{x(\sqrt{x+173}+173)}=\frac{1}{x(\sqrt{x+173}+173)}$$
\\
$$\lim\limits_{x\to0}\frac{\sqrt{x+173}-173}{x}=[\frac{0}{0}]=
\lim\limits_{x\to0}\frac{1}{x(\sqrt{x+173}+173)} = \frac{1}{\sqrt{173}+173}$$
\rozwStop
\odpStart
$\frac{1}{\sqrt{173}+173}$
\odpStop
\testStart
A.$\frac{1}{\sqrt{173}+173}$
B.$\frac{173}{\sqrt{173}+173}$
C.$0$
D.$\sqrt{173}+173$
E.$\infty$
F.$-\infty$
G.$\sqrt{173}-173$
H.$-173$
I.$173$
\testStop
\kluczStart
A
\kluczStop



\zadStart{Zadanie z Wikieł Z 4.3 a) moja wersja nr 71}
Obliczyć granicę funkcji $f(x)=\frac{\sqrt{x+179}-179}{x}$.
\zadStop
\rozwStart{Patryk Wirkus}{}
$$\frac{\sqrt{x+179}-179}{x}=\frac{(\sqrt{x+179}-179)(\sqrt{x+179}+179)}{x(\sqrt{x+179}+179)}=\frac{1}{x(\sqrt{x+179}+179)}$$
\\
$$\lim\limits_{x\to0}\frac{\sqrt{x+179}-179}{x}=[\frac{0}{0}]=
\lim\limits_{x\to0}\frac{1}{x(\sqrt{x+179}+179)} = \frac{1}{\sqrt{179}+179}$$
\rozwStop
\odpStart
$\frac{1}{\sqrt{179}+179}$
\odpStop
\testStart
A.$\frac{1}{\sqrt{179}+179}$
B.$\frac{179}{\sqrt{179}+179}$
C.$0$
D.$\sqrt{179}+179$
E.$\infty$
F.$-\infty$
G.$\sqrt{179}-179$
H.$-179$
I.$179$
\testStop
\kluczStart
A
\kluczStop



\zadStart{Zadanie z Wikieł Z 4.3 a) moja wersja nr 72}
Obliczyć granicę funkcji $f(x)=\frac{\sqrt{x+181}-181}{x}$.
\zadStop
\rozwStart{Patryk Wirkus}{}
$$\frac{\sqrt{x+181}-181}{x}=\frac{(\sqrt{x+181}-181)(\sqrt{x+181}+181)}{x(\sqrt{x+181}+181)}=\frac{1}{x(\sqrt{x+181}+181)}$$
\\
$$\lim\limits_{x\to0}\frac{\sqrt{x+181}-181}{x}=[\frac{0}{0}]=
\lim\limits_{x\to0}\frac{1}{x(\sqrt{x+181}+181)} = \frac{1}{\sqrt{181}+181}$$
\rozwStop
\odpStart
$\frac{1}{\sqrt{181}+181}$
\odpStop
\testStart
A.$\frac{1}{\sqrt{181}+181}$
B.$\frac{181}{\sqrt{181}+181}$
C.$0$
D.$\sqrt{181}+181$
E.$\infty$
F.$-\infty$
G.$\sqrt{181}-181$
H.$-181$
I.$181$
\testStop
\kluczStart
A
\kluczStop



\zadStart{Zadanie z Wikieł Z 4.3 a) moja wersja nr 73}
Obliczyć granicę funkcji $f(x)=\frac{\sqrt{x+191}-191}{x}$.
\zadStop
\rozwStart{Patryk Wirkus}{}
$$\frac{\sqrt{x+191}-191}{x}=\frac{(\sqrt{x+191}-191)(\sqrt{x+191}+191)}{x(\sqrt{x+191}+191)}=\frac{1}{x(\sqrt{x+191}+191)}$$
\\
$$\lim\limits_{x\to0}\frac{\sqrt{x+191}-191}{x}=[\frac{0}{0}]=
\lim\limits_{x\to0}\frac{1}{x(\sqrt{x+191}+191)} = \frac{1}{\sqrt{191}+191}$$
\rozwStop
\odpStart
$\frac{1}{\sqrt{191}+191}$
\odpStop
\testStart
A.$\frac{1}{\sqrt{191}+191}$
B.$\frac{191}{\sqrt{191}+191}$
C.$0$
D.$\sqrt{191}+191$
E.$\infty$
F.$-\infty$
G.$\sqrt{191}-191$
H.$-191$
I.$191$
\testStop
\kluczStart
A
\kluczStop



\zadStart{Zadanie z Wikieł Z 4.3 a) moja wersja nr 74}
Obliczyć granicę funkcji $f(x)=\frac{\sqrt{x+193}-193}{x}$.
\zadStop
\rozwStart{Patryk Wirkus}{}
$$\frac{\sqrt{x+193}-193}{x}=\frac{(\sqrt{x+193}-193)(\sqrt{x+193}+193)}{x(\sqrt{x+193}+193)}=\frac{1}{x(\sqrt{x+193}+193)}$$
\\
$$\lim\limits_{x\to0}\frac{\sqrt{x+193}-193}{x}=[\frac{0}{0}]=
\lim\limits_{x\to0}\frac{1}{x(\sqrt{x+193}+193)} = \frac{1}{\sqrt{193}+193}$$
\rozwStop
\odpStart
$\frac{1}{\sqrt{193}+193}$
\odpStop
\testStart
A.$\frac{1}{\sqrt{193}+193}$
B.$\frac{193}{\sqrt{193}+193}$
C.$0$
D.$\sqrt{193}+193$
E.$\infty$
F.$-\infty$
G.$\sqrt{193}-193$
H.$-193$
I.$193$
\testStop
\kluczStart
A
\kluczStop



\zadStart{Zadanie z Wikieł Z 4.3 a) moja wersja nr 75}
Obliczyć granicę funkcji $f(x)=\frac{\sqrt{x+197}-197}{x}$.
\zadStop
\rozwStart{Patryk Wirkus}{}
$$\frac{\sqrt{x+197}-197}{x}=\frac{(\sqrt{x+197}-197)(\sqrt{x+197}+197)}{x(\sqrt{x+197}+197)}=\frac{1}{x(\sqrt{x+197}+197)}$$
\\
$$\lim\limits_{x\to0}\frac{\sqrt{x+197}-197}{x}=[\frac{0}{0}]=
\lim\limits_{x\to0}\frac{1}{x(\sqrt{x+197}+197)} = \frac{1}{\sqrt{197}+197}$$
\rozwStop
\odpStart
$\frac{1}{\sqrt{197}+197}$
\odpStop
\testStart
A.$\frac{1}{\sqrt{197}+197}$
B.$\frac{197}{\sqrt{197}+197}$
C.$0$
D.$\sqrt{197}+197$
E.$\infty$
F.$-\infty$
G.$\sqrt{197}-197$
H.$-197$
I.$197$
\testStop
\kluczStart
A
\kluczStop



\zadStart{Zadanie z Wikieł Z 4.3 a) moja wersja nr 76}
Obliczyć granicę funkcji $f(x)=\frac{\sqrt{x+199}-199}{x}$.
\zadStop
\rozwStart{Patryk Wirkus}{}
$$\frac{\sqrt{x+199}-199}{x}=\frac{(\sqrt{x+199}-199)(\sqrt{x+199}+199)}{x(\sqrt{x+199}+199)}=\frac{1}{x(\sqrt{x+199}+199)}$$
\\
$$\lim\limits_{x\to0}\frac{\sqrt{x+199}-199}{x}=[\frac{0}{0}]=
\lim\limits_{x\to0}\frac{1}{x(\sqrt{x+199}+199)} = \frac{1}{\sqrt{199}+199}$$
\rozwStop
\odpStart
$\frac{1}{\sqrt{199}+199}$
\odpStop
\testStart
A.$\frac{1}{\sqrt{199}+199}$
B.$\frac{199}{\sqrt{199}+199}$
C.$0$
D.$\sqrt{199}+199$
E.$\infty$
F.$-\infty$
G.$\sqrt{199}-199$
H.$-199$
I.$199$
\testStop
\kluczStart
A
\kluczStop



\zadStart{Zadanie z Wikieł Z 4.3 a) moja wersja nr 77}
Obliczyć granicę funkcji $f(x)=\frac{\sqrt{x+211}-211}{x}$.
\zadStop
\rozwStart{Patryk Wirkus}{}
$$\frac{\sqrt{x+211}-211}{x}=\frac{(\sqrt{x+211}-211)(\sqrt{x+211}+211)}{x(\sqrt{x+211}+211)}=\frac{1}{x(\sqrt{x+211}+211)}$$
\\
$$\lim\limits_{x\to0}\frac{\sqrt{x+211}-211}{x}=[\frac{0}{0}]=
\lim\limits_{x\to0}\frac{1}{x(\sqrt{x+211}+211)} = \frac{1}{\sqrt{211}+211}$$
\rozwStop
\odpStart
$\frac{1}{\sqrt{211}+211}$
\odpStop
\testStart
A.$\frac{1}{\sqrt{211}+211}$
B.$\frac{211}{\sqrt{211}+211}$
C.$0$
D.$\sqrt{211}+211$
E.$\infty$
F.$-\infty$
G.$\sqrt{211}-211$
H.$-211$
I.$211$
\testStop
\kluczStart
A
\kluczStop



\zadStart{Zadanie z Wikieł Z 4.3 a) moja wersja nr 78}
Obliczyć granicę funkcji $f(x)=\frac{\sqrt{x+223}-223}{x}$.
\zadStop
\rozwStart{Patryk Wirkus}{}
$$\frac{\sqrt{x+223}-223}{x}=\frac{(\sqrt{x+223}-223)(\sqrt{x+223}+223)}{x(\sqrt{x+223}+223)}=\frac{1}{x(\sqrt{x+223}+223)}$$
\\
$$\lim\limits_{x\to0}\frac{\sqrt{x+223}-223}{x}=[\frac{0}{0}]=
\lim\limits_{x\to0}\frac{1}{x(\sqrt{x+223}+223)} = \frac{1}{\sqrt{223}+223}$$
\rozwStop
\odpStart
$\frac{1}{\sqrt{223}+223}$
\odpStop
\testStart
A.$\frac{1}{\sqrt{223}+223}$
B.$\frac{223}{\sqrt{223}+223}$
C.$0$
D.$\sqrt{223}+223$
E.$\infty$
F.$-\infty$
G.$\sqrt{223}-223$
H.$-223$
I.$223$
\testStop
\kluczStart
A
\kluczStop



\zadStart{Zadanie z Wikieł Z 4.3 a) moja wersja nr 79}
Obliczyć granicę funkcji $f(x)=\frac{\sqrt{x+227}-227}{x}$.
\zadStop
\rozwStart{Patryk Wirkus}{}
$$\frac{\sqrt{x+227}-227}{x}=\frac{(\sqrt{x+227}-227)(\sqrt{x+227}+227)}{x(\sqrt{x+227}+227)}=\frac{1}{x(\sqrt{x+227}+227)}$$
\\
$$\lim\limits_{x\to0}\frac{\sqrt{x+227}-227}{x}=[\frac{0}{0}]=
\lim\limits_{x\to0}\frac{1}{x(\sqrt{x+227}+227)} = \frac{1}{\sqrt{227}+227}$$
\rozwStop
\odpStart
$\frac{1}{\sqrt{227}+227}$
\odpStop
\testStart
A.$\frac{1}{\sqrt{227}+227}$
B.$\frac{227}{\sqrt{227}+227}$
C.$0$
D.$\sqrt{227}+227$
E.$\infty$
F.$-\infty$
G.$\sqrt{227}-227$
H.$-227$
I.$227$
\testStop
\kluczStart
A
\kluczStop



\zadStart{Zadanie z Wikieł Z 4.3 a) moja wersja nr 80}
Obliczyć granicę funkcji $f(x)=\frac{\sqrt{x+229}-229}{x}$.
\zadStop
\rozwStart{Patryk Wirkus}{}
$$\frac{\sqrt{x+229}-229}{x}=\frac{(\sqrt{x+229}-229)(\sqrt{x+229}+229)}{x(\sqrt{x+229}+229)}=\frac{1}{x(\sqrt{x+229}+229)}$$
\\
$$\lim\limits_{x\to0}\frac{\sqrt{x+229}-229}{x}=[\frac{0}{0}]=
\lim\limits_{x\to0}\frac{1}{x(\sqrt{x+229}+229)} = \frac{1}{\sqrt{229}+229}$$
\rozwStop
\odpStart
$\frac{1}{\sqrt{229}+229}$
\odpStop
\testStart
A.$\frac{1}{\sqrt{229}+229}$
B.$\frac{229}{\sqrt{229}+229}$
C.$0$
D.$\sqrt{229}+229$
E.$\infty$
F.$-\infty$
G.$\sqrt{229}-229$
H.$-229$
I.$229$
\testStop
\kluczStart
A
\kluczStop



\zadStart{Zadanie z Wikieł Z 4.3 a) moja wersja nr 81}
Obliczyć granicę funkcji $f(x)=\frac{\sqrt{x+233}-233}{x}$.
\zadStop
\rozwStart{Patryk Wirkus}{}
$$\frac{\sqrt{x+233}-233}{x}=\frac{(\sqrt{x+233}-233)(\sqrt{x+233}+233)}{x(\sqrt{x+233}+233)}=\frac{1}{x(\sqrt{x+233}+233)}$$
\\
$$\lim\limits_{x\to0}\frac{\sqrt{x+233}-233}{x}=[\frac{0}{0}]=
\lim\limits_{x\to0}\frac{1}{x(\sqrt{x+233}+233)} = \frac{1}{\sqrt{233}+233}$$
\rozwStop
\odpStart
$\frac{1}{\sqrt{233}+233}$
\odpStop
\testStart
A.$\frac{1}{\sqrt{233}+233}$
B.$\frac{233}{\sqrt{233}+233}$
C.$0$
D.$\sqrt{233}+233$
E.$\infty$
F.$-\infty$
G.$\sqrt{233}-233$
H.$-233$
I.$233$
\testStop
\kluczStart
A
\kluczStop



\zadStart{Zadanie z Wikieł Z 4.3 a) moja wersja nr 82}
Obliczyć granicę funkcji $f(x)=\frac{\sqrt{x+239}-239}{x}$.
\zadStop
\rozwStart{Patryk Wirkus}{}
$$\frac{\sqrt{x+239}-239}{x}=\frac{(\sqrt{x+239}-239)(\sqrt{x+239}+239)}{x(\sqrt{x+239}+239)}=\frac{1}{x(\sqrt{x+239}+239)}$$
\\
$$\lim\limits_{x\to0}\frac{\sqrt{x+239}-239}{x}=[\frac{0}{0}]=
\lim\limits_{x\to0}\frac{1}{x(\sqrt{x+239}+239)} = \frac{1}{\sqrt{239}+239}$$
\rozwStop
\odpStart
$\frac{1}{\sqrt{239}+239}$
\odpStop
\testStart
A.$\frac{1}{\sqrt{239}+239}$
B.$\frac{239}{\sqrt{239}+239}$
C.$0$
D.$\sqrt{239}+239$
E.$\infty$
F.$-\infty$
G.$\sqrt{239}-239$
H.$-239$
I.$239$
\testStop
\kluczStart
A
\kluczStop



\zadStart{Zadanie z Wikieł Z 4.3 a) moja wersja nr 83}
Obliczyć granicę funkcji $f(x)=\frac{\sqrt{x+241}-241}{x}$.
\zadStop
\rozwStart{Patryk Wirkus}{}
$$\frac{\sqrt{x+241}-241}{x}=\frac{(\sqrt{x+241}-241)(\sqrt{x+241}+241)}{x(\sqrt{x+241}+241)}=\frac{1}{x(\sqrt{x+241}+241)}$$
\\
$$\lim\limits_{x\to0}\frac{\sqrt{x+241}-241}{x}=[\frac{0}{0}]=
\lim\limits_{x\to0}\frac{1}{x(\sqrt{x+241}+241)} = \frac{1}{\sqrt{241}+241}$$
\rozwStop
\odpStart
$\frac{1}{\sqrt{241}+241}$
\odpStop
\testStart
A.$\frac{1}{\sqrt{241}+241}$
B.$\frac{241}{\sqrt{241}+241}$
C.$0$
D.$\sqrt{241}+241$
E.$\infty$
F.$-\infty$
G.$\sqrt{241}-241$
H.$-241$
I.$241$
\testStop
\kluczStart
A
\kluczStop



\zadStart{Zadanie z Wikieł Z 4.3 a) moja wersja nr 84}
Obliczyć granicę funkcji $f(x)=\frac{\sqrt{x+251}-251}{x}$.
\zadStop
\rozwStart{Patryk Wirkus}{}
$$\frac{\sqrt{x+251}-251}{x}=\frac{(\sqrt{x+251}-251)(\sqrt{x+251}+251)}{x(\sqrt{x+251}+251)}=\frac{1}{x(\sqrt{x+251}+251)}$$
\\
$$\lim\limits_{x\to0}\frac{\sqrt{x+251}-251}{x}=[\frac{0}{0}]=
\lim\limits_{x\to0}\frac{1}{x(\sqrt{x+251}+251)} = \frac{1}{\sqrt{251}+251}$$
\rozwStop
\odpStart
$\frac{1}{\sqrt{251}+251}$
\odpStop
\testStart
A.$\frac{1}{\sqrt{251}+251}$
B.$\frac{251}{\sqrt{251}+251}$
C.$0$
D.$\sqrt{251}+251$
E.$\infty$
F.$-\infty$
G.$\sqrt{251}-251$
H.$-251$
I.$251$
\testStop
\kluczStart
A
\kluczStop



\zadStart{Zadanie z Wikieł Z 4.3 a) moja wersja nr 85}
Obliczyć granicę funkcji $f(x)=\frac{\sqrt{x+257}-257}{x}$.
\zadStop
\rozwStart{Patryk Wirkus}{}
$$\frac{\sqrt{x+257}-257}{x}=\frac{(\sqrt{x+257}-257)(\sqrt{x+257}+257)}{x(\sqrt{x+257}+257)}=\frac{1}{x(\sqrt{x+257}+257)}$$
\\
$$\lim\limits_{x\to0}\frac{\sqrt{x+257}-257}{x}=[\frac{0}{0}]=
\lim\limits_{x\to0}\frac{1}{x(\sqrt{x+257}+257)} = \frac{1}{\sqrt{257}+257}$$
\rozwStop
\odpStart
$\frac{1}{\sqrt{257}+257}$
\odpStop
\testStart
A.$\frac{1}{\sqrt{257}+257}$
B.$\frac{257}{\sqrt{257}+257}$
C.$0$
D.$\sqrt{257}+257$
E.$\infty$
F.$-\infty$
G.$\sqrt{257}-257$
H.$-257$
I.$257$
\testStop
\kluczStart
A
\kluczStop



\zadStart{Zadanie z Wikieł Z 4.3 a) moja wersja nr 86}
Obliczyć granicę funkcji $f(x)=\frac{\sqrt{x+263}-263}{x}$.
\zadStop
\rozwStart{Patryk Wirkus}{}
$$\frac{\sqrt{x+263}-263}{x}=\frac{(\sqrt{x+263}-263)(\sqrt{x+263}+263)}{x(\sqrt{x+263}+263)}=\frac{1}{x(\sqrt{x+263}+263)}$$
\\
$$\lim\limits_{x\to0}\frac{\sqrt{x+263}-263}{x}=[\frac{0}{0}]=
\lim\limits_{x\to0}\frac{1}{x(\sqrt{x+263}+263)} = \frac{1}{\sqrt{263}+263}$$
\rozwStop
\odpStart
$\frac{1}{\sqrt{263}+263}$
\odpStop
\testStart
A.$\frac{1}{\sqrt{263}+263}$
B.$\frac{263}{\sqrt{263}+263}$
C.$0$
D.$\sqrt{263}+263$
E.$\infty$
F.$-\infty$
G.$\sqrt{263}-263$
H.$-263$
I.$263$
\testStop
\kluczStart
A
\kluczStop



\zadStart{Zadanie z Wikieł Z 4.3 a) moja wersja nr 87}
Obliczyć granicę funkcji $f(x)=\frac{\sqrt{x+269}-269}{x}$.
\zadStop
\rozwStart{Patryk Wirkus}{}
$$\frac{\sqrt{x+269}-269}{x}=\frac{(\sqrt{x+269}-269)(\sqrt{x+269}+269)}{x(\sqrt{x+269}+269)}=\frac{1}{x(\sqrt{x+269}+269)}$$
\\
$$\lim\limits_{x\to0}\frac{\sqrt{x+269}-269}{x}=[\frac{0}{0}]=
\lim\limits_{x\to0}\frac{1}{x(\sqrt{x+269}+269)} = \frac{1}{\sqrt{269}+269}$$
\rozwStop
\odpStart
$\frac{1}{\sqrt{269}+269}$
\odpStop
\testStart
A.$\frac{1}{\sqrt{269}+269}$
B.$\frac{269}{\sqrt{269}+269}$
C.$0$
D.$\sqrt{269}+269$
E.$\infty$
F.$-\infty$
G.$\sqrt{269}-269$
H.$-269$
I.$269$
\testStop
\kluczStart
A
\kluczStop



\zadStart{Zadanie z Wikieł Z 4.3 a) moja wersja nr 88}
Obliczyć granicę funkcji $f(x)=\frac{\sqrt{x+271}-271}{x}$.
\zadStop
\rozwStart{Patryk Wirkus}{}
$$\frac{\sqrt{x+271}-271}{x}=\frac{(\sqrt{x+271}-271)(\sqrt{x+271}+271)}{x(\sqrt{x+271}+271)}=\frac{1}{x(\sqrt{x+271}+271)}$$
\\
$$\lim\limits_{x\to0}\frac{\sqrt{x+271}-271}{x}=[\frac{0}{0}]=
\lim\limits_{x\to0}\frac{1}{x(\sqrt{x+271}+271)} = \frac{1}{\sqrt{271}+271}$$
\rozwStop
\odpStart
$\frac{1}{\sqrt{271}+271}$
\odpStop
\testStart
A.$\frac{1}{\sqrt{271}+271}$
B.$\frac{271}{\sqrt{271}+271}$
C.$0$
D.$\sqrt{271}+271$
E.$\infty$
F.$-\infty$
G.$\sqrt{271}-271$
H.$-271$
I.$271$
\testStop
\kluczStart
A
\kluczStop





\end{document}
