\documentclass[12pt, a4paper]{article}
\usepackage[utf8]{inputenc}
\usepackage{polski}

\usepackage{amsthm}  %pakiet do tworzenia twierdzeń itp.
\usepackage{amsmath} %pakiet do niektórych symboli matematycznych
\usepackage{amssymb} %pakiet do symboli mat., np. \nsubseteq
\usepackage{amsfonts}
\usepackage{graphicx} %obsługa plików graficznych z rozszerzeniem png, jpg
\theoremstyle{definition} %styl dla definicji
\newtheorem{zad}{} 
\title{Multizestaw zadań}
\author{Robert Fidytek}
%\date{\today}
\date{}
\newcounter{liczniksekcji}
\newcommand{\kategoria}[1]{\section{#1}} %olreślamy nazwę kateforii zadań
\newcommand{\zadStart}[1]{\begin{zad}#1\newline} %oznaczenie początku zadania
\newcommand{\zadStop}{\end{zad}}   %oznaczenie końca zadania
%Makra opcjonarne (nie muszą występować):
\newcommand{\rozwStart}[2]{\noindent \textbf{Rozwiązanie (autor #1 , recenzent #2): }\newline} %oznaczenie początku rozwiązania, opcjonarnie można wprowadzić informację o autorze rozwiązania zadania i recenzencie poprawności wykonania rozwiązania zadania
\newcommand{\rozwStop}{\newline}                                            %oznaczenie końca rozwiązania
\newcommand{\odpStart}{\noindent \textbf{Odpowiedź:}\newline}    %oznaczenie początku odpowiedzi końcowej (wypisanie wyniku)
\newcommand{\odpStop}{\newline}                                             %oznaczenie końca odpowiedzi końcowej (wypisanie wyniku)
\newcommand{\testStart}{\noindent \textbf{Test:}\newline} %ewentualne możliwe opcje odpowiedzi testowej: A. ? B. ? C. ? D. ? itd.
\newcommand{\testStop}{\newline} %koniec wprowadzania odpowiedzi testowych
\newcommand{\kluczStart}{\noindent \textbf{Test poprawna odpowiedź:}\newline} %klucz, poprawna odpowiedź pytania testowego (jedna literka): A lub B lub C lub D itd.
\newcommand{\kluczStop}{\newline} %koniec poprawnej odpowiedzi pytania testowego 
\newcommand{\wstawGrafike}[2]{\begin{figure}[h] \includegraphics[scale=#2] {#1} \end{figure}} %gdyby była potrzeba wstawienia obrazka, parametry: nazwa pliku, skala (jak nie wiesz co wpisać, to wpisz 1)

\begin{document}
\maketitle


\kategoria{Wikieł/Z3.5b}
\zadStart{Zadanie z Wikieł Z 3.5 b)  moja wersja nr [nrWersji]}
%[p1]:[2,3,4,5,6,7,8,9,10]
%[p0]:[1]
%[p2]:[2,3,4,5]
%[p4]:[2,3,4,5,6,7,8,9,10]
%[p3]=random.randint(1,100)
%[q]=round([p1]/([p4]+0.0000001),2)
%[a]=round(pow([q],[p2]),2)
%[a1]=round([p0]-[a],2)
%[b]=round([p0]-[q],2)
%[pb]=round([p3]*[b],2)
%[w]=round([pb]/([a1]+0.0000001),2)
%math.gcd([p1],[p4])==1 and [p4]>[p1] and [p4]!=0 and [a1]!=0 


Znaleźć pierwszy wyraz ciągu geometrycznego $(a_{n})$, w którym $q=\frac{[p1]}{[p4]}$ oraz $S_{[p2]}=[p3].$
\zadStop
\rozwStart{Maja Szabłowska}{}
Korzystając ze wzoru na sumę ciągu geometrycznego mamy:
$$ [p3]=a_{1}\cdot \frac{1-(\frac{[p1]}{[p4]})^{[p2]}}{1-\frac{[p1]}{[p4]}}$$
$$ [p3]=a_{1}\cdot \frac{1-[a]}{[b]}$$
$$ [p3]=a_{1}\cdot \frac{[a1]}{[b]}$$
$$ a_{1}=\frac{[p3]}{\frac{[a1]}{[b]}}=\frac{[p3]\cdot[b]}{[a1]}=\frac{[pb]}{[a1]}=[w]$$
\rozwStop
\odpStart
$[w]$
\odpStop
\testStart
A.$[w]$
B.$[b]$
C.$[pb]$
D.$[p0]$
E.$[p3]$
F.$[p1]$
G.$[p2]$
\testStop
\kluczStart
A
\kluczStop
\end{document}