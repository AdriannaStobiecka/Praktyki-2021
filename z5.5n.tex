\documentclass[12pt, a4paper]{article}
\usepackage[utf8]{inputenc}
\usepackage{polski}

\usepackage{amsthm}  %pakiet do tworzenia twierdzeń itp.
\usepackage{amsmath} %pakiet do niektórych symboli matematycznych
\usepackage{amssymb} %pakiet do symboli mat., np. \nsubseteq
\usepackage{amsfonts}
\usepackage{graphicx} %obsługa plików graficznych z rozszerzeniem png, jpg
\theoremstyle{definition} %styl dla definicji
\newtheorem{zad}{} 
\title{Multizestaw zadań}
\author{Robert Fidytek}
%\date{\today}
\date{}
\newcounter{liczniksekcji}
\newcommand{\kategoria}[1]{\section{#1}} %olreślamy nazwę kateforii zadań
\newcommand{\zadStart}[1]{\begin{zad}#1\newline} %oznaczenie początku zadania
\newcommand{\zadStop}{\end{zad}}   %oznaczenie końca zadania
%Makra opcjonarne (nie muszą występować):
\newcommand{\rozwStart}[2]{\noindent \textbf{Rozwiązanie (autor #1 , recenzent #2): }\newline} %oznaczenie początku rozwiązania, opcjonarnie można wprowadzić informację o autorze rozwiązania zadania i recenzencie poprawności wykonania rozwiązania zadania
\newcommand{\rozwStop}{\newline}                                            %oznaczenie końca rozwiązania
\newcommand{\odpStart}{\noindent \textbf{Odpowiedź:}\newline}    %oznaczenie początku odpowiedzi końcowej (wypisanie wyniku)
\newcommand{\odpStop}{\newline}                                             %oznaczenie końca odpowiedzi końcowej (wypisanie wyniku)
\newcommand{\testStart}{\noindent \textbf{Test:}\newline} %ewentualne możliwe opcje odpowiedzi testowej: A. ? B. ? C. ? D. ? itd.
\newcommand{\testStop}{\newline} %koniec wprowadzania odpowiedzi testowych
\newcommand{\kluczStart}{\noindent \textbf{Test poprawna odpowiedź:}\newline} %klucz, poprawna odpowiedź pytania testowego (jedna literka): A lub B lub C lub D itd.
\newcommand{\kluczStop}{\newline} %koniec poprawnej odpowiedzi pytania testowego 
\newcommand{\wstawGrafike}[2]{\begin{figure}[h] \includegraphics[scale=#2] {#1} \end{figure}} %gdyby była potrzeba wstawienia obrazka, parametry: nazwa pliku, skala (jak nie wiesz co wpisać, to wpisz 1)

\begin{document}
\maketitle


\kategoria{Wikieł/Z5.5n}
\zadStart{Zadanie z Wikieł Z 5.5 n) moja wersja nr [nrWersji]}
%[a]:[2,3,4,5,6,7,8,9]
%[a1]=3*[a]
%[b]:[2,3,4,5,6,7,8,9]
%[b1]=2*[b]
%[c]=random.randint(2,10)
%[d]=random.randint(2,10)
%[a]!=0 
Wyznacz pochodną funkcji \\ $f(x)=\ln\left([a]x^3-[b]x^2-[c]x+[d]\right) $.
\zadStop
\rozwStart{Joanna Świerzbin}{}
$$f(x)=\ln{\left([a]x^3-[b]x^2-[c]x+[d]\right)} $$
$$u(x)=\ln{(x)} \ \ \ \ \ \ u'(x)=\frac{1}{x}$$
$$g(x)=[a]x^3-[b]x^2-[c]x+[d] $$ $$ g'(x)=3\cdot[a]x^2-2\cdot[b]x-[c]$$
$$f'(x)=(u(g(x)))'=u'(g(x))\cdot g'(x) = $$
 $$= \frac{3\cdot[a]x^2-2\cdot[b]x-[c]}{[a]x^3-[b]x^2-[c]x+[d]}=$$
$$= \frac{[a1]x^2-[b1]x-[c]}{[a]x^3-[b]x^2-[c]x+[d]}$$
\rozwStop
\odpStart
$ f'(x) = \frac{[a1]x^2-[b1]x-[c]}{[a]x^3-[b]x^2-[c]x+[d]}$
\odpStop
\testStart
A. $ f'(x) = \frac{[a1]x^2-[b1]x-[c]}{[a]x^3-[b]x^2-[c]x+[d]}$\\
B. $ f'(x) = {[a1]x^2-[b1]x-[c]}$ \\
C. $ f'(x) = {[a]x^3-[b]x^2-[c]x+[d]}$ \\
D. $ f'(x) = \frac{1}{[a]x^3-[b]x^2-[c]x+[d]}$\\
E. $ f'(x) = \frac{[a1]x^2-[b1]x-[c]}{[a1]x^3-[b1]x^2-[c]x+[d]}$\\
F. $ f'(x) = \frac{[a]x^2-[b]x-[c]}{[a]x^3-[b]x^2-[c]x+[d]}$
\testStop
\kluczStart
A
\kluczStop
\end{document}