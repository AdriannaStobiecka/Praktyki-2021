\documentclass[12pt, a4paper]{article}
\usepackage[utf8]{inputenc}
\usepackage{polski}

\usepackage{amsthm}  %pakiet do tworzenia twierdzeń itp.
\usepackage{amsmath} %pakiet do niektórych symboli matematycznych
\usepackage{amssymb} %pakiet do symboli mat., np. \nsubseteq
\usepackage{amsfonts}
\usepackage{graphicx} %obsługa plików graficznych z rozszerzeniem png, jpg
\theoremstyle{definition} %styl dla definicji
\newtheorem{zad}{} 
\title{Multizestaw zadań}
\author{Robert Fidytek}
%\date{\today}
\date{}
\newcounter{liczniksekcji}
\newcommand{\kategoria}[1]{\section{#1}} %olreślamy nazwę kateforii zadań
\newcommand{\zadStart}[1]{\begin{zad}#1\newline} %oznaczenie początku zadania
\newcommand{\zadStop}{\end{zad}}   %oznaczenie końca zadania
%Makra opcjonarne (nie muszą występować):
\newcommand{\rozwStart}[2]{\noindent \textbf{Rozwiązanie (autor #1 , recenzent #2): }\newline} %oznaczenie początku rozwiązania, opcjonarnie można wprowadzić informację o autorze rozwiązania zadania i recenzencie poprawności wykonania rozwiązania zadania
\newcommand{\rozwStop}{\newline}                                            %oznaczenie końca rozwiązania
\newcommand{\odpStart}{\noindent \textbf{Odpowiedź:}\newline}    %oznaczenie początku odpowiedzi końcowej (wypisanie wyniku)
\newcommand{\odpStop}{\newline}                                             %oznaczenie końca odpowiedzi końcowej (wypisanie wyniku)
\newcommand{\testStart}{\noindent \textbf{Test:}\newline} %ewentualne możliwe opcje odpowiedzi testowej: A. ? B. ? C. ? D. ? itd.
\newcommand{\testStop}{\newline} %koniec wprowadzania odpowiedzi testowych
\newcommand{\kluczStart}{\noindent \textbf{Test poprawna odpowiedź:}\newline} %klucz, poprawna odpowiedź pytania testowego (jedna literka): A lub B lub C lub D itd.
\newcommand{\kluczStop}{\newline} %koniec poprawnej odpowiedzi pytania testowego 
\newcommand{\wstawGrafike}[2]{\begin{figure}[h] \includegraphics[scale=#2] {#1} \end{figure}} %gdyby była potrzeba wstawienia obrazka, parametry: nazwa pliku, skala (jak nie wiesz co wpisać, to wpisz 1)

\begin{document}
\maketitle


\kategoria{Wikieł/Z1.70b}
\zadStart{Zadanie z Wikieł Z 1.70 b) moja wersja nr [nrWersji]}
%[a]:[2,3,4,5,6,7,8,9,10]
%[b]:[2,3,4,5,6,7,8,9,10]
%[c]:[2,3,4,5,6,7,8,9,10]
%[d]:[2,3,4,5,6,7,8,9,10]
%[a]=random.randint(2,25)
%[b]=random.randint(2,25)
%[c]=random.randint(2,25)
%[d]=random.randint(2,25)
%[-c]=(-1)*[c]
%[a]>[c] and [c]>[b] and [b]>[d] and [d]>[-c]
Rozwiązać nierówność $\frac{(x+[a])^{4}([b]-x)^{3}}{(x+[c])(x-[d])^{2}}\geq0$
\zadStop
\rozwStart{Jakub Ulrych}{}
$$\frac{(x+[a])^{4}([b]-x)^{3}}{(x+[c])(x-[d])^{2}}\geq0$$
założenie: $$x+[c]\neq0\land x-[d]]\neq0$$
$$x\neq-[c]\land x\neq[d]$$
dziedzina:$$x\in \mathbb{R}-\{-[c],[d]\}$$
rozwiązanie:$$\frac{(x+[a])^{4}([b]-x)^{3}}{(x+[c])(x-[d])^{2}}\geq0$$
$$\textbf{1a)}(x+[a])^{4}([b]-x)^{3}\geq0 \land \textbf{1b)}(x+[c])(x-[d])^{2}\geq0$$ $$\vee$$ $$\textbf{2a)}(x+[a])^{4}([b]-x)^{3}\leq0 \land \textbf{2b)}(x+[c])(x-[d])^{2}\leq0$$
$$\textbf{1a)}(x+[a])^{4}([b]-x)^{3}\geq0\Leftrightarrow x\leq[b]$$
$$\textbf{1b)}(x+[c])(x-[d])^{2}\geq0\Leftrightarrow x\geq-[c]$$
$$\textbf{2a)}(x+[a])^{4}([b]-x)^{3}\leq0\Leftrightarrow x\geq[b] \vee x=-[a]$$
$$\textbf{2b)}(x+[c])(x-[d])^{2}\leq0\Leftrightarrow x\leq-[c]\vee x=[d]$$
$$\textbf{1a)}\land\textbf{1b)}\Rightarrow x\in[-[c],[b]]$$
$$\textbf{2a)}\land\textbf{2b)}\Rightarrow x\in\{-[a]\}$$
$$(\textbf{1a)}\land\textbf{1b)}\vee\textbf{2a)}\land\textbf{2b)})\land \text{dziedzina}\Rightarrow x\in(-[c],[b]]\cup\{-[a]\}-\{[d]\}$$
\rozwStop
\odpStart
$$x\in(-[c],[b]]\cup\{-[a]\}-\{[d]\}$$
\odpStop
\testStart
A.$$x\in(-[c],[b]]\cup\{-[a]\}-\{[d]\}$$
B.$$x\in(-[c],[b]]\cup\{-[a]\}$$
C.$$x\in([d],[a]]$$
D.$$x\in(-[b],[b]]$$
\testStop
\kluczStart
A
\kluczStop
\end{document}