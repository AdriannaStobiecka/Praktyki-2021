\documentclass[12pt, a4paper]{article}
\usepackage[utf8]{inputenc}
\usepackage{polski}
\usepackage{amsthm}  %pakiet do tworzenia twierdzeń itp.
\usepackage{amsmath} %pakiet do niektórych symboli matematycznych
\usepackage{amssymb} %pakiet do symboli mat., np. \nsubseteq
\usepackage{amsfonts}
\usepackage{graphicx} %obsługa plików graficznych z rozszerzeniem png, jpg
\theoremstyle{definition} %styl dla definicji
\newtheorem{zad}{} 
\title{Multizestaw zadań}
\author{Patryk Wirkus}
%\date{\today}
\date{}
\newcommand{\kategoria}[1]{\section{#1}}
\newcommand{\zadStart}[1]{\begin{zad}#1\newline}
\newcommand{\zadStop}{\end{zad}}
\newcommand{\rozwStart}[2]{\noindent \textbf{Rozwiązanie (autor #1 , recenzent #2): }\newline}
\newcommand{\rozwStop}{\newline}                                           
\newcommand{\odpStart}{\noindent \textbf{Odpowiedź:}\newline}
\newcommand{\odpStop}{\newline}
\newcommand{\testStart}{\noindent \textbf{Test:}\newline}
\newcommand{\testStop}{\newline}
\newcommand{\kluczStart}{\noindent \textbf{Test poprawna odpowiedź:}\newline}
\newcommand{\kluczStop}{\newline}
\newcommand{\wstawGrafike}[2]{\begin{figure}[h] \includegraphics[scale=#2] {#1} \end{figure}}

\begin{document}
\maketitle

\kategoria{Wikieł/1.62b}


\zadStart{Zadanie z Wikieł Z 1.62 b) moja wersja nr 1}

Rozwiązać nierówności $(x+3)(2-x)(x+1)\ge0$.
\zadStop
\rozwStart{Patryk Wirkus}{}
Miejsca zerowe naszego wielomianu to: $-3, 2, -1$.\\
Wielomian jest stopnia nieparzystego, ponadto znak współczynnika przy\linebreak najwyższej potędze x jest ujemny.\\ W związku z tym wykres wielomianu zaczyna się od lewej strony powyżej osi OX. A więc $$x \in (-\infty,-3) \cup (-1,2).$$
\rozwStop
\odpStart
$x \in (-\infty,-3) \cup (-1,2)$
\odpStop
\testStart
A.$x \in (-\infty,-3) \cup (-1,2)$\\
B.$x \in (-\infty,-3) \cup (-1,2]$\\
C.$x \in (-\infty,-3) \cup [-1,2)$\\
D.$x \in (-\infty,-3] \cup (-1,2)$\\
E.$x \in (-\infty,-3] \cup (-1,2]$\\
F.$x \in (-\infty,-3] \cup [-1,2)$\\
G.$x \in (-\infty,-3) \cup [-1,2]$\\
H.$x \in (-\infty,-3] \cup [-1,2]$
\testStop
\kluczStart
A
\kluczStop



\zadStart{Zadanie z Wikieł Z 1.62 b) moja wersja nr 2}

Rozwiązać nierówności $(x+4)(2-x)(x+1)\ge0$.
\zadStop
\rozwStart{Patryk Wirkus}{}
Miejsca zerowe naszego wielomianu to: $-4, 2, -1$.\\
Wielomian jest stopnia nieparzystego, ponadto znak współczynnika przy\linebreak najwyższej potędze x jest ujemny.\\ W związku z tym wykres wielomianu zaczyna się od lewej strony powyżej osi OX. A więc $$x \in (-\infty,-4) \cup (-1,2).$$
\rozwStop
\odpStart
$x \in (-\infty,-4) \cup (-1,2)$
\odpStop
\testStart
A.$x \in (-\infty,-4) \cup (-1,2)$\\
B.$x \in (-\infty,-4) \cup (-1,2]$\\
C.$x \in (-\infty,-4) \cup [-1,2)$\\
D.$x \in (-\infty,-4] \cup (-1,2)$\\
E.$x \in (-\infty,-4] \cup (-1,2]$\\
F.$x \in (-\infty,-4] \cup [-1,2)$\\
G.$x \in (-\infty,-4) \cup [-1,2]$\\
H.$x \in (-\infty,-4] \cup [-1,2]$
\testStop
\kluczStart
A
\kluczStop



\zadStart{Zadanie z Wikieł Z 1.62 b) moja wersja nr 3}

Rozwiązać nierówności $(x+4)(3-x)(x+1)\ge0$.
\zadStop
\rozwStart{Patryk Wirkus}{}
Miejsca zerowe naszego wielomianu to: $-4, 3, -1$.\\
Wielomian jest stopnia nieparzystego, ponadto znak współczynnika przy\linebreak najwyższej potędze x jest ujemny.\\ W związku z tym wykres wielomianu zaczyna się od lewej strony powyżej osi OX. A więc $$x \in (-\infty,-4) \cup (-1,3).$$
\rozwStop
\odpStart
$x \in (-\infty,-4) \cup (-1,3)$
\odpStop
\testStart
A.$x \in (-\infty,-4) \cup (-1,3)$\\
B.$x \in (-\infty,-4) \cup (-1,3]$\\
C.$x \in (-\infty,-4) \cup [-1,3)$\\
D.$x \in (-\infty,-4] \cup (-1,3)$\\
E.$x \in (-\infty,-4] \cup (-1,3]$\\
F.$x \in (-\infty,-4] \cup [-1,3)$\\
G.$x \in (-\infty,-4) \cup [-1,3]$\\
H.$x \in (-\infty,-4] \cup [-1,3]$
\testStop
\kluczStart
A
\kluczStop



\zadStart{Zadanie z Wikieł Z 1.62 b) moja wersja nr 4}

Rozwiązać nierówności $(x+4)(3-x)(x+2)\ge0$.
\zadStop
\rozwStart{Patryk Wirkus}{}
Miejsca zerowe naszego wielomianu to: $-4, 3, -2$.\\
Wielomian jest stopnia nieparzystego, ponadto znak współczynnika przy\linebreak najwyższej potędze x jest ujemny.\\ W związku z tym wykres wielomianu zaczyna się od lewej strony powyżej osi OX. A więc $$x \in (-\infty,-4) \cup (-2,3).$$
\rozwStop
\odpStart
$x \in (-\infty,-4) \cup (-2,3)$
\odpStop
\testStart
A.$x \in (-\infty,-4) \cup (-2,3)$\\
B.$x \in (-\infty,-4) \cup (-2,3]$\\
C.$x \in (-\infty,-4) \cup [-2,3)$\\
D.$x \in (-\infty,-4] \cup (-2,3)$\\
E.$x \in (-\infty,-4] \cup (-2,3]$\\
F.$x \in (-\infty,-4] \cup [-2,3)$\\
G.$x \in (-\infty,-4) \cup [-2,3]$\\
H.$x \in (-\infty,-4] \cup [-2,3]$
\testStop
\kluczStart
A
\kluczStop



\zadStart{Zadanie z Wikieł Z 1.62 b) moja wersja nr 5}

Rozwiązać nierówności $(x+5)(2-x)(x+1)\ge0$.
\zadStop
\rozwStart{Patryk Wirkus}{}
Miejsca zerowe naszego wielomianu to: $-5, 2, -1$.\\
Wielomian jest stopnia nieparzystego, ponadto znak współczynnika przy\linebreak najwyższej potędze x jest ujemny.\\ W związku z tym wykres wielomianu zaczyna się od lewej strony powyżej osi OX. A więc $$x \in (-\infty,-5) \cup (-1,2).$$
\rozwStop
\odpStart
$x \in (-\infty,-5) \cup (-1,2)$
\odpStop
\testStart
A.$x \in (-\infty,-5) \cup (-1,2)$\\
B.$x \in (-\infty,-5) \cup (-1,2]$\\
C.$x \in (-\infty,-5) \cup [-1,2)$\\
D.$x \in (-\infty,-5] \cup (-1,2)$\\
E.$x \in (-\infty,-5] \cup (-1,2]$\\
F.$x \in (-\infty,-5] \cup [-1,2)$\\
G.$x \in (-\infty,-5) \cup [-1,2]$\\
H.$x \in (-\infty,-5] \cup [-1,2]$
\testStop
\kluczStart
A
\kluczStop



\zadStart{Zadanie z Wikieł Z 1.62 b) moja wersja nr 6}

Rozwiązać nierówności $(x+5)(3-x)(x+1)\ge0$.
\zadStop
\rozwStart{Patryk Wirkus}{}
Miejsca zerowe naszego wielomianu to: $-5, 3, -1$.\\
Wielomian jest stopnia nieparzystego, ponadto znak współczynnika przy\linebreak najwyższej potędze x jest ujemny.\\ W związku z tym wykres wielomianu zaczyna się od lewej strony powyżej osi OX. A więc $$x \in (-\infty,-5) \cup (-1,3).$$
\rozwStop
\odpStart
$x \in (-\infty,-5) \cup (-1,3)$
\odpStop
\testStart
A.$x \in (-\infty,-5) \cup (-1,3)$\\
B.$x \in (-\infty,-5) \cup (-1,3]$\\
C.$x \in (-\infty,-5) \cup [-1,3)$\\
D.$x \in (-\infty,-5] \cup (-1,3)$\\
E.$x \in (-\infty,-5] \cup (-1,3]$\\
F.$x \in (-\infty,-5] \cup [-1,3)$\\
G.$x \in (-\infty,-5) \cup [-1,3]$\\
H.$x \in (-\infty,-5] \cup [-1,3]$
\testStop
\kluczStart
A
\kluczStop



\zadStart{Zadanie z Wikieł Z 1.62 b) moja wersja nr 7}

Rozwiązać nierówności $(x+5)(3-x)(x+2)\ge0$.
\zadStop
\rozwStart{Patryk Wirkus}{}
Miejsca zerowe naszego wielomianu to: $-5, 3, -2$.\\
Wielomian jest stopnia nieparzystego, ponadto znak współczynnika przy\linebreak najwyższej potędze x jest ujemny.\\ W związku z tym wykres wielomianu zaczyna się od lewej strony powyżej osi OX. A więc $$x \in (-\infty,-5) \cup (-2,3).$$
\rozwStop
\odpStart
$x \in (-\infty,-5) \cup (-2,3)$
\odpStop
\testStart
A.$x \in (-\infty,-5) \cup (-2,3)$\\
B.$x \in (-\infty,-5) \cup (-2,3]$\\
C.$x \in (-\infty,-5) \cup [-2,3)$\\
D.$x \in (-\infty,-5] \cup (-2,3)$\\
E.$x \in (-\infty,-5] \cup (-2,3]$\\
F.$x \in (-\infty,-5] \cup [-2,3)$\\
G.$x \in (-\infty,-5) \cup [-2,3]$\\
H.$x \in (-\infty,-5] \cup [-2,3]$
\testStop
\kluczStart
A
\kluczStop



\zadStart{Zadanie z Wikieł Z 1.62 b) moja wersja nr 8}

Rozwiązać nierówności $(x+5)(4-x)(x+1)\ge0$.
\zadStop
\rozwStart{Patryk Wirkus}{}
Miejsca zerowe naszego wielomianu to: $-5, 4, -1$.\\
Wielomian jest stopnia nieparzystego, ponadto znak współczynnika przy\linebreak najwyższej potędze x jest ujemny.\\ W związku z tym wykres wielomianu zaczyna się od lewej strony powyżej osi OX. A więc $$x \in (-\infty,-5) \cup (-1,4).$$
\rozwStop
\odpStart
$x \in (-\infty,-5) \cup (-1,4)$
\odpStop
\testStart
A.$x \in (-\infty,-5) \cup (-1,4)$\\
B.$x \in (-\infty,-5) \cup (-1,4]$\\
C.$x \in (-\infty,-5) \cup [-1,4)$\\
D.$x \in (-\infty,-5] \cup (-1,4)$\\
E.$x \in (-\infty,-5] \cup (-1,4]$\\
F.$x \in (-\infty,-5] \cup [-1,4)$\\
G.$x \in (-\infty,-5) \cup [-1,4]$\\
H.$x \in (-\infty,-5] \cup [-1,4]$
\testStop
\kluczStart
A
\kluczStop



\zadStart{Zadanie z Wikieł Z 1.62 b) moja wersja nr 9}

Rozwiązać nierówności $(x+5)(4-x)(x+2)\ge0$.
\zadStop
\rozwStart{Patryk Wirkus}{}
Miejsca zerowe naszego wielomianu to: $-5, 4, -2$.\\
Wielomian jest stopnia nieparzystego, ponadto znak współczynnika przy\linebreak najwyższej potędze x jest ujemny.\\ W związku z tym wykres wielomianu zaczyna się od lewej strony powyżej osi OX. A więc $$x \in (-\infty,-5) \cup (-2,4).$$
\rozwStop
\odpStart
$x \in (-\infty,-5) \cup (-2,4)$
\odpStop
\testStart
A.$x \in (-\infty,-5) \cup (-2,4)$\\
B.$x \in (-\infty,-5) \cup (-2,4]$\\
C.$x \in (-\infty,-5) \cup [-2,4)$\\
D.$x \in (-\infty,-5] \cup (-2,4)$\\
E.$x \in (-\infty,-5] \cup (-2,4]$\\
F.$x \in (-\infty,-5] \cup [-2,4)$\\
G.$x \in (-\infty,-5) \cup [-2,4]$\\
H.$x \in (-\infty,-5] \cup [-2,4]$
\testStop
\kluczStart
A
\kluczStop



\zadStart{Zadanie z Wikieł Z 1.62 b) moja wersja nr 10}

Rozwiązać nierówności $(x+5)(4-x)(x+3)\ge0$.
\zadStop
\rozwStart{Patryk Wirkus}{}
Miejsca zerowe naszego wielomianu to: $-5, 4, -3$.\\
Wielomian jest stopnia nieparzystego, ponadto znak współczynnika przy\linebreak najwyższej potędze x jest ujemny.\\ W związku z tym wykres wielomianu zaczyna się od lewej strony powyżej osi OX. A więc $$x \in (-\infty,-5) \cup (-3,4).$$
\rozwStop
\odpStart
$x \in (-\infty,-5) \cup (-3,4)$
\odpStop
\testStart
A.$x \in (-\infty,-5) \cup (-3,4)$\\
B.$x \in (-\infty,-5) \cup (-3,4]$\\
C.$x \in (-\infty,-5) \cup [-3,4)$\\
D.$x \in (-\infty,-5] \cup (-3,4)$\\
E.$x \in (-\infty,-5] \cup (-3,4]$\\
F.$x \in (-\infty,-5] \cup [-3,4)$\\
G.$x \in (-\infty,-5) \cup [-3,4]$\\
H.$x \in (-\infty,-5] \cup [-3,4]$
\testStop
\kluczStart
A
\kluczStop



\zadStart{Zadanie z Wikieł Z 1.62 b) moja wersja nr 11}

Rozwiązać nierówności $(x+6)(2-x)(x+1)\ge0$.
\zadStop
\rozwStart{Patryk Wirkus}{}
Miejsca zerowe naszego wielomianu to: $-6, 2, -1$.\\
Wielomian jest stopnia nieparzystego, ponadto znak współczynnika przy\linebreak najwyższej potędze x jest ujemny.\\ W związku z tym wykres wielomianu zaczyna się od lewej strony powyżej osi OX. A więc $$x \in (-\infty,-6) \cup (-1,2).$$
\rozwStop
\odpStart
$x \in (-\infty,-6) \cup (-1,2)$
\odpStop
\testStart
A.$x \in (-\infty,-6) \cup (-1,2)$\\
B.$x \in (-\infty,-6) \cup (-1,2]$\\
C.$x \in (-\infty,-6) \cup [-1,2)$\\
D.$x \in (-\infty,-6] \cup (-1,2)$\\
E.$x \in (-\infty,-6] \cup (-1,2]$\\
F.$x \in (-\infty,-6] \cup [-1,2)$\\
G.$x \in (-\infty,-6) \cup [-1,2]$\\
H.$x \in (-\infty,-6] \cup [-1,2]$
\testStop
\kluczStart
A
\kluczStop



\zadStart{Zadanie z Wikieł Z 1.62 b) moja wersja nr 12}

Rozwiązać nierówności $(x+6)(3-x)(x+1)\ge0$.
\zadStop
\rozwStart{Patryk Wirkus}{}
Miejsca zerowe naszego wielomianu to: $-6, 3, -1$.\\
Wielomian jest stopnia nieparzystego, ponadto znak współczynnika przy\linebreak najwyższej potędze x jest ujemny.\\ W związku z tym wykres wielomianu zaczyna się od lewej strony powyżej osi OX. A więc $$x \in (-\infty,-6) \cup (-1,3).$$
\rozwStop
\odpStart
$x \in (-\infty,-6) \cup (-1,3)$
\odpStop
\testStart
A.$x \in (-\infty,-6) \cup (-1,3)$\\
B.$x \in (-\infty,-6) \cup (-1,3]$\\
C.$x \in (-\infty,-6) \cup [-1,3)$\\
D.$x \in (-\infty,-6] \cup (-1,3)$\\
E.$x \in (-\infty,-6] \cup (-1,3]$\\
F.$x \in (-\infty,-6] \cup [-1,3)$\\
G.$x \in (-\infty,-6) \cup [-1,3]$\\
H.$x \in (-\infty,-6] \cup [-1,3]$
\testStop
\kluczStart
A
\kluczStop



\zadStart{Zadanie z Wikieł Z 1.62 b) moja wersja nr 13}

Rozwiązać nierówności $(x+6)(3-x)(x+2)\ge0$.
\zadStop
\rozwStart{Patryk Wirkus}{}
Miejsca zerowe naszego wielomianu to: $-6, 3, -2$.\\
Wielomian jest stopnia nieparzystego, ponadto znak współczynnika przy\linebreak najwyższej potędze x jest ujemny.\\ W związku z tym wykres wielomianu zaczyna się od lewej strony powyżej osi OX. A więc $$x \in (-\infty,-6) \cup (-2,3).$$
\rozwStop
\odpStart
$x \in (-\infty,-6) \cup (-2,3)$
\odpStop
\testStart
A.$x \in (-\infty,-6) \cup (-2,3)$\\
B.$x \in (-\infty,-6) \cup (-2,3]$\\
C.$x \in (-\infty,-6) \cup [-2,3)$\\
D.$x \in (-\infty,-6] \cup (-2,3)$\\
E.$x \in (-\infty,-6] \cup (-2,3]$\\
F.$x \in (-\infty,-6] \cup [-2,3)$\\
G.$x \in (-\infty,-6) \cup [-2,3]$\\
H.$x \in (-\infty,-6] \cup [-2,3]$
\testStop
\kluczStart
A
\kluczStop



\zadStart{Zadanie z Wikieł Z 1.62 b) moja wersja nr 14}

Rozwiązać nierówności $(x+6)(4-x)(x+1)\ge0$.
\zadStop
\rozwStart{Patryk Wirkus}{}
Miejsca zerowe naszego wielomianu to: $-6, 4, -1$.\\
Wielomian jest stopnia nieparzystego, ponadto znak współczynnika przy\linebreak najwyższej potędze x jest ujemny.\\ W związku z tym wykres wielomianu zaczyna się od lewej strony powyżej osi OX. A więc $$x \in (-\infty,-6) \cup (-1,4).$$
\rozwStop
\odpStart
$x \in (-\infty,-6) \cup (-1,4)$
\odpStop
\testStart
A.$x \in (-\infty,-6) \cup (-1,4)$\\
B.$x \in (-\infty,-6) \cup (-1,4]$\\
C.$x \in (-\infty,-6) \cup [-1,4)$\\
D.$x \in (-\infty,-6] \cup (-1,4)$\\
E.$x \in (-\infty,-6] \cup (-1,4]$\\
F.$x \in (-\infty,-6] \cup [-1,4)$\\
G.$x \in (-\infty,-6) \cup [-1,4]$\\
H.$x \in (-\infty,-6] \cup [-1,4]$
\testStop
\kluczStart
A
\kluczStop



\zadStart{Zadanie z Wikieł Z 1.62 b) moja wersja nr 15}

Rozwiązać nierówności $(x+6)(4-x)(x+2)\ge0$.
\zadStop
\rozwStart{Patryk Wirkus}{}
Miejsca zerowe naszego wielomianu to: $-6, 4, -2$.\\
Wielomian jest stopnia nieparzystego, ponadto znak współczynnika przy\linebreak najwyższej potędze x jest ujemny.\\ W związku z tym wykres wielomianu zaczyna się od lewej strony powyżej osi OX. A więc $$x \in (-\infty,-6) \cup (-2,4).$$
\rozwStop
\odpStart
$x \in (-\infty,-6) \cup (-2,4)$
\odpStop
\testStart
A.$x \in (-\infty,-6) \cup (-2,4)$\\
B.$x \in (-\infty,-6) \cup (-2,4]$\\
C.$x \in (-\infty,-6) \cup [-2,4)$\\
D.$x \in (-\infty,-6] \cup (-2,4)$\\
E.$x \in (-\infty,-6] \cup (-2,4]$\\
F.$x \in (-\infty,-6] \cup [-2,4)$\\
G.$x \in (-\infty,-6) \cup [-2,4]$\\
H.$x \in (-\infty,-6] \cup [-2,4]$
\testStop
\kluczStart
A
\kluczStop



\zadStart{Zadanie z Wikieł Z 1.62 b) moja wersja nr 16}

Rozwiązać nierówności $(x+6)(4-x)(x+3)\ge0$.
\zadStop
\rozwStart{Patryk Wirkus}{}
Miejsca zerowe naszego wielomianu to: $-6, 4, -3$.\\
Wielomian jest stopnia nieparzystego, ponadto znak współczynnika przy\linebreak najwyższej potędze x jest ujemny.\\ W związku z tym wykres wielomianu zaczyna się od lewej strony powyżej osi OX. A więc $$x \in (-\infty,-6) \cup (-3,4).$$
\rozwStop
\odpStart
$x \in (-\infty,-6) \cup (-3,4)$
\odpStop
\testStart
A.$x \in (-\infty,-6) \cup (-3,4)$\\
B.$x \in (-\infty,-6) \cup (-3,4]$\\
C.$x \in (-\infty,-6) \cup [-3,4)$\\
D.$x \in (-\infty,-6] \cup (-3,4)$\\
E.$x \in (-\infty,-6] \cup (-3,4]$\\
F.$x \in (-\infty,-6] \cup [-3,4)$\\
G.$x \in (-\infty,-6) \cup [-3,4]$\\
H.$x \in (-\infty,-6] \cup [-3,4]$
\testStop
\kluczStart
A
\kluczStop



\zadStart{Zadanie z Wikieł Z 1.62 b) moja wersja nr 17}

Rozwiązać nierówności $(x+6)(5-x)(x+1)\ge0$.
\zadStop
\rozwStart{Patryk Wirkus}{}
Miejsca zerowe naszego wielomianu to: $-6, 5, -1$.\\
Wielomian jest stopnia nieparzystego, ponadto znak współczynnika przy\linebreak najwyższej potędze x jest ujemny.\\ W związku z tym wykres wielomianu zaczyna się od lewej strony powyżej osi OX. A więc $$x \in (-\infty,-6) \cup (-1,5).$$
\rozwStop
\odpStart
$x \in (-\infty,-6) \cup (-1,5)$
\odpStop
\testStart
A.$x \in (-\infty,-6) \cup (-1,5)$\\
B.$x \in (-\infty,-6) \cup (-1,5]$\\
C.$x \in (-\infty,-6) \cup [-1,5)$\\
D.$x \in (-\infty,-6] \cup (-1,5)$\\
E.$x \in (-\infty,-6] \cup (-1,5]$\\
F.$x \in (-\infty,-6] \cup [-1,5)$\\
G.$x \in (-\infty,-6) \cup [-1,5]$\\
H.$x \in (-\infty,-6] \cup [-1,5]$
\testStop
\kluczStart
A
\kluczStop



\zadStart{Zadanie z Wikieł Z 1.62 b) moja wersja nr 18}

Rozwiązać nierówności $(x+6)(5-x)(x+2)\ge0$.
\zadStop
\rozwStart{Patryk Wirkus}{}
Miejsca zerowe naszego wielomianu to: $-6, 5, -2$.\\
Wielomian jest stopnia nieparzystego, ponadto znak współczynnika przy\linebreak najwyższej potędze x jest ujemny.\\ W związku z tym wykres wielomianu zaczyna się od lewej strony powyżej osi OX. A więc $$x \in (-\infty,-6) \cup (-2,5).$$
\rozwStop
\odpStart
$x \in (-\infty,-6) \cup (-2,5)$
\odpStop
\testStart
A.$x \in (-\infty,-6) \cup (-2,5)$\\
B.$x \in (-\infty,-6) \cup (-2,5]$\\
C.$x \in (-\infty,-6) \cup [-2,5)$\\
D.$x \in (-\infty,-6] \cup (-2,5)$\\
E.$x \in (-\infty,-6] \cup (-2,5]$\\
F.$x \in (-\infty,-6] \cup [-2,5)$\\
G.$x \in (-\infty,-6) \cup [-2,5]$\\
H.$x \in (-\infty,-6] \cup [-2,5]$
\testStop
\kluczStart
A
\kluczStop



\zadStart{Zadanie z Wikieł Z 1.62 b) moja wersja nr 19}

Rozwiązać nierówności $(x+6)(5-x)(x+3)\ge0$.
\zadStop
\rozwStart{Patryk Wirkus}{}
Miejsca zerowe naszego wielomianu to: $-6, 5, -3$.\\
Wielomian jest stopnia nieparzystego, ponadto znak współczynnika przy\linebreak najwyższej potędze x jest ujemny.\\ W związku z tym wykres wielomianu zaczyna się od lewej strony powyżej osi OX. A więc $$x \in (-\infty,-6) \cup (-3,5).$$
\rozwStop
\odpStart
$x \in (-\infty,-6) \cup (-3,5)$
\odpStop
\testStart
A.$x \in (-\infty,-6) \cup (-3,5)$\\
B.$x \in (-\infty,-6) \cup (-3,5]$\\
C.$x \in (-\infty,-6) \cup [-3,5)$\\
D.$x \in (-\infty,-6] \cup (-3,5)$\\
E.$x \in (-\infty,-6] \cup (-3,5]$\\
F.$x \in (-\infty,-6] \cup [-3,5)$\\
G.$x \in (-\infty,-6) \cup [-3,5]$\\
H.$x \in (-\infty,-6] \cup [-3,5]$
\testStop
\kluczStart
A
\kluczStop



\zadStart{Zadanie z Wikieł Z 1.62 b) moja wersja nr 20}

Rozwiązać nierówności $(x+6)(5-x)(x+4)\ge0$.
\zadStop
\rozwStart{Patryk Wirkus}{}
Miejsca zerowe naszego wielomianu to: $-6, 5, -4$.\\
Wielomian jest stopnia nieparzystego, ponadto znak współczynnika przy\linebreak najwyższej potędze x jest ujemny.\\ W związku z tym wykres wielomianu zaczyna się od lewej strony powyżej osi OX. A więc $$x \in (-\infty,-6) \cup (-4,5).$$
\rozwStop
\odpStart
$x \in (-\infty,-6) \cup (-4,5)$
\odpStop
\testStart
A.$x \in (-\infty,-6) \cup (-4,5)$\\
B.$x \in (-\infty,-6) \cup (-4,5]$\\
C.$x \in (-\infty,-6) \cup [-4,5)$\\
D.$x \in (-\infty,-6] \cup (-4,5)$\\
E.$x \in (-\infty,-6] \cup (-4,5]$\\
F.$x \in (-\infty,-6] \cup [-4,5)$\\
G.$x \in (-\infty,-6) \cup [-4,5]$\\
H.$x \in (-\infty,-6] \cup [-4,5]$
\testStop
\kluczStart
A
\kluczStop



\zadStart{Zadanie z Wikieł Z 1.62 b) moja wersja nr 21}

Rozwiązać nierówności $(x+7)(2-x)(x+1)\ge0$.
\zadStop
\rozwStart{Patryk Wirkus}{}
Miejsca zerowe naszego wielomianu to: $-7, 2, -1$.\\
Wielomian jest stopnia nieparzystego, ponadto znak współczynnika przy\linebreak najwyższej potędze x jest ujemny.\\ W związku z tym wykres wielomianu zaczyna się od lewej strony powyżej osi OX. A więc $$x \in (-\infty,-7) \cup (-1,2).$$
\rozwStop
\odpStart
$x \in (-\infty,-7) \cup (-1,2)$
\odpStop
\testStart
A.$x \in (-\infty,-7) \cup (-1,2)$\\
B.$x \in (-\infty,-7) \cup (-1,2]$\\
C.$x \in (-\infty,-7) \cup [-1,2)$\\
D.$x \in (-\infty,-7] \cup (-1,2)$\\
E.$x \in (-\infty,-7] \cup (-1,2]$\\
F.$x \in (-\infty,-7] \cup [-1,2)$\\
G.$x \in (-\infty,-7) \cup [-1,2]$\\
H.$x \in (-\infty,-7] \cup [-1,2]$
\testStop
\kluczStart
A
\kluczStop



\zadStart{Zadanie z Wikieł Z 1.62 b) moja wersja nr 22}

Rozwiązać nierówności $(x+7)(3-x)(x+1)\ge0$.
\zadStop
\rozwStart{Patryk Wirkus}{}
Miejsca zerowe naszego wielomianu to: $-7, 3, -1$.\\
Wielomian jest stopnia nieparzystego, ponadto znak współczynnika przy\linebreak najwyższej potędze x jest ujemny.\\ W związku z tym wykres wielomianu zaczyna się od lewej strony powyżej osi OX. A więc $$x \in (-\infty,-7) \cup (-1,3).$$
\rozwStop
\odpStart
$x \in (-\infty,-7) \cup (-1,3)$
\odpStop
\testStart
A.$x \in (-\infty,-7) \cup (-1,3)$\\
B.$x \in (-\infty,-7) \cup (-1,3]$\\
C.$x \in (-\infty,-7) \cup [-1,3)$\\
D.$x \in (-\infty,-7] \cup (-1,3)$\\
E.$x \in (-\infty,-7] \cup (-1,3]$\\
F.$x \in (-\infty,-7] \cup [-1,3)$\\
G.$x \in (-\infty,-7) \cup [-1,3]$\\
H.$x \in (-\infty,-7] \cup [-1,3]$
\testStop
\kluczStart
A
\kluczStop



\zadStart{Zadanie z Wikieł Z 1.62 b) moja wersja nr 23}

Rozwiązać nierówności $(x+7)(3-x)(x+2)\ge0$.
\zadStop
\rozwStart{Patryk Wirkus}{}
Miejsca zerowe naszego wielomianu to: $-7, 3, -2$.\\
Wielomian jest stopnia nieparzystego, ponadto znak współczynnika przy\linebreak najwyższej potędze x jest ujemny.\\ W związku z tym wykres wielomianu zaczyna się od lewej strony powyżej osi OX. A więc $$x \in (-\infty,-7) \cup (-2,3).$$
\rozwStop
\odpStart
$x \in (-\infty,-7) \cup (-2,3)$
\odpStop
\testStart
A.$x \in (-\infty,-7) \cup (-2,3)$\\
B.$x \in (-\infty,-7) \cup (-2,3]$\\
C.$x \in (-\infty,-7) \cup [-2,3)$\\
D.$x \in (-\infty,-7] \cup (-2,3)$\\
E.$x \in (-\infty,-7] \cup (-2,3]$\\
F.$x \in (-\infty,-7] \cup [-2,3)$\\
G.$x \in (-\infty,-7) \cup [-2,3]$\\
H.$x \in (-\infty,-7] \cup [-2,3]$
\testStop
\kluczStart
A
\kluczStop



\zadStart{Zadanie z Wikieł Z 1.62 b) moja wersja nr 24}

Rozwiązać nierówności $(x+7)(4-x)(x+1)\ge0$.
\zadStop
\rozwStart{Patryk Wirkus}{}
Miejsca zerowe naszego wielomianu to: $-7, 4, -1$.\\
Wielomian jest stopnia nieparzystego, ponadto znak współczynnika przy\linebreak najwyższej potędze x jest ujemny.\\ W związku z tym wykres wielomianu zaczyna się od lewej strony powyżej osi OX. A więc $$x \in (-\infty,-7) \cup (-1,4).$$
\rozwStop
\odpStart
$x \in (-\infty,-7) \cup (-1,4)$
\odpStop
\testStart
A.$x \in (-\infty,-7) \cup (-1,4)$\\
B.$x \in (-\infty,-7) \cup (-1,4]$\\
C.$x \in (-\infty,-7) \cup [-1,4)$\\
D.$x \in (-\infty,-7] \cup (-1,4)$\\
E.$x \in (-\infty,-7] \cup (-1,4]$\\
F.$x \in (-\infty,-7] \cup [-1,4)$\\
G.$x \in (-\infty,-7) \cup [-1,4]$\\
H.$x \in (-\infty,-7] \cup [-1,4]$
\testStop
\kluczStart
A
\kluczStop



\zadStart{Zadanie z Wikieł Z 1.62 b) moja wersja nr 25}

Rozwiązać nierówności $(x+7)(4-x)(x+2)\ge0$.
\zadStop
\rozwStart{Patryk Wirkus}{}
Miejsca zerowe naszego wielomianu to: $-7, 4, -2$.\\
Wielomian jest stopnia nieparzystego, ponadto znak współczynnika przy\linebreak najwyższej potędze x jest ujemny.\\ W związku z tym wykres wielomianu zaczyna się od lewej strony powyżej osi OX. A więc $$x \in (-\infty,-7) \cup (-2,4).$$
\rozwStop
\odpStart
$x \in (-\infty,-7) \cup (-2,4)$
\odpStop
\testStart
A.$x \in (-\infty,-7) \cup (-2,4)$\\
B.$x \in (-\infty,-7) \cup (-2,4]$\\
C.$x \in (-\infty,-7) \cup [-2,4)$\\
D.$x \in (-\infty,-7] \cup (-2,4)$\\
E.$x \in (-\infty,-7] \cup (-2,4]$\\
F.$x \in (-\infty,-7] \cup [-2,4)$\\
G.$x \in (-\infty,-7) \cup [-2,4]$\\
H.$x \in (-\infty,-7] \cup [-2,4]$
\testStop
\kluczStart
A
\kluczStop



\zadStart{Zadanie z Wikieł Z 1.62 b) moja wersja nr 26}

Rozwiązać nierówności $(x+7)(4-x)(x+3)\ge0$.
\zadStop
\rozwStart{Patryk Wirkus}{}
Miejsca zerowe naszego wielomianu to: $-7, 4, -3$.\\
Wielomian jest stopnia nieparzystego, ponadto znak współczynnika przy\linebreak najwyższej potędze x jest ujemny.\\ W związku z tym wykres wielomianu zaczyna się od lewej strony powyżej osi OX. A więc $$x \in (-\infty,-7) \cup (-3,4).$$
\rozwStop
\odpStart
$x \in (-\infty,-7) \cup (-3,4)$
\odpStop
\testStart
A.$x \in (-\infty,-7) \cup (-3,4)$\\
B.$x \in (-\infty,-7) \cup (-3,4]$\\
C.$x \in (-\infty,-7) \cup [-3,4)$\\
D.$x \in (-\infty,-7] \cup (-3,4)$\\
E.$x \in (-\infty,-7] \cup (-3,4]$\\
F.$x \in (-\infty,-7] \cup [-3,4)$\\
G.$x \in (-\infty,-7) \cup [-3,4]$\\
H.$x \in (-\infty,-7] \cup [-3,4]$
\testStop
\kluczStart
A
\kluczStop



\zadStart{Zadanie z Wikieł Z 1.62 b) moja wersja nr 27}

Rozwiązać nierówności $(x+7)(5-x)(x+1)\ge0$.
\zadStop
\rozwStart{Patryk Wirkus}{}
Miejsca zerowe naszego wielomianu to: $-7, 5, -1$.\\
Wielomian jest stopnia nieparzystego, ponadto znak współczynnika przy\linebreak najwyższej potędze x jest ujemny.\\ W związku z tym wykres wielomianu zaczyna się od lewej strony powyżej osi OX. A więc $$x \in (-\infty,-7) \cup (-1,5).$$
\rozwStop
\odpStart
$x \in (-\infty,-7) \cup (-1,5)$
\odpStop
\testStart
A.$x \in (-\infty,-7) \cup (-1,5)$\\
B.$x \in (-\infty,-7) \cup (-1,5]$\\
C.$x \in (-\infty,-7) \cup [-1,5)$\\
D.$x \in (-\infty,-7] \cup (-1,5)$\\
E.$x \in (-\infty,-7] \cup (-1,5]$\\
F.$x \in (-\infty,-7] \cup [-1,5)$\\
G.$x \in (-\infty,-7) \cup [-1,5]$\\
H.$x \in (-\infty,-7] \cup [-1,5]$
\testStop
\kluczStart
A
\kluczStop



\zadStart{Zadanie z Wikieł Z 1.62 b) moja wersja nr 28}

Rozwiązać nierówności $(x+7)(5-x)(x+2)\ge0$.
\zadStop
\rozwStart{Patryk Wirkus}{}
Miejsca zerowe naszego wielomianu to: $-7, 5, -2$.\\
Wielomian jest stopnia nieparzystego, ponadto znak współczynnika przy\linebreak najwyższej potędze x jest ujemny.\\ W związku z tym wykres wielomianu zaczyna się od lewej strony powyżej osi OX. A więc $$x \in (-\infty,-7) \cup (-2,5).$$
\rozwStop
\odpStart
$x \in (-\infty,-7) \cup (-2,5)$
\odpStop
\testStart
A.$x \in (-\infty,-7) \cup (-2,5)$\\
B.$x \in (-\infty,-7) \cup (-2,5]$\\
C.$x \in (-\infty,-7) \cup [-2,5)$\\
D.$x \in (-\infty,-7] \cup (-2,5)$\\
E.$x \in (-\infty,-7] \cup (-2,5]$\\
F.$x \in (-\infty,-7] \cup [-2,5)$\\
G.$x \in (-\infty,-7) \cup [-2,5]$\\
H.$x \in (-\infty,-7] \cup [-2,5]$
\testStop
\kluczStart
A
\kluczStop



\zadStart{Zadanie z Wikieł Z 1.62 b) moja wersja nr 29}

Rozwiązać nierówności $(x+7)(5-x)(x+3)\ge0$.
\zadStop
\rozwStart{Patryk Wirkus}{}
Miejsca zerowe naszego wielomianu to: $-7, 5, -3$.\\
Wielomian jest stopnia nieparzystego, ponadto znak współczynnika przy\linebreak najwyższej potędze x jest ujemny.\\ W związku z tym wykres wielomianu zaczyna się od lewej strony powyżej osi OX. A więc $$x \in (-\infty,-7) \cup (-3,5).$$
\rozwStop
\odpStart
$x \in (-\infty,-7) \cup (-3,5)$
\odpStop
\testStart
A.$x \in (-\infty,-7) \cup (-3,5)$\\
B.$x \in (-\infty,-7) \cup (-3,5]$\\
C.$x \in (-\infty,-7) \cup [-3,5)$\\
D.$x \in (-\infty,-7] \cup (-3,5)$\\
E.$x \in (-\infty,-7] \cup (-3,5]$\\
F.$x \in (-\infty,-7] \cup [-3,5)$\\
G.$x \in (-\infty,-7) \cup [-3,5]$\\
H.$x \in (-\infty,-7] \cup [-3,5]$
\testStop
\kluczStart
A
\kluczStop



\zadStart{Zadanie z Wikieł Z 1.62 b) moja wersja nr 30}

Rozwiązać nierówności $(x+7)(5-x)(x+4)\ge0$.
\zadStop
\rozwStart{Patryk Wirkus}{}
Miejsca zerowe naszego wielomianu to: $-7, 5, -4$.\\
Wielomian jest stopnia nieparzystego, ponadto znak współczynnika przy\linebreak najwyższej potędze x jest ujemny.\\ W związku z tym wykres wielomianu zaczyna się od lewej strony powyżej osi OX. A więc $$x \in (-\infty,-7) \cup (-4,5).$$
\rozwStop
\odpStart
$x \in (-\infty,-7) \cup (-4,5)$
\odpStop
\testStart
A.$x \in (-\infty,-7) \cup (-4,5)$\\
B.$x \in (-\infty,-7) \cup (-4,5]$\\
C.$x \in (-\infty,-7) \cup [-4,5)$\\
D.$x \in (-\infty,-7] \cup (-4,5)$\\
E.$x \in (-\infty,-7] \cup (-4,5]$\\
F.$x \in (-\infty,-7] \cup [-4,5)$\\
G.$x \in (-\infty,-7) \cup [-4,5]$\\
H.$x \in (-\infty,-7] \cup [-4,5]$
\testStop
\kluczStart
A
\kluczStop



\zadStart{Zadanie z Wikieł Z 1.62 b) moja wersja nr 31}

Rozwiązać nierówności $(x+7)(6-x)(x+1)\ge0$.
\zadStop
\rozwStart{Patryk Wirkus}{}
Miejsca zerowe naszego wielomianu to: $-7, 6, -1$.\\
Wielomian jest stopnia nieparzystego, ponadto znak współczynnika przy\linebreak najwyższej potędze x jest ujemny.\\ W związku z tym wykres wielomianu zaczyna się od lewej strony powyżej osi OX. A więc $$x \in (-\infty,-7) \cup (-1,6).$$
\rozwStop
\odpStart
$x \in (-\infty,-7) \cup (-1,6)$
\odpStop
\testStart
A.$x \in (-\infty,-7) \cup (-1,6)$\\
B.$x \in (-\infty,-7) \cup (-1,6]$\\
C.$x \in (-\infty,-7) \cup [-1,6)$\\
D.$x \in (-\infty,-7] \cup (-1,6)$\\
E.$x \in (-\infty,-7] \cup (-1,6]$\\
F.$x \in (-\infty,-7] \cup [-1,6)$\\
G.$x \in (-\infty,-7) \cup [-1,6]$\\
H.$x \in (-\infty,-7] \cup [-1,6]$
\testStop
\kluczStart
A
\kluczStop



\zadStart{Zadanie z Wikieł Z 1.62 b) moja wersja nr 32}

Rozwiązać nierówności $(x+7)(6-x)(x+2)\ge0$.
\zadStop
\rozwStart{Patryk Wirkus}{}
Miejsca zerowe naszego wielomianu to: $-7, 6, -2$.\\
Wielomian jest stopnia nieparzystego, ponadto znak współczynnika przy\linebreak najwyższej potędze x jest ujemny.\\ W związku z tym wykres wielomianu zaczyna się od lewej strony powyżej osi OX. A więc $$x \in (-\infty,-7) \cup (-2,6).$$
\rozwStop
\odpStart
$x \in (-\infty,-7) \cup (-2,6)$
\odpStop
\testStart
A.$x \in (-\infty,-7) \cup (-2,6)$\\
B.$x \in (-\infty,-7) \cup (-2,6]$\\
C.$x \in (-\infty,-7) \cup [-2,6)$\\
D.$x \in (-\infty,-7] \cup (-2,6)$\\
E.$x \in (-\infty,-7] \cup (-2,6]$\\
F.$x \in (-\infty,-7] \cup [-2,6)$\\
G.$x \in (-\infty,-7) \cup [-2,6]$\\
H.$x \in (-\infty,-7] \cup [-2,6]$
\testStop
\kluczStart
A
\kluczStop



\zadStart{Zadanie z Wikieł Z 1.62 b) moja wersja nr 33}

Rozwiązać nierówności $(x+7)(6-x)(x+3)\ge0$.
\zadStop
\rozwStart{Patryk Wirkus}{}
Miejsca zerowe naszego wielomianu to: $-7, 6, -3$.\\
Wielomian jest stopnia nieparzystego, ponadto znak współczynnika przy\linebreak najwyższej potędze x jest ujemny.\\ W związku z tym wykres wielomianu zaczyna się od lewej strony powyżej osi OX. A więc $$x \in (-\infty,-7) \cup (-3,6).$$
\rozwStop
\odpStart
$x \in (-\infty,-7) \cup (-3,6)$
\odpStop
\testStart
A.$x \in (-\infty,-7) \cup (-3,6)$\\
B.$x \in (-\infty,-7) \cup (-3,6]$\\
C.$x \in (-\infty,-7) \cup [-3,6)$\\
D.$x \in (-\infty,-7] \cup (-3,6)$\\
E.$x \in (-\infty,-7] \cup (-3,6]$\\
F.$x \in (-\infty,-7] \cup [-3,6)$\\
G.$x \in (-\infty,-7) \cup [-3,6]$\\
H.$x \in (-\infty,-7] \cup [-3,6]$
\testStop
\kluczStart
A
\kluczStop



\zadStart{Zadanie z Wikieł Z 1.62 b) moja wersja nr 34}

Rozwiązać nierówności $(x+7)(6-x)(x+4)\ge0$.
\zadStop
\rozwStart{Patryk Wirkus}{}
Miejsca zerowe naszego wielomianu to: $-7, 6, -4$.\\
Wielomian jest stopnia nieparzystego, ponadto znak współczynnika przy\linebreak najwyższej potędze x jest ujemny.\\ W związku z tym wykres wielomianu zaczyna się od lewej strony powyżej osi OX. A więc $$x \in (-\infty,-7) \cup (-4,6).$$
\rozwStop
\odpStart
$x \in (-\infty,-7) \cup (-4,6)$
\odpStop
\testStart
A.$x \in (-\infty,-7) \cup (-4,6)$\\
B.$x \in (-\infty,-7) \cup (-4,6]$\\
C.$x \in (-\infty,-7) \cup [-4,6)$\\
D.$x \in (-\infty,-7] \cup (-4,6)$\\
E.$x \in (-\infty,-7] \cup (-4,6]$\\
F.$x \in (-\infty,-7] \cup [-4,6)$\\
G.$x \in (-\infty,-7) \cup [-4,6]$\\
H.$x \in (-\infty,-7] \cup [-4,6]$
\testStop
\kluczStart
A
\kluczStop



\zadStart{Zadanie z Wikieł Z 1.62 b) moja wersja nr 35}

Rozwiązać nierówności $(x+7)(6-x)(x+5)\ge0$.
\zadStop
\rozwStart{Patryk Wirkus}{}
Miejsca zerowe naszego wielomianu to: $-7, 6, -5$.\\
Wielomian jest stopnia nieparzystego, ponadto znak współczynnika przy\linebreak najwyższej potędze x jest ujemny.\\ W związku z tym wykres wielomianu zaczyna się od lewej strony powyżej osi OX. A więc $$x \in (-\infty,-7) \cup (-5,6).$$
\rozwStop
\odpStart
$x \in (-\infty,-7) \cup (-5,6)$
\odpStop
\testStart
A.$x \in (-\infty,-7) \cup (-5,6)$\\
B.$x \in (-\infty,-7) \cup (-5,6]$\\
C.$x \in (-\infty,-7) \cup [-5,6)$\\
D.$x \in (-\infty,-7] \cup (-5,6)$\\
E.$x \in (-\infty,-7] \cup (-5,6]$\\
F.$x \in (-\infty,-7] \cup [-5,6)$\\
G.$x \in (-\infty,-7) \cup [-5,6]$\\
H.$x \in (-\infty,-7] \cup [-5,6]$
\testStop
\kluczStart
A
\kluczStop



\zadStart{Zadanie z Wikieł Z 1.62 b) moja wersja nr 36}

Rozwiązać nierówności $(x+8)(2-x)(x+1)\ge0$.
\zadStop
\rozwStart{Patryk Wirkus}{}
Miejsca zerowe naszego wielomianu to: $-8, 2, -1$.\\
Wielomian jest stopnia nieparzystego, ponadto znak współczynnika przy\linebreak najwyższej potędze x jest ujemny.\\ W związku z tym wykres wielomianu zaczyna się od lewej strony powyżej osi OX. A więc $$x \in (-\infty,-8) \cup (-1,2).$$
\rozwStop
\odpStart
$x \in (-\infty,-8) \cup (-1,2)$
\odpStop
\testStart
A.$x \in (-\infty,-8) \cup (-1,2)$\\
B.$x \in (-\infty,-8) \cup (-1,2]$\\
C.$x \in (-\infty,-8) \cup [-1,2)$\\
D.$x \in (-\infty,-8] \cup (-1,2)$\\
E.$x \in (-\infty,-8] \cup (-1,2]$\\
F.$x \in (-\infty,-8] \cup [-1,2)$\\
G.$x \in (-\infty,-8) \cup [-1,2]$\\
H.$x \in (-\infty,-8] \cup [-1,2]$
\testStop
\kluczStart
A
\kluczStop



\zadStart{Zadanie z Wikieł Z 1.62 b) moja wersja nr 37}

Rozwiązać nierówności $(x+8)(3-x)(x+1)\ge0$.
\zadStop
\rozwStart{Patryk Wirkus}{}
Miejsca zerowe naszego wielomianu to: $-8, 3, -1$.\\
Wielomian jest stopnia nieparzystego, ponadto znak współczynnika przy\linebreak najwyższej potędze x jest ujemny.\\ W związku z tym wykres wielomianu zaczyna się od lewej strony powyżej osi OX. A więc $$x \in (-\infty,-8) \cup (-1,3).$$
\rozwStop
\odpStart
$x \in (-\infty,-8) \cup (-1,3)$
\odpStop
\testStart
A.$x \in (-\infty,-8) \cup (-1,3)$\\
B.$x \in (-\infty,-8) \cup (-1,3]$\\
C.$x \in (-\infty,-8) \cup [-1,3)$\\
D.$x \in (-\infty,-8] \cup (-1,3)$\\
E.$x \in (-\infty,-8] \cup (-1,3]$\\
F.$x \in (-\infty,-8] \cup [-1,3)$\\
G.$x \in (-\infty,-8) \cup [-1,3]$\\
H.$x \in (-\infty,-8] \cup [-1,3]$
\testStop
\kluczStart
A
\kluczStop



\zadStart{Zadanie z Wikieł Z 1.62 b) moja wersja nr 38}

Rozwiązać nierówności $(x+8)(3-x)(x+2)\ge0$.
\zadStop
\rozwStart{Patryk Wirkus}{}
Miejsca zerowe naszego wielomianu to: $-8, 3, -2$.\\
Wielomian jest stopnia nieparzystego, ponadto znak współczynnika przy\linebreak najwyższej potędze x jest ujemny.\\ W związku z tym wykres wielomianu zaczyna się od lewej strony powyżej osi OX. A więc $$x \in (-\infty,-8) \cup (-2,3).$$
\rozwStop
\odpStart
$x \in (-\infty,-8) \cup (-2,3)$
\odpStop
\testStart
A.$x \in (-\infty,-8) \cup (-2,3)$\\
B.$x \in (-\infty,-8) \cup (-2,3]$\\
C.$x \in (-\infty,-8) \cup [-2,3)$\\
D.$x \in (-\infty,-8] \cup (-2,3)$\\
E.$x \in (-\infty,-8] \cup (-2,3]$\\
F.$x \in (-\infty,-8] \cup [-2,3)$\\
G.$x \in (-\infty,-8) \cup [-2,3]$\\
H.$x \in (-\infty,-8] \cup [-2,3]$
\testStop
\kluczStart
A
\kluczStop



\zadStart{Zadanie z Wikieł Z 1.62 b) moja wersja nr 39}

Rozwiązać nierówności $(x+8)(4-x)(x+1)\ge0$.
\zadStop
\rozwStart{Patryk Wirkus}{}
Miejsca zerowe naszego wielomianu to: $-8, 4, -1$.\\
Wielomian jest stopnia nieparzystego, ponadto znak współczynnika przy\linebreak najwyższej potędze x jest ujemny.\\ W związku z tym wykres wielomianu zaczyna się od lewej strony powyżej osi OX. A więc $$x \in (-\infty,-8) \cup (-1,4).$$
\rozwStop
\odpStart
$x \in (-\infty,-8) \cup (-1,4)$
\odpStop
\testStart
A.$x \in (-\infty,-8) \cup (-1,4)$\\
B.$x \in (-\infty,-8) \cup (-1,4]$\\
C.$x \in (-\infty,-8) \cup [-1,4)$\\
D.$x \in (-\infty,-8] \cup (-1,4)$\\
E.$x \in (-\infty,-8] \cup (-1,4]$\\
F.$x \in (-\infty,-8] \cup [-1,4)$\\
G.$x \in (-\infty,-8) \cup [-1,4]$\\
H.$x \in (-\infty,-8] \cup [-1,4]$
\testStop
\kluczStart
A
\kluczStop



\zadStart{Zadanie z Wikieł Z 1.62 b) moja wersja nr 40}

Rozwiązać nierówności $(x+8)(4-x)(x+2)\ge0$.
\zadStop
\rozwStart{Patryk Wirkus}{}
Miejsca zerowe naszego wielomianu to: $-8, 4, -2$.\\
Wielomian jest stopnia nieparzystego, ponadto znak współczynnika przy\linebreak najwyższej potędze x jest ujemny.\\ W związku z tym wykres wielomianu zaczyna się od lewej strony powyżej osi OX. A więc $$x \in (-\infty,-8) \cup (-2,4).$$
\rozwStop
\odpStart
$x \in (-\infty,-8) \cup (-2,4)$
\odpStop
\testStart
A.$x \in (-\infty,-8) \cup (-2,4)$\\
B.$x \in (-\infty,-8) \cup (-2,4]$\\
C.$x \in (-\infty,-8) \cup [-2,4)$\\
D.$x \in (-\infty,-8] \cup (-2,4)$\\
E.$x \in (-\infty,-8] \cup (-2,4]$\\
F.$x \in (-\infty,-8] \cup [-2,4)$\\
G.$x \in (-\infty,-8) \cup [-2,4]$\\
H.$x \in (-\infty,-8] \cup [-2,4]$
\testStop
\kluczStart
A
\kluczStop



\zadStart{Zadanie z Wikieł Z 1.62 b) moja wersja nr 41}

Rozwiązać nierówności $(x+8)(4-x)(x+3)\ge0$.
\zadStop
\rozwStart{Patryk Wirkus}{}
Miejsca zerowe naszego wielomianu to: $-8, 4, -3$.\\
Wielomian jest stopnia nieparzystego, ponadto znak współczynnika przy\linebreak najwyższej potędze x jest ujemny.\\ W związku z tym wykres wielomianu zaczyna się od lewej strony powyżej osi OX. A więc $$x \in (-\infty,-8) \cup (-3,4).$$
\rozwStop
\odpStart
$x \in (-\infty,-8) \cup (-3,4)$
\odpStop
\testStart
A.$x \in (-\infty,-8) \cup (-3,4)$\\
B.$x \in (-\infty,-8) \cup (-3,4]$\\
C.$x \in (-\infty,-8) \cup [-3,4)$\\
D.$x \in (-\infty,-8] \cup (-3,4)$\\
E.$x \in (-\infty,-8] \cup (-3,4]$\\
F.$x \in (-\infty,-8] \cup [-3,4)$\\
G.$x \in (-\infty,-8) \cup [-3,4]$\\
H.$x \in (-\infty,-8] \cup [-3,4]$
\testStop
\kluczStart
A
\kluczStop



\zadStart{Zadanie z Wikieł Z 1.62 b) moja wersja nr 42}

Rozwiązać nierówności $(x+8)(5-x)(x+1)\ge0$.
\zadStop
\rozwStart{Patryk Wirkus}{}
Miejsca zerowe naszego wielomianu to: $-8, 5, -1$.\\
Wielomian jest stopnia nieparzystego, ponadto znak współczynnika przy\linebreak najwyższej potędze x jest ujemny.\\ W związku z tym wykres wielomianu zaczyna się od lewej strony powyżej osi OX. A więc $$x \in (-\infty,-8) \cup (-1,5).$$
\rozwStop
\odpStart
$x \in (-\infty,-8) \cup (-1,5)$
\odpStop
\testStart
A.$x \in (-\infty,-8) \cup (-1,5)$\\
B.$x \in (-\infty,-8) \cup (-1,5]$\\
C.$x \in (-\infty,-8) \cup [-1,5)$\\
D.$x \in (-\infty,-8] \cup (-1,5)$\\
E.$x \in (-\infty,-8] \cup (-1,5]$\\
F.$x \in (-\infty,-8] \cup [-1,5)$\\
G.$x \in (-\infty,-8) \cup [-1,5]$\\
H.$x \in (-\infty,-8] \cup [-1,5]$
\testStop
\kluczStart
A
\kluczStop



\zadStart{Zadanie z Wikieł Z 1.62 b) moja wersja nr 43}

Rozwiązać nierówności $(x+8)(5-x)(x+2)\ge0$.
\zadStop
\rozwStart{Patryk Wirkus}{}
Miejsca zerowe naszego wielomianu to: $-8, 5, -2$.\\
Wielomian jest stopnia nieparzystego, ponadto znak współczynnika przy\linebreak najwyższej potędze x jest ujemny.\\ W związku z tym wykres wielomianu zaczyna się od lewej strony powyżej osi OX. A więc $$x \in (-\infty,-8) \cup (-2,5).$$
\rozwStop
\odpStart
$x \in (-\infty,-8) \cup (-2,5)$
\odpStop
\testStart
A.$x \in (-\infty,-8) \cup (-2,5)$\\
B.$x \in (-\infty,-8) \cup (-2,5]$\\
C.$x \in (-\infty,-8) \cup [-2,5)$\\
D.$x \in (-\infty,-8] \cup (-2,5)$\\
E.$x \in (-\infty,-8] \cup (-2,5]$\\
F.$x \in (-\infty,-8] \cup [-2,5)$\\
G.$x \in (-\infty,-8) \cup [-2,5]$\\
H.$x \in (-\infty,-8] \cup [-2,5]$
\testStop
\kluczStart
A
\kluczStop



\zadStart{Zadanie z Wikieł Z 1.62 b) moja wersja nr 44}

Rozwiązać nierówności $(x+8)(5-x)(x+3)\ge0$.
\zadStop
\rozwStart{Patryk Wirkus}{}
Miejsca zerowe naszego wielomianu to: $-8, 5, -3$.\\
Wielomian jest stopnia nieparzystego, ponadto znak współczynnika przy\linebreak najwyższej potędze x jest ujemny.\\ W związku z tym wykres wielomianu zaczyna się od lewej strony powyżej osi OX. A więc $$x \in (-\infty,-8) \cup (-3,5).$$
\rozwStop
\odpStart
$x \in (-\infty,-8) \cup (-3,5)$
\odpStop
\testStart
A.$x \in (-\infty,-8) \cup (-3,5)$\\
B.$x \in (-\infty,-8) \cup (-3,5]$\\
C.$x \in (-\infty,-8) \cup [-3,5)$\\
D.$x \in (-\infty,-8] \cup (-3,5)$\\
E.$x \in (-\infty,-8] \cup (-3,5]$\\
F.$x \in (-\infty,-8] \cup [-3,5)$\\
G.$x \in (-\infty,-8) \cup [-3,5]$\\
H.$x \in (-\infty,-8] \cup [-3,5]$
\testStop
\kluczStart
A
\kluczStop



\zadStart{Zadanie z Wikieł Z 1.62 b) moja wersja nr 45}

Rozwiązać nierówności $(x+8)(5-x)(x+4)\ge0$.
\zadStop
\rozwStart{Patryk Wirkus}{}
Miejsca zerowe naszego wielomianu to: $-8, 5, -4$.\\
Wielomian jest stopnia nieparzystego, ponadto znak współczynnika przy\linebreak najwyższej potędze x jest ujemny.\\ W związku z tym wykres wielomianu zaczyna się od lewej strony powyżej osi OX. A więc $$x \in (-\infty,-8) \cup (-4,5).$$
\rozwStop
\odpStart
$x \in (-\infty,-8) \cup (-4,5)$
\odpStop
\testStart
A.$x \in (-\infty,-8) \cup (-4,5)$\\
B.$x \in (-\infty,-8) \cup (-4,5]$\\
C.$x \in (-\infty,-8) \cup [-4,5)$\\
D.$x \in (-\infty,-8] \cup (-4,5)$\\
E.$x \in (-\infty,-8] \cup (-4,5]$\\
F.$x \in (-\infty,-8] \cup [-4,5)$\\
G.$x \in (-\infty,-8) \cup [-4,5]$\\
H.$x \in (-\infty,-8] \cup [-4,5]$
\testStop
\kluczStart
A
\kluczStop



\zadStart{Zadanie z Wikieł Z 1.62 b) moja wersja nr 46}

Rozwiązać nierówności $(x+8)(6-x)(x+1)\ge0$.
\zadStop
\rozwStart{Patryk Wirkus}{}
Miejsca zerowe naszego wielomianu to: $-8, 6, -1$.\\
Wielomian jest stopnia nieparzystego, ponadto znak współczynnika przy\linebreak najwyższej potędze x jest ujemny.\\ W związku z tym wykres wielomianu zaczyna się od lewej strony powyżej osi OX. A więc $$x \in (-\infty,-8) \cup (-1,6).$$
\rozwStop
\odpStart
$x \in (-\infty,-8) \cup (-1,6)$
\odpStop
\testStart
A.$x \in (-\infty,-8) \cup (-1,6)$\\
B.$x \in (-\infty,-8) \cup (-1,6]$\\
C.$x \in (-\infty,-8) \cup [-1,6)$\\
D.$x \in (-\infty,-8] \cup (-1,6)$\\
E.$x \in (-\infty,-8] \cup (-1,6]$\\
F.$x \in (-\infty,-8] \cup [-1,6)$\\
G.$x \in (-\infty,-8) \cup [-1,6]$\\
H.$x \in (-\infty,-8] \cup [-1,6]$
\testStop
\kluczStart
A
\kluczStop



\zadStart{Zadanie z Wikieł Z 1.62 b) moja wersja nr 47}

Rozwiązać nierówności $(x+8)(6-x)(x+2)\ge0$.
\zadStop
\rozwStart{Patryk Wirkus}{}
Miejsca zerowe naszego wielomianu to: $-8, 6, -2$.\\
Wielomian jest stopnia nieparzystego, ponadto znak współczynnika przy\linebreak najwyższej potędze x jest ujemny.\\ W związku z tym wykres wielomianu zaczyna się od lewej strony powyżej osi OX. A więc $$x \in (-\infty,-8) \cup (-2,6).$$
\rozwStop
\odpStart
$x \in (-\infty,-8) \cup (-2,6)$
\odpStop
\testStart
A.$x \in (-\infty,-8) \cup (-2,6)$\\
B.$x \in (-\infty,-8) \cup (-2,6]$\\
C.$x \in (-\infty,-8) \cup [-2,6)$\\
D.$x \in (-\infty,-8] \cup (-2,6)$\\
E.$x \in (-\infty,-8] \cup (-2,6]$\\
F.$x \in (-\infty,-8] \cup [-2,6)$\\
G.$x \in (-\infty,-8) \cup [-2,6]$\\
H.$x \in (-\infty,-8] \cup [-2,6]$
\testStop
\kluczStart
A
\kluczStop



\zadStart{Zadanie z Wikieł Z 1.62 b) moja wersja nr 48}

Rozwiązać nierówności $(x+8)(6-x)(x+3)\ge0$.
\zadStop
\rozwStart{Patryk Wirkus}{}
Miejsca zerowe naszego wielomianu to: $-8, 6, -3$.\\
Wielomian jest stopnia nieparzystego, ponadto znak współczynnika przy\linebreak najwyższej potędze x jest ujemny.\\ W związku z tym wykres wielomianu zaczyna się od lewej strony powyżej osi OX. A więc $$x \in (-\infty,-8) \cup (-3,6).$$
\rozwStop
\odpStart
$x \in (-\infty,-8) \cup (-3,6)$
\odpStop
\testStart
A.$x \in (-\infty,-8) \cup (-3,6)$\\
B.$x \in (-\infty,-8) \cup (-3,6]$\\
C.$x \in (-\infty,-8) \cup [-3,6)$\\
D.$x \in (-\infty,-8] \cup (-3,6)$\\
E.$x \in (-\infty,-8] \cup (-3,6]$\\
F.$x \in (-\infty,-8] \cup [-3,6)$\\
G.$x \in (-\infty,-8) \cup [-3,6]$\\
H.$x \in (-\infty,-8] \cup [-3,6]$
\testStop
\kluczStart
A
\kluczStop



\zadStart{Zadanie z Wikieł Z 1.62 b) moja wersja nr 49}

Rozwiązać nierówności $(x+8)(6-x)(x+4)\ge0$.
\zadStop
\rozwStart{Patryk Wirkus}{}
Miejsca zerowe naszego wielomianu to: $-8, 6, -4$.\\
Wielomian jest stopnia nieparzystego, ponadto znak współczynnika przy\linebreak najwyższej potędze x jest ujemny.\\ W związku z tym wykres wielomianu zaczyna się od lewej strony powyżej osi OX. A więc $$x \in (-\infty,-8) \cup (-4,6).$$
\rozwStop
\odpStart
$x \in (-\infty,-8) \cup (-4,6)$
\odpStop
\testStart
A.$x \in (-\infty,-8) \cup (-4,6)$\\
B.$x \in (-\infty,-8) \cup (-4,6]$\\
C.$x \in (-\infty,-8) \cup [-4,6)$\\
D.$x \in (-\infty,-8] \cup (-4,6)$\\
E.$x \in (-\infty,-8] \cup (-4,6]$\\
F.$x \in (-\infty,-8] \cup [-4,6)$\\
G.$x \in (-\infty,-8) \cup [-4,6]$\\
H.$x \in (-\infty,-8] \cup [-4,6]$
\testStop
\kluczStart
A
\kluczStop



\zadStart{Zadanie z Wikieł Z 1.62 b) moja wersja nr 50}

Rozwiązać nierówności $(x+8)(6-x)(x+5)\ge0$.
\zadStop
\rozwStart{Patryk Wirkus}{}
Miejsca zerowe naszego wielomianu to: $-8, 6, -5$.\\
Wielomian jest stopnia nieparzystego, ponadto znak współczynnika przy\linebreak najwyższej potędze x jest ujemny.\\ W związku z tym wykres wielomianu zaczyna się od lewej strony powyżej osi OX. A więc $$x \in (-\infty,-8) \cup (-5,6).$$
\rozwStop
\odpStart
$x \in (-\infty,-8) \cup (-5,6)$
\odpStop
\testStart
A.$x \in (-\infty,-8) \cup (-5,6)$\\
B.$x \in (-\infty,-8) \cup (-5,6]$\\
C.$x \in (-\infty,-8) \cup [-5,6)$\\
D.$x \in (-\infty,-8] \cup (-5,6)$\\
E.$x \in (-\infty,-8] \cup (-5,6]$\\
F.$x \in (-\infty,-8] \cup [-5,6)$\\
G.$x \in (-\infty,-8) \cup [-5,6]$\\
H.$x \in (-\infty,-8] \cup [-5,6]$
\testStop
\kluczStart
A
\kluczStop



\zadStart{Zadanie z Wikieł Z 1.62 b) moja wersja nr 51}

Rozwiązać nierówności $(x+8)(7-x)(x+1)\ge0$.
\zadStop
\rozwStart{Patryk Wirkus}{}
Miejsca zerowe naszego wielomianu to: $-8, 7, -1$.\\
Wielomian jest stopnia nieparzystego, ponadto znak współczynnika przy\linebreak najwyższej potędze x jest ujemny.\\ W związku z tym wykres wielomianu zaczyna się od lewej strony powyżej osi OX. A więc $$x \in (-\infty,-8) \cup (-1,7).$$
\rozwStop
\odpStart
$x \in (-\infty,-8) \cup (-1,7)$
\odpStop
\testStart
A.$x \in (-\infty,-8) \cup (-1,7)$\\
B.$x \in (-\infty,-8) \cup (-1,7]$\\
C.$x \in (-\infty,-8) \cup [-1,7)$\\
D.$x \in (-\infty,-8] \cup (-1,7)$\\
E.$x \in (-\infty,-8] \cup (-1,7]$\\
F.$x \in (-\infty,-8] \cup [-1,7)$\\
G.$x \in (-\infty,-8) \cup [-1,7]$\\
H.$x \in (-\infty,-8] \cup [-1,7]$
\testStop
\kluczStart
A
\kluczStop



\zadStart{Zadanie z Wikieł Z 1.62 b) moja wersja nr 52}

Rozwiązać nierówności $(x+8)(7-x)(x+2)\ge0$.
\zadStop
\rozwStart{Patryk Wirkus}{}
Miejsca zerowe naszego wielomianu to: $-8, 7, -2$.\\
Wielomian jest stopnia nieparzystego, ponadto znak współczynnika przy\linebreak najwyższej potędze x jest ujemny.\\ W związku z tym wykres wielomianu zaczyna się od lewej strony powyżej osi OX. A więc $$x \in (-\infty,-8) \cup (-2,7).$$
\rozwStop
\odpStart
$x \in (-\infty,-8) \cup (-2,7)$
\odpStop
\testStart
A.$x \in (-\infty,-8) \cup (-2,7)$\\
B.$x \in (-\infty,-8) \cup (-2,7]$\\
C.$x \in (-\infty,-8) \cup [-2,7)$\\
D.$x \in (-\infty,-8] \cup (-2,7)$\\
E.$x \in (-\infty,-8] \cup (-2,7]$\\
F.$x \in (-\infty,-8] \cup [-2,7)$\\
G.$x \in (-\infty,-8) \cup [-2,7]$\\
H.$x \in (-\infty,-8] \cup [-2,7]$
\testStop
\kluczStart
A
\kluczStop



\zadStart{Zadanie z Wikieł Z 1.62 b) moja wersja nr 53}

Rozwiązać nierówności $(x+8)(7-x)(x+3)\ge0$.
\zadStop
\rozwStart{Patryk Wirkus}{}
Miejsca zerowe naszego wielomianu to: $-8, 7, -3$.\\
Wielomian jest stopnia nieparzystego, ponadto znak współczynnika przy\linebreak najwyższej potędze x jest ujemny.\\ W związku z tym wykres wielomianu zaczyna się od lewej strony powyżej osi OX. A więc $$x \in (-\infty,-8) \cup (-3,7).$$
\rozwStop
\odpStart
$x \in (-\infty,-8) \cup (-3,7)$
\odpStop
\testStart
A.$x \in (-\infty,-8) \cup (-3,7)$\\
B.$x \in (-\infty,-8) \cup (-3,7]$\\
C.$x \in (-\infty,-8) \cup [-3,7)$\\
D.$x \in (-\infty,-8] \cup (-3,7)$\\
E.$x \in (-\infty,-8] \cup (-3,7]$\\
F.$x \in (-\infty,-8] \cup [-3,7)$\\
G.$x \in (-\infty,-8) \cup [-3,7]$\\
H.$x \in (-\infty,-8] \cup [-3,7]$
\testStop
\kluczStart
A
\kluczStop



\zadStart{Zadanie z Wikieł Z 1.62 b) moja wersja nr 54}

Rozwiązać nierówności $(x+8)(7-x)(x+4)\ge0$.
\zadStop
\rozwStart{Patryk Wirkus}{}
Miejsca zerowe naszego wielomianu to: $-8, 7, -4$.\\
Wielomian jest stopnia nieparzystego, ponadto znak współczynnika przy\linebreak najwyższej potędze x jest ujemny.\\ W związku z tym wykres wielomianu zaczyna się od lewej strony powyżej osi OX. A więc $$x \in (-\infty,-8) \cup (-4,7).$$
\rozwStop
\odpStart
$x \in (-\infty,-8) \cup (-4,7)$
\odpStop
\testStart
A.$x \in (-\infty,-8) \cup (-4,7)$\\
B.$x \in (-\infty,-8) \cup (-4,7]$\\
C.$x \in (-\infty,-8) \cup [-4,7)$\\
D.$x \in (-\infty,-8] \cup (-4,7)$\\
E.$x \in (-\infty,-8] \cup (-4,7]$\\
F.$x \in (-\infty,-8] \cup [-4,7)$\\
G.$x \in (-\infty,-8) \cup [-4,7]$\\
H.$x \in (-\infty,-8] \cup [-4,7]$
\testStop
\kluczStart
A
\kluczStop



\zadStart{Zadanie z Wikieł Z 1.62 b) moja wersja nr 55}

Rozwiązać nierówności $(x+8)(7-x)(x+5)\ge0$.
\zadStop
\rozwStart{Patryk Wirkus}{}
Miejsca zerowe naszego wielomianu to: $-8, 7, -5$.\\
Wielomian jest stopnia nieparzystego, ponadto znak współczynnika przy\linebreak najwyższej potędze x jest ujemny.\\ W związku z tym wykres wielomianu zaczyna się od lewej strony powyżej osi OX. A więc $$x \in (-\infty,-8) \cup (-5,7).$$
\rozwStop
\odpStart
$x \in (-\infty,-8) \cup (-5,7)$
\odpStop
\testStart
A.$x \in (-\infty,-8) \cup (-5,7)$\\
B.$x \in (-\infty,-8) \cup (-5,7]$\\
C.$x \in (-\infty,-8) \cup [-5,7)$\\
D.$x \in (-\infty,-8] \cup (-5,7)$\\
E.$x \in (-\infty,-8] \cup (-5,7]$\\
F.$x \in (-\infty,-8] \cup [-5,7)$\\
G.$x \in (-\infty,-8) \cup [-5,7]$\\
H.$x \in (-\infty,-8] \cup [-5,7]$
\testStop
\kluczStart
A
\kluczStop



\zadStart{Zadanie z Wikieł Z 1.62 b) moja wersja nr 56}

Rozwiązać nierówności $(x+8)(7-x)(x+6)\ge0$.
\zadStop
\rozwStart{Patryk Wirkus}{}
Miejsca zerowe naszego wielomianu to: $-8, 7, -6$.\\
Wielomian jest stopnia nieparzystego, ponadto znak współczynnika przy\linebreak najwyższej potędze x jest ujemny.\\ W związku z tym wykres wielomianu zaczyna się od lewej strony powyżej osi OX. A więc $$x \in (-\infty,-8) \cup (-6,7).$$
\rozwStop
\odpStart
$x \in (-\infty,-8) \cup (-6,7)$
\odpStop
\testStart
A.$x \in (-\infty,-8) \cup (-6,7)$\\
B.$x \in (-\infty,-8) \cup (-6,7]$\\
C.$x \in (-\infty,-8) \cup [-6,7)$\\
D.$x \in (-\infty,-8] \cup (-6,7)$\\
E.$x \in (-\infty,-8] \cup (-6,7]$\\
F.$x \in (-\infty,-8] \cup [-6,7)$\\
G.$x \in (-\infty,-8) \cup [-6,7]$\\
H.$x \in (-\infty,-8] \cup [-6,7]$
\testStop
\kluczStart
A
\kluczStop



\zadStart{Zadanie z Wikieł Z 1.62 b) moja wersja nr 57}

Rozwiązać nierówności $(x+9)(2-x)(x+1)\ge0$.
\zadStop
\rozwStart{Patryk Wirkus}{}
Miejsca zerowe naszego wielomianu to: $-9, 2, -1$.\\
Wielomian jest stopnia nieparzystego, ponadto znak współczynnika przy\linebreak najwyższej potędze x jest ujemny.\\ W związku z tym wykres wielomianu zaczyna się od lewej strony powyżej osi OX. A więc $$x \in (-\infty,-9) \cup (-1,2).$$
\rozwStop
\odpStart
$x \in (-\infty,-9) \cup (-1,2)$
\odpStop
\testStart
A.$x \in (-\infty,-9) \cup (-1,2)$\\
B.$x \in (-\infty,-9) \cup (-1,2]$\\
C.$x \in (-\infty,-9) \cup [-1,2)$\\
D.$x \in (-\infty,-9] \cup (-1,2)$\\
E.$x \in (-\infty,-9] \cup (-1,2]$\\
F.$x \in (-\infty,-9] \cup [-1,2)$\\
G.$x \in (-\infty,-9) \cup [-1,2]$\\
H.$x \in (-\infty,-9] \cup [-1,2]$
\testStop
\kluczStart
A
\kluczStop



\zadStart{Zadanie z Wikieł Z 1.62 b) moja wersja nr 58}

Rozwiązać nierówności $(x+9)(3-x)(x+1)\ge0$.
\zadStop
\rozwStart{Patryk Wirkus}{}
Miejsca zerowe naszego wielomianu to: $-9, 3, -1$.\\
Wielomian jest stopnia nieparzystego, ponadto znak współczynnika przy\linebreak najwyższej potędze x jest ujemny.\\ W związku z tym wykres wielomianu zaczyna się od lewej strony powyżej osi OX. A więc $$x \in (-\infty,-9) \cup (-1,3).$$
\rozwStop
\odpStart
$x \in (-\infty,-9) \cup (-1,3)$
\odpStop
\testStart
A.$x \in (-\infty,-9) \cup (-1,3)$\\
B.$x \in (-\infty,-9) \cup (-1,3]$\\
C.$x \in (-\infty,-9) \cup [-1,3)$\\
D.$x \in (-\infty,-9] \cup (-1,3)$\\
E.$x \in (-\infty,-9] \cup (-1,3]$\\
F.$x \in (-\infty,-9] \cup [-1,3)$\\
G.$x \in (-\infty,-9) \cup [-1,3]$\\
H.$x \in (-\infty,-9] \cup [-1,3]$
\testStop
\kluczStart
A
\kluczStop



\zadStart{Zadanie z Wikieł Z 1.62 b) moja wersja nr 59}

Rozwiązać nierówności $(x+9)(3-x)(x+2)\ge0$.
\zadStop
\rozwStart{Patryk Wirkus}{}
Miejsca zerowe naszego wielomianu to: $-9, 3, -2$.\\
Wielomian jest stopnia nieparzystego, ponadto znak współczynnika przy\linebreak najwyższej potędze x jest ujemny.\\ W związku z tym wykres wielomianu zaczyna się od lewej strony powyżej osi OX. A więc $$x \in (-\infty,-9) \cup (-2,3).$$
\rozwStop
\odpStart
$x \in (-\infty,-9) \cup (-2,3)$
\odpStop
\testStart
A.$x \in (-\infty,-9) \cup (-2,3)$\\
B.$x \in (-\infty,-9) \cup (-2,3]$\\
C.$x \in (-\infty,-9) \cup [-2,3)$\\
D.$x \in (-\infty,-9] \cup (-2,3)$\\
E.$x \in (-\infty,-9] \cup (-2,3]$\\
F.$x \in (-\infty,-9] \cup [-2,3)$\\
G.$x \in (-\infty,-9) \cup [-2,3]$\\
H.$x \in (-\infty,-9] \cup [-2,3]$
\testStop
\kluczStart
A
\kluczStop



\zadStart{Zadanie z Wikieł Z 1.62 b) moja wersja nr 60}

Rozwiązać nierówności $(x+9)(4-x)(x+1)\ge0$.
\zadStop
\rozwStart{Patryk Wirkus}{}
Miejsca zerowe naszego wielomianu to: $-9, 4, -1$.\\
Wielomian jest stopnia nieparzystego, ponadto znak współczynnika przy\linebreak najwyższej potędze x jest ujemny.\\ W związku z tym wykres wielomianu zaczyna się od lewej strony powyżej osi OX. A więc $$x \in (-\infty,-9) \cup (-1,4).$$
\rozwStop
\odpStart
$x \in (-\infty,-9) \cup (-1,4)$
\odpStop
\testStart
A.$x \in (-\infty,-9) \cup (-1,4)$\\
B.$x \in (-\infty,-9) \cup (-1,4]$\\
C.$x \in (-\infty,-9) \cup [-1,4)$\\
D.$x \in (-\infty,-9] \cup (-1,4)$\\
E.$x \in (-\infty,-9] \cup (-1,4]$\\
F.$x \in (-\infty,-9] \cup [-1,4)$\\
G.$x \in (-\infty,-9) \cup [-1,4]$\\
H.$x \in (-\infty,-9] \cup [-1,4]$
\testStop
\kluczStart
A
\kluczStop



\zadStart{Zadanie z Wikieł Z 1.62 b) moja wersja nr 61}

Rozwiązać nierówności $(x+9)(4-x)(x+2)\ge0$.
\zadStop
\rozwStart{Patryk Wirkus}{}
Miejsca zerowe naszego wielomianu to: $-9, 4, -2$.\\
Wielomian jest stopnia nieparzystego, ponadto znak współczynnika przy\linebreak najwyższej potędze x jest ujemny.\\ W związku z tym wykres wielomianu zaczyna się od lewej strony powyżej osi OX. A więc $$x \in (-\infty,-9) \cup (-2,4).$$
\rozwStop
\odpStart
$x \in (-\infty,-9) \cup (-2,4)$
\odpStop
\testStart
A.$x \in (-\infty,-9) \cup (-2,4)$\\
B.$x \in (-\infty,-9) \cup (-2,4]$\\
C.$x \in (-\infty,-9) \cup [-2,4)$\\
D.$x \in (-\infty,-9] \cup (-2,4)$\\
E.$x \in (-\infty,-9] \cup (-2,4]$\\
F.$x \in (-\infty,-9] \cup [-2,4)$\\
G.$x \in (-\infty,-9) \cup [-2,4]$\\
H.$x \in (-\infty,-9] \cup [-2,4]$
\testStop
\kluczStart
A
\kluczStop



\zadStart{Zadanie z Wikieł Z 1.62 b) moja wersja nr 62}

Rozwiązać nierówności $(x+9)(4-x)(x+3)\ge0$.
\zadStop
\rozwStart{Patryk Wirkus}{}
Miejsca zerowe naszego wielomianu to: $-9, 4, -3$.\\
Wielomian jest stopnia nieparzystego, ponadto znak współczynnika przy\linebreak najwyższej potędze x jest ujemny.\\ W związku z tym wykres wielomianu zaczyna się od lewej strony powyżej osi OX. A więc $$x \in (-\infty,-9) \cup (-3,4).$$
\rozwStop
\odpStart
$x \in (-\infty,-9) \cup (-3,4)$
\odpStop
\testStart
A.$x \in (-\infty,-9) \cup (-3,4)$\\
B.$x \in (-\infty,-9) \cup (-3,4]$\\
C.$x \in (-\infty,-9) \cup [-3,4)$\\
D.$x \in (-\infty,-9] \cup (-3,4)$\\
E.$x \in (-\infty,-9] \cup (-3,4]$\\
F.$x \in (-\infty,-9] \cup [-3,4)$\\
G.$x \in (-\infty,-9) \cup [-3,4]$\\
H.$x \in (-\infty,-9] \cup [-3,4]$
\testStop
\kluczStart
A
\kluczStop



\zadStart{Zadanie z Wikieł Z 1.62 b) moja wersja nr 63}

Rozwiązać nierówności $(x+9)(5-x)(x+1)\ge0$.
\zadStop
\rozwStart{Patryk Wirkus}{}
Miejsca zerowe naszego wielomianu to: $-9, 5, -1$.\\
Wielomian jest stopnia nieparzystego, ponadto znak współczynnika przy\linebreak najwyższej potędze x jest ujemny.\\ W związku z tym wykres wielomianu zaczyna się od lewej strony powyżej osi OX. A więc $$x \in (-\infty,-9) \cup (-1,5).$$
\rozwStop
\odpStart
$x \in (-\infty,-9) \cup (-1,5)$
\odpStop
\testStart
A.$x \in (-\infty,-9) \cup (-1,5)$\\
B.$x \in (-\infty,-9) \cup (-1,5]$\\
C.$x \in (-\infty,-9) \cup [-1,5)$\\
D.$x \in (-\infty,-9] \cup (-1,5)$\\
E.$x \in (-\infty,-9] \cup (-1,5]$\\
F.$x \in (-\infty,-9] \cup [-1,5)$\\
G.$x \in (-\infty,-9) \cup [-1,5]$\\
H.$x \in (-\infty,-9] \cup [-1,5]$
\testStop
\kluczStart
A
\kluczStop



\zadStart{Zadanie z Wikieł Z 1.62 b) moja wersja nr 64}

Rozwiązać nierówności $(x+9)(5-x)(x+2)\ge0$.
\zadStop
\rozwStart{Patryk Wirkus}{}
Miejsca zerowe naszego wielomianu to: $-9, 5, -2$.\\
Wielomian jest stopnia nieparzystego, ponadto znak współczynnika przy\linebreak najwyższej potędze x jest ujemny.\\ W związku z tym wykres wielomianu zaczyna się od lewej strony powyżej osi OX. A więc $$x \in (-\infty,-9) \cup (-2,5).$$
\rozwStop
\odpStart
$x \in (-\infty,-9) \cup (-2,5)$
\odpStop
\testStart
A.$x \in (-\infty,-9) \cup (-2,5)$\\
B.$x \in (-\infty,-9) \cup (-2,5]$\\
C.$x \in (-\infty,-9) \cup [-2,5)$\\
D.$x \in (-\infty,-9] \cup (-2,5)$\\
E.$x \in (-\infty,-9] \cup (-2,5]$\\
F.$x \in (-\infty,-9] \cup [-2,5)$\\
G.$x \in (-\infty,-9) \cup [-2,5]$\\
H.$x \in (-\infty,-9] \cup [-2,5]$
\testStop
\kluczStart
A
\kluczStop



\zadStart{Zadanie z Wikieł Z 1.62 b) moja wersja nr 65}

Rozwiązać nierówności $(x+9)(5-x)(x+3)\ge0$.
\zadStop
\rozwStart{Patryk Wirkus}{}
Miejsca zerowe naszego wielomianu to: $-9, 5, -3$.\\
Wielomian jest stopnia nieparzystego, ponadto znak współczynnika przy\linebreak najwyższej potędze x jest ujemny.\\ W związku z tym wykres wielomianu zaczyna się od lewej strony powyżej osi OX. A więc $$x \in (-\infty,-9) \cup (-3,5).$$
\rozwStop
\odpStart
$x \in (-\infty,-9) \cup (-3,5)$
\odpStop
\testStart
A.$x \in (-\infty,-9) \cup (-3,5)$\\
B.$x \in (-\infty,-9) \cup (-3,5]$\\
C.$x \in (-\infty,-9) \cup [-3,5)$\\
D.$x \in (-\infty,-9] \cup (-3,5)$\\
E.$x \in (-\infty,-9] \cup (-3,5]$\\
F.$x \in (-\infty,-9] \cup [-3,5)$\\
G.$x \in (-\infty,-9) \cup [-3,5]$\\
H.$x \in (-\infty,-9] \cup [-3,5]$
\testStop
\kluczStart
A
\kluczStop



\zadStart{Zadanie z Wikieł Z 1.62 b) moja wersja nr 66}

Rozwiązać nierówności $(x+9)(5-x)(x+4)\ge0$.
\zadStop
\rozwStart{Patryk Wirkus}{}
Miejsca zerowe naszego wielomianu to: $-9, 5, -4$.\\
Wielomian jest stopnia nieparzystego, ponadto znak współczynnika przy\linebreak najwyższej potędze x jest ujemny.\\ W związku z tym wykres wielomianu zaczyna się od lewej strony powyżej osi OX. A więc $$x \in (-\infty,-9) \cup (-4,5).$$
\rozwStop
\odpStart
$x \in (-\infty,-9) \cup (-4,5)$
\odpStop
\testStart
A.$x \in (-\infty,-9) \cup (-4,5)$\\
B.$x \in (-\infty,-9) \cup (-4,5]$\\
C.$x \in (-\infty,-9) \cup [-4,5)$\\
D.$x \in (-\infty,-9] \cup (-4,5)$\\
E.$x \in (-\infty,-9] \cup (-4,5]$\\
F.$x \in (-\infty,-9] \cup [-4,5)$\\
G.$x \in (-\infty,-9) \cup [-4,5]$\\
H.$x \in (-\infty,-9] \cup [-4,5]$
\testStop
\kluczStart
A
\kluczStop



\zadStart{Zadanie z Wikieł Z 1.62 b) moja wersja nr 67}

Rozwiązać nierówności $(x+9)(6-x)(x+1)\ge0$.
\zadStop
\rozwStart{Patryk Wirkus}{}
Miejsca zerowe naszego wielomianu to: $-9, 6, -1$.\\
Wielomian jest stopnia nieparzystego, ponadto znak współczynnika przy\linebreak najwyższej potędze x jest ujemny.\\ W związku z tym wykres wielomianu zaczyna się od lewej strony powyżej osi OX. A więc $$x \in (-\infty,-9) \cup (-1,6).$$
\rozwStop
\odpStart
$x \in (-\infty,-9) \cup (-1,6)$
\odpStop
\testStart
A.$x \in (-\infty,-9) \cup (-1,6)$\\
B.$x \in (-\infty,-9) \cup (-1,6]$\\
C.$x \in (-\infty,-9) \cup [-1,6)$\\
D.$x \in (-\infty,-9] \cup (-1,6)$\\
E.$x \in (-\infty,-9] \cup (-1,6]$\\
F.$x \in (-\infty,-9] \cup [-1,6)$\\
G.$x \in (-\infty,-9) \cup [-1,6]$\\
H.$x \in (-\infty,-9] \cup [-1,6]$
\testStop
\kluczStart
A
\kluczStop



\zadStart{Zadanie z Wikieł Z 1.62 b) moja wersja nr 68}

Rozwiązać nierówności $(x+9)(6-x)(x+2)\ge0$.
\zadStop
\rozwStart{Patryk Wirkus}{}
Miejsca zerowe naszego wielomianu to: $-9, 6, -2$.\\
Wielomian jest stopnia nieparzystego, ponadto znak współczynnika przy\linebreak najwyższej potędze x jest ujemny.\\ W związku z tym wykres wielomianu zaczyna się od lewej strony powyżej osi OX. A więc $$x \in (-\infty,-9) \cup (-2,6).$$
\rozwStop
\odpStart
$x \in (-\infty,-9) \cup (-2,6)$
\odpStop
\testStart
A.$x \in (-\infty,-9) \cup (-2,6)$\\
B.$x \in (-\infty,-9) \cup (-2,6]$\\
C.$x \in (-\infty,-9) \cup [-2,6)$\\
D.$x \in (-\infty,-9] \cup (-2,6)$\\
E.$x \in (-\infty,-9] \cup (-2,6]$\\
F.$x \in (-\infty,-9] \cup [-2,6)$\\
G.$x \in (-\infty,-9) \cup [-2,6]$\\
H.$x \in (-\infty,-9] \cup [-2,6]$
\testStop
\kluczStart
A
\kluczStop



\zadStart{Zadanie z Wikieł Z 1.62 b) moja wersja nr 69}

Rozwiązać nierówności $(x+9)(6-x)(x+3)\ge0$.
\zadStop
\rozwStart{Patryk Wirkus}{}
Miejsca zerowe naszego wielomianu to: $-9, 6, -3$.\\
Wielomian jest stopnia nieparzystego, ponadto znak współczynnika przy\linebreak najwyższej potędze x jest ujemny.\\ W związku z tym wykres wielomianu zaczyna się od lewej strony powyżej osi OX. A więc $$x \in (-\infty,-9) \cup (-3,6).$$
\rozwStop
\odpStart
$x \in (-\infty,-9) \cup (-3,6)$
\odpStop
\testStart
A.$x \in (-\infty,-9) \cup (-3,6)$\\
B.$x \in (-\infty,-9) \cup (-3,6]$\\
C.$x \in (-\infty,-9) \cup [-3,6)$\\
D.$x \in (-\infty,-9] \cup (-3,6)$\\
E.$x \in (-\infty,-9] \cup (-3,6]$\\
F.$x \in (-\infty,-9] \cup [-3,6)$\\
G.$x \in (-\infty,-9) \cup [-3,6]$\\
H.$x \in (-\infty,-9] \cup [-3,6]$
\testStop
\kluczStart
A
\kluczStop



\zadStart{Zadanie z Wikieł Z 1.62 b) moja wersja nr 70}

Rozwiązać nierówności $(x+9)(6-x)(x+4)\ge0$.
\zadStop
\rozwStart{Patryk Wirkus}{}
Miejsca zerowe naszego wielomianu to: $-9, 6, -4$.\\
Wielomian jest stopnia nieparzystego, ponadto znak współczynnika przy\linebreak najwyższej potędze x jest ujemny.\\ W związku z tym wykres wielomianu zaczyna się od lewej strony powyżej osi OX. A więc $$x \in (-\infty,-9) \cup (-4,6).$$
\rozwStop
\odpStart
$x \in (-\infty,-9) \cup (-4,6)$
\odpStop
\testStart
A.$x \in (-\infty,-9) \cup (-4,6)$\\
B.$x \in (-\infty,-9) \cup (-4,6]$\\
C.$x \in (-\infty,-9) \cup [-4,6)$\\
D.$x \in (-\infty,-9] \cup (-4,6)$\\
E.$x \in (-\infty,-9] \cup (-4,6]$\\
F.$x \in (-\infty,-9] \cup [-4,6)$\\
G.$x \in (-\infty,-9) \cup [-4,6]$\\
H.$x \in (-\infty,-9] \cup [-4,6]$
\testStop
\kluczStart
A
\kluczStop



\zadStart{Zadanie z Wikieł Z 1.62 b) moja wersja nr 71}

Rozwiązać nierówności $(x+9)(6-x)(x+5)\ge0$.
\zadStop
\rozwStart{Patryk Wirkus}{}
Miejsca zerowe naszego wielomianu to: $-9, 6, -5$.\\
Wielomian jest stopnia nieparzystego, ponadto znak współczynnika przy\linebreak najwyższej potędze x jest ujemny.\\ W związku z tym wykres wielomianu zaczyna się od lewej strony powyżej osi OX. A więc $$x \in (-\infty,-9) \cup (-5,6).$$
\rozwStop
\odpStart
$x \in (-\infty,-9) \cup (-5,6)$
\odpStop
\testStart
A.$x \in (-\infty,-9) \cup (-5,6)$\\
B.$x \in (-\infty,-9) \cup (-5,6]$\\
C.$x \in (-\infty,-9) \cup [-5,6)$\\
D.$x \in (-\infty,-9] \cup (-5,6)$\\
E.$x \in (-\infty,-9] \cup (-5,6]$\\
F.$x \in (-\infty,-9] \cup [-5,6)$\\
G.$x \in (-\infty,-9) \cup [-5,6]$\\
H.$x \in (-\infty,-9] \cup [-5,6]$
\testStop
\kluczStart
A
\kluczStop



\zadStart{Zadanie z Wikieł Z 1.62 b) moja wersja nr 72}

Rozwiązać nierówności $(x+9)(7-x)(x+1)\ge0$.
\zadStop
\rozwStart{Patryk Wirkus}{}
Miejsca zerowe naszego wielomianu to: $-9, 7, -1$.\\
Wielomian jest stopnia nieparzystego, ponadto znak współczynnika przy\linebreak najwyższej potędze x jest ujemny.\\ W związku z tym wykres wielomianu zaczyna się od lewej strony powyżej osi OX. A więc $$x \in (-\infty,-9) \cup (-1,7).$$
\rozwStop
\odpStart
$x \in (-\infty,-9) \cup (-1,7)$
\odpStop
\testStart
A.$x \in (-\infty,-9) \cup (-1,7)$\\
B.$x \in (-\infty,-9) \cup (-1,7]$\\
C.$x \in (-\infty,-9) \cup [-1,7)$\\
D.$x \in (-\infty,-9] \cup (-1,7)$\\
E.$x \in (-\infty,-9] \cup (-1,7]$\\
F.$x \in (-\infty,-9] \cup [-1,7)$\\
G.$x \in (-\infty,-9) \cup [-1,7]$\\
H.$x \in (-\infty,-9] \cup [-1,7]$
\testStop
\kluczStart
A
\kluczStop



\zadStart{Zadanie z Wikieł Z 1.62 b) moja wersja nr 73}

Rozwiązać nierówności $(x+9)(7-x)(x+2)\ge0$.
\zadStop
\rozwStart{Patryk Wirkus}{}
Miejsca zerowe naszego wielomianu to: $-9, 7, -2$.\\
Wielomian jest stopnia nieparzystego, ponadto znak współczynnika przy\linebreak najwyższej potędze x jest ujemny.\\ W związku z tym wykres wielomianu zaczyna się od lewej strony powyżej osi OX. A więc $$x \in (-\infty,-9) \cup (-2,7).$$
\rozwStop
\odpStart
$x \in (-\infty,-9) \cup (-2,7)$
\odpStop
\testStart
A.$x \in (-\infty,-9) \cup (-2,7)$\\
B.$x \in (-\infty,-9) \cup (-2,7]$\\
C.$x \in (-\infty,-9) \cup [-2,7)$\\
D.$x \in (-\infty,-9] \cup (-2,7)$\\
E.$x \in (-\infty,-9] \cup (-2,7]$\\
F.$x \in (-\infty,-9] \cup [-2,7)$\\
G.$x \in (-\infty,-9) \cup [-2,7]$\\
H.$x \in (-\infty,-9] \cup [-2,7]$
\testStop
\kluczStart
A
\kluczStop



\zadStart{Zadanie z Wikieł Z 1.62 b) moja wersja nr 74}

Rozwiązać nierówności $(x+9)(7-x)(x+3)\ge0$.
\zadStop
\rozwStart{Patryk Wirkus}{}
Miejsca zerowe naszego wielomianu to: $-9, 7, -3$.\\
Wielomian jest stopnia nieparzystego, ponadto znak współczynnika przy\linebreak najwyższej potędze x jest ujemny.\\ W związku z tym wykres wielomianu zaczyna się od lewej strony powyżej osi OX. A więc $$x \in (-\infty,-9) \cup (-3,7).$$
\rozwStop
\odpStart
$x \in (-\infty,-9) \cup (-3,7)$
\odpStop
\testStart
A.$x \in (-\infty,-9) \cup (-3,7)$\\
B.$x \in (-\infty,-9) \cup (-3,7]$\\
C.$x \in (-\infty,-9) \cup [-3,7)$\\
D.$x \in (-\infty,-9] \cup (-3,7)$\\
E.$x \in (-\infty,-9] \cup (-3,7]$\\
F.$x \in (-\infty,-9] \cup [-3,7)$\\
G.$x \in (-\infty,-9) \cup [-3,7]$\\
H.$x \in (-\infty,-9] \cup [-3,7]$
\testStop
\kluczStart
A
\kluczStop



\zadStart{Zadanie z Wikieł Z 1.62 b) moja wersja nr 75}

Rozwiązać nierówności $(x+9)(7-x)(x+4)\ge0$.
\zadStop
\rozwStart{Patryk Wirkus}{}
Miejsca zerowe naszego wielomianu to: $-9, 7, -4$.\\
Wielomian jest stopnia nieparzystego, ponadto znak współczynnika przy\linebreak najwyższej potędze x jest ujemny.\\ W związku z tym wykres wielomianu zaczyna się od lewej strony powyżej osi OX. A więc $$x \in (-\infty,-9) \cup (-4,7).$$
\rozwStop
\odpStart
$x \in (-\infty,-9) \cup (-4,7)$
\odpStop
\testStart
A.$x \in (-\infty,-9) \cup (-4,7)$\\
B.$x \in (-\infty,-9) \cup (-4,7]$\\
C.$x \in (-\infty,-9) \cup [-4,7)$\\
D.$x \in (-\infty,-9] \cup (-4,7)$\\
E.$x \in (-\infty,-9] \cup (-4,7]$\\
F.$x \in (-\infty,-9] \cup [-4,7)$\\
G.$x \in (-\infty,-9) \cup [-4,7]$\\
H.$x \in (-\infty,-9] \cup [-4,7]$
\testStop
\kluczStart
A
\kluczStop



\zadStart{Zadanie z Wikieł Z 1.62 b) moja wersja nr 76}

Rozwiązać nierówności $(x+9)(7-x)(x+5)\ge0$.
\zadStop
\rozwStart{Patryk Wirkus}{}
Miejsca zerowe naszego wielomianu to: $-9, 7, -5$.\\
Wielomian jest stopnia nieparzystego, ponadto znak współczynnika przy\linebreak najwyższej potędze x jest ujemny.\\ W związku z tym wykres wielomianu zaczyna się od lewej strony powyżej osi OX. A więc $$x \in (-\infty,-9) \cup (-5,7).$$
\rozwStop
\odpStart
$x \in (-\infty,-9) \cup (-5,7)$
\odpStop
\testStart
A.$x \in (-\infty,-9) \cup (-5,7)$\\
B.$x \in (-\infty,-9) \cup (-5,7]$\\
C.$x \in (-\infty,-9) \cup [-5,7)$\\
D.$x \in (-\infty,-9] \cup (-5,7)$\\
E.$x \in (-\infty,-9] \cup (-5,7]$\\
F.$x \in (-\infty,-9] \cup [-5,7)$\\
G.$x \in (-\infty,-9) \cup [-5,7]$\\
H.$x \in (-\infty,-9] \cup [-5,7]$
\testStop
\kluczStart
A
\kluczStop



\zadStart{Zadanie z Wikieł Z 1.62 b) moja wersja nr 77}

Rozwiązać nierówności $(x+9)(7-x)(x+6)\ge0$.
\zadStop
\rozwStart{Patryk Wirkus}{}
Miejsca zerowe naszego wielomianu to: $-9, 7, -6$.\\
Wielomian jest stopnia nieparzystego, ponadto znak współczynnika przy\linebreak najwyższej potędze x jest ujemny.\\ W związku z tym wykres wielomianu zaczyna się od lewej strony powyżej osi OX. A więc $$x \in (-\infty,-9) \cup (-6,7).$$
\rozwStop
\odpStart
$x \in (-\infty,-9) \cup (-6,7)$
\odpStop
\testStart
A.$x \in (-\infty,-9) \cup (-6,7)$\\
B.$x \in (-\infty,-9) \cup (-6,7]$\\
C.$x \in (-\infty,-9) \cup [-6,7)$\\
D.$x \in (-\infty,-9] \cup (-6,7)$\\
E.$x \in (-\infty,-9] \cup (-6,7]$\\
F.$x \in (-\infty,-9] \cup [-6,7)$\\
G.$x \in (-\infty,-9) \cup [-6,7]$\\
H.$x \in (-\infty,-9] \cup [-6,7]$
\testStop
\kluczStart
A
\kluczStop



\zadStart{Zadanie z Wikieł Z 1.62 b) moja wersja nr 78}

Rozwiązać nierówności $(x+9)(8-x)(x+1)\ge0$.
\zadStop
\rozwStart{Patryk Wirkus}{}
Miejsca zerowe naszego wielomianu to: $-9, 8, -1$.\\
Wielomian jest stopnia nieparzystego, ponadto znak współczynnika przy\linebreak najwyższej potędze x jest ujemny.\\ W związku z tym wykres wielomianu zaczyna się od lewej strony powyżej osi OX. A więc $$x \in (-\infty,-9) \cup (-1,8).$$
\rozwStop
\odpStart
$x \in (-\infty,-9) \cup (-1,8)$
\odpStop
\testStart
A.$x \in (-\infty,-9) \cup (-1,8)$\\
B.$x \in (-\infty,-9) \cup (-1,8]$\\
C.$x \in (-\infty,-9) \cup [-1,8)$\\
D.$x \in (-\infty,-9] \cup (-1,8)$\\
E.$x \in (-\infty,-9] \cup (-1,8]$\\
F.$x \in (-\infty,-9] \cup [-1,8)$\\
G.$x \in (-\infty,-9) \cup [-1,8]$\\
H.$x \in (-\infty,-9] \cup [-1,8]$
\testStop
\kluczStart
A
\kluczStop



\zadStart{Zadanie z Wikieł Z 1.62 b) moja wersja nr 79}

Rozwiązać nierówności $(x+9)(8-x)(x+2)\ge0$.
\zadStop
\rozwStart{Patryk Wirkus}{}
Miejsca zerowe naszego wielomianu to: $-9, 8, -2$.\\
Wielomian jest stopnia nieparzystego, ponadto znak współczynnika przy\linebreak najwyższej potędze x jest ujemny.\\ W związku z tym wykres wielomianu zaczyna się od lewej strony powyżej osi OX. A więc $$x \in (-\infty,-9) \cup (-2,8).$$
\rozwStop
\odpStart
$x \in (-\infty,-9) \cup (-2,8)$
\odpStop
\testStart
A.$x \in (-\infty,-9) \cup (-2,8)$\\
B.$x \in (-\infty,-9) \cup (-2,8]$\\
C.$x \in (-\infty,-9) \cup [-2,8)$\\
D.$x \in (-\infty,-9] \cup (-2,8)$\\
E.$x \in (-\infty,-9] \cup (-2,8]$\\
F.$x \in (-\infty,-9] \cup [-2,8)$\\
G.$x \in (-\infty,-9) \cup [-2,8]$\\
H.$x \in (-\infty,-9] \cup [-2,8]$
\testStop
\kluczStart
A
\kluczStop



\zadStart{Zadanie z Wikieł Z 1.62 b) moja wersja nr 80}

Rozwiązać nierówności $(x+9)(8-x)(x+3)\ge0$.
\zadStop
\rozwStart{Patryk Wirkus}{}
Miejsca zerowe naszego wielomianu to: $-9, 8, -3$.\\
Wielomian jest stopnia nieparzystego, ponadto znak współczynnika przy\linebreak najwyższej potędze x jest ujemny.\\ W związku z tym wykres wielomianu zaczyna się od lewej strony powyżej osi OX. A więc $$x \in (-\infty,-9) \cup (-3,8).$$
\rozwStop
\odpStart
$x \in (-\infty,-9) \cup (-3,8)$
\odpStop
\testStart
A.$x \in (-\infty,-9) \cup (-3,8)$\\
B.$x \in (-\infty,-9) \cup (-3,8]$\\
C.$x \in (-\infty,-9) \cup [-3,8)$\\
D.$x \in (-\infty,-9] \cup (-3,8)$\\
E.$x \in (-\infty,-9] \cup (-3,8]$\\
F.$x \in (-\infty,-9] \cup [-3,8)$\\
G.$x \in (-\infty,-9) \cup [-3,8]$\\
H.$x \in (-\infty,-9] \cup [-3,8]$
\testStop
\kluczStart
A
\kluczStop



\zadStart{Zadanie z Wikieł Z 1.62 b) moja wersja nr 81}

Rozwiązać nierówności $(x+9)(8-x)(x+4)\ge0$.
\zadStop
\rozwStart{Patryk Wirkus}{}
Miejsca zerowe naszego wielomianu to: $-9, 8, -4$.\\
Wielomian jest stopnia nieparzystego, ponadto znak współczynnika przy\linebreak najwyższej potędze x jest ujemny.\\ W związku z tym wykres wielomianu zaczyna się od lewej strony powyżej osi OX. A więc $$x \in (-\infty,-9) \cup (-4,8).$$
\rozwStop
\odpStart
$x \in (-\infty,-9) \cup (-4,8)$
\odpStop
\testStart
A.$x \in (-\infty,-9) \cup (-4,8)$\\
B.$x \in (-\infty,-9) \cup (-4,8]$\\
C.$x \in (-\infty,-9) \cup [-4,8)$\\
D.$x \in (-\infty,-9] \cup (-4,8)$\\
E.$x \in (-\infty,-9] \cup (-4,8]$\\
F.$x \in (-\infty,-9] \cup [-4,8)$\\
G.$x \in (-\infty,-9) \cup [-4,8]$\\
H.$x \in (-\infty,-9] \cup [-4,8]$
\testStop
\kluczStart
A
\kluczStop



\zadStart{Zadanie z Wikieł Z 1.62 b) moja wersja nr 82}

Rozwiązać nierówności $(x+9)(8-x)(x+5)\ge0$.
\zadStop
\rozwStart{Patryk Wirkus}{}
Miejsca zerowe naszego wielomianu to: $-9, 8, -5$.\\
Wielomian jest stopnia nieparzystego, ponadto znak współczynnika przy\linebreak najwyższej potędze x jest ujemny.\\ W związku z tym wykres wielomianu zaczyna się od lewej strony powyżej osi OX. A więc $$x \in (-\infty,-9) \cup (-5,8).$$
\rozwStop
\odpStart
$x \in (-\infty,-9) \cup (-5,8)$
\odpStop
\testStart
A.$x \in (-\infty,-9) \cup (-5,8)$\\
B.$x \in (-\infty,-9) \cup (-5,8]$\\
C.$x \in (-\infty,-9) \cup [-5,8)$\\
D.$x \in (-\infty,-9] \cup (-5,8)$\\
E.$x \in (-\infty,-9] \cup (-5,8]$\\
F.$x \in (-\infty,-9] \cup [-5,8)$\\
G.$x \in (-\infty,-9) \cup [-5,8]$\\
H.$x \in (-\infty,-9] \cup [-5,8]$
\testStop
\kluczStart
A
\kluczStop



\zadStart{Zadanie z Wikieł Z 1.62 b) moja wersja nr 83}

Rozwiązać nierówności $(x+9)(8-x)(x+6)\ge0$.
\zadStop
\rozwStart{Patryk Wirkus}{}
Miejsca zerowe naszego wielomianu to: $-9, 8, -6$.\\
Wielomian jest stopnia nieparzystego, ponadto znak współczynnika przy\linebreak najwyższej potędze x jest ujemny.\\ W związku z tym wykres wielomianu zaczyna się od lewej strony powyżej osi OX. A więc $$x \in (-\infty,-9) \cup (-6,8).$$
\rozwStop
\odpStart
$x \in (-\infty,-9) \cup (-6,8)$
\odpStop
\testStart
A.$x \in (-\infty,-9) \cup (-6,8)$\\
B.$x \in (-\infty,-9) \cup (-6,8]$\\
C.$x \in (-\infty,-9) \cup [-6,8)$\\
D.$x \in (-\infty,-9] \cup (-6,8)$\\
E.$x \in (-\infty,-9] \cup (-6,8]$\\
F.$x \in (-\infty,-9] \cup [-6,8)$\\
G.$x \in (-\infty,-9) \cup [-6,8]$\\
H.$x \in (-\infty,-9] \cup [-6,8]$
\testStop
\kluczStart
A
\kluczStop



\zadStart{Zadanie z Wikieł Z 1.62 b) moja wersja nr 84}

Rozwiązać nierówności $(x+9)(8-x)(x+7)\ge0$.
\zadStop
\rozwStart{Patryk Wirkus}{}
Miejsca zerowe naszego wielomianu to: $-9, 8, -7$.\\
Wielomian jest stopnia nieparzystego, ponadto znak współczynnika przy\linebreak najwyższej potędze x jest ujemny.\\ W związku z tym wykres wielomianu zaczyna się od lewej strony powyżej osi OX. A więc $$x \in (-\infty,-9) \cup (-7,8).$$
\rozwStop
\odpStart
$x \in (-\infty,-9) \cup (-7,8)$
\odpStop
\testStart
A.$x \in (-\infty,-9) \cup (-7,8)$\\
B.$x \in (-\infty,-9) \cup (-7,8]$\\
C.$x \in (-\infty,-9) \cup [-7,8)$\\
D.$x \in (-\infty,-9] \cup (-7,8)$\\
E.$x \in (-\infty,-9] \cup (-7,8]$\\
F.$x \in (-\infty,-9] \cup [-7,8)$\\
G.$x \in (-\infty,-9) \cup [-7,8]$\\
H.$x \in (-\infty,-9] \cup [-7,8]$
\testStop
\kluczStart
A
\kluczStop



\zadStart{Zadanie z Wikieł Z 1.62 b) moja wersja nr 85}

Rozwiązać nierówności $(x+10)(2-x)(x+1)\ge0$.
\zadStop
\rozwStart{Patryk Wirkus}{}
Miejsca zerowe naszego wielomianu to: $-10, 2, -1$.\\
Wielomian jest stopnia nieparzystego, ponadto znak współczynnika przy\linebreak najwyższej potędze x jest ujemny.\\ W związku z tym wykres wielomianu zaczyna się od lewej strony powyżej osi OX. A więc $$x \in (-\infty,-10) \cup (-1,2).$$
\rozwStop
\odpStart
$x \in (-\infty,-10) \cup (-1,2)$
\odpStop
\testStart
A.$x \in (-\infty,-10) \cup (-1,2)$\\
B.$x \in (-\infty,-10) \cup (-1,2]$\\
C.$x \in (-\infty,-10) \cup [-1,2)$\\
D.$x \in (-\infty,-10] \cup (-1,2)$\\
E.$x \in (-\infty,-10] \cup (-1,2]$\\
F.$x \in (-\infty,-10] \cup [-1,2)$\\
G.$x \in (-\infty,-10) \cup [-1,2]$\\
H.$x \in (-\infty,-10] \cup [-1,2]$
\testStop
\kluczStart
A
\kluczStop



\zadStart{Zadanie z Wikieł Z 1.62 b) moja wersja nr 86}

Rozwiązać nierówności $(x+10)(3-x)(x+1)\ge0$.
\zadStop
\rozwStart{Patryk Wirkus}{}
Miejsca zerowe naszego wielomianu to: $-10, 3, -1$.\\
Wielomian jest stopnia nieparzystego, ponadto znak współczynnika przy\linebreak najwyższej potędze x jest ujemny.\\ W związku z tym wykres wielomianu zaczyna się od lewej strony powyżej osi OX. A więc $$x \in (-\infty,-10) \cup (-1,3).$$
\rozwStop
\odpStart
$x \in (-\infty,-10) \cup (-1,3)$
\odpStop
\testStart
A.$x \in (-\infty,-10) \cup (-1,3)$\\
B.$x \in (-\infty,-10) \cup (-1,3]$\\
C.$x \in (-\infty,-10) \cup [-1,3)$\\
D.$x \in (-\infty,-10] \cup (-1,3)$\\
E.$x \in (-\infty,-10] \cup (-1,3]$\\
F.$x \in (-\infty,-10] \cup [-1,3)$\\
G.$x \in (-\infty,-10) \cup [-1,3]$\\
H.$x \in (-\infty,-10] \cup [-1,3]$
\testStop
\kluczStart
A
\kluczStop



\zadStart{Zadanie z Wikieł Z 1.62 b) moja wersja nr 87}

Rozwiązać nierówności $(x+10)(3-x)(x+2)\ge0$.
\zadStop
\rozwStart{Patryk Wirkus}{}
Miejsca zerowe naszego wielomianu to: $-10, 3, -2$.\\
Wielomian jest stopnia nieparzystego, ponadto znak współczynnika przy\linebreak najwyższej potędze x jest ujemny.\\ W związku z tym wykres wielomianu zaczyna się od lewej strony powyżej osi OX. A więc $$x \in (-\infty,-10) \cup (-2,3).$$
\rozwStop
\odpStart
$x \in (-\infty,-10) \cup (-2,3)$
\odpStop
\testStart
A.$x \in (-\infty,-10) \cup (-2,3)$\\
B.$x \in (-\infty,-10) \cup (-2,3]$\\
C.$x \in (-\infty,-10) \cup [-2,3)$\\
D.$x \in (-\infty,-10] \cup (-2,3)$\\
E.$x \in (-\infty,-10] \cup (-2,3]$\\
F.$x \in (-\infty,-10] \cup [-2,3)$\\
G.$x \in (-\infty,-10) \cup [-2,3]$\\
H.$x \in (-\infty,-10] \cup [-2,3]$
\testStop
\kluczStart
A
\kluczStop



\zadStart{Zadanie z Wikieł Z 1.62 b) moja wersja nr 88}

Rozwiązać nierówności $(x+10)(4-x)(x+1)\ge0$.
\zadStop
\rozwStart{Patryk Wirkus}{}
Miejsca zerowe naszego wielomianu to: $-10, 4, -1$.\\
Wielomian jest stopnia nieparzystego, ponadto znak współczynnika przy\linebreak najwyższej potędze x jest ujemny.\\ W związku z tym wykres wielomianu zaczyna się od lewej strony powyżej osi OX. A więc $$x \in (-\infty,-10) \cup (-1,4).$$
\rozwStop
\odpStart
$x \in (-\infty,-10) \cup (-1,4)$
\odpStop
\testStart
A.$x \in (-\infty,-10) \cup (-1,4)$\\
B.$x \in (-\infty,-10) \cup (-1,4]$\\
C.$x \in (-\infty,-10) \cup [-1,4)$\\
D.$x \in (-\infty,-10] \cup (-1,4)$\\
E.$x \in (-\infty,-10] \cup (-1,4]$\\
F.$x \in (-\infty,-10] \cup [-1,4)$\\
G.$x \in (-\infty,-10) \cup [-1,4]$\\
H.$x \in (-\infty,-10] \cup [-1,4]$
\testStop
\kluczStart
A
\kluczStop



\zadStart{Zadanie z Wikieł Z 1.62 b) moja wersja nr 89}

Rozwiązać nierówności $(x+10)(4-x)(x+2)\ge0$.
\zadStop
\rozwStart{Patryk Wirkus}{}
Miejsca zerowe naszego wielomianu to: $-10, 4, -2$.\\
Wielomian jest stopnia nieparzystego, ponadto znak współczynnika przy\linebreak najwyższej potędze x jest ujemny.\\ W związku z tym wykres wielomianu zaczyna się od lewej strony powyżej osi OX. A więc $$x \in (-\infty,-10) \cup (-2,4).$$
\rozwStop
\odpStart
$x \in (-\infty,-10) \cup (-2,4)$
\odpStop
\testStart
A.$x \in (-\infty,-10) \cup (-2,4)$\\
B.$x \in (-\infty,-10) \cup (-2,4]$\\
C.$x \in (-\infty,-10) \cup [-2,4)$\\
D.$x \in (-\infty,-10] \cup (-2,4)$\\
E.$x \in (-\infty,-10] \cup (-2,4]$\\
F.$x \in (-\infty,-10] \cup [-2,4)$\\
G.$x \in (-\infty,-10) \cup [-2,4]$\\
H.$x \in (-\infty,-10] \cup [-2,4]$
\testStop
\kluczStart
A
\kluczStop



\zadStart{Zadanie z Wikieł Z 1.62 b) moja wersja nr 90}

Rozwiązać nierówności $(x+10)(4-x)(x+3)\ge0$.
\zadStop
\rozwStart{Patryk Wirkus}{}
Miejsca zerowe naszego wielomianu to: $-10, 4, -3$.\\
Wielomian jest stopnia nieparzystego, ponadto znak współczynnika przy\linebreak najwyższej potędze x jest ujemny.\\ W związku z tym wykres wielomianu zaczyna się od lewej strony powyżej osi OX. A więc $$x \in (-\infty,-10) \cup (-3,4).$$
\rozwStop
\odpStart
$x \in (-\infty,-10) \cup (-3,4)$
\odpStop
\testStart
A.$x \in (-\infty,-10) \cup (-3,4)$\\
B.$x \in (-\infty,-10) \cup (-3,4]$\\
C.$x \in (-\infty,-10) \cup [-3,4)$\\
D.$x \in (-\infty,-10] \cup (-3,4)$\\
E.$x \in (-\infty,-10] \cup (-3,4]$\\
F.$x \in (-\infty,-10] \cup [-3,4)$\\
G.$x \in (-\infty,-10) \cup [-3,4]$\\
H.$x \in (-\infty,-10] \cup [-3,4]$
\testStop
\kluczStart
A
\kluczStop



\zadStart{Zadanie z Wikieł Z 1.62 b) moja wersja nr 91}

Rozwiązać nierówności $(x+10)(5-x)(x+1)\ge0$.
\zadStop
\rozwStart{Patryk Wirkus}{}
Miejsca zerowe naszego wielomianu to: $-10, 5, -1$.\\
Wielomian jest stopnia nieparzystego, ponadto znak współczynnika przy\linebreak najwyższej potędze x jest ujemny.\\ W związku z tym wykres wielomianu zaczyna się od lewej strony powyżej osi OX. A więc $$x \in (-\infty,-10) \cup (-1,5).$$
\rozwStop
\odpStart
$x \in (-\infty,-10) \cup (-1,5)$
\odpStop
\testStart
A.$x \in (-\infty,-10) \cup (-1,5)$\\
B.$x \in (-\infty,-10) \cup (-1,5]$\\
C.$x \in (-\infty,-10) \cup [-1,5)$\\
D.$x \in (-\infty,-10] \cup (-1,5)$\\
E.$x \in (-\infty,-10] \cup (-1,5]$\\
F.$x \in (-\infty,-10] \cup [-1,5)$\\
G.$x \in (-\infty,-10) \cup [-1,5]$\\
H.$x \in (-\infty,-10] \cup [-1,5]$
\testStop
\kluczStart
A
\kluczStop



\zadStart{Zadanie z Wikieł Z 1.62 b) moja wersja nr 92}

Rozwiązać nierówności $(x+10)(5-x)(x+2)\ge0$.
\zadStop
\rozwStart{Patryk Wirkus}{}
Miejsca zerowe naszego wielomianu to: $-10, 5, -2$.\\
Wielomian jest stopnia nieparzystego, ponadto znak współczynnika przy\linebreak najwyższej potędze x jest ujemny.\\ W związku z tym wykres wielomianu zaczyna się od lewej strony powyżej osi OX. A więc $$x \in (-\infty,-10) \cup (-2,5).$$
\rozwStop
\odpStart
$x \in (-\infty,-10) \cup (-2,5)$
\odpStop
\testStart
A.$x \in (-\infty,-10) \cup (-2,5)$\\
B.$x \in (-\infty,-10) \cup (-2,5]$\\
C.$x \in (-\infty,-10) \cup [-2,5)$\\
D.$x \in (-\infty,-10] \cup (-2,5)$\\
E.$x \in (-\infty,-10] \cup (-2,5]$\\
F.$x \in (-\infty,-10] \cup [-2,5)$\\
G.$x \in (-\infty,-10) \cup [-2,5]$\\
H.$x \in (-\infty,-10] \cup [-2,5]$
\testStop
\kluczStart
A
\kluczStop



\zadStart{Zadanie z Wikieł Z 1.62 b) moja wersja nr 93}

Rozwiązać nierówności $(x+10)(5-x)(x+3)\ge0$.
\zadStop
\rozwStart{Patryk Wirkus}{}
Miejsca zerowe naszego wielomianu to: $-10, 5, -3$.\\
Wielomian jest stopnia nieparzystego, ponadto znak współczynnika przy\linebreak najwyższej potędze x jest ujemny.\\ W związku z tym wykres wielomianu zaczyna się od lewej strony powyżej osi OX. A więc $$x \in (-\infty,-10) \cup (-3,5).$$
\rozwStop
\odpStart
$x \in (-\infty,-10) \cup (-3,5)$
\odpStop
\testStart
A.$x \in (-\infty,-10) \cup (-3,5)$\\
B.$x \in (-\infty,-10) \cup (-3,5]$\\
C.$x \in (-\infty,-10) \cup [-3,5)$\\
D.$x \in (-\infty,-10] \cup (-3,5)$\\
E.$x \in (-\infty,-10] \cup (-3,5]$\\
F.$x \in (-\infty,-10] \cup [-3,5)$\\
G.$x \in (-\infty,-10) \cup [-3,5]$\\
H.$x \in (-\infty,-10] \cup [-3,5]$
\testStop
\kluczStart
A
\kluczStop



\zadStart{Zadanie z Wikieł Z 1.62 b) moja wersja nr 94}

Rozwiązać nierówności $(x+10)(5-x)(x+4)\ge0$.
\zadStop
\rozwStart{Patryk Wirkus}{}
Miejsca zerowe naszego wielomianu to: $-10, 5, -4$.\\
Wielomian jest stopnia nieparzystego, ponadto znak współczynnika przy\linebreak najwyższej potędze x jest ujemny.\\ W związku z tym wykres wielomianu zaczyna się od lewej strony powyżej osi OX. A więc $$x \in (-\infty,-10) \cup (-4,5).$$
\rozwStop
\odpStart
$x \in (-\infty,-10) \cup (-4,5)$
\odpStop
\testStart
A.$x \in (-\infty,-10) \cup (-4,5)$\\
B.$x \in (-\infty,-10) \cup (-4,5]$\\
C.$x \in (-\infty,-10) \cup [-4,5)$\\
D.$x \in (-\infty,-10] \cup (-4,5)$\\
E.$x \in (-\infty,-10] \cup (-4,5]$\\
F.$x \in (-\infty,-10] \cup [-4,5)$\\
G.$x \in (-\infty,-10) \cup [-4,5]$\\
H.$x \in (-\infty,-10] \cup [-4,5]$
\testStop
\kluczStart
A
\kluczStop



\zadStart{Zadanie z Wikieł Z 1.62 b) moja wersja nr 95}

Rozwiązać nierówności $(x+10)(6-x)(x+1)\ge0$.
\zadStop
\rozwStart{Patryk Wirkus}{}
Miejsca zerowe naszego wielomianu to: $-10, 6, -1$.\\
Wielomian jest stopnia nieparzystego, ponadto znak współczynnika przy\linebreak najwyższej potędze x jest ujemny.\\ W związku z tym wykres wielomianu zaczyna się od lewej strony powyżej osi OX. A więc $$x \in (-\infty,-10) \cup (-1,6).$$
\rozwStop
\odpStart
$x \in (-\infty,-10) \cup (-1,6)$
\odpStop
\testStart
A.$x \in (-\infty,-10) \cup (-1,6)$\\
B.$x \in (-\infty,-10) \cup (-1,6]$\\
C.$x \in (-\infty,-10) \cup [-1,6)$\\
D.$x \in (-\infty,-10] \cup (-1,6)$\\
E.$x \in (-\infty,-10] \cup (-1,6]$\\
F.$x \in (-\infty,-10] \cup [-1,6)$\\
G.$x \in (-\infty,-10) \cup [-1,6]$\\
H.$x \in (-\infty,-10] \cup [-1,6]$
\testStop
\kluczStart
A
\kluczStop



\zadStart{Zadanie z Wikieł Z 1.62 b) moja wersja nr 96}

Rozwiązać nierówności $(x+10)(6-x)(x+2)\ge0$.
\zadStop
\rozwStart{Patryk Wirkus}{}
Miejsca zerowe naszego wielomianu to: $-10, 6, -2$.\\
Wielomian jest stopnia nieparzystego, ponadto znak współczynnika przy\linebreak najwyższej potędze x jest ujemny.\\ W związku z tym wykres wielomianu zaczyna się od lewej strony powyżej osi OX. A więc $$x \in (-\infty,-10) \cup (-2,6).$$
\rozwStop
\odpStart
$x \in (-\infty,-10) \cup (-2,6)$
\odpStop
\testStart
A.$x \in (-\infty,-10) \cup (-2,6)$\\
B.$x \in (-\infty,-10) \cup (-2,6]$\\
C.$x \in (-\infty,-10) \cup [-2,6)$\\
D.$x \in (-\infty,-10] \cup (-2,6)$\\
E.$x \in (-\infty,-10] \cup (-2,6]$\\
F.$x \in (-\infty,-10] \cup [-2,6)$\\
G.$x \in (-\infty,-10) \cup [-2,6]$\\
H.$x \in (-\infty,-10] \cup [-2,6]$
\testStop
\kluczStart
A
\kluczStop



\zadStart{Zadanie z Wikieł Z 1.62 b) moja wersja nr 97}

Rozwiązać nierówności $(x+10)(6-x)(x+3)\ge0$.
\zadStop
\rozwStart{Patryk Wirkus}{}
Miejsca zerowe naszego wielomianu to: $-10, 6, -3$.\\
Wielomian jest stopnia nieparzystego, ponadto znak współczynnika przy\linebreak najwyższej potędze x jest ujemny.\\ W związku z tym wykres wielomianu zaczyna się od lewej strony powyżej osi OX. A więc $$x \in (-\infty,-10) \cup (-3,6).$$
\rozwStop
\odpStart
$x \in (-\infty,-10) \cup (-3,6)$
\odpStop
\testStart
A.$x \in (-\infty,-10) \cup (-3,6)$\\
B.$x \in (-\infty,-10) \cup (-3,6]$\\
C.$x \in (-\infty,-10) \cup [-3,6)$\\
D.$x \in (-\infty,-10] \cup (-3,6)$\\
E.$x \in (-\infty,-10] \cup (-3,6]$\\
F.$x \in (-\infty,-10] \cup [-3,6)$\\
G.$x \in (-\infty,-10) \cup [-3,6]$\\
H.$x \in (-\infty,-10] \cup [-3,6]$
\testStop
\kluczStart
A
\kluczStop



\zadStart{Zadanie z Wikieł Z 1.62 b) moja wersja nr 98}

Rozwiązać nierówności $(x+10)(6-x)(x+4)\ge0$.
\zadStop
\rozwStart{Patryk Wirkus}{}
Miejsca zerowe naszego wielomianu to: $-10, 6, -4$.\\
Wielomian jest stopnia nieparzystego, ponadto znak współczynnika przy\linebreak najwyższej potędze x jest ujemny.\\ W związku z tym wykres wielomianu zaczyna się od lewej strony powyżej osi OX. A więc $$x \in (-\infty,-10) \cup (-4,6).$$
\rozwStop
\odpStart
$x \in (-\infty,-10) \cup (-4,6)$
\odpStop
\testStart
A.$x \in (-\infty,-10) \cup (-4,6)$\\
B.$x \in (-\infty,-10) \cup (-4,6]$\\
C.$x \in (-\infty,-10) \cup [-4,6)$\\
D.$x \in (-\infty,-10] \cup (-4,6)$\\
E.$x \in (-\infty,-10] \cup (-4,6]$\\
F.$x \in (-\infty,-10] \cup [-4,6)$\\
G.$x \in (-\infty,-10) \cup [-4,6]$\\
H.$x \in (-\infty,-10] \cup [-4,6]$
\testStop
\kluczStart
A
\kluczStop



\zadStart{Zadanie z Wikieł Z 1.62 b) moja wersja nr 99}

Rozwiązać nierówności $(x+10)(6-x)(x+5)\ge0$.
\zadStop
\rozwStart{Patryk Wirkus}{}
Miejsca zerowe naszego wielomianu to: $-10, 6, -5$.\\
Wielomian jest stopnia nieparzystego, ponadto znak współczynnika przy\linebreak najwyższej potędze x jest ujemny.\\ W związku z tym wykres wielomianu zaczyna się od lewej strony powyżej osi OX. A więc $$x \in (-\infty,-10) \cup (-5,6).$$
\rozwStop
\odpStart
$x \in (-\infty,-10) \cup (-5,6)$
\odpStop
\testStart
A.$x \in (-\infty,-10) \cup (-5,6)$\\
B.$x \in (-\infty,-10) \cup (-5,6]$\\
C.$x \in (-\infty,-10) \cup [-5,6)$\\
D.$x \in (-\infty,-10] \cup (-5,6)$\\
E.$x \in (-\infty,-10] \cup (-5,6]$\\
F.$x \in (-\infty,-10] \cup [-5,6)$\\
G.$x \in (-\infty,-10) \cup [-5,6]$\\
H.$x \in (-\infty,-10] \cup [-5,6]$
\testStop
\kluczStart
A
\kluczStop



\zadStart{Zadanie z Wikieł Z 1.62 b) moja wersja nr 100}

Rozwiązać nierówności $(x+10)(7-x)(x+1)\ge0$.
\zadStop
\rozwStart{Patryk Wirkus}{}
Miejsca zerowe naszego wielomianu to: $-10, 7, -1$.\\
Wielomian jest stopnia nieparzystego, ponadto znak współczynnika przy\linebreak najwyższej potędze x jest ujemny.\\ W związku z tym wykres wielomianu zaczyna się od lewej strony powyżej osi OX. A więc $$x \in (-\infty,-10) \cup (-1,7).$$
\rozwStop
\odpStart
$x \in (-\infty,-10) \cup (-1,7)$
\odpStop
\testStart
A.$x \in (-\infty,-10) \cup (-1,7)$\\
B.$x \in (-\infty,-10) \cup (-1,7]$\\
C.$x \in (-\infty,-10) \cup [-1,7)$\\
D.$x \in (-\infty,-10] \cup (-1,7)$\\
E.$x \in (-\infty,-10] \cup (-1,7]$\\
F.$x \in (-\infty,-10] \cup [-1,7)$\\
G.$x \in (-\infty,-10) \cup [-1,7]$\\
H.$x \in (-\infty,-10] \cup [-1,7]$
\testStop
\kluczStart
A
\kluczStop



\zadStart{Zadanie z Wikieł Z 1.62 b) moja wersja nr 101}

Rozwiązać nierówności $(x+10)(7-x)(x+2)\ge0$.
\zadStop
\rozwStart{Patryk Wirkus}{}
Miejsca zerowe naszego wielomianu to: $-10, 7, -2$.\\
Wielomian jest stopnia nieparzystego, ponadto znak współczynnika przy\linebreak najwyższej potędze x jest ujemny.\\ W związku z tym wykres wielomianu zaczyna się od lewej strony powyżej osi OX. A więc $$x \in (-\infty,-10) \cup (-2,7).$$
\rozwStop
\odpStart
$x \in (-\infty,-10) \cup (-2,7)$
\odpStop
\testStart
A.$x \in (-\infty,-10) \cup (-2,7)$\\
B.$x \in (-\infty,-10) \cup (-2,7]$\\
C.$x \in (-\infty,-10) \cup [-2,7)$\\
D.$x \in (-\infty,-10] \cup (-2,7)$\\
E.$x \in (-\infty,-10] \cup (-2,7]$\\
F.$x \in (-\infty,-10] \cup [-2,7)$\\
G.$x \in (-\infty,-10) \cup [-2,7]$\\
H.$x \in (-\infty,-10] \cup [-2,7]$
\testStop
\kluczStart
A
\kluczStop



\zadStart{Zadanie z Wikieł Z 1.62 b) moja wersja nr 102}

Rozwiązać nierówności $(x+10)(7-x)(x+3)\ge0$.
\zadStop
\rozwStart{Patryk Wirkus}{}
Miejsca zerowe naszego wielomianu to: $-10, 7, -3$.\\
Wielomian jest stopnia nieparzystego, ponadto znak współczynnika przy\linebreak najwyższej potędze x jest ujemny.\\ W związku z tym wykres wielomianu zaczyna się od lewej strony powyżej osi OX. A więc $$x \in (-\infty,-10) \cup (-3,7).$$
\rozwStop
\odpStart
$x \in (-\infty,-10) \cup (-3,7)$
\odpStop
\testStart
A.$x \in (-\infty,-10) \cup (-3,7)$\\
B.$x \in (-\infty,-10) \cup (-3,7]$\\
C.$x \in (-\infty,-10) \cup [-3,7)$\\
D.$x \in (-\infty,-10] \cup (-3,7)$\\
E.$x \in (-\infty,-10] \cup (-3,7]$\\
F.$x \in (-\infty,-10] \cup [-3,7)$\\
G.$x \in (-\infty,-10) \cup [-3,7]$\\
H.$x \in (-\infty,-10] \cup [-3,7]$
\testStop
\kluczStart
A
\kluczStop



\zadStart{Zadanie z Wikieł Z 1.62 b) moja wersja nr 103}

Rozwiązać nierówności $(x+10)(7-x)(x+4)\ge0$.
\zadStop
\rozwStart{Patryk Wirkus}{}
Miejsca zerowe naszego wielomianu to: $-10, 7, -4$.\\
Wielomian jest stopnia nieparzystego, ponadto znak współczynnika przy\linebreak najwyższej potędze x jest ujemny.\\ W związku z tym wykres wielomianu zaczyna się od lewej strony powyżej osi OX. A więc $$x \in (-\infty,-10) \cup (-4,7).$$
\rozwStop
\odpStart
$x \in (-\infty,-10) \cup (-4,7)$
\odpStop
\testStart
A.$x \in (-\infty,-10) \cup (-4,7)$\\
B.$x \in (-\infty,-10) \cup (-4,7]$\\
C.$x \in (-\infty,-10) \cup [-4,7)$\\
D.$x \in (-\infty,-10] \cup (-4,7)$\\
E.$x \in (-\infty,-10] \cup (-4,7]$\\
F.$x \in (-\infty,-10] \cup [-4,7)$\\
G.$x \in (-\infty,-10) \cup [-4,7]$\\
H.$x \in (-\infty,-10] \cup [-4,7]$
\testStop
\kluczStart
A
\kluczStop



\zadStart{Zadanie z Wikieł Z 1.62 b) moja wersja nr 104}

Rozwiązać nierówności $(x+10)(7-x)(x+5)\ge0$.
\zadStop
\rozwStart{Patryk Wirkus}{}
Miejsca zerowe naszego wielomianu to: $-10, 7, -5$.\\
Wielomian jest stopnia nieparzystego, ponadto znak współczynnika przy\linebreak najwyższej potędze x jest ujemny.\\ W związku z tym wykres wielomianu zaczyna się od lewej strony powyżej osi OX. A więc $$x \in (-\infty,-10) \cup (-5,7).$$
\rozwStop
\odpStart
$x \in (-\infty,-10) \cup (-5,7)$
\odpStop
\testStart
A.$x \in (-\infty,-10) \cup (-5,7)$\\
B.$x \in (-\infty,-10) \cup (-5,7]$\\
C.$x \in (-\infty,-10) \cup [-5,7)$\\
D.$x \in (-\infty,-10] \cup (-5,7)$\\
E.$x \in (-\infty,-10] \cup (-5,7]$\\
F.$x \in (-\infty,-10] \cup [-5,7)$\\
G.$x \in (-\infty,-10) \cup [-5,7]$\\
H.$x \in (-\infty,-10] \cup [-5,7]$
\testStop
\kluczStart
A
\kluczStop



\zadStart{Zadanie z Wikieł Z 1.62 b) moja wersja nr 105}

Rozwiązać nierówności $(x+10)(7-x)(x+6)\ge0$.
\zadStop
\rozwStart{Patryk Wirkus}{}
Miejsca zerowe naszego wielomianu to: $-10, 7, -6$.\\
Wielomian jest stopnia nieparzystego, ponadto znak współczynnika przy\linebreak najwyższej potędze x jest ujemny.\\ W związku z tym wykres wielomianu zaczyna się od lewej strony powyżej osi OX. A więc $$x \in (-\infty,-10) \cup (-6,7).$$
\rozwStop
\odpStart
$x \in (-\infty,-10) \cup (-6,7)$
\odpStop
\testStart
A.$x \in (-\infty,-10) \cup (-6,7)$\\
B.$x \in (-\infty,-10) \cup (-6,7]$\\
C.$x \in (-\infty,-10) \cup [-6,7)$\\
D.$x \in (-\infty,-10] \cup (-6,7)$\\
E.$x \in (-\infty,-10] \cup (-6,7]$\\
F.$x \in (-\infty,-10] \cup [-6,7)$\\
G.$x \in (-\infty,-10) \cup [-6,7]$\\
H.$x \in (-\infty,-10] \cup [-6,7]$
\testStop
\kluczStart
A
\kluczStop



\zadStart{Zadanie z Wikieł Z 1.62 b) moja wersja nr 106}

Rozwiązać nierówności $(x+10)(8-x)(x+1)\ge0$.
\zadStop
\rozwStart{Patryk Wirkus}{}
Miejsca zerowe naszego wielomianu to: $-10, 8, -1$.\\
Wielomian jest stopnia nieparzystego, ponadto znak współczynnika przy\linebreak najwyższej potędze x jest ujemny.\\ W związku z tym wykres wielomianu zaczyna się od lewej strony powyżej osi OX. A więc $$x \in (-\infty,-10) \cup (-1,8).$$
\rozwStop
\odpStart
$x \in (-\infty,-10) \cup (-1,8)$
\odpStop
\testStart
A.$x \in (-\infty,-10) \cup (-1,8)$\\
B.$x \in (-\infty,-10) \cup (-1,8]$\\
C.$x \in (-\infty,-10) \cup [-1,8)$\\
D.$x \in (-\infty,-10] \cup (-1,8)$\\
E.$x \in (-\infty,-10] \cup (-1,8]$\\
F.$x \in (-\infty,-10] \cup [-1,8)$\\
G.$x \in (-\infty,-10) \cup [-1,8]$\\
H.$x \in (-\infty,-10] \cup [-1,8]$
\testStop
\kluczStart
A
\kluczStop



\zadStart{Zadanie z Wikieł Z 1.62 b) moja wersja nr 107}

Rozwiązać nierówności $(x+10)(8-x)(x+2)\ge0$.
\zadStop
\rozwStart{Patryk Wirkus}{}
Miejsca zerowe naszego wielomianu to: $-10, 8, -2$.\\
Wielomian jest stopnia nieparzystego, ponadto znak współczynnika przy\linebreak najwyższej potędze x jest ujemny.\\ W związku z tym wykres wielomianu zaczyna się od lewej strony powyżej osi OX. A więc $$x \in (-\infty,-10) \cup (-2,8).$$
\rozwStop
\odpStart
$x \in (-\infty,-10) \cup (-2,8)$
\odpStop
\testStart
A.$x \in (-\infty,-10) \cup (-2,8)$\\
B.$x \in (-\infty,-10) \cup (-2,8]$\\
C.$x \in (-\infty,-10) \cup [-2,8)$\\
D.$x \in (-\infty,-10] \cup (-2,8)$\\
E.$x \in (-\infty,-10] \cup (-2,8]$\\
F.$x \in (-\infty,-10] \cup [-2,8)$\\
G.$x \in (-\infty,-10) \cup [-2,8]$\\
H.$x \in (-\infty,-10] \cup [-2,8]$
\testStop
\kluczStart
A
\kluczStop



\zadStart{Zadanie z Wikieł Z 1.62 b) moja wersja nr 108}

Rozwiązać nierówności $(x+10)(8-x)(x+3)\ge0$.
\zadStop
\rozwStart{Patryk Wirkus}{}
Miejsca zerowe naszego wielomianu to: $-10, 8, -3$.\\
Wielomian jest stopnia nieparzystego, ponadto znak współczynnika przy\linebreak najwyższej potędze x jest ujemny.\\ W związku z tym wykres wielomianu zaczyna się od lewej strony powyżej osi OX. A więc $$x \in (-\infty,-10) \cup (-3,8).$$
\rozwStop
\odpStart
$x \in (-\infty,-10) \cup (-3,8)$
\odpStop
\testStart
A.$x \in (-\infty,-10) \cup (-3,8)$\\
B.$x \in (-\infty,-10) \cup (-3,8]$\\
C.$x \in (-\infty,-10) \cup [-3,8)$\\
D.$x \in (-\infty,-10] \cup (-3,8)$\\
E.$x \in (-\infty,-10] \cup (-3,8]$\\
F.$x \in (-\infty,-10] \cup [-3,8)$\\
G.$x \in (-\infty,-10) \cup [-3,8]$\\
H.$x \in (-\infty,-10] \cup [-3,8]$
\testStop
\kluczStart
A
\kluczStop



\zadStart{Zadanie z Wikieł Z 1.62 b) moja wersja nr 109}

Rozwiązać nierówności $(x+10)(8-x)(x+4)\ge0$.
\zadStop
\rozwStart{Patryk Wirkus}{}
Miejsca zerowe naszego wielomianu to: $-10, 8, -4$.\\
Wielomian jest stopnia nieparzystego, ponadto znak współczynnika przy\linebreak najwyższej potędze x jest ujemny.\\ W związku z tym wykres wielomianu zaczyna się od lewej strony powyżej osi OX. A więc $$x \in (-\infty,-10) \cup (-4,8).$$
\rozwStop
\odpStart
$x \in (-\infty,-10) \cup (-4,8)$
\odpStop
\testStart
A.$x \in (-\infty,-10) \cup (-4,8)$\\
B.$x \in (-\infty,-10) \cup (-4,8]$\\
C.$x \in (-\infty,-10) \cup [-4,8)$\\
D.$x \in (-\infty,-10] \cup (-4,8)$\\
E.$x \in (-\infty,-10] \cup (-4,8]$\\
F.$x \in (-\infty,-10] \cup [-4,8)$\\
G.$x \in (-\infty,-10) \cup [-4,8]$\\
H.$x \in (-\infty,-10] \cup [-4,8]$
\testStop
\kluczStart
A
\kluczStop



\zadStart{Zadanie z Wikieł Z 1.62 b) moja wersja nr 110}

Rozwiązać nierówności $(x+10)(8-x)(x+5)\ge0$.
\zadStop
\rozwStart{Patryk Wirkus}{}
Miejsca zerowe naszego wielomianu to: $-10, 8, -5$.\\
Wielomian jest stopnia nieparzystego, ponadto znak współczynnika przy\linebreak najwyższej potędze x jest ujemny.\\ W związku z tym wykres wielomianu zaczyna się od lewej strony powyżej osi OX. A więc $$x \in (-\infty,-10) \cup (-5,8).$$
\rozwStop
\odpStart
$x \in (-\infty,-10) \cup (-5,8)$
\odpStop
\testStart
A.$x \in (-\infty,-10) \cup (-5,8)$\\
B.$x \in (-\infty,-10) \cup (-5,8]$\\
C.$x \in (-\infty,-10) \cup [-5,8)$\\
D.$x \in (-\infty,-10] \cup (-5,8)$\\
E.$x \in (-\infty,-10] \cup (-5,8]$\\
F.$x \in (-\infty,-10] \cup [-5,8)$\\
G.$x \in (-\infty,-10) \cup [-5,8]$\\
H.$x \in (-\infty,-10] \cup [-5,8]$
\testStop
\kluczStart
A
\kluczStop



\zadStart{Zadanie z Wikieł Z 1.62 b) moja wersja nr 111}

Rozwiązać nierówności $(x+10)(8-x)(x+6)\ge0$.
\zadStop
\rozwStart{Patryk Wirkus}{}
Miejsca zerowe naszego wielomianu to: $-10, 8, -6$.\\
Wielomian jest stopnia nieparzystego, ponadto znak współczynnika przy\linebreak najwyższej potędze x jest ujemny.\\ W związku z tym wykres wielomianu zaczyna się od lewej strony powyżej osi OX. A więc $$x \in (-\infty,-10) \cup (-6,8).$$
\rozwStop
\odpStart
$x \in (-\infty,-10) \cup (-6,8)$
\odpStop
\testStart
A.$x \in (-\infty,-10) \cup (-6,8)$\\
B.$x \in (-\infty,-10) \cup (-6,8]$\\
C.$x \in (-\infty,-10) \cup [-6,8)$\\
D.$x \in (-\infty,-10] \cup (-6,8)$\\
E.$x \in (-\infty,-10] \cup (-6,8]$\\
F.$x \in (-\infty,-10] \cup [-6,8)$\\
G.$x \in (-\infty,-10) \cup [-6,8]$\\
H.$x \in (-\infty,-10] \cup [-6,8]$
\testStop
\kluczStart
A
\kluczStop



\zadStart{Zadanie z Wikieł Z 1.62 b) moja wersja nr 112}

Rozwiązać nierówności $(x+10)(8-x)(x+7)\ge0$.
\zadStop
\rozwStart{Patryk Wirkus}{}
Miejsca zerowe naszego wielomianu to: $-10, 8, -7$.\\
Wielomian jest stopnia nieparzystego, ponadto znak współczynnika przy\linebreak najwyższej potędze x jest ujemny.\\ W związku z tym wykres wielomianu zaczyna się od lewej strony powyżej osi OX. A więc $$x \in (-\infty,-10) \cup (-7,8).$$
\rozwStop
\odpStart
$x \in (-\infty,-10) \cup (-7,8)$
\odpStop
\testStart
A.$x \in (-\infty,-10) \cup (-7,8)$\\
B.$x \in (-\infty,-10) \cup (-7,8]$\\
C.$x \in (-\infty,-10) \cup [-7,8)$\\
D.$x \in (-\infty,-10] \cup (-7,8)$\\
E.$x \in (-\infty,-10] \cup (-7,8]$\\
F.$x \in (-\infty,-10] \cup [-7,8)$\\
G.$x \in (-\infty,-10) \cup [-7,8]$\\
H.$x \in (-\infty,-10] \cup [-7,8]$
\testStop
\kluczStart
A
\kluczStop



\zadStart{Zadanie z Wikieł Z 1.62 b) moja wersja nr 113}

Rozwiązać nierówności $(x+10)(9-x)(x+1)\ge0$.
\zadStop
\rozwStart{Patryk Wirkus}{}
Miejsca zerowe naszego wielomianu to: $-10, 9, -1$.\\
Wielomian jest stopnia nieparzystego, ponadto znak współczynnika przy\linebreak najwyższej potędze x jest ujemny.\\ W związku z tym wykres wielomianu zaczyna się od lewej strony powyżej osi OX. A więc $$x \in (-\infty,-10) \cup (-1,9).$$
\rozwStop
\odpStart
$x \in (-\infty,-10) \cup (-1,9)$
\odpStop
\testStart
A.$x \in (-\infty,-10) \cup (-1,9)$\\
B.$x \in (-\infty,-10) \cup (-1,9]$\\
C.$x \in (-\infty,-10) \cup [-1,9)$\\
D.$x \in (-\infty,-10] \cup (-1,9)$\\
E.$x \in (-\infty,-10] \cup (-1,9]$\\
F.$x \in (-\infty,-10] \cup [-1,9)$\\
G.$x \in (-\infty,-10) \cup [-1,9]$\\
H.$x \in (-\infty,-10] \cup [-1,9]$
\testStop
\kluczStart
A
\kluczStop



\zadStart{Zadanie z Wikieł Z 1.62 b) moja wersja nr 114}

Rozwiązać nierówności $(x+10)(9-x)(x+2)\ge0$.
\zadStop
\rozwStart{Patryk Wirkus}{}
Miejsca zerowe naszego wielomianu to: $-10, 9, -2$.\\
Wielomian jest stopnia nieparzystego, ponadto znak współczynnika przy\linebreak najwyższej potędze x jest ujemny.\\ W związku z tym wykres wielomianu zaczyna się od lewej strony powyżej osi OX. A więc $$x \in (-\infty,-10) \cup (-2,9).$$
\rozwStop
\odpStart
$x \in (-\infty,-10) \cup (-2,9)$
\odpStop
\testStart
A.$x \in (-\infty,-10) \cup (-2,9)$\\
B.$x \in (-\infty,-10) \cup (-2,9]$\\
C.$x \in (-\infty,-10) \cup [-2,9)$\\
D.$x \in (-\infty,-10] \cup (-2,9)$\\
E.$x \in (-\infty,-10] \cup (-2,9]$\\
F.$x \in (-\infty,-10] \cup [-2,9)$\\
G.$x \in (-\infty,-10) \cup [-2,9]$\\
H.$x \in (-\infty,-10] \cup [-2,9]$
\testStop
\kluczStart
A
\kluczStop



\zadStart{Zadanie z Wikieł Z 1.62 b) moja wersja nr 115}

Rozwiązać nierówności $(x+10)(9-x)(x+3)\ge0$.
\zadStop
\rozwStart{Patryk Wirkus}{}
Miejsca zerowe naszego wielomianu to: $-10, 9, -3$.\\
Wielomian jest stopnia nieparzystego, ponadto znak współczynnika przy\linebreak najwyższej potędze x jest ujemny.\\ W związku z tym wykres wielomianu zaczyna się od lewej strony powyżej osi OX. A więc $$x \in (-\infty,-10) \cup (-3,9).$$
\rozwStop
\odpStart
$x \in (-\infty,-10) \cup (-3,9)$
\odpStop
\testStart
A.$x \in (-\infty,-10) \cup (-3,9)$\\
B.$x \in (-\infty,-10) \cup (-3,9]$\\
C.$x \in (-\infty,-10) \cup [-3,9)$\\
D.$x \in (-\infty,-10] \cup (-3,9)$\\
E.$x \in (-\infty,-10] \cup (-3,9]$\\
F.$x \in (-\infty,-10] \cup [-3,9)$\\
G.$x \in (-\infty,-10) \cup [-3,9]$\\
H.$x \in (-\infty,-10] \cup [-3,9]$
\testStop
\kluczStart
A
\kluczStop



\zadStart{Zadanie z Wikieł Z 1.62 b) moja wersja nr 116}

Rozwiązać nierówności $(x+10)(9-x)(x+4)\ge0$.
\zadStop
\rozwStart{Patryk Wirkus}{}
Miejsca zerowe naszego wielomianu to: $-10, 9, -4$.\\
Wielomian jest stopnia nieparzystego, ponadto znak współczynnika przy\linebreak najwyższej potędze x jest ujemny.\\ W związku z tym wykres wielomianu zaczyna się od lewej strony powyżej osi OX. A więc $$x \in (-\infty,-10) \cup (-4,9).$$
\rozwStop
\odpStart
$x \in (-\infty,-10) \cup (-4,9)$
\odpStop
\testStart
A.$x \in (-\infty,-10) \cup (-4,9)$\\
B.$x \in (-\infty,-10) \cup (-4,9]$\\
C.$x \in (-\infty,-10) \cup [-4,9)$\\
D.$x \in (-\infty,-10] \cup (-4,9)$\\
E.$x \in (-\infty,-10] \cup (-4,9]$\\
F.$x \in (-\infty,-10] \cup [-4,9)$\\
G.$x \in (-\infty,-10) \cup [-4,9]$\\
H.$x \in (-\infty,-10] \cup [-4,9]$
\testStop
\kluczStart
A
\kluczStop



\zadStart{Zadanie z Wikieł Z 1.62 b) moja wersja nr 117}

Rozwiązać nierówności $(x+10)(9-x)(x+5)\ge0$.
\zadStop
\rozwStart{Patryk Wirkus}{}
Miejsca zerowe naszego wielomianu to: $-10, 9, -5$.\\
Wielomian jest stopnia nieparzystego, ponadto znak współczynnika przy\linebreak najwyższej potędze x jest ujemny.\\ W związku z tym wykres wielomianu zaczyna się od lewej strony powyżej osi OX. A więc $$x \in (-\infty,-10) \cup (-5,9).$$
\rozwStop
\odpStart
$x \in (-\infty,-10) \cup (-5,9)$
\odpStop
\testStart
A.$x \in (-\infty,-10) \cup (-5,9)$\\
B.$x \in (-\infty,-10) \cup (-5,9]$\\
C.$x \in (-\infty,-10) \cup [-5,9)$\\
D.$x \in (-\infty,-10] \cup (-5,9)$\\
E.$x \in (-\infty,-10] \cup (-5,9]$\\
F.$x \in (-\infty,-10] \cup [-5,9)$\\
G.$x \in (-\infty,-10) \cup [-5,9]$\\
H.$x \in (-\infty,-10] \cup [-5,9]$
\testStop
\kluczStart
A
\kluczStop



\zadStart{Zadanie z Wikieł Z 1.62 b) moja wersja nr 118}

Rozwiązać nierówności $(x+10)(9-x)(x+6)\ge0$.
\zadStop
\rozwStart{Patryk Wirkus}{}
Miejsca zerowe naszego wielomianu to: $-10, 9, -6$.\\
Wielomian jest stopnia nieparzystego, ponadto znak współczynnika przy\linebreak najwyższej potędze x jest ujemny.\\ W związku z tym wykres wielomianu zaczyna się od lewej strony powyżej osi OX. A więc $$x \in (-\infty,-10) \cup (-6,9).$$
\rozwStop
\odpStart
$x \in (-\infty,-10) \cup (-6,9)$
\odpStop
\testStart
A.$x \in (-\infty,-10) \cup (-6,9)$\\
B.$x \in (-\infty,-10) \cup (-6,9]$\\
C.$x \in (-\infty,-10) \cup [-6,9)$\\
D.$x \in (-\infty,-10] \cup (-6,9)$\\
E.$x \in (-\infty,-10] \cup (-6,9]$\\
F.$x \in (-\infty,-10] \cup [-6,9)$\\
G.$x \in (-\infty,-10) \cup [-6,9]$\\
H.$x \in (-\infty,-10] \cup [-6,9]$
\testStop
\kluczStart
A
\kluczStop



\zadStart{Zadanie z Wikieł Z 1.62 b) moja wersja nr 119}

Rozwiązać nierówności $(x+10)(9-x)(x+7)\ge0$.
\zadStop
\rozwStart{Patryk Wirkus}{}
Miejsca zerowe naszego wielomianu to: $-10, 9, -7$.\\
Wielomian jest stopnia nieparzystego, ponadto znak współczynnika przy\linebreak najwyższej potędze x jest ujemny.\\ W związku z tym wykres wielomianu zaczyna się od lewej strony powyżej osi OX. A więc $$x \in (-\infty,-10) \cup (-7,9).$$
\rozwStop
\odpStart
$x \in (-\infty,-10) \cup (-7,9)$
\odpStop
\testStart
A.$x \in (-\infty,-10) \cup (-7,9)$\\
B.$x \in (-\infty,-10) \cup (-7,9]$\\
C.$x \in (-\infty,-10) \cup [-7,9)$\\
D.$x \in (-\infty,-10] \cup (-7,9)$\\
E.$x \in (-\infty,-10] \cup (-7,9]$\\
F.$x \in (-\infty,-10] \cup [-7,9)$\\
G.$x \in (-\infty,-10) \cup [-7,9]$\\
H.$x \in (-\infty,-10] \cup [-7,9]$
\testStop
\kluczStart
A
\kluczStop



\zadStart{Zadanie z Wikieł Z 1.62 b) moja wersja nr 120}

Rozwiązać nierówności $(x+10)(9-x)(x+8)\ge0$.
\zadStop
\rozwStart{Patryk Wirkus}{}
Miejsca zerowe naszego wielomianu to: $-10, 9, -8$.\\
Wielomian jest stopnia nieparzystego, ponadto znak współczynnika przy\linebreak najwyższej potędze x jest ujemny.\\ W związku z tym wykres wielomianu zaczyna się od lewej strony powyżej osi OX. A więc $$x \in (-\infty,-10) \cup (-8,9).$$
\rozwStop
\odpStart
$x \in (-\infty,-10) \cup (-8,9)$
\odpStop
\testStart
A.$x \in (-\infty,-10) \cup (-8,9)$\\
B.$x \in (-\infty,-10) \cup (-8,9]$\\
C.$x \in (-\infty,-10) \cup [-8,9)$\\
D.$x \in (-\infty,-10] \cup (-8,9)$\\
E.$x \in (-\infty,-10] \cup (-8,9]$\\
F.$x \in (-\infty,-10] \cup [-8,9)$\\
G.$x \in (-\infty,-10) \cup [-8,9]$\\
H.$x \in (-\infty,-10] \cup [-8,9]$
\testStop
\kluczStart
A
\kluczStop



\zadStart{Zadanie z Wikieł Z 1.62 b) moja wersja nr 121}

Rozwiązać nierówności $(x+11)(2-x)(x+1)\ge0$.
\zadStop
\rozwStart{Patryk Wirkus}{}
Miejsca zerowe naszego wielomianu to: $-11, 2, -1$.\\
Wielomian jest stopnia nieparzystego, ponadto znak współczynnika przy\linebreak najwyższej potędze x jest ujemny.\\ W związku z tym wykres wielomianu zaczyna się od lewej strony powyżej osi OX. A więc $$x \in (-\infty,-11) \cup (-1,2).$$
\rozwStop
\odpStart
$x \in (-\infty,-11) \cup (-1,2)$
\odpStop
\testStart
A.$x \in (-\infty,-11) \cup (-1,2)$\\
B.$x \in (-\infty,-11) \cup (-1,2]$\\
C.$x \in (-\infty,-11) \cup [-1,2)$\\
D.$x \in (-\infty,-11] \cup (-1,2)$\\
E.$x \in (-\infty,-11] \cup (-1,2]$\\
F.$x \in (-\infty,-11] \cup [-1,2)$\\
G.$x \in (-\infty,-11) \cup [-1,2]$\\
H.$x \in (-\infty,-11] \cup [-1,2]$
\testStop
\kluczStart
A
\kluczStop



\zadStart{Zadanie z Wikieł Z 1.62 b) moja wersja nr 122}

Rozwiązać nierówności $(x+11)(3-x)(x+1)\ge0$.
\zadStop
\rozwStart{Patryk Wirkus}{}
Miejsca zerowe naszego wielomianu to: $-11, 3, -1$.\\
Wielomian jest stopnia nieparzystego, ponadto znak współczynnika przy\linebreak najwyższej potędze x jest ujemny.\\ W związku z tym wykres wielomianu zaczyna się od lewej strony powyżej osi OX. A więc $$x \in (-\infty,-11) \cup (-1,3).$$
\rozwStop
\odpStart
$x \in (-\infty,-11) \cup (-1,3)$
\odpStop
\testStart
A.$x \in (-\infty,-11) \cup (-1,3)$\\
B.$x \in (-\infty,-11) \cup (-1,3]$\\
C.$x \in (-\infty,-11) \cup [-1,3)$\\
D.$x \in (-\infty,-11] \cup (-1,3)$\\
E.$x \in (-\infty,-11] \cup (-1,3]$\\
F.$x \in (-\infty,-11] \cup [-1,3)$\\
G.$x \in (-\infty,-11) \cup [-1,3]$\\
H.$x \in (-\infty,-11] \cup [-1,3]$
\testStop
\kluczStart
A
\kluczStop



\zadStart{Zadanie z Wikieł Z 1.62 b) moja wersja nr 123}

Rozwiązać nierówności $(x+11)(3-x)(x+2)\ge0$.
\zadStop
\rozwStart{Patryk Wirkus}{}
Miejsca zerowe naszego wielomianu to: $-11, 3, -2$.\\
Wielomian jest stopnia nieparzystego, ponadto znak współczynnika przy\linebreak najwyższej potędze x jest ujemny.\\ W związku z tym wykres wielomianu zaczyna się od lewej strony powyżej osi OX. A więc $$x \in (-\infty,-11) \cup (-2,3).$$
\rozwStop
\odpStart
$x \in (-\infty,-11) \cup (-2,3)$
\odpStop
\testStart
A.$x \in (-\infty,-11) \cup (-2,3)$\\
B.$x \in (-\infty,-11) \cup (-2,3]$\\
C.$x \in (-\infty,-11) \cup [-2,3)$\\
D.$x \in (-\infty,-11] \cup (-2,3)$\\
E.$x \in (-\infty,-11] \cup (-2,3]$\\
F.$x \in (-\infty,-11] \cup [-2,3)$\\
G.$x \in (-\infty,-11) \cup [-2,3]$\\
H.$x \in (-\infty,-11] \cup [-2,3]$
\testStop
\kluczStart
A
\kluczStop



\zadStart{Zadanie z Wikieł Z 1.62 b) moja wersja nr 124}

Rozwiązać nierówności $(x+11)(4-x)(x+1)\ge0$.
\zadStop
\rozwStart{Patryk Wirkus}{}
Miejsca zerowe naszego wielomianu to: $-11, 4, -1$.\\
Wielomian jest stopnia nieparzystego, ponadto znak współczynnika przy\linebreak najwyższej potędze x jest ujemny.\\ W związku z tym wykres wielomianu zaczyna się od lewej strony powyżej osi OX. A więc $$x \in (-\infty,-11) \cup (-1,4).$$
\rozwStop
\odpStart
$x \in (-\infty,-11) \cup (-1,4)$
\odpStop
\testStart
A.$x \in (-\infty,-11) \cup (-1,4)$\\
B.$x \in (-\infty,-11) \cup (-1,4]$\\
C.$x \in (-\infty,-11) \cup [-1,4)$\\
D.$x \in (-\infty,-11] \cup (-1,4)$\\
E.$x \in (-\infty,-11] \cup (-1,4]$\\
F.$x \in (-\infty,-11] \cup [-1,4)$\\
G.$x \in (-\infty,-11) \cup [-1,4]$\\
H.$x \in (-\infty,-11] \cup [-1,4]$
\testStop
\kluczStart
A
\kluczStop



\zadStart{Zadanie z Wikieł Z 1.62 b) moja wersja nr 125}

Rozwiązać nierówności $(x+11)(4-x)(x+2)\ge0$.
\zadStop
\rozwStart{Patryk Wirkus}{}
Miejsca zerowe naszego wielomianu to: $-11, 4, -2$.\\
Wielomian jest stopnia nieparzystego, ponadto znak współczynnika przy\linebreak najwyższej potędze x jest ujemny.\\ W związku z tym wykres wielomianu zaczyna się od lewej strony powyżej osi OX. A więc $$x \in (-\infty,-11) \cup (-2,4).$$
\rozwStop
\odpStart
$x \in (-\infty,-11) \cup (-2,4)$
\odpStop
\testStart
A.$x \in (-\infty,-11) \cup (-2,4)$\\
B.$x \in (-\infty,-11) \cup (-2,4]$\\
C.$x \in (-\infty,-11) \cup [-2,4)$\\
D.$x \in (-\infty,-11] \cup (-2,4)$\\
E.$x \in (-\infty,-11] \cup (-2,4]$\\
F.$x \in (-\infty,-11] \cup [-2,4)$\\
G.$x \in (-\infty,-11) \cup [-2,4]$\\
H.$x \in (-\infty,-11] \cup [-2,4]$
\testStop
\kluczStart
A
\kluczStop



\zadStart{Zadanie z Wikieł Z 1.62 b) moja wersja nr 126}

Rozwiązać nierówności $(x+11)(4-x)(x+3)\ge0$.
\zadStop
\rozwStart{Patryk Wirkus}{}
Miejsca zerowe naszego wielomianu to: $-11, 4, -3$.\\
Wielomian jest stopnia nieparzystego, ponadto znak współczynnika przy\linebreak najwyższej potędze x jest ujemny.\\ W związku z tym wykres wielomianu zaczyna się od lewej strony powyżej osi OX. A więc $$x \in (-\infty,-11) \cup (-3,4).$$
\rozwStop
\odpStart
$x \in (-\infty,-11) \cup (-3,4)$
\odpStop
\testStart
A.$x \in (-\infty,-11) \cup (-3,4)$\\
B.$x \in (-\infty,-11) \cup (-3,4]$\\
C.$x \in (-\infty,-11) \cup [-3,4)$\\
D.$x \in (-\infty,-11] \cup (-3,4)$\\
E.$x \in (-\infty,-11] \cup (-3,4]$\\
F.$x \in (-\infty,-11] \cup [-3,4)$\\
G.$x \in (-\infty,-11) \cup [-3,4]$\\
H.$x \in (-\infty,-11] \cup [-3,4]$
\testStop
\kluczStart
A
\kluczStop



\zadStart{Zadanie z Wikieł Z 1.62 b) moja wersja nr 127}

Rozwiązać nierówności $(x+11)(5-x)(x+1)\ge0$.
\zadStop
\rozwStart{Patryk Wirkus}{}
Miejsca zerowe naszego wielomianu to: $-11, 5, -1$.\\
Wielomian jest stopnia nieparzystego, ponadto znak współczynnika przy\linebreak najwyższej potędze x jest ujemny.\\ W związku z tym wykres wielomianu zaczyna się od lewej strony powyżej osi OX. A więc $$x \in (-\infty,-11) \cup (-1,5).$$
\rozwStop
\odpStart
$x \in (-\infty,-11) \cup (-1,5)$
\odpStop
\testStart
A.$x \in (-\infty,-11) \cup (-1,5)$\\
B.$x \in (-\infty,-11) \cup (-1,5]$\\
C.$x \in (-\infty,-11) \cup [-1,5)$\\
D.$x \in (-\infty,-11] \cup (-1,5)$\\
E.$x \in (-\infty,-11] \cup (-1,5]$\\
F.$x \in (-\infty,-11] \cup [-1,5)$\\
G.$x \in (-\infty,-11) \cup [-1,5]$\\
H.$x \in (-\infty,-11] \cup [-1,5]$
\testStop
\kluczStart
A
\kluczStop



\zadStart{Zadanie z Wikieł Z 1.62 b) moja wersja nr 128}

Rozwiązać nierówności $(x+11)(5-x)(x+2)\ge0$.
\zadStop
\rozwStart{Patryk Wirkus}{}
Miejsca zerowe naszego wielomianu to: $-11, 5, -2$.\\
Wielomian jest stopnia nieparzystego, ponadto znak współczynnika przy\linebreak najwyższej potędze x jest ujemny.\\ W związku z tym wykres wielomianu zaczyna się od lewej strony powyżej osi OX. A więc $$x \in (-\infty,-11) \cup (-2,5).$$
\rozwStop
\odpStart
$x \in (-\infty,-11) \cup (-2,5)$
\odpStop
\testStart
A.$x \in (-\infty,-11) \cup (-2,5)$\\
B.$x \in (-\infty,-11) \cup (-2,5]$\\
C.$x \in (-\infty,-11) \cup [-2,5)$\\
D.$x \in (-\infty,-11] \cup (-2,5)$\\
E.$x \in (-\infty,-11] \cup (-2,5]$\\
F.$x \in (-\infty,-11] \cup [-2,5)$\\
G.$x \in (-\infty,-11) \cup [-2,5]$\\
H.$x \in (-\infty,-11] \cup [-2,5]$
\testStop
\kluczStart
A
\kluczStop



\zadStart{Zadanie z Wikieł Z 1.62 b) moja wersja nr 129}

Rozwiązać nierówności $(x+11)(5-x)(x+3)\ge0$.
\zadStop
\rozwStart{Patryk Wirkus}{}
Miejsca zerowe naszego wielomianu to: $-11, 5, -3$.\\
Wielomian jest stopnia nieparzystego, ponadto znak współczynnika przy\linebreak najwyższej potędze x jest ujemny.\\ W związku z tym wykres wielomianu zaczyna się od lewej strony powyżej osi OX. A więc $$x \in (-\infty,-11) \cup (-3,5).$$
\rozwStop
\odpStart
$x \in (-\infty,-11) \cup (-3,5)$
\odpStop
\testStart
A.$x \in (-\infty,-11) \cup (-3,5)$\\
B.$x \in (-\infty,-11) \cup (-3,5]$\\
C.$x \in (-\infty,-11) \cup [-3,5)$\\
D.$x \in (-\infty,-11] \cup (-3,5)$\\
E.$x \in (-\infty,-11] \cup (-3,5]$\\
F.$x \in (-\infty,-11] \cup [-3,5)$\\
G.$x \in (-\infty,-11) \cup [-3,5]$\\
H.$x \in (-\infty,-11] \cup [-3,5]$
\testStop
\kluczStart
A
\kluczStop



\zadStart{Zadanie z Wikieł Z 1.62 b) moja wersja nr 130}

Rozwiązać nierówności $(x+11)(5-x)(x+4)\ge0$.
\zadStop
\rozwStart{Patryk Wirkus}{}
Miejsca zerowe naszego wielomianu to: $-11, 5, -4$.\\
Wielomian jest stopnia nieparzystego, ponadto znak współczynnika przy\linebreak najwyższej potędze x jest ujemny.\\ W związku z tym wykres wielomianu zaczyna się od lewej strony powyżej osi OX. A więc $$x \in (-\infty,-11) \cup (-4,5).$$
\rozwStop
\odpStart
$x \in (-\infty,-11) \cup (-4,5)$
\odpStop
\testStart
A.$x \in (-\infty,-11) \cup (-4,5)$\\
B.$x \in (-\infty,-11) \cup (-4,5]$\\
C.$x \in (-\infty,-11) \cup [-4,5)$\\
D.$x \in (-\infty,-11] \cup (-4,5)$\\
E.$x \in (-\infty,-11] \cup (-4,5]$\\
F.$x \in (-\infty,-11] \cup [-4,5)$\\
G.$x \in (-\infty,-11) \cup [-4,5]$\\
H.$x \in (-\infty,-11] \cup [-4,5]$
\testStop
\kluczStart
A
\kluczStop



\zadStart{Zadanie z Wikieł Z 1.62 b) moja wersja nr 131}

Rozwiązać nierówności $(x+11)(6-x)(x+1)\ge0$.
\zadStop
\rozwStart{Patryk Wirkus}{}
Miejsca zerowe naszego wielomianu to: $-11, 6, -1$.\\
Wielomian jest stopnia nieparzystego, ponadto znak współczynnika przy\linebreak najwyższej potędze x jest ujemny.\\ W związku z tym wykres wielomianu zaczyna się od lewej strony powyżej osi OX. A więc $$x \in (-\infty,-11) \cup (-1,6).$$
\rozwStop
\odpStart
$x \in (-\infty,-11) \cup (-1,6)$
\odpStop
\testStart
A.$x \in (-\infty,-11) \cup (-1,6)$\\
B.$x \in (-\infty,-11) \cup (-1,6]$\\
C.$x \in (-\infty,-11) \cup [-1,6)$\\
D.$x \in (-\infty,-11] \cup (-1,6)$\\
E.$x \in (-\infty,-11] \cup (-1,6]$\\
F.$x \in (-\infty,-11] \cup [-1,6)$\\
G.$x \in (-\infty,-11) \cup [-1,6]$\\
H.$x \in (-\infty,-11] \cup [-1,6]$
\testStop
\kluczStart
A
\kluczStop



\zadStart{Zadanie z Wikieł Z 1.62 b) moja wersja nr 132}

Rozwiązać nierówności $(x+11)(6-x)(x+2)\ge0$.
\zadStop
\rozwStart{Patryk Wirkus}{}
Miejsca zerowe naszego wielomianu to: $-11, 6, -2$.\\
Wielomian jest stopnia nieparzystego, ponadto znak współczynnika przy\linebreak najwyższej potędze x jest ujemny.\\ W związku z tym wykres wielomianu zaczyna się od lewej strony powyżej osi OX. A więc $$x \in (-\infty,-11) \cup (-2,6).$$
\rozwStop
\odpStart
$x \in (-\infty,-11) \cup (-2,6)$
\odpStop
\testStart
A.$x \in (-\infty,-11) \cup (-2,6)$\\
B.$x \in (-\infty,-11) \cup (-2,6]$\\
C.$x \in (-\infty,-11) \cup [-2,6)$\\
D.$x \in (-\infty,-11] \cup (-2,6)$\\
E.$x \in (-\infty,-11] \cup (-2,6]$\\
F.$x \in (-\infty,-11] \cup [-2,6)$\\
G.$x \in (-\infty,-11) \cup [-2,6]$\\
H.$x \in (-\infty,-11] \cup [-2,6]$
\testStop
\kluczStart
A
\kluczStop



\zadStart{Zadanie z Wikieł Z 1.62 b) moja wersja nr 133}

Rozwiązać nierówności $(x+11)(6-x)(x+3)\ge0$.
\zadStop
\rozwStart{Patryk Wirkus}{}
Miejsca zerowe naszego wielomianu to: $-11, 6, -3$.\\
Wielomian jest stopnia nieparzystego, ponadto znak współczynnika przy\linebreak najwyższej potędze x jest ujemny.\\ W związku z tym wykres wielomianu zaczyna się od lewej strony powyżej osi OX. A więc $$x \in (-\infty,-11) \cup (-3,6).$$
\rozwStop
\odpStart
$x \in (-\infty,-11) \cup (-3,6)$
\odpStop
\testStart
A.$x \in (-\infty,-11) \cup (-3,6)$\\
B.$x \in (-\infty,-11) \cup (-3,6]$\\
C.$x \in (-\infty,-11) \cup [-3,6)$\\
D.$x \in (-\infty,-11] \cup (-3,6)$\\
E.$x \in (-\infty,-11] \cup (-3,6]$\\
F.$x \in (-\infty,-11] \cup [-3,6)$\\
G.$x \in (-\infty,-11) \cup [-3,6]$\\
H.$x \in (-\infty,-11] \cup [-3,6]$
\testStop
\kluczStart
A
\kluczStop



\zadStart{Zadanie z Wikieł Z 1.62 b) moja wersja nr 134}

Rozwiązać nierówności $(x+11)(6-x)(x+4)\ge0$.
\zadStop
\rozwStart{Patryk Wirkus}{}
Miejsca zerowe naszego wielomianu to: $-11, 6, -4$.\\
Wielomian jest stopnia nieparzystego, ponadto znak współczynnika przy\linebreak najwyższej potędze x jest ujemny.\\ W związku z tym wykres wielomianu zaczyna się od lewej strony powyżej osi OX. A więc $$x \in (-\infty,-11) \cup (-4,6).$$
\rozwStop
\odpStart
$x \in (-\infty,-11) \cup (-4,6)$
\odpStop
\testStart
A.$x \in (-\infty,-11) \cup (-4,6)$\\
B.$x \in (-\infty,-11) \cup (-4,6]$\\
C.$x \in (-\infty,-11) \cup [-4,6)$\\
D.$x \in (-\infty,-11] \cup (-4,6)$\\
E.$x \in (-\infty,-11] \cup (-4,6]$\\
F.$x \in (-\infty,-11] \cup [-4,6)$\\
G.$x \in (-\infty,-11) \cup [-4,6]$\\
H.$x \in (-\infty,-11] \cup [-4,6]$
\testStop
\kluczStart
A
\kluczStop



\zadStart{Zadanie z Wikieł Z 1.62 b) moja wersja nr 135}

Rozwiązać nierówności $(x+11)(6-x)(x+5)\ge0$.
\zadStop
\rozwStart{Patryk Wirkus}{}
Miejsca zerowe naszego wielomianu to: $-11, 6, -5$.\\
Wielomian jest stopnia nieparzystego, ponadto znak współczynnika przy\linebreak najwyższej potędze x jest ujemny.\\ W związku z tym wykres wielomianu zaczyna się od lewej strony powyżej osi OX. A więc $$x \in (-\infty,-11) \cup (-5,6).$$
\rozwStop
\odpStart
$x \in (-\infty,-11) \cup (-5,6)$
\odpStop
\testStart
A.$x \in (-\infty,-11) \cup (-5,6)$\\
B.$x \in (-\infty,-11) \cup (-5,6]$\\
C.$x \in (-\infty,-11) \cup [-5,6)$\\
D.$x \in (-\infty,-11] \cup (-5,6)$\\
E.$x \in (-\infty,-11] \cup (-5,6]$\\
F.$x \in (-\infty,-11] \cup [-5,6)$\\
G.$x \in (-\infty,-11) \cup [-5,6]$\\
H.$x \in (-\infty,-11] \cup [-5,6]$
\testStop
\kluczStart
A
\kluczStop



\zadStart{Zadanie z Wikieł Z 1.62 b) moja wersja nr 136}

Rozwiązać nierówności $(x+11)(7-x)(x+1)\ge0$.
\zadStop
\rozwStart{Patryk Wirkus}{}
Miejsca zerowe naszego wielomianu to: $-11, 7, -1$.\\
Wielomian jest stopnia nieparzystego, ponadto znak współczynnika przy\linebreak najwyższej potędze x jest ujemny.\\ W związku z tym wykres wielomianu zaczyna się od lewej strony powyżej osi OX. A więc $$x \in (-\infty,-11) \cup (-1,7).$$
\rozwStop
\odpStart
$x \in (-\infty,-11) \cup (-1,7)$
\odpStop
\testStart
A.$x \in (-\infty,-11) \cup (-1,7)$\\
B.$x \in (-\infty,-11) \cup (-1,7]$\\
C.$x \in (-\infty,-11) \cup [-1,7)$\\
D.$x \in (-\infty,-11] \cup (-1,7)$\\
E.$x \in (-\infty,-11] \cup (-1,7]$\\
F.$x \in (-\infty,-11] \cup [-1,7)$\\
G.$x \in (-\infty,-11) \cup [-1,7]$\\
H.$x \in (-\infty,-11] \cup [-1,7]$
\testStop
\kluczStart
A
\kluczStop



\zadStart{Zadanie z Wikieł Z 1.62 b) moja wersja nr 137}

Rozwiązać nierówności $(x+11)(7-x)(x+2)\ge0$.
\zadStop
\rozwStart{Patryk Wirkus}{}
Miejsca zerowe naszego wielomianu to: $-11, 7, -2$.\\
Wielomian jest stopnia nieparzystego, ponadto znak współczynnika przy\linebreak najwyższej potędze x jest ujemny.\\ W związku z tym wykres wielomianu zaczyna się od lewej strony powyżej osi OX. A więc $$x \in (-\infty,-11) \cup (-2,7).$$
\rozwStop
\odpStart
$x \in (-\infty,-11) \cup (-2,7)$
\odpStop
\testStart
A.$x \in (-\infty,-11) \cup (-2,7)$\\
B.$x \in (-\infty,-11) \cup (-2,7]$\\
C.$x \in (-\infty,-11) \cup [-2,7)$\\
D.$x \in (-\infty,-11] \cup (-2,7)$\\
E.$x \in (-\infty,-11] \cup (-2,7]$\\
F.$x \in (-\infty,-11] \cup [-2,7)$\\
G.$x \in (-\infty,-11) \cup [-2,7]$\\
H.$x \in (-\infty,-11] \cup [-2,7]$
\testStop
\kluczStart
A
\kluczStop



\zadStart{Zadanie z Wikieł Z 1.62 b) moja wersja nr 138}

Rozwiązać nierówności $(x+11)(7-x)(x+3)\ge0$.
\zadStop
\rozwStart{Patryk Wirkus}{}
Miejsca zerowe naszego wielomianu to: $-11, 7, -3$.\\
Wielomian jest stopnia nieparzystego, ponadto znak współczynnika przy\linebreak najwyższej potędze x jest ujemny.\\ W związku z tym wykres wielomianu zaczyna się od lewej strony powyżej osi OX. A więc $$x \in (-\infty,-11) \cup (-3,7).$$
\rozwStop
\odpStart
$x \in (-\infty,-11) \cup (-3,7)$
\odpStop
\testStart
A.$x \in (-\infty,-11) \cup (-3,7)$\\
B.$x \in (-\infty,-11) \cup (-3,7]$\\
C.$x \in (-\infty,-11) \cup [-3,7)$\\
D.$x \in (-\infty,-11] \cup (-3,7)$\\
E.$x \in (-\infty,-11] \cup (-3,7]$\\
F.$x \in (-\infty,-11] \cup [-3,7)$\\
G.$x \in (-\infty,-11) \cup [-3,7]$\\
H.$x \in (-\infty,-11] \cup [-3,7]$
\testStop
\kluczStart
A
\kluczStop



\zadStart{Zadanie z Wikieł Z 1.62 b) moja wersja nr 139}

Rozwiązać nierówności $(x+11)(7-x)(x+4)\ge0$.
\zadStop
\rozwStart{Patryk Wirkus}{}
Miejsca zerowe naszego wielomianu to: $-11, 7, -4$.\\
Wielomian jest stopnia nieparzystego, ponadto znak współczynnika przy\linebreak najwyższej potędze x jest ujemny.\\ W związku z tym wykres wielomianu zaczyna się od lewej strony powyżej osi OX. A więc $$x \in (-\infty,-11) \cup (-4,7).$$
\rozwStop
\odpStart
$x \in (-\infty,-11) \cup (-4,7)$
\odpStop
\testStart
A.$x \in (-\infty,-11) \cup (-4,7)$\\
B.$x \in (-\infty,-11) \cup (-4,7]$\\
C.$x \in (-\infty,-11) \cup [-4,7)$\\
D.$x \in (-\infty,-11] \cup (-4,7)$\\
E.$x \in (-\infty,-11] \cup (-4,7]$\\
F.$x \in (-\infty,-11] \cup [-4,7)$\\
G.$x \in (-\infty,-11) \cup [-4,7]$\\
H.$x \in (-\infty,-11] \cup [-4,7]$
\testStop
\kluczStart
A
\kluczStop



\zadStart{Zadanie z Wikieł Z 1.62 b) moja wersja nr 140}

Rozwiązać nierówności $(x+11)(7-x)(x+5)\ge0$.
\zadStop
\rozwStart{Patryk Wirkus}{}
Miejsca zerowe naszego wielomianu to: $-11, 7, -5$.\\
Wielomian jest stopnia nieparzystego, ponadto znak współczynnika przy\linebreak najwyższej potędze x jest ujemny.\\ W związku z tym wykres wielomianu zaczyna się od lewej strony powyżej osi OX. A więc $$x \in (-\infty,-11) \cup (-5,7).$$
\rozwStop
\odpStart
$x \in (-\infty,-11) \cup (-5,7)$
\odpStop
\testStart
A.$x \in (-\infty,-11) \cup (-5,7)$\\
B.$x \in (-\infty,-11) \cup (-5,7]$\\
C.$x \in (-\infty,-11) \cup [-5,7)$\\
D.$x \in (-\infty,-11] \cup (-5,7)$\\
E.$x \in (-\infty,-11] \cup (-5,7]$\\
F.$x \in (-\infty,-11] \cup [-5,7)$\\
G.$x \in (-\infty,-11) \cup [-5,7]$\\
H.$x \in (-\infty,-11] \cup [-5,7]$
\testStop
\kluczStart
A
\kluczStop



\zadStart{Zadanie z Wikieł Z 1.62 b) moja wersja nr 141}

Rozwiązać nierówności $(x+11)(7-x)(x+6)\ge0$.
\zadStop
\rozwStart{Patryk Wirkus}{}
Miejsca zerowe naszego wielomianu to: $-11, 7, -6$.\\
Wielomian jest stopnia nieparzystego, ponadto znak współczynnika przy\linebreak najwyższej potędze x jest ujemny.\\ W związku z tym wykres wielomianu zaczyna się od lewej strony powyżej osi OX. A więc $$x \in (-\infty,-11) \cup (-6,7).$$
\rozwStop
\odpStart
$x \in (-\infty,-11) \cup (-6,7)$
\odpStop
\testStart
A.$x \in (-\infty,-11) \cup (-6,7)$\\
B.$x \in (-\infty,-11) \cup (-6,7]$\\
C.$x \in (-\infty,-11) \cup [-6,7)$\\
D.$x \in (-\infty,-11] \cup (-6,7)$\\
E.$x \in (-\infty,-11] \cup (-6,7]$\\
F.$x \in (-\infty,-11] \cup [-6,7)$\\
G.$x \in (-\infty,-11) \cup [-6,7]$\\
H.$x \in (-\infty,-11] \cup [-6,7]$
\testStop
\kluczStart
A
\kluczStop



\zadStart{Zadanie z Wikieł Z 1.62 b) moja wersja nr 142}

Rozwiązać nierówności $(x+11)(8-x)(x+1)\ge0$.
\zadStop
\rozwStart{Patryk Wirkus}{}
Miejsca zerowe naszego wielomianu to: $-11, 8, -1$.\\
Wielomian jest stopnia nieparzystego, ponadto znak współczynnika przy\linebreak najwyższej potędze x jest ujemny.\\ W związku z tym wykres wielomianu zaczyna się od lewej strony powyżej osi OX. A więc $$x \in (-\infty,-11) \cup (-1,8).$$
\rozwStop
\odpStart
$x \in (-\infty,-11) \cup (-1,8)$
\odpStop
\testStart
A.$x \in (-\infty,-11) \cup (-1,8)$\\
B.$x \in (-\infty,-11) \cup (-1,8]$\\
C.$x \in (-\infty,-11) \cup [-1,8)$\\
D.$x \in (-\infty,-11] \cup (-1,8)$\\
E.$x \in (-\infty,-11] \cup (-1,8]$\\
F.$x \in (-\infty,-11] \cup [-1,8)$\\
G.$x \in (-\infty,-11) \cup [-1,8]$\\
H.$x \in (-\infty,-11] \cup [-1,8]$
\testStop
\kluczStart
A
\kluczStop



\zadStart{Zadanie z Wikieł Z 1.62 b) moja wersja nr 143}

Rozwiązać nierówności $(x+11)(8-x)(x+2)\ge0$.
\zadStop
\rozwStart{Patryk Wirkus}{}
Miejsca zerowe naszego wielomianu to: $-11, 8, -2$.\\
Wielomian jest stopnia nieparzystego, ponadto znak współczynnika przy\linebreak najwyższej potędze x jest ujemny.\\ W związku z tym wykres wielomianu zaczyna się od lewej strony powyżej osi OX. A więc $$x \in (-\infty,-11) \cup (-2,8).$$
\rozwStop
\odpStart
$x \in (-\infty,-11) \cup (-2,8)$
\odpStop
\testStart
A.$x \in (-\infty,-11) \cup (-2,8)$\\
B.$x \in (-\infty,-11) \cup (-2,8]$\\
C.$x \in (-\infty,-11) \cup [-2,8)$\\
D.$x \in (-\infty,-11] \cup (-2,8)$\\
E.$x \in (-\infty,-11] \cup (-2,8]$\\
F.$x \in (-\infty,-11] \cup [-2,8)$\\
G.$x \in (-\infty,-11) \cup [-2,8]$\\
H.$x \in (-\infty,-11] \cup [-2,8]$
\testStop
\kluczStart
A
\kluczStop



\zadStart{Zadanie z Wikieł Z 1.62 b) moja wersja nr 144}

Rozwiązać nierówności $(x+11)(8-x)(x+3)\ge0$.
\zadStop
\rozwStart{Patryk Wirkus}{}
Miejsca zerowe naszego wielomianu to: $-11, 8, -3$.\\
Wielomian jest stopnia nieparzystego, ponadto znak współczynnika przy\linebreak najwyższej potędze x jest ujemny.\\ W związku z tym wykres wielomianu zaczyna się od lewej strony powyżej osi OX. A więc $$x \in (-\infty,-11) \cup (-3,8).$$
\rozwStop
\odpStart
$x \in (-\infty,-11) \cup (-3,8)$
\odpStop
\testStart
A.$x \in (-\infty,-11) \cup (-3,8)$\\
B.$x \in (-\infty,-11) \cup (-3,8]$\\
C.$x \in (-\infty,-11) \cup [-3,8)$\\
D.$x \in (-\infty,-11] \cup (-3,8)$\\
E.$x \in (-\infty,-11] \cup (-3,8]$\\
F.$x \in (-\infty,-11] \cup [-3,8)$\\
G.$x \in (-\infty,-11) \cup [-3,8]$\\
H.$x \in (-\infty,-11] \cup [-3,8]$
\testStop
\kluczStart
A
\kluczStop



\zadStart{Zadanie z Wikieł Z 1.62 b) moja wersja nr 145}

Rozwiązać nierówności $(x+11)(8-x)(x+4)\ge0$.
\zadStop
\rozwStart{Patryk Wirkus}{}
Miejsca zerowe naszego wielomianu to: $-11, 8, -4$.\\
Wielomian jest stopnia nieparzystego, ponadto znak współczynnika przy\linebreak najwyższej potędze x jest ujemny.\\ W związku z tym wykres wielomianu zaczyna się od lewej strony powyżej osi OX. A więc $$x \in (-\infty,-11) \cup (-4,8).$$
\rozwStop
\odpStart
$x \in (-\infty,-11) \cup (-4,8)$
\odpStop
\testStart
A.$x \in (-\infty,-11) \cup (-4,8)$\\
B.$x \in (-\infty,-11) \cup (-4,8]$\\
C.$x \in (-\infty,-11) \cup [-4,8)$\\
D.$x \in (-\infty,-11] \cup (-4,8)$\\
E.$x \in (-\infty,-11] \cup (-4,8]$\\
F.$x \in (-\infty,-11] \cup [-4,8)$\\
G.$x \in (-\infty,-11) \cup [-4,8]$\\
H.$x \in (-\infty,-11] \cup [-4,8]$
\testStop
\kluczStart
A
\kluczStop



\zadStart{Zadanie z Wikieł Z 1.62 b) moja wersja nr 146}

Rozwiązać nierówności $(x+11)(8-x)(x+5)\ge0$.
\zadStop
\rozwStart{Patryk Wirkus}{}
Miejsca zerowe naszego wielomianu to: $-11, 8, -5$.\\
Wielomian jest stopnia nieparzystego, ponadto znak współczynnika przy\linebreak najwyższej potędze x jest ujemny.\\ W związku z tym wykres wielomianu zaczyna się od lewej strony powyżej osi OX. A więc $$x \in (-\infty,-11) \cup (-5,8).$$
\rozwStop
\odpStart
$x \in (-\infty,-11) \cup (-5,8)$
\odpStop
\testStart
A.$x \in (-\infty,-11) \cup (-5,8)$\\
B.$x \in (-\infty,-11) \cup (-5,8]$\\
C.$x \in (-\infty,-11) \cup [-5,8)$\\
D.$x \in (-\infty,-11] \cup (-5,8)$\\
E.$x \in (-\infty,-11] \cup (-5,8]$\\
F.$x \in (-\infty,-11] \cup [-5,8)$\\
G.$x \in (-\infty,-11) \cup [-5,8]$\\
H.$x \in (-\infty,-11] \cup [-5,8]$
\testStop
\kluczStart
A
\kluczStop



\zadStart{Zadanie z Wikieł Z 1.62 b) moja wersja nr 147}

Rozwiązać nierówności $(x+11)(8-x)(x+6)\ge0$.
\zadStop
\rozwStart{Patryk Wirkus}{}
Miejsca zerowe naszego wielomianu to: $-11, 8, -6$.\\
Wielomian jest stopnia nieparzystego, ponadto znak współczynnika przy\linebreak najwyższej potędze x jest ujemny.\\ W związku z tym wykres wielomianu zaczyna się od lewej strony powyżej osi OX. A więc $$x \in (-\infty,-11) \cup (-6,8).$$
\rozwStop
\odpStart
$x \in (-\infty,-11) \cup (-6,8)$
\odpStop
\testStart
A.$x \in (-\infty,-11) \cup (-6,8)$\\
B.$x \in (-\infty,-11) \cup (-6,8]$\\
C.$x \in (-\infty,-11) \cup [-6,8)$\\
D.$x \in (-\infty,-11] \cup (-6,8)$\\
E.$x \in (-\infty,-11] \cup (-6,8]$\\
F.$x \in (-\infty,-11] \cup [-6,8)$\\
G.$x \in (-\infty,-11) \cup [-6,8]$\\
H.$x \in (-\infty,-11] \cup [-6,8]$
\testStop
\kluczStart
A
\kluczStop



\zadStart{Zadanie z Wikieł Z 1.62 b) moja wersja nr 148}

Rozwiązać nierówności $(x+11)(8-x)(x+7)\ge0$.
\zadStop
\rozwStart{Patryk Wirkus}{}
Miejsca zerowe naszego wielomianu to: $-11, 8, -7$.\\
Wielomian jest stopnia nieparzystego, ponadto znak współczynnika przy\linebreak najwyższej potędze x jest ujemny.\\ W związku z tym wykres wielomianu zaczyna się od lewej strony powyżej osi OX. A więc $$x \in (-\infty,-11) \cup (-7,8).$$
\rozwStop
\odpStart
$x \in (-\infty,-11) \cup (-7,8)$
\odpStop
\testStart
A.$x \in (-\infty,-11) \cup (-7,8)$\\
B.$x \in (-\infty,-11) \cup (-7,8]$\\
C.$x \in (-\infty,-11) \cup [-7,8)$\\
D.$x \in (-\infty,-11] \cup (-7,8)$\\
E.$x \in (-\infty,-11] \cup (-7,8]$\\
F.$x \in (-\infty,-11] \cup [-7,8)$\\
G.$x \in (-\infty,-11) \cup [-7,8]$\\
H.$x \in (-\infty,-11] \cup [-7,8]$
\testStop
\kluczStart
A
\kluczStop



\zadStart{Zadanie z Wikieł Z 1.62 b) moja wersja nr 149}

Rozwiązać nierówności $(x+11)(9-x)(x+1)\ge0$.
\zadStop
\rozwStart{Patryk Wirkus}{}
Miejsca zerowe naszego wielomianu to: $-11, 9, -1$.\\
Wielomian jest stopnia nieparzystego, ponadto znak współczynnika przy\linebreak najwyższej potędze x jest ujemny.\\ W związku z tym wykres wielomianu zaczyna się od lewej strony powyżej osi OX. A więc $$x \in (-\infty,-11) \cup (-1,9).$$
\rozwStop
\odpStart
$x \in (-\infty,-11) \cup (-1,9)$
\odpStop
\testStart
A.$x \in (-\infty,-11) \cup (-1,9)$\\
B.$x \in (-\infty,-11) \cup (-1,9]$\\
C.$x \in (-\infty,-11) \cup [-1,9)$\\
D.$x \in (-\infty,-11] \cup (-1,9)$\\
E.$x \in (-\infty,-11] \cup (-1,9]$\\
F.$x \in (-\infty,-11] \cup [-1,9)$\\
G.$x \in (-\infty,-11) \cup [-1,9]$\\
H.$x \in (-\infty,-11] \cup [-1,9]$
\testStop
\kluczStart
A
\kluczStop



\zadStart{Zadanie z Wikieł Z 1.62 b) moja wersja nr 150}

Rozwiązać nierówności $(x+11)(9-x)(x+2)\ge0$.
\zadStop
\rozwStart{Patryk Wirkus}{}
Miejsca zerowe naszego wielomianu to: $-11, 9, -2$.\\
Wielomian jest stopnia nieparzystego, ponadto znak współczynnika przy\linebreak najwyższej potędze x jest ujemny.\\ W związku z tym wykres wielomianu zaczyna się od lewej strony powyżej osi OX. A więc $$x \in (-\infty,-11) \cup (-2,9).$$
\rozwStop
\odpStart
$x \in (-\infty,-11) \cup (-2,9)$
\odpStop
\testStart
A.$x \in (-\infty,-11) \cup (-2,9)$\\
B.$x \in (-\infty,-11) \cup (-2,9]$\\
C.$x \in (-\infty,-11) \cup [-2,9)$\\
D.$x \in (-\infty,-11] \cup (-2,9)$\\
E.$x \in (-\infty,-11] \cup (-2,9]$\\
F.$x \in (-\infty,-11] \cup [-2,9)$\\
G.$x \in (-\infty,-11) \cup [-2,9]$\\
H.$x \in (-\infty,-11] \cup [-2,9]$
\testStop
\kluczStart
A
\kluczStop



\zadStart{Zadanie z Wikieł Z 1.62 b) moja wersja nr 151}

Rozwiązać nierówności $(x+11)(9-x)(x+3)\ge0$.
\zadStop
\rozwStart{Patryk Wirkus}{}
Miejsca zerowe naszego wielomianu to: $-11, 9, -3$.\\
Wielomian jest stopnia nieparzystego, ponadto znak współczynnika przy\linebreak najwyższej potędze x jest ujemny.\\ W związku z tym wykres wielomianu zaczyna się od lewej strony powyżej osi OX. A więc $$x \in (-\infty,-11) \cup (-3,9).$$
\rozwStop
\odpStart
$x \in (-\infty,-11) \cup (-3,9)$
\odpStop
\testStart
A.$x \in (-\infty,-11) \cup (-3,9)$\\
B.$x \in (-\infty,-11) \cup (-3,9]$\\
C.$x \in (-\infty,-11) \cup [-3,9)$\\
D.$x \in (-\infty,-11] \cup (-3,9)$\\
E.$x \in (-\infty,-11] \cup (-3,9]$\\
F.$x \in (-\infty,-11] \cup [-3,9)$\\
G.$x \in (-\infty,-11) \cup [-3,9]$\\
H.$x \in (-\infty,-11] \cup [-3,9]$
\testStop
\kluczStart
A
\kluczStop



\zadStart{Zadanie z Wikieł Z 1.62 b) moja wersja nr 152}

Rozwiązać nierówności $(x+11)(9-x)(x+4)\ge0$.
\zadStop
\rozwStart{Patryk Wirkus}{}
Miejsca zerowe naszego wielomianu to: $-11, 9, -4$.\\
Wielomian jest stopnia nieparzystego, ponadto znak współczynnika przy\linebreak najwyższej potędze x jest ujemny.\\ W związku z tym wykres wielomianu zaczyna się od lewej strony powyżej osi OX. A więc $$x \in (-\infty,-11) \cup (-4,9).$$
\rozwStop
\odpStart
$x \in (-\infty,-11) \cup (-4,9)$
\odpStop
\testStart
A.$x \in (-\infty,-11) \cup (-4,9)$\\
B.$x \in (-\infty,-11) \cup (-4,9]$\\
C.$x \in (-\infty,-11) \cup [-4,9)$\\
D.$x \in (-\infty,-11] \cup (-4,9)$\\
E.$x \in (-\infty,-11] \cup (-4,9]$\\
F.$x \in (-\infty,-11] \cup [-4,9)$\\
G.$x \in (-\infty,-11) \cup [-4,9]$\\
H.$x \in (-\infty,-11] \cup [-4,9]$
\testStop
\kluczStart
A
\kluczStop



\zadStart{Zadanie z Wikieł Z 1.62 b) moja wersja nr 153}

Rozwiązać nierówności $(x+11)(9-x)(x+5)\ge0$.
\zadStop
\rozwStart{Patryk Wirkus}{}
Miejsca zerowe naszego wielomianu to: $-11, 9, -5$.\\
Wielomian jest stopnia nieparzystego, ponadto znak współczynnika przy\linebreak najwyższej potędze x jest ujemny.\\ W związku z tym wykres wielomianu zaczyna się od lewej strony powyżej osi OX. A więc $$x \in (-\infty,-11) \cup (-5,9).$$
\rozwStop
\odpStart
$x \in (-\infty,-11) \cup (-5,9)$
\odpStop
\testStart
A.$x \in (-\infty,-11) \cup (-5,9)$\\
B.$x \in (-\infty,-11) \cup (-5,9]$\\
C.$x \in (-\infty,-11) \cup [-5,9)$\\
D.$x \in (-\infty,-11] \cup (-5,9)$\\
E.$x \in (-\infty,-11] \cup (-5,9]$\\
F.$x \in (-\infty,-11] \cup [-5,9)$\\
G.$x \in (-\infty,-11) \cup [-5,9]$\\
H.$x \in (-\infty,-11] \cup [-5,9]$
\testStop
\kluczStart
A
\kluczStop



\zadStart{Zadanie z Wikieł Z 1.62 b) moja wersja nr 154}

Rozwiązać nierówności $(x+11)(9-x)(x+6)\ge0$.
\zadStop
\rozwStart{Patryk Wirkus}{}
Miejsca zerowe naszego wielomianu to: $-11, 9, -6$.\\
Wielomian jest stopnia nieparzystego, ponadto znak współczynnika przy\linebreak najwyższej potędze x jest ujemny.\\ W związku z tym wykres wielomianu zaczyna się od lewej strony powyżej osi OX. A więc $$x \in (-\infty,-11) \cup (-6,9).$$
\rozwStop
\odpStart
$x \in (-\infty,-11) \cup (-6,9)$
\odpStop
\testStart
A.$x \in (-\infty,-11) \cup (-6,9)$\\
B.$x \in (-\infty,-11) \cup (-6,9]$\\
C.$x \in (-\infty,-11) \cup [-6,9)$\\
D.$x \in (-\infty,-11] \cup (-6,9)$\\
E.$x \in (-\infty,-11] \cup (-6,9]$\\
F.$x \in (-\infty,-11] \cup [-6,9)$\\
G.$x \in (-\infty,-11) \cup [-6,9]$\\
H.$x \in (-\infty,-11] \cup [-6,9]$
\testStop
\kluczStart
A
\kluczStop



\zadStart{Zadanie z Wikieł Z 1.62 b) moja wersja nr 155}

Rozwiązać nierówności $(x+11)(9-x)(x+7)\ge0$.
\zadStop
\rozwStart{Patryk Wirkus}{}
Miejsca zerowe naszego wielomianu to: $-11, 9, -7$.\\
Wielomian jest stopnia nieparzystego, ponadto znak współczynnika przy\linebreak najwyższej potędze x jest ujemny.\\ W związku z tym wykres wielomianu zaczyna się od lewej strony powyżej osi OX. A więc $$x \in (-\infty,-11) \cup (-7,9).$$
\rozwStop
\odpStart
$x \in (-\infty,-11) \cup (-7,9)$
\odpStop
\testStart
A.$x \in (-\infty,-11) \cup (-7,9)$\\
B.$x \in (-\infty,-11) \cup (-7,9]$\\
C.$x \in (-\infty,-11) \cup [-7,9)$\\
D.$x \in (-\infty,-11] \cup (-7,9)$\\
E.$x \in (-\infty,-11] \cup (-7,9]$\\
F.$x \in (-\infty,-11] \cup [-7,9)$\\
G.$x \in (-\infty,-11) \cup [-7,9]$\\
H.$x \in (-\infty,-11] \cup [-7,9]$
\testStop
\kluczStart
A
\kluczStop



\zadStart{Zadanie z Wikieł Z 1.62 b) moja wersja nr 156}

Rozwiązać nierówności $(x+11)(9-x)(x+8)\ge0$.
\zadStop
\rozwStart{Patryk Wirkus}{}
Miejsca zerowe naszego wielomianu to: $-11, 9, -8$.\\
Wielomian jest stopnia nieparzystego, ponadto znak współczynnika przy\linebreak najwyższej potędze x jest ujemny.\\ W związku z tym wykres wielomianu zaczyna się od lewej strony powyżej osi OX. A więc $$x \in (-\infty,-11) \cup (-8,9).$$
\rozwStop
\odpStart
$x \in (-\infty,-11) \cup (-8,9)$
\odpStop
\testStart
A.$x \in (-\infty,-11) \cup (-8,9)$\\
B.$x \in (-\infty,-11) \cup (-8,9]$\\
C.$x \in (-\infty,-11) \cup [-8,9)$\\
D.$x \in (-\infty,-11] \cup (-8,9)$\\
E.$x \in (-\infty,-11] \cup (-8,9]$\\
F.$x \in (-\infty,-11] \cup [-8,9)$\\
G.$x \in (-\infty,-11) \cup [-8,9]$\\
H.$x \in (-\infty,-11] \cup [-8,9]$
\testStop
\kluczStart
A
\kluczStop



\zadStart{Zadanie z Wikieł Z 1.62 b) moja wersja nr 157}

Rozwiązać nierówności $(x+11)(10-x)(x+1)\ge0$.
\zadStop
\rozwStart{Patryk Wirkus}{}
Miejsca zerowe naszego wielomianu to: $-11, 10, -1$.\\
Wielomian jest stopnia nieparzystego, ponadto znak współczynnika przy\linebreak najwyższej potędze x jest ujemny.\\ W związku z tym wykres wielomianu zaczyna się od lewej strony powyżej osi OX. A więc $$x \in (-\infty,-11) \cup (-1,10).$$
\rozwStop
\odpStart
$x \in (-\infty,-11) \cup (-1,10)$
\odpStop
\testStart
A.$x \in (-\infty,-11) \cup (-1,10)$\\
B.$x \in (-\infty,-11) \cup (-1,10]$\\
C.$x \in (-\infty,-11) \cup [-1,10)$\\
D.$x \in (-\infty,-11] \cup (-1,10)$\\
E.$x \in (-\infty,-11] \cup (-1,10]$\\
F.$x \in (-\infty,-11] \cup [-1,10)$\\
G.$x \in (-\infty,-11) \cup [-1,10]$\\
H.$x \in (-\infty,-11] \cup [-1,10]$
\testStop
\kluczStart
A
\kluczStop



\zadStart{Zadanie z Wikieł Z 1.62 b) moja wersja nr 158}

Rozwiązać nierówności $(x+11)(10-x)(x+2)\ge0$.
\zadStop
\rozwStart{Patryk Wirkus}{}
Miejsca zerowe naszego wielomianu to: $-11, 10, -2$.\\
Wielomian jest stopnia nieparzystego, ponadto znak współczynnika przy\linebreak najwyższej potędze x jest ujemny.\\ W związku z tym wykres wielomianu zaczyna się od lewej strony powyżej osi OX. A więc $$x \in (-\infty,-11) \cup (-2,10).$$
\rozwStop
\odpStart
$x \in (-\infty,-11) \cup (-2,10)$
\odpStop
\testStart
A.$x \in (-\infty,-11) \cup (-2,10)$\\
B.$x \in (-\infty,-11) \cup (-2,10]$\\
C.$x \in (-\infty,-11) \cup [-2,10)$\\
D.$x \in (-\infty,-11] \cup (-2,10)$\\
E.$x \in (-\infty,-11] \cup (-2,10]$\\
F.$x \in (-\infty,-11] \cup [-2,10)$\\
G.$x \in (-\infty,-11) \cup [-2,10]$\\
H.$x \in (-\infty,-11] \cup [-2,10]$
\testStop
\kluczStart
A
\kluczStop



\zadStart{Zadanie z Wikieł Z 1.62 b) moja wersja nr 159}

Rozwiązać nierówności $(x+11)(10-x)(x+3)\ge0$.
\zadStop
\rozwStart{Patryk Wirkus}{}
Miejsca zerowe naszego wielomianu to: $-11, 10, -3$.\\
Wielomian jest stopnia nieparzystego, ponadto znak współczynnika przy\linebreak najwyższej potędze x jest ujemny.\\ W związku z tym wykres wielomianu zaczyna się od lewej strony powyżej osi OX. A więc $$x \in (-\infty,-11) \cup (-3,10).$$
\rozwStop
\odpStart
$x \in (-\infty,-11) \cup (-3,10)$
\odpStop
\testStart
A.$x \in (-\infty,-11) \cup (-3,10)$\\
B.$x \in (-\infty,-11) \cup (-3,10]$\\
C.$x \in (-\infty,-11) \cup [-3,10)$\\
D.$x \in (-\infty,-11] \cup (-3,10)$\\
E.$x \in (-\infty,-11] \cup (-3,10]$\\
F.$x \in (-\infty,-11] \cup [-3,10)$\\
G.$x \in (-\infty,-11) \cup [-3,10]$\\
H.$x \in (-\infty,-11] \cup [-3,10]$
\testStop
\kluczStart
A
\kluczStop



\zadStart{Zadanie z Wikieł Z 1.62 b) moja wersja nr 160}

Rozwiązać nierówności $(x+11)(10-x)(x+4)\ge0$.
\zadStop
\rozwStart{Patryk Wirkus}{}
Miejsca zerowe naszego wielomianu to: $-11, 10, -4$.\\
Wielomian jest stopnia nieparzystego, ponadto znak współczynnika przy\linebreak najwyższej potędze x jest ujemny.\\ W związku z tym wykres wielomianu zaczyna się od lewej strony powyżej osi OX. A więc $$x \in (-\infty,-11) \cup (-4,10).$$
\rozwStop
\odpStart
$x \in (-\infty,-11) \cup (-4,10)$
\odpStop
\testStart
A.$x \in (-\infty,-11) \cup (-4,10)$\\
B.$x \in (-\infty,-11) \cup (-4,10]$\\
C.$x \in (-\infty,-11) \cup [-4,10)$\\
D.$x \in (-\infty,-11] \cup (-4,10)$\\
E.$x \in (-\infty,-11] \cup (-4,10]$\\
F.$x \in (-\infty,-11] \cup [-4,10)$\\
G.$x \in (-\infty,-11) \cup [-4,10]$\\
H.$x \in (-\infty,-11] \cup [-4,10]$
\testStop
\kluczStart
A
\kluczStop



\zadStart{Zadanie z Wikieł Z 1.62 b) moja wersja nr 161}

Rozwiązać nierówności $(x+11)(10-x)(x+5)\ge0$.
\zadStop
\rozwStart{Patryk Wirkus}{}
Miejsca zerowe naszego wielomianu to: $-11, 10, -5$.\\
Wielomian jest stopnia nieparzystego, ponadto znak współczynnika przy\linebreak najwyższej potędze x jest ujemny.\\ W związku z tym wykres wielomianu zaczyna się od lewej strony powyżej osi OX. A więc $$x \in (-\infty,-11) \cup (-5,10).$$
\rozwStop
\odpStart
$x \in (-\infty,-11) \cup (-5,10)$
\odpStop
\testStart
A.$x \in (-\infty,-11) \cup (-5,10)$\\
B.$x \in (-\infty,-11) \cup (-5,10]$\\
C.$x \in (-\infty,-11) \cup [-5,10)$\\
D.$x \in (-\infty,-11] \cup (-5,10)$\\
E.$x \in (-\infty,-11] \cup (-5,10]$\\
F.$x \in (-\infty,-11] \cup [-5,10)$\\
G.$x \in (-\infty,-11) \cup [-5,10]$\\
H.$x \in (-\infty,-11] \cup [-5,10]$
\testStop
\kluczStart
A
\kluczStop



\zadStart{Zadanie z Wikieł Z 1.62 b) moja wersja nr 162}

Rozwiązać nierówności $(x+11)(10-x)(x+6)\ge0$.
\zadStop
\rozwStart{Patryk Wirkus}{}
Miejsca zerowe naszego wielomianu to: $-11, 10, -6$.\\
Wielomian jest stopnia nieparzystego, ponadto znak współczynnika przy\linebreak najwyższej potędze x jest ujemny.\\ W związku z tym wykres wielomianu zaczyna się od lewej strony powyżej osi OX. A więc $$x \in (-\infty,-11) \cup (-6,10).$$
\rozwStop
\odpStart
$x \in (-\infty,-11) \cup (-6,10)$
\odpStop
\testStart
A.$x \in (-\infty,-11) \cup (-6,10)$\\
B.$x \in (-\infty,-11) \cup (-6,10]$\\
C.$x \in (-\infty,-11) \cup [-6,10)$\\
D.$x \in (-\infty,-11] \cup (-6,10)$\\
E.$x \in (-\infty,-11] \cup (-6,10]$\\
F.$x \in (-\infty,-11] \cup [-6,10)$\\
G.$x \in (-\infty,-11) \cup [-6,10]$\\
H.$x \in (-\infty,-11] \cup [-6,10]$
\testStop
\kluczStart
A
\kluczStop



\zadStart{Zadanie z Wikieł Z 1.62 b) moja wersja nr 163}

Rozwiązać nierówności $(x+11)(10-x)(x+7)\ge0$.
\zadStop
\rozwStart{Patryk Wirkus}{}
Miejsca zerowe naszego wielomianu to: $-11, 10, -7$.\\
Wielomian jest stopnia nieparzystego, ponadto znak współczynnika przy\linebreak najwyższej potędze x jest ujemny.\\ W związku z tym wykres wielomianu zaczyna się od lewej strony powyżej osi OX. A więc $$x \in (-\infty,-11) \cup (-7,10).$$
\rozwStop
\odpStart
$x \in (-\infty,-11) \cup (-7,10)$
\odpStop
\testStart
A.$x \in (-\infty,-11) \cup (-7,10)$\\
B.$x \in (-\infty,-11) \cup (-7,10]$\\
C.$x \in (-\infty,-11) \cup [-7,10)$\\
D.$x \in (-\infty,-11] \cup (-7,10)$\\
E.$x \in (-\infty,-11] \cup (-7,10]$\\
F.$x \in (-\infty,-11] \cup [-7,10)$\\
G.$x \in (-\infty,-11) \cup [-7,10]$\\
H.$x \in (-\infty,-11] \cup [-7,10]$
\testStop
\kluczStart
A
\kluczStop



\zadStart{Zadanie z Wikieł Z 1.62 b) moja wersja nr 164}

Rozwiązać nierówności $(x+11)(10-x)(x+8)\ge0$.
\zadStop
\rozwStart{Patryk Wirkus}{}
Miejsca zerowe naszego wielomianu to: $-11, 10, -8$.\\
Wielomian jest stopnia nieparzystego, ponadto znak współczynnika przy\linebreak najwyższej potędze x jest ujemny.\\ W związku z tym wykres wielomianu zaczyna się od lewej strony powyżej osi OX. A więc $$x \in (-\infty,-11) \cup (-8,10).$$
\rozwStop
\odpStart
$x \in (-\infty,-11) \cup (-8,10)$
\odpStop
\testStart
A.$x \in (-\infty,-11) \cup (-8,10)$\\
B.$x \in (-\infty,-11) \cup (-8,10]$\\
C.$x \in (-\infty,-11) \cup [-8,10)$\\
D.$x \in (-\infty,-11] \cup (-8,10)$\\
E.$x \in (-\infty,-11] \cup (-8,10]$\\
F.$x \in (-\infty,-11] \cup [-8,10)$\\
G.$x \in (-\infty,-11) \cup [-8,10]$\\
H.$x \in (-\infty,-11] \cup [-8,10]$
\testStop
\kluczStart
A
\kluczStop



\zadStart{Zadanie z Wikieł Z 1.62 b) moja wersja nr 165}

Rozwiązać nierówności $(x+11)(10-x)(x+9)\ge0$.
\zadStop
\rozwStart{Patryk Wirkus}{}
Miejsca zerowe naszego wielomianu to: $-11, 10, -9$.\\
Wielomian jest stopnia nieparzystego, ponadto znak współczynnika przy\linebreak najwyższej potędze x jest ujemny.\\ W związku z tym wykres wielomianu zaczyna się od lewej strony powyżej osi OX. A więc $$x \in (-\infty,-11) \cup (-9,10).$$
\rozwStop
\odpStart
$x \in (-\infty,-11) \cup (-9,10)$
\odpStop
\testStart
A.$x \in (-\infty,-11) \cup (-9,10)$\\
B.$x \in (-\infty,-11) \cup (-9,10]$\\
C.$x \in (-\infty,-11) \cup [-9,10)$\\
D.$x \in (-\infty,-11] \cup (-9,10)$\\
E.$x \in (-\infty,-11] \cup (-9,10]$\\
F.$x \in (-\infty,-11] \cup [-9,10)$\\
G.$x \in (-\infty,-11) \cup [-9,10]$\\
H.$x \in (-\infty,-11] \cup [-9,10]$
\testStop
\kluczStart
A
\kluczStop



\zadStart{Zadanie z Wikieł Z 1.62 b) moja wersja nr 166}

Rozwiązać nierówności $(x+12)(2-x)(x+1)\ge0$.
\zadStop
\rozwStart{Patryk Wirkus}{}
Miejsca zerowe naszego wielomianu to: $-12, 2, -1$.\\
Wielomian jest stopnia nieparzystego, ponadto znak współczynnika przy\linebreak najwyższej potędze x jest ujemny.\\ W związku z tym wykres wielomianu zaczyna się od lewej strony powyżej osi OX. A więc $$x \in (-\infty,-12) \cup (-1,2).$$
\rozwStop
\odpStart
$x \in (-\infty,-12) \cup (-1,2)$
\odpStop
\testStart
A.$x \in (-\infty,-12) \cup (-1,2)$\\
B.$x \in (-\infty,-12) \cup (-1,2]$\\
C.$x \in (-\infty,-12) \cup [-1,2)$\\
D.$x \in (-\infty,-12] \cup (-1,2)$\\
E.$x \in (-\infty,-12] \cup (-1,2]$\\
F.$x \in (-\infty,-12] \cup [-1,2)$\\
G.$x \in (-\infty,-12) \cup [-1,2]$\\
H.$x \in (-\infty,-12] \cup [-1,2]$
\testStop
\kluczStart
A
\kluczStop



\zadStart{Zadanie z Wikieł Z 1.62 b) moja wersja nr 167}

Rozwiązać nierówności $(x+12)(3-x)(x+1)\ge0$.
\zadStop
\rozwStart{Patryk Wirkus}{}
Miejsca zerowe naszego wielomianu to: $-12, 3, -1$.\\
Wielomian jest stopnia nieparzystego, ponadto znak współczynnika przy\linebreak najwyższej potędze x jest ujemny.\\ W związku z tym wykres wielomianu zaczyna się od lewej strony powyżej osi OX. A więc $$x \in (-\infty,-12) \cup (-1,3).$$
\rozwStop
\odpStart
$x \in (-\infty,-12) \cup (-1,3)$
\odpStop
\testStart
A.$x \in (-\infty,-12) \cup (-1,3)$\\
B.$x \in (-\infty,-12) \cup (-1,3]$\\
C.$x \in (-\infty,-12) \cup [-1,3)$\\
D.$x \in (-\infty,-12] \cup (-1,3)$\\
E.$x \in (-\infty,-12] \cup (-1,3]$\\
F.$x \in (-\infty,-12] \cup [-1,3)$\\
G.$x \in (-\infty,-12) \cup [-1,3]$\\
H.$x \in (-\infty,-12] \cup [-1,3]$
\testStop
\kluczStart
A
\kluczStop



\zadStart{Zadanie z Wikieł Z 1.62 b) moja wersja nr 168}

Rozwiązać nierówności $(x+12)(3-x)(x+2)\ge0$.
\zadStop
\rozwStart{Patryk Wirkus}{}
Miejsca zerowe naszego wielomianu to: $-12, 3, -2$.\\
Wielomian jest stopnia nieparzystego, ponadto znak współczynnika przy\linebreak najwyższej potędze x jest ujemny.\\ W związku z tym wykres wielomianu zaczyna się od lewej strony powyżej osi OX. A więc $$x \in (-\infty,-12) \cup (-2,3).$$
\rozwStop
\odpStart
$x \in (-\infty,-12) \cup (-2,3)$
\odpStop
\testStart
A.$x \in (-\infty,-12) \cup (-2,3)$\\
B.$x \in (-\infty,-12) \cup (-2,3]$\\
C.$x \in (-\infty,-12) \cup [-2,3)$\\
D.$x \in (-\infty,-12] \cup (-2,3)$\\
E.$x \in (-\infty,-12] \cup (-2,3]$\\
F.$x \in (-\infty,-12] \cup [-2,3)$\\
G.$x \in (-\infty,-12) \cup [-2,3]$\\
H.$x \in (-\infty,-12] \cup [-2,3]$
\testStop
\kluczStart
A
\kluczStop



\zadStart{Zadanie z Wikieł Z 1.62 b) moja wersja nr 169}

Rozwiązać nierówności $(x+12)(4-x)(x+1)\ge0$.
\zadStop
\rozwStart{Patryk Wirkus}{}
Miejsca zerowe naszego wielomianu to: $-12, 4, -1$.\\
Wielomian jest stopnia nieparzystego, ponadto znak współczynnika przy\linebreak najwyższej potędze x jest ujemny.\\ W związku z tym wykres wielomianu zaczyna się od lewej strony powyżej osi OX. A więc $$x \in (-\infty,-12) \cup (-1,4).$$
\rozwStop
\odpStart
$x \in (-\infty,-12) \cup (-1,4)$
\odpStop
\testStart
A.$x \in (-\infty,-12) \cup (-1,4)$\\
B.$x \in (-\infty,-12) \cup (-1,4]$\\
C.$x \in (-\infty,-12) \cup [-1,4)$\\
D.$x \in (-\infty,-12] \cup (-1,4)$\\
E.$x \in (-\infty,-12] \cup (-1,4]$\\
F.$x \in (-\infty,-12] \cup [-1,4)$\\
G.$x \in (-\infty,-12) \cup [-1,4]$\\
H.$x \in (-\infty,-12] \cup [-1,4]$
\testStop
\kluczStart
A
\kluczStop



\zadStart{Zadanie z Wikieł Z 1.62 b) moja wersja nr 170}

Rozwiązać nierówności $(x+12)(4-x)(x+2)\ge0$.
\zadStop
\rozwStart{Patryk Wirkus}{}
Miejsca zerowe naszego wielomianu to: $-12, 4, -2$.\\
Wielomian jest stopnia nieparzystego, ponadto znak współczynnika przy\linebreak najwyższej potędze x jest ujemny.\\ W związku z tym wykres wielomianu zaczyna się od lewej strony powyżej osi OX. A więc $$x \in (-\infty,-12) \cup (-2,4).$$
\rozwStop
\odpStart
$x \in (-\infty,-12) \cup (-2,4)$
\odpStop
\testStart
A.$x \in (-\infty,-12) \cup (-2,4)$\\
B.$x \in (-\infty,-12) \cup (-2,4]$\\
C.$x \in (-\infty,-12) \cup [-2,4)$\\
D.$x \in (-\infty,-12] \cup (-2,4)$\\
E.$x \in (-\infty,-12] \cup (-2,4]$\\
F.$x \in (-\infty,-12] \cup [-2,4)$\\
G.$x \in (-\infty,-12) \cup [-2,4]$\\
H.$x \in (-\infty,-12] \cup [-2,4]$
\testStop
\kluczStart
A
\kluczStop



\zadStart{Zadanie z Wikieł Z 1.62 b) moja wersja nr 171}

Rozwiązać nierówności $(x+12)(4-x)(x+3)\ge0$.
\zadStop
\rozwStart{Patryk Wirkus}{}
Miejsca zerowe naszego wielomianu to: $-12, 4, -3$.\\
Wielomian jest stopnia nieparzystego, ponadto znak współczynnika przy\linebreak najwyższej potędze x jest ujemny.\\ W związku z tym wykres wielomianu zaczyna się od lewej strony powyżej osi OX. A więc $$x \in (-\infty,-12) \cup (-3,4).$$
\rozwStop
\odpStart
$x \in (-\infty,-12) \cup (-3,4)$
\odpStop
\testStart
A.$x \in (-\infty,-12) \cup (-3,4)$\\
B.$x \in (-\infty,-12) \cup (-3,4]$\\
C.$x \in (-\infty,-12) \cup [-3,4)$\\
D.$x \in (-\infty,-12] \cup (-3,4)$\\
E.$x \in (-\infty,-12] \cup (-3,4]$\\
F.$x \in (-\infty,-12] \cup [-3,4)$\\
G.$x \in (-\infty,-12) \cup [-3,4]$\\
H.$x \in (-\infty,-12] \cup [-3,4]$
\testStop
\kluczStart
A
\kluczStop



\zadStart{Zadanie z Wikieł Z 1.62 b) moja wersja nr 172}

Rozwiązać nierówności $(x+12)(5-x)(x+1)\ge0$.
\zadStop
\rozwStart{Patryk Wirkus}{}
Miejsca zerowe naszego wielomianu to: $-12, 5, -1$.\\
Wielomian jest stopnia nieparzystego, ponadto znak współczynnika przy\linebreak najwyższej potędze x jest ujemny.\\ W związku z tym wykres wielomianu zaczyna się od lewej strony powyżej osi OX. A więc $$x \in (-\infty,-12) \cup (-1,5).$$
\rozwStop
\odpStart
$x \in (-\infty,-12) \cup (-1,5)$
\odpStop
\testStart
A.$x \in (-\infty,-12) \cup (-1,5)$\\
B.$x \in (-\infty,-12) \cup (-1,5]$\\
C.$x \in (-\infty,-12) \cup [-1,5)$\\
D.$x \in (-\infty,-12] \cup (-1,5)$\\
E.$x \in (-\infty,-12] \cup (-1,5]$\\
F.$x \in (-\infty,-12] \cup [-1,5)$\\
G.$x \in (-\infty,-12) \cup [-1,5]$\\
H.$x \in (-\infty,-12] \cup [-1,5]$
\testStop
\kluczStart
A
\kluczStop



\zadStart{Zadanie z Wikieł Z 1.62 b) moja wersja nr 173}

Rozwiązać nierówności $(x+12)(5-x)(x+2)\ge0$.
\zadStop
\rozwStart{Patryk Wirkus}{}
Miejsca zerowe naszego wielomianu to: $-12, 5, -2$.\\
Wielomian jest stopnia nieparzystego, ponadto znak współczynnika przy\linebreak najwyższej potędze x jest ujemny.\\ W związku z tym wykres wielomianu zaczyna się od lewej strony powyżej osi OX. A więc $$x \in (-\infty,-12) \cup (-2,5).$$
\rozwStop
\odpStart
$x \in (-\infty,-12) \cup (-2,5)$
\odpStop
\testStart
A.$x \in (-\infty,-12) \cup (-2,5)$\\
B.$x \in (-\infty,-12) \cup (-2,5]$\\
C.$x \in (-\infty,-12) \cup [-2,5)$\\
D.$x \in (-\infty,-12] \cup (-2,5)$\\
E.$x \in (-\infty,-12] \cup (-2,5]$\\
F.$x \in (-\infty,-12] \cup [-2,5)$\\
G.$x \in (-\infty,-12) \cup [-2,5]$\\
H.$x \in (-\infty,-12] \cup [-2,5]$
\testStop
\kluczStart
A
\kluczStop



\zadStart{Zadanie z Wikieł Z 1.62 b) moja wersja nr 174}

Rozwiązać nierówności $(x+12)(5-x)(x+3)\ge0$.
\zadStop
\rozwStart{Patryk Wirkus}{}
Miejsca zerowe naszego wielomianu to: $-12, 5, -3$.\\
Wielomian jest stopnia nieparzystego, ponadto znak współczynnika przy\linebreak najwyższej potędze x jest ujemny.\\ W związku z tym wykres wielomianu zaczyna się od lewej strony powyżej osi OX. A więc $$x \in (-\infty,-12) \cup (-3,5).$$
\rozwStop
\odpStart
$x \in (-\infty,-12) \cup (-3,5)$
\odpStop
\testStart
A.$x \in (-\infty,-12) \cup (-3,5)$\\
B.$x \in (-\infty,-12) \cup (-3,5]$\\
C.$x \in (-\infty,-12) \cup [-3,5)$\\
D.$x \in (-\infty,-12] \cup (-3,5)$\\
E.$x \in (-\infty,-12] \cup (-3,5]$\\
F.$x \in (-\infty,-12] \cup [-3,5)$\\
G.$x \in (-\infty,-12) \cup [-3,5]$\\
H.$x \in (-\infty,-12] \cup [-3,5]$
\testStop
\kluczStart
A
\kluczStop



\zadStart{Zadanie z Wikieł Z 1.62 b) moja wersja nr 175}

Rozwiązać nierówności $(x+12)(5-x)(x+4)\ge0$.
\zadStop
\rozwStart{Patryk Wirkus}{}
Miejsca zerowe naszego wielomianu to: $-12, 5, -4$.\\
Wielomian jest stopnia nieparzystego, ponadto znak współczynnika przy\linebreak najwyższej potędze x jest ujemny.\\ W związku z tym wykres wielomianu zaczyna się od lewej strony powyżej osi OX. A więc $$x \in (-\infty,-12) \cup (-4,5).$$
\rozwStop
\odpStart
$x \in (-\infty,-12) \cup (-4,5)$
\odpStop
\testStart
A.$x \in (-\infty,-12) \cup (-4,5)$\\
B.$x \in (-\infty,-12) \cup (-4,5]$\\
C.$x \in (-\infty,-12) \cup [-4,5)$\\
D.$x \in (-\infty,-12] \cup (-4,5)$\\
E.$x \in (-\infty,-12] \cup (-4,5]$\\
F.$x \in (-\infty,-12] \cup [-4,5)$\\
G.$x \in (-\infty,-12) \cup [-4,5]$\\
H.$x \in (-\infty,-12] \cup [-4,5]$
\testStop
\kluczStart
A
\kluczStop



\zadStart{Zadanie z Wikieł Z 1.62 b) moja wersja nr 176}

Rozwiązać nierówności $(x+12)(6-x)(x+1)\ge0$.
\zadStop
\rozwStart{Patryk Wirkus}{}
Miejsca zerowe naszego wielomianu to: $-12, 6, -1$.\\
Wielomian jest stopnia nieparzystego, ponadto znak współczynnika przy\linebreak najwyższej potędze x jest ujemny.\\ W związku z tym wykres wielomianu zaczyna się od lewej strony powyżej osi OX. A więc $$x \in (-\infty,-12) \cup (-1,6).$$
\rozwStop
\odpStart
$x \in (-\infty,-12) \cup (-1,6)$
\odpStop
\testStart
A.$x \in (-\infty,-12) \cup (-1,6)$\\
B.$x \in (-\infty,-12) \cup (-1,6]$\\
C.$x \in (-\infty,-12) \cup [-1,6)$\\
D.$x \in (-\infty,-12] \cup (-1,6)$\\
E.$x \in (-\infty,-12] \cup (-1,6]$\\
F.$x \in (-\infty,-12] \cup [-1,6)$\\
G.$x \in (-\infty,-12) \cup [-1,6]$\\
H.$x \in (-\infty,-12] \cup [-1,6]$
\testStop
\kluczStart
A
\kluczStop



\zadStart{Zadanie z Wikieł Z 1.62 b) moja wersja nr 177}

Rozwiązać nierówności $(x+12)(6-x)(x+2)\ge0$.
\zadStop
\rozwStart{Patryk Wirkus}{}
Miejsca zerowe naszego wielomianu to: $-12, 6, -2$.\\
Wielomian jest stopnia nieparzystego, ponadto znak współczynnika przy\linebreak najwyższej potędze x jest ujemny.\\ W związku z tym wykres wielomianu zaczyna się od lewej strony powyżej osi OX. A więc $$x \in (-\infty,-12) \cup (-2,6).$$
\rozwStop
\odpStart
$x \in (-\infty,-12) \cup (-2,6)$
\odpStop
\testStart
A.$x \in (-\infty,-12) \cup (-2,6)$\\
B.$x \in (-\infty,-12) \cup (-2,6]$\\
C.$x \in (-\infty,-12) \cup [-2,6)$\\
D.$x \in (-\infty,-12] \cup (-2,6)$\\
E.$x \in (-\infty,-12] \cup (-2,6]$\\
F.$x \in (-\infty,-12] \cup [-2,6)$\\
G.$x \in (-\infty,-12) \cup [-2,6]$\\
H.$x \in (-\infty,-12] \cup [-2,6]$
\testStop
\kluczStart
A
\kluczStop



\zadStart{Zadanie z Wikieł Z 1.62 b) moja wersja nr 178}

Rozwiązać nierówności $(x+12)(6-x)(x+3)\ge0$.
\zadStop
\rozwStart{Patryk Wirkus}{}
Miejsca zerowe naszego wielomianu to: $-12, 6, -3$.\\
Wielomian jest stopnia nieparzystego, ponadto znak współczynnika przy\linebreak najwyższej potędze x jest ujemny.\\ W związku z tym wykres wielomianu zaczyna się od lewej strony powyżej osi OX. A więc $$x \in (-\infty,-12) \cup (-3,6).$$
\rozwStop
\odpStart
$x \in (-\infty,-12) \cup (-3,6)$
\odpStop
\testStart
A.$x \in (-\infty,-12) \cup (-3,6)$\\
B.$x \in (-\infty,-12) \cup (-3,6]$\\
C.$x \in (-\infty,-12) \cup [-3,6)$\\
D.$x \in (-\infty,-12] \cup (-3,6)$\\
E.$x \in (-\infty,-12] \cup (-3,6]$\\
F.$x \in (-\infty,-12] \cup [-3,6)$\\
G.$x \in (-\infty,-12) \cup [-3,6]$\\
H.$x \in (-\infty,-12] \cup [-3,6]$
\testStop
\kluczStart
A
\kluczStop



\zadStart{Zadanie z Wikieł Z 1.62 b) moja wersja nr 179}

Rozwiązać nierówności $(x+12)(6-x)(x+4)\ge0$.
\zadStop
\rozwStart{Patryk Wirkus}{}
Miejsca zerowe naszego wielomianu to: $-12, 6, -4$.\\
Wielomian jest stopnia nieparzystego, ponadto znak współczynnika przy\linebreak najwyższej potędze x jest ujemny.\\ W związku z tym wykres wielomianu zaczyna się od lewej strony powyżej osi OX. A więc $$x \in (-\infty,-12) \cup (-4,6).$$
\rozwStop
\odpStart
$x \in (-\infty,-12) \cup (-4,6)$
\odpStop
\testStart
A.$x \in (-\infty,-12) \cup (-4,6)$\\
B.$x \in (-\infty,-12) \cup (-4,6]$\\
C.$x \in (-\infty,-12) \cup [-4,6)$\\
D.$x \in (-\infty,-12] \cup (-4,6)$\\
E.$x \in (-\infty,-12] \cup (-4,6]$\\
F.$x \in (-\infty,-12] \cup [-4,6)$\\
G.$x \in (-\infty,-12) \cup [-4,6]$\\
H.$x \in (-\infty,-12] \cup [-4,6]$
\testStop
\kluczStart
A
\kluczStop



\zadStart{Zadanie z Wikieł Z 1.62 b) moja wersja nr 180}

Rozwiązać nierówności $(x+12)(6-x)(x+5)\ge0$.
\zadStop
\rozwStart{Patryk Wirkus}{}
Miejsca zerowe naszego wielomianu to: $-12, 6, -5$.\\
Wielomian jest stopnia nieparzystego, ponadto znak współczynnika przy\linebreak najwyższej potędze x jest ujemny.\\ W związku z tym wykres wielomianu zaczyna się od lewej strony powyżej osi OX. A więc $$x \in (-\infty,-12) \cup (-5,6).$$
\rozwStop
\odpStart
$x \in (-\infty,-12) \cup (-5,6)$
\odpStop
\testStart
A.$x \in (-\infty,-12) \cup (-5,6)$\\
B.$x \in (-\infty,-12) \cup (-5,6]$\\
C.$x \in (-\infty,-12) \cup [-5,6)$\\
D.$x \in (-\infty,-12] \cup (-5,6)$\\
E.$x \in (-\infty,-12] \cup (-5,6]$\\
F.$x \in (-\infty,-12] \cup [-5,6)$\\
G.$x \in (-\infty,-12) \cup [-5,6]$\\
H.$x \in (-\infty,-12] \cup [-5,6]$
\testStop
\kluczStart
A
\kluczStop



\zadStart{Zadanie z Wikieł Z 1.62 b) moja wersja nr 181}

Rozwiązać nierówności $(x+12)(7-x)(x+1)\ge0$.
\zadStop
\rozwStart{Patryk Wirkus}{}
Miejsca zerowe naszego wielomianu to: $-12, 7, -1$.\\
Wielomian jest stopnia nieparzystego, ponadto znak współczynnika przy\linebreak najwyższej potędze x jest ujemny.\\ W związku z tym wykres wielomianu zaczyna się od lewej strony powyżej osi OX. A więc $$x \in (-\infty,-12) \cup (-1,7).$$
\rozwStop
\odpStart
$x \in (-\infty,-12) \cup (-1,7)$
\odpStop
\testStart
A.$x \in (-\infty,-12) \cup (-1,7)$\\
B.$x \in (-\infty,-12) \cup (-1,7]$\\
C.$x \in (-\infty,-12) \cup [-1,7)$\\
D.$x \in (-\infty,-12] \cup (-1,7)$\\
E.$x \in (-\infty,-12] \cup (-1,7]$\\
F.$x \in (-\infty,-12] \cup [-1,7)$\\
G.$x \in (-\infty,-12) \cup [-1,7]$\\
H.$x \in (-\infty,-12] \cup [-1,7]$
\testStop
\kluczStart
A
\kluczStop



\zadStart{Zadanie z Wikieł Z 1.62 b) moja wersja nr 182}

Rozwiązać nierówności $(x+12)(7-x)(x+2)\ge0$.
\zadStop
\rozwStart{Patryk Wirkus}{}
Miejsca zerowe naszego wielomianu to: $-12, 7, -2$.\\
Wielomian jest stopnia nieparzystego, ponadto znak współczynnika przy\linebreak najwyższej potędze x jest ujemny.\\ W związku z tym wykres wielomianu zaczyna się od lewej strony powyżej osi OX. A więc $$x \in (-\infty,-12) \cup (-2,7).$$
\rozwStop
\odpStart
$x \in (-\infty,-12) \cup (-2,7)$
\odpStop
\testStart
A.$x \in (-\infty,-12) \cup (-2,7)$\\
B.$x \in (-\infty,-12) \cup (-2,7]$\\
C.$x \in (-\infty,-12) \cup [-2,7)$\\
D.$x \in (-\infty,-12] \cup (-2,7)$\\
E.$x \in (-\infty,-12] \cup (-2,7]$\\
F.$x \in (-\infty,-12] \cup [-2,7)$\\
G.$x \in (-\infty,-12) \cup [-2,7]$\\
H.$x \in (-\infty,-12] \cup [-2,7]$
\testStop
\kluczStart
A
\kluczStop



\zadStart{Zadanie z Wikieł Z 1.62 b) moja wersja nr 183}

Rozwiązać nierówności $(x+12)(7-x)(x+3)\ge0$.
\zadStop
\rozwStart{Patryk Wirkus}{}
Miejsca zerowe naszego wielomianu to: $-12, 7, -3$.\\
Wielomian jest stopnia nieparzystego, ponadto znak współczynnika przy\linebreak najwyższej potędze x jest ujemny.\\ W związku z tym wykres wielomianu zaczyna się od lewej strony powyżej osi OX. A więc $$x \in (-\infty,-12) \cup (-3,7).$$
\rozwStop
\odpStart
$x \in (-\infty,-12) \cup (-3,7)$
\odpStop
\testStart
A.$x \in (-\infty,-12) \cup (-3,7)$\\
B.$x \in (-\infty,-12) \cup (-3,7]$\\
C.$x \in (-\infty,-12) \cup [-3,7)$\\
D.$x \in (-\infty,-12] \cup (-3,7)$\\
E.$x \in (-\infty,-12] \cup (-3,7]$\\
F.$x \in (-\infty,-12] \cup [-3,7)$\\
G.$x \in (-\infty,-12) \cup [-3,7]$\\
H.$x \in (-\infty,-12] \cup [-3,7]$
\testStop
\kluczStart
A
\kluczStop



\zadStart{Zadanie z Wikieł Z 1.62 b) moja wersja nr 184}

Rozwiązać nierówności $(x+12)(7-x)(x+4)\ge0$.
\zadStop
\rozwStart{Patryk Wirkus}{}
Miejsca zerowe naszego wielomianu to: $-12, 7, -4$.\\
Wielomian jest stopnia nieparzystego, ponadto znak współczynnika przy\linebreak najwyższej potędze x jest ujemny.\\ W związku z tym wykres wielomianu zaczyna się od lewej strony powyżej osi OX. A więc $$x \in (-\infty,-12) \cup (-4,7).$$
\rozwStop
\odpStart
$x \in (-\infty,-12) \cup (-4,7)$
\odpStop
\testStart
A.$x \in (-\infty,-12) \cup (-4,7)$\\
B.$x \in (-\infty,-12) \cup (-4,7]$\\
C.$x \in (-\infty,-12) \cup [-4,7)$\\
D.$x \in (-\infty,-12] \cup (-4,7)$\\
E.$x \in (-\infty,-12] \cup (-4,7]$\\
F.$x \in (-\infty,-12] \cup [-4,7)$\\
G.$x \in (-\infty,-12) \cup [-4,7]$\\
H.$x \in (-\infty,-12] \cup [-4,7]$
\testStop
\kluczStart
A
\kluczStop



\zadStart{Zadanie z Wikieł Z 1.62 b) moja wersja nr 185}

Rozwiązać nierówności $(x+12)(7-x)(x+5)\ge0$.
\zadStop
\rozwStart{Patryk Wirkus}{}
Miejsca zerowe naszego wielomianu to: $-12, 7, -5$.\\
Wielomian jest stopnia nieparzystego, ponadto znak współczynnika przy\linebreak najwyższej potędze x jest ujemny.\\ W związku z tym wykres wielomianu zaczyna się od lewej strony powyżej osi OX. A więc $$x \in (-\infty,-12) \cup (-5,7).$$
\rozwStop
\odpStart
$x \in (-\infty,-12) \cup (-5,7)$
\odpStop
\testStart
A.$x \in (-\infty,-12) \cup (-5,7)$\\
B.$x \in (-\infty,-12) \cup (-5,7]$\\
C.$x \in (-\infty,-12) \cup [-5,7)$\\
D.$x \in (-\infty,-12] \cup (-5,7)$\\
E.$x \in (-\infty,-12] \cup (-5,7]$\\
F.$x \in (-\infty,-12] \cup [-5,7)$\\
G.$x \in (-\infty,-12) \cup [-5,7]$\\
H.$x \in (-\infty,-12] \cup [-5,7]$
\testStop
\kluczStart
A
\kluczStop



\zadStart{Zadanie z Wikieł Z 1.62 b) moja wersja nr 186}

Rozwiązać nierówności $(x+12)(7-x)(x+6)\ge0$.
\zadStop
\rozwStart{Patryk Wirkus}{}
Miejsca zerowe naszego wielomianu to: $-12, 7, -6$.\\
Wielomian jest stopnia nieparzystego, ponadto znak współczynnika przy\linebreak najwyższej potędze x jest ujemny.\\ W związku z tym wykres wielomianu zaczyna się od lewej strony powyżej osi OX. A więc $$x \in (-\infty,-12) \cup (-6,7).$$
\rozwStop
\odpStart
$x \in (-\infty,-12) \cup (-6,7)$
\odpStop
\testStart
A.$x \in (-\infty,-12) \cup (-6,7)$\\
B.$x \in (-\infty,-12) \cup (-6,7]$\\
C.$x \in (-\infty,-12) \cup [-6,7)$\\
D.$x \in (-\infty,-12] \cup (-6,7)$\\
E.$x \in (-\infty,-12] \cup (-6,7]$\\
F.$x \in (-\infty,-12] \cup [-6,7)$\\
G.$x \in (-\infty,-12) \cup [-6,7]$\\
H.$x \in (-\infty,-12] \cup [-6,7]$
\testStop
\kluczStart
A
\kluczStop



\zadStart{Zadanie z Wikieł Z 1.62 b) moja wersja nr 187}

Rozwiązać nierówności $(x+12)(8-x)(x+1)\ge0$.
\zadStop
\rozwStart{Patryk Wirkus}{}
Miejsca zerowe naszego wielomianu to: $-12, 8, -1$.\\
Wielomian jest stopnia nieparzystego, ponadto znak współczynnika przy\linebreak najwyższej potędze x jest ujemny.\\ W związku z tym wykres wielomianu zaczyna się od lewej strony powyżej osi OX. A więc $$x \in (-\infty,-12) \cup (-1,8).$$
\rozwStop
\odpStart
$x \in (-\infty,-12) \cup (-1,8)$
\odpStop
\testStart
A.$x \in (-\infty,-12) \cup (-1,8)$\\
B.$x \in (-\infty,-12) \cup (-1,8]$\\
C.$x \in (-\infty,-12) \cup [-1,8)$\\
D.$x \in (-\infty,-12] \cup (-1,8)$\\
E.$x \in (-\infty,-12] \cup (-1,8]$\\
F.$x \in (-\infty,-12] \cup [-1,8)$\\
G.$x \in (-\infty,-12) \cup [-1,8]$\\
H.$x \in (-\infty,-12] \cup [-1,8]$
\testStop
\kluczStart
A
\kluczStop



\zadStart{Zadanie z Wikieł Z 1.62 b) moja wersja nr 188}

Rozwiązać nierówności $(x+12)(8-x)(x+2)\ge0$.
\zadStop
\rozwStart{Patryk Wirkus}{}
Miejsca zerowe naszego wielomianu to: $-12, 8, -2$.\\
Wielomian jest stopnia nieparzystego, ponadto znak współczynnika przy\linebreak najwyższej potędze x jest ujemny.\\ W związku z tym wykres wielomianu zaczyna się od lewej strony powyżej osi OX. A więc $$x \in (-\infty,-12) \cup (-2,8).$$
\rozwStop
\odpStart
$x \in (-\infty,-12) \cup (-2,8)$
\odpStop
\testStart
A.$x \in (-\infty,-12) \cup (-2,8)$\\
B.$x \in (-\infty,-12) \cup (-2,8]$\\
C.$x \in (-\infty,-12) \cup [-2,8)$\\
D.$x \in (-\infty,-12] \cup (-2,8)$\\
E.$x \in (-\infty,-12] \cup (-2,8]$\\
F.$x \in (-\infty,-12] \cup [-2,8)$\\
G.$x \in (-\infty,-12) \cup [-2,8]$\\
H.$x \in (-\infty,-12] \cup [-2,8]$
\testStop
\kluczStart
A
\kluczStop



\zadStart{Zadanie z Wikieł Z 1.62 b) moja wersja nr 189}

Rozwiązać nierówności $(x+12)(8-x)(x+3)\ge0$.
\zadStop
\rozwStart{Patryk Wirkus}{}
Miejsca zerowe naszego wielomianu to: $-12, 8, -3$.\\
Wielomian jest stopnia nieparzystego, ponadto znak współczynnika przy\linebreak najwyższej potędze x jest ujemny.\\ W związku z tym wykres wielomianu zaczyna się od lewej strony powyżej osi OX. A więc $$x \in (-\infty,-12) \cup (-3,8).$$
\rozwStop
\odpStart
$x \in (-\infty,-12) \cup (-3,8)$
\odpStop
\testStart
A.$x \in (-\infty,-12) \cup (-3,8)$\\
B.$x \in (-\infty,-12) \cup (-3,8]$\\
C.$x \in (-\infty,-12) \cup [-3,8)$\\
D.$x \in (-\infty,-12] \cup (-3,8)$\\
E.$x \in (-\infty,-12] \cup (-3,8]$\\
F.$x \in (-\infty,-12] \cup [-3,8)$\\
G.$x \in (-\infty,-12) \cup [-3,8]$\\
H.$x \in (-\infty,-12] \cup [-3,8]$
\testStop
\kluczStart
A
\kluczStop



\zadStart{Zadanie z Wikieł Z 1.62 b) moja wersja nr 190}

Rozwiązać nierówności $(x+12)(8-x)(x+4)\ge0$.
\zadStop
\rozwStart{Patryk Wirkus}{}
Miejsca zerowe naszego wielomianu to: $-12, 8, -4$.\\
Wielomian jest stopnia nieparzystego, ponadto znak współczynnika przy\linebreak najwyższej potędze x jest ujemny.\\ W związku z tym wykres wielomianu zaczyna się od lewej strony powyżej osi OX. A więc $$x \in (-\infty,-12) \cup (-4,8).$$
\rozwStop
\odpStart
$x \in (-\infty,-12) \cup (-4,8)$
\odpStop
\testStart
A.$x \in (-\infty,-12) \cup (-4,8)$\\
B.$x \in (-\infty,-12) \cup (-4,8]$\\
C.$x \in (-\infty,-12) \cup [-4,8)$\\
D.$x \in (-\infty,-12] \cup (-4,8)$\\
E.$x \in (-\infty,-12] \cup (-4,8]$\\
F.$x \in (-\infty,-12] \cup [-4,8)$\\
G.$x \in (-\infty,-12) \cup [-4,8]$\\
H.$x \in (-\infty,-12] \cup [-4,8]$
\testStop
\kluczStart
A
\kluczStop



\zadStart{Zadanie z Wikieł Z 1.62 b) moja wersja nr 191}

Rozwiązać nierówności $(x+12)(8-x)(x+5)\ge0$.
\zadStop
\rozwStart{Patryk Wirkus}{}
Miejsca zerowe naszego wielomianu to: $-12, 8, -5$.\\
Wielomian jest stopnia nieparzystego, ponadto znak współczynnika przy\linebreak najwyższej potędze x jest ujemny.\\ W związku z tym wykres wielomianu zaczyna się od lewej strony powyżej osi OX. A więc $$x \in (-\infty,-12) \cup (-5,8).$$
\rozwStop
\odpStart
$x \in (-\infty,-12) \cup (-5,8)$
\odpStop
\testStart
A.$x \in (-\infty,-12) \cup (-5,8)$\\
B.$x \in (-\infty,-12) \cup (-5,8]$\\
C.$x \in (-\infty,-12) \cup [-5,8)$\\
D.$x \in (-\infty,-12] \cup (-5,8)$\\
E.$x \in (-\infty,-12] \cup (-5,8]$\\
F.$x \in (-\infty,-12] \cup [-5,8)$\\
G.$x \in (-\infty,-12) \cup [-5,8]$\\
H.$x \in (-\infty,-12] \cup [-5,8]$
\testStop
\kluczStart
A
\kluczStop



\zadStart{Zadanie z Wikieł Z 1.62 b) moja wersja nr 192}

Rozwiązać nierówności $(x+12)(8-x)(x+6)\ge0$.
\zadStop
\rozwStart{Patryk Wirkus}{}
Miejsca zerowe naszego wielomianu to: $-12, 8, -6$.\\
Wielomian jest stopnia nieparzystego, ponadto znak współczynnika przy\linebreak najwyższej potędze x jest ujemny.\\ W związku z tym wykres wielomianu zaczyna się od lewej strony powyżej osi OX. A więc $$x \in (-\infty,-12) \cup (-6,8).$$
\rozwStop
\odpStart
$x \in (-\infty,-12) \cup (-6,8)$
\odpStop
\testStart
A.$x \in (-\infty,-12) \cup (-6,8)$\\
B.$x \in (-\infty,-12) \cup (-6,8]$\\
C.$x \in (-\infty,-12) \cup [-6,8)$\\
D.$x \in (-\infty,-12] \cup (-6,8)$\\
E.$x \in (-\infty,-12] \cup (-6,8]$\\
F.$x \in (-\infty,-12] \cup [-6,8)$\\
G.$x \in (-\infty,-12) \cup [-6,8]$\\
H.$x \in (-\infty,-12] \cup [-6,8]$
\testStop
\kluczStart
A
\kluczStop



\zadStart{Zadanie z Wikieł Z 1.62 b) moja wersja nr 193}

Rozwiązać nierówności $(x+12)(8-x)(x+7)\ge0$.
\zadStop
\rozwStart{Patryk Wirkus}{}
Miejsca zerowe naszego wielomianu to: $-12, 8, -7$.\\
Wielomian jest stopnia nieparzystego, ponadto znak współczynnika przy\linebreak najwyższej potędze x jest ujemny.\\ W związku z tym wykres wielomianu zaczyna się od lewej strony powyżej osi OX. A więc $$x \in (-\infty,-12) \cup (-7,8).$$
\rozwStop
\odpStart
$x \in (-\infty,-12) \cup (-7,8)$
\odpStop
\testStart
A.$x \in (-\infty,-12) \cup (-7,8)$\\
B.$x \in (-\infty,-12) \cup (-7,8]$\\
C.$x \in (-\infty,-12) \cup [-7,8)$\\
D.$x \in (-\infty,-12] \cup (-7,8)$\\
E.$x \in (-\infty,-12] \cup (-7,8]$\\
F.$x \in (-\infty,-12] \cup [-7,8)$\\
G.$x \in (-\infty,-12) \cup [-7,8]$\\
H.$x \in (-\infty,-12] \cup [-7,8]$
\testStop
\kluczStart
A
\kluczStop



\zadStart{Zadanie z Wikieł Z 1.62 b) moja wersja nr 194}

Rozwiązać nierówności $(x+12)(9-x)(x+1)\ge0$.
\zadStop
\rozwStart{Patryk Wirkus}{}
Miejsca zerowe naszego wielomianu to: $-12, 9, -1$.\\
Wielomian jest stopnia nieparzystego, ponadto znak współczynnika przy\linebreak najwyższej potędze x jest ujemny.\\ W związku z tym wykres wielomianu zaczyna się od lewej strony powyżej osi OX. A więc $$x \in (-\infty,-12) \cup (-1,9).$$
\rozwStop
\odpStart
$x \in (-\infty,-12) \cup (-1,9)$
\odpStop
\testStart
A.$x \in (-\infty,-12) \cup (-1,9)$\\
B.$x \in (-\infty,-12) \cup (-1,9]$\\
C.$x \in (-\infty,-12) \cup [-1,9)$\\
D.$x \in (-\infty,-12] \cup (-1,9)$\\
E.$x \in (-\infty,-12] \cup (-1,9]$\\
F.$x \in (-\infty,-12] \cup [-1,9)$\\
G.$x \in (-\infty,-12) \cup [-1,9]$\\
H.$x \in (-\infty,-12] \cup [-1,9]$
\testStop
\kluczStart
A
\kluczStop



\zadStart{Zadanie z Wikieł Z 1.62 b) moja wersja nr 195}

Rozwiązać nierówności $(x+12)(9-x)(x+2)\ge0$.
\zadStop
\rozwStart{Patryk Wirkus}{}
Miejsca zerowe naszego wielomianu to: $-12, 9, -2$.\\
Wielomian jest stopnia nieparzystego, ponadto znak współczynnika przy\linebreak najwyższej potędze x jest ujemny.\\ W związku z tym wykres wielomianu zaczyna się od lewej strony powyżej osi OX. A więc $$x \in (-\infty,-12) \cup (-2,9).$$
\rozwStop
\odpStart
$x \in (-\infty,-12) \cup (-2,9)$
\odpStop
\testStart
A.$x \in (-\infty,-12) \cup (-2,9)$\\
B.$x \in (-\infty,-12) \cup (-2,9]$\\
C.$x \in (-\infty,-12) \cup [-2,9)$\\
D.$x \in (-\infty,-12] \cup (-2,9)$\\
E.$x \in (-\infty,-12] \cup (-2,9]$\\
F.$x \in (-\infty,-12] \cup [-2,9)$\\
G.$x \in (-\infty,-12) \cup [-2,9]$\\
H.$x \in (-\infty,-12] \cup [-2,9]$
\testStop
\kluczStart
A
\kluczStop



\zadStart{Zadanie z Wikieł Z 1.62 b) moja wersja nr 196}

Rozwiązać nierówności $(x+12)(9-x)(x+3)\ge0$.
\zadStop
\rozwStart{Patryk Wirkus}{}
Miejsca zerowe naszego wielomianu to: $-12, 9, -3$.\\
Wielomian jest stopnia nieparzystego, ponadto znak współczynnika przy\linebreak najwyższej potędze x jest ujemny.\\ W związku z tym wykres wielomianu zaczyna się od lewej strony powyżej osi OX. A więc $$x \in (-\infty,-12) \cup (-3,9).$$
\rozwStop
\odpStart
$x \in (-\infty,-12) \cup (-3,9)$
\odpStop
\testStart
A.$x \in (-\infty,-12) \cup (-3,9)$\\
B.$x \in (-\infty,-12) \cup (-3,9]$\\
C.$x \in (-\infty,-12) \cup [-3,9)$\\
D.$x \in (-\infty,-12] \cup (-3,9)$\\
E.$x \in (-\infty,-12] \cup (-3,9]$\\
F.$x \in (-\infty,-12] \cup [-3,9)$\\
G.$x \in (-\infty,-12) \cup [-3,9]$\\
H.$x \in (-\infty,-12] \cup [-3,9]$
\testStop
\kluczStart
A
\kluczStop



\zadStart{Zadanie z Wikieł Z 1.62 b) moja wersja nr 197}

Rozwiązać nierówności $(x+12)(9-x)(x+4)\ge0$.
\zadStop
\rozwStart{Patryk Wirkus}{}
Miejsca zerowe naszego wielomianu to: $-12, 9, -4$.\\
Wielomian jest stopnia nieparzystego, ponadto znak współczynnika przy\linebreak najwyższej potędze x jest ujemny.\\ W związku z tym wykres wielomianu zaczyna się od lewej strony powyżej osi OX. A więc $$x \in (-\infty,-12) \cup (-4,9).$$
\rozwStop
\odpStart
$x \in (-\infty,-12) \cup (-4,9)$
\odpStop
\testStart
A.$x \in (-\infty,-12) \cup (-4,9)$\\
B.$x \in (-\infty,-12) \cup (-4,9]$\\
C.$x \in (-\infty,-12) \cup [-4,9)$\\
D.$x \in (-\infty,-12] \cup (-4,9)$\\
E.$x \in (-\infty,-12] \cup (-4,9]$\\
F.$x \in (-\infty,-12] \cup [-4,9)$\\
G.$x \in (-\infty,-12) \cup [-4,9]$\\
H.$x \in (-\infty,-12] \cup [-4,9]$
\testStop
\kluczStart
A
\kluczStop



\zadStart{Zadanie z Wikieł Z 1.62 b) moja wersja nr 198}

Rozwiązać nierówności $(x+12)(9-x)(x+5)\ge0$.
\zadStop
\rozwStart{Patryk Wirkus}{}
Miejsca zerowe naszego wielomianu to: $-12, 9, -5$.\\
Wielomian jest stopnia nieparzystego, ponadto znak współczynnika przy\linebreak najwyższej potędze x jest ujemny.\\ W związku z tym wykres wielomianu zaczyna się od lewej strony powyżej osi OX. A więc $$x \in (-\infty,-12) \cup (-5,9).$$
\rozwStop
\odpStart
$x \in (-\infty,-12) \cup (-5,9)$
\odpStop
\testStart
A.$x \in (-\infty,-12) \cup (-5,9)$\\
B.$x \in (-\infty,-12) \cup (-5,9]$\\
C.$x \in (-\infty,-12) \cup [-5,9)$\\
D.$x \in (-\infty,-12] \cup (-5,9)$\\
E.$x \in (-\infty,-12] \cup (-5,9]$\\
F.$x \in (-\infty,-12] \cup [-5,9)$\\
G.$x \in (-\infty,-12) \cup [-5,9]$\\
H.$x \in (-\infty,-12] \cup [-5,9]$
\testStop
\kluczStart
A
\kluczStop



\zadStart{Zadanie z Wikieł Z 1.62 b) moja wersja nr 199}

Rozwiązać nierówności $(x+12)(9-x)(x+6)\ge0$.
\zadStop
\rozwStart{Patryk Wirkus}{}
Miejsca zerowe naszego wielomianu to: $-12, 9, -6$.\\
Wielomian jest stopnia nieparzystego, ponadto znak współczynnika przy\linebreak najwyższej potędze x jest ujemny.\\ W związku z tym wykres wielomianu zaczyna się od lewej strony powyżej osi OX. A więc $$x \in (-\infty,-12) \cup (-6,9).$$
\rozwStop
\odpStart
$x \in (-\infty,-12) \cup (-6,9)$
\odpStop
\testStart
A.$x \in (-\infty,-12) \cup (-6,9)$\\
B.$x \in (-\infty,-12) \cup (-6,9]$\\
C.$x \in (-\infty,-12) \cup [-6,9)$\\
D.$x \in (-\infty,-12] \cup (-6,9)$\\
E.$x \in (-\infty,-12] \cup (-6,9]$\\
F.$x \in (-\infty,-12] \cup [-6,9)$\\
G.$x \in (-\infty,-12) \cup [-6,9]$\\
H.$x \in (-\infty,-12] \cup [-6,9]$
\testStop
\kluczStart
A
\kluczStop



\zadStart{Zadanie z Wikieł Z 1.62 b) moja wersja nr 200}

Rozwiązać nierówności $(x+12)(9-x)(x+7)\ge0$.
\zadStop
\rozwStart{Patryk Wirkus}{}
Miejsca zerowe naszego wielomianu to: $-12, 9, -7$.\\
Wielomian jest stopnia nieparzystego, ponadto znak współczynnika przy\linebreak najwyższej potędze x jest ujemny.\\ W związku z tym wykres wielomianu zaczyna się od lewej strony powyżej osi OX. A więc $$x \in (-\infty,-12) \cup (-7,9).$$
\rozwStop
\odpStart
$x \in (-\infty,-12) \cup (-7,9)$
\odpStop
\testStart
A.$x \in (-\infty,-12) \cup (-7,9)$\\
B.$x \in (-\infty,-12) \cup (-7,9]$\\
C.$x \in (-\infty,-12) \cup [-7,9)$\\
D.$x \in (-\infty,-12] \cup (-7,9)$\\
E.$x \in (-\infty,-12] \cup (-7,9]$\\
F.$x \in (-\infty,-12] \cup [-7,9)$\\
G.$x \in (-\infty,-12) \cup [-7,9]$\\
H.$x \in (-\infty,-12] \cup [-7,9]$
\testStop
\kluczStart
A
\kluczStop



\zadStart{Zadanie z Wikieł Z 1.62 b) moja wersja nr 201}

Rozwiązać nierówności $(x+12)(9-x)(x+8)\ge0$.
\zadStop
\rozwStart{Patryk Wirkus}{}
Miejsca zerowe naszego wielomianu to: $-12, 9, -8$.\\
Wielomian jest stopnia nieparzystego, ponadto znak współczynnika przy\linebreak najwyższej potędze x jest ujemny.\\ W związku z tym wykres wielomianu zaczyna się od lewej strony powyżej osi OX. A więc $$x \in (-\infty,-12) \cup (-8,9).$$
\rozwStop
\odpStart
$x \in (-\infty,-12) \cup (-8,9)$
\odpStop
\testStart
A.$x \in (-\infty,-12) \cup (-8,9)$\\
B.$x \in (-\infty,-12) \cup (-8,9]$\\
C.$x \in (-\infty,-12) \cup [-8,9)$\\
D.$x \in (-\infty,-12] \cup (-8,9)$\\
E.$x \in (-\infty,-12] \cup (-8,9]$\\
F.$x \in (-\infty,-12] \cup [-8,9)$\\
G.$x \in (-\infty,-12) \cup [-8,9]$\\
H.$x \in (-\infty,-12] \cup [-8,9]$
\testStop
\kluczStart
A
\kluczStop



\zadStart{Zadanie z Wikieł Z 1.62 b) moja wersja nr 202}

Rozwiązać nierówności $(x+12)(10-x)(x+1)\ge0$.
\zadStop
\rozwStart{Patryk Wirkus}{}
Miejsca zerowe naszego wielomianu to: $-12, 10, -1$.\\
Wielomian jest stopnia nieparzystego, ponadto znak współczynnika przy\linebreak najwyższej potędze x jest ujemny.\\ W związku z tym wykres wielomianu zaczyna się od lewej strony powyżej osi OX. A więc $$x \in (-\infty,-12) \cup (-1,10).$$
\rozwStop
\odpStart
$x \in (-\infty,-12) \cup (-1,10)$
\odpStop
\testStart
A.$x \in (-\infty,-12) \cup (-1,10)$\\
B.$x \in (-\infty,-12) \cup (-1,10]$\\
C.$x \in (-\infty,-12) \cup [-1,10)$\\
D.$x \in (-\infty,-12] \cup (-1,10)$\\
E.$x \in (-\infty,-12] \cup (-1,10]$\\
F.$x \in (-\infty,-12] \cup [-1,10)$\\
G.$x \in (-\infty,-12) \cup [-1,10]$\\
H.$x \in (-\infty,-12] \cup [-1,10]$
\testStop
\kluczStart
A
\kluczStop



\zadStart{Zadanie z Wikieł Z 1.62 b) moja wersja nr 203}

Rozwiązać nierówności $(x+12)(10-x)(x+2)\ge0$.
\zadStop
\rozwStart{Patryk Wirkus}{}
Miejsca zerowe naszego wielomianu to: $-12, 10, -2$.\\
Wielomian jest stopnia nieparzystego, ponadto znak współczynnika przy\linebreak najwyższej potędze x jest ujemny.\\ W związku z tym wykres wielomianu zaczyna się od lewej strony powyżej osi OX. A więc $$x \in (-\infty,-12) \cup (-2,10).$$
\rozwStop
\odpStart
$x \in (-\infty,-12) \cup (-2,10)$
\odpStop
\testStart
A.$x \in (-\infty,-12) \cup (-2,10)$\\
B.$x \in (-\infty,-12) \cup (-2,10]$\\
C.$x \in (-\infty,-12) \cup [-2,10)$\\
D.$x \in (-\infty,-12] \cup (-2,10)$\\
E.$x \in (-\infty,-12] \cup (-2,10]$\\
F.$x \in (-\infty,-12] \cup [-2,10)$\\
G.$x \in (-\infty,-12) \cup [-2,10]$\\
H.$x \in (-\infty,-12] \cup [-2,10]$
\testStop
\kluczStart
A
\kluczStop



\zadStart{Zadanie z Wikieł Z 1.62 b) moja wersja nr 204}

Rozwiązać nierówności $(x+12)(10-x)(x+3)\ge0$.
\zadStop
\rozwStart{Patryk Wirkus}{}
Miejsca zerowe naszego wielomianu to: $-12, 10, -3$.\\
Wielomian jest stopnia nieparzystego, ponadto znak współczynnika przy\linebreak najwyższej potędze x jest ujemny.\\ W związku z tym wykres wielomianu zaczyna się od lewej strony powyżej osi OX. A więc $$x \in (-\infty,-12) \cup (-3,10).$$
\rozwStop
\odpStart
$x \in (-\infty,-12) \cup (-3,10)$
\odpStop
\testStart
A.$x \in (-\infty,-12) \cup (-3,10)$\\
B.$x \in (-\infty,-12) \cup (-3,10]$\\
C.$x \in (-\infty,-12) \cup [-3,10)$\\
D.$x \in (-\infty,-12] \cup (-3,10)$\\
E.$x \in (-\infty,-12] \cup (-3,10]$\\
F.$x \in (-\infty,-12] \cup [-3,10)$\\
G.$x \in (-\infty,-12) \cup [-3,10]$\\
H.$x \in (-\infty,-12] \cup [-3,10]$
\testStop
\kluczStart
A
\kluczStop



\zadStart{Zadanie z Wikieł Z 1.62 b) moja wersja nr 205}

Rozwiązać nierówności $(x+12)(10-x)(x+4)\ge0$.
\zadStop
\rozwStart{Patryk Wirkus}{}
Miejsca zerowe naszego wielomianu to: $-12, 10, -4$.\\
Wielomian jest stopnia nieparzystego, ponadto znak współczynnika przy\linebreak najwyższej potędze x jest ujemny.\\ W związku z tym wykres wielomianu zaczyna się od lewej strony powyżej osi OX. A więc $$x \in (-\infty,-12) \cup (-4,10).$$
\rozwStop
\odpStart
$x \in (-\infty,-12) \cup (-4,10)$
\odpStop
\testStart
A.$x \in (-\infty,-12) \cup (-4,10)$\\
B.$x \in (-\infty,-12) \cup (-4,10]$\\
C.$x \in (-\infty,-12) \cup [-4,10)$\\
D.$x \in (-\infty,-12] \cup (-4,10)$\\
E.$x \in (-\infty,-12] \cup (-4,10]$\\
F.$x \in (-\infty,-12] \cup [-4,10)$\\
G.$x \in (-\infty,-12) \cup [-4,10]$\\
H.$x \in (-\infty,-12] \cup [-4,10]$
\testStop
\kluczStart
A
\kluczStop



\zadStart{Zadanie z Wikieł Z 1.62 b) moja wersja nr 206}

Rozwiązać nierówności $(x+12)(10-x)(x+5)\ge0$.
\zadStop
\rozwStart{Patryk Wirkus}{}
Miejsca zerowe naszego wielomianu to: $-12, 10, -5$.\\
Wielomian jest stopnia nieparzystego, ponadto znak współczynnika przy\linebreak najwyższej potędze x jest ujemny.\\ W związku z tym wykres wielomianu zaczyna się od lewej strony powyżej osi OX. A więc $$x \in (-\infty,-12) \cup (-5,10).$$
\rozwStop
\odpStart
$x \in (-\infty,-12) \cup (-5,10)$
\odpStop
\testStart
A.$x \in (-\infty,-12) \cup (-5,10)$\\
B.$x \in (-\infty,-12) \cup (-5,10]$\\
C.$x \in (-\infty,-12) \cup [-5,10)$\\
D.$x \in (-\infty,-12] \cup (-5,10)$\\
E.$x \in (-\infty,-12] \cup (-5,10]$\\
F.$x \in (-\infty,-12] \cup [-5,10)$\\
G.$x \in (-\infty,-12) \cup [-5,10]$\\
H.$x \in (-\infty,-12] \cup [-5,10]$
\testStop
\kluczStart
A
\kluczStop



\zadStart{Zadanie z Wikieł Z 1.62 b) moja wersja nr 207}

Rozwiązać nierówności $(x+12)(10-x)(x+6)\ge0$.
\zadStop
\rozwStart{Patryk Wirkus}{}
Miejsca zerowe naszego wielomianu to: $-12, 10, -6$.\\
Wielomian jest stopnia nieparzystego, ponadto znak współczynnika przy\linebreak najwyższej potędze x jest ujemny.\\ W związku z tym wykres wielomianu zaczyna się od lewej strony powyżej osi OX. A więc $$x \in (-\infty,-12) \cup (-6,10).$$
\rozwStop
\odpStart
$x \in (-\infty,-12) \cup (-6,10)$
\odpStop
\testStart
A.$x \in (-\infty,-12) \cup (-6,10)$\\
B.$x \in (-\infty,-12) \cup (-6,10]$\\
C.$x \in (-\infty,-12) \cup [-6,10)$\\
D.$x \in (-\infty,-12] \cup (-6,10)$\\
E.$x \in (-\infty,-12] \cup (-6,10]$\\
F.$x \in (-\infty,-12] \cup [-6,10)$\\
G.$x \in (-\infty,-12) \cup [-6,10]$\\
H.$x \in (-\infty,-12] \cup [-6,10]$
\testStop
\kluczStart
A
\kluczStop



\zadStart{Zadanie z Wikieł Z 1.62 b) moja wersja nr 208}

Rozwiązać nierówności $(x+12)(10-x)(x+7)\ge0$.
\zadStop
\rozwStart{Patryk Wirkus}{}
Miejsca zerowe naszego wielomianu to: $-12, 10, -7$.\\
Wielomian jest stopnia nieparzystego, ponadto znak współczynnika przy\linebreak najwyższej potędze x jest ujemny.\\ W związku z tym wykres wielomianu zaczyna się od lewej strony powyżej osi OX. A więc $$x \in (-\infty,-12) \cup (-7,10).$$
\rozwStop
\odpStart
$x \in (-\infty,-12) \cup (-7,10)$
\odpStop
\testStart
A.$x \in (-\infty,-12) \cup (-7,10)$\\
B.$x \in (-\infty,-12) \cup (-7,10]$\\
C.$x \in (-\infty,-12) \cup [-7,10)$\\
D.$x \in (-\infty,-12] \cup (-7,10)$\\
E.$x \in (-\infty,-12] \cup (-7,10]$\\
F.$x \in (-\infty,-12] \cup [-7,10)$\\
G.$x \in (-\infty,-12) \cup [-7,10]$\\
H.$x \in (-\infty,-12] \cup [-7,10]$
\testStop
\kluczStart
A
\kluczStop



\zadStart{Zadanie z Wikieł Z 1.62 b) moja wersja nr 209}

Rozwiązać nierówności $(x+12)(10-x)(x+8)\ge0$.
\zadStop
\rozwStart{Patryk Wirkus}{}
Miejsca zerowe naszego wielomianu to: $-12, 10, -8$.\\
Wielomian jest stopnia nieparzystego, ponadto znak współczynnika przy\linebreak najwyższej potędze x jest ujemny.\\ W związku z tym wykres wielomianu zaczyna się od lewej strony powyżej osi OX. A więc $$x \in (-\infty,-12) \cup (-8,10).$$
\rozwStop
\odpStart
$x \in (-\infty,-12) \cup (-8,10)$
\odpStop
\testStart
A.$x \in (-\infty,-12) \cup (-8,10)$\\
B.$x \in (-\infty,-12) \cup (-8,10]$\\
C.$x \in (-\infty,-12) \cup [-8,10)$\\
D.$x \in (-\infty,-12] \cup (-8,10)$\\
E.$x \in (-\infty,-12] \cup (-8,10]$\\
F.$x \in (-\infty,-12] \cup [-8,10)$\\
G.$x \in (-\infty,-12) \cup [-8,10]$\\
H.$x \in (-\infty,-12] \cup [-8,10]$
\testStop
\kluczStart
A
\kluczStop



\zadStart{Zadanie z Wikieł Z 1.62 b) moja wersja nr 210}

Rozwiązać nierówności $(x+12)(10-x)(x+9)\ge0$.
\zadStop
\rozwStart{Patryk Wirkus}{}
Miejsca zerowe naszego wielomianu to: $-12, 10, -9$.\\
Wielomian jest stopnia nieparzystego, ponadto znak współczynnika przy\linebreak najwyższej potędze x jest ujemny.\\ W związku z tym wykres wielomianu zaczyna się od lewej strony powyżej osi OX. A więc $$x \in (-\infty,-12) \cup (-9,10).$$
\rozwStop
\odpStart
$x \in (-\infty,-12) \cup (-9,10)$
\odpStop
\testStart
A.$x \in (-\infty,-12) \cup (-9,10)$\\
B.$x \in (-\infty,-12) \cup (-9,10]$\\
C.$x \in (-\infty,-12) \cup [-9,10)$\\
D.$x \in (-\infty,-12] \cup (-9,10)$\\
E.$x \in (-\infty,-12] \cup (-9,10]$\\
F.$x \in (-\infty,-12] \cup [-9,10)$\\
G.$x \in (-\infty,-12) \cup [-9,10]$\\
H.$x \in (-\infty,-12] \cup [-9,10]$
\testStop
\kluczStart
A
\kluczStop



\zadStart{Zadanie z Wikieł Z 1.62 b) moja wersja nr 211}

Rozwiązać nierówności $(x+12)(11-x)(x+1)\ge0$.
\zadStop
\rozwStart{Patryk Wirkus}{}
Miejsca zerowe naszego wielomianu to: $-12, 11, -1$.\\
Wielomian jest stopnia nieparzystego, ponadto znak współczynnika przy\linebreak najwyższej potędze x jest ujemny.\\ W związku z tym wykres wielomianu zaczyna się od lewej strony powyżej osi OX. A więc $$x \in (-\infty,-12) \cup (-1,11).$$
\rozwStop
\odpStart
$x \in (-\infty,-12) \cup (-1,11)$
\odpStop
\testStart
A.$x \in (-\infty,-12) \cup (-1,11)$\\
B.$x \in (-\infty,-12) \cup (-1,11]$\\
C.$x \in (-\infty,-12) \cup [-1,11)$\\
D.$x \in (-\infty,-12] \cup (-1,11)$\\
E.$x \in (-\infty,-12] \cup (-1,11]$\\
F.$x \in (-\infty,-12] \cup [-1,11)$\\
G.$x \in (-\infty,-12) \cup [-1,11]$\\
H.$x \in (-\infty,-12] \cup [-1,11]$
\testStop
\kluczStart
A
\kluczStop



\zadStart{Zadanie z Wikieł Z 1.62 b) moja wersja nr 212}

Rozwiązać nierówności $(x+12)(11-x)(x+2)\ge0$.
\zadStop
\rozwStart{Patryk Wirkus}{}
Miejsca zerowe naszego wielomianu to: $-12, 11, -2$.\\
Wielomian jest stopnia nieparzystego, ponadto znak współczynnika przy\linebreak najwyższej potędze x jest ujemny.\\ W związku z tym wykres wielomianu zaczyna się od lewej strony powyżej osi OX. A więc $$x \in (-\infty,-12) \cup (-2,11).$$
\rozwStop
\odpStart
$x \in (-\infty,-12) \cup (-2,11)$
\odpStop
\testStart
A.$x \in (-\infty,-12) \cup (-2,11)$\\
B.$x \in (-\infty,-12) \cup (-2,11]$\\
C.$x \in (-\infty,-12) \cup [-2,11)$\\
D.$x \in (-\infty,-12] \cup (-2,11)$\\
E.$x \in (-\infty,-12] \cup (-2,11]$\\
F.$x \in (-\infty,-12] \cup [-2,11)$\\
G.$x \in (-\infty,-12) \cup [-2,11]$\\
H.$x \in (-\infty,-12] \cup [-2,11]$
\testStop
\kluczStart
A
\kluczStop



\zadStart{Zadanie z Wikieł Z 1.62 b) moja wersja nr 213}

Rozwiązać nierówności $(x+12)(11-x)(x+3)\ge0$.
\zadStop
\rozwStart{Patryk Wirkus}{}
Miejsca zerowe naszego wielomianu to: $-12, 11, -3$.\\
Wielomian jest stopnia nieparzystego, ponadto znak współczynnika przy\linebreak najwyższej potędze x jest ujemny.\\ W związku z tym wykres wielomianu zaczyna się od lewej strony powyżej osi OX. A więc $$x \in (-\infty,-12) \cup (-3,11).$$
\rozwStop
\odpStart
$x \in (-\infty,-12) \cup (-3,11)$
\odpStop
\testStart
A.$x \in (-\infty,-12) \cup (-3,11)$\\
B.$x \in (-\infty,-12) \cup (-3,11]$\\
C.$x \in (-\infty,-12) \cup [-3,11)$\\
D.$x \in (-\infty,-12] \cup (-3,11)$\\
E.$x \in (-\infty,-12] \cup (-3,11]$\\
F.$x \in (-\infty,-12] \cup [-3,11)$\\
G.$x \in (-\infty,-12) \cup [-3,11]$\\
H.$x \in (-\infty,-12] \cup [-3,11]$
\testStop
\kluczStart
A
\kluczStop



\zadStart{Zadanie z Wikieł Z 1.62 b) moja wersja nr 214}

Rozwiązać nierówności $(x+12)(11-x)(x+4)\ge0$.
\zadStop
\rozwStart{Patryk Wirkus}{}
Miejsca zerowe naszego wielomianu to: $-12, 11, -4$.\\
Wielomian jest stopnia nieparzystego, ponadto znak współczynnika przy\linebreak najwyższej potędze x jest ujemny.\\ W związku z tym wykres wielomianu zaczyna się od lewej strony powyżej osi OX. A więc $$x \in (-\infty,-12) \cup (-4,11).$$
\rozwStop
\odpStart
$x \in (-\infty,-12) \cup (-4,11)$
\odpStop
\testStart
A.$x \in (-\infty,-12) \cup (-4,11)$\\
B.$x \in (-\infty,-12) \cup (-4,11]$\\
C.$x \in (-\infty,-12) \cup [-4,11)$\\
D.$x \in (-\infty,-12] \cup (-4,11)$\\
E.$x \in (-\infty,-12] \cup (-4,11]$\\
F.$x \in (-\infty,-12] \cup [-4,11)$\\
G.$x \in (-\infty,-12) \cup [-4,11]$\\
H.$x \in (-\infty,-12] \cup [-4,11]$
\testStop
\kluczStart
A
\kluczStop



\zadStart{Zadanie z Wikieł Z 1.62 b) moja wersja nr 215}

Rozwiązać nierówności $(x+12)(11-x)(x+5)\ge0$.
\zadStop
\rozwStart{Patryk Wirkus}{}
Miejsca zerowe naszego wielomianu to: $-12, 11, -5$.\\
Wielomian jest stopnia nieparzystego, ponadto znak współczynnika przy\linebreak najwyższej potędze x jest ujemny.\\ W związku z tym wykres wielomianu zaczyna się od lewej strony powyżej osi OX. A więc $$x \in (-\infty,-12) \cup (-5,11).$$
\rozwStop
\odpStart
$x \in (-\infty,-12) \cup (-5,11)$
\odpStop
\testStart
A.$x \in (-\infty,-12) \cup (-5,11)$\\
B.$x \in (-\infty,-12) \cup (-5,11]$\\
C.$x \in (-\infty,-12) \cup [-5,11)$\\
D.$x \in (-\infty,-12] \cup (-5,11)$\\
E.$x \in (-\infty,-12] \cup (-5,11]$\\
F.$x \in (-\infty,-12] \cup [-5,11)$\\
G.$x \in (-\infty,-12) \cup [-5,11]$\\
H.$x \in (-\infty,-12] \cup [-5,11]$
\testStop
\kluczStart
A
\kluczStop



\zadStart{Zadanie z Wikieł Z 1.62 b) moja wersja nr 216}

Rozwiązać nierówności $(x+12)(11-x)(x+6)\ge0$.
\zadStop
\rozwStart{Patryk Wirkus}{}
Miejsca zerowe naszego wielomianu to: $-12, 11, -6$.\\
Wielomian jest stopnia nieparzystego, ponadto znak współczynnika przy\linebreak najwyższej potędze x jest ujemny.\\ W związku z tym wykres wielomianu zaczyna się od lewej strony powyżej osi OX. A więc $$x \in (-\infty,-12) \cup (-6,11).$$
\rozwStop
\odpStart
$x \in (-\infty,-12) \cup (-6,11)$
\odpStop
\testStart
A.$x \in (-\infty,-12) \cup (-6,11)$\\
B.$x \in (-\infty,-12) \cup (-6,11]$\\
C.$x \in (-\infty,-12) \cup [-6,11)$\\
D.$x \in (-\infty,-12] \cup (-6,11)$\\
E.$x \in (-\infty,-12] \cup (-6,11]$\\
F.$x \in (-\infty,-12] \cup [-6,11)$\\
G.$x \in (-\infty,-12) \cup [-6,11]$\\
H.$x \in (-\infty,-12] \cup [-6,11]$
\testStop
\kluczStart
A
\kluczStop



\zadStart{Zadanie z Wikieł Z 1.62 b) moja wersja nr 217}

Rozwiązać nierówności $(x+12)(11-x)(x+7)\ge0$.
\zadStop
\rozwStart{Patryk Wirkus}{}
Miejsca zerowe naszego wielomianu to: $-12, 11, -7$.\\
Wielomian jest stopnia nieparzystego, ponadto znak współczynnika przy\linebreak najwyższej potędze x jest ujemny.\\ W związku z tym wykres wielomianu zaczyna się od lewej strony powyżej osi OX. A więc $$x \in (-\infty,-12) \cup (-7,11).$$
\rozwStop
\odpStart
$x \in (-\infty,-12) \cup (-7,11)$
\odpStop
\testStart
A.$x \in (-\infty,-12) \cup (-7,11)$\\
B.$x \in (-\infty,-12) \cup (-7,11]$\\
C.$x \in (-\infty,-12) \cup [-7,11)$\\
D.$x \in (-\infty,-12] \cup (-7,11)$\\
E.$x \in (-\infty,-12] \cup (-7,11]$\\
F.$x \in (-\infty,-12] \cup [-7,11)$\\
G.$x \in (-\infty,-12) \cup [-7,11]$\\
H.$x \in (-\infty,-12] \cup [-7,11]$
\testStop
\kluczStart
A
\kluczStop



\zadStart{Zadanie z Wikieł Z 1.62 b) moja wersja nr 218}

Rozwiązać nierówności $(x+12)(11-x)(x+8)\ge0$.
\zadStop
\rozwStart{Patryk Wirkus}{}
Miejsca zerowe naszego wielomianu to: $-12, 11, -8$.\\
Wielomian jest stopnia nieparzystego, ponadto znak współczynnika przy\linebreak najwyższej potędze x jest ujemny.\\ W związku z tym wykres wielomianu zaczyna się od lewej strony powyżej osi OX. A więc $$x \in (-\infty,-12) \cup (-8,11).$$
\rozwStop
\odpStart
$x \in (-\infty,-12) \cup (-8,11)$
\odpStop
\testStart
A.$x \in (-\infty,-12) \cup (-8,11)$\\
B.$x \in (-\infty,-12) \cup (-8,11]$\\
C.$x \in (-\infty,-12) \cup [-8,11)$\\
D.$x \in (-\infty,-12] \cup (-8,11)$\\
E.$x \in (-\infty,-12] \cup (-8,11]$\\
F.$x \in (-\infty,-12] \cup [-8,11)$\\
G.$x \in (-\infty,-12) \cup [-8,11]$\\
H.$x \in (-\infty,-12] \cup [-8,11]$
\testStop
\kluczStart
A
\kluczStop



\zadStart{Zadanie z Wikieł Z 1.62 b) moja wersja nr 219}

Rozwiązać nierówności $(x+12)(11-x)(x+9)\ge0$.
\zadStop
\rozwStart{Patryk Wirkus}{}
Miejsca zerowe naszego wielomianu to: $-12, 11, -9$.\\
Wielomian jest stopnia nieparzystego, ponadto znak współczynnika przy\linebreak najwyższej potędze x jest ujemny.\\ W związku z tym wykres wielomianu zaczyna się od lewej strony powyżej osi OX. A więc $$x \in (-\infty,-12) \cup (-9,11).$$
\rozwStop
\odpStart
$x \in (-\infty,-12) \cup (-9,11)$
\odpStop
\testStart
A.$x \in (-\infty,-12) \cup (-9,11)$\\
B.$x \in (-\infty,-12) \cup (-9,11]$\\
C.$x \in (-\infty,-12) \cup [-9,11)$\\
D.$x \in (-\infty,-12] \cup (-9,11)$\\
E.$x \in (-\infty,-12] \cup (-9,11]$\\
F.$x \in (-\infty,-12] \cup [-9,11)$\\
G.$x \in (-\infty,-12) \cup [-9,11]$\\
H.$x \in (-\infty,-12] \cup [-9,11]$
\testStop
\kluczStart
A
\kluczStop



\zadStart{Zadanie z Wikieł Z 1.62 b) moja wersja nr 220}

Rozwiązać nierówności $(x+12)(11-x)(x+10)\ge0$.
\zadStop
\rozwStart{Patryk Wirkus}{}
Miejsca zerowe naszego wielomianu to: $-12, 11, -10$.\\
Wielomian jest stopnia nieparzystego, ponadto znak współczynnika przy\linebreak najwyższej potędze x jest ujemny.\\ W związku z tym wykres wielomianu zaczyna się od lewej strony powyżej osi OX. A więc $$x \in (-\infty,-12) \cup (-10,11).$$
\rozwStop
\odpStart
$x \in (-\infty,-12) \cup (-10,11)$
\odpStop
\testStart
A.$x \in (-\infty,-12) \cup (-10,11)$\\
B.$x \in (-\infty,-12) \cup (-10,11]$\\
C.$x \in (-\infty,-12) \cup [-10,11)$\\
D.$x \in (-\infty,-12] \cup (-10,11)$\\
E.$x \in (-\infty,-12] \cup (-10,11]$\\
F.$x \in (-\infty,-12] \cup [-10,11)$\\
G.$x \in (-\infty,-12) \cup [-10,11]$\\
H.$x \in (-\infty,-12] \cup [-10,11]$
\testStop
\kluczStart
A
\kluczStop



\zadStart{Zadanie z Wikieł Z 1.62 b) moja wersja nr 221}

Rozwiązać nierówności $(x+13)(2-x)(x+1)\ge0$.
\zadStop
\rozwStart{Patryk Wirkus}{}
Miejsca zerowe naszego wielomianu to: $-13, 2, -1$.\\
Wielomian jest stopnia nieparzystego, ponadto znak współczynnika przy\linebreak najwyższej potędze x jest ujemny.\\ W związku z tym wykres wielomianu zaczyna się od lewej strony powyżej osi OX. A więc $$x \in (-\infty,-13) \cup (-1,2).$$
\rozwStop
\odpStart
$x \in (-\infty,-13) \cup (-1,2)$
\odpStop
\testStart
A.$x \in (-\infty,-13) \cup (-1,2)$\\
B.$x \in (-\infty,-13) \cup (-1,2]$\\
C.$x \in (-\infty,-13) \cup [-1,2)$\\
D.$x \in (-\infty,-13] \cup (-1,2)$\\
E.$x \in (-\infty,-13] \cup (-1,2]$\\
F.$x \in (-\infty,-13] \cup [-1,2)$\\
G.$x \in (-\infty,-13) \cup [-1,2]$\\
H.$x \in (-\infty,-13] \cup [-1,2]$
\testStop
\kluczStart
A
\kluczStop



\zadStart{Zadanie z Wikieł Z 1.62 b) moja wersja nr 222}

Rozwiązać nierówności $(x+13)(3-x)(x+1)\ge0$.
\zadStop
\rozwStart{Patryk Wirkus}{}
Miejsca zerowe naszego wielomianu to: $-13, 3, -1$.\\
Wielomian jest stopnia nieparzystego, ponadto znak współczynnika przy\linebreak najwyższej potędze x jest ujemny.\\ W związku z tym wykres wielomianu zaczyna się od lewej strony powyżej osi OX. A więc $$x \in (-\infty,-13) \cup (-1,3).$$
\rozwStop
\odpStart
$x \in (-\infty,-13) \cup (-1,3)$
\odpStop
\testStart
A.$x \in (-\infty,-13) \cup (-1,3)$\\
B.$x \in (-\infty,-13) \cup (-1,3]$\\
C.$x \in (-\infty,-13) \cup [-1,3)$\\
D.$x \in (-\infty,-13] \cup (-1,3)$\\
E.$x \in (-\infty,-13] \cup (-1,3]$\\
F.$x \in (-\infty,-13] \cup [-1,3)$\\
G.$x \in (-\infty,-13) \cup [-1,3]$\\
H.$x \in (-\infty,-13] \cup [-1,3]$
\testStop
\kluczStart
A
\kluczStop



\zadStart{Zadanie z Wikieł Z 1.62 b) moja wersja nr 223}

Rozwiązać nierówności $(x+13)(3-x)(x+2)\ge0$.
\zadStop
\rozwStart{Patryk Wirkus}{}
Miejsca zerowe naszego wielomianu to: $-13, 3, -2$.\\
Wielomian jest stopnia nieparzystego, ponadto znak współczynnika przy\linebreak najwyższej potędze x jest ujemny.\\ W związku z tym wykres wielomianu zaczyna się od lewej strony powyżej osi OX. A więc $$x \in (-\infty,-13) \cup (-2,3).$$
\rozwStop
\odpStart
$x \in (-\infty,-13) \cup (-2,3)$
\odpStop
\testStart
A.$x \in (-\infty,-13) \cup (-2,3)$\\
B.$x \in (-\infty,-13) \cup (-2,3]$\\
C.$x \in (-\infty,-13) \cup [-2,3)$\\
D.$x \in (-\infty,-13] \cup (-2,3)$\\
E.$x \in (-\infty,-13] \cup (-2,3]$\\
F.$x \in (-\infty,-13] \cup [-2,3)$\\
G.$x \in (-\infty,-13) \cup [-2,3]$\\
H.$x \in (-\infty,-13] \cup [-2,3]$
\testStop
\kluczStart
A
\kluczStop



\zadStart{Zadanie z Wikieł Z 1.62 b) moja wersja nr 224}

Rozwiązać nierówności $(x+13)(4-x)(x+1)\ge0$.
\zadStop
\rozwStart{Patryk Wirkus}{}
Miejsca zerowe naszego wielomianu to: $-13, 4, -1$.\\
Wielomian jest stopnia nieparzystego, ponadto znak współczynnika przy\linebreak najwyższej potędze x jest ujemny.\\ W związku z tym wykres wielomianu zaczyna się od lewej strony powyżej osi OX. A więc $$x \in (-\infty,-13) \cup (-1,4).$$
\rozwStop
\odpStart
$x \in (-\infty,-13) \cup (-1,4)$
\odpStop
\testStart
A.$x \in (-\infty,-13) \cup (-1,4)$\\
B.$x \in (-\infty,-13) \cup (-1,4]$\\
C.$x \in (-\infty,-13) \cup [-1,4)$\\
D.$x \in (-\infty,-13] \cup (-1,4)$\\
E.$x \in (-\infty,-13] \cup (-1,4]$\\
F.$x \in (-\infty,-13] \cup [-1,4)$\\
G.$x \in (-\infty,-13) \cup [-1,4]$\\
H.$x \in (-\infty,-13] \cup [-1,4]$
\testStop
\kluczStart
A
\kluczStop



\zadStart{Zadanie z Wikieł Z 1.62 b) moja wersja nr 225}

Rozwiązać nierówności $(x+13)(4-x)(x+2)\ge0$.
\zadStop
\rozwStart{Patryk Wirkus}{}
Miejsca zerowe naszego wielomianu to: $-13, 4, -2$.\\
Wielomian jest stopnia nieparzystego, ponadto znak współczynnika przy\linebreak najwyższej potędze x jest ujemny.\\ W związku z tym wykres wielomianu zaczyna się od lewej strony powyżej osi OX. A więc $$x \in (-\infty,-13) \cup (-2,4).$$
\rozwStop
\odpStart
$x \in (-\infty,-13) \cup (-2,4)$
\odpStop
\testStart
A.$x \in (-\infty,-13) \cup (-2,4)$\\
B.$x \in (-\infty,-13) \cup (-2,4]$\\
C.$x \in (-\infty,-13) \cup [-2,4)$\\
D.$x \in (-\infty,-13] \cup (-2,4)$\\
E.$x \in (-\infty,-13] \cup (-2,4]$\\
F.$x \in (-\infty,-13] \cup [-2,4)$\\
G.$x \in (-\infty,-13) \cup [-2,4]$\\
H.$x \in (-\infty,-13] \cup [-2,4]$
\testStop
\kluczStart
A
\kluczStop



\zadStart{Zadanie z Wikieł Z 1.62 b) moja wersja nr 226}

Rozwiązać nierówności $(x+13)(4-x)(x+3)\ge0$.
\zadStop
\rozwStart{Patryk Wirkus}{}
Miejsca zerowe naszego wielomianu to: $-13, 4, -3$.\\
Wielomian jest stopnia nieparzystego, ponadto znak współczynnika przy\linebreak najwyższej potędze x jest ujemny.\\ W związku z tym wykres wielomianu zaczyna się od lewej strony powyżej osi OX. A więc $$x \in (-\infty,-13) \cup (-3,4).$$
\rozwStop
\odpStart
$x \in (-\infty,-13) \cup (-3,4)$
\odpStop
\testStart
A.$x \in (-\infty,-13) \cup (-3,4)$\\
B.$x \in (-\infty,-13) \cup (-3,4]$\\
C.$x \in (-\infty,-13) \cup [-3,4)$\\
D.$x \in (-\infty,-13] \cup (-3,4)$\\
E.$x \in (-\infty,-13] \cup (-3,4]$\\
F.$x \in (-\infty,-13] \cup [-3,4)$\\
G.$x \in (-\infty,-13) \cup [-3,4]$\\
H.$x \in (-\infty,-13] \cup [-3,4]$
\testStop
\kluczStart
A
\kluczStop



\zadStart{Zadanie z Wikieł Z 1.62 b) moja wersja nr 227}

Rozwiązać nierówności $(x+13)(5-x)(x+1)\ge0$.
\zadStop
\rozwStart{Patryk Wirkus}{}
Miejsca zerowe naszego wielomianu to: $-13, 5, -1$.\\
Wielomian jest stopnia nieparzystego, ponadto znak współczynnika przy\linebreak najwyższej potędze x jest ujemny.\\ W związku z tym wykres wielomianu zaczyna się od lewej strony powyżej osi OX. A więc $$x \in (-\infty,-13) \cup (-1,5).$$
\rozwStop
\odpStart
$x \in (-\infty,-13) \cup (-1,5)$
\odpStop
\testStart
A.$x \in (-\infty,-13) \cup (-1,5)$\\
B.$x \in (-\infty,-13) \cup (-1,5]$\\
C.$x \in (-\infty,-13) \cup [-1,5)$\\
D.$x \in (-\infty,-13] \cup (-1,5)$\\
E.$x \in (-\infty,-13] \cup (-1,5]$\\
F.$x \in (-\infty,-13] \cup [-1,5)$\\
G.$x \in (-\infty,-13) \cup [-1,5]$\\
H.$x \in (-\infty,-13] \cup [-1,5]$
\testStop
\kluczStart
A
\kluczStop



\zadStart{Zadanie z Wikieł Z 1.62 b) moja wersja nr 228}

Rozwiązać nierówności $(x+13)(5-x)(x+2)\ge0$.
\zadStop
\rozwStart{Patryk Wirkus}{}
Miejsca zerowe naszego wielomianu to: $-13, 5, -2$.\\
Wielomian jest stopnia nieparzystego, ponadto znak współczynnika przy\linebreak najwyższej potędze x jest ujemny.\\ W związku z tym wykres wielomianu zaczyna się od lewej strony powyżej osi OX. A więc $$x \in (-\infty,-13) \cup (-2,5).$$
\rozwStop
\odpStart
$x \in (-\infty,-13) \cup (-2,5)$
\odpStop
\testStart
A.$x \in (-\infty,-13) \cup (-2,5)$\\
B.$x \in (-\infty,-13) \cup (-2,5]$\\
C.$x \in (-\infty,-13) \cup [-2,5)$\\
D.$x \in (-\infty,-13] \cup (-2,5)$\\
E.$x \in (-\infty,-13] \cup (-2,5]$\\
F.$x \in (-\infty,-13] \cup [-2,5)$\\
G.$x \in (-\infty,-13) \cup [-2,5]$\\
H.$x \in (-\infty,-13] \cup [-2,5]$
\testStop
\kluczStart
A
\kluczStop



\zadStart{Zadanie z Wikieł Z 1.62 b) moja wersja nr 229}

Rozwiązać nierówności $(x+13)(5-x)(x+3)\ge0$.
\zadStop
\rozwStart{Patryk Wirkus}{}
Miejsca zerowe naszego wielomianu to: $-13, 5, -3$.\\
Wielomian jest stopnia nieparzystego, ponadto znak współczynnika przy\linebreak najwyższej potędze x jest ujemny.\\ W związku z tym wykres wielomianu zaczyna się od lewej strony powyżej osi OX. A więc $$x \in (-\infty,-13) \cup (-3,5).$$
\rozwStop
\odpStart
$x \in (-\infty,-13) \cup (-3,5)$
\odpStop
\testStart
A.$x \in (-\infty,-13) \cup (-3,5)$\\
B.$x \in (-\infty,-13) \cup (-3,5]$\\
C.$x \in (-\infty,-13) \cup [-3,5)$\\
D.$x \in (-\infty,-13] \cup (-3,5)$\\
E.$x \in (-\infty,-13] \cup (-3,5]$\\
F.$x \in (-\infty,-13] \cup [-3,5)$\\
G.$x \in (-\infty,-13) \cup [-3,5]$\\
H.$x \in (-\infty,-13] \cup [-3,5]$
\testStop
\kluczStart
A
\kluczStop



\zadStart{Zadanie z Wikieł Z 1.62 b) moja wersja nr 230}

Rozwiązać nierówności $(x+13)(5-x)(x+4)\ge0$.
\zadStop
\rozwStart{Patryk Wirkus}{}
Miejsca zerowe naszego wielomianu to: $-13, 5, -4$.\\
Wielomian jest stopnia nieparzystego, ponadto znak współczynnika przy\linebreak najwyższej potędze x jest ujemny.\\ W związku z tym wykres wielomianu zaczyna się od lewej strony powyżej osi OX. A więc $$x \in (-\infty,-13) \cup (-4,5).$$
\rozwStop
\odpStart
$x \in (-\infty,-13) \cup (-4,5)$
\odpStop
\testStart
A.$x \in (-\infty,-13) \cup (-4,5)$\\
B.$x \in (-\infty,-13) \cup (-4,5]$\\
C.$x \in (-\infty,-13) \cup [-4,5)$\\
D.$x \in (-\infty,-13] \cup (-4,5)$\\
E.$x \in (-\infty,-13] \cup (-4,5]$\\
F.$x \in (-\infty,-13] \cup [-4,5)$\\
G.$x \in (-\infty,-13) \cup [-4,5]$\\
H.$x \in (-\infty,-13] \cup [-4,5]$
\testStop
\kluczStart
A
\kluczStop



\zadStart{Zadanie z Wikieł Z 1.62 b) moja wersja nr 231}

Rozwiązać nierówności $(x+13)(6-x)(x+1)\ge0$.
\zadStop
\rozwStart{Patryk Wirkus}{}
Miejsca zerowe naszego wielomianu to: $-13, 6, -1$.\\
Wielomian jest stopnia nieparzystego, ponadto znak współczynnika przy\linebreak najwyższej potędze x jest ujemny.\\ W związku z tym wykres wielomianu zaczyna się od lewej strony powyżej osi OX. A więc $$x \in (-\infty,-13) \cup (-1,6).$$
\rozwStop
\odpStart
$x \in (-\infty,-13) \cup (-1,6)$
\odpStop
\testStart
A.$x \in (-\infty,-13) \cup (-1,6)$\\
B.$x \in (-\infty,-13) \cup (-1,6]$\\
C.$x \in (-\infty,-13) \cup [-1,6)$\\
D.$x \in (-\infty,-13] \cup (-1,6)$\\
E.$x \in (-\infty,-13] \cup (-1,6]$\\
F.$x \in (-\infty,-13] \cup [-1,6)$\\
G.$x \in (-\infty,-13) \cup [-1,6]$\\
H.$x \in (-\infty,-13] \cup [-1,6]$
\testStop
\kluczStart
A
\kluczStop



\zadStart{Zadanie z Wikieł Z 1.62 b) moja wersja nr 232}

Rozwiązać nierówności $(x+13)(6-x)(x+2)\ge0$.
\zadStop
\rozwStart{Patryk Wirkus}{}
Miejsca zerowe naszego wielomianu to: $-13, 6, -2$.\\
Wielomian jest stopnia nieparzystego, ponadto znak współczynnika przy\linebreak najwyższej potędze x jest ujemny.\\ W związku z tym wykres wielomianu zaczyna się od lewej strony powyżej osi OX. A więc $$x \in (-\infty,-13) \cup (-2,6).$$
\rozwStop
\odpStart
$x \in (-\infty,-13) \cup (-2,6)$
\odpStop
\testStart
A.$x \in (-\infty,-13) \cup (-2,6)$\\
B.$x \in (-\infty,-13) \cup (-2,6]$\\
C.$x \in (-\infty,-13) \cup [-2,6)$\\
D.$x \in (-\infty,-13] \cup (-2,6)$\\
E.$x \in (-\infty,-13] \cup (-2,6]$\\
F.$x \in (-\infty,-13] \cup [-2,6)$\\
G.$x \in (-\infty,-13) \cup [-2,6]$\\
H.$x \in (-\infty,-13] \cup [-2,6]$
\testStop
\kluczStart
A
\kluczStop



\zadStart{Zadanie z Wikieł Z 1.62 b) moja wersja nr 233}

Rozwiązać nierówności $(x+13)(6-x)(x+3)\ge0$.
\zadStop
\rozwStart{Patryk Wirkus}{}
Miejsca zerowe naszego wielomianu to: $-13, 6, -3$.\\
Wielomian jest stopnia nieparzystego, ponadto znak współczynnika przy\linebreak najwyższej potędze x jest ujemny.\\ W związku z tym wykres wielomianu zaczyna się od lewej strony powyżej osi OX. A więc $$x \in (-\infty,-13) \cup (-3,6).$$
\rozwStop
\odpStart
$x \in (-\infty,-13) \cup (-3,6)$
\odpStop
\testStart
A.$x \in (-\infty,-13) \cup (-3,6)$\\
B.$x \in (-\infty,-13) \cup (-3,6]$\\
C.$x \in (-\infty,-13) \cup [-3,6)$\\
D.$x \in (-\infty,-13] \cup (-3,6)$\\
E.$x \in (-\infty,-13] \cup (-3,6]$\\
F.$x \in (-\infty,-13] \cup [-3,6)$\\
G.$x \in (-\infty,-13) \cup [-3,6]$\\
H.$x \in (-\infty,-13] \cup [-3,6]$
\testStop
\kluczStart
A
\kluczStop



\zadStart{Zadanie z Wikieł Z 1.62 b) moja wersja nr 234}

Rozwiązać nierówności $(x+13)(6-x)(x+4)\ge0$.
\zadStop
\rozwStart{Patryk Wirkus}{}
Miejsca zerowe naszego wielomianu to: $-13, 6, -4$.\\
Wielomian jest stopnia nieparzystego, ponadto znak współczynnika przy\linebreak najwyższej potędze x jest ujemny.\\ W związku z tym wykres wielomianu zaczyna się od lewej strony powyżej osi OX. A więc $$x \in (-\infty,-13) \cup (-4,6).$$
\rozwStop
\odpStart
$x \in (-\infty,-13) \cup (-4,6)$
\odpStop
\testStart
A.$x \in (-\infty,-13) \cup (-4,6)$\\
B.$x \in (-\infty,-13) \cup (-4,6]$\\
C.$x \in (-\infty,-13) \cup [-4,6)$\\
D.$x \in (-\infty,-13] \cup (-4,6)$\\
E.$x \in (-\infty,-13] \cup (-4,6]$\\
F.$x \in (-\infty,-13] \cup [-4,6)$\\
G.$x \in (-\infty,-13) \cup [-4,6]$\\
H.$x \in (-\infty,-13] \cup [-4,6]$
\testStop
\kluczStart
A
\kluczStop



\zadStart{Zadanie z Wikieł Z 1.62 b) moja wersja nr 235}

Rozwiązać nierówności $(x+13)(6-x)(x+5)\ge0$.
\zadStop
\rozwStart{Patryk Wirkus}{}
Miejsca zerowe naszego wielomianu to: $-13, 6, -5$.\\
Wielomian jest stopnia nieparzystego, ponadto znak współczynnika przy\linebreak najwyższej potędze x jest ujemny.\\ W związku z tym wykres wielomianu zaczyna się od lewej strony powyżej osi OX. A więc $$x \in (-\infty,-13) \cup (-5,6).$$
\rozwStop
\odpStart
$x \in (-\infty,-13) \cup (-5,6)$
\odpStop
\testStart
A.$x \in (-\infty,-13) \cup (-5,6)$\\
B.$x \in (-\infty,-13) \cup (-5,6]$\\
C.$x \in (-\infty,-13) \cup [-5,6)$\\
D.$x \in (-\infty,-13] \cup (-5,6)$\\
E.$x \in (-\infty,-13] \cup (-5,6]$\\
F.$x \in (-\infty,-13] \cup [-5,6)$\\
G.$x \in (-\infty,-13) \cup [-5,6]$\\
H.$x \in (-\infty,-13] \cup [-5,6]$
\testStop
\kluczStart
A
\kluczStop



\zadStart{Zadanie z Wikieł Z 1.62 b) moja wersja nr 236}

Rozwiązać nierówności $(x+13)(7-x)(x+1)\ge0$.
\zadStop
\rozwStart{Patryk Wirkus}{}
Miejsca zerowe naszego wielomianu to: $-13, 7, -1$.\\
Wielomian jest stopnia nieparzystego, ponadto znak współczynnika przy\linebreak najwyższej potędze x jest ujemny.\\ W związku z tym wykres wielomianu zaczyna się od lewej strony powyżej osi OX. A więc $$x \in (-\infty,-13) \cup (-1,7).$$
\rozwStop
\odpStart
$x \in (-\infty,-13) \cup (-1,7)$
\odpStop
\testStart
A.$x \in (-\infty,-13) \cup (-1,7)$\\
B.$x \in (-\infty,-13) \cup (-1,7]$\\
C.$x \in (-\infty,-13) \cup [-1,7)$\\
D.$x \in (-\infty,-13] \cup (-1,7)$\\
E.$x \in (-\infty,-13] \cup (-1,7]$\\
F.$x \in (-\infty,-13] \cup [-1,7)$\\
G.$x \in (-\infty,-13) \cup [-1,7]$\\
H.$x \in (-\infty,-13] \cup [-1,7]$
\testStop
\kluczStart
A
\kluczStop



\zadStart{Zadanie z Wikieł Z 1.62 b) moja wersja nr 237}

Rozwiązać nierówności $(x+13)(7-x)(x+2)\ge0$.
\zadStop
\rozwStart{Patryk Wirkus}{}
Miejsca zerowe naszego wielomianu to: $-13, 7, -2$.\\
Wielomian jest stopnia nieparzystego, ponadto znak współczynnika przy\linebreak najwyższej potędze x jest ujemny.\\ W związku z tym wykres wielomianu zaczyna się od lewej strony powyżej osi OX. A więc $$x \in (-\infty,-13) \cup (-2,7).$$
\rozwStop
\odpStart
$x \in (-\infty,-13) \cup (-2,7)$
\odpStop
\testStart
A.$x \in (-\infty,-13) \cup (-2,7)$\\
B.$x \in (-\infty,-13) \cup (-2,7]$\\
C.$x \in (-\infty,-13) \cup [-2,7)$\\
D.$x \in (-\infty,-13] \cup (-2,7)$\\
E.$x \in (-\infty,-13] \cup (-2,7]$\\
F.$x \in (-\infty,-13] \cup [-2,7)$\\
G.$x \in (-\infty,-13) \cup [-2,7]$\\
H.$x \in (-\infty,-13] \cup [-2,7]$
\testStop
\kluczStart
A
\kluczStop



\zadStart{Zadanie z Wikieł Z 1.62 b) moja wersja nr 238}

Rozwiązać nierówności $(x+13)(7-x)(x+3)\ge0$.
\zadStop
\rozwStart{Patryk Wirkus}{}
Miejsca zerowe naszego wielomianu to: $-13, 7, -3$.\\
Wielomian jest stopnia nieparzystego, ponadto znak współczynnika przy\linebreak najwyższej potędze x jest ujemny.\\ W związku z tym wykres wielomianu zaczyna się od lewej strony powyżej osi OX. A więc $$x \in (-\infty,-13) \cup (-3,7).$$
\rozwStop
\odpStart
$x \in (-\infty,-13) \cup (-3,7)$
\odpStop
\testStart
A.$x \in (-\infty,-13) \cup (-3,7)$\\
B.$x \in (-\infty,-13) \cup (-3,7]$\\
C.$x \in (-\infty,-13) \cup [-3,7)$\\
D.$x \in (-\infty,-13] \cup (-3,7)$\\
E.$x \in (-\infty,-13] \cup (-3,7]$\\
F.$x \in (-\infty,-13] \cup [-3,7)$\\
G.$x \in (-\infty,-13) \cup [-3,7]$\\
H.$x \in (-\infty,-13] \cup [-3,7]$
\testStop
\kluczStart
A
\kluczStop



\zadStart{Zadanie z Wikieł Z 1.62 b) moja wersja nr 239}

Rozwiązać nierówności $(x+13)(7-x)(x+4)\ge0$.
\zadStop
\rozwStart{Patryk Wirkus}{}
Miejsca zerowe naszego wielomianu to: $-13, 7, -4$.\\
Wielomian jest stopnia nieparzystego, ponadto znak współczynnika przy\linebreak najwyższej potędze x jest ujemny.\\ W związku z tym wykres wielomianu zaczyna się od lewej strony powyżej osi OX. A więc $$x \in (-\infty,-13) \cup (-4,7).$$
\rozwStop
\odpStart
$x \in (-\infty,-13) \cup (-4,7)$
\odpStop
\testStart
A.$x \in (-\infty,-13) \cup (-4,7)$\\
B.$x \in (-\infty,-13) \cup (-4,7]$\\
C.$x \in (-\infty,-13) \cup [-4,7)$\\
D.$x \in (-\infty,-13] \cup (-4,7)$\\
E.$x \in (-\infty,-13] \cup (-4,7]$\\
F.$x \in (-\infty,-13] \cup [-4,7)$\\
G.$x \in (-\infty,-13) \cup [-4,7]$\\
H.$x \in (-\infty,-13] \cup [-4,7]$
\testStop
\kluczStart
A
\kluczStop



\zadStart{Zadanie z Wikieł Z 1.62 b) moja wersja nr 240}

Rozwiązać nierówności $(x+13)(7-x)(x+5)\ge0$.
\zadStop
\rozwStart{Patryk Wirkus}{}
Miejsca zerowe naszego wielomianu to: $-13, 7, -5$.\\
Wielomian jest stopnia nieparzystego, ponadto znak współczynnika przy\linebreak najwyższej potędze x jest ujemny.\\ W związku z tym wykres wielomianu zaczyna się od lewej strony powyżej osi OX. A więc $$x \in (-\infty,-13) \cup (-5,7).$$
\rozwStop
\odpStart
$x \in (-\infty,-13) \cup (-5,7)$
\odpStop
\testStart
A.$x \in (-\infty,-13) \cup (-5,7)$\\
B.$x \in (-\infty,-13) \cup (-5,7]$\\
C.$x \in (-\infty,-13) \cup [-5,7)$\\
D.$x \in (-\infty,-13] \cup (-5,7)$\\
E.$x \in (-\infty,-13] \cup (-5,7]$\\
F.$x \in (-\infty,-13] \cup [-5,7)$\\
G.$x \in (-\infty,-13) \cup [-5,7]$\\
H.$x \in (-\infty,-13] \cup [-5,7]$
\testStop
\kluczStart
A
\kluczStop



\zadStart{Zadanie z Wikieł Z 1.62 b) moja wersja nr 241}

Rozwiązać nierówności $(x+13)(7-x)(x+6)\ge0$.
\zadStop
\rozwStart{Patryk Wirkus}{}
Miejsca zerowe naszego wielomianu to: $-13, 7, -6$.\\
Wielomian jest stopnia nieparzystego, ponadto znak współczynnika przy\linebreak najwyższej potędze x jest ujemny.\\ W związku z tym wykres wielomianu zaczyna się od lewej strony powyżej osi OX. A więc $$x \in (-\infty,-13) \cup (-6,7).$$
\rozwStop
\odpStart
$x \in (-\infty,-13) \cup (-6,7)$
\odpStop
\testStart
A.$x \in (-\infty,-13) \cup (-6,7)$\\
B.$x \in (-\infty,-13) \cup (-6,7]$\\
C.$x \in (-\infty,-13) \cup [-6,7)$\\
D.$x \in (-\infty,-13] \cup (-6,7)$\\
E.$x \in (-\infty,-13] \cup (-6,7]$\\
F.$x \in (-\infty,-13] \cup [-6,7)$\\
G.$x \in (-\infty,-13) \cup [-6,7]$\\
H.$x \in (-\infty,-13] \cup [-6,7]$
\testStop
\kluczStart
A
\kluczStop



\zadStart{Zadanie z Wikieł Z 1.62 b) moja wersja nr 242}

Rozwiązać nierówności $(x+13)(8-x)(x+1)\ge0$.
\zadStop
\rozwStart{Patryk Wirkus}{}
Miejsca zerowe naszego wielomianu to: $-13, 8, -1$.\\
Wielomian jest stopnia nieparzystego, ponadto znak współczynnika przy\linebreak najwyższej potędze x jest ujemny.\\ W związku z tym wykres wielomianu zaczyna się od lewej strony powyżej osi OX. A więc $$x \in (-\infty,-13) \cup (-1,8).$$
\rozwStop
\odpStart
$x \in (-\infty,-13) \cup (-1,8)$
\odpStop
\testStart
A.$x \in (-\infty,-13) \cup (-1,8)$\\
B.$x \in (-\infty,-13) \cup (-1,8]$\\
C.$x \in (-\infty,-13) \cup [-1,8)$\\
D.$x \in (-\infty,-13] \cup (-1,8)$\\
E.$x \in (-\infty,-13] \cup (-1,8]$\\
F.$x \in (-\infty,-13] \cup [-1,8)$\\
G.$x \in (-\infty,-13) \cup [-1,8]$\\
H.$x \in (-\infty,-13] \cup [-1,8]$
\testStop
\kluczStart
A
\kluczStop



\zadStart{Zadanie z Wikieł Z 1.62 b) moja wersja nr 243}

Rozwiązać nierówności $(x+13)(8-x)(x+2)\ge0$.
\zadStop
\rozwStart{Patryk Wirkus}{}
Miejsca zerowe naszego wielomianu to: $-13, 8, -2$.\\
Wielomian jest stopnia nieparzystego, ponadto znak współczynnika przy\linebreak najwyższej potędze x jest ujemny.\\ W związku z tym wykres wielomianu zaczyna się od lewej strony powyżej osi OX. A więc $$x \in (-\infty,-13) \cup (-2,8).$$
\rozwStop
\odpStart
$x \in (-\infty,-13) \cup (-2,8)$
\odpStop
\testStart
A.$x \in (-\infty,-13) \cup (-2,8)$\\
B.$x \in (-\infty,-13) \cup (-2,8]$\\
C.$x \in (-\infty,-13) \cup [-2,8)$\\
D.$x \in (-\infty,-13] \cup (-2,8)$\\
E.$x \in (-\infty,-13] \cup (-2,8]$\\
F.$x \in (-\infty,-13] \cup [-2,8)$\\
G.$x \in (-\infty,-13) \cup [-2,8]$\\
H.$x \in (-\infty,-13] \cup [-2,8]$
\testStop
\kluczStart
A
\kluczStop



\zadStart{Zadanie z Wikieł Z 1.62 b) moja wersja nr 244}

Rozwiązać nierówności $(x+13)(8-x)(x+3)\ge0$.
\zadStop
\rozwStart{Patryk Wirkus}{}
Miejsca zerowe naszego wielomianu to: $-13, 8, -3$.\\
Wielomian jest stopnia nieparzystego, ponadto znak współczynnika przy\linebreak najwyższej potędze x jest ujemny.\\ W związku z tym wykres wielomianu zaczyna się od lewej strony powyżej osi OX. A więc $$x \in (-\infty,-13) \cup (-3,8).$$
\rozwStop
\odpStart
$x \in (-\infty,-13) \cup (-3,8)$
\odpStop
\testStart
A.$x \in (-\infty,-13) \cup (-3,8)$\\
B.$x \in (-\infty,-13) \cup (-3,8]$\\
C.$x \in (-\infty,-13) \cup [-3,8)$\\
D.$x \in (-\infty,-13] \cup (-3,8)$\\
E.$x \in (-\infty,-13] \cup (-3,8]$\\
F.$x \in (-\infty,-13] \cup [-3,8)$\\
G.$x \in (-\infty,-13) \cup [-3,8]$\\
H.$x \in (-\infty,-13] \cup [-3,8]$
\testStop
\kluczStart
A
\kluczStop



\zadStart{Zadanie z Wikieł Z 1.62 b) moja wersja nr 245}

Rozwiązać nierówności $(x+13)(8-x)(x+4)\ge0$.
\zadStop
\rozwStart{Patryk Wirkus}{}
Miejsca zerowe naszego wielomianu to: $-13, 8, -4$.\\
Wielomian jest stopnia nieparzystego, ponadto znak współczynnika przy\linebreak najwyższej potędze x jest ujemny.\\ W związku z tym wykres wielomianu zaczyna się od lewej strony powyżej osi OX. A więc $$x \in (-\infty,-13) \cup (-4,8).$$
\rozwStop
\odpStart
$x \in (-\infty,-13) \cup (-4,8)$
\odpStop
\testStart
A.$x \in (-\infty,-13) \cup (-4,8)$\\
B.$x \in (-\infty,-13) \cup (-4,8]$\\
C.$x \in (-\infty,-13) \cup [-4,8)$\\
D.$x \in (-\infty,-13] \cup (-4,8)$\\
E.$x \in (-\infty,-13] \cup (-4,8]$\\
F.$x \in (-\infty,-13] \cup [-4,8)$\\
G.$x \in (-\infty,-13) \cup [-4,8]$\\
H.$x \in (-\infty,-13] \cup [-4,8]$
\testStop
\kluczStart
A
\kluczStop



\zadStart{Zadanie z Wikieł Z 1.62 b) moja wersja nr 246}

Rozwiązać nierówności $(x+13)(8-x)(x+5)\ge0$.
\zadStop
\rozwStart{Patryk Wirkus}{}
Miejsca zerowe naszego wielomianu to: $-13, 8, -5$.\\
Wielomian jest stopnia nieparzystego, ponadto znak współczynnika przy\linebreak najwyższej potędze x jest ujemny.\\ W związku z tym wykres wielomianu zaczyna się od lewej strony powyżej osi OX. A więc $$x \in (-\infty,-13) \cup (-5,8).$$
\rozwStop
\odpStart
$x \in (-\infty,-13) \cup (-5,8)$
\odpStop
\testStart
A.$x \in (-\infty,-13) \cup (-5,8)$\\
B.$x \in (-\infty,-13) \cup (-5,8]$\\
C.$x \in (-\infty,-13) \cup [-5,8)$\\
D.$x \in (-\infty,-13] \cup (-5,8)$\\
E.$x \in (-\infty,-13] \cup (-5,8]$\\
F.$x \in (-\infty,-13] \cup [-5,8)$\\
G.$x \in (-\infty,-13) \cup [-5,8]$\\
H.$x \in (-\infty,-13] \cup [-5,8]$
\testStop
\kluczStart
A
\kluczStop



\zadStart{Zadanie z Wikieł Z 1.62 b) moja wersja nr 247}

Rozwiązać nierówności $(x+13)(8-x)(x+6)\ge0$.
\zadStop
\rozwStart{Patryk Wirkus}{}
Miejsca zerowe naszego wielomianu to: $-13, 8, -6$.\\
Wielomian jest stopnia nieparzystego, ponadto znak współczynnika przy\linebreak najwyższej potędze x jest ujemny.\\ W związku z tym wykres wielomianu zaczyna się od lewej strony powyżej osi OX. A więc $$x \in (-\infty,-13) \cup (-6,8).$$
\rozwStop
\odpStart
$x \in (-\infty,-13) \cup (-6,8)$
\odpStop
\testStart
A.$x \in (-\infty,-13) \cup (-6,8)$\\
B.$x \in (-\infty,-13) \cup (-6,8]$\\
C.$x \in (-\infty,-13) \cup [-6,8)$\\
D.$x \in (-\infty,-13] \cup (-6,8)$\\
E.$x \in (-\infty,-13] \cup (-6,8]$\\
F.$x \in (-\infty,-13] \cup [-6,8)$\\
G.$x \in (-\infty,-13) \cup [-6,8]$\\
H.$x \in (-\infty,-13] \cup [-6,8]$
\testStop
\kluczStart
A
\kluczStop



\zadStart{Zadanie z Wikieł Z 1.62 b) moja wersja nr 248}

Rozwiązać nierówności $(x+13)(8-x)(x+7)\ge0$.
\zadStop
\rozwStart{Patryk Wirkus}{}
Miejsca zerowe naszego wielomianu to: $-13, 8, -7$.\\
Wielomian jest stopnia nieparzystego, ponadto znak współczynnika przy\linebreak najwyższej potędze x jest ujemny.\\ W związku z tym wykres wielomianu zaczyna się od lewej strony powyżej osi OX. A więc $$x \in (-\infty,-13) \cup (-7,8).$$
\rozwStop
\odpStart
$x \in (-\infty,-13) \cup (-7,8)$
\odpStop
\testStart
A.$x \in (-\infty,-13) \cup (-7,8)$\\
B.$x \in (-\infty,-13) \cup (-7,8]$\\
C.$x \in (-\infty,-13) \cup [-7,8)$\\
D.$x \in (-\infty,-13] \cup (-7,8)$\\
E.$x \in (-\infty,-13] \cup (-7,8]$\\
F.$x \in (-\infty,-13] \cup [-7,8)$\\
G.$x \in (-\infty,-13) \cup [-7,8]$\\
H.$x \in (-\infty,-13] \cup [-7,8]$
\testStop
\kluczStart
A
\kluczStop



\zadStart{Zadanie z Wikieł Z 1.62 b) moja wersja nr 249}

Rozwiązać nierówności $(x+13)(9-x)(x+1)\ge0$.
\zadStop
\rozwStart{Patryk Wirkus}{}
Miejsca zerowe naszego wielomianu to: $-13, 9, -1$.\\
Wielomian jest stopnia nieparzystego, ponadto znak współczynnika przy\linebreak najwyższej potędze x jest ujemny.\\ W związku z tym wykres wielomianu zaczyna się od lewej strony powyżej osi OX. A więc $$x \in (-\infty,-13) \cup (-1,9).$$
\rozwStop
\odpStart
$x \in (-\infty,-13) \cup (-1,9)$
\odpStop
\testStart
A.$x \in (-\infty,-13) \cup (-1,9)$\\
B.$x \in (-\infty,-13) \cup (-1,9]$\\
C.$x \in (-\infty,-13) \cup [-1,9)$\\
D.$x \in (-\infty,-13] \cup (-1,9)$\\
E.$x \in (-\infty,-13] \cup (-1,9]$\\
F.$x \in (-\infty,-13] \cup [-1,9)$\\
G.$x \in (-\infty,-13) \cup [-1,9]$\\
H.$x \in (-\infty,-13] \cup [-1,9]$
\testStop
\kluczStart
A
\kluczStop



\zadStart{Zadanie z Wikieł Z 1.62 b) moja wersja nr 250}

Rozwiązać nierówności $(x+13)(9-x)(x+2)\ge0$.
\zadStop
\rozwStart{Patryk Wirkus}{}
Miejsca zerowe naszego wielomianu to: $-13, 9, -2$.\\
Wielomian jest stopnia nieparzystego, ponadto znak współczynnika przy\linebreak najwyższej potędze x jest ujemny.\\ W związku z tym wykres wielomianu zaczyna się od lewej strony powyżej osi OX. A więc $$x \in (-\infty,-13) \cup (-2,9).$$
\rozwStop
\odpStart
$x \in (-\infty,-13) \cup (-2,9)$
\odpStop
\testStart
A.$x \in (-\infty,-13) \cup (-2,9)$\\
B.$x \in (-\infty,-13) \cup (-2,9]$\\
C.$x \in (-\infty,-13) \cup [-2,9)$\\
D.$x \in (-\infty,-13] \cup (-2,9)$\\
E.$x \in (-\infty,-13] \cup (-2,9]$\\
F.$x \in (-\infty,-13] \cup [-2,9)$\\
G.$x \in (-\infty,-13) \cup [-2,9]$\\
H.$x \in (-\infty,-13] \cup [-2,9]$
\testStop
\kluczStart
A
\kluczStop



\zadStart{Zadanie z Wikieł Z 1.62 b) moja wersja nr 251}

Rozwiązać nierówności $(x+13)(9-x)(x+3)\ge0$.
\zadStop
\rozwStart{Patryk Wirkus}{}
Miejsca zerowe naszego wielomianu to: $-13, 9, -3$.\\
Wielomian jest stopnia nieparzystego, ponadto znak współczynnika przy\linebreak najwyższej potędze x jest ujemny.\\ W związku z tym wykres wielomianu zaczyna się od lewej strony powyżej osi OX. A więc $$x \in (-\infty,-13) \cup (-3,9).$$
\rozwStop
\odpStart
$x \in (-\infty,-13) \cup (-3,9)$
\odpStop
\testStart
A.$x \in (-\infty,-13) \cup (-3,9)$\\
B.$x \in (-\infty,-13) \cup (-3,9]$\\
C.$x \in (-\infty,-13) \cup [-3,9)$\\
D.$x \in (-\infty,-13] \cup (-3,9)$\\
E.$x \in (-\infty,-13] \cup (-3,9]$\\
F.$x \in (-\infty,-13] \cup [-3,9)$\\
G.$x \in (-\infty,-13) \cup [-3,9]$\\
H.$x \in (-\infty,-13] \cup [-3,9]$
\testStop
\kluczStart
A
\kluczStop



\zadStart{Zadanie z Wikieł Z 1.62 b) moja wersja nr 252}

Rozwiązać nierówności $(x+13)(9-x)(x+4)\ge0$.
\zadStop
\rozwStart{Patryk Wirkus}{}
Miejsca zerowe naszego wielomianu to: $-13, 9, -4$.\\
Wielomian jest stopnia nieparzystego, ponadto znak współczynnika przy\linebreak najwyższej potędze x jest ujemny.\\ W związku z tym wykres wielomianu zaczyna się od lewej strony powyżej osi OX. A więc $$x \in (-\infty,-13) \cup (-4,9).$$
\rozwStop
\odpStart
$x \in (-\infty,-13) \cup (-4,9)$
\odpStop
\testStart
A.$x \in (-\infty,-13) \cup (-4,9)$\\
B.$x \in (-\infty,-13) \cup (-4,9]$\\
C.$x \in (-\infty,-13) \cup [-4,9)$\\
D.$x \in (-\infty,-13] \cup (-4,9)$\\
E.$x \in (-\infty,-13] \cup (-4,9]$\\
F.$x \in (-\infty,-13] \cup [-4,9)$\\
G.$x \in (-\infty,-13) \cup [-4,9]$\\
H.$x \in (-\infty,-13] \cup [-4,9]$
\testStop
\kluczStart
A
\kluczStop



\zadStart{Zadanie z Wikieł Z 1.62 b) moja wersja nr 253}

Rozwiązać nierówności $(x+13)(9-x)(x+5)\ge0$.
\zadStop
\rozwStart{Patryk Wirkus}{}
Miejsca zerowe naszego wielomianu to: $-13, 9, -5$.\\
Wielomian jest stopnia nieparzystego, ponadto znak współczynnika przy\linebreak najwyższej potędze x jest ujemny.\\ W związku z tym wykres wielomianu zaczyna się od lewej strony powyżej osi OX. A więc $$x \in (-\infty,-13) \cup (-5,9).$$
\rozwStop
\odpStart
$x \in (-\infty,-13) \cup (-5,9)$
\odpStop
\testStart
A.$x \in (-\infty,-13) \cup (-5,9)$\\
B.$x \in (-\infty,-13) \cup (-5,9]$\\
C.$x \in (-\infty,-13) \cup [-5,9)$\\
D.$x \in (-\infty,-13] \cup (-5,9)$\\
E.$x \in (-\infty,-13] \cup (-5,9]$\\
F.$x \in (-\infty,-13] \cup [-5,9)$\\
G.$x \in (-\infty,-13) \cup [-5,9]$\\
H.$x \in (-\infty,-13] \cup [-5,9]$
\testStop
\kluczStart
A
\kluczStop



\zadStart{Zadanie z Wikieł Z 1.62 b) moja wersja nr 254}

Rozwiązać nierówności $(x+13)(9-x)(x+6)\ge0$.
\zadStop
\rozwStart{Patryk Wirkus}{}
Miejsca zerowe naszego wielomianu to: $-13, 9, -6$.\\
Wielomian jest stopnia nieparzystego, ponadto znak współczynnika przy\linebreak najwyższej potędze x jest ujemny.\\ W związku z tym wykres wielomianu zaczyna się od lewej strony powyżej osi OX. A więc $$x \in (-\infty,-13) \cup (-6,9).$$
\rozwStop
\odpStart
$x \in (-\infty,-13) \cup (-6,9)$
\odpStop
\testStart
A.$x \in (-\infty,-13) \cup (-6,9)$\\
B.$x \in (-\infty,-13) \cup (-6,9]$\\
C.$x \in (-\infty,-13) \cup [-6,9)$\\
D.$x \in (-\infty,-13] \cup (-6,9)$\\
E.$x \in (-\infty,-13] \cup (-6,9]$\\
F.$x \in (-\infty,-13] \cup [-6,9)$\\
G.$x \in (-\infty,-13) \cup [-6,9]$\\
H.$x \in (-\infty,-13] \cup [-6,9]$
\testStop
\kluczStart
A
\kluczStop



\zadStart{Zadanie z Wikieł Z 1.62 b) moja wersja nr 255}

Rozwiązać nierówności $(x+13)(9-x)(x+7)\ge0$.
\zadStop
\rozwStart{Patryk Wirkus}{}
Miejsca zerowe naszego wielomianu to: $-13, 9, -7$.\\
Wielomian jest stopnia nieparzystego, ponadto znak współczynnika przy\linebreak najwyższej potędze x jest ujemny.\\ W związku z tym wykres wielomianu zaczyna się od lewej strony powyżej osi OX. A więc $$x \in (-\infty,-13) \cup (-7,9).$$
\rozwStop
\odpStart
$x \in (-\infty,-13) \cup (-7,9)$
\odpStop
\testStart
A.$x \in (-\infty,-13) \cup (-7,9)$\\
B.$x \in (-\infty,-13) \cup (-7,9]$\\
C.$x \in (-\infty,-13) \cup [-7,9)$\\
D.$x \in (-\infty,-13] \cup (-7,9)$\\
E.$x \in (-\infty,-13] \cup (-7,9]$\\
F.$x \in (-\infty,-13] \cup [-7,9)$\\
G.$x \in (-\infty,-13) \cup [-7,9]$\\
H.$x \in (-\infty,-13] \cup [-7,9]$
\testStop
\kluczStart
A
\kluczStop



\zadStart{Zadanie z Wikieł Z 1.62 b) moja wersja nr 256}

Rozwiązać nierówności $(x+13)(9-x)(x+8)\ge0$.
\zadStop
\rozwStart{Patryk Wirkus}{}
Miejsca zerowe naszego wielomianu to: $-13, 9, -8$.\\
Wielomian jest stopnia nieparzystego, ponadto znak współczynnika przy\linebreak najwyższej potędze x jest ujemny.\\ W związku z tym wykres wielomianu zaczyna się od lewej strony powyżej osi OX. A więc $$x \in (-\infty,-13) \cup (-8,9).$$
\rozwStop
\odpStart
$x \in (-\infty,-13) \cup (-8,9)$
\odpStop
\testStart
A.$x \in (-\infty,-13) \cup (-8,9)$\\
B.$x \in (-\infty,-13) \cup (-8,9]$\\
C.$x \in (-\infty,-13) \cup [-8,9)$\\
D.$x \in (-\infty,-13] \cup (-8,9)$\\
E.$x \in (-\infty,-13] \cup (-8,9]$\\
F.$x \in (-\infty,-13] \cup [-8,9)$\\
G.$x \in (-\infty,-13) \cup [-8,9]$\\
H.$x \in (-\infty,-13] \cup [-8,9]$
\testStop
\kluczStart
A
\kluczStop



\zadStart{Zadanie z Wikieł Z 1.62 b) moja wersja nr 257}

Rozwiązać nierówności $(x+13)(10-x)(x+1)\ge0$.
\zadStop
\rozwStart{Patryk Wirkus}{}
Miejsca zerowe naszego wielomianu to: $-13, 10, -1$.\\
Wielomian jest stopnia nieparzystego, ponadto znak współczynnika przy\linebreak najwyższej potędze x jest ujemny.\\ W związku z tym wykres wielomianu zaczyna się od lewej strony powyżej osi OX. A więc $$x \in (-\infty,-13) \cup (-1,10).$$
\rozwStop
\odpStart
$x \in (-\infty,-13) \cup (-1,10)$
\odpStop
\testStart
A.$x \in (-\infty,-13) \cup (-1,10)$\\
B.$x \in (-\infty,-13) \cup (-1,10]$\\
C.$x \in (-\infty,-13) \cup [-1,10)$\\
D.$x \in (-\infty,-13] \cup (-1,10)$\\
E.$x \in (-\infty,-13] \cup (-1,10]$\\
F.$x \in (-\infty,-13] \cup [-1,10)$\\
G.$x \in (-\infty,-13) \cup [-1,10]$\\
H.$x \in (-\infty,-13] \cup [-1,10]$
\testStop
\kluczStart
A
\kluczStop



\zadStart{Zadanie z Wikieł Z 1.62 b) moja wersja nr 258}

Rozwiązać nierówności $(x+13)(10-x)(x+2)\ge0$.
\zadStop
\rozwStart{Patryk Wirkus}{}
Miejsca zerowe naszego wielomianu to: $-13, 10, -2$.\\
Wielomian jest stopnia nieparzystego, ponadto znak współczynnika przy\linebreak najwyższej potędze x jest ujemny.\\ W związku z tym wykres wielomianu zaczyna się od lewej strony powyżej osi OX. A więc $$x \in (-\infty,-13) \cup (-2,10).$$
\rozwStop
\odpStart
$x \in (-\infty,-13) \cup (-2,10)$
\odpStop
\testStart
A.$x \in (-\infty,-13) \cup (-2,10)$\\
B.$x \in (-\infty,-13) \cup (-2,10]$\\
C.$x \in (-\infty,-13) \cup [-2,10)$\\
D.$x \in (-\infty,-13] \cup (-2,10)$\\
E.$x \in (-\infty,-13] \cup (-2,10]$\\
F.$x \in (-\infty,-13] \cup [-2,10)$\\
G.$x \in (-\infty,-13) \cup [-2,10]$\\
H.$x \in (-\infty,-13] \cup [-2,10]$
\testStop
\kluczStart
A
\kluczStop



\zadStart{Zadanie z Wikieł Z 1.62 b) moja wersja nr 259}

Rozwiązać nierówności $(x+13)(10-x)(x+3)\ge0$.
\zadStop
\rozwStart{Patryk Wirkus}{}
Miejsca zerowe naszego wielomianu to: $-13, 10, -3$.\\
Wielomian jest stopnia nieparzystego, ponadto znak współczynnika przy\linebreak najwyższej potędze x jest ujemny.\\ W związku z tym wykres wielomianu zaczyna się od lewej strony powyżej osi OX. A więc $$x \in (-\infty,-13) \cup (-3,10).$$
\rozwStop
\odpStart
$x \in (-\infty,-13) \cup (-3,10)$
\odpStop
\testStart
A.$x \in (-\infty,-13) \cup (-3,10)$\\
B.$x \in (-\infty,-13) \cup (-3,10]$\\
C.$x \in (-\infty,-13) \cup [-3,10)$\\
D.$x \in (-\infty,-13] \cup (-3,10)$\\
E.$x \in (-\infty,-13] \cup (-3,10]$\\
F.$x \in (-\infty,-13] \cup [-3,10)$\\
G.$x \in (-\infty,-13) \cup [-3,10]$\\
H.$x \in (-\infty,-13] \cup [-3,10]$
\testStop
\kluczStart
A
\kluczStop



\zadStart{Zadanie z Wikieł Z 1.62 b) moja wersja nr 260}

Rozwiązać nierówności $(x+13)(10-x)(x+4)\ge0$.
\zadStop
\rozwStart{Patryk Wirkus}{}
Miejsca zerowe naszego wielomianu to: $-13, 10, -4$.\\
Wielomian jest stopnia nieparzystego, ponadto znak współczynnika przy\linebreak najwyższej potędze x jest ujemny.\\ W związku z tym wykres wielomianu zaczyna się od lewej strony powyżej osi OX. A więc $$x \in (-\infty,-13) \cup (-4,10).$$
\rozwStop
\odpStart
$x \in (-\infty,-13) \cup (-4,10)$
\odpStop
\testStart
A.$x \in (-\infty,-13) \cup (-4,10)$\\
B.$x \in (-\infty,-13) \cup (-4,10]$\\
C.$x \in (-\infty,-13) \cup [-4,10)$\\
D.$x \in (-\infty,-13] \cup (-4,10)$\\
E.$x \in (-\infty,-13] \cup (-4,10]$\\
F.$x \in (-\infty,-13] \cup [-4,10)$\\
G.$x \in (-\infty,-13) \cup [-4,10]$\\
H.$x \in (-\infty,-13] \cup [-4,10]$
\testStop
\kluczStart
A
\kluczStop



\zadStart{Zadanie z Wikieł Z 1.62 b) moja wersja nr 261}

Rozwiązać nierówności $(x+13)(10-x)(x+5)\ge0$.
\zadStop
\rozwStart{Patryk Wirkus}{}
Miejsca zerowe naszego wielomianu to: $-13, 10, -5$.\\
Wielomian jest stopnia nieparzystego, ponadto znak współczynnika przy\linebreak najwyższej potędze x jest ujemny.\\ W związku z tym wykres wielomianu zaczyna się od lewej strony powyżej osi OX. A więc $$x \in (-\infty,-13) \cup (-5,10).$$
\rozwStop
\odpStart
$x \in (-\infty,-13) \cup (-5,10)$
\odpStop
\testStart
A.$x \in (-\infty,-13) \cup (-5,10)$\\
B.$x \in (-\infty,-13) \cup (-5,10]$\\
C.$x \in (-\infty,-13) \cup [-5,10)$\\
D.$x \in (-\infty,-13] \cup (-5,10)$\\
E.$x \in (-\infty,-13] \cup (-5,10]$\\
F.$x \in (-\infty,-13] \cup [-5,10)$\\
G.$x \in (-\infty,-13) \cup [-5,10]$\\
H.$x \in (-\infty,-13] \cup [-5,10]$
\testStop
\kluczStart
A
\kluczStop



\zadStart{Zadanie z Wikieł Z 1.62 b) moja wersja nr 262}

Rozwiązać nierówności $(x+13)(10-x)(x+6)\ge0$.
\zadStop
\rozwStart{Patryk Wirkus}{}
Miejsca zerowe naszego wielomianu to: $-13, 10, -6$.\\
Wielomian jest stopnia nieparzystego, ponadto znak współczynnika przy\linebreak najwyższej potędze x jest ujemny.\\ W związku z tym wykres wielomianu zaczyna się od lewej strony powyżej osi OX. A więc $$x \in (-\infty,-13) \cup (-6,10).$$
\rozwStop
\odpStart
$x \in (-\infty,-13) \cup (-6,10)$
\odpStop
\testStart
A.$x \in (-\infty,-13) \cup (-6,10)$\\
B.$x \in (-\infty,-13) \cup (-6,10]$\\
C.$x \in (-\infty,-13) \cup [-6,10)$\\
D.$x \in (-\infty,-13] \cup (-6,10)$\\
E.$x \in (-\infty,-13] \cup (-6,10]$\\
F.$x \in (-\infty,-13] \cup [-6,10)$\\
G.$x \in (-\infty,-13) \cup [-6,10]$\\
H.$x \in (-\infty,-13] \cup [-6,10]$
\testStop
\kluczStart
A
\kluczStop



\zadStart{Zadanie z Wikieł Z 1.62 b) moja wersja nr 263}

Rozwiązać nierówności $(x+13)(10-x)(x+7)\ge0$.
\zadStop
\rozwStart{Patryk Wirkus}{}
Miejsca zerowe naszego wielomianu to: $-13, 10, -7$.\\
Wielomian jest stopnia nieparzystego, ponadto znak współczynnika przy\linebreak najwyższej potędze x jest ujemny.\\ W związku z tym wykres wielomianu zaczyna się od lewej strony powyżej osi OX. A więc $$x \in (-\infty,-13) \cup (-7,10).$$
\rozwStop
\odpStart
$x \in (-\infty,-13) \cup (-7,10)$
\odpStop
\testStart
A.$x \in (-\infty,-13) \cup (-7,10)$\\
B.$x \in (-\infty,-13) \cup (-7,10]$\\
C.$x \in (-\infty,-13) \cup [-7,10)$\\
D.$x \in (-\infty,-13] \cup (-7,10)$\\
E.$x \in (-\infty,-13] \cup (-7,10]$\\
F.$x \in (-\infty,-13] \cup [-7,10)$\\
G.$x \in (-\infty,-13) \cup [-7,10]$\\
H.$x \in (-\infty,-13] \cup [-7,10]$
\testStop
\kluczStart
A
\kluczStop



\zadStart{Zadanie z Wikieł Z 1.62 b) moja wersja nr 264}

Rozwiązać nierówności $(x+13)(10-x)(x+8)\ge0$.
\zadStop
\rozwStart{Patryk Wirkus}{}
Miejsca zerowe naszego wielomianu to: $-13, 10, -8$.\\
Wielomian jest stopnia nieparzystego, ponadto znak współczynnika przy\linebreak najwyższej potędze x jest ujemny.\\ W związku z tym wykres wielomianu zaczyna się od lewej strony powyżej osi OX. A więc $$x \in (-\infty,-13) \cup (-8,10).$$
\rozwStop
\odpStart
$x \in (-\infty,-13) \cup (-8,10)$
\odpStop
\testStart
A.$x \in (-\infty,-13) \cup (-8,10)$\\
B.$x \in (-\infty,-13) \cup (-8,10]$\\
C.$x \in (-\infty,-13) \cup [-8,10)$\\
D.$x \in (-\infty,-13] \cup (-8,10)$\\
E.$x \in (-\infty,-13] \cup (-8,10]$\\
F.$x \in (-\infty,-13] \cup [-8,10)$\\
G.$x \in (-\infty,-13) \cup [-8,10]$\\
H.$x \in (-\infty,-13] \cup [-8,10]$
\testStop
\kluczStart
A
\kluczStop



\zadStart{Zadanie z Wikieł Z 1.62 b) moja wersja nr 265}

Rozwiązać nierówności $(x+13)(10-x)(x+9)\ge0$.
\zadStop
\rozwStart{Patryk Wirkus}{}
Miejsca zerowe naszego wielomianu to: $-13, 10, -9$.\\
Wielomian jest stopnia nieparzystego, ponadto znak współczynnika przy\linebreak najwyższej potędze x jest ujemny.\\ W związku z tym wykres wielomianu zaczyna się od lewej strony powyżej osi OX. A więc $$x \in (-\infty,-13) \cup (-9,10).$$
\rozwStop
\odpStart
$x \in (-\infty,-13) \cup (-9,10)$
\odpStop
\testStart
A.$x \in (-\infty,-13) \cup (-9,10)$\\
B.$x \in (-\infty,-13) \cup (-9,10]$\\
C.$x \in (-\infty,-13) \cup [-9,10)$\\
D.$x \in (-\infty,-13] \cup (-9,10)$\\
E.$x \in (-\infty,-13] \cup (-9,10]$\\
F.$x \in (-\infty,-13] \cup [-9,10)$\\
G.$x \in (-\infty,-13) \cup [-9,10]$\\
H.$x \in (-\infty,-13] \cup [-9,10]$
\testStop
\kluczStart
A
\kluczStop



\zadStart{Zadanie z Wikieł Z 1.62 b) moja wersja nr 266}

Rozwiązać nierówności $(x+13)(11-x)(x+1)\ge0$.
\zadStop
\rozwStart{Patryk Wirkus}{}
Miejsca zerowe naszego wielomianu to: $-13, 11, -1$.\\
Wielomian jest stopnia nieparzystego, ponadto znak współczynnika przy\linebreak najwyższej potędze x jest ujemny.\\ W związku z tym wykres wielomianu zaczyna się od lewej strony powyżej osi OX. A więc $$x \in (-\infty,-13) \cup (-1,11).$$
\rozwStop
\odpStart
$x \in (-\infty,-13) \cup (-1,11)$
\odpStop
\testStart
A.$x \in (-\infty,-13) \cup (-1,11)$\\
B.$x \in (-\infty,-13) \cup (-1,11]$\\
C.$x \in (-\infty,-13) \cup [-1,11)$\\
D.$x \in (-\infty,-13] \cup (-1,11)$\\
E.$x \in (-\infty,-13] \cup (-1,11]$\\
F.$x \in (-\infty,-13] \cup [-1,11)$\\
G.$x \in (-\infty,-13) \cup [-1,11]$\\
H.$x \in (-\infty,-13] \cup [-1,11]$
\testStop
\kluczStart
A
\kluczStop



\zadStart{Zadanie z Wikieł Z 1.62 b) moja wersja nr 267}

Rozwiązać nierówności $(x+13)(11-x)(x+2)\ge0$.
\zadStop
\rozwStart{Patryk Wirkus}{}
Miejsca zerowe naszego wielomianu to: $-13, 11, -2$.\\
Wielomian jest stopnia nieparzystego, ponadto znak współczynnika przy\linebreak najwyższej potędze x jest ujemny.\\ W związku z tym wykres wielomianu zaczyna się od lewej strony powyżej osi OX. A więc $$x \in (-\infty,-13) \cup (-2,11).$$
\rozwStop
\odpStart
$x \in (-\infty,-13) \cup (-2,11)$
\odpStop
\testStart
A.$x \in (-\infty,-13) \cup (-2,11)$\\
B.$x \in (-\infty,-13) \cup (-2,11]$\\
C.$x \in (-\infty,-13) \cup [-2,11)$\\
D.$x \in (-\infty,-13] \cup (-2,11)$\\
E.$x \in (-\infty,-13] \cup (-2,11]$\\
F.$x \in (-\infty,-13] \cup [-2,11)$\\
G.$x \in (-\infty,-13) \cup [-2,11]$\\
H.$x \in (-\infty,-13] \cup [-2,11]$
\testStop
\kluczStart
A
\kluczStop



\zadStart{Zadanie z Wikieł Z 1.62 b) moja wersja nr 268}

Rozwiązać nierówności $(x+13)(11-x)(x+3)\ge0$.
\zadStop
\rozwStart{Patryk Wirkus}{}
Miejsca zerowe naszego wielomianu to: $-13, 11, -3$.\\
Wielomian jest stopnia nieparzystego, ponadto znak współczynnika przy\linebreak najwyższej potędze x jest ujemny.\\ W związku z tym wykres wielomianu zaczyna się od lewej strony powyżej osi OX. A więc $$x \in (-\infty,-13) \cup (-3,11).$$
\rozwStop
\odpStart
$x \in (-\infty,-13) \cup (-3,11)$
\odpStop
\testStart
A.$x \in (-\infty,-13) \cup (-3,11)$\\
B.$x \in (-\infty,-13) \cup (-3,11]$\\
C.$x \in (-\infty,-13) \cup [-3,11)$\\
D.$x \in (-\infty,-13] \cup (-3,11)$\\
E.$x \in (-\infty,-13] \cup (-3,11]$\\
F.$x \in (-\infty,-13] \cup [-3,11)$\\
G.$x \in (-\infty,-13) \cup [-3,11]$\\
H.$x \in (-\infty,-13] \cup [-3,11]$
\testStop
\kluczStart
A
\kluczStop



\zadStart{Zadanie z Wikieł Z 1.62 b) moja wersja nr 269}

Rozwiązać nierówności $(x+13)(11-x)(x+4)\ge0$.
\zadStop
\rozwStart{Patryk Wirkus}{}
Miejsca zerowe naszego wielomianu to: $-13, 11, -4$.\\
Wielomian jest stopnia nieparzystego, ponadto znak współczynnika przy\linebreak najwyższej potędze x jest ujemny.\\ W związku z tym wykres wielomianu zaczyna się od lewej strony powyżej osi OX. A więc $$x \in (-\infty,-13) \cup (-4,11).$$
\rozwStop
\odpStart
$x \in (-\infty,-13) \cup (-4,11)$
\odpStop
\testStart
A.$x \in (-\infty,-13) \cup (-4,11)$\\
B.$x \in (-\infty,-13) \cup (-4,11]$\\
C.$x \in (-\infty,-13) \cup [-4,11)$\\
D.$x \in (-\infty,-13] \cup (-4,11)$\\
E.$x \in (-\infty,-13] \cup (-4,11]$\\
F.$x \in (-\infty,-13] \cup [-4,11)$\\
G.$x \in (-\infty,-13) \cup [-4,11]$\\
H.$x \in (-\infty,-13] \cup [-4,11]$
\testStop
\kluczStart
A
\kluczStop



\zadStart{Zadanie z Wikieł Z 1.62 b) moja wersja nr 270}

Rozwiązać nierówności $(x+13)(11-x)(x+5)\ge0$.
\zadStop
\rozwStart{Patryk Wirkus}{}
Miejsca zerowe naszego wielomianu to: $-13, 11, -5$.\\
Wielomian jest stopnia nieparzystego, ponadto znak współczynnika przy\linebreak najwyższej potędze x jest ujemny.\\ W związku z tym wykres wielomianu zaczyna się od lewej strony powyżej osi OX. A więc $$x \in (-\infty,-13) \cup (-5,11).$$
\rozwStop
\odpStart
$x \in (-\infty,-13) \cup (-5,11)$
\odpStop
\testStart
A.$x \in (-\infty,-13) \cup (-5,11)$\\
B.$x \in (-\infty,-13) \cup (-5,11]$\\
C.$x \in (-\infty,-13) \cup [-5,11)$\\
D.$x \in (-\infty,-13] \cup (-5,11)$\\
E.$x \in (-\infty,-13] \cup (-5,11]$\\
F.$x \in (-\infty,-13] \cup [-5,11)$\\
G.$x \in (-\infty,-13) \cup [-5,11]$\\
H.$x \in (-\infty,-13] \cup [-5,11]$
\testStop
\kluczStart
A
\kluczStop



\zadStart{Zadanie z Wikieł Z 1.62 b) moja wersja nr 271}

Rozwiązać nierówności $(x+13)(11-x)(x+6)\ge0$.
\zadStop
\rozwStart{Patryk Wirkus}{}
Miejsca zerowe naszego wielomianu to: $-13, 11, -6$.\\
Wielomian jest stopnia nieparzystego, ponadto znak współczynnika przy\linebreak najwyższej potędze x jest ujemny.\\ W związku z tym wykres wielomianu zaczyna się od lewej strony powyżej osi OX. A więc $$x \in (-\infty,-13) \cup (-6,11).$$
\rozwStop
\odpStart
$x \in (-\infty,-13) \cup (-6,11)$
\odpStop
\testStart
A.$x \in (-\infty,-13) \cup (-6,11)$\\
B.$x \in (-\infty,-13) \cup (-6,11]$\\
C.$x \in (-\infty,-13) \cup [-6,11)$\\
D.$x \in (-\infty,-13] \cup (-6,11)$\\
E.$x \in (-\infty,-13] \cup (-6,11]$\\
F.$x \in (-\infty,-13] \cup [-6,11)$\\
G.$x \in (-\infty,-13) \cup [-6,11]$\\
H.$x \in (-\infty,-13] \cup [-6,11]$
\testStop
\kluczStart
A
\kluczStop



\zadStart{Zadanie z Wikieł Z 1.62 b) moja wersja nr 272}

Rozwiązać nierówności $(x+13)(11-x)(x+7)\ge0$.
\zadStop
\rozwStart{Patryk Wirkus}{}
Miejsca zerowe naszego wielomianu to: $-13, 11, -7$.\\
Wielomian jest stopnia nieparzystego, ponadto znak współczynnika przy\linebreak najwyższej potędze x jest ujemny.\\ W związku z tym wykres wielomianu zaczyna się od lewej strony powyżej osi OX. A więc $$x \in (-\infty,-13) \cup (-7,11).$$
\rozwStop
\odpStart
$x \in (-\infty,-13) \cup (-7,11)$
\odpStop
\testStart
A.$x \in (-\infty,-13) \cup (-7,11)$\\
B.$x \in (-\infty,-13) \cup (-7,11]$\\
C.$x \in (-\infty,-13) \cup [-7,11)$\\
D.$x \in (-\infty,-13] \cup (-7,11)$\\
E.$x \in (-\infty,-13] \cup (-7,11]$\\
F.$x \in (-\infty,-13] \cup [-7,11)$\\
G.$x \in (-\infty,-13) \cup [-7,11]$\\
H.$x \in (-\infty,-13] \cup [-7,11]$
\testStop
\kluczStart
A
\kluczStop



\zadStart{Zadanie z Wikieł Z 1.62 b) moja wersja nr 273}

Rozwiązać nierówności $(x+13)(11-x)(x+8)\ge0$.
\zadStop
\rozwStart{Patryk Wirkus}{}
Miejsca zerowe naszego wielomianu to: $-13, 11, -8$.\\
Wielomian jest stopnia nieparzystego, ponadto znak współczynnika przy\linebreak najwyższej potędze x jest ujemny.\\ W związku z tym wykres wielomianu zaczyna się od lewej strony powyżej osi OX. A więc $$x \in (-\infty,-13) \cup (-8,11).$$
\rozwStop
\odpStart
$x \in (-\infty,-13) \cup (-8,11)$
\odpStop
\testStart
A.$x \in (-\infty,-13) \cup (-8,11)$\\
B.$x \in (-\infty,-13) \cup (-8,11]$\\
C.$x \in (-\infty,-13) \cup [-8,11)$\\
D.$x \in (-\infty,-13] \cup (-8,11)$\\
E.$x \in (-\infty,-13] \cup (-8,11]$\\
F.$x \in (-\infty,-13] \cup [-8,11)$\\
G.$x \in (-\infty,-13) \cup [-8,11]$\\
H.$x \in (-\infty,-13] \cup [-8,11]$
\testStop
\kluczStart
A
\kluczStop



\zadStart{Zadanie z Wikieł Z 1.62 b) moja wersja nr 274}

Rozwiązać nierówności $(x+13)(11-x)(x+9)\ge0$.
\zadStop
\rozwStart{Patryk Wirkus}{}
Miejsca zerowe naszego wielomianu to: $-13, 11, -9$.\\
Wielomian jest stopnia nieparzystego, ponadto znak współczynnika przy\linebreak najwyższej potędze x jest ujemny.\\ W związku z tym wykres wielomianu zaczyna się od lewej strony powyżej osi OX. A więc $$x \in (-\infty,-13) \cup (-9,11).$$
\rozwStop
\odpStart
$x \in (-\infty,-13) \cup (-9,11)$
\odpStop
\testStart
A.$x \in (-\infty,-13) \cup (-9,11)$\\
B.$x \in (-\infty,-13) \cup (-9,11]$\\
C.$x \in (-\infty,-13) \cup [-9,11)$\\
D.$x \in (-\infty,-13] \cup (-9,11)$\\
E.$x \in (-\infty,-13] \cup (-9,11]$\\
F.$x \in (-\infty,-13] \cup [-9,11)$\\
G.$x \in (-\infty,-13) \cup [-9,11]$\\
H.$x \in (-\infty,-13] \cup [-9,11]$
\testStop
\kluczStart
A
\kluczStop



\zadStart{Zadanie z Wikieł Z 1.62 b) moja wersja nr 275}

Rozwiązać nierówności $(x+13)(11-x)(x+10)\ge0$.
\zadStop
\rozwStart{Patryk Wirkus}{}
Miejsca zerowe naszego wielomianu to: $-13, 11, -10$.\\
Wielomian jest stopnia nieparzystego, ponadto znak współczynnika przy\linebreak najwyższej potędze x jest ujemny.\\ W związku z tym wykres wielomianu zaczyna się od lewej strony powyżej osi OX. A więc $$x \in (-\infty,-13) \cup (-10,11).$$
\rozwStop
\odpStart
$x \in (-\infty,-13) \cup (-10,11)$
\odpStop
\testStart
A.$x \in (-\infty,-13) \cup (-10,11)$\\
B.$x \in (-\infty,-13) \cup (-10,11]$\\
C.$x \in (-\infty,-13) \cup [-10,11)$\\
D.$x \in (-\infty,-13] \cup (-10,11)$\\
E.$x \in (-\infty,-13] \cup (-10,11]$\\
F.$x \in (-\infty,-13] \cup [-10,11)$\\
G.$x \in (-\infty,-13) \cup [-10,11]$\\
H.$x \in (-\infty,-13] \cup [-10,11]$
\testStop
\kluczStart
A
\kluczStop



\zadStart{Zadanie z Wikieł Z 1.62 b) moja wersja nr 276}

Rozwiązać nierówności $(x+13)(12-x)(x+1)\ge0$.
\zadStop
\rozwStart{Patryk Wirkus}{}
Miejsca zerowe naszego wielomianu to: $-13, 12, -1$.\\
Wielomian jest stopnia nieparzystego, ponadto znak współczynnika przy\linebreak najwyższej potędze x jest ujemny.\\ W związku z tym wykres wielomianu zaczyna się od lewej strony powyżej osi OX. A więc $$x \in (-\infty,-13) \cup (-1,12).$$
\rozwStop
\odpStart
$x \in (-\infty,-13) \cup (-1,12)$
\odpStop
\testStart
A.$x \in (-\infty,-13) \cup (-1,12)$\\
B.$x \in (-\infty,-13) \cup (-1,12]$\\
C.$x \in (-\infty,-13) \cup [-1,12)$\\
D.$x \in (-\infty,-13] \cup (-1,12)$\\
E.$x \in (-\infty,-13] \cup (-1,12]$\\
F.$x \in (-\infty,-13] \cup [-1,12)$\\
G.$x \in (-\infty,-13) \cup [-1,12]$\\
H.$x \in (-\infty,-13] \cup [-1,12]$
\testStop
\kluczStart
A
\kluczStop



\zadStart{Zadanie z Wikieł Z 1.62 b) moja wersja nr 277}

Rozwiązać nierówności $(x+13)(12-x)(x+2)\ge0$.
\zadStop
\rozwStart{Patryk Wirkus}{}
Miejsca zerowe naszego wielomianu to: $-13, 12, -2$.\\
Wielomian jest stopnia nieparzystego, ponadto znak współczynnika przy\linebreak najwyższej potędze x jest ujemny.\\ W związku z tym wykres wielomianu zaczyna się od lewej strony powyżej osi OX. A więc $$x \in (-\infty,-13) \cup (-2,12).$$
\rozwStop
\odpStart
$x \in (-\infty,-13) \cup (-2,12)$
\odpStop
\testStart
A.$x \in (-\infty,-13) \cup (-2,12)$\\
B.$x \in (-\infty,-13) \cup (-2,12]$\\
C.$x \in (-\infty,-13) \cup [-2,12)$\\
D.$x \in (-\infty,-13] \cup (-2,12)$\\
E.$x \in (-\infty,-13] \cup (-2,12]$\\
F.$x \in (-\infty,-13] \cup [-2,12)$\\
G.$x \in (-\infty,-13) \cup [-2,12]$\\
H.$x \in (-\infty,-13] \cup [-2,12]$
\testStop
\kluczStart
A
\kluczStop



\zadStart{Zadanie z Wikieł Z 1.62 b) moja wersja nr 278}

Rozwiązać nierówności $(x+13)(12-x)(x+3)\ge0$.
\zadStop
\rozwStart{Patryk Wirkus}{}
Miejsca zerowe naszego wielomianu to: $-13, 12, -3$.\\
Wielomian jest stopnia nieparzystego, ponadto znak współczynnika przy\linebreak najwyższej potędze x jest ujemny.\\ W związku z tym wykres wielomianu zaczyna się od lewej strony powyżej osi OX. A więc $$x \in (-\infty,-13) \cup (-3,12).$$
\rozwStop
\odpStart
$x \in (-\infty,-13) \cup (-3,12)$
\odpStop
\testStart
A.$x \in (-\infty,-13) \cup (-3,12)$\\
B.$x \in (-\infty,-13) \cup (-3,12]$\\
C.$x \in (-\infty,-13) \cup [-3,12)$\\
D.$x \in (-\infty,-13] \cup (-3,12)$\\
E.$x \in (-\infty,-13] \cup (-3,12]$\\
F.$x \in (-\infty,-13] \cup [-3,12)$\\
G.$x \in (-\infty,-13) \cup [-3,12]$\\
H.$x \in (-\infty,-13] \cup [-3,12]$
\testStop
\kluczStart
A
\kluczStop



\zadStart{Zadanie z Wikieł Z 1.62 b) moja wersja nr 279}

Rozwiązać nierówności $(x+13)(12-x)(x+4)\ge0$.
\zadStop
\rozwStart{Patryk Wirkus}{}
Miejsca zerowe naszego wielomianu to: $-13, 12, -4$.\\
Wielomian jest stopnia nieparzystego, ponadto znak współczynnika przy\linebreak najwyższej potędze x jest ujemny.\\ W związku z tym wykres wielomianu zaczyna się od lewej strony powyżej osi OX. A więc $$x \in (-\infty,-13) \cup (-4,12).$$
\rozwStop
\odpStart
$x \in (-\infty,-13) \cup (-4,12)$
\odpStop
\testStart
A.$x \in (-\infty,-13) \cup (-4,12)$\\
B.$x \in (-\infty,-13) \cup (-4,12]$\\
C.$x \in (-\infty,-13) \cup [-4,12)$\\
D.$x \in (-\infty,-13] \cup (-4,12)$\\
E.$x \in (-\infty,-13] \cup (-4,12]$\\
F.$x \in (-\infty,-13] \cup [-4,12)$\\
G.$x \in (-\infty,-13) \cup [-4,12]$\\
H.$x \in (-\infty,-13] \cup [-4,12]$
\testStop
\kluczStart
A
\kluczStop



\zadStart{Zadanie z Wikieł Z 1.62 b) moja wersja nr 280}

Rozwiązać nierówności $(x+13)(12-x)(x+5)\ge0$.
\zadStop
\rozwStart{Patryk Wirkus}{}
Miejsca zerowe naszego wielomianu to: $-13, 12, -5$.\\
Wielomian jest stopnia nieparzystego, ponadto znak współczynnika przy\linebreak najwyższej potędze x jest ujemny.\\ W związku z tym wykres wielomianu zaczyna się od lewej strony powyżej osi OX. A więc $$x \in (-\infty,-13) \cup (-5,12).$$
\rozwStop
\odpStart
$x \in (-\infty,-13) \cup (-5,12)$
\odpStop
\testStart
A.$x \in (-\infty,-13) \cup (-5,12)$\\
B.$x \in (-\infty,-13) \cup (-5,12]$\\
C.$x \in (-\infty,-13) \cup [-5,12)$\\
D.$x \in (-\infty,-13] \cup (-5,12)$\\
E.$x \in (-\infty,-13] \cup (-5,12]$\\
F.$x \in (-\infty,-13] \cup [-5,12)$\\
G.$x \in (-\infty,-13) \cup [-5,12]$\\
H.$x \in (-\infty,-13] \cup [-5,12]$
\testStop
\kluczStart
A
\kluczStop



\zadStart{Zadanie z Wikieł Z 1.62 b) moja wersja nr 281}

Rozwiązać nierówności $(x+13)(12-x)(x+6)\ge0$.
\zadStop
\rozwStart{Patryk Wirkus}{}
Miejsca zerowe naszego wielomianu to: $-13, 12, -6$.\\
Wielomian jest stopnia nieparzystego, ponadto znak współczynnika przy\linebreak najwyższej potędze x jest ujemny.\\ W związku z tym wykres wielomianu zaczyna się od lewej strony powyżej osi OX. A więc $$x \in (-\infty,-13) \cup (-6,12).$$
\rozwStop
\odpStart
$x \in (-\infty,-13) \cup (-6,12)$
\odpStop
\testStart
A.$x \in (-\infty,-13) \cup (-6,12)$\\
B.$x \in (-\infty,-13) \cup (-6,12]$\\
C.$x \in (-\infty,-13) \cup [-6,12)$\\
D.$x \in (-\infty,-13] \cup (-6,12)$\\
E.$x \in (-\infty,-13] \cup (-6,12]$\\
F.$x \in (-\infty,-13] \cup [-6,12)$\\
G.$x \in (-\infty,-13) \cup [-6,12]$\\
H.$x \in (-\infty,-13] \cup [-6,12]$
\testStop
\kluczStart
A
\kluczStop



\zadStart{Zadanie z Wikieł Z 1.62 b) moja wersja nr 282}

Rozwiązać nierówności $(x+13)(12-x)(x+7)\ge0$.
\zadStop
\rozwStart{Patryk Wirkus}{}
Miejsca zerowe naszego wielomianu to: $-13, 12, -7$.\\
Wielomian jest stopnia nieparzystego, ponadto znak współczynnika przy\linebreak najwyższej potędze x jest ujemny.\\ W związku z tym wykres wielomianu zaczyna się od lewej strony powyżej osi OX. A więc $$x \in (-\infty,-13) \cup (-7,12).$$
\rozwStop
\odpStart
$x \in (-\infty,-13) \cup (-7,12)$
\odpStop
\testStart
A.$x \in (-\infty,-13) \cup (-7,12)$\\
B.$x \in (-\infty,-13) \cup (-7,12]$\\
C.$x \in (-\infty,-13) \cup [-7,12)$\\
D.$x \in (-\infty,-13] \cup (-7,12)$\\
E.$x \in (-\infty,-13] \cup (-7,12]$\\
F.$x \in (-\infty,-13] \cup [-7,12)$\\
G.$x \in (-\infty,-13) \cup [-7,12]$\\
H.$x \in (-\infty,-13] \cup [-7,12]$
\testStop
\kluczStart
A
\kluczStop



\zadStart{Zadanie z Wikieł Z 1.62 b) moja wersja nr 283}

Rozwiązać nierówności $(x+13)(12-x)(x+8)\ge0$.
\zadStop
\rozwStart{Patryk Wirkus}{}
Miejsca zerowe naszego wielomianu to: $-13, 12, -8$.\\
Wielomian jest stopnia nieparzystego, ponadto znak współczynnika przy\linebreak najwyższej potędze x jest ujemny.\\ W związku z tym wykres wielomianu zaczyna się od lewej strony powyżej osi OX. A więc $$x \in (-\infty,-13) \cup (-8,12).$$
\rozwStop
\odpStart
$x \in (-\infty,-13) \cup (-8,12)$
\odpStop
\testStart
A.$x \in (-\infty,-13) \cup (-8,12)$\\
B.$x \in (-\infty,-13) \cup (-8,12]$\\
C.$x \in (-\infty,-13) \cup [-8,12)$\\
D.$x \in (-\infty,-13] \cup (-8,12)$\\
E.$x \in (-\infty,-13] \cup (-8,12]$\\
F.$x \in (-\infty,-13] \cup [-8,12)$\\
G.$x \in (-\infty,-13) \cup [-8,12]$\\
H.$x \in (-\infty,-13] \cup [-8,12]$
\testStop
\kluczStart
A
\kluczStop



\zadStart{Zadanie z Wikieł Z 1.62 b) moja wersja nr 284}

Rozwiązać nierówności $(x+13)(12-x)(x+9)\ge0$.
\zadStop
\rozwStart{Patryk Wirkus}{}
Miejsca zerowe naszego wielomianu to: $-13, 12, -9$.\\
Wielomian jest stopnia nieparzystego, ponadto znak współczynnika przy\linebreak najwyższej potędze x jest ujemny.\\ W związku z tym wykres wielomianu zaczyna się od lewej strony powyżej osi OX. A więc $$x \in (-\infty,-13) \cup (-9,12).$$
\rozwStop
\odpStart
$x \in (-\infty,-13) \cup (-9,12)$
\odpStop
\testStart
A.$x \in (-\infty,-13) \cup (-9,12)$\\
B.$x \in (-\infty,-13) \cup (-9,12]$\\
C.$x \in (-\infty,-13) \cup [-9,12)$\\
D.$x \in (-\infty,-13] \cup (-9,12)$\\
E.$x \in (-\infty,-13] \cup (-9,12]$\\
F.$x \in (-\infty,-13] \cup [-9,12)$\\
G.$x \in (-\infty,-13) \cup [-9,12]$\\
H.$x \in (-\infty,-13] \cup [-9,12]$
\testStop
\kluczStart
A
\kluczStop



\zadStart{Zadanie z Wikieł Z 1.62 b) moja wersja nr 285}

Rozwiązać nierówności $(x+13)(12-x)(x+10)\ge0$.
\zadStop
\rozwStart{Patryk Wirkus}{}
Miejsca zerowe naszego wielomianu to: $-13, 12, -10$.\\
Wielomian jest stopnia nieparzystego, ponadto znak współczynnika przy\linebreak najwyższej potędze x jest ujemny.\\ W związku z tym wykres wielomianu zaczyna się od lewej strony powyżej osi OX. A więc $$x \in (-\infty,-13) \cup (-10,12).$$
\rozwStop
\odpStart
$x \in (-\infty,-13) \cup (-10,12)$
\odpStop
\testStart
A.$x \in (-\infty,-13) \cup (-10,12)$\\
B.$x \in (-\infty,-13) \cup (-10,12]$\\
C.$x \in (-\infty,-13) \cup [-10,12)$\\
D.$x \in (-\infty,-13] \cup (-10,12)$\\
E.$x \in (-\infty,-13] \cup (-10,12]$\\
F.$x \in (-\infty,-13] \cup [-10,12)$\\
G.$x \in (-\infty,-13) \cup [-10,12]$\\
H.$x \in (-\infty,-13] \cup [-10,12]$
\testStop
\kluczStart
A
\kluczStop



\zadStart{Zadanie z Wikieł Z 1.62 b) moja wersja nr 286}

Rozwiązać nierówności $(x+13)(12-x)(x+11)\ge0$.
\zadStop
\rozwStart{Patryk Wirkus}{}
Miejsca zerowe naszego wielomianu to: $-13, 12, -11$.\\
Wielomian jest stopnia nieparzystego, ponadto znak współczynnika przy\linebreak najwyższej potędze x jest ujemny.\\ W związku z tym wykres wielomianu zaczyna się od lewej strony powyżej osi OX. A więc $$x \in (-\infty,-13) \cup (-11,12).$$
\rozwStop
\odpStart
$x \in (-\infty,-13) \cup (-11,12)$
\odpStop
\testStart
A.$x \in (-\infty,-13) \cup (-11,12)$\\
B.$x \in (-\infty,-13) \cup (-11,12]$\\
C.$x \in (-\infty,-13) \cup [-11,12)$\\
D.$x \in (-\infty,-13] \cup (-11,12)$\\
E.$x \in (-\infty,-13] \cup (-11,12]$\\
F.$x \in (-\infty,-13] \cup [-11,12)$\\
G.$x \in (-\infty,-13) \cup [-11,12]$\\
H.$x \in (-\infty,-13] \cup [-11,12]$
\testStop
\kluczStart
A
\kluczStop



\zadStart{Zadanie z Wikieł Z 1.62 b) moja wersja nr 287}

Rozwiązać nierówności $(x+14)(2-x)(x+1)\ge0$.
\zadStop
\rozwStart{Patryk Wirkus}{}
Miejsca zerowe naszego wielomianu to: $-14, 2, -1$.\\
Wielomian jest stopnia nieparzystego, ponadto znak współczynnika przy\linebreak najwyższej potędze x jest ujemny.\\ W związku z tym wykres wielomianu zaczyna się od lewej strony powyżej osi OX. A więc $$x \in (-\infty,-14) \cup (-1,2).$$
\rozwStop
\odpStart
$x \in (-\infty,-14) \cup (-1,2)$
\odpStop
\testStart
A.$x \in (-\infty,-14) \cup (-1,2)$\\
B.$x \in (-\infty,-14) \cup (-1,2]$\\
C.$x \in (-\infty,-14) \cup [-1,2)$\\
D.$x \in (-\infty,-14] \cup (-1,2)$\\
E.$x \in (-\infty,-14] \cup (-1,2]$\\
F.$x \in (-\infty,-14] \cup [-1,2)$\\
G.$x \in (-\infty,-14) \cup [-1,2]$\\
H.$x \in (-\infty,-14] \cup [-1,2]$
\testStop
\kluczStart
A
\kluczStop



\zadStart{Zadanie z Wikieł Z 1.62 b) moja wersja nr 288}

Rozwiązać nierówności $(x+14)(3-x)(x+1)\ge0$.
\zadStop
\rozwStart{Patryk Wirkus}{}
Miejsca zerowe naszego wielomianu to: $-14, 3, -1$.\\
Wielomian jest stopnia nieparzystego, ponadto znak współczynnika przy\linebreak najwyższej potędze x jest ujemny.\\ W związku z tym wykres wielomianu zaczyna się od lewej strony powyżej osi OX. A więc $$x \in (-\infty,-14) \cup (-1,3).$$
\rozwStop
\odpStart
$x \in (-\infty,-14) \cup (-1,3)$
\odpStop
\testStart
A.$x \in (-\infty,-14) \cup (-1,3)$\\
B.$x \in (-\infty,-14) \cup (-1,3]$\\
C.$x \in (-\infty,-14) \cup [-1,3)$\\
D.$x \in (-\infty,-14] \cup (-1,3)$\\
E.$x \in (-\infty,-14] \cup (-1,3]$\\
F.$x \in (-\infty,-14] \cup [-1,3)$\\
G.$x \in (-\infty,-14) \cup [-1,3]$\\
H.$x \in (-\infty,-14] \cup [-1,3]$
\testStop
\kluczStart
A
\kluczStop



\zadStart{Zadanie z Wikieł Z 1.62 b) moja wersja nr 289}

Rozwiązać nierówności $(x+14)(3-x)(x+2)\ge0$.
\zadStop
\rozwStart{Patryk Wirkus}{}
Miejsca zerowe naszego wielomianu to: $-14, 3, -2$.\\
Wielomian jest stopnia nieparzystego, ponadto znak współczynnika przy\linebreak najwyższej potędze x jest ujemny.\\ W związku z tym wykres wielomianu zaczyna się od lewej strony powyżej osi OX. A więc $$x \in (-\infty,-14) \cup (-2,3).$$
\rozwStop
\odpStart
$x \in (-\infty,-14) \cup (-2,3)$
\odpStop
\testStart
A.$x \in (-\infty,-14) \cup (-2,3)$\\
B.$x \in (-\infty,-14) \cup (-2,3]$\\
C.$x \in (-\infty,-14) \cup [-2,3)$\\
D.$x \in (-\infty,-14] \cup (-2,3)$\\
E.$x \in (-\infty,-14] \cup (-2,3]$\\
F.$x \in (-\infty,-14] \cup [-2,3)$\\
G.$x \in (-\infty,-14) \cup [-2,3]$\\
H.$x \in (-\infty,-14] \cup [-2,3]$
\testStop
\kluczStart
A
\kluczStop



\zadStart{Zadanie z Wikieł Z 1.62 b) moja wersja nr 290}

Rozwiązać nierówności $(x+14)(4-x)(x+1)\ge0$.
\zadStop
\rozwStart{Patryk Wirkus}{}
Miejsca zerowe naszego wielomianu to: $-14, 4, -1$.\\
Wielomian jest stopnia nieparzystego, ponadto znak współczynnika przy\linebreak najwyższej potędze x jest ujemny.\\ W związku z tym wykres wielomianu zaczyna się od lewej strony powyżej osi OX. A więc $$x \in (-\infty,-14) \cup (-1,4).$$
\rozwStop
\odpStart
$x \in (-\infty,-14) \cup (-1,4)$
\odpStop
\testStart
A.$x \in (-\infty,-14) \cup (-1,4)$\\
B.$x \in (-\infty,-14) \cup (-1,4]$\\
C.$x \in (-\infty,-14) \cup [-1,4)$\\
D.$x \in (-\infty,-14] \cup (-1,4)$\\
E.$x \in (-\infty,-14] \cup (-1,4]$\\
F.$x \in (-\infty,-14] \cup [-1,4)$\\
G.$x \in (-\infty,-14) \cup [-1,4]$\\
H.$x \in (-\infty,-14] \cup [-1,4]$
\testStop
\kluczStart
A
\kluczStop



\zadStart{Zadanie z Wikieł Z 1.62 b) moja wersja nr 291}

Rozwiązać nierówności $(x+14)(4-x)(x+2)\ge0$.
\zadStop
\rozwStart{Patryk Wirkus}{}
Miejsca zerowe naszego wielomianu to: $-14, 4, -2$.\\
Wielomian jest stopnia nieparzystego, ponadto znak współczynnika przy\linebreak najwyższej potędze x jest ujemny.\\ W związku z tym wykres wielomianu zaczyna się od lewej strony powyżej osi OX. A więc $$x \in (-\infty,-14) \cup (-2,4).$$
\rozwStop
\odpStart
$x \in (-\infty,-14) \cup (-2,4)$
\odpStop
\testStart
A.$x \in (-\infty,-14) \cup (-2,4)$\\
B.$x \in (-\infty,-14) \cup (-2,4]$\\
C.$x \in (-\infty,-14) \cup [-2,4)$\\
D.$x \in (-\infty,-14] \cup (-2,4)$\\
E.$x \in (-\infty,-14] \cup (-2,4]$\\
F.$x \in (-\infty,-14] \cup [-2,4)$\\
G.$x \in (-\infty,-14) \cup [-2,4]$\\
H.$x \in (-\infty,-14] \cup [-2,4]$
\testStop
\kluczStart
A
\kluczStop



\zadStart{Zadanie z Wikieł Z 1.62 b) moja wersja nr 292}

Rozwiązać nierówności $(x+14)(4-x)(x+3)\ge0$.
\zadStop
\rozwStart{Patryk Wirkus}{}
Miejsca zerowe naszego wielomianu to: $-14, 4, -3$.\\
Wielomian jest stopnia nieparzystego, ponadto znak współczynnika przy\linebreak najwyższej potędze x jest ujemny.\\ W związku z tym wykres wielomianu zaczyna się od lewej strony powyżej osi OX. A więc $$x \in (-\infty,-14) \cup (-3,4).$$
\rozwStop
\odpStart
$x \in (-\infty,-14) \cup (-3,4)$
\odpStop
\testStart
A.$x \in (-\infty,-14) \cup (-3,4)$\\
B.$x \in (-\infty,-14) \cup (-3,4]$\\
C.$x \in (-\infty,-14) \cup [-3,4)$\\
D.$x \in (-\infty,-14] \cup (-3,4)$\\
E.$x \in (-\infty,-14] \cup (-3,4]$\\
F.$x \in (-\infty,-14] \cup [-3,4)$\\
G.$x \in (-\infty,-14) \cup [-3,4]$\\
H.$x \in (-\infty,-14] \cup [-3,4]$
\testStop
\kluczStart
A
\kluczStop



\zadStart{Zadanie z Wikieł Z 1.62 b) moja wersja nr 293}

Rozwiązać nierówności $(x+14)(5-x)(x+1)\ge0$.
\zadStop
\rozwStart{Patryk Wirkus}{}
Miejsca zerowe naszego wielomianu to: $-14, 5, -1$.\\
Wielomian jest stopnia nieparzystego, ponadto znak współczynnika przy\linebreak najwyższej potędze x jest ujemny.\\ W związku z tym wykres wielomianu zaczyna się od lewej strony powyżej osi OX. A więc $$x \in (-\infty,-14) \cup (-1,5).$$
\rozwStop
\odpStart
$x \in (-\infty,-14) \cup (-1,5)$
\odpStop
\testStart
A.$x \in (-\infty,-14) \cup (-1,5)$\\
B.$x \in (-\infty,-14) \cup (-1,5]$\\
C.$x \in (-\infty,-14) \cup [-1,5)$\\
D.$x \in (-\infty,-14] \cup (-1,5)$\\
E.$x \in (-\infty,-14] \cup (-1,5]$\\
F.$x \in (-\infty,-14] \cup [-1,5)$\\
G.$x \in (-\infty,-14) \cup [-1,5]$\\
H.$x \in (-\infty,-14] \cup [-1,5]$
\testStop
\kluczStart
A
\kluczStop



\zadStart{Zadanie z Wikieł Z 1.62 b) moja wersja nr 294}

Rozwiązać nierówności $(x+14)(5-x)(x+2)\ge0$.
\zadStop
\rozwStart{Patryk Wirkus}{}
Miejsca zerowe naszego wielomianu to: $-14, 5, -2$.\\
Wielomian jest stopnia nieparzystego, ponadto znak współczynnika przy\linebreak najwyższej potędze x jest ujemny.\\ W związku z tym wykres wielomianu zaczyna się od lewej strony powyżej osi OX. A więc $$x \in (-\infty,-14) \cup (-2,5).$$
\rozwStop
\odpStart
$x \in (-\infty,-14) \cup (-2,5)$
\odpStop
\testStart
A.$x \in (-\infty,-14) \cup (-2,5)$\\
B.$x \in (-\infty,-14) \cup (-2,5]$\\
C.$x \in (-\infty,-14) \cup [-2,5)$\\
D.$x \in (-\infty,-14] \cup (-2,5)$\\
E.$x \in (-\infty,-14] \cup (-2,5]$\\
F.$x \in (-\infty,-14] \cup [-2,5)$\\
G.$x \in (-\infty,-14) \cup [-2,5]$\\
H.$x \in (-\infty,-14] \cup [-2,5]$
\testStop
\kluczStart
A
\kluczStop



\zadStart{Zadanie z Wikieł Z 1.62 b) moja wersja nr 295}

Rozwiązać nierówności $(x+14)(5-x)(x+3)\ge0$.
\zadStop
\rozwStart{Patryk Wirkus}{}
Miejsca zerowe naszego wielomianu to: $-14, 5, -3$.\\
Wielomian jest stopnia nieparzystego, ponadto znak współczynnika przy\linebreak najwyższej potędze x jest ujemny.\\ W związku z tym wykres wielomianu zaczyna się od lewej strony powyżej osi OX. A więc $$x \in (-\infty,-14) \cup (-3,5).$$
\rozwStop
\odpStart
$x \in (-\infty,-14) \cup (-3,5)$
\odpStop
\testStart
A.$x \in (-\infty,-14) \cup (-3,5)$\\
B.$x \in (-\infty,-14) \cup (-3,5]$\\
C.$x \in (-\infty,-14) \cup [-3,5)$\\
D.$x \in (-\infty,-14] \cup (-3,5)$\\
E.$x \in (-\infty,-14] \cup (-3,5]$\\
F.$x \in (-\infty,-14] \cup [-3,5)$\\
G.$x \in (-\infty,-14) \cup [-3,5]$\\
H.$x \in (-\infty,-14] \cup [-3,5]$
\testStop
\kluczStart
A
\kluczStop



\zadStart{Zadanie z Wikieł Z 1.62 b) moja wersja nr 296}

Rozwiązać nierówności $(x+14)(5-x)(x+4)\ge0$.
\zadStop
\rozwStart{Patryk Wirkus}{}
Miejsca zerowe naszego wielomianu to: $-14, 5, -4$.\\
Wielomian jest stopnia nieparzystego, ponadto znak współczynnika przy\linebreak najwyższej potędze x jest ujemny.\\ W związku z tym wykres wielomianu zaczyna się od lewej strony powyżej osi OX. A więc $$x \in (-\infty,-14) \cup (-4,5).$$
\rozwStop
\odpStart
$x \in (-\infty,-14) \cup (-4,5)$
\odpStop
\testStart
A.$x \in (-\infty,-14) \cup (-4,5)$\\
B.$x \in (-\infty,-14) \cup (-4,5]$\\
C.$x \in (-\infty,-14) \cup [-4,5)$\\
D.$x \in (-\infty,-14] \cup (-4,5)$\\
E.$x \in (-\infty,-14] \cup (-4,5]$\\
F.$x \in (-\infty,-14] \cup [-4,5)$\\
G.$x \in (-\infty,-14) \cup [-4,5]$\\
H.$x \in (-\infty,-14] \cup [-4,5]$
\testStop
\kluczStart
A
\kluczStop



\zadStart{Zadanie z Wikieł Z 1.62 b) moja wersja nr 297}

Rozwiązać nierówności $(x+14)(6-x)(x+1)\ge0$.
\zadStop
\rozwStart{Patryk Wirkus}{}
Miejsca zerowe naszego wielomianu to: $-14, 6, -1$.\\
Wielomian jest stopnia nieparzystego, ponadto znak współczynnika przy\linebreak najwyższej potędze x jest ujemny.\\ W związku z tym wykres wielomianu zaczyna się od lewej strony powyżej osi OX. A więc $$x \in (-\infty,-14) \cup (-1,6).$$
\rozwStop
\odpStart
$x \in (-\infty,-14) \cup (-1,6)$
\odpStop
\testStart
A.$x \in (-\infty,-14) \cup (-1,6)$\\
B.$x \in (-\infty,-14) \cup (-1,6]$\\
C.$x \in (-\infty,-14) \cup [-1,6)$\\
D.$x \in (-\infty,-14] \cup (-1,6)$\\
E.$x \in (-\infty,-14] \cup (-1,6]$\\
F.$x \in (-\infty,-14] \cup [-1,6)$\\
G.$x \in (-\infty,-14) \cup [-1,6]$\\
H.$x \in (-\infty,-14] \cup [-1,6]$
\testStop
\kluczStart
A
\kluczStop



\zadStart{Zadanie z Wikieł Z 1.62 b) moja wersja nr 298}

Rozwiązać nierówności $(x+14)(6-x)(x+2)\ge0$.
\zadStop
\rozwStart{Patryk Wirkus}{}
Miejsca zerowe naszego wielomianu to: $-14, 6, -2$.\\
Wielomian jest stopnia nieparzystego, ponadto znak współczynnika przy\linebreak najwyższej potędze x jest ujemny.\\ W związku z tym wykres wielomianu zaczyna się od lewej strony powyżej osi OX. A więc $$x \in (-\infty,-14) \cup (-2,6).$$
\rozwStop
\odpStart
$x \in (-\infty,-14) \cup (-2,6)$
\odpStop
\testStart
A.$x \in (-\infty,-14) \cup (-2,6)$\\
B.$x \in (-\infty,-14) \cup (-2,6]$\\
C.$x \in (-\infty,-14) \cup [-2,6)$\\
D.$x \in (-\infty,-14] \cup (-2,6)$\\
E.$x \in (-\infty,-14] \cup (-2,6]$\\
F.$x \in (-\infty,-14] \cup [-2,6)$\\
G.$x \in (-\infty,-14) \cup [-2,6]$\\
H.$x \in (-\infty,-14] \cup [-2,6]$
\testStop
\kluczStart
A
\kluczStop



\zadStart{Zadanie z Wikieł Z 1.62 b) moja wersja nr 299}

Rozwiązać nierówności $(x+14)(6-x)(x+3)\ge0$.
\zadStop
\rozwStart{Patryk Wirkus}{}
Miejsca zerowe naszego wielomianu to: $-14, 6, -3$.\\
Wielomian jest stopnia nieparzystego, ponadto znak współczynnika przy\linebreak najwyższej potędze x jest ujemny.\\ W związku z tym wykres wielomianu zaczyna się od lewej strony powyżej osi OX. A więc $$x \in (-\infty,-14) \cup (-3,6).$$
\rozwStop
\odpStart
$x \in (-\infty,-14) \cup (-3,6)$
\odpStop
\testStart
A.$x \in (-\infty,-14) \cup (-3,6)$\\
B.$x \in (-\infty,-14) \cup (-3,6]$\\
C.$x \in (-\infty,-14) \cup [-3,6)$\\
D.$x \in (-\infty,-14] \cup (-3,6)$\\
E.$x \in (-\infty,-14] \cup (-3,6]$\\
F.$x \in (-\infty,-14] \cup [-3,6)$\\
G.$x \in (-\infty,-14) \cup [-3,6]$\\
H.$x \in (-\infty,-14] \cup [-3,6]$
\testStop
\kluczStart
A
\kluczStop



\zadStart{Zadanie z Wikieł Z 1.62 b) moja wersja nr 300}

Rozwiązać nierówności $(x+14)(6-x)(x+4)\ge0$.
\zadStop
\rozwStart{Patryk Wirkus}{}
Miejsca zerowe naszego wielomianu to: $-14, 6, -4$.\\
Wielomian jest stopnia nieparzystego, ponadto znak współczynnika przy\linebreak najwyższej potędze x jest ujemny.\\ W związku z tym wykres wielomianu zaczyna się od lewej strony powyżej osi OX. A więc $$x \in (-\infty,-14) \cup (-4,6).$$
\rozwStop
\odpStart
$x \in (-\infty,-14) \cup (-4,6)$
\odpStop
\testStart
A.$x \in (-\infty,-14) \cup (-4,6)$\\
B.$x \in (-\infty,-14) \cup (-4,6]$\\
C.$x \in (-\infty,-14) \cup [-4,6)$\\
D.$x \in (-\infty,-14] \cup (-4,6)$\\
E.$x \in (-\infty,-14] \cup (-4,6]$\\
F.$x \in (-\infty,-14] \cup [-4,6)$\\
G.$x \in (-\infty,-14) \cup [-4,6]$\\
H.$x \in (-\infty,-14] \cup [-4,6]$
\testStop
\kluczStart
A
\kluczStop



\zadStart{Zadanie z Wikieł Z 1.62 b) moja wersja nr 301}

Rozwiązać nierówności $(x+14)(6-x)(x+5)\ge0$.
\zadStop
\rozwStart{Patryk Wirkus}{}
Miejsca zerowe naszego wielomianu to: $-14, 6, -5$.\\
Wielomian jest stopnia nieparzystego, ponadto znak współczynnika przy\linebreak najwyższej potędze x jest ujemny.\\ W związku z tym wykres wielomianu zaczyna się od lewej strony powyżej osi OX. A więc $$x \in (-\infty,-14) \cup (-5,6).$$
\rozwStop
\odpStart
$x \in (-\infty,-14) \cup (-5,6)$
\odpStop
\testStart
A.$x \in (-\infty,-14) \cup (-5,6)$\\
B.$x \in (-\infty,-14) \cup (-5,6]$\\
C.$x \in (-\infty,-14) \cup [-5,6)$\\
D.$x \in (-\infty,-14] \cup (-5,6)$\\
E.$x \in (-\infty,-14] \cup (-5,6]$\\
F.$x \in (-\infty,-14] \cup [-5,6)$\\
G.$x \in (-\infty,-14) \cup [-5,6]$\\
H.$x \in (-\infty,-14] \cup [-5,6]$
\testStop
\kluczStart
A
\kluczStop



\zadStart{Zadanie z Wikieł Z 1.62 b) moja wersja nr 302}

Rozwiązać nierówności $(x+14)(7-x)(x+1)\ge0$.
\zadStop
\rozwStart{Patryk Wirkus}{}
Miejsca zerowe naszego wielomianu to: $-14, 7, -1$.\\
Wielomian jest stopnia nieparzystego, ponadto znak współczynnika przy\linebreak najwyższej potędze x jest ujemny.\\ W związku z tym wykres wielomianu zaczyna się od lewej strony powyżej osi OX. A więc $$x \in (-\infty,-14) \cup (-1,7).$$
\rozwStop
\odpStart
$x \in (-\infty,-14) \cup (-1,7)$
\odpStop
\testStart
A.$x \in (-\infty,-14) \cup (-1,7)$\\
B.$x \in (-\infty,-14) \cup (-1,7]$\\
C.$x \in (-\infty,-14) \cup [-1,7)$\\
D.$x \in (-\infty,-14] \cup (-1,7)$\\
E.$x \in (-\infty,-14] \cup (-1,7]$\\
F.$x \in (-\infty,-14] \cup [-1,7)$\\
G.$x \in (-\infty,-14) \cup [-1,7]$\\
H.$x \in (-\infty,-14] \cup [-1,7]$
\testStop
\kluczStart
A
\kluczStop



\zadStart{Zadanie z Wikieł Z 1.62 b) moja wersja nr 303}

Rozwiązać nierówności $(x+14)(7-x)(x+2)\ge0$.
\zadStop
\rozwStart{Patryk Wirkus}{}
Miejsca zerowe naszego wielomianu to: $-14, 7, -2$.\\
Wielomian jest stopnia nieparzystego, ponadto znak współczynnika przy\linebreak najwyższej potędze x jest ujemny.\\ W związku z tym wykres wielomianu zaczyna się od lewej strony powyżej osi OX. A więc $$x \in (-\infty,-14) \cup (-2,7).$$
\rozwStop
\odpStart
$x \in (-\infty,-14) \cup (-2,7)$
\odpStop
\testStart
A.$x \in (-\infty,-14) \cup (-2,7)$\\
B.$x \in (-\infty,-14) \cup (-2,7]$\\
C.$x \in (-\infty,-14) \cup [-2,7)$\\
D.$x \in (-\infty,-14] \cup (-2,7)$\\
E.$x \in (-\infty,-14] \cup (-2,7]$\\
F.$x \in (-\infty,-14] \cup [-2,7)$\\
G.$x \in (-\infty,-14) \cup [-2,7]$\\
H.$x \in (-\infty,-14] \cup [-2,7]$
\testStop
\kluczStart
A
\kluczStop



\zadStart{Zadanie z Wikieł Z 1.62 b) moja wersja nr 304}

Rozwiązać nierówności $(x+14)(7-x)(x+3)\ge0$.
\zadStop
\rozwStart{Patryk Wirkus}{}
Miejsca zerowe naszego wielomianu to: $-14, 7, -3$.\\
Wielomian jest stopnia nieparzystego, ponadto znak współczynnika przy\linebreak najwyższej potędze x jest ujemny.\\ W związku z tym wykres wielomianu zaczyna się od lewej strony powyżej osi OX. A więc $$x \in (-\infty,-14) \cup (-3,7).$$
\rozwStop
\odpStart
$x \in (-\infty,-14) \cup (-3,7)$
\odpStop
\testStart
A.$x \in (-\infty,-14) \cup (-3,7)$\\
B.$x \in (-\infty,-14) \cup (-3,7]$\\
C.$x \in (-\infty,-14) \cup [-3,7)$\\
D.$x \in (-\infty,-14] \cup (-3,7)$\\
E.$x \in (-\infty,-14] \cup (-3,7]$\\
F.$x \in (-\infty,-14] \cup [-3,7)$\\
G.$x \in (-\infty,-14) \cup [-3,7]$\\
H.$x \in (-\infty,-14] \cup [-3,7]$
\testStop
\kluczStart
A
\kluczStop



\zadStart{Zadanie z Wikieł Z 1.62 b) moja wersja nr 305}

Rozwiązać nierówności $(x+14)(7-x)(x+4)\ge0$.
\zadStop
\rozwStart{Patryk Wirkus}{}
Miejsca zerowe naszego wielomianu to: $-14, 7, -4$.\\
Wielomian jest stopnia nieparzystego, ponadto znak współczynnika przy\linebreak najwyższej potędze x jest ujemny.\\ W związku z tym wykres wielomianu zaczyna się od lewej strony powyżej osi OX. A więc $$x \in (-\infty,-14) \cup (-4,7).$$
\rozwStop
\odpStart
$x \in (-\infty,-14) \cup (-4,7)$
\odpStop
\testStart
A.$x \in (-\infty,-14) \cup (-4,7)$\\
B.$x \in (-\infty,-14) \cup (-4,7]$\\
C.$x \in (-\infty,-14) \cup [-4,7)$\\
D.$x \in (-\infty,-14] \cup (-4,7)$\\
E.$x \in (-\infty,-14] \cup (-4,7]$\\
F.$x \in (-\infty,-14] \cup [-4,7)$\\
G.$x \in (-\infty,-14) \cup [-4,7]$\\
H.$x \in (-\infty,-14] \cup [-4,7]$
\testStop
\kluczStart
A
\kluczStop



\zadStart{Zadanie z Wikieł Z 1.62 b) moja wersja nr 306}

Rozwiązać nierówności $(x+14)(7-x)(x+5)\ge0$.
\zadStop
\rozwStart{Patryk Wirkus}{}
Miejsca zerowe naszego wielomianu to: $-14, 7, -5$.\\
Wielomian jest stopnia nieparzystego, ponadto znak współczynnika przy\linebreak najwyższej potędze x jest ujemny.\\ W związku z tym wykres wielomianu zaczyna się od lewej strony powyżej osi OX. A więc $$x \in (-\infty,-14) \cup (-5,7).$$
\rozwStop
\odpStart
$x \in (-\infty,-14) \cup (-5,7)$
\odpStop
\testStart
A.$x \in (-\infty,-14) \cup (-5,7)$\\
B.$x \in (-\infty,-14) \cup (-5,7]$\\
C.$x \in (-\infty,-14) \cup [-5,7)$\\
D.$x \in (-\infty,-14] \cup (-5,7)$\\
E.$x \in (-\infty,-14] \cup (-5,7]$\\
F.$x \in (-\infty,-14] \cup [-5,7)$\\
G.$x \in (-\infty,-14) \cup [-5,7]$\\
H.$x \in (-\infty,-14] \cup [-5,7]$
\testStop
\kluczStart
A
\kluczStop



\zadStart{Zadanie z Wikieł Z 1.62 b) moja wersja nr 307}

Rozwiązać nierówności $(x+14)(7-x)(x+6)\ge0$.
\zadStop
\rozwStart{Patryk Wirkus}{}
Miejsca zerowe naszego wielomianu to: $-14, 7, -6$.\\
Wielomian jest stopnia nieparzystego, ponadto znak współczynnika przy\linebreak najwyższej potędze x jest ujemny.\\ W związku z tym wykres wielomianu zaczyna się od lewej strony powyżej osi OX. A więc $$x \in (-\infty,-14) \cup (-6,7).$$
\rozwStop
\odpStart
$x \in (-\infty,-14) \cup (-6,7)$
\odpStop
\testStart
A.$x \in (-\infty,-14) \cup (-6,7)$\\
B.$x \in (-\infty,-14) \cup (-6,7]$\\
C.$x \in (-\infty,-14) \cup [-6,7)$\\
D.$x \in (-\infty,-14] \cup (-6,7)$\\
E.$x \in (-\infty,-14] \cup (-6,7]$\\
F.$x \in (-\infty,-14] \cup [-6,7)$\\
G.$x \in (-\infty,-14) \cup [-6,7]$\\
H.$x \in (-\infty,-14] \cup [-6,7]$
\testStop
\kluczStart
A
\kluczStop



\zadStart{Zadanie z Wikieł Z 1.62 b) moja wersja nr 308}

Rozwiązać nierówności $(x+14)(8-x)(x+1)\ge0$.
\zadStop
\rozwStart{Patryk Wirkus}{}
Miejsca zerowe naszego wielomianu to: $-14, 8, -1$.\\
Wielomian jest stopnia nieparzystego, ponadto znak współczynnika przy\linebreak najwyższej potędze x jest ujemny.\\ W związku z tym wykres wielomianu zaczyna się od lewej strony powyżej osi OX. A więc $$x \in (-\infty,-14) \cup (-1,8).$$
\rozwStop
\odpStart
$x \in (-\infty,-14) \cup (-1,8)$
\odpStop
\testStart
A.$x \in (-\infty,-14) \cup (-1,8)$\\
B.$x \in (-\infty,-14) \cup (-1,8]$\\
C.$x \in (-\infty,-14) \cup [-1,8)$\\
D.$x \in (-\infty,-14] \cup (-1,8)$\\
E.$x \in (-\infty,-14] \cup (-1,8]$\\
F.$x \in (-\infty,-14] \cup [-1,8)$\\
G.$x \in (-\infty,-14) \cup [-1,8]$\\
H.$x \in (-\infty,-14] \cup [-1,8]$
\testStop
\kluczStart
A
\kluczStop



\zadStart{Zadanie z Wikieł Z 1.62 b) moja wersja nr 309}

Rozwiązać nierówności $(x+14)(8-x)(x+2)\ge0$.
\zadStop
\rozwStart{Patryk Wirkus}{}
Miejsca zerowe naszego wielomianu to: $-14, 8, -2$.\\
Wielomian jest stopnia nieparzystego, ponadto znak współczynnika przy\linebreak najwyższej potędze x jest ujemny.\\ W związku z tym wykres wielomianu zaczyna się od lewej strony powyżej osi OX. A więc $$x \in (-\infty,-14) \cup (-2,8).$$
\rozwStop
\odpStart
$x \in (-\infty,-14) \cup (-2,8)$
\odpStop
\testStart
A.$x \in (-\infty,-14) \cup (-2,8)$\\
B.$x \in (-\infty,-14) \cup (-2,8]$\\
C.$x \in (-\infty,-14) \cup [-2,8)$\\
D.$x \in (-\infty,-14] \cup (-2,8)$\\
E.$x \in (-\infty,-14] \cup (-2,8]$\\
F.$x \in (-\infty,-14] \cup [-2,8)$\\
G.$x \in (-\infty,-14) \cup [-2,8]$\\
H.$x \in (-\infty,-14] \cup [-2,8]$
\testStop
\kluczStart
A
\kluczStop



\zadStart{Zadanie z Wikieł Z 1.62 b) moja wersja nr 310}

Rozwiązać nierówności $(x+14)(8-x)(x+3)\ge0$.
\zadStop
\rozwStart{Patryk Wirkus}{}
Miejsca zerowe naszego wielomianu to: $-14, 8, -3$.\\
Wielomian jest stopnia nieparzystego, ponadto znak współczynnika przy\linebreak najwyższej potędze x jest ujemny.\\ W związku z tym wykres wielomianu zaczyna się od lewej strony powyżej osi OX. A więc $$x \in (-\infty,-14) \cup (-3,8).$$
\rozwStop
\odpStart
$x \in (-\infty,-14) \cup (-3,8)$
\odpStop
\testStart
A.$x \in (-\infty,-14) \cup (-3,8)$\\
B.$x \in (-\infty,-14) \cup (-3,8]$\\
C.$x \in (-\infty,-14) \cup [-3,8)$\\
D.$x \in (-\infty,-14] \cup (-3,8)$\\
E.$x \in (-\infty,-14] \cup (-3,8]$\\
F.$x \in (-\infty,-14] \cup [-3,8)$\\
G.$x \in (-\infty,-14) \cup [-3,8]$\\
H.$x \in (-\infty,-14] \cup [-3,8]$
\testStop
\kluczStart
A
\kluczStop



\zadStart{Zadanie z Wikieł Z 1.62 b) moja wersja nr 311}

Rozwiązać nierówności $(x+14)(8-x)(x+4)\ge0$.
\zadStop
\rozwStart{Patryk Wirkus}{}
Miejsca zerowe naszego wielomianu to: $-14, 8, -4$.\\
Wielomian jest stopnia nieparzystego, ponadto znak współczynnika przy\linebreak najwyższej potędze x jest ujemny.\\ W związku z tym wykres wielomianu zaczyna się od lewej strony powyżej osi OX. A więc $$x \in (-\infty,-14) \cup (-4,8).$$
\rozwStop
\odpStart
$x \in (-\infty,-14) \cup (-4,8)$
\odpStop
\testStart
A.$x \in (-\infty,-14) \cup (-4,8)$\\
B.$x \in (-\infty,-14) \cup (-4,8]$\\
C.$x \in (-\infty,-14) \cup [-4,8)$\\
D.$x \in (-\infty,-14] \cup (-4,8)$\\
E.$x \in (-\infty,-14] \cup (-4,8]$\\
F.$x \in (-\infty,-14] \cup [-4,8)$\\
G.$x \in (-\infty,-14) \cup [-4,8]$\\
H.$x \in (-\infty,-14] \cup [-4,8]$
\testStop
\kluczStart
A
\kluczStop



\zadStart{Zadanie z Wikieł Z 1.62 b) moja wersja nr 312}

Rozwiązać nierówności $(x+14)(8-x)(x+5)\ge0$.
\zadStop
\rozwStart{Patryk Wirkus}{}
Miejsca zerowe naszego wielomianu to: $-14, 8, -5$.\\
Wielomian jest stopnia nieparzystego, ponadto znak współczynnika przy\linebreak najwyższej potędze x jest ujemny.\\ W związku z tym wykres wielomianu zaczyna się od lewej strony powyżej osi OX. A więc $$x \in (-\infty,-14) \cup (-5,8).$$
\rozwStop
\odpStart
$x \in (-\infty,-14) \cup (-5,8)$
\odpStop
\testStart
A.$x \in (-\infty,-14) \cup (-5,8)$\\
B.$x \in (-\infty,-14) \cup (-5,8]$\\
C.$x \in (-\infty,-14) \cup [-5,8)$\\
D.$x \in (-\infty,-14] \cup (-5,8)$\\
E.$x \in (-\infty,-14] \cup (-5,8]$\\
F.$x \in (-\infty,-14] \cup [-5,8)$\\
G.$x \in (-\infty,-14) \cup [-5,8]$\\
H.$x \in (-\infty,-14] \cup [-5,8]$
\testStop
\kluczStart
A
\kluczStop



\zadStart{Zadanie z Wikieł Z 1.62 b) moja wersja nr 313}

Rozwiązać nierówności $(x+14)(8-x)(x+6)\ge0$.
\zadStop
\rozwStart{Patryk Wirkus}{}
Miejsca zerowe naszego wielomianu to: $-14, 8, -6$.\\
Wielomian jest stopnia nieparzystego, ponadto znak współczynnika przy\linebreak najwyższej potędze x jest ujemny.\\ W związku z tym wykres wielomianu zaczyna się od lewej strony powyżej osi OX. A więc $$x \in (-\infty,-14) \cup (-6,8).$$
\rozwStop
\odpStart
$x \in (-\infty,-14) \cup (-6,8)$
\odpStop
\testStart
A.$x \in (-\infty,-14) \cup (-6,8)$\\
B.$x \in (-\infty,-14) \cup (-6,8]$\\
C.$x \in (-\infty,-14) \cup [-6,8)$\\
D.$x \in (-\infty,-14] \cup (-6,8)$\\
E.$x \in (-\infty,-14] \cup (-6,8]$\\
F.$x \in (-\infty,-14] \cup [-6,8)$\\
G.$x \in (-\infty,-14) \cup [-6,8]$\\
H.$x \in (-\infty,-14] \cup [-6,8]$
\testStop
\kluczStart
A
\kluczStop



\zadStart{Zadanie z Wikieł Z 1.62 b) moja wersja nr 314}

Rozwiązać nierówności $(x+14)(8-x)(x+7)\ge0$.
\zadStop
\rozwStart{Patryk Wirkus}{}
Miejsca zerowe naszego wielomianu to: $-14, 8, -7$.\\
Wielomian jest stopnia nieparzystego, ponadto znak współczynnika przy\linebreak najwyższej potędze x jest ujemny.\\ W związku z tym wykres wielomianu zaczyna się od lewej strony powyżej osi OX. A więc $$x \in (-\infty,-14) \cup (-7,8).$$
\rozwStop
\odpStart
$x \in (-\infty,-14) \cup (-7,8)$
\odpStop
\testStart
A.$x \in (-\infty,-14) \cup (-7,8)$\\
B.$x \in (-\infty,-14) \cup (-7,8]$\\
C.$x \in (-\infty,-14) \cup [-7,8)$\\
D.$x \in (-\infty,-14] \cup (-7,8)$\\
E.$x \in (-\infty,-14] \cup (-7,8]$\\
F.$x \in (-\infty,-14] \cup [-7,8)$\\
G.$x \in (-\infty,-14) \cup [-7,8]$\\
H.$x \in (-\infty,-14] \cup [-7,8]$
\testStop
\kluczStart
A
\kluczStop



\zadStart{Zadanie z Wikieł Z 1.62 b) moja wersja nr 315}

Rozwiązać nierówności $(x+14)(9-x)(x+1)\ge0$.
\zadStop
\rozwStart{Patryk Wirkus}{}
Miejsca zerowe naszego wielomianu to: $-14, 9, -1$.\\
Wielomian jest stopnia nieparzystego, ponadto znak współczynnika przy\linebreak najwyższej potędze x jest ujemny.\\ W związku z tym wykres wielomianu zaczyna się od lewej strony powyżej osi OX. A więc $$x \in (-\infty,-14) \cup (-1,9).$$
\rozwStop
\odpStart
$x \in (-\infty,-14) \cup (-1,9)$
\odpStop
\testStart
A.$x \in (-\infty,-14) \cup (-1,9)$\\
B.$x \in (-\infty,-14) \cup (-1,9]$\\
C.$x \in (-\infty,-14) \cup [-1,9)$\\
D.$x \in (-\infty,-14] \cup (-1,9)$\\
E.$x \in (-\infty,-14] \cup (-1,9]$\\
F.$x \in (-\infty,-14] \cup [-1,9)$\\
G.$x \in (-\infty,-14) \cup [-1,9]$\\
H.$x \in (-\infty,-14] \cup [-1,9]$
\testStop
\kluczStart
A
\kluczStop



\zadStart{Zadanie z Wikieł Z 1.62 b) moja wersja nr 316}

Rozwiązać nierówności $(x+14)(9-x)(x+2)\ge0$.
\zadStop
\rozwStart{Patryk Wirkus}{}
Miejsca zerowe naszego wielomianu to: $-14, 9, -2$.\\
Wielomian jest stopnia nieparzystego, ponadto znak współczynnika przy\linebreak najwyższej potędze x jest ujemny.\\ W związku z tym wykres wielomianu zaczyna się od lewej strony powyżej osi OX. A więc $$x \in (-\infty,-14) \cup (-2,9).$$
\rozwStop
\odpStart
$x \in (-\infty,-14) \cup (-2,9)$
\odpStop
\testStart
A.$x \in (-\infty,-14) \cup (-2,9)$\\
B.$x \in (-\infty,-14) \cup (-2,9]$\\
C.$x \in (-\infty,-14) \cup [-2,9)$\\
D.$x \in (-\infty,-14] \cup (-2,9)$\\
E.$x \in (-\infty,-14] \cup (-2,9]$\\
F.$x \in (-\infty,-14] \cup [-2,9)$\\
G.$x \in (-\infty,-14) \cup [-2,9]$\\
H.$x \in (-\infty,-14] \cup [-2,9]$
\testStop
\kluczStart
A
\kluczStop



\zadStart{Zadanie z Wikieł Z 1.62 b) moja wersja nr 317}

Rozwiązać nierówności $(x+14)(9-x)(x+3)\ge0$.
\zadStop
\rozwStart{Patryk Wirkus}{}
Miejsca zerowe naszego wielomianu to: $-14, 9, -3$.\\
Wielomian jest stopnia nieparzystego, ponadto znak współczynnika przy\linebreak najwyższej potędze x jest ujemny.\\ W związku z tym wykres wielomianu zaczyna się od lewej strony powyżej osi OX. A więc $$x \in (-\infty,-14) \cup (-3,9).$$
\rozwStop
\odpStart
$x \in (-\infty,-14) \cup (-3,9)$
\odpStop
\testStart
A.$x \in (-\infty,-14) \cup (-3,9)$\\
B.$x \in (-\infty,-14) \cup (-3,9]$\\
C.$x \in (-\infty,-14) \cup [-3,9)$\\
D.$x \in (-\infty,-14] \cup (-3,9)$\\
E.$x \in (-\infty,-14] \cup (-3,9]$\\
F.$x \in (-\infty,-14] \cup [-3,9)$\\
G.$x \in (-\infty,-14) \cup [-3,9]$\\
H.$x \in (-\infty,-14] \cup [-3,9]$
\testStop
\kluczStart
A
\kluczStop



\zadStart{Zadanie z Wikieł Z 1.62 b) moja wersja nr 318}

Rozwiązać nierówności $(x+14)(9-x)(x+4)\ge0$.
\zadStop
\rozwStart{Patryk Wirkus}{}
Miejsca zerowe naszego wielomianu to: $-14, 9, -4$.\\
Wielomian jest stopnia nieparzystego, ponadto znak współczynnika przy\linebreak najwyższej potędze x jest ujemny.\\ W związku z tym wykres wielomianu zaczyna się od lewej strony powyżej osi OX. A więc $$x \in (-\infty,-14) \cup (-4,9).$$
\rozwStop
\odpStart
$x \in (-\infty,-14) \cup (-4,9)$
\odpStop
\testStart
A.$x \in (-\infty,-14) \cup (-4,9)$\\
B.$x \in (-\infty,-14) \cup (-4,9]$\\
C.$x \in (-\infty,-14) \cup [-4,9)$\\
D.$x \in (-\infty,-14] \cup (-4,9)$\\
E.$x \in (-\infty,-14] \cup (-4,9]$\\
F.$x \in (-\infty,-14] \cup [-4,9)$\\
G.$x \in (-\infty,-14) \cup [-4,9]$\\
H.$x \in (-\infty,-14] \cup [-4,9]$
\testStop
\kluczStart
A
\kluczStop



\zadStart{Zadanie z Wikieł Z 1.62 b) moja wersja nr 319}

Rozwiązać nierówności $(x+14)(9-x)(x+5)\ge0$.
\zadStop
\rozwStart{Patryk Wirkus}{}
Miejsca zerowe naszego wielomianu to: $-14, 9, -5$.\\
Wielomian jest stopnia nieparzystego, ponadto znak współczynnika przy\linebreak najwyższej potędze x jest ujemny.\\ W związku z tym wykres wielomianu zaczyna się od lewej strony powyżej osi OX. A więc $$x \in (-\infty,-14) \cup (-5,9).$$
\rozwStop
\odpStart
$x \in (-\infty,-14) \cup (-5,9)$
\odpStop
\testStart
A.$x \in (-\infty,-14) \cup (-5,9)$\\
B.$x \in (-\infty,-14) \cup (-5,9]$\\
C.$x \in (-\infty,-14) \cup [-5,9)$\\
D.$x \in (-\infty,-14] \cup (-5,9)$\\
E.$x \in (-\infty,-14] \cup (-5,9]$\\
F.$x \in (-\infty,-14] \cup [-5,9)$\\
G.$x \in (-\infty,-14) \cup [-5,9]$\\
H.$x \in (-\infty,-14] \cup [-5,9]$
\testStop
\kluczStart
A
\kluczStop



\zadStart{Zadanie z Wikieł Z 1.62 b) moja wersja nr 320}

Rozwiązać nierówności $(x+14)(9-x)(x+6)\ge0$.
\zadStop
\rozwStart{Patryk Wirkus}{}
Miejsca zerowe naszego wielomianu to: $-14, 9, -6$.\\
Wielomian jest stopnia nieparzystego, ponadto znak współczynnika przy\linebreak najwyższej potędze x jest ujemny.\\ W związku z tym wykres wielomianu zaczyna się od lewej strony powyżej osi OX. A więc $$x \in (-\infty,-14) \cup (-6,9).$$
\rozwStop
\odpStart
$x \in (-\infty,-14) \cup (-6,9)$
\odpStop
\testStart
A.$x \in (-\infty,-14) \cup (-6,9)$\\
B.$x \in (-\infty,-14) \cup (-6,9]$\\
C.$x \in (-\infty,-14) \cup [-6,9)$\\
D.$x \in (-\infty,-14] \cup (-6,9)$\\
E.$x \in (-\infty,-14] \cup (-6,9]$\\
F.$x \in (-\infty,-14] \cup [-6,9)$\\
G.$x \in (-\infty,-14) \cup [-6,9]$\\
H.$x \in (-\infty,-14] \cup [-6,9]$
\testStop
\kluczStart
A
\kluczStop



\zadStart{Zadanie z Wikieł Z 1.62 b) moja wersja nr 321}

Rozwiązać nierówności $(x+14)(9-x)(x+7)\ge0$.
\zadStop
\rozwStart{Patryk Wirkus}{}
Miejsca zerowe naszego wielomianu to: $-14, 9, -7$.\\
Wielomian jest stopnia nieparzystego, ponadto znak współczynnika przy\linebreak najwyższej potędze x jest ujemny.\\ W związku z tym wykres wielomianu zaczyna się od lewej strony powyżej osi OX. A więc $$x \in (-\infty,-14) \cup (-7,9).$$
\rozwStop
\odpStart
$x \in (-\infty,-14) \cup (-7,9)$
\odpStop
\testStart
A.$x \in (-\infty,-14) \cup (-7,9)$\\
B.$x \in (-\infty,-14) \cup (-7,9]$\\
C.$x \in (-\infty,-14) \cup [-7,9)$\\
D.$x \in (-\infty,-14] \cup (-7,9)$\\
E.$x \in (-\infty,-14] \cup (-7,9]$\\
F.$x \in (-\infty,-14] \cup [-7,9)$\\
G.$x \in (-\infty,-14) \cup [-7,9]$\\
H.$x \in (-\infty,-14] \cup [-7,9]$
\testStop
\kluczStart
A
\kluczStop



\zadStart{Zadanie z Wikieł Z 1.62 b) moja wersja nr 322}

Rozwiązać nierówności $(x+14)(9-x)(x+8)\ge0$.
\zadStop
\rozwStart{Patryk Wirkus}{}
Miejsca zerowe naszego wielomianu to: $-14, 9, -8$.\\
Wielomian jest stopnia nieparzystego, ponadto znak współczynnika przy\linebreak najwyższej potędze x jest ujemny.\\ W związku z tym wykres wielomianu zaczyna się od lewej strony powyżej osi OX. A więc $$x \in (-\infty,-14) \cup (-8,9).$$
\rozwStop
\odpStart
$x \in (-\infty,-14) \cup (-8,9)$
\odpStop
\testStart
A.$x \in (-\infty,-14) \cup (-8,9)$\\
B.$x \in (-\infty,-14) \cup (-8,9]$\\
C.$x \in (-\infty,-14) \cup [-8,9)$\\
D.$x \in (-\infty,-14] \cup (-8,9)$\\
E.$x \in (-\infty,-14] \cup (-8,9]$\\
F.$x \in (-\infty,-14] \cup [-8,9)$\\
G.$x \in (-\infty,-14) \cup [-8,9]$\\
H.$x \in (-\infty,-14] \cup [-8,9]$
\testStop
\kluczStart
A
\kluczStop



\zadStart{Zadanie z Wikieł Z 1.62 b) moja wersja nr 323}

Rozwiązać nierówności $(x+14)(10-x)(x+1)\ge0$.
\zadStop
\rozwStart{Patryk Wirkus}{}
Miejsca zerowe naszego wielomianu to: $-14, 10, -1$.\\
Wielomian jest stopnia nieparzystego, ponadto znak współczynnika przy\linebreak najwyższej potędze x jest ujemny.\\ W związku z tym wykres wielomianu zaczyna się od lewej strony powyżej osi OX. A więc $$x \in (-\infty,-14) \cup (-1,10).$$
\rozwStop
\odpStart
$x \in (-\infty,-14) \cup (-1,10)$
\odpStop
\testStart
A.$x \in (-\infty,-14) \cup (-1,10)$\\
B.$x \in (-\infty,-14) \cup (-1,10]$\\
C.$x \in (-\infty,-14) \cup [-1,10)$\\
D.$x \in (-\infty,-14] \cup (-1,10)$\\
E.$x \in (-\infty,-14] \cup (-1,10]$\\
F.$x \in (-\infty,-14] \cup [-1,10)$\\
G.$x \in (-\infty,-14) \cup [-1,10]$\\
H.$x \in (-\infty,-14] \cup [-1,10]$
\testStop
\kluczStart
A
\kluczStop



\zadStart{Zadanie z Wikieł Z 1.62 b) moja wersja nr 324}

Rozwiązać nierówności $(x+14)(10-x)(x+2)\ge0$.
\zadStop
\rozwStart{Patryk Wirkus}{}
Miejsca zerowe naszego wielomianu to: $-14, 10, -2$.\\
Wielomian jest stopnia nieparzystego, ponadto znak współczynnika przy\linebreak najwyższej potędze x jest ujemny.\\ W związku z tym wykres wielomianu zaczyna się od lewej strony powyżej osi OX. A więc $$x \in (-\infty,-14) \cup (-2,10).$$
\rozwStop
\odpStart
$x \in (-\infty,-14) \cup (-2,10)$
\odpStop
\testStart
A.$x \in (-\infty,-14) \cup (-2,10)$\\
B.$x \in (-\infty,-14) \cup (-2,10]$\\
C.$x \in (-\infty,-14) \cup [-2,10)$\\
D.$x \in (-\infty,-14] \cup (-2,10)$\\
E.$x \in (-\infty,-14] \cup (-2,10]$\\
F.$x \in (-\infty,-14] \cup [-2,10)$\\
G.$x \in (-\infty,-14) \cup [-2,10]$\\
H.$x \in (-\infty,-14] \cup [-2,10]$
\testStop
\kluczStart
A
\kluczStop



\zadStart{Zadanie z Wikieł Z 1.62 b) moja wersja nr 325}

Rozwiązać nierówności $(x+14)(10-x)(x+3)\ge0$.
\zadStop
\rozwStart{Patryk Wirkus}{}
Miejsca zerowe naszego wielomianu to: $-14, 10, -3$.\\
Wielomian jest stopnia nieparzystego, ponadto znak współczynnika przy\linebreak najwyższej potędze x jest ujemny.\\ W związku z tym wykres wielomianu zaczyna się od lewej strony powyżej osi OX. A więc $$x \in (-\infty,-14) \cup (-3,10).$$
\rozwStop
\odpStart
$x \in (-\infty,-14) \cup (-3,10)$
\odpStop
\testStart
A.$x \in (-\infty,-14) \cup (-3,10)$\\
B.$x \in (-\infty,-14) \cup (-3,10]$\\
C.$x \in (-\infty,-14) \cup [-3,10)$\\
D.$x \in (-\infty,-14] \cup (-3,10)$\\
E.$x \in (-\infty,-14] \cup (-3,10]$\\
F.$x \in (-\infty,-14] \cup [-3,10)$\\
G.$x \in (-\infty,-14) \cup [-3,10]$\\
H.$x \in (-\infty,-14] \cup [-3,10]$
\testStop
\kluczStart
A
\kluczStop



\zadStart{Zadanie z Wikieł Z 1.62 b) moja wersja nr 326}

Rozwiązać nierówności $(x+14)(10-x)(x+4)\ge0$.
\zadStop
\rozwStart{Patryk Wirkus}{}
Miejsca zerowe naszego wielomianu to: $-14, 10, -4$.\\
Wielomian jest stopnia nieparzystego, ponadto znak współczynnika przy\linebreak najwyższej potędze x jest ujemny.\\ W związku z tym wykres wielomianu zaczyna się od lewej strony powyżej osi OX. A więc $$x \in (-\infty,-14) \cup (-4,10).$$
\rozwStop
\odpStart
$x \in (-\infty,-14) \cup (-4,10)$
\odpStop
\testStart
A.$x \in (-\infty,-14) \cup (-4,10)$\\
B.$x \in (-\infty,-14) \cup (-4,10]$\\
C.$x \in (-\infty,-14) \cup [-4,10)$\\
D.$x \in (-\infty,-14] \cup (-4,10)$\\
E.$x \in (-\infty,-14] \cup (-4,10]$\\
F.$x \in (-\infty,-14] \cup [-4,10)$\\
G.$x \in (-\infty,-14) \cup [-4,10]$\\
H.$x \in (-\infty,-14] \cup [-4,10]$
\testStop
\kluczStart
A
\kluczStop



\zadStart{Zadanie z Wikieł Z 1.62 b) moja wersja nr 327}

Rozwiązać nierówności $(x+14)(10-x)(x+5)\ge0$.
\zadStop
\rozwStart{Patryk Wirkus}{}
Miejsca zerowe naszego wielomianu to: $-14, 10, -5$.\\
Wielomian jest stopnia nieparzystego, ponadto znak współczynnika przy\linebreak najwyższej potędze x jest ujemny.\\ W związku z tym wykres wielomianu zaczyna się od lewej strony powyżej osi OX. A więc $$x \in (-\infty,-14) \cup (-5,10).$$
\rozwStop
\odpStart
$x \in (-\infty,-14) \cup (-5,10)$
\odpStop
\testStart
A.$x \in (-\infty,-14) \cup (-5,10)$\\
B.$x \in (-\infty,-14) \cup (-5,10]$\\
C.$x \in (-\infty,-14) \cup [-5,10)$\\
D.$x \in (-\infty,-14] \cup (-5,10)$\\
E.$x \in (-\infty,-14] \cup (-5,10]$\\
F.$x \in (-\infty,-14] \cup [-5,10)$\\
G.$x \in (-\infty,-14) \cup [-5,10]$\\
H.$x \in (-\infty,-14] \cup [-5,10]$
\testStop
\kluczStart
A
\kluczStop



\zadStart{Zadanie z Wikieł Z 1.62 b) moja wersja nr 328}

Rozwiązać nierówności $(x+14)(10-x)(x+6)\ge0$.
\zadStop
\rozwStart{Patryk Wirkus}{}
Miejsca zerowe naszego wielomianu to: $-14, 10, -6$.\\
Wielomian jest stopnia nieparzystego, ponadto znak współczynnika przy\linebreak najwyższej potędze x jest ujemny.\\ W związku z tym wykres wielomianu zaczyna się od lewej strony powyżej osi OX. A więc $$x \in (-\infty,-14) \cup (-6,10).$$
\rozwStop
\odpStart
$x \in (-\infty,-14) \cup (-6,10)$
\odpStop
\testStart
A.$x \in (-\infty,-14) \cup (-6,10)$\\
B.$x \in (-\infty,-14) \cup (-6,10]$\\
C.$x \in (-\infty,-14) \cup [-6,10)$\\
D.$x \in (-\infty,-14] \cup (-6,10)$\\
E.$x \in (-\infty,-14] \cup (-6,10]$\\
F.$x \in (-\infty,-14] \cup [-6,10)$\\
G.$x \in (-\infty,-14) \cup [-6,10]$\\
H.$x \in (-\infty,-14] \cup [-6,10]$
\testStop
\kluczStart
A
\kluczStop



\zadStart{Zadanie z Wikieł Z 1.62 b) moja wersja nr 329}

Rozwiązać nierówności $(x+14)(10-x)(x+7)\ge0$.
\zadStop
\rozwStart{Patryk Wirkus}{}
Miejsca zerowe naszego wielomianu to: $-14, 10, -7$.\\
Wielomian jest stopnia nieparzystego, ponadto znak współczynnika przy\linebreak najwyższej potędze x jest ujemny.\\ W związku z tym wykres wielomianu zaczyna się od lewej strony powyżej osi OX. A więc $$x \in (-\infty,-14) \cup (-7,10).$$
\rozwStop
\odpStart
$x \in (-\infty,-14) \cup (-7,10)$
\odpStop
\testStart
A.$x \in (-\infty,-14) \cup (-7,10)$\\
B.$x \in (-\infty,-14) \cup (-7,10]$\\
C.$x \in (-\infty,-14) \cup [-7,10)$\\
D.$x \in (-\infty,-14] \cup (-7,10)$\\
E.$x \in (-\infty,-14] \cup (-7,10]$\\
F.$x \in (-\infty,-14] \cup [-7,10)$\\
G.$x \in (-\infty,-14) \cup [-7,10]$\\
H.$x \in (-\infty,-14] \cup [-7,10]$
\testStop
\kluczStart
A
\kluczStop



\zadStart{Zadanie z Wikieł Z 1.62 b) moja wersja nr 330}

Rozwiązać nierówności $(x+14)(10-x)(x+8)\ge0$.
\zadStop
\rozwStart{Patryk Wirkus}{}
Miejsca zerowe naszego wielomianu to: $-14, 10, -8$.\\
Wielomian jest stopnia nieparzystego, ponadto znak współczynnika przy\linebreak najwyższej potędze x jest ujemny.\\ W związku z tym wykres wielomianu zaczyna się od lewej strony powyżej osi OX. A więc $$x \in (-\infty,-14) \cup (-8,10).$$
\rozwStop
\odpStart
$x \in (-\infty,-14) \cup (-8,10)$
\odpStop
\testStart
A.$x \in (-\infty,-14) \cup (-8,10)$\\
B.$x \in (-\infty,-14) \cup (-8,10]$\\
C.$x \in (-\infty,-14) \cup [-8,10)$\\
D.$x \in (-\infty,-14] \cup (-8,10)$\\
E.$x \in (-\infty,-14] \cup (-8,10]$\\
F.$x \in (-\infty,-14] \cup [-8,10)$\\
G.$x \in (-\infty,-14) \cup [-8,10]$\\
H.$x \in (-\infty,-14] \cup [-8,10]$
\testStop
\kluczStart
A
\kluczStop



\zadStart{Zadanie z Wikieł Z 1.62 b) moja wersja nr 331}

Rozwiązać nierówności $(x+14)(10-x)(x+9)\ge0$.
\zadStop
\rozwStart{Patryk Wirkus}{}
Miejsca zerowe naszego wielomianu to: $-14, 10, -9$.\\
Wielomian jest stopnia nieparzystego, ponadto znak współczynnika przy\linebreak najwyższej potędze x jest ujemny.\\ W związku z tym wykres wielomianu zaczyna się od lewej strony powyżej osi OX. A więc $$x \in (-\infty,-14) \cup (-9,10).$$
\rozwStop
\odpStart
$x \in (-\infty,-14) \cup (-9,10)$
\odpStop
\testStart
A.$x \in (-\infty,-14) \cup (-9,10)$\\
B.$x \in (-\infty,-14) \cup (-9,10]$\\
C.$x \in (-\infty,-14) \cup [-9,10)$\\
D.$x \in (-\infty,-14] \cup (-9,10)$\\
E.$x \in (-\infty,-14] \cup (-9,10]$\\
F.$x \in (-\infty,-14] \cup [-9,10)$\\
G.$x \in (-\infty,-14) \cup [-9,10]$\\
H.$x \in (-\infty,-14] \cup [-9,10]$
\testStop
\kluczStart
A
\kluczStop



\zadStart{Zadanie z Wikieł Z 1.62 b) moja wersja nr 332}

Rozwiązać nierówności $(x+14)(11-x)(x+1)\ge0$.
\zadStop
\rozwStart{Patryk Wirkus}{}
Miejsca zerowe naszego wielomianu to: $-14, 11, -1$.\\
Wielomian jest stopnia nieparzystego, ponadto znak współczynnika przy\linebreak najwyższej potędze x jest ujemny.\\ W związku z tym wykres wielomianu zaczyna się od lewej strony powyżej osi OX. A więc $$x \in (-\infty,-14) \cup (-1,11).$$
\rozwStop
\odpStart
$x \in (-\infty,-14) \cup (-1,11)$
\odpStop
\testStart
A.$x \in (-\infty,-14) \cup (-1,11)$\\
B.$x \in (-\infty,-14) \cup (-1,11]$\\
C.$x \in (-\infty,-14) \cup [-1,11)$\\
D.$x \in (-\infty,-14] \cup (-1,11)$\\
E.$x \in (-\infty,-14] \cup (-1,11]$\\
F.$x \in (-\infty,-14] \cup [-1,11)$\\
G.$x \in (-\infty,-14) \cup [-1,11]$\\
H.$x \in (-\infty,-14] \cup [-1,11]$
\testStop
\kluczStart
A
\kluczStop



\zadStart{Zadanie z Wikieł Z 1.62 b) moja wersja nr 333}

Rozwiązać nierówności $(x+14)(11-x)(x+2)\ge0$.
\zadStop
\rozwStart{Patryk Wirkus}{}
Miejsca zerowe naszego wielomianu to: $-14, 11, -2$.\\
Wielomian jest stopnia nieparzystego, ponadto znak współczynnika przy\linebreak najwyższej potędze x jest ujemny.\\ W związku z tym wykres wielomianu zaczyna się od lewej strony powyżej osi OX. A więc $$x \in (-\infty,-14) \cup (-2,11).$$
\rozwStop
\odpStart
$x \in (-\infty,-14) \cup (-2,11)$
\odpStop
\testStart
A.$x \in (-\infty,-14) \cup (-2,11)$\\
B.$x \in (-\infty,-14) \cup (-2,11]$\\
C.$x \in (-\infty,-14) \cup [-2,11)$\\
D.$x \in (-\infty,-14] \cup (-2,11)$\\
E.$x \in (-\infty,-14] \cup (-2,11]$\\
F.$x \in (-\infty,-14] \cup [-2,11)$\\
G.$x \in (-\infty,-14) \cup [-2,11]$\\
H.$x \in (-\infty,-14] \cup [-2,11]$
\testStop
\kluczStart
A
\kluczStop



\zadStart{Zadanie z Wikieł Z 1.62 b) moja wersja nr 334}

Rozwiązać nierówności $(x+14)(11-x)(x+3)\ge0$.
\zadStop
\rozwStart{Patryk Wirkus}{}
Miejsca zerowe naszego wielomianu to: $-14, 11, -3$.\\
Wielomian jest stopnia nieparzystego, ponadto znak współczynnika przy\linebreak najwyższej potędze x jest ujemny.\\ W związku z tym wykres wielomianu zaczyna się od lewej strony powyżej osi OX. A więc $$x \in (-\infty,-14) \cup (-3,11).$$
\rozwStop
\odpStart
$x \in (-\infty,-14) \cup (-3,11)$
\odpStop
\testStart
A.$x \in (-\infty,-14) \cup (-3,11)$\\
B.$x \in (-\infty,-14) \cup (-3,11]$\\
C.$x \in (-\infty,-14) \cup [-3,11)$\\
D.$x \in (-\infty,-14] \cup (-3,11)$\\
E.$x \in (-\infty,-14] \cup (-3,11]$\\
F.$x \in (-\infty,-14] \cup [-3,11)$\\
G.$x \in (-\infty,-14) \cup [-3,11]$\\
H.$x \in (-\infty,-14] \cup [-3,11]$
\testStop
\kluczStart
A
\kluczStop



\zadStart{Zadanie z Wikieł Z 1.62 b) moja wersja nr 335}

Rozwiązać nierówności $(x+14)(11-x)(x+4)\ge0$.
\zadStop
\rozwStart{Patryk Wirkus}{}
Miejsca zerowe naszego wielomianu to: $-14, 11, -4$.\\
Wielomian jest stopnia nieparzystego, ponadto znak współczynnika przy\linebreak najwyższej potędze x jest ujemny.\\ W związku z tym wykres wielomianu zaczyna się od lewej strony powyżej osi OX. A więc $$x \in (-\infty,-14) \cup (-4,11).$$
\rozwStop
\odpStart
$x \in (-\infty,-14) \cup (-4,11)$
\odpStop
\testStart
A.$x \in (-\infty,-14) \cup (-4,11)$\\
B.$x \in (-\infty,-14) \cup (-4,11]$\\
C.$x \in (-\infty,-14) \cup [-4,11)$\\
D.$x \in (-\infty,-14] \cup (-4,11)$\\
E.$x \in (-\infty,-14] \cup (-4,11]$\\
F.$x \in (-\infty,-14] \cup [-4,11)$\\
G.$x \in (-\infty,-14) \cup [-4,11]$\\
H.$x \in (-\infty,-14] \cup [-4,11]$
\testStop
\kluczStart
A
\kluczStop



\zadStart{Zadanie z Wikieł Z 1.62 b) moja wersja nr 336}

Rozwiązać nierówności $(x+14)(11-x)(x+5)\ge0$.
\zadStop
\rozwStart{Patryk Wirkus}{}
Miejsca zerowe naszego wielomianu to: $-14, 11, -5$.\\
Wielomian jest stopnia nieparzystego, ponadto znak współczynnika przy\linebreak najwyższej potędze x jest ujemny.\\ W związku z tym wykres wielomianu zaczyna się od lewej strony powyżej osi OX. A więc $$x \in (-\infty,-14) \cup (-5,11).$$
\rozwStop
\odpStart
$x \in (-\infty,-14) \cup (-5,11)$
\odpStop
\testStart
A.$x \in (-\infty,-14) \cup (-5,11)$\\
B.$x \in (-\infty,-14) \cup (-5,11]$\\
C.$x \in (-\infty,-14) \cup [-5,11)$\\
D.$x \in (-\infty,-14] \cup (-5,11)$\\
E.$x \in (-\infty,-14] \cup (-5,11]$\\
F.$x \in (-\infty,-14] \cup [-5,11)$\\
G.$x \in (-\infty,-14) \cup [-5,11]$\\
H.$x \in (-\infty,-14] \cup [-5,11]$
\testStop
\kluczStart
A
\kluczStop



\zadStart{Zadanie z Wikieł Z 1.62 b) moja wersja nr 337}

Rozwiązać nierówności $(x+14)(11-x)(x+6)\ge0$.
\zadStop
\rozwStart{Patryk Wirkus}{}
Miejsca zerowe naszego wielomianu to: $-14, 11, -6$.\\
Wielomian jest stopnia nieparzystego, ponadto znak współczynnika przy\linebreak najwyższej potędze x jest ujemny.\\ W związku z tym wykres wielomianu zaczyna się od lewej strony powyżej osi OX. A więc $$x \in (-\infty,-14) \cup (-6,11).$$
\rozwStop
\odpStart
$x \in (-\infty,-14) \cup (-6,11)$
\odpStop
\testStart
A.$x \in (-\infty,-14) \cup (-6,11)$\\
B.$x \in (-\infty,-14) \cup (-6,11]$\\
C.$x \in (-\infty,-14) \cup [-6,11)$\\
D.$x \in (-\infty,-14] \cup (-6,11)$\\
E.$x \in (-\infty,-14] \cup (-6,11]$\\
F.$x \in (-\infty,-14] \cup [-6,11)$\\
G.$x \in (-\infty,-14) \cup [-6,11]$\\
H.$x \in (-\infty,-14] \cup [-6,11]$
\testStop
\kluczStart
A
\kluczStop



\zadStart{Zadanie z Wikieł Z 1.62 b) moja wersja nr 338}

Rozwiązać nierówności $(x+14)(11-x)(x+7)\ge0$.
\zadStop
\rozwStart{Patryk Wirkus}{}
Miejsca zerowe naszego wielomianu to: $-14, 11, -7$.\\
Wielomian jest stopnia nieparzystego, ponadto znak współczynnika przy\linebreak najwyższej potędze x jest ujemny.\\ W związku z tym wykres wielomianu zaczyna się od lewej strony powyżej osi OX. A więc $$x \in (-\infty,-14) \cup (-7,11).$$
\rozwStop
\odpStart
$x \in (-\infty,-14) \cup (-7,11)$
\odpStop
\testStart
A.$x \in (-\infty,-14) \cup (-7,11)$\\
B.$x \in (-\infty,-14) \cup (-7,11]$\\
C.$x \in (-\infty,-14) \cup [-7,11)$\\
D.$x \in (-\infty,-14] \cup (-7,11)$\\
E.$x \in (-\infty,-14] \cup (-7,11]$\\
F.$x \in (-\infty,-14] \cup [-7,11)$\\
G.$x \in (-\infty,-14) \cup [-7,11]$\\
H.$x \in (-\infty,-14] \cup [-7,11]$
\testStop
\kluczStart
A
\kluczStop



\zadStart{Zadanie z Wikieł Z 1.62 b) moja wersja nr 339}

Rozwiązać nierówności $(x+14)(11-x)(x+8)\ge0$.
\zadStop
\rozwStart{Patryk Wirkus}{}
Miejsca zerowe naszego wielomianu to: $-14, 11, -8$.\\
Wielomian jest stopnia nieparzystego, ponadto znak współczynnika przy\linebreak najwyższej potędze x jest ujemny.\\ W związku z tym wykres wielomianu zaczyna się od lewej strony powyżej osi OX. A więc $$x \in (-\infty,-14) \cup (-8,11).$$
\rozwStop
\odpStart
$x \in (-\infty,-14) \cup (-8,11)$
\odpStop
\testStart
A.$x \in (-\infty,-14) \cup (-8,11)$\\
B.$x \in (-\infty,-14) \cup (-8,11]$\\
C.$x \in (-\infty,-14) \cup [-8,11)$\\
D.$x \in (-\infty,-14] \cup (-8,11)$\\
E.$x \in (-\infty,-14] \cup (-8,11]$\\
F.$x \in (-\infty,-14] \cup [-8,11)$\\
G.$x \in (-\infty,-14) \cup [-8,11]$\\
H.$x \in (-\infty,-14] \cup [-8,11]$
\testStop
\kluczStart
A
\kluczStop



\zadStart{Zadanie z Wikieł Z 1.62 b) moja wersja nr 340}

Rozwiązać nierówności $(x+14)(11-x)(x+9)\ge0$.
\zadStop
\rozwStart{Patryk Wirkus}{}
Miejsca zerowe naszego wielomianu to: $-14, 11, -9$.\\
Wielomian jest stopnia nieparzystego, ponadto znak współczynnika przy\linebreak najwyższej potędze x jest ujemny.\\ W związku z tym wykres wielomianu zaczyna się od lewej strony powyżej osi OX. A więc $$x \in (-\infty,-14) \cup (-9,11).$$
\rozwStop
\odpStart
$x \in (-\infty,-14) \cup (-9,11)$
\odpStop
\testStart
A.$x \in (-\infty,-14) \cup (-9,11)$\\
B.$x \in (-\infty,-14) \cup (-9,11]$\\
C.$x \in (-\infty,-14) \cup [-9,11)$\\
D.$x \in (-\infty,-14] \cup (-9,11)$\\
E.$x \in (-\infty,-14] \cup (-9,11]$\\
F.$x \in (-\infty,-14] \cup [-9,11)$\\
G.$x \in (-\infty,-14) \cup [-9,11]$\\
H.$x \in (-\infty,-14] \cup [-9,11]$
\testStop
\kluczStart
A
\kluczStop



\zadStart{Zadanie z Wikieł Z 1.62 b) moja wersja nr 341}

Rozwiązać nierówności $(x+14)(11-x)(x+10)\ge0$.
\zadStop
\rozwStart{Patryk Wirkus}{}
Miejsca zerowe naszego wielomianu to: $-14, 11, -10$.\\
Wielomian jest stopnia nieparzystego, ponadto znak współczynnika przy\linebreak najwyższej potędze x jest ujemny.\\ W związku z tym wykres wielomianu zaczyna się od lewej strony powyżej osi OX. A więc $$x \in (-\infty,-14) \cup (-10,11).$$
\rozwStop
\odpStart
$x \in (-\infty,-14) \cup (-10,11)$
\odpStop
\testStart
A.$x \in (-\infty,-14) \cup (-10,11)$\\
B.$x \in (-\infty,-14) \cup (-10,11]$\\
C.$x \in (-\infty,-14) \cup [-10,11)$\\
D.$x \in (-\infty,-14] \cup (-10,11)$\\
E.$x \in (-\infty,-14] \cup (-10,11]$\\
F.$x \in (-\infty,-14] \cup [-10,11)$\\
G.$x \in (-\infty,-14) \cup [-10,11]$\\
H.$x \in (-\infty,-14] \cup [-10,11]$
\testStop
\kluczStart
A
\kluczStop



\zadStart{Zadanie z Wikieł Z 1.62 b) moja wersja nr 342}

Rozwiązać nierówności $(x+14)(12-x)(x+1)\ge0$.
\zadStop
\rozwStart{Patryk Wirkus}{}
Miejsca zerowe naszego wielomianu to: $-14, 12, -1$.\\
Wielomian jest stopnia nieparzystego, ponadto znak współczynnika przy\linebreak najwyższej potędze x jest ujemny.\\ W związku z tym wykres wielomianu zaczyna się od lewej strony powyżej osi OX. A więc $$x \in (-\infty,-14) \cup (-1,12).$$
\rozwStop
\odpStart
$x \in (-\infty,-14) \cup (-1,12)$
\odpStop
\testStart
A.$x \in (-\infty,-14) \cup (-1,12)$\\
B.$x \in (-\infty,-14) \cup (-1,12]$\\
C.$x \in (-\infty,-14) \cup [-1,12)$\\
D.$x \in (-\infty,-14] \cup (-1,12)$\\
E.$x \in (-\infty,-14] \cup (-1,12]$\\
F.$x \in (-\infty,-14] \cup [-1,12)$\\
G.$x \in (-\infty,-14) \cup [-1,12]$\\
H.$x \in (-\infty,-14] \cup [-1,12]$
\testStop
\kluczStart
A
\kluczStop



\zadStart{Zadanie z Wikieł Z 1.62 b) moja wersja nr 343}

Rozwiązać nierówności $(x+14)(12-x)(x+2)\ge0$.
\zadStop
\rozwStart{Patryk Wirkus}{}
Miejsca zerowe naszego wielomianu to: $-14, 12, -2$.\\
Wielomian jest stopnia nieparzystego, ponadto znak współczynnika przy\linebreak najwyższej potędze x jest ujemny.\\ W związku z tym wykres wielomianu zaczyna się od lewej strony powyżej osi OX. A więc $$x \in (-\infty,-14) \cup (-2,12).$$
\rozwStop
\odpStart
$x \in (-\infty,-14) \cup (-2,12)$
\odpStop
\testStart
A.$x \in (-\infty,-14) \cup (-2,12)$\\
B.$x \in (-\infty,-14) \cup (-2,12]$\\
C.$x \in (-\infty,-14) \cup [-2,12)$\\
D.$x \in (-\infty,-14] \cup (-2,12)$\\
E.$x \in (-\infty,-14] \cup (-2,12]$\\
F.$x \in (-\infty,-14] \cup [-2,12)$\\
G.$x \in (-\infty,-14) \cup [-2,12]$\\
H.$x \in (-\infty,-14] \cup [-2,12]$
\testStop
\kluczStart
A
\kluczStop



\zadStart{Zadanie z Wikieł Z 1.62 b) moja wersja nr 344}

Rozwiązać nierówności $(x+14)(12-x)(x+3)\ge0$.
\zadStop
\rozwStart{Patryk Wirkus}{}
Miejsca zerowe naszego wielomianu to: $-14, 12, -3$.\\
Wielomian jest stopnia nieparzystego, ponadto znak współczynnika przy\linebreak najwyższej potędze x jest ujemny.\\ W związku z tym wykres wielomianu zaczyna się od lewej strony powyżej osi OX. A więc $$x \in (-\infty,-14) \cup (-3,12).$$
\rozwStop
\odpStart
$x \in (-\infty,-14) \cup (-3,12)$
\odpStop
\testStart
A.$x \in (-\infty,-14) \cup (-3,12)$\\
B.$x \in (-\infty,-14) \cup (-3,12]$\\
C.$x \in (-\infty,-14) \cup [-3,12)$\\
D.$x \in (-\infty,-14] \cup (-3,12)$\\
E.$x \in (-\infty,-14] \cup (-3,12]$\\
F.$x \in (-\infty,-14] \cup [-3,12)$\\
G.$x \in (-\infty,-14) \cup [-3,12]$\\
H.$x \in (-\infty,-14] \cup [-3,12]$
\testStop
\kluczStart
A
\kluczStop



\zadStart{Zadanie z Wikieł Z 1.62 b) moja wersja nr 345}

Rozwiązać nierówności $(x+14)(12-x)(x+4)\ge0$.
\zadStop
\rozwStart{Patryk Wirkus}{}
Miejsca zerowe naszego wielomianu to: $-14, 12, -4$.\\
Wielomian jest stopnia nieparzystego, ponadto znak współczynnika przy\linebreak najwyższej potędze x jest ujemny.\\ W związku z tym wykres wielomianu zaczyna się od lewej strony powyżej osi OX. A więc $$x \in (-\infty,-14) \cup (-4,12).$$
\rozwStop
\odpStart
$x \in (-\infty,-14) \cup (-4,12)$
\odpStop
\testStart
A.$x \in (-\infty,-14) \cup (-4,12)$\\
B.$x \in (-\infty,-14) \cup (-4,12]$\\
C.$x \in (-\infty,-14) \cup [-4,12)$\\
D.$x \in (-\infty,-14] \cup (-4,12)$\\
E.$x \in (-\infty,-14] \cup (-4,12]$\\
F.$x \in (-\infty,-14] \cup [-4,12)$\\
G.$x \in (-\infty,-14) \cup [-4,12]$\\
H.$x \in (-\infty,-14] \cup [-4,12]$
\testStop
\kluczStart
A
\kluczStop



\zadStart{Zadanie z Wikieł Z 1.62 b) moja wersja nr 346}

Rozwiązać nierówności $(x+14)(12-x)(x+5)\ge0$.
\zadStop
\rozwStart{Patryk Wirkus}{}
Miejsca zerowe naszego wielomianu to: $-14, 12, -5$.\\
Wielomian jest stopnia nieparzystego, ponadto znak współczynnika przy\linebreak najwyższej potędze x jest ujemny.\\ W związku z tym wykres wielomianu zaczyna się od lewej strony powyżej osi OX. A więc $$x \in (-\infty,-14) \cup (-5,12).$$
\rozwStop
\odpStart
$x \in (-\infty,-14) \cup (-5,12)$
\odpStop
\testStart
A.$x \in (-\infty,-14) \cup (-5,12)$\\
B.$x \in (-\infty,-14) \cup (-5,12]$\\
C.$x \in (-\infty,-14) \cup [-5,12)$\\
D.$x \in (-\infty,-14] \cup (-5,12)$\\
E.$x \in (-\infty,-14] \cup (-5,12]$\\
F.$x \in (-\infty,-14] \cup [-5,12)$\\
G.$x \in (-\infty,-14) \cup [-5,12]$\\
H.$x \in (-\infty,-14] \cup [-5,12]$
\testStop
\kluczStart
A
\kluczStop



\zadStart{Zadanie z Wikieł Z 1.62 b) moja wersja nr 347}

Rozwiązać nierówności $(x+14)(12-x)(x+6)\ge0$.
\zadStop
\rozwStart{Patryk Wirkus}{}
Miejsca zerowe naszego wielomianu to: $-14, 12, -6$.\\
Wielomian jest stopnia nieparzystego, ponadto znak współczynnika przy\linebreak najwyższej potędze x jest ujemny.\\ W związku z tym wykres wielomianu zaczyna się od lewej strony powyżej osi OX. A więc $$x \in (-\infty,-14) \cup (-6,12).$$
\rozwStop
\odpStart
$x \in (-\infty,-14) \cup (-6,12)$
\odpStop
\testStart
A.$x \in (-\infty,-14) \cup (-6,12)$\\
B.$x \in (-\infty,-14) \cup (-6,12]$\\
C.$x \in (-\infty,-14) \cup [-6,12)$\\
D.$x \in (-\infty,-14] \cup (-6,12)$\\
E.$x \in (-\infty,-14] \cup (-6,12]$\\
F.$x \in (-\infty,-14] \cup [-6,12)$\\
G.$x \in (-\infty,-14) \cup [-6,12]$\\
H.$x \in (-\infty,-14] \cup [-6,12]$
\testStop
\kluczStart
A
\kluczStop



\zadStart{Zadanie z Wikieł Z 1.62 b) moja wersja nr 348}

Rozwiązać nierówności $(x+14)(12-x)(x+7)\ge0$.
\zadStop
\rozwStart{Patryk Wirkus}{}
Miejsca zerowe naszego wielomianu to: $-14, 12, -7$.\\
Wielomian jest stopnia nieparzystego, ponadto znak współczynnika przy\linebreak najwyższej potędze x jest ujemny.\\ W związku z tym wykres wielomianu zaczyna się od lewej strony powyżej osi OX. A więc $$x \in (-\infty,-14) \cup (-7,12).$$
\rozwStop
\odpStart
$x \in (-\infty,-14) \cup (-7,12)$
\odpStop
\testStart
A.$x \in (-\infty,-14) \cup (-7,12)$\\
B.$x \in (-\infty,-14) \cup (-7,12]$\\
C.$x \in (-\infty,-14) \cup [-7,12)$\\
D.$x \in (-\infty,-14] \cup (-7,12)$\\
E.$x \in (-\infty,-14] \cup (-7,12]$\\
F.$x \in (-\infty,-14] \cup [-7,12)$\\
G.$x \in (-\infty,-14) \cup [-7,12]$\\
H.$x \in (-\infty,-14] \cup [-7,12]$
\testStop
\kluczStart
A
\kluczStop



\zadStart{Zadanie z Wikieł Z 1.62 b) moja wersja nr 349}

Rozwiązać nierówności $(x+14)(12-x)(x+8)\ge0$.
\zadStop
\rozwStart{Patryk Wirkus}{}
Miejsca zerowe naszego wielomianu to: $-14, 12, -8$.\\
Wielomian jest stopnia nieparzystego, ponadto znak współczynnika przy\linebreak najwyższej potędze x jest ujemny.\\ W związku z tym wykres wielomianu zaczyna się od lewej strony powyżej osi OX. A więc $$x \in (-\infty,-14) \cup (-8,12).$$
\rozwStop
\odpStart
$x \in (-\infty,-14) \cup (-8,12)$
\odpStop
\testStart
A.$x \in (-\infty,-14) \cup (-8,12)$\\
B.$x \in (-\infty,-14) \cup (-8,12]$\\
C.$x \in (-\infty,-14) \cup [-8,12)$\\
D.$x \in (-\infty,-14] \cup (-8,12)$\\
E.$x \in (-\infty,-14] \cup (-8,12]$\\
F.$x \in (-\infty,-14] \cup [-8,12)$\\
G.$x \in (-\infty,-14) \cup [-8,12]$\\
H.$x \in (-\infty,-14] \cup [-8,12]$
\testStop
\kluczStart
A
\kluczStop



\zadStart{Zadanie z Wikieł Z 1.62 b) moja wersja nr 350}

Rozwiązać nierówności $(x+14)(12-x)(x+9)\ge0$.
\zadStop
\rozwStart{Patryk Wirkus}{}
Miejsca zerowe naszego wielomianu to: $-14, 12, -9$.\\
Wielomian jest stopnia nieparzystego, ponadto znak współczynnika przy\linebreak najwyższej potędze x jest ujemny.\\ W związku z tym wykres wielomianu zaczyna się od lewej strony powyżej osi OX. A więc $$x \in (-\infty,-14) \cup (-9,12).$$
\rozwStop
\odpStart
$x \in (-\infty,-14) \cup (-9,12)$
\odpStop
\testStart
A.$x \in (-\infty,-14) \cup (-9,12)$\\
B.$x \in (-\infty,-14) \cup (-9,12]$\\
C.$x \in (-\infty,-14) \cup [-9,12)$\\
D.$x \in (-\infty,-14] \cup (-9,12)$\\
E.$x \in (-\infty,-14] \cup (-9,12]$\\
F.$x \in (-\infty,-14] \cup [-9,12)$\\
G.$x \in (-\infty,-14) \cup [-9,12]$\\
H.$x \in (-\infty,-14] \cup [-9,12]$
\testStop
\kluczStart
A
\kluczStop



\zadStart{Zadanie z Wikieł Z 1.62 b) moja wersja nr 351}

Rozwiązać nierówności $(x+14)(12-x)(x+10)\ge0$.
\zadStop
\rozwStart{Patryk Wirkus}{}
Miejsca zerowe naszego wielomianu to: $-14, 12, -10$.\\
Wielomian jest stopnia nieparzystego, ponadto znak współczynnika przy\linebreak najwyższej potędze x jest ujemny.\\ W związku z tym wykres wielomianu zaczyna się od lewej strony powyżej osi OX. A więc $$x \in (-\infty,-14) \cup (-10,12).$$
\rozwStop
\odpStart
$x \in (-\infty,-14) \cup (-10,12)$
\odpStop
\testStart
A.$x \in (-\infty,-14) \cup (-10,12)$\\
B.$x \in (-\infty,-14) \cup (-10,12]$\\
C.$x \in (-\infty,-14) \cup [-10,12)$\\
D.$x \in (-\infty,-14] \cup (-10,12)$\\
E.$x \in (-\infty,-14] \cup (-10,12]$\\
F.$x \in (-\infty,-14] \cup [-10,12)$\\
G.$x \in (-\infty,-14) \cup [-10,12]$\\
H.$x \in (-\infty,-14] \cup [-10,12]$
\testStop
\kluczStart
A
\kluczStop



\zadStart{Zadanie z Wikieł Z 1.62 b) moja wersja nr 352}

Rozwiązać nierówności $(x+14)(12-x)(x+11)\ge0$.
\zadStop
\rozwStart{Patryk Wirkus}{}
Miejsca zerowe naszego wielomianu to: $-14, 12, -11$.\\
Wielomian jest stopnia nieparzystego, ponadto znak współczynnika przy\linebreak najwyższej potędze x jest ujemny.\\ W związku z tym wykres wielomianu zaczyna się od lewej strony powyżej osi OX. A więc $$x \in (-\infty,-14) \cup (-11,12).$$
\rozwStop
\odpStart
$x \in (-\infty,-14) \cup (-11,12)$
\odpStop
\testStart
A.$x \in (-\infty,-14) \cup (-11,12)$\\
B.$x \in (-\infty,-14) \cup (-11,12]$\\
C.$x \in (-\infty,-14) \cup [-11,12)$\\
D.$x \in (-\infty,-14] \cup (-11,12)$\\
E.$x \in (-\infty,-14] \cup (-11,12]$\\
F.$x \in (-\infty,-14] \cup [-11,12)$\\
G.$x \in (-\infty,-14) \cup [-11,12]$\\
H.$x \in (-\infty,-14] \cup [-11,12]$
\testStop
\kluczStart
A
\kluczStop



\zadStart{Zadanie z Wikieł Z 1.62 b) moja wersja nr 353}

Rozwiązać nierówności $(x+14)(13-x)(x+1)\ge0$.
\zadStop
\rozwStart{Patryk Wirkus}{}
Miejsca zerowe naszego wielomianu to: $-14, 13, -1$.\\
Wielomian jest stopnia nieparzystego, ponadto znak współczynnika przy\linebreak najwyższej potędze x jest ujemny.\\ W związku z tym wykres wielomianu zaczyna się od lewej strony powyżej osi OX. A więc $$x \in (-\infty,-14) \cup (-1,13).$$
\rozwStop
\odpStart
$x \in (-\infty,-14) \cup (-1,13)$
\odpStop
\testStart
A.$x \in (-\infty,-14) \cup (-1,13)$\\
B.$x \in (-\infty,-14) \cup (-1,13]$\\
C.$x \in (-\infty,-14) \cup [-1,13)$\\
D.$x \in (-\infty,-14] \cup (-1,13)$\\
E.$x \in (-\infty,-14] \cup (-1,13]$\\
F.$x \in (-\infty,-14] \cup [-1,13)$\\
G.$x \in (-\infty,-14) \cup [-1,13]$\\
H.$x \in (-\infty,-14] \cup [-1,13]$
\testStop
\kluczStart
A
\kluczStop



\zadStart{Zadanie z Wikieł Z 1.62 b) moja wersja nr 354}

Rozwiązać nierówności $(x+14)(13-x)(x+2)\ge0$.
\zadStop
\rozwStart{Patryk Wirkus}{}
Miejsca zerowe naszego wielomianu to: $-14, 13, -2$.\\
Wielomian jest stopnia nieparzystego, ponadto znak współczynnika przy\linebreak najwyższej potędze x jest ujemny.\\ W związku z tym wykres wielomianu zaczyna się od lewej strony powyżej osi OX. A więc $$x \in (-\infty,-14) \cup (-2,13).$$
\rozwStop
\odpStart
$x \in (-\infty,-14) \cup (-2,13)$
\odpStop
\testStart
A.$x \in (-\infty,-14) \cup (-2,13)$\\
B.$x \in (-\infty,-14) \cup (-2,13]$\\
C.$x \in (-\infty,-14) \cup [-2,13)$\\
D.$x \in (-\infty,-14] \cup (-2,13)$\\
E.$x \in (-\infty,-14] \cup (-2,13]$\\
F.$x \in (-\infty,-14] \cup [-2,13)$\\
G.$x \in (-\infty,-14) \cup [-2,13]$\\
H.$x \in (-\infty,-14] \cup [-2,13]$
\testStop
\kluczStart
A
\kluczStop



\zadStart{Zadanie z Wikieł Z 1.62 b) moja wersja nr 355}

Rozwiązać nierówności $(x+14)(13-x)(x+3)\ge0$.
\zadStop
\rozwStart{Patryk Wirkus}{}
Miejsca zerowe naszego wielomianu to: $-14, 13, -3$.\\
Wielomian jest stopnia nieparzystego, ponadto znak współczynnika przy\linebreak najwyższej potędze x jest ujemny.\\ W związku z tym wykres wielomianu zaczyna się od lewej strony powyżej osi OX. A więc $$x \in (-\infty,-14) \cup (-3,13).$$
\rozwStop
\odpStart
$x \in (-\infty,-14) \cup (-3,13)$
\odpStop
\testStart
A.$x \in (-\infty,-14) \cup (-3,13)$\\
B.$x \in (-\infty,-14) \cup (-3,13]$\\
C.$x \in (-\infty,-14) \cup [-3,13)$\\
D.$x \in (-\infty,-14] \cup (-3,13)$\\
E.$x \in (-\infty,-14] \cup (-3,13]$\\
F.$x \in (-\infty,-14] \cup [-3,13)$\\
G.$x \in (-\infty,-14) \cup [-3,13]$\\
H.$x \in (-\infty,-14] \cup [-3,13]$
\testStop
\kluczStart
A
\kluczStop



\zadStart{Zadanie z Wikieł Z 1.62 b) moja wersja nr 356}

Rozwiązać nierówności $(x+14)(13-x)(x+4)\ge0$.
\zadStop
\rozwStart{Patryk Wirkus}{}
Miejsca zerowe naszego wielomianu to: $-14, 13, -4$.\\
Wielomian jest stopnia nieparzystego, ponadto znak współczynnika przy\linebreak najwyższej potędze x jest ujemny.\\ W związku z tym wykres wielomianu zaczyna się od lewej strony powyżej osi OX. A więc $$x \in (-\infty,-14) \cup (-4,13).$$
\rozwStop
\odpStart
$x \in (-\infty,-14) \cup (-4,13)$
\odpStop
\testStart
A.$x \in (-\infty,-14) \cup (-4,13)$\\
B.$x \in (-\infty,-14) \cup (-4,13]$\\
C.$x \in (-\infty,-14) \cup [-4,13)$\\
D.$x \in (-\infty,-14] \cup (-4,13)$\\
E.$x \in (-\infty,-14] \cup (-4,13]$\\
F.$x \in (-\infty,-14] \cup [-4,13)$\\
G.$x \in (-\infty,-14) \cup [-4,13]$\\
H.$x \in (-\infty,-14] \cup [-4,13]$
\testStop
\kluczStart
A
\kluczStop



\zadStart{Zadanie z Wikieł Z 1.62 b) moja wersja nr 357}

Rozwiązać nierówności $(x+14)(13-x)(x+5)\ge0$.
\zadStop
\rozwStart{Patryk Wirkus}{}
Miejsca zerowe naszego wielomianu to: $-14, 13, -5$.\\
Wielomian jest stopnia nieparzystego, ponadto znak współczynnika przy\linebreak najwyższej potędze x jest ujemny.\\ W związku z tym wykres wielomianu zaczyna się od lewej strony powyżej osi OX. A więc $$x \in (-\infty,-14) \cup (-5,13).$$
\rozwStop
\odpStart
$x \in (-\infty,-14) \cup (-5,13)$
\odpStop
\testStart
A.$x \in (-\infty,-14) \cup (-5,13)$\\
B.$x \in (-\infty,-14) \cup (-5,13]$\\
C.$x \in (-\infty,-14) \cup [-5,13)$\\
D.$x \in (-\infty,-14] \cup (-5,13)$\\
E.$x \in (-\infty,-14] \cup (-5,13]$\\
F.$x \in (-\infty,-14] \cup [-5,13)$\\
G.$x \in (-\infty,-14) \cup [-5,13]$\\
H.$x \in (-\infty,-14] \cup [-5,13]$
\testStop
\kluczStart
A
\kluczStop



\zadStart{Zadanie z Wikieł Z 1.62 b) moja wersja nr 358}

Rozwiązać nierówności $(x+14)(13-x)(x+6)\ge0$.
\zadStop
\rozwStart{Patryk Wirkus}{}
Miejsca zerowe naszego wielomianu to: $-14, 13, -6$.\\
Wielomian jest stopnia nieparzystego, ponadto znak współczynnika przy\linebreak najwyższej potędze x jest ujemny.\\ W związku z tym wykres wielomianu zaczyna się od lewej strony powyżej osi OX. A więc $$x \in (-\infty,-14) \cup (-6,13).$$
\rozwStop
\odpStart
$x \in (-\infty,-14) \cup (-6,13)$
\odpStop
\testStart
A.$x \in (-\infty,-14) \cup (-6,13)$\\
B.$x \in (-\infty,-14) \cup (-6,13]$\\
C.$x \in (-\infty,-14) \cup [-6,13)$\\
D.$x \in (-\infty,-14] \cup (-6,13)$\\
E.$x \in (-\infty,-14] \cup (-6,13]$\\
F.$x \in (-\infty,-14] \cup [-6,13)$\\
G.$x \in (-\infty,-14) \cup [-6,13]$\\
H.$x \in (-\infty,-14] \cup [-6,13]$
\testStop
\kluczStart
A
\kluczStop



\zadStart{Zadanie z Wikieł Z 1.62 b) moja wersja nr 359}

Rozwiązać nierówności $(x+14)(13-x)(x+7)\ge0$.
\zadStop
\rozwStart{Patryk Wirkus}{}
Miejsca zerowe naszego wielomianu to: $-14, 13, -7$.\\
Wielomian jest stopnia nieparzystego, ponadto znak współczynnika przy\linebreak najwyższej potędze x jest ujemny.\\ W związku z tym wykres wielomianu zaczyna się od lewej strony powyżej osi OX. A więc $$x \in (-\infty,-14) \cup (-7,13).$$
\rozwStop
\odpStart
$x \in (-\infty,-14) \cup (-7,13)$
\odpStop
\testStart
A.$x \in (-\infty,-14) \cup (-7,13)$\\
B.$x \in (-\infty,-14) \cup (-7,13]$\\
C.$x \in (-\infty,-14) \cup [-7,13)$\\
D.$x \in (-\infty,-14] \cup (-7,13)$\\
E.$x \in (-\infty,-14] \cup (-7,13]$\\
F.$x \in (-\infty,-14] \cup [-7,13)$\\
G.$x \in (-\infty,-14) \cup [-7,13]$\\
H.$x \in (-\infty,-14] \cup [-7,13]$
\testStop
\kluczStart
A
\kluczStop



\zadStart{Zadanie z Wikieł Z 1.62 b) moja wersja nr 360}

Rozwiązać nierówności $(x+14)(13-x)(x+8)\ge0$.
\zadStop
\rozwStart{Patryk Wirkus}{}
Miejsca zerowe naszego wielomianu to: $-14, 13, -8$.\\
Wielomian jest stopnia nieparzystego, ponadto znak współczynnika przy\linebreak najwyższej potędze x jest ujemny.\\ W związku z tym wykres wielomianu zaczyna się od lewej strony powyżej osi OX. A więc $$x \in (-\infty,-14) \cup (-8,13).$$
\rozwStop
\odpStart
$x \in (-\infty,-14) \cup (-8,13)$
\odpStop
\testStart
A.$x \in (-\infty,-14) \cup (-8,13)$\\
B.$x \in (-\infty,-14) \cup (-8,13]$\\
C.$x \in (-\infty,-14) \cup [-8,13)$\\
D.$x \in (-\infty,-14] \cup (-8,13)$\\
E.$x \in (-\infty,-14] \cup (-8,13]$\\
F.$x \in (-\infty,-14] \cup [-8,13)$\\
G.$x \in (-\infty,-14) \cup [-8,13]$\\
H.$x \in (-\infty,-14] \cup [-8,13]$
\testStop
\kluczStart
A
\kluczStop



\zadStart{Zadanie z Wikieł Z 1.62 b) moja wersja nr 361}

Rozwiązać nierówności $(x+14)(13-x)(x+9)\ge0$.
\zadStop
\rozwStart{Patryk Wirkus}{}
Miejsca zerowe naszego wielomianu to: $-14, 13, -9$.\\
Wielomian jest stopnia nieparzystego, ponadto znak współczynnika przy\linebreak najwyższej potędze x jest ujemny.\\ W związku z tym wykres wielomianu zaczyna się od lewej strony powyżej osi OX. A więc $$x \in (-\infty,-14) \cup (-9,13).$$
\rozwStop
\odpStart
$x \in (-\infty,-14) \cup (-9,13)$
\odpStop
\testStart
A.$x \in (-\infty,-14) \cup (-9,13)$\\
B.$x \in (-\infty,-14) \cup (-9,13]$\\
C.$x \in (-\infty,-14) \cup [-9,13)$\\
D.$x \in (-\infty,-14] \cup (-9,13)$\\
E.$x \in (-\infty,-14] \cup (-9,13]$\\
F.$x \in (-\infty,-14] \cup [-9,13)$\\
G.$x \in (-\infty,-14) \cup [-9,13]$\\
H.$x \in (-\infty,-14] \cup [-9,13]$
\testStop
\kluczStart
A
\kluczStop



\zadStart{Zadanie z Wikieł Z 1.62 b) moja wersja nr 362}

Rozwiązać nierówności $(x+14)(13-x)(x+10)\ge0$.
\zadStop
\rozwStart{Patryk Wirkus}{}
Miejsca zerowe naszego wielomianu to: $-14, 13, -10$.\\
Wielomian jest stopnia nieparzystego, ponadto znak współczynnika przy\linebreak najwyższej potędze x jest ujemny.\\ W związku z tym wykres wielomianu zaczyna się od lewej strony powyżej osi OX. A więc $$x \in (-\infty,-14) \cup (-10,13).$$
\rozwStop
\odpStart
$x \in (-\infty,-14) \cup (-10,13)$
\odpStop
\testStart
A.$x \in (-\infty,-14) \cup (-10,13)$\\
B.$x \in (-\infty,-14) \cup (-10,13]$\\
C.$x \in (-\infty,-14) \cup [-10,13)$\\
D.$x \in (-\infty,-14] \cup (-10,13)$\\
E.$x \in (-\infty,-14] \cup (-10,13]$\\
F.$x \in (-\infty,-14] \cup [-10,13)$\\
G.$x \in (-\infty,-14) \cup [-10,13]$\\
H.$x \in (-\infty,-14] \cup [-10,13]$
\testStop
\kluczStart
A
\kluczStop



\zadStart{Zadanie z Wikieł Z 1.62 b) moja wersja nr 363}

Rozwiązać nierówności $(x+14)(13-x)(x+11)\ge0$.
\zadStop
\rozwStart{Patryk Wirkus}{}
Miejsca zerowe naszego wielomianu to: $-14, 13, -11$.\\
Wielomian jest stopnia nieparzystego, ponadto znak współczynnika przy\linebreak najwyższej potędze x jest ujemny.\\ W związku z tym wykres wielomianu zaczyna się od lewej strony powyżej osi OX. A więc $$x \in (-\infty,-14) \cup (-11,13).$$
\rozwStop
\odpStart
$x \in (-\infty,-14) \cup (-11,13)$
\odpStop
\testStart
A.$x \in (-\infty,-14) \cup (-11,13)$\\
B.$x \in (-\infty,-14) \cup (-11,13]$\\
C.$x \in (-\infty,-14) \cup [-11,13)$\\
D.$x \in (-\infty,-14] \cup (-11,13)$\\
E.$x \in (-\infty,-14] \cup (-11,13]$\\
F.$x \in (-\infty,-14] \cup [-11,13)$\\
G.$x \in (-\infty,-14) \cup [-11,13]$\\
H.$x \in (-\infty,-14] \cup [-11,13]$
\testStop
\kluczStart
A
\kluczStop



\zadStart{Zadanie z Wikieł Z 1.62 b) moja wersja nr 364}

Rozwiązać nierówności $(x+14)(13-x)(x+12)\ge0$.
\zadStop
\rozwStart{Patryk Wirkus}{}
Miejsca zerowe naszego wielomianu to: $-14, 13, -12$.\\
Wielomian jest stopnia nieparzystego, ponadto znak współczynnika przy\linebreak najwyższej potędze x jest ujemny.\\ W związku z tym wykres wielomianu zaczyna się od lewej strony powyżej osi OX. A więc $$x \in (-\infty,-14) \cup (-12,13).$$
\rozwStop
\odpStart
$x \in (-\infty,-14) \cup (-12,13)$
\odpStop
\testStart
A.$x \in (-\infty,-14) \cup (-12,13)$\\
B.$x \in (-\infty,-14) \cup (-12,13]$\\
C.$x \in (-\infty,-14) \cup [-12,13)$\\
D.$x \in (-\infty,-14] \cup (-12,13)$\\
E.$x \in (-\infty,-14] \cup (-12,13]$\\
F.$x \in (-\infty,-14] \cup [-12,13)$\\
G.$x \in (-\infty,-14) \cup [-12,13]$\\
H.$x \in (-\infty,-14] \cup [-12,13]$
\testStop
\kluczStart
A
\kluczStop



\zadStart{Zadanie z Wikieł Z 1.62 b) moja wersja nr 365}

Rozwiązać nierówności $(x+15)(2-x)(x+1)\ge0$.
\zadStop
\rozwStart{Patryk Wirkus}{}
Miejsca zerowe naszego wielomianu to: $-15, 2, -1$.\\
Wielomian jest stopnia nieparzystego, ponadto znak współczynnika przy\linebreak najwyższej potędze x jest ujemny.\\ W związku z tym wykres wielomianu zaczyna się od lewej strony powyżej osi OX. A więc $$x \in (-\infty,-15) \cup (-1,2).$$
\rozwStop
\odpStart
$x \in (-\infty,-15) \cup (-1,2)$
\odpStop
\testStart
A.$x \in (-\infty,-15) \cup (-1,2)$\\
B.$x \in (-\infty,-15) \cup (-1,2]$\\
C.$x \in (-\infty,-15) \cup [-1,2)$\\
D.$x \in (-\infty,-15] \cup (-1,2)$\\
E.$x \in (-\infty,-15] \cup (-1,2]$\\
F.$x \in (-\infty,-15] \cup [-1,2)$\\
G.$x \in (-\infty,-15) \cup [-1,2]$\\
H.$x \in (-\infty,-15] \cup [-1,2]$
\testStop
\kluczStart
A
\kluczStop



\zadStart{Zadanie z Wikieł Z 1.62 b) moja wersja nr 366}

Rozwiązać nierówności $(x+15)(3-x)(x+1)\ge0$.
\zadStop
\rozwStart{Patryk Wirkus}{}
Miejsca zerowe naszego wielomianu to: $-15, 3, -1$.\\
Wielomian jest stopnia nieparzystego, ponadto znak współczynnika przy\linebreak najwyższej potędze x jest ujemny.\\ W związku z tym wykres wielomianu zaczyna się od lewej strony powyżej osi OX. A więc $$x \in (-\infty,-15) \cup (-1,3).$$
\rozwStop
\odpStart
$x \in (-\infty,-15) \cup (-1,3)$
\odpStop
\testStart
A.$x \in (-\infty,-15) \cup (-1,3)$\\
B.$x \in (-\infty,-15) \cup (-1,3]$\\
C.$x \in (-\infty,-15) \cup [-1,3)$\\
D.$x \in (-\infty,-15] \cup (-1,3)$\\
E.$x \in (-\infty,-15] \cup (-1,3]$\\
F.$x \in (-\infty,-15] \cup [-1,3)$\\
G.$x \in (-\infty,-15) \cup [-1,3]$\\
H.$x \in (-\infty,-15] \cup [-1,3]$
\testStop
\kluczStart
A
\kluczStop



\zadStart{Zadanie z Wikieł Z 1.62 b) moja wersja nr 367}

Rozwiązać nierówności $(x+15)(3-x)(x+2)\ge0$.
\zadStop
\rozwStart{Patryk Wirkus}{}
Miejsca zerowe naszego wielomianu to: $-15, 3, -2$.\\
Wielomian jest stopnia nieparzystego, ponadto znak współczynnika przy\linebreak najwyższej potędze x jest ujemny.\\ W związku z tym wykres wielomianu zaczyna się od lewej strony powyżej osi OX. A więc $$x \in (-\infty,-15) \cup (-2,3).$$
\rozwStop
\odpStart
$x \in (-\infty,-15) \cup (-2,3)$
\odpStop
\testStart
A.$x \in (-\infty,-15) \cup (-2,3)$\\
B.$x \in (-\infty,-15) \cup (-2,3]$\\
C.$x \in (-\infty,-15) \cup [-2,3)$\\
D.$x \in (-\infty,-15] \cup (-2,3)$\\
E.$x \in (-\infty,-15] \cup (-2,3]$\\
F.$x \in (-\infty,-15] \cup [-2,3)$\\
G.$x \in (-\infty,-15) \cup [-2,3]$\\
H.$x \in (-\infty,-15] \cup [-2,3]$
\testStop
\kluczStart
A
\kluczStop



\zadStart{Zadanie z Wikieł Z 1.62 b) moja wersja nr 368}

Rozwiązać nierówności $(x+15)(4-x)(x+1)\ge0$.
\zadStop
\rozwStart{Patryk Wirkus}{}
Miejsca zerowe naszego wielomianu to: $-15, 4, -1$.\\
Wielomian jest stopnia nieparzystego, ponadto znak współczynnika przy\linebreak najwyższej potędze x jest ujemny.\\ W związku z tym wykres wielomianu zaczyna się od lewej strony powyżej osi OX. A więc $$x \in (-\infty,-15) \cup (-1,4).$$
\rozwStop
\odpStart
$x \in (-\infty,-15) \cup (-1,4)$
\odpStop
\testStart
A.$x \in (-\infty,-15) \cup (-1,4)$\\
B.$x \in (-\infty,-15) \cup (-1,4]$\\
C.$x \in (-\infty,-15) \cup [-1,4)$\\
D.$x \in (-\infty,-15] \cup (-1,4)$\\
E.$x \in (-\infty,-15] \cup (-1,4]$\\
F.$x \in (-\infty,-15] \cup [-1,4)$\\
G.$x \in (-\infty,-15) \cup [-1,4]$\\
H.$x \in (-\infty,-15] \cup [-1,4]$
\testStop
\kluczStart
A
\kluczStop



\zadStart{Zadanie z Wikieł Z 1.62 b) moja wersja nr 369}

Rozwiązać nierówności $(x+15)(4-x)(x+2)\ge0$.
\zadStop
\rozwStart{Patryk Wirkus}{}
Miejsca zerowe naszego wielomianu to: $-15, 4, -2$.\\
Wielomian jest stopnia nieparzystego, ponadto znak współczynnika przy\linebreak najwyższej potędze x jest ujemny.\\ W związku z tym wykres wielomianu zaczyna się od lewej strony powyżej osi OX. A więc $$x \in (-\infty,-15) \cup (-2,4).$$
\rozwStop
\odpStart
$x \in (-\infty,-15) \cup (-2,4)$
\odpStop
\testStart
A.$x \in (-\infty,-15) \cup (-2,4)$\\
B.$x \in (-\infty,-15) \cup (-2,4]$\\
C.$x \in (-\infty,-15) \cup [-2,4)$\\
D.$x \in (-\infty,-15] \cup (-2,4)$\\
E.$x \in (-\infty,-15] \cup (-2,4]$\\
F.$x \in (-\infty,-15] \cup [-2,4)$\\
G.$x \in (-\infty,-15) \cup [-2,4]$\\
H.$x \in (-\infty,-15] \cup [-2,4]$
\testStop
\kluczStart
A
\kluczStop



\zadStart{Zadanie z Wikieł Z 1.62 b) moja wersja nr 370}

Rozwiązać nierówności $(x+15)(4-x)(x+3)\ge0$.
\zadStop
\rozwStart{Patryk Wirkus}{}
Miejsca zerowe naszego wielomianu to: $-15, 4, -3$.\\
Wielomian jest stopnia nieparzystego, ponadto znak współczynnika przy\linebreak najwyższej potędze x jest ujemny.\\ W związku z tym wykres wielomianu zaczyna się od lewej strony powyżej osi OX. A więc $$x \in (-\infty,-15) \cup (-3,4).$$
\rozwStop
\odpStart
$x \in (-\infty,-15) \cup (-3,4)$
\odpStop
\testStart
A.$x \in (-\infty,-15) \cup (-3,4)$\\
B.$x \in (-\infty,-15) \cup (-3,4]$\\
C.$x \in (-\infty,-15) \cup [-3,4)$\\
D.$x \in (-\infty,-15] \cup (-3,4)$\\
E.$x \in (-\infty,-15] \cup (-3,4]$\\
F.$x \in (-\infty,-15] \cup [-3,4)$\\
G.$x \in (-\infty,-15) \cup [-3,4]$\\
H.$x \in (-\infty,-15] \cup [-3,4]$
\testStop
\kluczStart
A
\kluczStop



\zadStart{Zadanie z Wikieł Z 1.62 b) moja wersja nr 371}

Rozwiązać nierówności $(x+15)(5-x)(x+1)\ge0$.
\zadStop
\rozwStart{Patryk Wirkus}{}
Miejsca zerowe naszego wielomianu to: $-15, 5, -1$.\\
Wielomian jest stopnia nieparzystego, ponadto znak współczynnika przy\linebreak najwyższej potędze x jest ujemny.\\ W związku z tym wykres wielomianu zaczyna się od lewej strony powyżej osi OX. A więc $$x \in (-\infty,-15) \cup (-1,5).$$
\rozwStop
\odpStart
$x \in (-\infty,-15) \cup (-1,5)$
\odpStop
\testStart
A.$x \in (-\infty,-15) \cup (-1,5)$\\
B.$x \in (-\infty,-15) \cup (-1,5]$\\
C.$x \in (-\infty,-15) \cup [-1,5)$\\
D.$x \in (-\infty,-15] \cup (-1,5)$\\
E.$x \in (-\infty,-15] \cup (-1,5]$\\
F.$x \in (-\infty,-15] \cup [-1,5)$\\
G.$x \in (-\infty,-15) \cup [-1,5]$\\
H.$x \in (-\infty,-15] \cup [-1,5]$
\testStop
\kluczStart
A
\kluczStop



\zadStart{Zadanie z Wikieł Z 1.62 b) moja wersja nr 372}

Rozwiązać nierówności $(x+15)(5-x)(x+2)\ge0$.
\zadStop
\rozwStart{Patryk Wirkus}{}
Miejsca zerowe naszego wielomianu to: $-15, 5, -2$.\\
Wielomian jest stopnia nieparzystego, ponadto znak współczynnika przy\linebreak najwyższej potędze x jest ujemny.\\ W związku z tym wykres wielomianu zaczyna się od lewej strony powyżej osi OX. A więc $$x \in (-\infty,-15) \cup (-2,5).$$
\rozwStop
\odpStart
$x \in (-\infty,-15) \cup (-2,5)$
\odpStop
\testStart
A.$x \in (-\infty,-15) \cup (-2,5)$\\
B.$x \in (-\infty,-15) \cup (-2,5]$\\
C.$x \in (-\infty,-15) \cup [-2,5)$\\
D.$x \in (-\infty,-15] \cup (-2,5)$\\
E.$x \in (-\infty,-15] \cup (-2,5]$\\
F.$x \in (-\infty,-15] \cup [-2,5)$\\
G.$x \in (-\infty,-15) \cup [-2,5]$\\
H.$x \in (-\infty,-15] \cup [-2,5]$
\testStop
\kluczStart
A
\kluczStop



\zadStart{Zadanie z Wikieł Z 1.62 b) moja wersja nr 373}

Rozwiązać nierówności $(x+15)(5-x)(x+3)\ge0$.
\zadStop
\rozwStart{Patryk Wirkus}{}
Miejsca zerowe naszego wielomianu to: $-15, 5, -3$.\\
Wielomian jest stopnia nieparzystego, ponadto znak współczynnika przy\linebreak najwyższej potędze x jest ujemny.\\ W związku z tym wykres wielomianu zaczyna się od lewej strony powyżej osi OX. A więc $$x \in (-\infty,-15) \cup (-3,5).$$
\rozwStop
\odpStart
$x \in (-\infty,-15) \cup (-3,5)$
\odpStop
\testStart
A.$x \in (-\infty,-15) \cup (-3,5)$\\
B.$x \in (-\infty,-15) \cup (-3,5]$\\
C.$x \in (-\infty,-15) \cup [-3,5)$\\
D.$x \in (-\infty,-15] \cup (-3,5)$\\
E.$x \in (-\infty,-15] \cup (-3,5]$\\
F.$x \in (-\infty,-15] \cup [-3,5)$\\
G.$x \in (-\infty,-15) \cup [-3,5]$\\
H.$x \in (-\infty,-15] \cup [-3,5]$
\testStop
\kluczStart
A
\kluczStop



\zadStart{Zadanie z Wikieł Z 1.62 b) moja wersja nr 374}

Rozwiązać nierówności $(x+15)(5-x)(x+4)\ge0$.
\zadStop
\rozwStart{Patryk Wirkus}{}
Miejsca zerowe naszego wielomianu to: $-15, 5, -4$.\\
Wielomian jest stopnia nieparzystego, ponadto znak współczynnika przy\linebreak najwyższej potędze x jest ujemny.\\ W związku z tym wykres wielomianu zaczyna się od lewej strony powyżej osi OX. A więc $$x \in (-\infty,-15) \cup (-4,5).$$
\rozwStop
\odpStart
$x \in (-\infty,-15) \cup (-4,5)$
\odpStop
\testStart
A.$x \in (-\infty,-15) \cup (-4,5)$\\
B.$x \in (-\infty,-15) \cup (-4,5]$\\
C.$x \in (-\infty,-15) \cup [-4,5)$\\
D.$x \in (-\infty,-15] \cup (-4,5)$\\
E.$x \in (-\infty,-15] \cup (-4,5]$\\
F.$x \in (-\infty,-15] \cup [-4,5)$\\
G.$x \in (-\infty,-15) \cup [-4,5]$\\
H.$x \in (-\infty,-15] \cup [-4,5]$
\testStop
\kluczStart
A
\kluczStop



\zadStart{Zadanie z Wikieł Z 1.62 b) moja wersja nr 375}

Rozwiązać nierówności $(x+15)(6-x)(x+1)\ge0$.
\zadStop
\rozwStart{Patryk Wirkus}{}
Miejsca zerowe naszego wielomianu to: $-15, 6, -1$.\\
Wielomian jest stopnia nieparzystego, ponadto znak współczynnika przy\linebreak najwyższej potędze x jest ujemny.\\ W związku z tym wykres wielomianu zaczyna się od lewej strony powyżej osi OX. A więc $$x \in (-\infty,-15) \cup (-1,6).$$
\rozwStop
\odpStart
$x \in (-\infty,-15) \cup (-1,6)$
\odpStop
\testStart
A.$x \in (-\infty,-15) \cup (-1,6)$\\
B.$x \in (-\infty,-15) \cup (-1,6]$\\
C.$x \in (-\infty,-15) \cup [-1,6)$\\
D.$x \in (-\infty,-15] \cup (-1,6)$\\
E.$x \in (-\infty,-15] \cup (-1,6]$\\
F.$x \in (-\infty,-15] \cup [-1,6)$\\
G.$x \in (-\infty,-15) \cup [-1,6]$\\
H.$x \in (-\infty,-15] \cup [-1,6]$
\testStop
\kluczStart
A
\kluczStop



\zadStart{Zadanie z Wikieł Z 1.62 b) moja wersja nr 376}

Rozwiązać nierówności $(x+15)(6-x)(x+2)\ge0$.
\zadStop
\rozwStart{Patryk Wirkus}{}
Miejsca zerowe naszego wielomianu to: $-15, 6, -2$.\\
Wielomian jest stopnia nieparzystego, ponadto znak współczynnika przy\linebreak najwyższej potędze x jest ujemny.\\ W związku z tym wykres wielomianu zaczyna się od lewej strony powyżej osi OX. A więc $$x \in (-\infty,-15) \cup (-2,6).$$
\rozwStop
\odpStart
$x \in (-\infty,-15) \cup (-2,6)$
\odpStop
\testStart
A.$x \in (-\infty,-15) \cup (-2,6)$\\
B.$x \in (-\infty,-15) \cup (-2,6]$\\
C.$x \in (-\infty,-15) \cup [-2,6)$\\
D.$x \in (-\infty,-15] \cup (-2,6)$\\
E.$x \in (-\infty,-15] \cup (-2,6]$\\
F.$x \in (-\infty,-15] \cup [-2,6)$\\
G.$x \in (-\infty,-15) \cup [-2,6]$\\
H.$x \in (-\infty,-15] \cup [-2,6]$
\testStop
\kluczStart
A
\kluczStop



\zadStart{Zadanie z Wikieł Z 1.62 b) moja wersja nr 377}

Rozwiązać nierówności $(x+15)(6-x)(x+3)\ge0$.
\zadStop
\rozwStart{Patryk Wirkus}{}
Miejsca zerowe naszego wielomianu to: $-15, 6, -3$.\\
Wielomian jest stopnia nieparzystego, ponadto znak współczynnika przy\linebreak najwyższej potędze x jest ujemny.\\ W związku z tym wykres wielomianu zaczyna się od lewej strony powyżej osi OX. A więc $$x \in (-\infty,-15) \cup (-3,6).$$
\rozwStop
\odpStart
$x \in (-\infty,-15) \cup (-3,6)$
\odpStop
\testStart
A.$x \in (-\infty,-15) \cup (-3,6)$\\
B.$x \in (-\infty,-15) \cup (-3,6]$\\
C.$x \in (-\infty,-15) \cup [-3,6)$\\
D.$x \in (-\infty,-15] \cup (-3,6)$\\
E.$x \in (-\infty,-15] \cup (-3,6]$\\
F.$x \in (-\infty,-15] \cup [-3,6)$\\
G.$x \in (-\infty,-15) \cup [-3,6]$\\
H.$x \in (-\infty,-15] \cup [-3,6]$
\testStop
\kluczStart
A
\kluczStop



\zadStart{Zadanie z Wikieł Z 1.62 b) moja wersja nr 378}

Rozwiązać nierówności $(x+15)(6-x)(x+4)\ge0$.
\zadStop
\rozwStart{Patryk Wirkus}{}
Miejsca zerowe naszego wielomianu to: $-15, 6, -4$.\\
Wielomian jest stopnia nieparzystego, ponadto znak współczynnika przy\linebreak najwyższej potędze x jest ujemny.\\ W związku z tym wykres wielomianu zaczyna się od lewej strony powyżej osi OX. A więc $$x \in (-\infty,-15) \cup (-4,6).$$
\rozwStop
\odpStart
$x \in (-\infty,-15) \cup (-4,6)$
\odpStop
\testStart
A.$x \in (-\infty,-15) \cup (-4,6)$\\
B.$x \in (-\infty,-15) \cup (-4,6]$\\
C.$x \in (-\infty,-15) \cup [-4,6)$\\
D.$x \in (-\infty,-15] \cup (-4,6)$\\
E.$x \in (-\infty,-15] \cup (-4,6]$\\
F.$x \in (-\infty,-15] \cup [-4,6)$\\
G.$x \in (-\infty,-15) \cup [-4,6]$\\
H.$x \in (-\infty,-15] \cup [-4,6]$
\testStop
\kluczStart
A
\kluczStop



\zadStart{Zadanie z Wikieł Z 1.62 b) moja wersja nr 379}

Rozwiązać nierówności $(x+15)(6-x)(x+5)\ge0$.
\zadStop
\rozwStart{Patryk Wirkus}{}
Miejsca zerowe naszego wielomianu to: $-15, 6, -5$.\\
Wielomian jest stopnia nieparzystego, ponadto znak współczynnika przy\linebreak najwyższej potędze x jest ujemny.\\ W związku z tym wykres wielomianu zaczyna się od lewej strony powyżej osi OX. A więc $$x \in (-\infty,-15) \cup (-5,6).$$
\rozwStop
\odpStart
$x \in (-\infty,-15) \cup (-5,6)$
\odpStop
\testStart
A.$x \in (-\infty,-15) \cup (-5,6)$\\
B.$x \in (-\infty,-15) \cup (-5,6]$\\
C.$x \in (-\infty,-15) \cup [-5,6)$\\
D.$x \in (-\infty,-15] \cup (-5,6)$\\
E.$x \in (-\infty,-15] \cup (-5,6]$\\
F.$x \in (-\infty,-15] \cup [-5,6)$\\
G.$x \in (-\infty,-15) \cup [-5,6]$\\
H.$x \in (-\infty,-15] \cup [-5,6]$
\testStop
\kluczStart
A
\kluczStop



\zadStart{Zadanie z Wikieł Z 1.62 b) moja wersja nr 380}

Rozwiązać nierówności $(x+15)(7-x)(x+1)\ge0$.
\zadStop
\rozwStart{Patryk Wirkus}{}
Miejsca zerowe naszego wielomianu to: $-15, 7, -1$.\\
Wielomian jest stopnia nieparzystego, ponadto znak współczynnika przy\linebreak najwyższej potędze x jest ujemny.\\ W związku z tym wykres wielomianu zaczyna się od lewej strony powyżej osi OX. A więc $$x \in (-\infty,-15) \cup (-1,7).$$
\rozwStop
\odpStart
$x \in (-\infty,-15) \cup (-1,7)$
\odpStop
\testStart
A.$x \in (-\infty,-15) \cup (-1,7)$\\
B.$x \in (-\infty,-15) \cup (-1,7]$\\
C.$x \in (-\infty,-15) \cup [-1,7)$\\
D.$x \in (-\infty,-15] \cup (-1,7)$\\
E.$x \in (-\infty,-15] \cup (-1,7]$\\
F.$x \in (-\infty,-15] \cup [-1,7)$\\
G.$x \in (-\infty,-15) \cup [-1,7]$\\
H.$x \in (-\infty,-15] \cup [-1,7]$
\testStop
\kluczStart
A
\kluczStop



\zadStart{Zadanie z Wikieł Z 1.62 b) moja wersja nr 381}

Rozwiązać nierówności $(x+15)(7-x)(x+2)\ge0$.
\zadStop
\rozwStart{Patryk Wirkus}{}
Miejsca zerowe naszego wielomianu to: $-15, 7, -2$.\\
Wielomian jest stopnia nieparzystego, ponadto znak współczynnika przy\linebreak najwyższej potędze x jest ujemny.\\ W związku z tym wykres wielomianu zaczyna się od lewej strony powyżej osi OX. A więc $$x \in (-\infty,-15) \cup (-2,7).$$
\rozwStop
\odpStart
$x \in (-\infty,-15) \cup (-2,7)$
\odpStop
\testStart
A.$x \in (-\infty,-15) \cup (-2,7)$\\
B.$x \in (-\infty,-15) \cup (-2,7]$\\
C.$x \in (-\infty,-15) \cup [-2,7)$\\
D.$x \in (-\infty,-15] \cup (-2,7)$\\
E.$x \in (-\infty,-15] \cup (-2,7]$\\
F.$x \in (-\infty,-15] \cup [-2,7)$\\
G.$x \in (-\infty,-15) \cup [-2,7]$\\
H.$x \in (-\infty,-15] \cup [-2,7]$
\testStop
\kluczStart
A
\kluczStop



\zadStart{Zadanie z Wikieł Z 1.62 b) moja wersja nr 382}

Rozwiązać nierówności $(x+15)(7-x)(x+3)\ge0$.
\zadStop
\rozwStart{Patryk Wirkus}{}
Miejsca zerowe naszego wielomianu to: $-15, 7, -3$.\\
Wielomian jest stopnia nieparzystego, ponadto znak współczynnika przy\linebreak najwyższej potędze x jest ujemny.\\ W związku z tym wykres wielomianu zaczyna się od lewej strony powyżej osi OX. A więc $$x \in (-\infty,-15) \cup (-3,7).$$
\rozwStop
\odpStart
$x \in (-\infty,-15) \cup (-3,7)$
\odpStop
\testStart
A.$x \in (-\infty,-15) \cup (-3,7)$\\
B.$x \in (-\infty,-15) \cup (-3,7]$\\
C.$x \in (-\infty,-15) \cup [-3,7)$\\
D.$x \in (-\infty,-15] \cup (-3,7)$\\
E.$x \in (-\infty,-15] \cup (-3,7]$\\
F.$x \in (-\infty,-15] \cup [-3,7)$\\
G.$x \in (-\infty,-15) \cup [-3,7]$\\
H.$x \in (-\infty,-15] \cup [-3,7]$
\testStop
\kluczStart
A
\kluczStop



\zadStart{Zadanie z Wikieł Z 1.62 b) moja wersja nr 383}

Rozwiązać nierówności $(x+15)(7-x)(x+4)\ge0$.
\zadStop
\rozwStart{Patryk Wirkus}{}
Miejsca zerowe naszego wielomianu to: $-15, 7, -4$.\\
Wielomian jest stopnia nieparzystego, ponadto znak współczynnika przy\linebreak najwyższej potędze x jest ujemny.\\ W związku z tym wykres wielomianu zaczyna się od lewej strony powyżej osi OX. A więc $$x \in (-\infty,-15) \cup (-4,7).$$
\rozwStop
\odpStart
$x \in (-\infty,-15) \cup (-4,7)$
\odpStop
\testStart
A.$x \in (-\infty,-15) \cup (-4,7)$\\
B.$x \in (-\infty,-15) \cup (-4,7]$\\
C.$x \in (-\infty,-15) \cup [-4,7)$\\
D.$x \in (-\infty,-15] \cup (-4,7)$\\
E.$x \in (-\infty,-15] \cup (-4,7]$\\
F.$x \in (-\infty,-15] \cup [-4,7)$\\
G.$x \in (-\infty,-15) \cup [-4,7]$\\
H.$x \in (-\infty,-15] \cup [-4,7]$
\testStop
\kluczStart
A
\kluczStop



\zadStart{Zadanie z Wikieł Z 1.62 b) moja wersja nr 384}

Rozwiązać nierówności $(x+15)(7-x)(x+5)\ge0$.
\zadStop
\rozwStart{Patryk Wirkus}{}
Miejsca zerowe naszego wielomianu to: $-15, 7, -5$.\\
Wielomian jest stopnia nieparzystego, ponadto znak współczynnika przy\linebreak najwyższej potędze x jest ujemny.\\ W związku z tym wykres wielomianu zaczyna się od lewej strony powyżej osi OX. A więc $$x \in (-\infty,-15) \cup (-5,7).$$
\rozwStop
\odpStart
$x \in (-\infty,-15) \cup (-5,7)$
\odpStop
\testStart
A.$x \in (-\infty,-15) \cup (-5,7)$\\
B.$x \in (-\infty,-15) \cup (-5,7]$\\
C.$x \in (-\infty,-15) \cup [-5,7)$\\
D.$x \in (-\infty,-15] \cup (-5,7)$\\
E.$x \in (-\infty,-15] \cup (-5,7]$\\
F.$x \in (-\infty,-15] \cup [-5,7)$\\
G.$x \in (-\infty,-15) \cup [-5,7]$\\
H.$x \in (-\infty,-15] \cup [-5,7]$
\testStop
\kluczStart
A
\kluczStop



\zadStart{Zadanie z Wikieł Z 1.62 b) moja wersja nr 385}

Rozwiązać nierówności $(x+15)(7-x)(x+6)\ge0$.
\zadStop
\rozwStart{Patryk Wirkus}{}
Miejsca zerowe naszego wielomianu to: $-15, 7, -6$.\\
Wielomian jest stopnia nieparzystego, ponadto znak współczynnika przy\linebreak najwyższej potędze x jest ujemny.\\ W związku z tym wykres wielomianu zaczyna się od lewej strony powyżej osi OX. A więc $$x \in (-\infty,-15) \cup (-6,7).$$
\rozwStop
\odpStart
$x \in (-\infty,-15) \cup (-6,7)$
\odpStop
\testStart
A.$x \in (-\infty,-15) \cup (-6,7)$\\
B.$x \in (-\infty,-15) \cup (-6,7]$\\
C.$x \in (-\infty,-15) \cup [-6,7)$\\
D.$x \in (-\infty,-15] \cup (-6,7)$\\
E.$x \in (-\infty,-15] \cup (-6,7]$\\
F.$x \in (-\infty,-15] \cup [-6,7)$\\
G.$x \in (-\infty,-15) \cup [-6,7]$\\
H.$x \in (-\infty,-15] \cup [-6,7]$
\testStop
\kluczStart
A
\kluczStop



\zadStart{Zadanie z Wikieł Z 1.62 b) moja wersja nr 386}

Rozwiązać nierówności $(x+15)(8-x)(x+1)\ge0$.
\zadStop
\rozwStart{Patryk Wirkus}{}
Miejsca zerowe naszego wielomianu to: $-15, 8, -1$.\\
Wielomian jest stopnia nieparzystego, ponadto znak współczynnika przy\linebreak najwyższej potędze x jest ujemny.\\ W związku z tym wykres wielomianu zaczyna się od lewej strony powyżej osi OX. A więc $$x \in (-\infty,-15) \cup (-1,8).$$
\rozwStop
\odpStart
$x \in (-\infty,-15) \cup (-1,8)$
\odpStop
\testStart
A.$x \in (-\infty,-15) \cup (-1,8)$\\
B.$x \in (-\infty,-15) \cup (-1,8]$\\
C.$x \in (-\infty,-15) \cup [-1,8)$\\
D.$x \in (-\infty,-15] \cup (-1,8)$\\
E.$x \in (-\infty,-15] \cup (-1,8]$\\
F.$x \in (-\infty,-15] \cup [-1,8)$\\
G.$x \in (-\infty,-15) \cup [-1,8]$\\
H.$x \in (-\infty,-15] \cup [-1,8]$
\testStop
\kluczStart
A
\kluczStop



\zadStart{Zadanie z Wikieł Z 1.62 b) moja wersja nr 387}

Rozwiązać nierówności $(x+15)(8-x)(x+2)\ge0$.
\zadStop
\rozwStart{Patryk Wirkus}{}
Miejsca zerowe naszego wielomianu to: $-15, 8, -2$.\\
Wielomian jest stopnia nieparzystego, ponadto znak współczynnika przy\linebreak najwyższej potędze x jest ujemny.\\ W związku z tym wykres wielomianu zaczyna się od lewej strony powyżej osi OX. A więc $$x \in (-\infty,-15) \cup (-2,8).$$
\rozwStop
\odpStart
$x \in (-\infty,-15) \cup (-2,8)$
\odpStop
\testStart
A.$x \in (-\infty,-15) \cup (-2,8)$\\
B.$x \in (-\infty,-15) \cup (-2,8]$\\
C.$x \in (-\infty,-15) \cup [-2,8)$\\
D.$x \in (-\infty,-15] \cup (-2,8)$\\
E.$x \in (-\infty,-15] \cup (-2,8]$\\
F.$x \in (-\infty,-15] \cup [-2,8)$\\
G.$x \in (-\infty,-15) \cup [-2,8]$\\
H.$x \in (-\infty,-15] \cup [-2,8]$
\testStop
\kluczStart
A
\kluczStop



\zadStart{Zadanie z Wikieł Z 1.62 b) moja wersja nr 388}

Rozwiązać nierówności $(x+15)(8-x)(x+3)\ge0$.
\zadStop
\rozwStart{Patryk Wirkus}{}
Miejsca zerowe naszego wielomianu to: $-15, 8, -3$.\\
Wielomian jest stopnia nieparzystego, ponadto znak współczynnika przy\linebreak najwyższej potędze x jest ujemny.\\ W związku z tym wykres wielomianu zaczyna się od lewej strony powyżej osi OX. A więc $$x \in (-\infty,-15) \cup (-3,8).$$
\rozwStop
\odpStart
$x \in (-\infty,-15) \cup (-3,8)$
\odpStop
\testStart
A.$x \in (-\infty,-15) \cup (-3,8)$\\
B.$x \in (-\infty,-15) \cup (-3,8]$\\
C.$x \in (-\infty,-15) \cup [-3,8)$\\
D.$x \in (-\infty,-15] \cup (-3,8)$\\
E.$x \in (-\infty,-15] \cup (-3,8]$\\
F.$x \in (-\infty,-15] \cup [-3,8)$\\
G.$x \in (-\infty,-15) \cup [-3,8]$\\
H.$x \in (-\infty,-15] \cup [-3,8]$
\testStop
\kluczStart
A
\kluczStop



\zadStart{Zadanie z Wikieł Z 1.62 b) moja wersja nr 389}

Rozwiązać nierówności $(x+15)(8-x)(x+4)\ge0$.
\zadStop
\rozwStart{Patryk Wirkus}{}
Miejsca zerowe naszego wielomianu to: $-15, 8, -4$.\\
Wielomian jest stopnia nieparzystego, ponadto znak współczynnika przy\linebreak najwyższej potędze x jest ujemny.\\ W związku z tym wykres wielomianu zaczyna się od lewej strony powyżej osi OX. A więc $$x \in (-\infty,-15) \cup (-4,8).$$
\rozwStop
\odpStart
$x \in (-\infty,-15) \cup (-4,8)$
\odpStop
\testStart
A.$x \in (-\infty,-15) \cup (-4,8)$\\
B.$x \in (-\infty,-15) \cup (-4,8]$\\
C.$x \in (-\infty,-15) \cup [-4,8)$\\
D.$x \in (-\infty,-15] \cup (-4,8)$\\
E.$x \in (-\infty,-15] \cup (-4,8]$\\
F.$x \in (-\infty,-15] \cup [-4,8)$\\
G.$x \in (-\infty,-15) \cup [-4,8]$\\
H.$x \in (-\infty,-15] \cup [-4,8]$
\testStop
\kluczStart
A
\kluczStop



\zadStart{Zadanie z Wikieł Z 1.62 b) moja wersja nr 390}

Rozwiązać nierówności $(x+15)(8-x)(x+5)\ge0$.
\zadStop
\rozwStart{Patryk Wirkus}{}
Miejsca zerowe naszego wielomianu to: $-15, 8, -5$.\\
Wielomian jest stopnia nieparzystego, ponadto znak współczynnika przy\linebreak najwyższej potędze x jest ujemny.\\ W związku z tym wykres wielomianu zaczyna się od lewej strony powyżej osi OX. A więc $$x \in (-\infty,-15) \cup (-5,8).$$
\rozwStop
\odpStart
$x \in (-\infty,-15) \cup (-5,8)$
\odpStop
\testStart
A.$x \in (-\infty,-15) \cup (-5,8)$\\
B.$x \in (-\infty,-15) \cup (-5,8]$\\
C.$x \in (-\infty,-15) \cup [-5,8)$\\
D.$x \in (-\infty,-15] \cup (-5,8)$\\
E.$x \in (-\infty,-15] \cup (-5,8]$\\
F.$x \in (-\infty,-15] \cup [-5,8)$\\
G.$x \in (-\infty,-15) \cup [-5,8]$\\
H.$x \in (-\infty,-15] \cup [-5,8]$
\testStop
\kluczStart
A
\kluczStop



\zadStart{Zadanie z Wikieł Z 1.62 b) moja wersja nr 391}

Rozwiązać nierówności $(x+15)(8-x)(x+6)\ge0$.
\zadStop
\rozwStart{Patryk Wirkus}{}
Miejsca zerowe naszego wielomianu to: $-15, 8, -6$.\\
Wielomian jest stopnia nieparzystego, ponadto znak współczynnika przy\linebreak najwyższej potędze x jest ujemny.\\ W związku z tym wykres wielomianu zaczyna się od lewej strony powyżej osi OX. A więc $$x \in (-\infty,-15) \cup (-6,8).$$
\rozwStop
\odpStart
$x \in (-\infty,-15) \cup (-6,8)$
\odpStop
\testStart
A.$x \in (-\infty,-15) \cup (-6,8)$\\
B.$x \in (-\infty,-15) \cup (-6,8]$\\
C.$x \in (-\infty,-15) \cup [-6,8)$\\
D.$x \in (-\infty,-15] \cup (-6,8)$\\
E.$x \in (-\infty,-15] \cup (-6,8]$\\
F.$x \in (-\infty,-15] \cup [-6,8)$\\
G.$x \in (-\infty,-15) \cup [-6,8]$\\
H.$x \in (-\infty,-15] \cup [-6,8]$
\testStop
\kluczStart
A
\kluczStop



\zadStart{Zadanie z Wikieł Z 1.62 b) moja wersja nr 392}

Rozwiązać nierówności $(x+15)(8-x)(x+7)\ge0$.
\zadStop
\rozwStart{Patryk Wirkus}{}
Miejsca zerowe naszego wielomianu to: $-15, 8, -7$.\\
Wielomian jest stopnia nieparzystego, ponadto znak współczynnika przy\linebreak najwyższej potędze x jest ujemny.\\ W związku z tym wykres wielomianu zaczyna się od lewej strony powyżej osi OX. A więc $$x \in (-\infty,-15) \cup (-7,8).$$
\rozwStop
\odpStart
$x \in (-\infty,-15) \cup (-7,8)$
\odpStop
\testStart
A.$x \in (-\infty,-15) \cup (-7,8)$\\
B.$x \in (-\infty,-15) \cup (-7,8]$\\
C.$x \in (-\infty,-15) \cup [-7,8)$\\
D.$x \in (-\infty,-15] \cup (-7,8)$\\
E.$x \in (-\infty,-15] \cup (-7,8]$\\
F.$x \in (-\infty,-15] \cup [-7,8)$\\
G.$x \in (-\infty,-15) \cup [-7,8]$\\
H.$x \in (-\infty,-15] \cup [-7,8]$
\testStop
\kluczStart
A
\kluczStop



\zadStart{Zadanie z Wikieł Z 1.62 b) moja wersja nr 393}

Rozwiązać nierówności $(x+15)(9-x)(x+1)\ge0$.
\zadStop
\rozwStart{Patryk Wirkus}{}
Miejsca zerowe naszego wielomianu to: $-15, 9, -1$.\\
Wielomian jest stopnia nieparzystego, ponadto znak współczynnika przy\linebreak najwyższej potędze x jest ujemny.\\ W związku z tym wykres wielomianu zaczyna się od lewej strony powyżej osi OX. A więc $$x \in (-\infty,-15) \cup (-1,9).$$
\rozwStop
\odpStart
$x \in (-\infty,-15) \cup (-1,9)$
\odpStop
\testStart
A.$x \in (-\infty,-15) \cup (-1,9)$\\
B.$x \in (-\infty,-15) \cup (-1,9]$\\
C.$x \in (-\infty,-15) \cup [-1,9)$\\
D.$x \in (-\infty,-15] \cup (-1,9)$\\
E.$x \in (-\infty,-15] \cup (-1,9]$\\
F.$x \in (-\infty,-15] \cup [-1,9)$\\
G.$x \in (-\infty,-15) \cup [-1,9]$\\
H.$x \in (-\infty,-15] \cup [-1,9]$
\testStop
\kluczStart
A
\kluczStop



\zadStart{Zadanie z Wikieł Z 1.62 b) moja wersja nr 394}

Rozwiązać nierówności $(x+15)(9-x)(x+2)\ge0$.
\zadStop
\rozwStart{Patryk Wirkus}{}
Miejsca zerowe naszego wielomianu to: $-15, 9, -2$.\\
Wielomian jest stopnia nieparzystego, ponadto znak współczynnika przy\linebreak najwyższej potędze x jest ujemny.\\ W związku z tym wykres wielomianu zaczyna się od lewej strony powyżej osi OX. A więc $$x \in (-\infty,-15) \cup (-2,9).$$
\rozwStop
\odpStart
$x \in (-\infty,-15) \cup (-2,9)$
\odpStop
\testStart
A.$x \in (-\infty,-15) \cup (-2,9)$\\
B.$x \in (-\infty,-15) \cup (-2,9]$\\
C.$x \in (-\infty,-15) \cup [-2,9)$\\
D.$x \in (-\infty,-15] \cup (-2,9)$\\
E.$x \in (-\infty,-15] \cup (-2,9]$\\
F.$x \in (-\infty,-15] \cup [-2,9)$\\
G.$x \in (-\infty,-15) \cup [-2,9]$\\
H.$x \in (-\infty,-15] \cup [-2,9]$
\testStop
\kluczStart
A
\kluczStop



\zadStart{Zadanie z Wikieł Z 1.62 b) moja wersja nr 395}

Rozwiązać nierówności $(x+15)(9-x)(x+3)\ge0$.
\zadStop
\rozwStart{Patryk Wirkus}{}
Miejsca zerowe naszego wielomianu to: $-15, 9, -3$.\\
Wielomian jest stopnia nieparzystego, ponadto znak współczynnika przy\linebreak najwyższej potędze x jest ujemny.\\ W związku z tym wykres wielomianu zaczyna się od lewej strony powyżej osi OX. A więc $$x \in (-\infty,-15) \cup (-3,9).$$
\rozwStop
\odpStart
$x \in (-\infty,-15) \cup (-3,9)$
\odpStop
\testStart
A.$x \in (-\infty,-15) \cup (-3,9)$\\
B.$x \in (-\infty,-15) \cup (-3,9]$\\
C.$x \in (-\infty,-15) \cup [-3,9)$\\
D.$x \in (-\infty,-15] \cup (-3,9)$\\
E.$x \in (-\infty,-15] \cup (-3,9]$\\
F.$x \in (-\infty,-15] \cup [-3,9)$\\
G.$x \in (-\infty,-15) \cup [-3,9]$\\
H.$x \in (-\infty,-15] \cup [-3,9]$
\testStop
\kluczStart
A
\kluczStop



\zadStart{Zadanie z Wikieł Z 1.62 b) moja wersja nr 396}

Rozwiązać nierówności $(x+15)(9-x)(x+4)\ge0$.
\zadStop
\rozwStart{Patryk Wirkus}{}
Miejsca zerowe naszego wielomianu to: $-15, 9, -4$.\\
Wielomian jest stopnia nieparzystego, ponadto znak współczynnika przy\linebreak najwyższej potędze x jest ujemny.\\ W związku z tym wykres wielomianu zaczyna się od lewej strony powyżej osi OX. A więc $$x \in (-\infty,-15) \cup (-4,9).$$
\rozwStop
\odpStart
$x \in (-\infty,-15) \cup (-4,9)$
\odpStop
\testStart
A.$x \in (-\infty,-15) \cup (-4,9)$\\
B.$x \in (-\infty,-15) \cup (-4,9]$\\
C.$x \in (-\infty,-15) \cup [-4,9)$\\
D.$x \in (-\infty,-15] \cup (-4,9)$\\
E.$x \in (-\infty,-15] \cup (-4,9]$\\
F.$x \in (-\infty,-15] \cup [-4,9)$\\
G.$x \in (-\infty,-15) \cup [-4,9]$\\
H.$x \in (-\infty,-15] \cup [-4,9]$
\testStop
\kluczStart
A
\kluczStop



\zadStart{Zadanie z Wikieł Z 1.62 b) moja wersja nr 397}

Rozwiązać nierówności $(x+15)(9-x)(x+5)\ge0$.
\zadStop
\rozwStart{Patryk Wirkus}{}
Miejsca zerowe naszego wielomianu to: $-15, 9, -5$.\\
Wielomian jest stopnia nieparzystego, ponadto znak współczynnika przy\linebreak najwyższej potędze x jest ujemny.\\ W związku z tym wykres wielomianu zaczyna się od lewej strony powyżej osi OX. A więc $$x \in (-\infty,-15) \cup (-5,9).$$
\rozwStop
\odpStart
$x \in (-\infty,-15) \cup (-5,9)$
\odpStop
\testStart
A.$x \in (-\infty,-15) \cup (-5,9)$\\
B.$x \in (-\infty,-15) \cup (-5,9]$\\
C.$x \in (-\infty,-15) \cup [-5,9)$\\
D.$x \in (-\infty,-15] \cup (-5,9)$\\
E.$x \in (-\infty,-15] \cup (-5,9]$\\
F.$x \in (-\infty,-15] \cup [-5,9)$\\
G.$x \in (-\infty,-15) \cup [-5,9]$\\
H.$x \in (-\infty,-15] \cup [-5,9]$
\testStop
\kluczStart
A
\kluczStop



\zadStart{Zadanie z Wikieł Z 1.62 b) moja wersja nr 398}

Rozwiązać nierówności $(x+15)(9-x)(x+6)\ge0$.
\zadStop
\rozwStart{Patryk Wirkus}{}
Miejsca zerowe naszego wielomianu to: $-15, 9, -6$.\\
Wielomian jest stopnia nieparzystego, ponadto znak współczynnika przy\linebreak najwyższej potędze x jest ujemny.\\ W związku z tym wykres wielomianu zaczyna się od lewej strony powyżej osi OX. A więc $$x \in (-\infty,-15) \cup (-6,9).$$
\rozwStop
\odpStart
$x \in (-\infty,-15) \cup (-6,9)$
\odpStop
\testStart
A.$x \in (-\infty,-15) \cup (-6,9)$\\
B.$x \in (-\infty,-15) \cup (-6,9]$\\
C.$x \in (-\infty,-15) \cup [-6,9)$\\
D.$x \in (-\infty,-15] \cup (-6,9)$\\
E.$x \in (-\infty,-15] \cup (-6,9]$\\
F.$x \in (-\infty,-15] \cup [-6,9)$\\
G.$x \in (-\infty,-15) \cup [-6,9]$\\
H.$x \in (-\infty,-15] \cup [-6,9]$
\testStop
\kluczStart
A
\kluczStop



\zadStart{Zadanie z Wikieł Z 1.62 b) moja wersja nr 399}

Rozwiązać nierówności $(x+15)(9-x)(x+7)\ge0$.
\zadStop
\rozwStart{Patryk Wirkus}{}
Miejsca zerowe naszego wielomianu to: $-15, 9, -7$.\\
Wielomian jest stopnia nieparzystego, ponadto znak współczynnika przy\linebreak najwyższej potędze x jest ujemny.\\ W związku z tym wykres wielomianu zaczyna się od lewej strony powyżej osi OX. A więc $$x \in (-\infty,-15) \cup (-7,9).$$
\rozwStop
\odpStart
$x \in (-\infty,-15) \cup (-7,9)$
\odpStop
\testStart
A.$x \in (-\infty,-15) \cup (-7,9)$\\
B.$x \in (-\infty,-15) \cup (-7,9]$\\
C.$x \in (-\infty,-15) \cup [-7,9)$\\
D.$x \in (-\infty,-15] \cup (-7,9)$\\
E.$x \in (-\infty,-15] \cup (-7,9]$\\
F.$x \in (-\infty,-15] \cup [-7,9)$\\
G.$x \in (-\infty,-15) \cup [-7,9]$\\
H.$x \in (-\infty,-15] \cup [-7,9]$
\testStop
\kluczStart
A
\kluczStop



\zadStart{Zadanie z Wikieł Z 1.62 b) moja wersja nr 400}

Rozwiązać nierówności $(x+15)(9-x)(x+8)\ge0$.
\zadStop
\rozwStart{Patryk Wirkus}{}
Miejsca zerowe naszego wielomianu to: $-15, 9, -8$.\\
Wielomian jest stopnia nieparzystego, ponadto znak współczynnika przy\linebreak najwyższej potędze x jest ujemny.\\ W związku z tym wykres wielomianu zaczyna się od lewej strony powyżej osi OX. A więc $$x \in (-\infty,-15) \cup (-8,9).$$
\rozwStop
\odpStart
$x \in (-\infty,-15) \cup (-8,9)$
\odpStop
\testStart
A.$x \in (-\infty,-15) \cup (-8,9)$\\
B.$x \in (-\infty,-15) \cup (-8,9]$\\
C.$x \in (-\infty,-15) \cup [-8,9)$\\
D.$x \in (-\infty,-15] \cup (-8,9)$\\
E.$x \in (-\infty,-15] \cup (-8,9]$\\
F.$x \in (-\infty,-15] \cup [-8,9)$\\
G.$x \in (-\infty,-15) \cup [-8,9]$\\
H.$x \in (-\infty,-15] \cup [-8,9]$
\testStop
\kluczStart
A
\kluczStop



\zadStart{Zadanie z Wikieł Z 1.62 b) moja wersja nr 401}

Rozwiązać nierówności $(x+15)(10-x)(x+1)\ge0$.
\zadStop
\rozwStart{Patryk Wirkus}{}
Miejsca zerowe naszego wielomianu to: $-15, 10, -1$.\\
Wielomian jest stopnia nieparzystego, ponadto znak współczynnika przy\linebreak najwyższej potędze x jest ujemny.\\ W związku z tym wykres wielomianu zaczyna się od lewej strony powyżej osi OX. A więc $$x \in (-\infty,-15) \cup (-1,10).$$
\rozwStop
\odpStart
$x \in (-\infty,-15) \cup (-1,10)$
\odpStop
\testStart
A.$x \in (-\infty,-15) \cup (-1,10)$\\
B.$x \in (-\infty,-15) \cup (-1,10]$\\
C.$x \in (-\infty,-15) \cup [-1,10)$\\
D.$x \in (-\infty,-15] \cup (-1,10)$\\
E.$x \in (-\infty,-15] \cup (-1,10]$\\
F.$x \in (-\infty,-15] \cup [-1,10)$\\
G.$x \in (-\infty,-15) \cup [-1,10]$\\
H.$x \in (-\infty,-15] \cup [-1,10]$
\testStop
\kluczStart
A
\kluczStop



\zadStart{Zadanie z Wikieł Z 1.62 b) moja wersja nr 402}

Rozwiązać nierówności $(x+15)(10-x)(x+2)\ge0$.
\zadStop
\rozwStart{Patryk Wirkus}{}
Miejsca zerowe naszego wielomianu to: $-15, 10, -2$.\\
Wielomian jest stopnia nieparzystego, ponadto znak współczynnika przy\linebreak najwyższej potędze x jest ujemny.\\ W związku z tym wykres wielomianu zaczyna się od lewej strony powyżej osi OX. A więc $$x \in (-\infty,-15) \cup (-2,10).$$
\rozwStop
\odpStart
$x \in (-\infty,-15) \cup (-2,10)$
\odpStop
\testStart
A.$x \in (-\infty,-15) \cup (-2,10)$\\
B.$x \in (-\infty,-15) \cup (-2,10]$\\
C.$x \in (-\infty,-15) \cup [-2,10)$\\
D.$x \in (-\infty,-15] \cup (-2,10)$\\
E.$x \in (-\infty,-15] \cup (-2,10]$\\
F.$x \in (-\infty,-15] \cup [-2,10)$\\
G.$x \in (-\infty,-15) \cup [-2,10]$\\
H.$x \in (-\infty,-15] \cup [-2,10]$
\testStop
\kluczStart
A
\kluczStop



\zadStart{Zadanie z Wikieł Z 1.62 b) moja wersja nr 403}

Rozwiązać nierówności $(x+15)(10-x)(x+3)\ge0$.
\zadStop
\rozwStart{Patryk Wirkus}{}
Miejsca zerowe naszego wielomianu to: $-15, 10, -3$.\\
Wielomian jest stopnia nieparzystego, ponadto znak współczynnika przy\linebreak najwyższej potędze x jest ujemny.\\ W związku z tym wykres wielomianu zaczyna się od lewej strony powyżej osi OX. A więc $$x \in (-\infty,-15) \cup (-3,10).$$
\rozwStop
\odpStart
$x \in (-\infty,-15) \cup (-3,10)$
\odpStop
\testStart
A.$x \in (-\infty,-15) \cup (-3,10)$\\
B.$x \in (-\infty,-15) \cup (-3,10]$\\
C.$x \in (-\infty,-15) \cup [-3,10)$\\
D.$x \in (-\infty,-15] \cup (-3,10)$\\
E.$x \in (-\infty,-15] \cup (-3,10]$\\
F.$x \in (-\infty,-15] \cup [-3,10)$\\
G.$x \in (-\infty,-15) \cup [-3,10]$\\
H.$x \in (-\infty,-15] \cup [-3,10]$
\testStop
\kluczStart
A
\kluczStop



\zadStart{Zadanie z Wikieł Z 1.62 b) moja wersja nr 404}

Rozwiązać nierówności $(x+15)(10-x)(x+4)\ge0$.
\zadStop
\rozwStart{Patryk Wirkus}{}
Miejsca zerowe naszego wielomianu to: $-15, 10, -4$.\\
Wielomian jest stopnia nieparzystego, ponadto znak współczynnika przy\linebreak najwyższej potędze x jest ujemny.\\ W związku z tym wykres wielomianu zaczyna się od lewej strony powyżej osi OX. A więc $$x \in (-\infty,-15) \cup (-4,10).$$
\rozwStop
\odpStart
$x \in (-\infty,-15) \cup (-4,10)$
\odpStop
\testStart
A.$x \in (-\infty,-15) \cup (-4,10)$\\
B.$x \in (-\infty,-15) \cup (-4,10]$\\
C.$x \in (-\infty,-15) \cup [-4,10)$\\
D.$x \in (-\infty,-15] \cup (-4,10)$\\
E.$x \in (-\infty,-15] \cup (-4,10]$\\
F.$x \in (-\infty,-15] \cup [-4,10)$\\
G.$x \in (-\infty,-15) \cup [-4,10]$\\
H.$x \in (-\infty,-15] \cup [-4,10]$
\testStop
\kluczStart
A
\kluczStop



\zadStart{Zadanie z Wikieł Z 1.62 b) moja wersja nr 405}

Rozwiązać nierówności $(x+15)(10-x)(x+5)\ge0$.
\zadStop
\rozwStart{Patryk Wirkus}{}
Miejsca zerowe naszego wielomianu to: $-15, 10, -5$.\\
Wielomian jest stopnia nieparzystego, ponadto znak współczynnika przy\linebreak najwyższej potędze x jest ujemny.\\ W związku z tym wykres wielomianu zaczyna się od lewej strony powyżej osi OX. A więc $$x \in (-\infty,-15) \cup (-5,10).$$
\rozwStop
\odpStart
$x \in (-\infty,-15) \cup (-5,10)$
\odpStop
\testStart
A.$x \in (-\infty,-15) \cup (-5,10)$\\
B.$x \in (-\infty,-15) \cup (-5,10]$\\
C.$x \in (-\infty,-15) \cup [-5,10)$\\
D.$x \in (-\infty,-15] \cup (-5,10)$\\
E.$x \in (-\infty,-15] \cup (-5,10]$\\
F.$x \in (-\infty,-15] \cup [-5,10)$\\
G.$x \in (-\infty,-15) \cup [-5,10]$\\
H.$x \in (-\infty,-15] \cup [-5,10]$
\testStop
\kluczStart
A
\kluczStop



\zadStart{Zadanie z Wikieł Z 1.62 b) moja wersja nr 406}

Rozwiązać nierówności $(x+15)(10-x)(x+6)\ge0$.
\zadStop
\rozwStart{Patryk Wirkus}{}
Miejsca zerowe naszego wielomianu to: $-15, 10, -6$.\\
Wielomian jest stopnia nieparzystego, ponadto znak współczynnika przy\linebreak najwyższej potędze x jest ujemny.\\ W związku z tym wykres wielomianu zaczyna się od lewej strony powyżej osi OX. A więc $$x \in (-\infty,-15) \cup (-6,10).$$
\rozwStop
\odpStart
$x \in (-\infty,-15) \cup (-6,10)$
\odpStop
\testStart
A.$x \in (-\infty,-15) \cup (-6,10)$\\
B.$x \in (-\infty,-15) \cup (-6,10]$\\
C.$x \in (-\infty,-15) \cup [-6,10)$\\
D.$x \in (-\infty,-15] \cup (-6,10)$\\
E.$x \in (-\infty,-15] \cup (-6,10]$\\
F.$x \in (-\infty,-15] \cup [-6,10)$\\
G.$x \in (-\infty,-15) \cup [-6,10]$\\
H.$x \in (-\infty,-15] \cup [-6,10]$
\testStop
\kluczStart
A
\kluczStop



\zadStart{Zadanie z Wikieł Z 1.62 b) moja wersja nr 407}

Rozwiązać nierówności $(x+15)(10-x)(x+7)\ge0$.
\zadStop
\rozwStart{Patryk Wirkus}{}
Miejsca zerowe naszego wielomianu to: $-15, 10, -7$.\\
Wielomian jest stopnia nieparzystego, ponadto znak współczynnika przy\linebreak najwyższej potędze x jest ujemny.\\ W związku z tym wykres wielomianu zaczyna się od lewej strony powyżej osi OX. A więc $$x \in (-\infty,-15) \cup (-7,10).$$
\rozwStop
\odpStart
$x \in (-\infty,-15) \cup (-7,10)$
\odpStop
\testStart
A.$x \in (-\infty,-15) \cup (-7,10)$\\
B.$x \in (-\infty,-15) \cup (-7,10]$\\
C.$x \in (-\infty,-15) \cup [-7,10)$\\
D.$x \in (-\infty,-15] \cup (-7,10)$\\
E.$x \in (-\infty,-15] \cup (-7,10]$\\
F.$x \in (-\infty,-15] \cup [-7,10)$\\
G.$x \in (-\infty,-15) \cup [-7,10]$\\
H.$x \in (-\infty,-15] \cup [-7,10]$
\testStop
\kluczStart
A
\kluczStop



\zadStart{Zadanie z Wikieł Z 1.62 b) moja wersja nr 408}

Rozwiązać nierówności $(x+15)(10-x)(x+8)\ge0$.
\zadStop
\rozwStart{Patryk Wirkus}{}
Miejsca zerowe naszego wielomianu to: $-15, 10, -8$.\\
Wielomian jest stopnia nieparzystego, ponadto znak współczynnika przy\linebreak najwyższej potędze x jest ujemny.\\ W związku z tym wykres wielomianu zaczyna się od lewej strony powyżej osi OX. A więc $$x \in (-\infty,-15) \cup (-8,10).$$
\rozwStop
\odpStart
$x \in (-\infty,-15) \cup (-8,10)$
\odpStop
\testStart
A.$x \in (-\infty,-15) \cup (-8,10)$\\
B.$x \in (-\infty,-15) \cup (-8,10]$\\
C.$x \in (-\infty,-15) \cup [-8,10)$\\
D.$x \in (-\infty,-15] \cup (-8,10)$\\
E.$x \in (-\infty,-15] \cup (-8,10]$\\
F.$x \in (-\infty,-15] \cup [-8,10)$\\
G.$x \in (-\infty,-15) \cup [-8,10]$\\
H.$x \in (-\infty,-15] \cup [-8,10]$
\testStop
\kluczStart
A
\kluczStop



\zadStart{Zadanie z Wikieł Z 1.62 b) moja wersja nr 409}

Rozwiązać nierówności $(x+15)(10-x)(x+9)\ge0$.
\zadStop
\rozwStart{Patryk Wirkus}{}
Miejsca zerowe naszego wielomianu to: $-15, 10, -9$.\\
Wielomian jest stopnia nieparzystego, ponadto znak współczynnika przy\linebreak najwyższej potędze x jest ujemny.\\ W związku z tym wykres wielomianu zaczyna się od lewej strony powyżej osi OX. A więc $$x \in (-\infty,-15) \cup (-9,10).$$
\rozwStop
\odpStart
$x \in (-\infty,-15) \cup (-9,10)$
\odpStop
\testStart
A.$x \in (-\infty,-15) \cup (-9,10)$\\
B.$x \in (-\infty,-15) \cup (-9,10]$\\
C.$x \in (-\infty,-15) \cup [-9,10)$\\
D.$x \in (-\infty,-15] \cup (-9,10)$\\
E.$x \in (-\infty,-15] \cup (-9,10]$\\
F.$x \in (-\infty,-15] \cup [-9,10)$\\
G.$x \in (-\infty,-15) \cup [-9,10]$\\
H.$x \in (-\infty,-15] \cup [-9,10]$
\testStop
\kluczStart
A
\kluczStop



\zadStart{Zadanie z Wikieł Z 1.62 b) moja wersja nr 410}

Rozwiązać nierówności $(x+15)(11-x)(x+1)\ge0$.
\zadStop
\rozwStart{Patryk Wirkus}{}
Miejsca zerowe naszego wielomianu to: $-15, 11, -1$.\\
Wielomian jest stopnia nieparzystego, ponadto znak współczynnika przy\linebreak najwyższej potędze x jest ujemny.\\ W związku z tym wykres wielomianu zaczyna się od lewej strony powyżej osi OX. A więc $$x \in (-\infty,-15) \cup (-1,11).$$
\rozwStop
\odpStart
$x \in (-\infty,-15) \cup (-1,11)$
\odpStop
\testStart
A.$x \in (-\infty,-15) \cup (-1,11)$\\
B.$x \in (-\infty,-15) \cup (-1,11]$\\
C.$x \in (-\infty,-15) \cup [-1,11)$\\
D.$x \in (-\infty,-15] \cup (-1,11)$\\
E.$x \in (-\infty,-15] \cup (-1,11]$\\
F.$x \in (-\infty,-15] \cup [-1,11)$\\
G.$x \in (-\infty,-15) \cup [-1,11]$\\
H.$x \in (-\infty,-15] \cup [-1,11]$
\testStop
\kluczStart
A
\kluczStop



\zadStart{Zadanie z Wikieł Z 1.62 b) moja wersja nr 411}

Rozwiązać nierówności $(x+15)(11-x)(x+2)\ge0$.
\zadStop
\rozwStart{Patryk Wirkus}{}
Miejsca zerowe naszego wielomianu to: $-15, 11, -2$.\\
Wielomian jest stopnia nieparzystego, ponadto znak współczynnika przy\linebreak najwyższej potędze x jest ujemny.\\ W związku z tym wykres wielomianu zaczyna się od lewej strony powyżej osi OX. A więc $$x \in (-\infty,-15) \cup (-2,11).$$
\rozwStop
\odpStart
$x \in (-\infty,-15) \cup (-2,11)$
\odpStop
\testStart
A.$x \in (-\infty,-15) \cup (-2,11)$\\
B.$x \in (-\infty,-15) \cup (-2,11]$\\
C.$x \in (-\infty,-15) \cup [-2,11)$\\
D.$x \in (-\infty,-15] \cup (-2,11)$\\
E.$x \in (-\infty,-15] \cup (-2,11]$\\
F.$x \in (-\infty,-15] \cup [-2,11)$\\
G.$x \in (-\infty,-15) \cup [-2,11]$\\
H.$x \in (-\infty,-15] \cup [-2,11]$
\testStop
\kluczStart
A
\kluczStop



\zadStart{Zadanie z Wikieł Z 1.62 b) moja wersja nr 412}

Rozwiązać nierówności $(x+15)(11-x)(x+3)\ge0$.
\zadStop
\rozwStart{Patryk Wirkus}{}
Miejsca zerowe naszego wielomianu to: $-15, 11, -3$.\\
Wielomian jest stopnia nieparzystego, ponadto znak współczynnika przy\linebreak najwyższej potędze x jest ujemny.\\ W związku z tym wykres wielomianu zaczyna się od lewej strony powyżej osi OX. A więc $$x \in (-\infty,-15) \cup (-3,11).$$
\rozwStop
\odpStart
$x \in (-\infty,-15) \cup (-3,11)$
\odpStop
\testStart
A.$x \in (-\infty,-15) \cup (-3,11)$\\
B.$x \in (-\infty,-15) \cup (-3,11]$\\
C.$x \in (-\infty,-15) \cup [-3,11)$\\
D.$x \in (-\infty,-15] \cup (-3,11)$\\
E.$x \in (-\infty,-15] \cup (-3,11]$\\
F.$x \in (-\infty,-15] \cup [-3,11)$\\
G.$x \in (-\infty,-15) \cup [-3,11]$\\
H.$x \in (-\infty,-15] \cup [-3,11]$
\testStop
\kluczStart
A
\kluczStop



\zadStart{Zadanie z Wikieł Z 1.62 b) moja wersja nr 413}

Rozwiązać nierówności $(x+15)(11-x)(x+4)\ge0$.
\zadStop
\rozwStart{Patryk Wirkus}{}
Miejsca zerowe naszego wielomianu to: $-15, 11, -4$.\\
Wielomian jest stopnia nieparzystego, ponadto znak współczynnika przy\linebreak najwyższej potędze x jest ujemny.\\ W związku z tym wykres wielomianu zaczyna się od lewej strony powyżej osi OX. A więc $$x \in (-\infty,-15) \cup (-4,11).$$
\rozwStop
\odpStart
$x \in (-\infty,-15) \cup (-4,11)$
\odpStop
\testStart
A.$x \in (-\infty,-15) \cup (-4,11)$\\
B.$x \in (-\infty,-15) \cup (-4,11]$\\
C.$x \in (-\infty,-15) \cup [-4,11)$\\
D.$x \in (-\infty,-15] \cup (-4,11)$\\
E.$x \in (-\infty,-15] \cup (-4,11]$\\
F.$x \in (-\infty,-15] \cup [-4,11)$\\
G.$x \in (-\infty,-15) \cup [-4,11]$\\
H.$x \in (-\infty,-15] \cup [-4,11]$
\testStop
\kluczStart
A
\kluczStop



\zadStart{Zadanie z Wikieł Z 1.62 b) moja wersja nr 414}

Rozwiązać nierówności $(x+15)(11-x)(x+5)\ge0$.
\zadStop
\rozwStart{Patryk Wirkus}{}
Miejsca zerowe naszego wielomianu to: $-15, 11, -5$.\\
Wielomian jest stopnia nieparzystego, ponadto znak współczynnika przy\linebreak najwyższej potędze x jest ujemny.\\ W związku z tym wykres wielomianu zaczyna się od lewej strony powyżej osi OX. A więc $$x \in (-\infty,-15) \cup (-5,11).$$
\rozwStop
\odpStart
$x \in (-\infty,-15) \cup (-5,11)$
\odpStop
\testStart
A.$x \in (-\infty,-15) \cup (-5,11)$\\
B.$x \in (-\infty,-15) \cup (-5,11]$\\
C.$x \in (-\infty,-15) \cup [-5,11)$\\
D.$x \in (-\infty,-15] \cup (-5,11)$\\
E.$x \in (-\infty,-15] \cup (-5,11]$\\
F.$x \in (-\infty,-15] \cup [-5,11)$\\
G.$x \in (-\infty,-15) \cup [-5,11]$\\
H.$x \in (-\infty,-15] \cup [-5,11]$
\testStop
\kluczStart
A
\kluczStop



\zadStart{Zadanie z Wikieł Z 1.62 b) moja wersja nr 415}

Rozwiązać nierówności $(x+15)(11-x)(x+6)\ge0$.
\zadStop
\rozwStart{Patryk Wirkus}{}
Miejsca zerowe naszego wielomianu to: $-15, 11, -6$.\\
Wielomian jest stopnia nieparzystego, ponadto znak współczynnika przy\linebreak najwyższej potędze x jest ujemny.\\ W związku z tym wykres wielomianu zaczyna się od lewej strony powyżej osi OX. A więc $$x \in (-\infty,-15) \cup (-6,11).$$
\rozwStop
\odpStart
$x \in (-\infty,-15) \cup (-6,11)$
\odpStop
\testStart
A.$x \in (-\infty,-15) \cup (-6,11)$\\
B.$x \in (-\infty,-15) \cup (-6,11]$\\
C.$x \in (-\infty,-15) \cup [-6,11)$\\
D.$x \in (-\infty,-15] \cup (-6,11)$\\
E.$x \in (-\infty,-15] \cup (-6,11]$\\
F.$x \in (-\infty,-15] \cup [-6,11)$\\
G.$x \in (-\infty,-15) \cup [-6,11]$\\
H.$x \in (-\infty,-15] \cup [-6,11]$
\testStop
\kluczStart
A
\kluczStop



\zadStart{Zadanie z Wikieł Z 1.62 b) moja wersja nr 416}

Rozwiązać nierówności $(x+15)(11-x)(x+7)\ge0$.
\zadStop
\rozwStart{Patryk Wirkus}{}
Miejsca zerowe naszego wielomianu to: $-15, 11, -7$.\\
Wielomian jest stopnia nieparzystego, ponadto znak współczynnika przy\linebreak najwyższej potędze x jest ujemny.\\ W związku z tym wykres wielomianu zaczyna się od lewej strony powyżej osi OX. A więc $$x \in (-\infty,-15) \cup (-7,11).$$
\rozwStop
\odpStart
$x \in (-\infty,-15) \cup (-7,11)$
\odpStop
\testStart
A.$x \in (-\infty,-15) \cup (-7,11)$\\
B.$x \in (-\infty,-15) \cup (-7,11]$\\
C.$x \in (-\infty,-15) \cup [-7,11)$\\
D.$x \in (-\infty,-15] \cup (-7,11)$\\
E.$x \in (-\infty,-15] \cup (-7,11]$\\
F.$x \in (-\infty,-15] \cup [-7,11)$\\
G.$x \in (-\infty,-15) \cup [-7,11]$\\
H.$x \in (-\infty,-15] \cup [-7,11]$
\testStop
\kluczStart
A
\kluczStop



\zadStart{Zadanie z Wikieł Z 1.62 b) moja wersja nr 417}

Rozwiązać nierówności $(x+15)(11-x)(x+8)\ge0$.
\zadStop
\rozwStart{Patryk Wirkus}{}
Miejsca zerowe naszego wielomianu to: $-15, 11, -8$.\\
Wielomian jest stopnia nieparzystego, ponadto znak współczynnika przy\linebreak najwyższej potędze x jest ujemny.\\ W związku z tym wykres wielomianu zaczyna się od lewej strony powyżej osi OX. A więc $$x \in (-\infty,-15) \cup (-8,11).$$
\rozwStop
\odpStart
$x \in (-\infty,-15) \cup (-8,11)$
\odpStop
\testStart
A.$x \in (-\infty,-15) \cup (-8,11)$\\
B.$x \in (-\infty,-15) \cup (-8,11]$\\
C.$x \in (-\infty,-15) \cup [-8,11)$\\
D.$x \in (-\infty,-15] \cup (-8,11)$\\
E.$x \in (-\infty,-15] \cup (-8,11]$\\
F.$x \in (-\infty,-15] \cup [-8,11)$\\
G.$x \in (-\infty,-15) \cup [-8,11]$\\
H.$x \in (-\infty,-15] \cup [-8,11]$
\testStop
\kluczStart
A
\kluczStop



\zadStart{Zadanie z Wikieł Z 1.62 b) moja wersja nr 418}

Rozwiązać nierówności $(x+15)(11-x)(x+9)\ge0$.
\zadStop
\rozwStart{Patryk Wirkus}{}
Miejsca zerowe naszego wielomianu to: $-15, 11, -9$.\\
Wielomian jest stopnia nieparzystego, ponadto znak współczynnika przy\linebreak najwyższej potędze x jest ujemny.\\ W związku z tym wykres wielomianu zaczyna się od lewej strony powyżej osi OX. A więc $$x \in (-\infty,-15) \cup (-9,11).$$
\rozwStop
\odpStart
$x \in (-\infty,-15) \cup (-9,11)$
\odpStop
\testStart
A.$x \in (-\infty,-15) \cup (-9,11)$\\
B.$x \in (-\infty,-15) \cup (-9,11]$\\
C.$x \in (-\infty,-15) \cup [-9,11)$\\
D.$x \in (-\infty,-15] \cup (-9,11)$\\
E.$x \in (-\infty,-15] \cup (-9,11]$\\
F.$x \in (-\infty,-15] \cup [-9,11)$\\
G.$x \in (-\infty,-15) \cup [-9,11]$\\
H.$x \in (-\infty,-15] \cup [-9,11]$
\testStop
\kluczStart
A
\kluczStop



\zadStart{Zadanie z Wikieł Z 1.62 b) moja wersja nr 419}

Rozwiązać nierówności $(x+15)(11-x)(x+10)\ge0$.
\zadStop
\rozwStart{Patryk Wirkus}{}
Miejsca zerowe naszego wielomianu to: $-15, 11, -10$.\\
Wielomian jest stopnia nieparzystego, ponadto znak współczynnika przy\linebreak najwyższej potędze x jest ujemny.\\ W związku z tym wykres wielomianu zaczyna się od lewej strony powyżej osi OX. A więc $$x \in (-\infty,-15) \cup (-10,11).$$
\rozwStop
\odpStart
$x \in (-\infty,-15) \cup (-10,11)$
\odpStop
\testStart
A.$x \in (-\infty,-15) \cup (-10,11)$\\
B.$x \in (-\infty,-15) \cup (-10,11]$\\
C.$x \in (-\infty,-15) \cup [-10,11)$\\
D.$x \in (-\infty,-15] \cup (-10,11)$\\
E.$x \in (-\infty,-15] \cup (-10,11]$\\
F.$x \in (-\infty,-15] \cup [-10,11)$\\
G.$x \in (-\infty,-15) \cup [-10,11]$\\
H.$x \in (-\infty,-15] \cup [-10,11]$
\testStop
\kluczStart
A
\kluczStop



\zadStart{Zadanie z Wikieł Z 1.62 b) moja wersja nr 420}

Rozwiązać nierówności $(x+15)(12-x)(x+1)\ge0$.
\zadStop
\rozwStart{Patryk Wirkus}{}
Miejsca zerowe naszego wielomianu to: $-15, 12, -1$.\\
Wielomian jest stopnia nieparzystego, ponadto znak współczynnika przy\linebreak najwyższej potędze x jest ujemny.\\ W związku z tym wykres wielomianu zaczyna się od lewej strony powyżej osi OX. A więc $$x \in (-\infty,-15) \cup (-1,12).$$
\rozwStop
\odpStart
$x \in (-\infty,-15) \cup (-1,12)$
\odpStop
\testStart
A.$x \in (-\infty,-15) \cup (-1,12)$\\
B.$x \in (-\infty,-15) \cup (-1,12]$\\
C.$x \in (-\infty,-15) \cup [-1,12)$\\
D.$x \in (-\infty,-15] \cup (-1,12)$\\
E.$x \in (-\infty,-15] \cup (-1,12]$\\
F.$x \in (-\infty,-15] \cup [-1,12)$\\
G.$x \in (-\infty,-15) \cup [-1,12]$\\
H.$x \in (-\infty,-15] \cup [-1,12]$
\testStop
\kluczStart
A
\kluczStop



\zadStart{Zadanie z Wikieł Z 1.62 b) moja wersja nr 421}

Rozwiązać nierówności $(x+15)(12-x)(x+2)\ge0$.
\zadStop
\rozwStart{Patryk Wirkus}{}
Miejsca zerowe naszego wielomianu to: $-15, 12, -2$.\\
Wielomian jest stopnia nieparzystego, ponadto znak współczynnika przy\linebreak najwyższej potędze x jest ujemny.\\ W związku z tym wykres wielomianu zaczyna się od lewej strony powyżej osi OX. A więc $$x \in (-\infty,-15) \cup (-2,12).$$
\rozwStop
\odpStart
$x \in (-\infty,-15) \cup (-2,12)$
\odpStop
\testStart
A.$x \in (-\infty,-15) \cup (-2,12)$\\
B.$x \in (-\infty,-15) \cup (-2,12]$\\
C.$x \in (-\infty,-15) \cup [-2,12)$\\
D.$x \in (-\infty,-15] \cup (-2,12)$\\
E.$x \in (-\infty,-15] \cup (-2,12]$\\
F.$x \in (-\infty,-15] \cup [-2,12)$\\
G.$x \in (-\infty,-15) \cup [-2,12]$\\
H.$x \in (-\infty,-15] \cup [-2,12]$
\testStop
\kluczStart
A
\kluczStop



\zadStart{Zadanie z Wikieł Z 1.62 b) moja wersja nr 422}

Rozwiązać nierówności $(x+15)(12-x)(x+3)\ge0$.
\zadStop
\rozwStart{Patryk Wirkus}{}
Miejsca zerowe naszego wielomianu to: $-15, 12, -3$.\\
Wielomian jest stopnia nieparzystego, ponadto znak współczynnika przy\linebreak najwyższej potędze x jest ujemny.\\ W związku z tym wykres wielomianu zaczyna się od lewej strony powyżej osi OX. A więc $$x \in (-\infty,-15) \cup (-3,12).$$
\rozwStop
\odpStart
$x \in (-\infty,-15) \cup (-3,12)$
\odpStop
\testStart
A.$x \in (-\infty,-15) \cup (-3,12)$\\
B.$x \in (-\infty,-15) \cup (-3,12]$\\
C.$x \in (-\infty,-15) \cup [-3,12)$\\
D.$x \in (-\infty,-15] \cup (-3,12)$\\
E.$x \in (-\infty,-15] \cup (-3,12]$\\
F.$x \in (-\infty,-15] \cup [-3,12)$\\
G.$x \in (-\infty,-15) \cup [-3,12]$\\
H.$x \in (-\infty,-15] \cup [-3,12]$
\testStop
\kluczStart
A
\kluczStop



\zadStart{Zadanie z Wikieł Z 1.62 b) moja wersja nr 423}

Rozwiązać nierówności $(x+15)(12-x)(x+4)\ge0$.
\zadStop
\rozwStart{Patryk Wirkus}{}
Miejsca zerowe naszego wielomianu to: $-15, 12, -4$.\\
Wielomian jest stopnia nieparzystego, ponadto znak współczynnika przy\linebreak najwyższej potędze x jest ujemny.\\ W związku z tym wykres wielomianu zaczyna się od lewej strony powyżej osi OX. A więc $$x \in (-\infty,-15) \cup (-4,12).$$
\rozwStop
\odpStart
$x \in (-\infty,-15) \cup (-4,12)$
\odpStop
\testStart
A.$x \in (-\infty,-15) \cup (-4,12)$\\
B.$x \in (-\infty,-15) \cup (-4,12]$\\
C.$x \in (-\infty,-15) \cup [-4,12)$\\
D.$x \in (-\infty,-15] \cup (-4,12)$\\
E.$x \in (-\infty,-15] \cup (-4,12]$\\
F.$x \in (-\infty,-15] \cup [-4,12)$\\
G.$x \in (-\infty,-15) \cup [-4,12]$\\
H.$x \in (-\infty,-15] \cup [-4,12]$
\testStop
\kluczStart
A
\kluczStop



\zadStart{Zadanie z Wikieł Z 1.62 b) moja wersja nr 424}

Rozwiązać nierówności $(x+15)(12-x)(x+5)\ge0$.
\zadStop
\rozwStart{Patryk Wirkus}{}
Miejsca zerowe naszego wielomianu to: $-15, 12, -5$.\\
Wielomian jest stopnia nieparzystego, ponadto znak współczynnika przy\linebreak najwyższej potędze x jest ujemny.\\ W związku z tym wykres wielomianu zaczyna się od lewej strony powyżej osi OX. A więc $$x \in (-\infty,-15) \cup (-5,12).$$
\rozwStop
\odpStart
$x \in (-\infty,-15) \cup (-5,12)$
\odpStop
\testStart
A.$x \in (-\infty,-15) \cup (-5,12)$\\
B.$x \in (-\infty,-15) \cup (-5,12]$\\
C.$x \in (-\infty,-15) \cup [-5,12)$\\
D.$x \in (-\infty,-15] \cup (-5,12)$\\
E.$x \in (-\infty,-15] \cup (-5,12]$\\
F.$x \in (-\infty,-15] \cup [-5,12)$\\
G.$x \in (-\infty,-15) \cup [-5,12]$\\
H.$x \in (-\infty,-15] \cup [-5,12]$
\testStop
\kluczStart
A
\kluczStop



\zadStart{Zadanie z Wikieł Z 1.62 b) moja wersja nr 425}

Rozwiązać nierówności $(x+15)(12-x)(x+6)\ge0$.
\zadStop
\rozwStart{Patryk Wirkus}{}
Miejsca zerowe naszego wielomianu to: $-15, 12, -6$.\\
Wielomian jest stopnia nieparzystego, ponadto znak współczynnika przy\linebreak najwyższej potędze x jest ujemny.\\ W związku z tym wykres wielomianu zaczyna się od lewej strony powyżej osi OX. A więc $$x \in (-\infty,-15) \cup (-6,12).$$
\rozwStop
\odpStart
$x \in (-\infty,-15) \cup (-6,12)$
\odpStop
\testStart
A.$x \in (-\infty,-15) \cup (-6,12)$\\
B.$x \in (-\infty,-15) \cup (-6,12]$\\
C.$x \in (-\infty,-15) \cup [-6,12)$\\
D.$x \in (-\infty,-15] \cup (-6,12)$\\
E.$x \in (-\infty,-15] \cup (-6,12]$\\
F.$x \in (-\infty,-15] \cup [-6,12)$\\
G.$x \in (-\infty,-15) \cup [-6,12]$\\
H.$x \in (-\infty,-15] \cup [-6,12]$
\testStop
\kluczStart
A
\kluczStop



\zadStart{Zadanie z Wikieł Z 1.62 b) moja wersja nr 426}

Rozwiązać nierówności $(x+15)(12-x)(x+7)\ge0$.
\zadStop
\rozwStart{Patryk Wirkus}{}
Miejsca zerowe naszego wielomianu to: $-15, 12, -7$.\\
Wielomian jest stopnia nieparzystego, ponadto znak współczynnika przy\linebreak najwyższej potędze x jest ujemny.\\ W związku z tym wykres wielomianu zaczyna się od lewej strony powyżej osi OX. A więc $$x \in (-\infty,-15) \cup (-7,12).$$
\rozwStop
\odpStart
$x \in (-\infty,-15) \cup (-7,12)$
\odpStop
\testStart
A.$x \in (-\infty,-15) \cup (-7,12)$\\
B.$x \in (-\infty,-15) \cup (-7,12]$\\
C.$x \in (-\infty,-15) \cup [-7,12)$\\
D.$x \in (-\infty,-15] \cup (-7,12)$\\
E.$x \in (-\infty,-15] \cup (-7,12]$\\
F.$x \in (-\infty,-15] \cup [-7,12)$\\
G.$x \in (-\infty,-15) \cup [-7,12]$\\
H.$x \in (-\infty,-15] \cup [-7,12]$
\testStop
\kluczStart
A
\kluczStop



\zadStart{Zadanie z Wikieł Z 1.62 b) moja wersja nr 427}

Rozwiązać nierówności $(x+15)(12-x)(x+8)\ge0$.
\zadStop
\rozwStart{Patryk Wirkus}{}
Miejsca zerowe naszego wielomianu to: $-15, 12, -8$.\\
Wielomian jest stopnia nieparzystego, ponadto znak współczynnika przy\linebreak najwyższej potędze x jest ujemny.\\ W związku z tym wykres wielomianu zaczyna się od lewej strony powyżej osi OX. A więc $$x \in (-\infty,-15) \cup (-8,12).$$
\rozwStop
\odpStart
$x \in (-\infty,-15) \cup (-8,12)$
\odpStop
\testStart
A.$x \in (-\infty,-15) \cup (-8,12)$\\
B.$x \in (-\infty,-15) \cup (-8,12]$\\
C.$x \in (-\infty,-15) \cup [-8,12)$\\
D.$x \in (-\infty,-15] \cup (-8,12)$\\
E.$x \in (-\infty,-15] \cup (-8,12]$\\
F.$x \in (-\infty,-15] \cup [-8,12)$\\
G.$x \in (-\infty,-15) \cup [-8,12]$\\
H.$x \in (-\infty,-15] \cup [-8,12]$
\testStop
\kluczStart
A
\kluczStop



\zadStart{Zadanie z Wikieł Z 1.62 b) moja wersja nr 428}

Rozwiązać nierówności $(x+15)(12-x)(x+9)\ge0$.
\zadStop
\rozwStart{Patryk Wirkus}{}
Miejsca zerowe naszego wielomianu to: $-15, 12, -9$.\\
Wielomian jest stopnia nieparzystego, ponadto znak współczynnika przy\linebreak najwyższej potędze x jest ujemny.\\ W związku z tym wykres wielomianu zaczyna się od lewej strony powyżej osi OX. A więc $$x \in (-\infty,-15) \cup (-9,12).$$
\rozwStop
\odpStart
$x \in (-\infty,-15) \cup (-9,12)$
\odpStop
\testStart
A.$x \in (-\infty,-15) \cup (-9,12)$\\
B.$x \in (-\infty,-15) \cup (-9,12]$\\
C.$x \in (-\infty,-15) \cup [-9,12)$\\
D.$x \in (-\infty,-15] \cup (-9,12)$\\
E.$x \in (-\infty,-15] \cup (-9,12]$\\
F.$x \in (-\infty,-15] \cup [-9,12)$\\
G.$x \in (-\infty,-15) \cup [-9,12]$\\
H.$x \in (-\infty,-15] \cup [-9,12]$
\testStop
\kluczStart
A
\kluczStop



\zadStart{Zadanie z Wikieł Z 1.62 b) moja wersja nr 429}

Rozwiązać nierówności $(x+15)(12-x)(x+10)\ge0$.
\zadStop
\rozwStart{Patryk Wirkus}{}
Miejsca zerowe naszego wielomianu to: $-15, 12, -10$.\\
Wielomian jest stopnia nieparzystego, ponadto znak współczynnika przy\linebreak najwyższej potędze x jest ujemny.\\ W związku z tym wykres wielomianu zaczyna się od lewej strony powyżej osi OX. A więc $$x \in (-\infty,-15) \cup (-10,12).$$
\rozwStop
\odpStart
$x \in (-\infty,-15) \cup (-10,12)$
\odpStop
\testStart
A.$x \in (-\infty,-15) \cup (-10,12)$\\
B.$x \in (-\infty,-15) \cup (-10,12]$\\
C.$x \in (-\infty,-15) \cup [-10,12)$\\
D.$x \in (-\infty,-15] \cup (-10,12)$\\
E.$x \in (-\infty,-15] \cup (-10,12]$\\
F.$x \in (-\infty,-15] \cup [-10,12)$\\
G.$x \in (-\infty,-15) \cup [-10,12]$\\
H.$x \in (-\infty,-15] \cup [-10,12]$
\testStop
\kluczStart
A
\kluczStop



\zadStart{Zadanie z Wikieł Z 1.62 b) moja wersja nr 430}

Rozwiązać nierówności $(x+15)(12-x)(x+11)\ge0$.
\zadStop
\rozwStart{Patryk Wirkus}{}
Miejsca zerowe naszego wielomianu to: $-15, 12, -11$.\\
Wielomian jest stopnia nieparzystego, ponadto znak współczynnika przy\linebreak najwyższej potędze x jest ujemny.\\ W związku z tym wykres wielomianu zaczyna się od lewej strony powyżej osi OX. A więc $$x \in (-\infty,-15) \cup (-11,12).$$
\rozwStop
\odpStart
$x \in (-\infty,-15) \cup (-11,12)$
\odpStop
\testStart
A.$x \in (-\infty,-15) \cup (-11,12)$\\
B.$x \in (-\infty,-15) \cup (-11,12]$\\
C.$x \in (-\infty,-15) \cup [-11,12)$\\
D.$x \in (-\infty,-15] \cup (-11,12)$\\
E.$x \in (-\infty,-15] \cup (-11,12]$\\
F.$x \in (-\infty,-15] \cup [-11,12)$\\
G.$x \in (-\infty,-15) \cup [-11,12]$\\
H.$x \in (-\infty,-15] \cup [-11,12]$
\testStop
\kluczStart
A
\kluczStop



\zadStart{Zadanie z Wikieł Z 1.62 b) moja wersja nr 431}

Rozwiązać nierówności $(x+15)(13-x)(x+1)\ge0$.
\zadStop
\rozwStart{Patryk Wirkus}{}
Miejsca zerowe naszego wielomianu to: $-15, 13, -1$.\\
Wielomian jest stopnia nieparzystego, ponadto znak współczynnika przy\linebreak najwyższej potędze x jest ujemny.\\ W związku z tym wykres wielomianu zaczyna się od lewej strony powyżej osi OX. A więc $$x \in (-\infty,-15) \cup (-1,13).$$
\rozwStop
\odpStart
$x \in (-\infty,-15) \cup (-1,13)$
\odpStop
\testStart
A.$x \in (-\infty,-15) \cup (-1,13)$\\
B.$x \in (-\infty,-15) \cup (-1,13]$\\
C.$x \in (-\infty,-15) \cup [-1,13)$\\
D.$x \in (-\infty,-15] \cup (-1,13)$\\
E.$x \in (-\infty,-15] \cup (-1,13]$\\
F.$x \in (-\infty,-15] \cup [-1,13)$\\
G.$x \in (-\infty,-15) \cup [-1,13]$\\
H.$x \in (-\infty,-15] \cup [-1,13]$
\testStop
\kluczStart
A
\kluczStop



\zadStart{Zadanie z Wikieł Z 1.62 b) moja wersja nr 432}

Rozwiązać nierówności $(x+15)(13-x)(x+2)\ge0$.
\zadStop
\rozwStart{Patryk Wirkus}{}
Miejsca zerowe naszego wielomianu to: $-15, 13, -2$.\\
Wielomian jest stopnia nieparzystego, ponadto znak współczynnika przy\linebreak najwyższej potędze x jest ujemny.\\ W związku z tym wykres wielomianu zaczyna się od lewej strony powyżej osi OX. A więc $$x \in (-\infty,-15) \cup (-2,13).$$
\rozwStop
\odpStart
$x \in (-\infty,-15) \cup (-2,13)$
\odpStop
\testStart
A.$x \in (-\infty,-15) \cup (-2,13)$\\
B.$x \in (-\infty,-15) \cup (-2,13]$\\
C.$x \in (-\infty,-15) \cup [-2,13)$\\
D.$x \in (-\infty,-15] \cup (-2,13)$\\
E.$x \in (-\infty,-15] \cup (-2,13]$\\
F.$x \in (-\infty,-15] \cup [-2,13)$\\
G.$x \in (-\infty,-15) \cup [-2,13]$\\
H.$x \in (-\infty,-15] \cup [-2,13]$
\testStop
\kluczStart
A
\kluczStop



\zadStart{Zadanie z Wikieł Z 1.62 b) moja wersja nr 433}

Rozwiązać nierówności $(x+15)(13-x)(x+3)\ge0$.
\zadStop
\rozwStart{Patryk Wirkus}{}
Miejsca zerowe naszego wielomianu to: $-15, 13, -3$.\\
Wielomian jest stopnia nieparzystego, ponadto znak współczynnika przy\linebreak najwyższej potędze x jest ujemny.\\ W związku z tym wykres wielomianu zaczyna się od lewej strony powyżej osi OX. A więc $$x \in (-\infty,-15) \cup (-3,13).$$
\rozwStop
\odpStart
$x \in (-\infty,-15) \cup (-3,13)$
\odpStop
\testStart
A.$x \in (-\infty,-15) \cup (-3,13)$\\
B.$x \in (-\infty,-15) \cup (-3,13]$\\
C.$x \in (-\infty,-15) \cup [-3,13)$\\
D.$x \in (-\infty,-15] \cup (-3,13)$\\
E.$x \in (-\infty,-15] \cup (-3,13]$\\
F.$x \in (-\infty,-15] \cup [-3,13)$\\
G.$x \in (-\infty,-15) \cup [-3,13]$\\
H.$x \in (-\infty,-15] \cup [-3,13]$
\testStop
\kluczStart
A
\kluczStop



\zadStart{Zadanie z Wikieł Z 1.62 b) moja wersja nr 434}

Rozwiązać nierówności $(x+15)(13-x)(x+4)\ge0$.
\zadStop
\rozwStart{Patryk Wirkus}{}
Miejsca zerowe naszego wielomianu to: $-15, 13, -4$.\\
Wielomian jest stopnia nieparzystego, ponadto znak współczynnika przy\linebreak najwyższej potędze x jest ujemny.\\ W związku z tym wykres wielomianu zaczyna się od lewej strony powyżej osi OX. A więc $$x \in (-\infty,-15) \cup (-4,13).$$
\rozwStop
\odpStart
$x \in (-\infty,-15) \cup (-4,13)$
\odpStop
\testStart
A.$x \in (-\infty,-15) \cup (-4,13)$\\
B.$x \in (-\infty,-15) \cup (-4,13]$\\
C.$x \in (-\infty,-15) \cup [-4,13)$\\
D.$x \in (-\infty,-15] \cup (-4,13)$\\
E.$x \in (-\infty,-15] \cup (-4,13]$\\
F.$x \in (-\infty,-15] \cup [-4,13)$\\
G.$x \in (-\infty,-15) \cup [-4,13]$\\
H.$x \in (-\infty,-15] \cup [-4,13]$
\testStop
\kluczStart
A
\kluczStop



\zadStart{Zadanie z Wikieł Z 1.62 b) moja wersja nr 435}

Rozwiązać nierówności $(x+15)(13-x)(x+5)\ge0$.
\zadStop
\rozwStart{Patryk Wirkus}{}
Miejsca zerowe naszego wielomianu to: $-15, 13, -5$.\\
Wielomian jest stopnia nieparzystego, ponadto znak współczynnika przy\linebreak najwyższej potędze x jest ujemny.\\ W związku z tym wykres wielomianu zaczyna się od lewej strony powyżej osi OX. A więc $$x \in (-\infty,-15) \cup (-5,13).$$
\rozwStop
\odpStart
$x \in (-\infty,-15) \cup (-5,13)$
\odpStop
\testStart
A.$x \in (-\infty,-15) \cup (-5,13)$\\
B.$x \in (-\infty,-15) \cup (-5,13]$\\
C.$x \in (-\infty,-15) \cup [-5,13)$\\
D.$x \in (-\infty,-15] \cup (-5,13)$\\
E.$x \in (-\infty,-15] \cup (-5,13]$\\
F.$x \in (-\infty,-15] \cup [-5,13)$\\
G.$x \in (-\infty,-15) \cup [-5,13]$\\
H.$x \in (-\infty,-15] \cup [-5,13]$
\testStop
\kluczStart
A
\kluczStop



\zadStart{Zadanie z Wikieł Z 1.62 b) moja wersja nr 436}

Rozwiązać nierówności $(x+15)(13-x)(x+6)\ge0$.
\zadStop
\rozwStart{Patryk Wirkus}{}
Miejsca zerowe naszego wielomianu to: $-15, 13, -6$.\\
Wielomian jest stopnia nieparzystego, ponadto znak współczynnika przy\linebreak najwyższej potędze x jest ujemny.\\ W związku z tym wykres wielomianu zaczyna się od lewej strony powyżej osi OX. A więc $$x \in (-\infty,-15) \cup (-6,13).$$
\rozwStop
\odpStart
$x \in (-\infty,-15) \cup (-6,13)$
\odpStop
\testStart
A.$x \in (-\infty,-15) \cup (-6,13)$\\
B.$x \in (-\infty,-15) \cup (-6,13]$\\
C.$x \in (-\infty,-15) \cup [-6,13)$\\
D.$x \in (-\infty,-15] \cup (-6,13)$\\
E.$x \in (-\infty,-15] \cup (-6,13]$\\
F.$x \in (-\infty,-15] \cup [-6,13)$\\
G.$x \in (-\infty,-15) \cup [-6,13]$\\
H.$x \in (-\infty,-15] \cup [-6,13]$
\testStop
\kluczStart
A
\kluczStop



\zadStart{Zadanie z Wikieł Z 1.62 b) moja wersja nr 437}

Rozwiązać nierówności $(x+15)(13-x)(x+7)\ge0$.
\zadStop
\rozwStart{Patryk Wirkus}{}
Miejsca zerowe naszego wielomianu to: $-15, 13, -7$.\\
Wielomian jest stopnia nieparzystego, ponadto znak współczynnika przy\linebreak najwyższej potędze x jest ujemny.\\ W związku z tym wykres wielomianu zaczyna się od lewej strony powyżej osi OX. A więc $$x \in (-\infty,-15) \cup (-7,13).$$
\rozwStop
\odpStart
$x \in (-\infty,-15) \cup (-7,13)$
\odpStop
\testStart
A.$x \in (-\infty,-15) \cup (-7,13)$\\
B.$x \in (-\infty,-15) \cup (-7,13]$\\
C.$x \in (-\infty,-15) \cup [-7,13)$\\
D.$x \in (-\infty,-15] \cup (-7,13)$\\
E.$x \in (-\infty,-15] \cup (-7,13]$\\
F.$x \in (-\infty,-15] \cup [-7,13)$\\
G.$x \in (-\infty,-15) \cup [-7,13]$\\
H.$x \in (-\infty,-15] \cup [-7,13]$
\testStop
\kluczStart
A
\kluczStop



\zadStart{Zadanie z Wikieł Z 1.62 b) moja wersja nr 438}

Rozwiązać nierówności $(x+15)(13-x)(x+8)\ge0$.
\zadStop
\rozwStart{Patryk Wirkus}{}
Miejsca zerowe naszego wielomianu to: $-15, 13, -8$.\\
Wielomian jest stopnia nieparzystego, ponadto znak współczynnika przy\linebreak najwyższej potędze x jest ujemny.\\ W związku z tym wykres wielomianu zaczyna się od lewej strony powyżej osi OX. A więc $$x \in (-\infty,-15) \cup (-8,13).$$
\rozwStop
\odpStart
$x \in (-\infty,-15) \cup (-8,13)$
\odpStop
\testStart
A.$x \in (-\infty,-15) \cup (-8,13)$\\
B.$x \in (-\infty,-15) \cup (-8,13]$\\
C.$x \in (-\infty,-15) \cup [-8,13)$\\
D.$x \in (-\infty,-15] \cup (-8,13)$\\
E.$x \in (-\infty,-15] \cup (-8,13]$\\
F.$x \in (-\infty,-15] \cup [-8,13)$\\
G.$x \in (-\infty,-15) \cup [-8,13]$\\
H.$x \in (-\infty,-15] \cup [-8,13]$
\testStop
\kluczStart
A
\kluczStop



\zadStart{Zadanie z Wikieł Z 1.62 b) moja wersja nr 439}

Rozwiązać nierówności $(x+15)(13-x)(x+9)\ge0$.
\zadStop
\rozwStart{Patryk Wirkus}{}
Miejsca zerowe naszego wielomianu to: $-15, 13, -9$.\\
Wielomian jest stopnia nieparzystego, ponadto znak współczynnika przy\linebreak najwyższej potędze x jest ujemny.\\ W związku z tym wykres wielomianu zaczyna się od lewej strony powyżej osi OX. A więc $$x \in (-\infty,-15) \cup (-9,13).$$
\rozwStop
\odpStart
$x \in (-\infty,-15) \cup (-9,13)$
\odpStop
\testStart
A.$x \in (-\infty,-15) \cup (-9,13)$\\
B.$x \in (-\infty,-15) \cup (-9,13]$\\
C.$x \in (-\infty,-15) \cup [-9,13)$\\
D.$x \in (-\infty,-15] \cup (-9,13)$\\
E.$x \in (-\infty,-15] \cup (-9,13]$\\
F.$x \in (-\infty,-15] \cup [-9,13)$\\
G.$x \in (-\infty,-15) \cup [-9,13]$\\
H.$x \in (-\infty,-15] \cup [-9,13]$
\testStop
\kluczStart
A
\kluczStop



\zadStart{Zadanie z Wikieł Z 1.62 b) moja wersja nr 440}

Rozwiązać nierówności $(x+15)(13-x)(x+10)\ge0$.
\zadStop
\rozwStart{Patryk Wirkus}{}
Miejsca zerowe naszego wielomianu to: $-15, 13, -10$.\\
Wielomian jest stopnia nieparzystego, ponadto znak współczynnika przy\linebreak najwyższej potędze x jest ujemny.\\ W związku z tym wykres wielomianu zaczyna się od lewej strony powyżej osi OX. A więc $$x \in (-\infty,-15) \cup (-10,13).$$
\rozwStop
\odpStart
$x \in (-\infty,-15) \cup (-10,13)$
\odpStop
\testStart
A.$x \in (-\infty,-15) \cup (-10,13)$\\
B.$x \in (-\infty,-15) \cup (-10,13]$\\
C.$x \in (-\infty,-15) \cup [-10,13)$\\
D.$x \in (-\infty,-15] \cup (-10,13)$\\
E.$x \in (-\infty,-15] \cup (-10,13]$\\
F.$x \in (-\infty,-15] \cup [-10,13)$\\
G.$x \in (-\infty,-15) \cup [-10,13]$\\
H.$x \in (-\infty,-15] \cup [-10,13]$
\testStop
\kluczStart
A
\kluczStop



\zadStart{Zadanie z Wikieł Z 1.62 b) moja wersja nr 441}

Rozwiązać nierówności $(x+15)(13-x)(x+11)\ge0$.
\zadStop
\rozwStart{Patryk Wirkus}{}
Miejsca zerowe naszego wielomianu to: $-15, 13, -11$.\\
Wielomian jest stopnia nieparzystego, ponadto znak współczynnika przy\linebreak najwyższej potędze x jest ujemny.\\ W związku z tym wykres wielomianu zaczyna się od lewej strony powyżej osi OX. A więc $$x \in (-\infty,-15) \cup (-11,13).$$
\rozwStop
\odpStart
$x \in (-\infty,-15) \cup (-11,13)$
\odpStop
\testStart
A.$x \in (-\infty,-15) \cup (-11,13)$\\
B.$x \in (-\infty,-15) \cup (-11,13]$\\
C.$x \in (-\infty,-15) \cup [-11,13)$\\
D.$x \in (-\infty,-15] \cup (-11,13)$\\
E.$x \in (-\infty,-15] \cup (-11,13]$\\
F.$x \in (-\infty,-15] \cup [-11,13)$\\
G.$x \in (-\infty,-15) \cup [-11,13]$\\
H.$x \in (-\infty,-15] \cup [-11,13]$
\testStop
\kluczStart
A
\kluczStop



\zadStart{Zadanie z Wikieł Z 1.62 b) moja wersja nr 442}

Rozwiązać nierówności $(x+15)(13-x)(x+12)\ge0$.
\zadStop
\rozwStart{Patryk Wirkus}{}
Miejsca zerowe naszego wielomianu to: $-15, 13, -12$.\\
Wielomian jest stopnia nieparzystego, ponadto znak współczynnika przy\linebreak najwyższej potędze x jest ujemny.\\ W związku z tym wykres wielomianu zaczyna się od lewej strony powyżej osi OX. A więc $$x \in (-\infty,-15) \cup (-12,13).$$
\rozwStop
\odpStart
$x \in (-\infty,-15) \cup (-12,13)$
\odpStop
\testStart
A.$x \in (-\infty,-15) \cup (-12,13)$\\
B.$x \in (-\infty,-15) \cup (-12,13]$\\
C.$x \in (-\infty,-15) \cup [-12,13)$\\
D.$x \in (-\infty,-15] \cup (-12,13)$\\
E.$x \in (-\infty,-15] \cup (-12,13]$\\
F.$x \in (-\infty,-15] \cup [-12,13)$\\
G.$x \in (-\infty,-15) \cup [-12,13]$\\
H.$x \in (-\infty,-15] \cup [-12,13]$
\testStop
\kluczStart
A
\kluczStop



\zadStart{Zadanie z Wikieł Z 1.62 b) moja wersja nr 443}

Rozwiązać nierówności $(x+15)(14-x)(x+1)\ge0$.
\zadStop
\rozwStart{Patryk Wirkus}{}
Miejsca zerowe naszego wielomianu to: $-15, 14, -1$.\\
Wielomian jest stopnia nieparzystego, ponadto znak współczynnika przy\linebreak najwyższej potędze x jest ujemny.\\ W związku z tym wykres wielomianu zaczyna się od lewej strony powyżej osi OX. A więc $$x \in (-\infty,-15) \cup (-1,14).$$
\rozwStop
\odpStart
$x \in (-\infty,-15) \cup (-1,14)$
\odpStop
\testStart
A.$x \in (-\infty,-15) \cup (-1,14)$\\
B.$x \in (-\infty,-15) \cup (-1,14]$\\
C.$x \in (-\infty,-15) \cup [-1,14)$\\
D.$x \in (-\infty,-15] \cup (-1,14)$\\
E.$x \in (-\infty,-15] \cup (-1,14]$\\
F.$x \in (-\infty,-15] \cup [-1,14)$\\
G.$x \in (-\infty,-15) \cup [-1,14]$\\
H.$x \in (-\infty,-15] \cup [-1,14]$
\testStop
\kluczStart
A
\kluczStop



\zadStart{Zadanie z Wikieł Z 1.62 b) moja wersja nr 444}

Rozwiązać nierówności $(x+15)(14-x)(x+2)\ge0$.
\zadStop
\rozwStart{Patryk Wirkus}{}
Miejsca zerowe naszego wielomianu to: $-15, 14, -2$.\\
Wielomian jest stopnia nieparzystego, ponadto znak współczynnika przy\linebreak najwyższej potędze x jest ujemny.\\ W związku z tym wykres wielomianu zaczyna się od lewej strony powyżej osi OX. A więc $$x \in (-\infty,-15) \cup (-2,14).$$
\rozwStop
\odpStart
$x \in (-\infty,-15) \cup (-2,14)$
\odpStop
\testStart
A.$x \in (-\infty,-15) \cup (-2,14)$\\
B.$x \in (-\infty,-15) \cup (-2,14]$\\
C.$x \in (-\infty,-15) \cup [-2,14)$\\
D.$x \in (-\infty,-15] \cup (-2,14)$\\
E.$x \in (-\infty,-15] \cup (-2,14]$\\
F.$x \in (-\infty,-15] \cup [-2,14)$\\
G.$x \in (-\infty,-15) \cup [-2,14]$\\
H.$x \in (-\infty,-15] \cup [-2,14]$
\testStop
\kluczStart
A
\kluczStop



\zadStart{Zadanie z Wikieł Z 1.62 b) moja wersja nr 445}

Rozwiązać nierówności $(x+15)(14-x)(x+3)\ge0$.
\zadStop
\rozwStart{Patryk Wirkus}{}
Miejsca zerowe naszego wielomianu to: $-15, 14, -3$.\\
Wielomian jest stopnia nieparzystego, ponadto znak współczynnika przy\linebreak najwyższej potędze x jest ujemny.\\ W związku z tym wykres wielomianu zaczyna się od lewej strony powyżej osi OX. A więc $$x \in (-\infty,-15) \cup (-3,14).$$
\rozwStop
\odpStart
$x \in (-\infty,-15) \cup (-3,14)$
\odpStop
\testStart
A.$x \in (-\infty,-15) \cup (-3,14)$\\
B.$x \in (-\infty,-15) \cup (-3,14]$\\
C.$x \in (-\infty,-15) \cup [-3,14)$\\
D.$x \in (-\infty,-15] \cup (-3,14)$\\
E.$x \in (-\infty,-15] \cup (-3,14]$\\
F.$x \in (-\infty,-15] \cup [-3,14)$\\
G.$x \in (-\infty,-15) \cup [-3,14]$\\
H.$x \in (-\infty,-15] \cup [-3,14]$
\testStop
\kluczStart
A
\kluczStop



\zadStart{Zadanie z Wikieł Z 1.62 b) moja wersja nr 446}

Rozwiązać nierówności $(x+15)(14-x)(x+4)\ge0$.
\zadStop
\rozwStart{Patryk Wirkus}{}
Miejsca zerowe naszego wielomianu to: $-15, 14, -4$.\\
Wielomian jest stopnia nieparzystego, ponadto znak współczynnika przy\linebreak najwyższej potędze x jest ujemny.\\ W związku z tym wykres wielomianu zaczyna się od lewej strony powyżej osi OX. A więc $$x \in (-\infty,-15) \cup (-4,14).$$
\rozwStop
\odpStart
$x \in (-\infty,-15) \cup (-4,14)$
\odpStop
\testStart
A.$x \in (-\infty,-15) \cup (-4,14)$\\
B.$x \in (-\infty,-15) \cup (-4,14]$\\
C.$x \in (-\infty,-15) \cup [-4,14)$\\
D.$x \in (-\infty,-15] \cup (-4,14)$\\
E.$x \in (-\infty,-15] \cup (-4,14]$\\
F.$x \in (-\infty,-15] \cup [-4,14)$\\
G.$x \in (-\infty,-15) \cup [-4,14]$\\
H.$x \in (-\infty,-15] \cup [-4,14]$
\testStop
\kluczStart
A
\kluczStop



\zadStart{Zadanie z Wikieł Z 1.62 b) moja wersja nr 447}

Rozwiązać nierówności $(x+15)(14-x)(x+5)\ge0$.
\zadStop
\rozwStart{Patryk Wirkus}{}
Miejsca zerowe naszego wielomianu to: $-15, 14, -5$.\\
Wielomian jest stopnia nieparzystego, ponadto znak współczynnika przy\linebreak najwyższej potędze x jest ujemny.\\ W związku z tym wykres wielomianu zaczyna się od lewej strony powyżej osi OX. A więc $$x \in (-\infty,-15) \cup (-5,14).$$
\rozwStop
\odpStart
$x \in (-\infty,-15) \cup (-5,14)$
\odpStop
\testStart
A.$x \in (-\infty,-15) \cup (-5,14)$\\
B.$x \in (-\infty,-15) \cup (-5,14]$\\
C.$x \in (-\infty,-15) \cup [-5,14)$\\
D.$x \in (-\infty,-15] \cup (-5,14)$\\
E.$x \in (-\infty,-15] \cup (-5,14]$\\
F.$x \in (-\infty,-15] \cup [-5,14)$\\
G.$x \in (-\infty,-15) \cup [-5,14]$\\
H.$x \in (-\infty,-15] \cup [-5,14]$
\testStop
\kluczStart
A
\kluczStop



\zadStart{Zadanie z Wikieł Z 1.62 b) moja wersja nr 448}

Rozwiązać nierówności $(x+15)(14-x)(x+6)\ge0$.
\zadStop
\rozwStart{Patryk Wirkus}{}
Miejsca zerowe naszego wielomianu to: $-15, 14, -6$.\\
Wielomian jest stopnia nieparzystego, ponadto znak współczynnika przy\linebreak najwyższej potędze x jest ujemny.\\ W związku z tym wykres wielomianu zaczyna się od lewej strony powyżej osi OX. A więc $$x \in (-\infty,-15) \cup (-6,14).$$
\rozwStop
\odpStart
$x \in (-\infty,-15) \cup (-6,14)$
\odpStop
\testStart
A.$x \in (-\infty,-15) \cup (-6,14)$\\
B.$x \in (-\infty,-15) \cup (-6,14]$\\
C.$x \in (-\infty,-15) \cup [-6,14)$\\
D.$x \in (-\infty,-15] \cup (-6,14)$\\
E.$x \in (-\infty,-15] \cup (-6,14]$\\
F.$x \in (-\infty,-15] \cup [-6,14)$\\
G.$x \in (-\infty,-15) \cup [-6,14]$\\
H.$x \in (-\infty,-15] \cup [-6,14]$
\testStop
\kluczStart
A
\kluczStop



\zadStart{Zadanie z Wikieł Z 1.62 b) moja wersja nr 449}

Rozwiązać nierówności $(x+15)(14-x)(x+7)\ge0$.
\zadStop
\rozwStart{Patryk Wirkus}{}
Miejsca zerowe naszego wielomianu to: $-15, 14, -7$.\\
Wielomian jest stopnia nieparzystego, ponadto znak współczynnika przy\linebreak najwyższej potędze x jest ujemny.\\ W związku z tym wykres wielomianu zaczyna się od lewej strony powyżej osi OX. A więc $$x \in (-\infty,-15) \cup (-7,14).$$
\rozwStop
\odpStart
$x \in (-\infty,-15) \cup (-7,14)$
\odpStop
\testStart
A.$x \in (-\infty,-15) \cup (-7,14)$\\
B.$x \in (-\infty,-15) \cup (-7,14]$\\
C.$x \in (-\infty,-15) \cup [-7,14)$\\
D.$x \in (-\infty,-15] \cup (-7,14)$\\
E.$x \in (-\infty,-15] \cup (-7,14]$\\
F.$x \in (-\infty,-15] \cup [-7,14)$\\
G.$x \in (-\infty,-15) \cup [-7,14]$\\
H.$x \in (-\infty,-15] \cup [-7,14]$
\testStop
\kluczStart
A
\kluczStop



\zadStart{Zadanie z Wikieł Z 1.62 b) moja wersja nr 450}

Rozwiązać nierówności $(x+15)(14-x)(x+8)\ge0$.
\zadStop
\rozwStart{Patryk Wirkus}{}
Miejsca zerowe naszego wielomianu to: $-15, 14, -8$.\\
Wielomian jest stopnia nieparzystego, ponadto znak współczynnika przy\linebreak najwyższej potędze x jest ujemny.\\ W związku z tym wykres wielomianu zaczyna się od lewej strony powyżej osi OX. A więc $$x \in (-\infty,-15) \cup (-8,14).$$
\rozwStop
\odpStart
$x \in (-\infty,-15) \cup (-8,14)$
\odpStop
\testStart
A.$x \in (-\infty,-15) \cup (-8,14)$\\
B.$x \in (-\infty,-15) \cup (-8,14]$\\
C.$x \in (-\infty,-15) \cup [-8,14)$\\
D.$x \in (-\infty,-15] \cup (-8,14)$\\
E.$x \in (-\infty,-15] \cup (-8,14]$\\
F.$x \in (-\infty,-15] \cup [-8,14)$\\
G.$x \in (-\infty,-15) \cup [-8,14]$\\
H.$x \in (-\infty,-15] \cup [-8,14]$
\testStop
\kluczStart
A
\kluczStop



\zadStart{Zadanie z Wikieł Z 1.62 b) moja wersja nr 451}

Rozwiązać nierówności $(x+15)(14-x)(x+9)\ge0$.
\zadStop
\rozwStart{Patryk Wirkus}{}
Miejsca zerowe naszego wielomianu to: $-15, 14, -9$.\\
Wielomian jest stopnia nieparzystego, ponadto znak współczynnika przy\linebreak najwyższej potędze x jest ujemny.\\ W związku z tym wykres wielomianu zaczyna się od lewej strony powyżej osi OX. A więc $$x \in (-\infty,-15) \cup (-9,14).$$
\rozwStop
\odpStart
$x \in (-\infty,-15) \cup (-9,14)$
\odpStop
\testStart
A.$x \in (-\infty,-15) \cup (-9,14)$\\
B.$x \in (-\infty,-15) \cup (-9,14]$\\
C.$x \in (-\infty,-15) \cup [-9,14)$\\
D.$x \in (-\infty,-15] \cup (-9,14)$\\
E.$x \in (-\infty,-15] \cup (-9,14]$\\
F.$x \in (-\infty,-15] \cup [-9,14)$\\
G.$x \in (-\infty,-15) \cup [-9,14]$\\
H.$x \in (-\infty,-15] \cup [-9,14]$
\testStop
\kluczStart
A
\kluczStop



\zadStart{Zadanie z Wikieł Z 1.62 b) moja wersja nr 452}

Rozwiązać nierówności $(x+15)(14-x)(x+10)\ge0$.
\zadStop
\rozwStart{Patryk Wirkus}{}
Miejsca zerowe naszego wielomianu to: $-15, 14, -10$.\\
Wielomian jest stopnia nieparzystego, ponadto znak współczynnika przy\linebreak najwyższej potędze x jest ujemny.\\ W związku z tym wykres wielomianu zaczyna się od lewej strony powyżej osi OX. A więc $$x \in (-\infty,-15) \cup (-10,14).$$
\rozwStop
\odpStart
$x \in (-\infty,-15) \cup (-10,14)$
\odpStop
\testStart
A.$x \in (-\infty,-15) \cup (-10,14)$\\
B.$x \in (-\infty,-15) \cup (-10,14]$\\
C.$x \in (-\infty,-15) \cup [-10,14)$\\
D.$x \in (-\infty,-15] \cup (-10,14)$\\
E.$x \in (-\infty,-15] \cup (-10,14]$\\
F.$x \in (-\infty,-15] \cup [-10,14)$\\
G.$x \in (-\infty,-15) \cup [-10,14]$\\
H.$x \in (-\infty,-15] \cup [-10,14]$
\testStop
\kluczStart
A
\kluczStop



\zadStart{Zadanie z Wikieł Z 1.62 b) moja wersja nr 453}

Rozwiązać nierówności $(x+15)(14-x)(x+11)\ge0$.
\zadStop
\rozwStart{Patryk Wirkus}{}
Miejsca zerowe naszego wielomianu to: $-15, 14, -11$.\\
Wielomian jest stopnia nieparzystego, ponadto znak współczynnika przy\linebreak najwyższej potędze x jest ujemny.\\ W związku z tym wykres wielomianu zaczyna się od lewej strony powyżej osi OX. A więc $$x \in (-\infty,-15) \cup (-11,14).$$
\rozwStop
\odpStart
$x \in (-\infty,-15) \cup (-11,14)$
\odpStop
\testStart
A.$x \in (-\infty,-15) \cup (-11,14)$\\
B.$x \in (-\infty,-15) \cup (-11,14]$\\
C.$x \in (-\infty,-15) \cup [-11,14)$\\
D.$x \in (-\infty,-15] \cup (-11,14)$\\
E.$x \in (-\infty,-15] \cup (-11,14]$\\
F.$x \in (-\infty,-15] \cup [-11,14)$\\
G.$x \in (-\infty,-15) \cup [-11,14]$\\
H.$x \in (-\infty,-15] \cup [-11,14]$
\testStop
\kluczStart
A
\kluczStop



\zadStart{Zadanie z Wikieł Z 1.62 b) moja wersja nr 454}

Rozwiązać nierówności $(x+15)(14-x)(x+12)\ge0$.
\zadStop
\rozwStart{Patryk Wirkus}{}
Miejsca zerowe naszego wielomianu to: $-15, 14, -12$.\\
Wielomian jest stopnia nieparzystego, ponadto znak współczynnika przy\linebreak najwyższej potędze x jest ujemny.\\ W związku z tym wykres wielomianu zaczyna się od lewej strony powyżej osi OX. A więc $$x \in (-\infty,-15) \cup (-12,14).$$
\rozwStop
\odpStart
$x \in (-\infty,-15) \cup (-12,14)$
\odpStop
\testStart
A.$x \in (-\infty,-15) \cup (-12,14)$\\
B.$x \in (-\infty,-15) \cup (-12,14]$\\
C.$x \in (-\infty,-15) \cup [-12,14)$\\
D.$x \in (-\infty,-15] \cup (-12,14)$\\
E.$x \in (-\infty,-15] \cup (-12,14]$\\
F.$x \in (-\infty,-15] \cup [-12,14)$\\
G.$x \in (-\infty,-15) \cup [-12,14]$\\
H.$x \in (-\infty,-15] \cup [-12,14]$
\testStop
\kluczStart
A
\kluczStop



\zadStart{Zadanie z Wikieł Z 1.62 b) moja wersja nr 455}

Rozwiązać nierówności $(x+15)(14-x)(x+13)\ge0$.
\zadStop
\rozwStart{Patryk Wirkus}{}
Miejsca zerowe naszego wielomianu to: $-15, 14, -13$.\\
Wielomian jest stopnia nieparzystego, ponadto znak współczynnika przy\linebreak najwyższej potędze x jest ujemny.\\ W związku z tym wykres wielomianu zaczyna się od lewej strony powyżej osi OX. A więc $$x \in (-\infty,-15) \cup (-13,14).$$
\rozwStop
\odpStart
$x \in (-\infty,-15) \cup (-13,14)$
\odpStop
\testStart
A.$x \in (-\infty,-15) \cup (-13,14)$\\
B.$x \in (-\infty,-15) \cup (-13,14]$\\
C.$x \in (-\infty,-15) \cup [-13,14)$\\
D.$x \in (-\infty,-15] \cup (-13,14)$\\
E.$x \in (-\infty,-15] \cup (-13,14]$\\
F.$x \in (-\infty,-15] \cup [-13,14)$\\
G.$x \in (-\infty,-15) \cup [-13,14]$\\
H.$x \in (-\infty,-15] \cup [-13,14]$
\testStop
\kluczStart
A
\kluczStop



\zadStart{Zadanie z Wikieł Z 1.62 b) moja wersja nr 456}

Rozwiązać nierówności $(x+16)(2-x)(x+1)\ge0$.
\zadStop
\rozwStart{Patryk Wirkus}{}
Miejsca zerowe naszego wielomianu to: $-16, 2, -1$.\\
Wielomian jest stopnia nieparzystego, ponadto znak współczynnika przy\linebreak najwyższej potędze x jest ujemny.\\ W związku z tym wykres wielomianu zaczyna się od lewej strony powyżej osi OX. A więc $$x \in (-\infty,-16) \cup (-1,2).$$
\rozwStop
\odpStart
$x \in (-\infty,-16) \cup (-1,2)$
\odpStop
\testStart
A.$x \in (-\infty,-16) \cup (-1,2)$\\
B.$x \in (-\infty,-16) \cup (-1,2]$\\
C.$x \in (-\infty,-16) \cup [-1,2)$\\
D.$x \in (-\infty,-16] \cup (-1,2)$\\
E.$x \in (-\infty,-16] \cup (-1,2]$\\
F.$x \in (-\infty,-16] \cup [-1,2)$\\
G.$x \in (-\infty,-16) \cup [-1,2]$\\
H.$x \in (-\infty,-16] \cup [-1,2]$
\testStop
\kluczStart
A
\kluczStop



\zadStart{Zadanie z Wikieł Z 1.62 b) moja wersja nr 457}

Rozwiązać nierówności $(x+16)(3-x)(x+1)\ge0$.
\zadStop
\rozwStart{Patryk Wirkus}{}
Miejsca zerowe naszego wielomianu to: $-16, 3, -1$.\\
Wielomian jest stopnia nieparzystego, ponadto znak współczynnika przy\linebreak najwyższej potędze x jest ujemny.\\ W związku z tym wykres wielomianu zaczyna się od lewej strony powyżej osi OX. A więc $$x \in (-\infty,-16) \cup (-1,3).$$
\rozwStop
\odpStart
$x \in (-\infty,-16) \cup (-1,3)$
\odpStop
\testStart
A.$x \in (-\infty,-16) \cup (-1,3)$\\
B.$x \in (-\infty,-16) \cup (-1,3]$\\
C.$x \in (-\infty,-16) \cup [-1,3)$\\
D.$x \in (-\infty,-16] \cup (-1,3)$\\
E.$x \in (-\infty,-16] \cup (-1,3]$\\
F.$x \in (-\infty,-16] \cup [-1,3)$\\
G.$x \in (-\infty,-16) \cup [-1,3]$\\
H.$x \in (-\infty,-16] \cup [-1,3]$
\testStop
\kluczStart
A
\kluczStop



\zadStart{Zadanie z Wikieł Z 1.62 b) moja wersja nr 458}

Rozwiązać nierówności $(x+16)(3-x)(x+2)\ge0$.
\zadStop
\rozwStart{Patryk Wirkus}{}
Miejsca zerowe naszego wielomianu to: $-16, 3, -2$.\\
Wielomian jest stopnia nieparzystego, ponadto znak współczynnika przy\linebreak najwyższej potędze x jest ujemny.\\ W związku z tym wykres wielomianu zaczyna się od lewej strony powyżej osi OX. A więc $$x \in (-\infty,-16) \cup (-2,3).$$
\rozwStop
\odpStart
$x \in (-\infty,-16) \cup (-2,3)$
\odpStop
\testStart
A.$x \in (-\infty,-16) \cup (-2,3)$\\
B.$x \in (-\infty,-16) \cup (-2,3]$\\
C.$x \in (-\infty,-16) \cup [-2,3)$\\
D.$x \in (-\infty,-16] \cup (-2,3)$\\
E.$x \in (-\infty,-16] \cup (-2,3]$\\
F.$x \in (-\infty,-16] \cup [-2,3)$\\
G.$x \in (-\infty,-16) \cup [-2,3]$\\
H.$x \in (-\infty,-16] \cup [-2,3]$
\testStop
\kluczStart
A
\kluczStop



\zadStart{Zadanie z Wikieł Z 1.62 b) moja wersja nr 459}

Rozwiązać nierówności $(x+16)(4-x)(x+1)\ge0$.
\zadStop
\rozwStart{Patryk Wirkus}{}
Miejsca zerowe naszego wielomianu to: $-16, 4, -1$.\\
Wielomian jest stopnia nieparzystego, ponadto znak współczynnika przy\linebreak najwyższej potędze x jest ujemny.\\ W związku z tym wykres wielomianu zaczyna się od lewej strony powyżej osi OX. A więc $$x \in (-\infty,-16) \cup (-1,4).$$
\rozwStop
\odpStart
$x \in (-\infty,-16) \cup (-1,4)$
\odpStop
\testStart
A.$x \in (-\infty,-16) \cup (-1,4)$\\
B.$x \in (-\infty,-16) \cup (-1,4]$\\
C.$x \in (-\infty,-16) \cup [-1,4)$\\
D.$x \in (-\infty,-16] \cup (-1,4)$\\
E.$x \in (-\infty,-16] \cup (-1,4]$\\
F.$x \in (-\infty,-16] \cup [-1,4)$\\
G.$x \in (-\infty,-16) \cup [-1,4]$\\
H.$x \in (-\infty,-16] \cup [-1,4]$
\testStop
\kluczStart
A
\kluczStop



\zadStart{Zadanie z Wikieł Z 1.62 b) moja wersja nr 460}

Rozwiązać nierówności $(x+16)(4-x)(x+2)\ge0$.
\zadStop
\rozwStart{Patryk Wirkus}{}
Miejsca zerowe naszego wielomianu to: $-16, 4, -2$.\\
Wielomian jest stopnia nieparzystego, ponadto znak współczynnika przy\linebreak najwyższej potędze x jest ujemny.\\ W związku z tym wykres wielomianu zaczyna się od lewej strony powyżej osi OX. A więc $$x \in (-\infty,-16) \cup (-2,4).$$
\rozwStop
\odpStart
$x \in (-\infty,-16) \cup (-2,4)$
\odpStop
\testStart
A.$x \in (-\infty,-16) \cup (-2,4)$\\
B.$x \in (-\infty,-16) \cup (-2,4]$\\
C.$x \in (-\infty,-16) \cup [-2,4)$\\
D.$x \in (-\infty,-16] \cup (-2,4)$\\
E.$x \in (-\infty,-16] \cup (-2,4]$\\
F.$x \in (-\infty,-16] \cup [-2,4)$\\
G.$x \in (-\infty,-16) \cup [-2,4]$\\
H.$x \in (-\infty,-16] \cup [-2,4]$
\testStop
\kluczStart
A
\kluczStop



\zadStart{Zadanie z Wikieł Z 1.62 b) moja wersja nr 461}

Rozwiązać nierówności $(x+16)(4-x)(x+3)\ge0$.
\zadStop
\rozwStart{Patryk Wirkus}{}
Miejsca zerowe naszego wielomianu to: $-16, 4, -3$.\\
Wielomian jest stopnia nieparzystego, ponadto znak współczynnika przy\linebreak najwyższej potędze x jest ujemny.\\ W związku z tym wykres wielomianu zaczyna się od lewej strony powyżej osi OX. A więc $$x \in (-\infty,-16) \cup (-3,4).$$
\rozwStop
\odpStart
$x \in (-\infty,-16) \cup (-3,4)$
\odpStop
\testStart
A.$x \in (-\infty,-16) \cup (-3,4)$\\
B.$x \in (-\infty,-16) \cup (-3,4]$\\
C.$x \in (-\infty,-16) \cup [-3,4)$\\
D.$x \in (-\infty,-16] \cup (-3,4)$\\
E.$x \in (-\infty,-16] \cup (-3,4]$\\
F.$x \in (-\infty,-16] \cup [-3,4)$\\
G.$x \in (-\infty,-16) \cup [-3,4]$\\
H.$x \in (-\infty,-16] \cup [-3,4]$
\testStop
\kluczStart
A
\kluczStop



\zadStart{Zadanie z Wikieł Z 1.62 b) moja wersja nr 462}

Rozwiązać nierówności $(x+16)(5-x)(x+1)\ge0$.
\zadStop
\rozwStart{Patryk Wirkus}{}
Miejsca zerowe naszego wielomianu to: $-16, 5, -1$.\\
Wielomian jest stopnia nieparzystego, ponadto znak współczynnika przy\linebreak najwyższej potędze x jest ujemny.\\ W związku z tym wykres wielomianu zaczyna się od lewej strony powyżej osi OX. A więc $$x \in (-\infty,-16) \cup (-1,5).$$
\rozwStop
\odpStart
$x \in (-\infty,-16) \cup (-1,5)$
\odpStop
\testStart
A.$x \in (-\infty,-16) \cup (-1,5)$\\
B.$x \in (-\infty,-16) \cup (-1,5]$\\
C.$x \in (-\infty,-16) \cup [-1,5)$\\
D.$x \in (-\infty,-16] \cup (-1,5)$\\
E.$x \in (-\infty,-16] \cup (-1,5]$\\
F.$x \in (-\infty,-16] \cup [-1,5)$\\
G.$x \in (-\infty,-16) \cup [-1,5]$\\
H.$x \in (-\infty,-16] \cup [-1,5]$
\testStop
\kluczStart
A
\kluczStop



\zadStart{Zadanie z Wikieł Z 1.62 b) moja wersja nr 463}

Rozwiązać nierówności $(x+16)(5-x)(x+2)\ge0$.
\zadStop
\rozwStart{Patryk Wirkus}{}
Miejsca zerowe naszego wielomianu to: $-16, 5, -2$.\\
Wielomian jest stopnia nieparzystego, ponadto znak współczynnika przy\linebreak najwyższej potędze x jest ujemny.\\ W związku z tym wykres wielomianu zaczyna się od lewej strony powyżej osi OX. A więc $$x \in (-\infty,-16) \cup (-2,5).$$
\rozwStop
\odpStart
$x \in (-\infty,-16) \cup (-2,5)$
\odpStop
\testStart
A.$x \in (-\infty,-16) \cup (-2,5)$\\
B.$x \in (-\infty,-16) \cup (-2,5]$\\
C.$x \in (-\infty,-16) \cup [-2,5)$\\
D.$x \in (-\infty,-16] \cup (-2,5)$\\
E.$x \in (-\infty,-16] \cup (-2,5]$\\
F.$x \in (-\infty,-16] \cup [-2,5)$\\
G.$x \in (-\infty,-16) \cup [-2,5]$\\
H.$x \in (-\infty,-16] \cup [-2,5]$
\testStop
\kluczStart
A
\kluczStop



\zadStart{Zadanie z Wikieł Z 1.62 b) moja wersja nr 464}

Rozwiązać nierówności $(x+16)(5-x)(x+3)\ge0$.
\zadStop
\rozwStart{Patryk Wirkus}{}
Miejsca zerowe naszego wielomianu to: $-16, 5, -3$.\\
Wielomian jest stopnia nieparzystego, ponadto znak współczynnika przy\linebreak najwyższej potędze x jest ujemny.\\ W związku z tym wykres wielomianu zaczyna się od lewej strony powyżej osi OX. A więc $$x \in (-\infty,-16) \cup (-3,5).$$
\rozwStop
\odpStart
$x \in (-\infty,-16) \cup (-3,5)$
\odpStop
\testStart
A.$x \in (-\infty,-16) \cup (-3,5)$\\
B.$x \in (-\infty,-16) \cup (-3,5]$\\
C.$x \in (-\infty,-16) \cup [-3,5)$\\
D.$x \in (-\infty,-16] \cup (-3,5)$\\
E.$x \in (-\infty,-16] \cup (-3,5]$\\
F.$x \in (-\infty,-16] \cup [-3,5)$\\
G.$x \in (-\infty,-16) \cup [-3,5]$\\
H.$x \in (-\infty,-16] \cup [-3,5]$
\testStop
\kluczStart
A
\kluczStop



\zadStart{Zadanie z Wikieł Z 1.62 b) moja wersja nr 465}

Rozwiązać nierówności $(x+16)(5-x)(x+4)\ge0$.
\zadStop
\rozwStart{Patryk Wirkus}{}
Miejsca zerowe naszego wielomianu to: $-16, 5, -4$.\\
Wielomian jest stopnia nieparzystego, ponadto znak współczynnika przy\linebreak najwyższej potędze x jest ujemny.\\ W związku z tym wykres wielomianu zaczyna się od lewej strony powyżej osi OX. A więc $$x \in (-\infty,-16) \cup (-4,5).$$
\rozwStop
\odpStart
$x \in (-\infty,-16) \cup (-4,5)$
\odpStop
\testStart
A.$x \in (-\infty,-16) \cup (-4,5)$\\
B.$x \in (-\infty,-16) \cup (-4,5]$\\
C.$x \in (-\infty,-16) \cup [-4,5)$\\
D.$x \in (-\infty,-16] \cup (-4,5)$\\
E.$x \in (-\infty,-16] \cup (-4,5]$\\
F.$x \in (-\infty,-16] \cup [-4,5)$\\
G.$x \in (-\infty,-16) \cup [-4,5]$\\
H.$x \in (-\infty,-16] \cup [-4,5]$
\testStop
\kluczStart
A
\kluczStop



\zadStart{Zadanie z Wikieł Z 1.62 b) moja wersja nr 466}

Rozwiązać nierówności $(x+16)(6-x)(x+1)\ge0$.
\zadStop
\rozwStart{Patryk Wirkus}{}
Miejsca zerowe naszego wielomianu to: $-16, 6, -1$.\\
Wielomian jest stopnia nieparzystego, ponadto znak współczynnika przy\linebreak najwyższej potędze x jest ujemny.\\ W związku z tym wykres wielomianu zaczyna się od lewej strony powyżej osi OX. A więc $$x \in (-\infty,-16) \cup (-1,6).$$
\rozwStop
\odpStart
$x \in (-\infty,-16) \cup (-1,6)$
\odpStop
\testStart
A.$x \in (-\infty,-16) \cup (-1,6)$\\
B.$x \in (-\infty,-16) \cup (-1,6]$\\
C.$x \in (-\infty,-16) \cup [-1,6)$\\
D.$x \in (-\infty,-16] \cup (-1,6)$\\
E.$x \in (-\infty,-16] \cup (-1,6]$\\
F.$x \in (-\infty,-16] \cup [-1,6)$\\
G.$x \in (-\infty,-16) \cup [-1,6]$\\
H.$x \in (-\infty,-16] \cup [-1,6]$
\testStop
\kluczStart
A
\kluczStop



\zadStart{Zadanie z Wikieł Z 1.62 b) moja wersja nr 467}

Rozwiązać nierówności $(x+16)(6-x)(x+2)\ge0$.
\zadStop
\rozwStart{Patryk Wirkus}{}
Miejsca zerowe naszego wielomianu to: $-16, 6, -2$.\\
Wielomian jest stopnia nieparzystego, ponadto znak współczynnika przy\linebreak najwyższej potędze x jest ujemny.\\ W związku z tym wykres wielomianu zaczyna się od lewej strony powyżej osi OX. A więc $$x \in (-\infty,-16) \cup (-2,6).$$
\rozwStop
\odpStart
$x \in (-\infty,-16) \cup (-2,6)$
\odpStop
\testStart
A.$x \in (-\infty,-16) \cup (-2,6)$\\
B.$x \in (-\infty,-16) \cup (-2,6]$\\
C.$x \in (-\infty,-16) \cup [-2,6)$\\
D.$x \in (-\infty,-16] \cup (-2,6)$\\
E.$x \in (-\infty,-16] \cup (-2,6]$\\
F.$x \in (-\infty,-16] \cup [-2,6)$\\
G.$x \in (-\infty,-16) \cup [-2,6]$\\
H.$x \in (-\infty,-16] \cup [-2,6]$
\testStop
\kluczStart
A
\kluczStop



\zadStart{Zadanie z Wikieł Z 1.62 b) moja wersja nr 468}

Rozwiązać nierówności $(x+16)(6-x)(x+3)\ge0$.
\zadStop
\rozwStart{Patryk Wirkus}{}
Miejsca zerowe naszego wielomianu to: $-16, 6, -3$.\\
Wielomian jest stopnia nieparzystego, ponadto znak współczynnika przy\linebreak najwyższej potędze x jest ujemny.\\ W związku z tym wykres wielomianu zaczyna się od lewej strony powyżej osi OX. A więc $$x \in (-\infty,-16) \cup (-3,6).$$
\rozwStop
\odpStart
$x \in (-\infty,-16) \cup (-3,6)$
\odpStop
\testStart
A.$x \in (-\infty,-16) \cup (-3,6)$\\
B.$x \in (-\infty,-16) \cup (-3,6]$\\
C.$x \in (-\infty,-16) \cup [-3,6)$\\
D.$x \in (-\infty,-16] \cup (-3,6)$\\
E.$x \in (-\infty,-16] \cup (-3,6]$\\
F.$x \in (-\infty,-16] \cup [-3,6)$\\
G.$x \in (-\infty,-16) \cup [-3,6]$\\
H.$x \in (-\infty,-16] \cup [-3,6]$
\testStop
\kluczStart
A
\kluczStop



\zadStart{Zadanie z Wikieł Z 1.62 b) moja wersja nr 469}

Rozwiązać nierówności $(x+16)(6-x)(x+4)\ge0$.
\zadStop
\rozwStart{Patryk Wirkus}{}
Miejsca zerowe naszego wielomianu to: $-16, 6, -4$.\\
Wielomian jest stopnia nieparzystego, ponadto znak współczynnika przy\linebreak najwyższej potędze x jest ujemny.\\ W związku z tym wykres wielomianu zaczyna się od lewej strony powyżej osi OX. A więc $$x \in (-\infty,-16) \cup (-4,6).$$
\rozwStop
\odpStart
$x \in (-\infty,-16) \cup (-4,6)$
\odpStop
\testStart
A.$x \in (-\infty,-16) \cup (-4,6)$\\
B.$x \in (-\infty,-16) \cup (-4,6]$\\
C.$x \in (-\infty,-16) \cup [-4,6)$\\
D.$x \in (-\infty,-16] \cup (-4,6)$\\
E.$x \in (-\infty,-16] \cup (-4,6]$\\
F.$x \in (-\infty,-16] \cup [-4,6)$\\
G.$x \in (-\infty,-16) \cup [-4,6]$\\
H.$x \in (-\infty,-16] \cup [-4,6]$
\testStop
\kluczStart
A
\kluczStop



\zadStart{Zadanie z Wikieł Z 1.62 b) moja wersja nr 470}

Rozwiązać nierówności $(x+16)(6-x)(x+5)\ge0$.
\zadStop
\rozwStart{Patryk Wirkus}{}
Miejsca zerowe naszego wielomianu to: $-16, 6, -5$.\\
Wielomian jest stopnia nieparzystego, ponadto znak współczynnika przy\linebreak najwyższej potędze x jest ujemny.\\ W związku z tym wykres wielomianu zaczyna się od lewej strony powyżej osi OX. A więc $$x \in (-\infty,-16) \cup (-5,6).$$
\rozwStop
\odpStart
$x \in (-\infty,-16) \cup (-5,6)$
\odpStop
\testStart
A.$x \in (-\infty,-16) \cup (-5,6)$\\
B.$x \in (-\infty,-16) \cup (-5,6]$\\
C.$x \in (-\infty,-16) \cup [-5,6)$\\
D.$x \in (-\infty,-16] \cup (-5,6)$\\
E.$x \in (-\infty,-16] \cup (-5,6]$\\
F.$x \in (-\infty,-16] \cup [-5,6)$\\
G.$x \in (-\infty,-16) \cup [-5,6]$\\
H.$x \in (-\infty,-16] \cup [-5,6]$
\testStop
\kluczStart
A
\kluczStop



\zadStart{Zadanie z Wikieł Z 1.62 b) moja wersja nr 471}

Rozwiązać nierówności $(x+16)(7-x)(x+1)\ge0$.
\zadStop
\rozwStart{Patryk Wirkus}{}
Miejsca zerowe naszego wielomianu to: $-16, 7, -1$.\\
Wielomian jest stopnia nieparzystego, ponadto znak współczynnika przy\linebreak najwyższej potędze x jest ujemny.\\ W związku z tym wykres wielomianu zaczyna się od lewej strony powyżej osi OX. A więc $$x \in (-\infty,-16) \cup (-1,7).$$
\rozwStop
\odpStart
$x \in (-\infty,-16) \cup (-1,7)$
\odpStop
\testStart
A.$x \in (-\infty,-16) \cup (-1,7)$\\
B.$x \in (-\infty,-16) \cup (-1,7]$\\
C.$x \in (-\infty,-16) \cup [-1,7)$\\
D.$x \in (-\infty,-16] \cup (-1,7)$\\
E.$x \in (-\infty,-16] \cup (-1,7]$\\
F.$x \in (-\infty,-16] \cup [-1,7)$\\
G.$x \in (-\infty,-16) \cup [-1,7]$\\
H.$x \in (-\infty,-16] \cup [-1,7]$
\testStop
\kluczStart
A
\kluczStop



\zadStart{Zadanie z Wikieł Z 1.62 b) moja wersja nr 472}

Rozwiązać nierówności $(x+16)(7-x)(x+2)\ge0$.
\zadStop
\rozwStart{Patryk Wirkus}{}
Miejsca zerowe naszego wielomianu to: $-16, 7, -2$.\\
Wielomian jest stopnia nieparzystego, ponadto znak współczynnika przy\linebreak najwyższej potędze x jest ujemny.\\ W związku z tym wykres wielomianu zaczyna się od lewej strony powyżej osi OX. A więc $$x \in (-\infty,-16) \cup (-2,7).$$
\rozwStop
\odpStart
$x \in (-\infty,-16) \cup (-2,7)$
\odpStop
\testStart
A.$x \in (-\infty,-16) \cup (-2,7)$\\
B.$x \in (-\infty,-16) \cup (-2,7]$\\
C.$x \in (-\infty,-16) \cup [-2,7)$\\
D.$x \in (-\infty,-16] \cup (-2,7)$\\
E.$x \in (-\infty,-16] \cup (-2,7]$\\
F.$x \in (-\infty,-16] \cup [-2,7)$\\
G.$x \in (-\infty,-16) \cup [-2,7]$\\
H.$x \in (-\infty,-16] \cup [-2,7]$
\testStop
\kluczStart
A
\kluczStop



\zadStart{Zadanie z Wikieł Z 1.62 b) moja wersja nr 473}

Rozwiązać nierówności $(x+16)(7-x)(x+3)\ge0$.
\zadStop
\rozwStart{Patryk Wirkus}{}
Miejsca zerowe naszego wielomianu to: $-16, 7, -3$.\\
Wielomian jest stopnia nieparzystego, ponadto znak współczynnika przy\linebreak najwyższej potędze x jest ujemny.\\ W związku z tym wykres wielomianu zaczyna się od lewej strony powyżej osi OX. A więc $$x \in (-\infty,-16) \cup (-3,7).$$
\rozwStop
\odpStart
$x \in (-\infty,-16) \cup (-3,7)$
\odpStop
\testStart
A.$x \in (-\infty,-16) \cup (-3,7)$\\
B.$x \in (-\infty,-16) \cup (-3,7]$\\
C.$x \in (-\infty,-16) \cup [-3,7)$\\
D.$x \in (-\infty,-16] \cup (-3,7)$\\
E.$x \in (-\infty,-16] \cup (-3,7]$\\
F.$x \in (-\infty,-16] \cup [-3,7)$\\
G.$x \in (-\infty,-16) \cup [-3,7]$\\
H.$x \in (-\infty,-16] \cup [-3,7]$
\testStop
\kluczStart
A
\kluczStop



\zadStart{Zadanie z Wikieł Z 1.62 b) moja wersja nr 474}

Rozwiązać nierówności $(x+16)(7-x)(x+4)\ge0$.
\zadStop
\rozwStart{Patryk Wirkus}{}
Miejsca zerowe naszego wielomianu to: $-16, 7, -4$.\\
Wielomian jest stopnia nieparzystego, ponadto znak współczynnika przy\linebreak najwyższej potędze x jest ujemny.\\ W związku z tym wykres wielomianu zaczyna się od lewej strony powyżej osi OX. A więc $$x \in (-\infty,-16) \cup (-4,7).$$
\rozwStop
\odpStart
$x \in (-\infty,-16) \cup (-4,7)$
\odpStop
\testStart
A.$x \in (-\infty,-16) \cup (-4,7)$\\
B.$x \in (-\infty,-16) \cup (-4,7]$\\
C.$x \in (-\infty,-16) \cup [-4,7)$\\
D.$x \in (-\infty,-16] \cup (-4,7)$\\
E.$x \in (-\infty,-16] \cup (-4,7]$\\
F.$x \in (-\infty,-16] \cup [-4,7)$\\
G.$x \in (-\infty,-16) \cup [-4,7]$\\
H.$x \in (-\infty,-16] \cup [-4,7]$
\testStop
\kluczStart
A
\kluczStop



\zadStart{Zadanie z Wikieł Z 1.62 b) moja wersja nr 475}

Rozwiązać nierówności $(x+16)(7-x)(x+5)\ge0$.
\zadStop
\rozwStart{Patryk Wirkus}{}
Miejsca zerowe naszego wielomianu to: $-16, 7, -5$.\\
Wielomian jest stopnia nieparzystego, ponadto znak współczynnika przy\linebreak najwyższej potędze x jest ujemny.\\ W związku z tym wykres wielomianu zaczyna się od lewej strony powyżej osi OX. A więc $$x \in (-\infty,-16) \cup (-5,7).$$
\rozwStop
\odpStart
$x \in (-\infty,-16) \cup (-5,7)$
\odpStop
\testStart
A.$x \in (-\infty,-16) \cup (-5,7)$\\
B.$x \in (-\infty,-16) \cup (-5,7]$\\
C.$x \in (-\infty,-16) \cup [-5,7)$\\
D.$x \in (-\infty,-16] \cup (-5,7)$\\
E.$x \in (-\infty,-16] \cup (-5,7]$\\
F.$x \in (-\infty,-16] \cup [-5,7)$\\
G.$x \in (-\infty,-16) \cup [-5,7]$\\
H.$x \in (-\infty,-16] \cup [-5,7]$
\testStop
\kluczStart
A
\kluczStop



\zadStart{Zadanie z Wikieł Z 1.62 b) moja wersja nr 476}

Rozwiązać nierówności $(x+16)(7-x)(x+6)\ge0$.
\zadStop
\rozwStart{Patryk Wirkus}{}
Miejsca zerowe naszego wielomianu to: $-16, 7, -6$.\\
Wielomian jest stopnia nieparzystego, ponadto znak współczynnika przy\linebreak najwyższej potędze x jest ujemny.\\ W związku z tym wykres wielomianu zaczyna się od lewej strony powyżej osi OX. A więc $$x \in (-\infty,-16) \cup (-6,7).$$
\rozwStop
\odpStart
$x \in (-\infty,-16) \cup (-6,7)$
\odpStop
\testStart
A.$x \in (-\infty,-16) \cup (-6,7)$\\
B.$x \in (-\infty,-16) \cup (-6,7]$\\
C.$x \in (-\infty,-16) \cup [-6,7)$\\
D.$x \in (-\infty,-16] \cup (-6,7)$\\
E.$x \in (-\infty,-16] \cup (-6,7]$\\
F.$x \in (-\infty,-16] \cup [-6,7)$\\
G.$x \in (-\infty,-16) \cup [-6,7]$\\
H.$x \in (-\infty,-16] \cup [-6,7]$
\testStop
\kluczStart
A
\kluczStop



\zadStart{Zadanie z Wikieł Z 1.62 b) moja wersja nr 477}

Rozwiązać nierówności $(x+16)(8-x)(x+1)\ge0$.
\zadStop
\rozwStart{Patryk Wirkus}{}
Miejsca zerowe naszego wielomianu to: $-16, 8, -1$.\\
Wielomian jest stopnia nieparzystego, ponadto znak współczynnika przy\linebreak najwyższej potędze x jest ujemny.\\ W związku z tym wykres wielomianu zaczyna się od lewej strony powyżej osi OX. A więc $$x \in (-\infty,-16) \cup (-1,8).$$
\rozwStop
\odpStart
$x \in (-\infty,-16) \cup (-1,8)$
\odpStop
\testStart
A.$x \in (-\infty,-16) \cup (-1,8)$\\
B.$x \in (-\infty,-16) \cup (-1,8]$\\
C.$x \in (-\infty,-16) \cup [-1,8)$\\
D.$x \in (-\infty,-16] \cup (-1,8)$\\
E.$x \in (-\infty,-16] \cup (-1,8]$\\
F.$x \in (-\infty,-16] \cup [-1,8)$\\
G.$x \in (-\infty,-16) \cup [-1,8]$\\
H.$x \in (-\infty,-16] \cup [-1,8]$
\testStop
\kluczStart
A
\kluczStop



\zadStart{Zadanie z Wikieł Z 1.62 b) moja wersja nr 478}

Rozwiązać nierówności $(x+16)(8-x)(x+2)\ge0$.
\zadStop
\rozwStart{Patryk Wirkus}{}
Miejsca zerowe naszego wielomianu to: $-16, 8, -2$.\\
Wielomian jest stopnia nieparzystego, ponadto znak współczynnika przy\linebreak najwyższej potędze x jest ujemny.\\ W związku z tym wykres wielomianu zaczyna się od lewej strony powyżej osi OX. A więc $$x \in (-\infty,-16) \cup (-2,8).$$
\rozwStop
\odpStart
$x \in (-\infty,-16) \cup (-2,8)$
\odpStop
\testStart
A.$x \in (-\infty,-16) \cup (-2,8)$\\
B.$x \in (-\infty,-16) \cup (-2,8]$\\
C.$x \in (-\infty,-16) \cup [-2,8)$\\
D.$x \in (-\infty,-16] \cup (-2,8)$\\
E.$x \in (-\infty,-16] \cup (-2,8]$\\
F.$x \in (-\infty,-16] \cup [-2,8)$\\
G.$x \in (-\infty,-16) \cup [-2,8]$\\
H.$x \in (-\infty,-16] \cup [-2,8]$
\testStop
\kluczStart
A
\kluczStop



\zadStart{Zadanie z Wikieł Z 1.62 b) moja wersja nr 479}

Rozwiązać nierówności $(x+16)(8-x)(x+3)\ge0$.
\zadStop
\rozwStart{Patryk Wirkus}{}
Miejsca zerowe naszego wielomianu to: $-16, 8, -3$.\\
Wielomian jest stopnia nieparzystego, ponadto znak współczynnika przy\linebreak najwyższej potędze x jest ujemny.\\ W związku z tym wykres wielomianu zaczyna się od lewej strony powyżej osi OX. A więc $$x \in (-\infty,-16) \cup (-3,8).$$
\rozwStop
\odpStart
$x \in (-\infty,-16) \cup (-3,8)$
\odpStop
\testStart
A.$x \in (-\infty,-16) \cup (-3,8)$\\
B.$x \in (-\infty,-16) \cup (-3,8]$\\
C.$x \in (-\infty,-16) \cup [-3,8)$\\
D.$x \in (-\infty,-16] \cup (-3,8)$\\
E.$x \in (-\infty,-16] \cup (-3,8]$\\
F.$x \in (-\infty,-16] \cup [-3,8)$\\
G.$x \in (-\infty,-16) \cup [-3,8]$\\
H.$x \in (-\infty,-16] \cup [-3,8]$
\testStop
\kluczStart
A
\kluczStop



\zadStart{Zadanie z Wikieł Z 1.62 b) moja wersja nr 480}

Rozwiązać nierówności $(x+16)(8-x)(x+4)\ge0$.
\zadStop
\rozwStart{Patryk Wirkus}{}
Miejsca zerowe naszego wielomianu to: $-16, 8, -4$.\\
Wielomian jest stopnia nieparzystego, ponadto znak współczynnika przy\linebreak najwyższej potędze x jest ujemny.\\ W związku z tym wykres wielomianu zaczyna się od lewej strony powyżej osi OX. A więc $$x \in (-\infty,-16) \cup (-4,8).$$
\rozwStop
\odpStart
$x \in (-\infty,-16) \cup (-4,8)$
\odpStop
\testStart
A.$x \in (-\infty,-16) \cup (-4,8)$\\
B.$x \in (-\infty,-16) \cup (-4,8]$\\
C.$x \in (-\infty,-16) \cup [-4,8)$\\
D.$x \in (-\infty,-16] \cup (-4,8)$\\
E.$x \in (-\infty,-16] \cup (-4,8]$\\
F.$x \in (-\infty,-16] \cup [-4,8)$\\
G.$x \in (-\infty,-16) \cup [-4,8]$\\
H.$x \in (-\infty,-16] \cup [-4,8]$
\testStop
\kluczStart
A
\kluczStop



\zadStart{Zadanie z Wikieł Z 1.62 b) moja wersja nr 481}

Rozwiązać nierówności $(x+16)(8-x)(x+5)\ge0$.
\zadStop
\rozwStart{Patryk Wirkus}{}
Miejsca zerowe naszego wielomianu to: $-16, 8, -5$.\\
Wielomian jest stopnia nieparzystego, ponadto znak współczynnika przy\linebreak najwyższej potędze x jest ujemny.\\ W związku z tym wykres wielomianu zaczyna się od lewej strony powyżej osi OX. A więc $$x \in (-\infty,-16) \cup (-5,8).$$
\rozwStop
\odpStart
$x \in (-\infty,-16) \cup (-5,8)$
\odpStop
\testStart
A.$x \in (-\infty,-16) \cup (-5,8)$\\
B.$x \in (-\infty,-16) \cup (-5,8]$\\
C.$x \in (-\infty,-16) \cup [-5,8)$\\
D.$x \in (-\infty,-16] \cup (-5,8)$\\
E.$x \in (-\infty,-16] \cup (-5,8]$\\
F.$x \in (-\infty,-16] \cup [-5,8)$\\
G.$x \in (-\infty,-16) \cup [-5,8]$\\
H.$x \in (-\infty,-16] \cup [-5,8]$
\testStop
\kluczStart
A
\kluczStop



\zadStart{Zadanie z Wikieł Z 1.62 b) moja wersja nr 482}

Rozwiązać nierówności $(x+16)(8-x)(x+6)\ge0$.
\zadStop
\rozwStart{Patryk Wirkus}{}
Miejsca zerowe naszego wielomianu to: $-16, 8, -6$.\\
Wielomian jest stopnia nieparzystego, ponadto znak współczynnika przy\linebreak najwyższej potędze x jest ujemny.\\ W związku z tym wykres wielomianu zaczyna się od lewej strony powyżej osi OX. A więc $$x \in (-\infty,-16) \cup (-6,8).$$
\rozwStop
\odpStart
$x \in (-\infty,-16) \cup (-6,8)$
\odpStop
\testStart
A.$x \in (-\infty,-16) \cup (-6,8)$\\
B.$x \in (-\infty,-16) \cup (-6,8]$\\
C.$x \in (-\infty,-16) \cup [-6,8)$\\
D.$x \in (-\infty,-16] \cup (-6,8)$\\
E.$x \in (-\infty,-16] \cup (-6,8]$\\
F.$x \in (-\infty,-16] \cup [-6,8)$\\
G.$x \in (-\infty,-16) \cup [-6,8]$\\
H.$x \in (-\infty,-16] \cup [-6,8]$
\testStop
\kluczStart
A
\kluczStop



\zadStart{Zadanie z Wikieł Z 1.62 b) moja wersja nr 483}

Rozwiązać nierówności $(x+16)(8-x)(x+7)\ge0$.
\zadStop
\rozwStart{Patryk Wirkus}{}
Miejsca zerowe naszego wielomianu to: $-16, 8, -7$.\\
Wielomian jest stopnia nieparzystego, ponadto znak współczynnika przy\linebreak najwyższej potędze x jest ujemny.\\ W związku z tym wykres wielomianu zaczyna się od lewej strony powyżej osi OX. A więc $$x \in (-\infty,-16) \cup (-7,8).$$
\rozwStop
\odpStart
$x \in (-\infty,-16) \cup (-7,8)$
\odpStop
\testStart
A.$x \in (-\infty,-16) \cup (-7,8)$\\
B.$x \in (-\infty,-16) \cup (-7,8]$\\
C.$x \in (-\infty,-16) \cup [-7,8)$\\
D.$x \in (-\infty,-16] \cup (-7,8)$\\
E.$x \in (-\infty,-16] \cup (-7,8]$\\
F.$x \in (-\infty,-16] \cup [-7,8)$\\
G.$x \in (-\infty,-16) \cup [-7,8]$\\
H.$x \in (-\infty,-16] \cup [-7,8]$
\testStop
\kluczStart
A
\kluczStop



\zadStart{Zadanie z Wikieł Z 1.62 b) moja wersja nr 484}

Rozwiązać nierówności $(x+16)(9-x)(x+1)\ge0$.
\zadStop
\rozwStart{Patryk Wirkus}{}
Miejsca zerowe naszego wielomianu to: $-16, 9, -1$.\\
Wielomian jest stopnia nieparzystego, ponadto znak współczynnika przy\linebreak najwyższej potędze x jest ujemny.\\ W związku z tym wykres wielomianu zaczyna się od lewej strony powyżej osi OX. A więc $$x \in (-\infty,-16) \cup (-1,9).$$
\rozwStop
\odpStart
$x \in (-\infty,-16) \cup (-1,9)$
\odpStop
\testStart
A.$x \in (-\infty,-16) \cup (-1,9)$\\
B.$x \in (-\infty,-16) \cup (-1,9]$\\
C.$x \in (-\infty,-16) \cup [-1,9)$\\
D.$x \in (-\infty,-16] \cup (-1,9)$\\
E.$x \in (-\infty,-16] \cup (-1,9]$\\
F.$x \in (-\infty,-16] \cup [-1,9)$\\
G.$x \in (-\infty,-16) \cup [-1,9]$\\
H.$x \in (-\infty,-16] \cup [-1,9]$
\testStop
\kluczStart
A
\kluczStop



\zadStart{Zadanie z Wikieł Z 1.62 b) moja wersja nr 485}

Rozwiązać nierówności $(x+16)(9-x)(x+2)\ge0$.
\zadStop
\rozwStart{Patryk Wirkus}{}
Miejsca zerowe naszego wielomianu to: $-16, 9, -2$.\\
Wielomian jest stopnia nieparzystego, ponadto znak współczynnika przy\linebreak najwyższej potędze x jest ujemny.\\ W związku z tym wykres wielomianu zaczyna się od lewej strony powyżej osi OX. A więc $$x \in (-\infty,-16) \cup (-2,9).$$
\rozwStop
\odpStart
$x \in (-\infty,-16) \cup (-2,9)$
\odpStop
\testStart
A.$x \in (-\infty,-16) \cup (-2,9)$\\
B.$x \in (-\infty,-16) \cup (-2,9]$\\
C.$x \in (-\infty,-16) \cup [-2,9)$\\
D.$x \in (-\infty,-16] \cup (-2,9)$\\
E.$x \in (-\infty,-16] \cup (-2,9]$\\
F.$x \in (-\infty,-16] \cup [-2,9)$\\
G.$x \in (-\infty,-16) \cup [-2,9]$\\
H.$x \in (-\infty,-16] \cup [-2,9]$
\testStop
\kluczStart
A
\kluczStop



\zadStart{Zadanie z Wikieł Z 1.62 b) moja wersja nr 486}

Rozwiązać nierówności $(x+16)(9-x)(x+3)\ge0$.
\zadStop
\rozwStart{Patryk Wirkus}{}
Miejsca zerowe naszego wielomianu to: $-16, 9, -3$.\\
Wielomian jest stopnia nieparzystego, ponadto znak współczynnika przy\linebreak najwyższej potędze x jest ujemny.\\ W związku z tym wykres wielomianu zaczyna się od lewej strony powyżej osi OX. A więc $$x \in (-\infty,-16) \cup (-3,9).$$
\rozwStop
\odpStart
$x \in (-\infty,-16) \cup (-3,9)$
\odpStop
\testStart
A.$x \in (-\infty,-16) \cup (-3,9)$\\
B.$x \in (-\infty,-16) \cup (-3,9]$\\
C.$x \in (-\infty,-16) \cup [-3,9)$\\
D.$x \in (-\infty,-16] \cup (-3,9)$\\
E.$x \in (-\infty,-16] \cup (-3,9]$\\
F.$x \in (-\infty,-16] \cup [-3,9)$\\
G.$x \in (-\infty,-16) \cup [-3,9]$\\
H.$x \in (-\infty,-16] \cup [-3,9]$
\testStop
\kluczStart
A
\kluczStop



\zadStart{Zadanie z Wikieł Z 1.62 b) moja wersja nr 487}

Rozwiązać nierówności $(x+16)(9-x)(x+4)\ge0$.
\zadStop
\rozwStart{Patryk Wirkus}{}
Miejsca zerowe naszego wielomianu to: $-16, 9, -4$.\\
Wielomian jest stopnia nieparzystego, ponadto znak współczynnika przy\linebreak najwyższej potędze x jest ujemny.\\ W związku z tym wykres wielomianu zaczyna się od lewej strony powyżej osi OX. A więc $$x \in (-\infty,-16) \cup (-4,9).$$
\rozwStop
\odpStart
$x \in (-\infty,-16) \cup (-4,9)$
\odpStop
\testStart
A.$x \in (-\infty,-16) \cup (-4,9)$\\
B.$x \in (-\infty,-16) \cup (-4,9]$\\
C.$x \in (-\infty,-16) \cup [-4,9)$\\
D.$x \in (-\infty,-16] \cup (-4,9)$\\
E.$x \in (-\infty,-16] \cup (-4,9]$\\
F.$x \in (-\infty,-16] \cup [-4,9)$\\
G.$x \in (-\infty,-16) \cup [-4,9]$\\
H.$x \in (-\infty,-16] \cup [-4,9]$
\testStop
\kluczStart
A
\kluczStop



\zadStart{Zadanie z Wikieł Z 1.62 b) moja wersja nr 488}

Rozwiązać nierówności $(x+16)(9-x)(x+5)\ge0$.
\zadStop
\rozwStart{Patryk Wirkus}{}
Miejsca zerowe naszego wielomianu to: $-16, 9, -5$.\\
Wielomian jest stopnia nieparzystego, ponadto znak współczynnika przy\linebreak najwyższej potędze x jest ujemny.\\ W związku z tym wykres wielomianu zaczyna się od lewej strony powyżej osi OX. A więc $$x \in (-\infty,-16) \cup (-5,9).$$
\rozwStop
\odpStart
$x \in (-\infty,-16) \cup (-5,9)$
\odpStop
\testStart
A.$x \in (-\infty,-16) \cup (-5,9)$\\
B.$x \in (-\infty,-16) \cup (-5,9]$\\
C.$x \in (-\infty,-16) \cup [-5,9)$\\
D.$x \in (-\infty,-16] \cup (-5,9)$\\
E.$x \in (-\infty,-16] \cup (-5,9]$\\
F.$x \in (-\infty,-16] \cup [-5,9)$\\
G.$x \in (-\infty,-16) \cup [-5,9]$\\
H.$x \in (-\infty,-16] \cup [-5,9]$
\testStop
\kluczStart
A
\kluczStop



\zadStart{Zadanie z Wikieł Z 1.62 b) moja wersja nr 489}

Rozwiązać nierówności $(x+16)(9-x)(x+6)\ge0$.
\zadStop
\rozwStart{Patryk Wirkus}{}
Miejsca zerowe naszego wielomianu to: $-16, 9, -6$.\\
Wielomian jest stopnia nieparzystego, ponadto znak współczynnika przy\linebreak najwyższej potędze x jest ujemny.\\ W związku z tym wykres wielomianu zaczyna się od lewej strony powyżej osi OX. A więc $$x \in (-\infty,-16) \cup (-6,9).$$
\rozwStop
\odpStart
$x \in (-\infty,-16) \cup (-6,9)$
\odpStop
\testStart
A.$x \in (-\infty,-16) \cup (-6,9)$\\
B.$x \in (-\infty,-16) \cup (-6,9]$\\
C.$x \in (-\infty,-16) \cup [-6,9)$\\
D.$x \in (-\infty,-16] \cup (-6,9)$\\
E.$x \in (-\infty,-16] \cup (-6,9]$\\
F.$x \in (-\infty,-16] \cup [-6,9)$\\
G.$x \in (-\infty,-16) \cup [-6,9]$\\
H.$x \in (-\infty,-16] \cup [-6,9]$
\testStop
\kluczStart
A
\kluczStop



\zadStart{Zadanie z Wikieł Z 1.62 b) moja wersja nr 490}

Rozwiązać nierówności $(x+16)(9-x)(x+7)\ge0$.
\zadStop
\rozwStart{Patryk Wirkus}{}
Miejsca zerowe naszego wielomianu to: $-16, 9, -7$.\\
Wielomian jest stopnia nieparzystego, ponadto znak współczynnika przy\linebreak najwyższej potędze x jest ujemny.\\ W związku z tym wykres wielomianu zaczyna się od lewej strony powyżej osi OX. A więc $$x \in (-\infty,-16) \cup (-7,9).$$
\rozwStop
\odpStart
$x \in (-\infty,-16) \cup (-7,9)$
\odpStop
\testStart
A.$x \in (-\infty,-16) \cup (-7,9)$\\
B.$x \in (-\infty,-16) \cup (-7,9]$\\
C.$x \in (-\infty,-16) \cup [-7,9)$\\
D.$x \in (-\infty,-16] \cup (-7,9)$\\
E.$x \in (-\infty,-16] \cup (-7,9]$\\
F.$x \in (-\infty,-16] \cup [-7,9)$\\
G.$x \in (-\infty,-16) \cup [-7,9]$\\
H.$x \in (-\infty,-16] \cup [-7,9]$
\testStop
\kluczStart
A
\kluczStop



\zadStart{Zadanie z Wikieł Z 1.62 b) moja wersja nr 491}

Rozwiązać nierówności $(x+16)(9-x)(x+8)\ge0$.
\zadStop
\rozwStart{Patryk Wirkus}{}
Miejsca zerowe naszego wielomianu to: $-16, 9, -8$.\\
Wielomian jest stopnia nieparzystego, ponadto znak współczynnika przy\linebreak najwyższej potędze x jest ujemny.\\ W związku z tym wykres wielomianu zaczyna się od lewej strony powyżej osi OX. A więc $$x \in (-\infty,-16) \cup (-8,9).$$
\rozwStop
\odpStart
$x \in (-\infty,-16) \cup (-8,9)$
\odpStop
\testStart
A.$x \in (-\infty,-16) \cup (-8,9)$\\
B.$x \in (-\infty,-16) \cup (-8,9]$\\
C.$x \in (-\infty,-16) \cup [-8,9)$\\
D.$x \in (-\infty,-16] \cup (-8,9)$\\
E.$x \in (-\infty,-16] \cup (-8,9]$\\
F.$x \in (-\infty,-16] \cup [-8,9)$\\
G.$x \in (-\infty,-16) \cup [-8,9]$\\
H.$x \in (-\infty,-16] \cup [-8,9]$
\testStop
\kluczStart
A
\kluczStop



\zadStart{Zadanie z Wikieł Z 1.62 b) moja wersja nr 492}

Rozwiązać nierówności $(x+16)(10-x)(x+1)\ge0$.
\zadStop
\rozwStart{Patryk Wirkus}{}
Miejsca zerowe naszego wielomianu to: $-16, 10, -1$.\\
Wielomian jest stopnia nieparzystego, ponadto znak współczynnika przy\linebreak najwyższej potędze x jest ujemny.\\ W związku z tym wykres wielomianu zaczyna się od lewej strony powyżej osi OX. A więc $$x \in (-\infty,-16) \cup (-1,10).$$
\rozwStop
\odpStart
$x \in (-\infty,-16) \cup (-1,10)$
\odpStop
\testStart
A.$x \in (-\infty,-16) \cup (-1,10)$\\
B.$x \in (-\infty,-16) \cup (-1,10]$\\
C.$x \in (-\infty,-16) \cup [-1,10)$\\
D.$x \in (-\infty,-16] \cup (-1,10)$\\
E.$x \in (-\infty,-16] \cup (-1,10]$\\
F.$x \in (-\infty,-16] \cup [-1,10)$\\
G.$x \in (-\infty,-16) \cup [-1,10]$\\
H.$x \in (-\infty,-16] \cup [-1,10]$
\testStop
\kluczStart
A
\kluczStop



\zadStart{Zadanie z Wikieł Z 1.62 b) moja wersja nr 493}

Rozwiązać nierówności $(x+16)(10-x)(x+2)\ge0$.
\zadStop
\rozwStart{Patryk Wirkus}{}
Miejsca zerowe naszego wielomianu to: $-16, 10, -2$.\\
Wielomian jest stopnia nieparzystego, ponadto znak współczynnika przy\linebreak najwyższej potędze x jest ujemny.\\ W związku z tym wykres wielomianu zaczyna się od lewej strony powyżej osi OX. A więc $$x \in (-\infty,-16) \cup (-2,10).$$
\rozwStop
\odpStart
$x \in (-\infty,-16) \cup (-2,10)$
\odpStop
\testStart
A.$x \in (-\infty,-16) \cup (-2,10)$\\
B.$x \in (-\infty,-16) \cup (-2,10]$\\
C.$x \in (-\infty,-16) \cup [-2,10)$\\
D.$x \in (-\infty,-16] \cup (-2,10)$\\
E.$x \in (-\infty,-16] \cup (-2,10]$\\
F.$x \in (-\infty,-16] \cup [-2,10)$\\
G.$x \in (-\infty,-16) \cup [-2,10]$\\
H.$x \in (-\infty,-16] \cup [-2,10]$
\testStop
\kluczStart
A
\kluczStop



\zadStart{Zadanie z Wikieł Z 1.62 b) moja wersja nr 494}

Rozwiązać nierówności $(x+16)(10-x)(x+3)\ge0$.
\zadStop
\rozwStart{Patryk Wirkus}{}
Miejsca zerowe naszego wielomianu to: $-16, 10, -3$.\\
Wielomian jest stopnia nieparzystego, ponadto znak współczynnika przy\linebreak najwyższej potędze x jest ujemny.\\ W związku z tym wykres wielomianu zaczyna się od lewej strony powyżej osi OX. A więc $$x \in (-\infty,-16) \cup (-3,10).$$
\rozwStop
\odpStart
$x \in (-\infty,-16) \cup (-3,10)$
\odpStop
\testStart
A.$x \in (-\infty,-16) \cup (-3,10)$\\
B.$x \in (-\infty,-16) \cup (-3,10]$\\
C.$x \in (-\infty,-16) \cup [-3,10)$\\
D.$x \in (-\infty,-16] \cup (-3,10)$\\
E.$x \in (-\infty,-16] \cup (-3,10]$\\
F.$x \in (-\infty,-16] \cup [-3,10)$\\
G.$x \in (-\infty,-16) \cup [-3,10]$\\
H.$x \in (-\infty,-16] \cup [-3,10]$
\testStop
\kluczStart
A
\kluczStop



\zadStart{Zadanie z Wikieł Z 1.62 b) moja wersja nr 495}

Rozwiązać nierówności $(x+16)(10-x)(x+4)\ge0$.
\zadStop
\rozwStart{Patryk Wirkus}{}
Miejsca zerowe naszego wielomianu to: $-16, 10, -4$.\\
Wielomian jest stopnia nieparzystego, ponadto znak współczynnika przy\linebreak najwyższej potędze x jest ujemny.\\ W związku z tym wykres wielomianu zaczyna się od lewej strony powyżej osi OX. A więc $$x \in (-\infty,-16) \cup (-4,10).$$
\rozwStop
\odpStart
$x \in (-\infty,-16) \cup (-4,10)$
\odpStop
\testStart
A.$x \in (-\infty,-16) \cup (-4,10)$\\
B.$x \in (-\infty,-16) \cup (-4,10]$\\
C.$x \in (-\infty,-16) \cup [-4,10)$\\
D.$x \in (-\infty,-16] \cup (-4,10)$\\
E.$x \in (-\infty,-16] \cup (-4,10]$\\
F.$x \in (-\infty,-16] \cup [-4,10)$\\
G.$x \in (-\infty,-16) \cup [-4,10]$\\
H.$x \in (-\infty,-16] \cup [-4,10]$
\testStop
\kluczStart
A
\kluczStop



\zadStart{Zadanie z Wikieł Z 1.62 b) moja wersja nr 496}

Rozwiązać nierówności $(x+16)(10-x)(x+5)\ge0$.
\zadStop
\rozwStart{Patryk Wirkus}{}
Miejsca zerowe naszego wielomianu to: $-16, 10, -5$.\\
Wielomian jest stopnia nieparzystego, ponadto znak współczynnika przy\linebreak najwyższej potędze x jest ujemny.\\ W związku z tym wykres wielomianu zaczyna się od lewej strony powyżej osi OX. A więc $$x \in (-\infty,-16) \cup (-5,10).$$
\rozwStop
\odpStart
$x \in (-\infty,-16) \cup (-5,10)$
\odpStop
\testStart
A.$x \in (-\infty,-16) \cup (-5,10)$\\
B.$x \in (-\infty,-16) \cup (-5,10]$\\
C.$x \in (-\infty,-16) \cup [-5,10)$\\
D.$x \in (-\infty,-16] \cup (-5,10)$\\
E.$x \in (-\infty,-16] \cup (-5,10]$\\
F.$x \in (-\infty,-16] \cup [-5,10)$\\
G.$x \in (-\infty,-16) \cup [-5,10]$\\
H.$x \in (-\infty,-16] \cup [-5,10]$
\testStop
\kluczStart
A
\kluczStop



\zadStart{Zadanie z Wikieł Z 1.62 b) moja wersja nr 497}

Rozwiązać nierówności $(x+16)(10-x)(x+6)\ge0$.
\zadStop
\rozwStart{Patryk Wirkus}{}
Miejsca zerowe naszego wielomianu to: $-16, 10, -6$.\\
Wielomian jest stopnia nieparzystego, ponadto znak współczynnika przy\linebreak najwyższej potędze x jest ujemny.\\ W związku z tym wykres wielomianu zaczyna się od lewej strony powyżej osi OX. A więc $$x \in (-\infty,-16) \cup (-6,10).$$
\rozwStop
\odpStart
$x \in (-\infty,-16) \cup (-6,10)$
\odpStop
\testStart
A.$x \in (-\infty,-16) \cup (-6,10)$\\
B.$x \in (-\infty,-16) \cup (-6,10]$\\
C.$x \in (-\infty,-16) \cup [-6,10)$\\
D.$x \in (-\infty,-16] \cup (-6,10)$\\
E.$x \in (-\infty,-16] \cup (-6,10]$\\
F.$x \in (-\infty,-16] \cup [-6,10)$\\
G.$x \in (-\infty,-16) \cup [-6,10]$\\
H.$x \in (-\infty,-16] \cup [-6,10]$
\testStop
\kluczStart
A
\kluczStop



\zadStart{Zadanie z Wikieł Z 1.62 b) moja wersja nr 498}

Rozwiązać nierówności $(x+16)(10-x)(x+7)\ge0$.
\zadStop
\rozwStart{Patryk Wirkus}{}
Miejsca zerowe naszego wielomianu to: $-16, 10, -7$.\\
Wielomian jest stopnia nieparzystego, ponadto znak współczynnika przy\linebreak najwyższej potędze x jest ujemny.\\ W związku z tym wykres wielomianu zaczyna się od lewej strony powyżej osi OX. A więc $$x \in (-\infty,-16) \cup (-7,10).$$
\rozwStop
\odpStart
$x \in (-\infty,-16) \cup (-7,10)$
\odpStop
\testStart
A.$x \in (-\infty,-16) \cup (-7,10)$\\
B.$x \in (-\infty,-16) \cup (-7,10]$\\
C.$x \in (-\infty,-16) \cup [-7,10)$\\
D.$x \in (-\infty,-16] \cup (-7,10)$\\
E.$x \in (-\infty,-16] \cup (-7,10]$\\
F.$x \in (-\infty,-16] \cup [-7,10)$\\
G.$x \in (-\infty,-16) \cup [-7,10]$\\
H.$x \in (-\infty,-16] \cup [-7,10]$
\testStop
\kluczStart
A
\kluczStop



\zadStart{Zadanie z Wikieł Z 1.62 b) moja wersja nr 499}

Rozwiązać nierówności $(x+16)(10-x)(x+8)\ge0$.
\zadStop
\rozwStart{Patryk Wirkus}{}
Miejsca zerowe naszego wielomianu to: $-16, 10, -8$.\\
Wielomian jest stopnia nieparzystego, ponadto znak współczynnika przy\linebreak najwyższej potędze x jest ujemny.\\ W związku z tym wykres wielomianu zaczyna się od lewej strony powyżej osi OX. A więc $$x \in (-\infty,-16) \cup (-8,10).$$
\rozwStop
\odpStart
$x \in (-\infty,-16) \cup (-8,10)$
\odpStop
\testStart
A.$x \in (-\infty,-16) \cup (-8,10)$\\
B.$x \in (-\infty,-16) \cup (-8,10]$\\
C.$x \in (-\infty,-16) \cup [-8,10)$\\
D.$x \in (-\infty,-16] \cup (-8,10)$\\
E.$x \in (-\infty,-16] \cup (-8,10]$\\
F.$x \in (-\infty,-16] \cup [-8,10)$\\
G.$x \in (-\infty,-16) \cup [-8,10]$\\
H.$x \in (-\infty,-16] \cup [-8,10]$
\testStop
\kluczStart
A
\kluczStop



\zadStart{Zadanie z Wikieł Z 1.62 b) moja wersja nr 500}

Rozwiązać nierówności $(x+16)(10-x)(x+9)\ge0$.
\zadStop
\rozwStart{Patryk Wirkus}{}
Miejsca zerowe naszego wielomianu to: $-16, 10, -9$.\\
Wielomian jest stopnia nieparzystego, ponadto znak współczynnika przy\linebreak najwyższej potędze x jest ujemny.\\ W związku z tym wykres wielomianu zaczyna się od lewej strony powyżej osi OX. A więc $$x \in (-\infty,-16) \cup (-9,10).$$
\rozwStop
\odpStart
$x \in (-\infty,-16) \cup (-9,10)$
\odpStop
\testStart
A.$x \in (-\infty,-16) \cup (-9,10)$\\
B.$x \in (-\infty,-16) \cup (-9,10]$\\
C.$x \in (-\infty,-16) \cup [-9,10)$\\
D.$x \in (-\infty,-16] \cup (-9,10)$\\
E.$x \in (-\infty,-16] \cup (-9,10]$\\
F.$x \in (-\infty,-16] \cup [-9,10)$\\
G.$x \in (-\infty,-16) \cup [-9,10]$\\
H.$x \in (-\infty,-16] \cup [-9,10]$
\testStop
\kluczStart
A
\kluczStop



\zadStart{Zadanie z Wikieł Z 1.62 b) moja wersja nr 501}

Rozwiązać nierówności $(x+16)(11-x)(x+1)\ge0$.
\zadStop
\rozwStart{Patryk Wirkus}{}
Miejsca zerowe naszego wielomianu to: $-16, 11, -1$.\\
Wielomian jest stopnia nieparzystego, ponadto znak współczynnika przy\linebreak najwyższej potędze x jest ujemny.\\ W związku z tym wykres wielomianu zaczyna się od lewej strony powyżej osi OX. A więc $$x \in (-\infty,-16) \cup (-1,11).$$
\rozwStop
\odpStart
$x \in (-\infty,-16) \cup (-1,11)$
\odpStop
\testStart
A.$x \in (-\infty,-16) \cup (-1,11)$\\
B.$x \in (-\infty,-16) \cup (-1,11]$\\
C.$x \in (-\infty,-16) \cup [-1,11)$\\
D.$x \in (-\infty,-16] \cup (-1,11)$\\
E.$x \in (-\infty,-16] \cup (-1,11]$\\
F.$x \in (-\infty,-16] \cup [-1,11)$\\
G.$x \in (-\infty,-16) \cup [-1,11]$\\
H.$x \in (-\infty,-16] \cup [-1,11]$
\testStop
\kluczStart
A
\kluczStop



\zadStart{Zadanie z Wikieł Z 1.62 b) moja wersja nr 502}

Rozwiązać nierówności $(x+16)(11-x)(x+2)\ge0$.
\zadStop
\rozwStart{Patryk Wirkus}{}
Miejsca zerowe naszego wielomianu to: $-16, 11, -2$.\\
Wielomian jest stopnia nieparzystego, ponadto znak współczynnika przy\linebreak najwyższej potędze x jest ujemny.\\ W związku z tym wykres wielomianu zaczyna się od lewej strony powyżej osi OX. A więc $$x \in (-\infty,-16) \cup (-2,11).$$
\rozwStop
\odpStart
$x \in (-\infty,-16) \cup (-2,11)$
\odpStop
\testStart
A.$x \in (-\infty,-16) \cup (-2,11)$\\
B.$x \in (-\infty,-16) \cup (-2,11]$\\
C.$x \in (-\infty,-16) \cup [-2,11)$\\
D.$x \in (-\infty,-16] \cup (-2,11)$\\
E.$x \in (-\infty,-16] \cup (-2,11]$\\
F.$x \in (-\infty,-16] \cup [-2,11)$\\
G.$x \in (-\infty,-16) \cup [-2,11]$\\
H.$x \in (-\infty,-16] \cup [-2,11]$
\testStop
\kluczStart
A
\kluczStop



\zadStart{Zadanie z Wikieł Z 1.62 b) moja wersja nr 503}

Rozwiązać nierówności $(x+16)(11-x)(x+3)\ge0$.
\zadStop
\rozwStart{Patryk Wirkus}{}
Miejsca zerowe naszego wielomianu to: $-16, 11, -3$.\\
Wielomian jest stopnia nieparzystego, ponadto znak współczynnika przy\linebreak najwyższej potędze x jest ujemny.\\ W związku z tym wykres wielomianu zaczyna się od lewej strony powyżej osi OX. A więc $$x \in (-\infty,-16) \cup (-3,11).$$
\rozwStop
\odpStart
$x \in (-\infty,-16) \cup (-3,11)$
\odpStop
\testStart
A.$x \in (-\infty,-16) \cup (-3,11)$\\
B.$x \in (-\infty,-16) \cup (-3,11]$\\
C.$x \in (-\infty,-16) \cup [-3,11)$\\
D.$x \in (-\infty,-16] \cup (-3,11)$\\
E.$x \in (-\infty,-16] \cup (-3,11]$\\
F.$x \in (-\infty,-16] \cup [-3,11)$\\
G.$x \in (-\infty,-16) \cup [-3,11]$\\
H.$x \in (-\infty,-16] \cup [-3,11]$
\testStop
\kluczStart
A
\kluczStop



\zadStart{Zadanie z Wikieł Z 1.62 b) moja wersja nr 504}

Rozwiązać nierówności $(x+16)(11-x)(x+4)\ge0$.
\zadStop
\rozwStart{Patryk Wirkus}{}
Miejsca zerowe naszego wielomianu to: $-16, 11, -4$.\\
Wielomian jest stopnia nieparzystego, ponadto znak współczynnika przy\linebreak najwyższej potędze x jest ujemny.\\ W związku z tym wykres wielomianu zaczyna się od lewej strony powyżej osi OX. A więc $$x \in (-\infty,-16) \cup (-4,11).$$
\rozwStop
\odpStart
$x \in (-\infty,-16) \cup (-4,11)$
\odpStop
\testStart
A.$x \in (-\infty,-16) \cup (-4,11)$\\
B.$x \in (-\infty,-16) \cup (-4,11]$\\
C.$x \in (-\infty,-16) \cup [-4,11)$\\
D.$x \in (-\infty,-16] \cup (-4,11)$\\
E.$x \in (-\infty,-16] \cup (-4,11]$\\
F.$x \in (-\infty,-16] \cup [-4,11)$\\
G.$x \in (-\infty,-16) \cup [-4,11]$\\
H.$x \in (-\infty,-16] \cup [-4,11]$
\testStop
\kluczStart
A
\kluczStop



\zadStart{Zadanie z Wikieł Z 1.62 b) moja wersja nr 505}

Rozwiązać nierówności $(x+16)(11-x)(x+5)\ge0$.
\zadStop
\rozwStart{Patryk Wirkus}{}
Miejsca zerowe naszego wielomianu to: $-16, 11, -5$.\\
Wielomian jest stopnia nieparzystego, ponadto znak współczynnika przy\linebreak najwyższej potędze x jest ujemny.\\ W związku z tym wykres wielomianu zaczyna się od lewej strony powyżej osi OX. A więc $$x \in (-\infty,-16) \cup (-5,11).$$
\rozwStop
\odpStart
$x \in (-\infty,-16) \cup (-5,11)$
\odpStop
\testStart
A.$x \in (-\infty,-16) \cup (-5,11)$\\
B.$x \in (-\infty,-16) \cup (-5,11]$\\
C.$x \in (-\infty,-16) \cup [-5,11)$\\
D.$x \in (-\infty,-16] \cup (-5,11)$\\
E.$x \in (-\infty,-16] \cup (-5,11]$\\
F.$x \in (-\infty,-16] \cup [-5,11)$\\
G.$x \in (-\infty,-16) \cup [-5,11]$\\
H.$x \in (-\infty,-16] \cup [-5,11]$
\testStop
\kluczStart
A
\kluczStop



\zadStart{Zadanie z Wikieł Z 1.62 b) moja wersja nr 506}

Rozwiązać nierówności $(x+16)(11-x)(x+6)\ge0$.
\zadStop
\rozwStart{Patryk Wirkus}{}
Miejsca zerowe naszego wielomianu to: $-16, 11, -6$.\\
Wielomian jest stopnia nieparzystego, ponadto znak współczynnika przy\linebreak najwyższej potędze x jest ujemny.\\ W związku z tym wykres wielomianu zaczyna się od lewej strony powyżej osi OX. A więc $$x \in (-\infty,-16) \cup (-6,11).$$
\rozwStop
\odpStart
$x \in (-\infty,-16) \cup (-6,11)$
\odpStop
\testStart
A.$x \in (-\infty,-16) \cup (-6,11)$\\
B.$x \in (-\infty,-16) \cup (-6,11]$\\
C.$x \in (-\infty,-16) \cup [-6,11)$\\
D.$x \in (-\infty,-16] \cup (-6,11)$\\
E.$x \in (-\infty,-16] \cup (-6,11]$\\
F.$x \in (-\infty,-16] \cup [-6,11)$\\
G.$x \in (-\infty,-16) \cup [-6,11]$\\
H.$x \in (-\infty,-16] \cup [-6,11]$
\testStop
\kluczStart
A
\kluczStop



\zadStart{Zadanie z Wikieł Z 1.62 b) moja wersja nr 507}

Rozwiązać nierówności $(x+16)(11-x)(x+7)\ge0$.
\zadStop
\rozwStart{Patryk Wirkus}{}
Miejsca zerowe naszego wielomianu to: $-16, 11, -7$.\\
Wielomian jest stopnia nieparzystego, ponadto znak współczynnika przy\linebreak najwyższej potędze x jest ujemny.\\ W związku z tym wykres wielomianu zaczyna się od lewej strony powyżej osi OX. A więc $$x \in (-\infty,-16) \cup (-7,11).$$
\rozwStop
\odpStart
$x \in (-\infty,-16) \cup (-7,11)$
\odpStop
\testStart
A.$x \in (-\infty,-16) \cup (-7,11)$\\
B.$x \in (-\infty,-16) \cup (-7,11]$\\
C.$x \in (-\infty,-16) \cup [-7,11)$\\
D.$x \in (-\infty,-16] \cup (-7,11)$\\
E.$x \in (-\infty,-16] \cup (-7,11]$\\
F.$x \in (-\infty,-16] \cup [-7,11)$\\
G.$x \in (-\infty,-16) \cup [-7,11]$\\
H.$x \in (-\infty,-16] \cup [-7,11]$
\testStop
\kluczStart
A
\kluczStop



\zadStart{Zadanie z Wikieł Z 1.62 b) moja wersja nr 508}

Rozwiązać nierówności $(x+16)(11-x)(x+8)\ge0$.
\zadStop
\rozwStart{Patryk Wirkus}{}
Miejsca zerowe naszego wielomianu to: $-16, 11, -8$.\\
Wielomian jest stopnia nieparzystego, ponadto znak współczynnika przy\linebreak najwyższej potędze x jest ujemny.\\ W związku z tym wykres wielomianu zaczyna się od lewej strony powyżej osi OX. A więc $$x \in (-\infty,-16) \cup (-8,11).$$
\rozwStop
\odpStart
$x \in (-\infty,-16) \cup (-8,11)$
\odpStop
\testStart
A.$x \in (-\infty,-16) \cup (-8,11)$\\
B.$x \in (-\infty,-16) \cup (-8,11]$\\
C.$x \in (-\infty,-16) \cup [-8,11)$\\
D.$x \in (-\infty,-16] \cup (-8,11)$\\
E.$x \in (-\infty,-16] \cup (-8,11]$\\
F.$x \in (-\infty,-16] \cup [-8,11)$\\
G.$x \in (-\infty,-16) \cup [-8,11]$\\
H.$x \in (-\infty,-16] \cup [-8,11]$
\testStop
\kluczStart
A
\kluczStop



\zadStart{Zadanie z Wikieł Z 1.62 b) moja wersja nr 509}

Rozwiązać nierówności $(x+16)(11-x)(x+9)\ge0$.
\zadStop
\rozwStart{Patryk Wirkus}{}
Miejsca zerowe naszego wielomianu to: $-16, 11, -9$.\\
Wielomian jest stopnia nieparzystego, ponadto znak współczynnika przy\linebreak najwyższej potędze x jest ujemny.\\ W związku z tym wykres wielomianu zaczyna się od lewej strony powyżej osi OX. A więc $$x \in (-\infty,-16) \cup (-9,11).$$
\rozwStop
\odpStart
$x \in (-\infty,-16) \cup (-9,11)$
\odpStop
\testStart
A.$x \in (-\infty,-16) \cup (-9,11)$\\
B.$x \in (-\infty,-16) \cup (-9,11]$\\
C.$x \in (-\infty,-16) \cup [-9,11)$\\
D.$x \in (-\infty,-16] \cup (-9,11)$\\
E.$x \in (-\infty,-16] \cup (-9,11]$\\
F.$x \in (-\infty,-16] \cup [-9,11)$\\
G.$x \in (-\infty,-16) \cup [-9,11]$\\
H.$x \in (-\infty,-16] \cup [-9,11]$
\testStop
\kluczStart
A
\kluczStop



\zadStart{Zadanie z Wikieł Z 1.62 b) moja wersja nr 510}

Rozwiązać nierówności $(x+16)(11-x)(x+10)\ge0$.
\zadStop
\rozwStart{Patryk Wirkus}{}
Miejsca zerowe naszego wielomianu to: $-16, 11, -10$.\\
Wielomian jest stopnia nieparzystego, ponadto znak współczynnika przy\linebreak najwyższej potędze x jest ujemny.\\ W związku z tym wykres wielomianu zaczyna się od lewej strony powyżej osi OX. A więc $$x \in (-\infty,-16) \cup (-10,11).$$
\rozwStop
\odpStart
$x \in (-\infty,-16) \cup (-10,11)$
\odpStop
\testStart
A.$x \in (-\infty,-16) \cup (-10,11)$\\
B.$x \in (-\infty,-16) \cup (-10,11]$\\
C.$x \in (-\infty,-16) \cup [-10,11)$\\
D.$x \in (-\infty,-16] \cup (-10,11)$\\
E.$x \in (-\infty,-16] \cup (-10,11]$\\
F.$x \in (-\infty,-16] \cup [-10,11)$\\
G.$x \in (-\infty,-16) \cup [-10,11]$\\
H.$x \in (-\infty,-16] \cup [-10,11]$
\testStop
\kluczStart
A
\kluczStop



\zadStart{Zadanie z Wikieł Z 1.62 b) moja wersja nr 511}

Rozwiązać nierówności $(x+16)(12-x)(x+1)\ge0$.
\zadStop
\rozwStart{Patryk Wirkus}{}
Miejsca zerowe naszego wielomianu to: $-16, 12, -1$.\\
Wielomian jest stopnia nieparzystego, ponadto znak współczynnika przy\linebreak najwyższej potędze x jest ujemny.\\ W związku z tym wykres wielomianu zaczyna się od lewej strony powyżej osi OX. A więc $$x \in (-\infty,-16) \cup (-1,12).$$
\rozwStop
\odpStart
$x \in (-\infty,-16) \cup (-1,12)$
\odpStop
\testStart
A.$x \in (-\infty,-16) \cup (-1,12)$\\
B.$x \in (-\infty,-16) \cup (-1,12]$\\
C.$x \in (-\infty,-16) \cup [-1,12)$\\
D.$x \in (-\infty,-16] \cup (-1,12)$\\
E.$x \in (-\infty,-16] \cup (-1,12]$\\
F.$x \in (-\infty,-16] \cup [-1,12)$\\
G.$x \in (-\infty,-16) \cup [-1,12]$\\
H.$x \in (-\infty,-16] \cup [-1,12]$
\testStop
\kluczStart
A
\kluczStop



\zadStart{Zadanie z Wikieł Z 1.62 b) moja wersja nr 512}

Rozwiązać nierówności $(x+16)(12-x)(x+2)\ge0$.
\zadStop
\rozwStart{Patryk Wirkus}{}
Miejsca zerowe naszego wielomianu to: $-16, 12, -2$.\\
Wielomian jest stopnia nieparzystego, ponadto znak współczynnika przy\linebreak najwyższej potędze x jest ujemny.\\ W związku z tym wykres wielomianu zaczyna się od lewej strony powyżej osi OX. A więc $$x \in (-\infty,-16) \cup (-2,12).$$
\rozwStop
\odpStart
$x \in (-\infty,-16) \cup (-2,12)$
\odpStop
\testStart
A.$x \in (-\infty,-16) \cup (-2,12)$\\
B.$x \in (-\infty,-16) \cup (-2,12]$\\
C.$x \in (-\infty,-16) \cup [-2,12)$\\
D.$x \in (-\infty,-16] \cup (-2,12)$\\
E.$x \in (-\infty,-16] \cup (-2,12]$\\
F.$x \in (-\infty,-16] \cup [-2,12)$\\
G.$x \in (-\infty,-16) \cup [-2,12]$\\
H.$x \in (-\infty,-16] \cup [-2,12]$
\testStop
\kluczStart
A
\kluczStop



\zadStart{Zadanie z Wikieł Z 1.62 b) moja wersja nr 513}

Rozwiązać nierówności $(x+16)(12-x)(x+3)\ge0$.
\zadStop
\rozwStart{Patryk Wirkus}{}
Miejsca zerowe naszego wielomianu to: $-16, 12, -3$.\\
Wielomian jest stopnia nieparzystego, ponadto znak współczynnika przy\linebreak najwyższej potędze x jest ujemny.\\ W związku z tym wykres wielomianu zaczyna się od lewej strony powyżej osi OX. A więc $$x \in (-\infty,-16) \cup (-3,12).$$
\rozwStop
\odpStart
$x \in (-\infty,-16) \cup (-3,12)$
\odpStop
\testStart
A.$x \in (-\infty,-16) \cup (-3,12)$\\
B.$x \in (-\infty,-16) \cup (-3,12]$\\
C.$x \in (-\infty,-16) \cup [-3,12)$\\
D.$x \in (-\infty,-16] \cup (-3,12)$\\
E.$x \in (-\infty,-16] \cup (-3,12]$\\
F.$x \in (-\infty,-16] \cup [-3,12)$\\
G.$x \in (-\infty,-16) \cup [-3,12]$\\
H.$x \in (-\infty,-16] \cup [-3,12]$
\testStop
\kluczStart
A
\kluczStop



\zadStart{Zadanie z Wikieł Z 1.62 b) moja wersja nr 514}

Rozwiązać nierówności $(x+16)(12-x)(x+4)\ge0$.
\zadStop
\rozwStart{Patryk Wirkus}{}
Miejsca zerowe naszego wielomianu to: $-16, 12, -4$.\\
Wielomian jest stopnia nieparzystego, ponadto znak współczynnika przy\linebreak najwyższej potędze x jest ujemny.\\ W związku z tym wykres wielomianu zaczyna się od lewej strony powyżej osi OX. A więc $$x \in (-\infty,-16) \cup (-4,12).$$
\rozwStop
\odpStart
$x \in (-\infty,-16) \cup (-4,12)$
\odpStop
\testStart
A.$x \in (-\infty,-16) \cup (-4,12)$\\
B.$x \in (-\infty,-16) \cup (-4,12]$\\
C.$x \in (-\infty,-16) \cup [-4,12)$\\
D.$x \in (-\infty,-16] \cup (-4,12)$\\
E.$x \in (-\infty,-16] \cup (-4,12]$\\
F.$x \in (-\infty,-16] \cup [-4,12)$\\
G.$x \in (-\infty,-16) \cup [-4,12]$\\
H.$x \in (-\infty,-16] \cup [-4,12]$
\testStop
\kluczStart
A
\kluczStop



\zadStart{Zadanie z Wikieł Z 1.62 b) moja wersja nr 515}

Rozwiązać nierówności $(x+16)(12-x)(x+5)\ge0$.
\zadStop
\rozwStart{Patryk Wirkus}{}
Miejsca zerowe naszego wielomianu to: $-16, 12, -5$.\\
Wielomian jest stopnia nieparzystego, ponadto znak współczynnika przy\linebreak najwyższej potędze x jest ujemny.\\ W związku z tym wykres wielomianu zaczyna się od lewej strony powyżej osi OX. A więc $$x \in (-\infty,-16) \cup (-5,12).$$
\rozwStop
\odpStart
$x \in (-\infty,-16) \cup (-5,12)$
\odpStop
\testStart
A.$x \in (-\infty,-16) \cup (-5,12)$\\
B.$x \in (-\infty,-16) \cup (-5,12]$\\
C.$x \in (-\infty,-16) \cup [-5,12)$\\
D.$x \in (-\infty,-16] \cup (-5,12)$\\
E.$x \in (-\infty,-16] \cup (-5,12]$\\
F.$x \in (-\infty,-16] \cup [-5,12)$\\
G.$x \in (-\infty,-16) \cup [-5,12]$\\
H.$x \in (-\infty,-16] \cup [-5,12]$
\testStop
\kluczStart
A
\kluczStop



\zadStart{Zadanie z Wikieł Z 1.62 b) moja wersja nr 516}

Rozwiązać nierówności $(x+16)(12-x)(x+6)\ge0$.
\zadStop
\rozwStart{Patryk Wirkus}{}
Miejsca zerowe naszego wielomianu to: $-16, 12, -6$.\\
Wielomian jest stopnia nieparzystego, ponadto znak współczynnika przy\linebreak najwyższej potędze x jest ujemny.\\ W związku z tym wykres wielomianu zaczyna się od lewej strony powyżej osi OX. A więc $$x \in (-\infty,-16) \cup (-6,12).$$
\rozwStop
\odpStart
$x \in (-\infty,-16) \cup (-6,12)$
\odpStop
\testStart
A.$x \in (-\infty,-16) \cup (-6,12)$\\
B.$x \in (-\infty,-16) \cup (-6,12]$\\
C.$x \in (-\infty,-16) \cup [-6,12)$\\
D.$x \in (-\infty,-16] \cup (-6,12)$\\
E.$x \in (-\infty,-16] \cup (-6,12]$\\
F.$x \in (-\infty,-16] \cup [-6,12)$\\
G.$x \in (-\infty,-16) \cup [-6,12]$\\
H.$x \in (-\infty,-16] \cup [-6,12]$
\testStop
\kluczStart
A
\kluczStop



\zadStart{Zadanie z Wikieł Z 1.62 b) moja wersja nr 517}

Rozwiązać nierówności $(x+16)(12-x)(x+7)\ge0$.
\zadStop
\rozwStart{Patryk Wirkus}{}
Miejsca zerowe naszego wielomianu to: $-16, 12, -7$.\\
Wielomian jest stopnia nieparzystego, ponadto znak współczynnika przy\linebreak najwyższej potędze x jest ujemny.\\ W związku z tym wykres wielomianu zaczyna się od lewej strony powyżej osi OX. A więc $$x \in (-\infty,-16) \cup (-7,12).$$
\rozwStop
\odpStart
$x \in (-\infty,-16) \cup (-7,12)$
\odpStop
\testStart
A.$x \in (-\infty,-16) \cup (-7,12)$\\
B.$x \in (-\infty,-16) \cup (-7,12]$\\
C.$x \in (-\infty,-16) \cup [-7,12)$\\
D.$x \in (-\infty,-16] \cup (-7,12)$\\
E.$x \in (-\infty,-16] \cup (-7,12]$\\
F.$x \in (-\infty,-16] \cup [-7,12)$\\
G.$x \in (-\infty,-16) \cup [-7,12]$\\
H.$x \in (-\infty,-16] \cup [-7,12]$
\testStop
\kluczStart
A
\kluczStop



\zadStart{Zadanie z Wikieł Z 1.62 b) moja wersja nr 518}

Rozwiązać nierówności $(x+16)(12-x)(x+8)\ge0$.
\zadStop
\rozwStart{Patryk Wirkus}{}
Miejsca zerowe naszego wielomianu to: $-16, 12, -8$.\\
Wielomian jest stopnia nieparzystego, ponadto znak współczynnika przy\linebreak najwyższej potędze x jest ujemny.\\ W związku z tym wykres wielomianu zaczyna się od lewej strony powyżej osi OX. A więc $$x \in (-\infty,-16) \cup (-8,12).$$
\rozwStop
\odpStart
$x \in (-\infty,-16) \cup (-8,12)$
\odpStop
\testStart
A.$x \in (-\infty,-16) \cup (-8,12)$\\
B.$x \in (-\infty,-16) \cup (-8,12]$\\
C.$x \in (-\infty,-16) \cup [-8,12)$\\
D.$x \in (-\infty,-16] \cup (-8,12)$\\
E.$x \in (-\infty,-16] \cup (-8,12]$\\
F.$x \in (-\infty,-16] \cup [-8,12)$\\
G.$x \in (-\infty,-16) \cup [-8,12]$\\
H.$x \in (-\infty,-16] \cup [-8,12]$
\testStop
\kluczStart
A
\kluczStop



\zadStart{Zadanie z Wikieł Z 1.62 b) moja wersja nr 519}

Rozwiązać nierówności $(x+16)(12-x)(x+9)\ge0$.
\zadStop
\rozwStart{Patryk Wirkus}{}
Miejsca zerowe naszego wielomianu to: $-16, 12, -9$.\\
Wielomian jest stopnia nieparzystego, ponadto znak współczynnika przy\linebreak najwyższej potędze x jest ujemny.\\ W związku z tym wykres wielomianu zaczyna się od lewej strony powyżej osi OX. A więc $$x \in (-\infty,-16) \cup (-9,12).$$
\rozwStop
\odpStart
$x \in (-\infty,-16) \cup (-9,12)$
\odpStop
\testStart
A.$x \in (-\infty,-16) \cup (-9,12)$\\
B.$x \in (-\infty,-16) \cup (-9,12]$\\
C.$x \in (-\infty,-16) \cup [-9,12)$\\
D.$x \in (-\infty,-16] \cup (-9,12)$\\
E.$x \in (-\infty,-16] \cup (-9,12]$\\
F.$x \in (-\infty,-16] \cup [-9,12)$\\
G.$x \in (-\infty,-16) \cup [-9,12]$\\
H.$x \in (-\infty,-16] \cup [-9,12]$
\testStop
\kluczStart
A
\kluczStop



\zadStart{Zadanie z Wikieł Z 1.62 b) moja wersja nr 520}

Rozwiązać nierówności $(x+16)(12-x)(x+10)\ge0$.
\zadStop
\rozwStart{Patryk Wirkus}{}
Miejsca zerowe naszego wielomianu to: $-16, 12, -10$.\\
Wielomian jest stopnia nieparzystego, ponadto znak współczynnika przy\linebreak najwyższej potędze x jest ujemny.\\ W związku z tym wykres wielomianu zaczyna się od lewej strony powyżej osi OX. A więc $$x \in (-\infty,-16) \cup (-10,12).$$
\rozwStop
\odpStart
$x \in (-\infty,-16) \cup (-10,12)$
\odpStop
\testStart
A.$x \in (-\infty,-16) \cup (-10,12)$\\
B.$x \in (-\infty,-16) \cup (-10,12]$\\
C.$x \in (-\infty,-16) \cup [-10,12)$\\
D.$x \in (-\infty,-16] \cup (-10,12)$\\
E.$x \in (-\infty,-16] \cup (-10,12]$\\
F.$x \in (-\infty,-16] \cup [-10,12)$\\
G.$x \in (-\infty,-16) \cup [-10,12]$\\
H.$x \in (-\infty,-16] \cup [-10,12]$
\testStop
\kluczStart
A
\kluczStop



\zadStart{Zadanie z Wikieł Z 1.62 b) moja wersja nr 521}

Rozwiązać nierówności $(x+16)(12-x)(x+11)\ge0$.
\zadStop
\rozwStart{Patryk Wirkus}{}
Miejsca zerowe naszego wielomianu to: $-16, 12, -11$.\\
Wielomian jest stopnia nieparzystego, ponadto znak współczynnika przy\linebreak najwyższej potędze x jest ujemny.\\ W związku z tym wykres wielomianu zaczyna się od lewej strony powyżej osi OX. A więc $$x \in (-\infty,-16) \cup (-11,12).$$
\rozwStop
\odpStart
$x \in (-\infty,-16) \cup (-11,12)$
\odpStop
\testStart
A.$x \in (-\infty,-16) \cup (-11,12)$\\
B.$x \in (-\infty,-16) \cup (-11,12]$\\
C.$x \in (-\infty,-16) \cup [-11,12)$\\
D.$x \in (-\infty,-16] \cup (-11,12)$\\
E.$x \in (-\infty,-16] \cup (-11,12]$\\
F.$x \in (-\infty,-16] \cup [-11,12)$\\
G.$x \in (-\infty,-16) \cup [-11,12]$\\
H.$x \in (-\infty,-16] \cup [-11,12]$
\testStop
\kluczStart
A
\kluczStop



\zadStart{Zadanie z Wikieł Z 1.62 b) moja wersja nr 522}

Rozwiązać nierówności $(x+16)(13-x)(x+1)\ge0$.
\zadStop
\rozwStart{Patryk Wirkus}{}
Miejsca zerowe naszego wielomianu to: $-16, 13, -1$.\\
Wielomian jest stopnia nieparzystego, ponadto znak współczynnika przy\linebreak najwyższej potędze x jest ujemny.\\ W związku z tym wykres wielomianu zaczyna się od lewej strony powyżej osi OX. A więc $$x \in (-\infty,-16) \cup (-1,13).$$
\rozwStop
\odpStart
$x \in (-\infty,-16) \cup (-1,13)$
\odpStop
\testStart
A.$x \in (-\infty,-16) \cup (-1,13)$\\
B.$x \in (-\infty,-16) \cup (-1,13]$\\
C.$x \in (-\infty,-16) \cup [-1,13)$\\
D.$x \in (-\infty,-16] \cup (-1,13)$\\
E.$x \in (-\infty,-16] \cup (-1,13]$\\
F.$x \in (-\infty,-16] \cup [-1,13)$\\
G.$x \in (-\infty,-16) \cup [-1,13]$\\
H.$x \in (-\infty,-16] \cup [-1,13]$
\testStop
\kluczStart
A
\kluczStop



\zadStart{Zadanie z Wikieł Z 1.62 b) moja wersja nr 523}

Rozwiązać nierówności $(x+16)(13-x)(x+2)\ge0$.
\zadStop
\rozwStart{Patryk Wirkus}{}
Miejsca zerowe naszego wielomianu to: $-16, 13, -2$.\\
Wielomian jest stopnia nieparzystego, ponadto znak współczynnika przy\linebreak najwyższej potędze x jest ujemny.\\ W związku z tym wykres wielomianu zaczyna się od lewej strony powyżej osi OX. A więc $$x \in (-\infty,-16) \cup (-2,13).$$
\rozwStop
\odpStart
$x \in (-\infty,-16) \cup (-2,13)$
\odpStop
\testStart
A.$x \in (-\infty,-16) \cup (-2,13)$\\
B.$x \in (-\infty,-16) \cup (-2,13]$\\
C.$x \in (-\infty,-16) \cup [-2,13)$\\
D.$x \in (-\infty,-16] \cup (-2,13)$\\
E.$x \in (-\infty,-16] \cup (-2,13]$\\
F.$x \in (-\infty,-16] \cup [-2,13)$\\
G.$x \in (-\infty,-16) \cup [-2,13]$\\
H.$x \in (-\infty,-16] \cup [-2,13]$
\testStop
\kluczStart
A
\kluczStop



\zadStart{Zadanie z Wikieł Z 1.62 b) moja wersja nr 524}

Rozwiązać nierówności $(x+16)(13-x)(x+3)\ge0$.
\zadStop
\rozwStart{Patryk Wirkus}{}
Miejsca zerowe naszego wielomianu to: $-16, 13, -3$.\\
Wielomian jest stopnia nieparzystego, ponadto znak współczynnika przy\linebreak najwyższej potędze x jest ujemny.\\ W związku z tym wykres wielomianu zaczyna się od lewej strony powyżej osi OX. A więc $$x \in (-\infty,-16) \cup (-3,13).$$
\rozwStop
\odpStart
$x \in (-\infty,-16) \cup (-3,13)$
\odpStop
\testStart
A.$x \in (-\infty,-16) \cup (-3,13)$\\
B.$x \in (-\infty,-16) \cup (-3,13]$\\
C.$x \in (-\infty,-16) \cup [-3,13)$\\
D.$x \in (-\infty,-16] \cup (-3,13)$\\
E.$x \in (-\infty,-16] \cup (-3,13]$\\
F.$x \in (-\infty,-16] \cup [-3,13)$\\
G.$x \in (-\infty,-16) \cup [-3,13]$\\
H.$x \in (-\infty,-16] \cup [-3,13]$
\testStop
\kluczStart
A
\kluczStop



\zadStart{Zadanie z Wikieł Z 1.62 b) moja wersja nr 525}

Rozwiązać nierówności $(x+16)(13-x)(x+4)\ge0$.
\zadStop
\rozwStart{Patryk Wirkus}{}
Miejsca zerowe naszego wielomianu to: $-16, 13, -4$.\\
Wielomian jest stopnia nieparzystego, ponadto znak współczynnika przy\linebreak najwyższej potędze x jest ujemny.\\ W związku z tym wykres wielomianu zaczyna się od lewej strony powyżej osi OX. A więc $$x \in (-\infty,-16) \cup (-4,13).$$
\rozwStop
\odpStart
$x \in (-\infty,-16) \cup (-4,13)$
\odpStop
\testStart
A.$x \in (-\infty,-16) \cup (-4,13)$\\
B.$x \in (-\infty,-16) \cup (-4,13]$\\
C.$x \in (-\infty,-16) \cup [-4,13)$\\
D.$x \in (-\infty,-16] \cup (-4,13)$\\
E.$x \in (-\infty,-16] \cup (-4,13]$\\
F.$x \in (-\infty,-16] \cup [-4,13)$\\
G.$x \in (-\infty,-16) \cup [-4,13]$\\
H.$x \in (-\infty,-16] \cup [-4,13]$
\testStop
\kluczStart
A
\kluczStop



\zadStart{Zadanie z Wikieł Z 1.62 b) moja wersja nr 526}

Rozwiązać nierówności $(x+16)(13-x)(x+5)\ge0$.
\zadStop
\rozwStart{Patryk Wirkus}{}
Miejsca zerowe naszego wielomianu to: $-16, 13, -5$.\\
Wielomian jest stopnia nieparzystego, ponadto znak współczynnika przy\linebreak najwyższej potędze x jest ujemny.\\ W związku z tym wykres wielomianu zaczyna się od lewej strony powyżej osi OX. A więc $$x \in (-\infty,-16) \cup (-5,13).$$
\rozwStop
\odpStart
$x \in (-\infty,-16) \cup (-5,13)$
\odpStop
\testStart
A.$x \in (-\infty,-16) \cup (-5,13)$\\
B.$x \in (-\infty,-16) \cup (-5,13]$\\
C.$x \in (-\infty,-16) \cup [-5,13)$\\
D.$x \in (-\infty,-16] \cup (-5,13)$\\
E.$x \in (-\infty,-16] \cup (-5,13]$\\
F.$x \in (-\infty,-16] \cup [-5,13)$\\
G.$x \in (-\infty,-16) \cup [-5,13]$\\
H.$x \in (-\infty,-16] \cup [-5,13]$
\testStop
\kluczStart
A
\kluczStop



\zadStart{Zadanie z Wikieł Z 1.62 b) moja wersja nr 527}

Rozwiązać nierówności $(x+16)(13-x)(x+6)\ge0$.
\zadStop
\rozwStart{Patryk Wirkus}{}
Miejsca zerowe naszego wielomianu to: $-16, 13, -6$.\\
Wielomian jest stopnia nieparzystego, ponadto znak współczynnika przy\linebreak najwyższej potędze x jest ujemny.\\ W związku z tym wykres wielomianu zaczyna się od lewej strony powyżej osi OX. A więc $$x \in (-\infty,-16) \cup (-6,13).$$
\rozwStop
\odpStart
$x \in (-\infty,-16) \cup (-6,13)$
\odpStop
\testStart
A.$x \in (-\infty,-16) \cup (-6,13)$\\
B.$x \in (-\infty,-16) \cup (-6,13]$\\
C.$x \in (-\infty,-16) \cup [-6,13)$\\
D.$x \in (-\infty,-16] \cup (-6,13)$\\
E.$x \in (-\infty,-16] \cup (-6,13]$\\
F.$x \in (-\infty,-16] \cup [-6,13)$\\
G.$x \in (-\infty,-16) \cup [-6,13]$\\
H.$x \in (-\infty,-16] \cup [-6,13]$
\testStop
\kluczStart
A
\kluczStop



\zadStart{Zadanie z Wikieł Z 1.62 b) moja wersja nr 528}

Rozwiązać nierówności $(x+16)(13-x)(x+7)\ge0$.
\zadStop
\rozwStart{Patryk Wirkus}{}
Miejsca zerowe naszego wielomianu to: $-16, 13, -7$.\\
Wielomian jest stopnia nieparzystego, ponadto znak współczynnika przy\linebreak najwyższej potędze x jest ujemny.\\ W związku z tym wykres wielomianu zaczyna się od lewej strony powyżej osi OX. A więc $$x \in (-\infty,-16) \cup (-7,13).$$
\rozwStop
\odpStart
$x \in (-\infty,-16) \cup (-7,13)$
\odpStop
\testStart
A.$x \in (-\infty,-16) \cup (-7,13)$\\
B.$x \in (-\infty,-16) \cup (-7,13]$\\
C.$x \in (-\infty,-16) \cup [-7,13)$\\
D.$x \in (-\infty,-16] \cup (-7,13)$\\
E.$x \in (-\infty,-16] \cup (-7,13]$\\
F.$x \in (-\infty,-16] \cup [-7,13)$\\
G.$x \in (-\infty,-16) \cup [-7,13]$\\
H.$x \in (-\infty,-16] \cup [-7,13]$
\testStop
\kluczStart
A
\kluczStop



\zadStart{Zadanie z Wikieł Z 1.62 b) moja wersja nr 529}

Rozwiązać nierówności $(x+16)(13-x)(x+8)\ge0$.
\zadStop
\rozwStart{Patryk Wirkus}{}
Miejsca zerowe naszego wielomianu to: $-16, 13, -8$.\\
Wielomian jest stopnia nieparzystego, ponadto znak współczynnika przy\linebreak najwyższej potędze x jest ujemny.\\ W związku z tym wykres wielomianu zaczyna się od lewej strony powyżej osi OX. A więc $$x \in (-\infty,-16) \cup (-8,13).$$
\rozwStop
\odpStart
$x \in (-\infty,-16) \cup (-8,13)$
\odpStop
\testStart
A.$x \in (-\infty,-16) \cup (-8,13)$\\
B.$x \in (-\infty,-16) \cup (-8,13]$\\
C.$x \in (-\infty,-16) \cup [-8,13)$\\
D.$x \in (-\infty,-16] \cup (-8,13)$\\
E.$x \in (-\infty,-16] \cup (-8,13]$\\
F.$x \in (-\infty,-16] \cup [-8,13)$\\
G.$x \in (-\infty,-16) \cup [-8,13]$\\
H.$x \in (-\infty,-16] \cup [-8,13]$
\testStop
\kluczStart
A
\kluczStop



\zadStart{Zadanie z Wikieł Z 1.62 b) moja wersja nr 530}

Rozwiązać nierówności $(x+16)(13-x)(x+9)\ge0$.
\zadStop
\rozwStart{Patryk Wirkus}{}
Miejsca zerowe naszego wielomianu to: $-16, 13, -9$.\\
Wielomian jest stopnia nieparzystego, ponadto znak współczynnika przy\linebreak najwyższej potędze x jest ujemny.\\ W związku z tym wykres wielomianu zaczyna się od lewej strony powyżej osi OX. A więc $$x \in (-\infty,-16) \cup (-9,13).$$
\rozwStop
\odpStart
$x \in (-\infty,-16) \cup (-9,13)$
\odpStop
\testStart
A.$x \in (-\infty,-16) \cup (-9,13)$\\
B.$x \in (-\infty,-16) \cup (-9,13]$\\
C.$x \in (-\infty,-16) \cup [-9,13)$\\
D.$x \in (-\infty,-16] \cup (-9,13)$\\
E.$x \in (-\infty,-16] \cup (-9,13]$\\
F.$x \in (-\infty,-16] \cup [-9,13)$\\
G.$x \in (-\infty,-16) \cup [-9,13]$\\
H.$x \in (-\infty,-16] \cup [-9,13]$
\testStop
\kluczStart
A
\kluczStop



\zadStart{Zadanie z Wikieł Z 1.62 b) moja wersja nr 531}

Rozwiązać nierówności $(x+16)(13-x)(x+10)\ge0$.
\zadStop
\rozwStart{Patryk Wirkus}{}
Miejsca zerowe naszego wielomianu to: $-16, 13, -10$.\\
Wielomian jest stopnia nieparzystego, ponadto znak współczynnika przy\linebreak najwyższej potędze x jest ujemny.\\ W związku z tym wykres wielomianu zaczyna się od lewej strony powyżej osi OX. A więc $$x \in (-\infty,-16) \cup (-10,13).$$
\rozwStop
\odpStart
$x \in (-\infty,-16) \cup (-10,13)$
\odpStop
\testStart
A.$x \in (-\infty,-16) \cup (-10,13)$\\
B.$x \in (-\infty,-16) \cup (-10,13]$\\
C.$x \in (-\infty,-16) \cup [-10,13)$\\
D.$x \in (-\infty,-16] \cup (-10,13)$\\
E.$x \in (-\infty,-16] \cup (-10,13]$\\
F.$x \in (-\infty,-16] \cup [-10,13)$\\
G.$x \in (-\infty,-16) \cup [-10,13]$\\
H.$x \in (-\infty,-16] \cup [-10,13]$
\testStop
\kluczStart
A
\kluczStop



\zadStart{Zadanie z Wikieł Z 1.62 b) moja wersja nr 532}

Rozwiązać nierówności $(x+16)(13-x)(x+11)\ge0$.
\zadStop
\rozwStart{Patryk Wirkus}{}
Miejsca zerowe naszego wielomianu to: $-16, 13, -11$.\\
Wielomian jest stopnia nieparzystego, ponadto znak współczynnika przy\linebreak najwyższej potędze x jest ujemny.\\ W związku z tym wykres wielomianu zaczyna się od lewej strony powyżej osi OX. A więc $$x \in (-\infty,-16) \cup (-11,13).$$
\rozwStop
\odpStart
$x \in (-\infty,-16) \cup (-11,13)$
\odpStop
\testStart
A.$x \in (-\infty,-16) \cup (-11,13)$\\
B.$x \in (-\infty,-16) \cup (-11,13]$\\
C.$x \in (-\infty,-16) \cup [-11,13)$\\
D.$x \in (-\infty,-16] \cup (-11,13)$\\
E.$x \in (-\infty,-16] \cup (-11,13]$\\
F.$x \in (-\infty,-16] \cup [-11,13)$\\
G.$x \in (-\infty,-16) \cup [-11,13]$\\
H.$x \in (-\infty,-16] \cup [-11,13]$
\testStop
\kluczStart
A
\kluczStop



\zadStart{Zadanie z Wikieł Z 1.62 b) moja wersja nr 533}

Rozwiązać nierówności $(x+16)(13-x)(x+12)\ge0$.
\zadStop
\rozwStart{Patryk Wirkus}{}
Miejsca zerowe naszego wielomianu to: $-16, 13, -12$.\\
Wielomian jest stopnia nieparzystego, ponadto znak współczynnika przy\linebreak najwyższej potędze x jest ujemny.\\ W związku z tym wykres wielomianu zaczyna się od lewej strony powyżej osi OX. A więc $$x \in (-\infty,-16) \cup (-12,13).$$
\rozwStop
\odpStart
$x \in (-\infty,-16) \cup (-12,13)$
\odpStop
\testStart
A.$x \in (-\infty,-16) \cup (-12,13)$\\
B.$x \in (-\infty,-16) \cup (-12,13]$\\
C.$x \in (-\infty,-16) \cup [-12,13)$\\
D.$x \in (-\infty,-16] \cup (-12,13)$\\
E.$x \in (-\infty,-16] \cup (-12,13]$\\
F.$x \in (-\infty,-16] \cup [-12,13)$\\
G.$x \in (-\infty,-16) \cup [-12,13]$\\
H.$x \in (-\infty,-16] \cup [-12,13]$
\testStop
\kluczStart
A
\kluczStop



\zadStart{Zadanie z Wikieł Z 1.62 b) moja wersja nr 534}

Rozwiązać nierówności $(x+16)(14-x)(x+1)\ge0$.
\zadStop
\rozwStart{Patryk Wirkus}{}
Miejsca zerowe naszego wielomianu to: $-16, 14, -1$.\\
Wielomian jest stopnia nieparzystego, ponadto znak współczynnika przy\linebreak najwyższej potędze x jest ujemny.\\ W związku z tym wykres wielomianu zaczyna się od lewej strony powyżej osi OX. A więc $$x \in (-\infty,-16) \cup (-1,14).$$
\rozwStop
\odpStart
$x \in (-\infty,-16) \cup (-1,14)$
\odpStop
\testStart
A.$x \in (-\infty,-16) \cup (-1,14)$\\
B.$x \in (-\infty,-16) \cup (-1,14]$\\
C.$x \in (-\infty,-16) \cup [-1,14)$\\
D.$x \in (-\infty,-16] \cup (-1,14)$\\
E.$x \in (-\infty,-16] \cup (-1,14]$\\
F.$x \in (-\infty,-16] \cup [-1,14)$\\
G.$x \in (-\infty,-16) \cup [-1,14]$\\
H.$x \in (-\infty,-16] \cup [-1,14]$
\testStop
\kluczStart
A
\kluczStop



\zadStart{Zadanie z Wikieł Z 1.62 b) moja wersja nr 535}

Rozwiązać nierówności $(x+16)(14-x)(x+2)\ge0$.
\zadStop
\rozwStart{Patryk Wirkus}{}
Miejsca zerowe naszego wielomianu to: $-16, 14, -2$.\\
Wielomian jest stopnia nieparzystego, ponadto znak współczynnika przy\linebreak najwyższej potędze x jest ujemny.\\ W związku z tym wykres wielomianu zaczyna się od lewej strony powyżej osi OX. A więc $$x \in (-\infty,-16) \cup (-2,14).$$
\rozwStop
\odpStart
$x \in (-\infty,-16) \cup (-2,14)$
\odpStop
\testStart
A.$x \in (-\infty,-16) \cup (-2,14)$\\
B.$x \in (-\infty,-16) \cup (-2,14]$\\
C.$x \in (-\infty,-16) \cup [-2,14)$\\
D.$x \in (-\infty,-16] \cup (-2,14)$\\
E.$x \in (-\infty,-16] \cup (-2,14]$\\
F.$x \in (-\infty,-16] \cup [-2,14)$\\
G.$x \in (-\infty,-16) \cup [-2,14]$\\
H.$x \in (-\infty,-16] \cup [-2,14]$
\testStop
\kluczStart
A
\kluczStop



\zadStart{Zadanie z Wikieł Z 1.62 b) moja wersja nr 536}

Rozwiązać nierówności $(x+16)(14-x)(x+3)\ge0$.
\zadStop
\rozwStart{Patryk Wirkus}{}
Miejsca zerowe naszego wielomianu to: $-16, 14, -3$.\\
Wielomian jest stopnia nieparzystego, ponadto znak współczynnika przy\linebreak najwyższej potędze x jest ujemny.\\ W związku z tym wykres wielomianu zaczyna się od lewej strony powyżej osi OX. A więc $$x \in (-\infty,-16) \cup (-3,14).$$
\rozwStop
\odpStart
$x \in (-\infty,-16) \cup (-3,14)$
\odpStop
\testStart
A.$x \in (-\infty,-16) \cup (-3,14)$\\
B.$x \in (-\infty,-16) \cup (-3,14]$\\
C.$x \in (-\infty,-16) \cup [-3,14)$\\
D.$x \in (-\infty,-16] \cup (-3,14)$\\
E.$x \in (-\infty,-16] \cup (-3,14]$\\
F.$x \in (-\infty,-16] \cup [-3,14)$\\
G.$x \in (-\infty,-16) \cup [-3,14]$\\
H.$x \in (-\infty,-16] \cup [-3,14]$
\testStop
\kluczStart
A
\kluczStop



\zadStart{Zadanie z Wikieł Z 1.62 b) moja wersja nr 537}

Rozwiązać nierówności $(x+16)(14-x)(x+4)\ge0$.
\zadStop
\rozwStart{Patryk Wirkus}{}
Miejsca zerowe naszego wielomianu to: $-16, 14, -4$.\\
Wielomian jest stopnia nieparzystego, ponadto znak współczynnika przy\linebreak najwyższej potędze x jest ujemny.\\ W związku z tym wykres wielomianu zaczyna się od lewej strony powyżej osi OX. A więc $$x \in (-\infty,-16) \cup (-4,14).$$
\rozwStop
\odpStart
$x \in (-\infty,-16) \cup (-4,14)$
\odpStop
\testStart
A.$x \in (-\infty,-16) \cup (-4,14)$\\
B.$x \in (-\infty,-16) \cup (-4,14]$\\
C.$x \in (-\infty,-16) \cup [-4,14)$\\
D.$x \in (-\infty,-16] \cup (-4,14)$\\
E.$x \in (-\infty,-16] \cup (-4,14]$\\
F.$x \in (-\infty,-16] \cup [-4,14)$\\
G.$x \in (-\infty,-16) \cup [-4,14]$\\
H.$x \in (-\infty,-16] \cup [-4,14]$
\testStop
\kluczStart
A
\kluczStop



\zadStart{Zadanie z Wikieł Z 1.62 b) moja wersja nr 538}

Rozwiązać nierówności $(x+16)(14-x)(x+5)\ge0$.
\zadStop
\rozwStart{Patryk Wirkus}{}
Miejsca zerowe naszego wielomianu to: $-16, 14, -5$.\\
Wielomian jest stopnia nieparzystego, ponadto znak współczynnika przy\linebreak najwyższej potędze x jest ujemny.\\ W związku z tym wykres wielomianu zaczyna się od lewej strony powyżej osi OX. A więc $$x \in (-\infty,-16) \cup (-5,14).$$
\rozwStop
\odpStart
$x \in (-\infty,-16) \cup (-5,14)$
\odpStop
\testStart
A.$x \in (-\infty,-16) \cup (-5,14)$\\
B.$x \in (-\infty,-16) \cup (-5,14]$\\
C.$x \in (-\infty,-16) \cup [-5,14)$\\
D.$x \in (-\infty,-16] \cup (-5,14)$\\
E.$x \in (-\infty,-16] \cup (-5,14]$\\
F.$x \in (-\infty,-16] \cup [-5,14)$\\
G.$x \in (-\infty,-16) \cup [-5,14]$\\
H.$x \in (-\infty,-16] \cup [-5,14]$
\testStop
\kluczStart
A
\kluczStop



\zadStart{Zadanie z Wikieł Z 1.62 b) moja wersja nr 539}

Rozwiązać nierówności $(x+16)(14-x)(x+6)\ge0$.
\zadStop
\rozwStart{Patryk Wirkus}{}
Miejsca zerowe naszego wielomianu to: $-16, 14, -6$.\\
Wielomian jest stopnia nieparzystego, ponadto znak współczynnika przy\linebreak najwyższej potędze x jest ujemny.\\ W związku z tym wykres wielomianu zaczyna się od lewej strony powyżej osi OX. A więc $$x \in (-\infty,-16) \cup (-6,14).$$
\rozwStop
\odpStart
$x \in (-\infty,-16) \cup (-6,14)$
\odpStop
\testStart
A.$x \in (-\infty,-16) \cup (-6,14)$\\
B.$x \in (-\infty,-16) \cup (-6,14]$\\
C.$x \in (-\infty,-16) \cup [-6,14)$\\
D.$x \in (-\infty,-16] \cup (-6,14)$\\
E.$x \in (-\infty,-16] \cup (-6,14]$\\
F.$x \in (-\infty,-16] \cup [-6,14)$\\
G.$x \in (-\infty,-16) \cup [-6,14]$\\
H.$x \in (-\infty,-16] \cup [-6,14]$
\testStop
\kluczStart
A
\kluczStop



\zadStart{Zadanie z Wikieł Z 1.62 b) moja wersja nr 540}

Rozwiązać nierówności $(x+16)(14-x)(x+7)\ge0$.
\zadStop
\rozwStart{Patryk Wirkus}{}
Miejsca zerowe naszego wielomianu to: $-16, 14, -7$.\\
Wielomian jest stopnia nieparzystego, ponadto znak współczynnika przy\linebreak najwyższej potędze x jest ujemny.\\ W związku z tym wykres wielomianu zaczyna się od lewej strony powyżej osi OX. A więc $$x \in (-\infty,-16) \cup (-7,14).$$
\rozwStop
\odpStart
$x \in (-\infty,-16) \cup (-7,14)$
\odpStop
\testStart
A.$x \in (-\infty,-16) \cup (-7,14)$\\
B.$x \in (-\infty,-16) \cup (-7,14]$\\
C.$x \in (-\infty,-16) \cup [-7,14)$\\
D.$x \in (-\infty,-16] \cup (-7,14)$\\
E.$x \in (-\infty,-16] \cup (-7,14]$\\
F.$x \in (-\infty,-16] \cup [-7,14)$\\
G.$x \in (-\infty,-16) \cup [-7,14]$\\
H.$x \in (-\infty,-16] \cup [-7,14]$
\testStop
\kluczStart
A
\kluczStop



\zadStart{Zadanie z Wikieł Z 1.62 b) moja wersja nr 541}

Rozwiązać nierówności $(x+16)(14-x)(x+8)\ge0$.
\zadStop
\rozwStart{Patryk Wirkus}{}
Miejsca zerowe naszego wielomianu to: $-16, 14, -8$.\\
Wielomian jest stopnia nieparzystego, ponadto znak współczynnika przy\linebreak najwyższej potędze x jest ujemny.\\ W związku z tym wykres wielomianu zaczyna się od lewej strony powyżej osi OX. A więc $$x \in (-\infty,-16) \cup (-8,14).$$
\rozwStop
\odpStart
$x \in (-\infty,-16) \cup (-8,14)$
\odpStop
\testStart
A.$x \in (-\infty,-16) \cup (-8,14)$\\
B.$x \in (-\infty,-16) \cup (-8,14]$\\
C.$x \in (-\infty,-16) \cup [-8,14)$\\
D.$x \in (-\infty,-16] \cup (-8,14)$\\
E.$x \in (-\infty,-16] \cup (-8,14]$\\
F.$x \in (-\infty,-16] \cup [-8,14)$\\
G.$x \in (-\infty,-16) \cup [-8,14]$\\
H.$x \in (-\infty,-16] \cup [-8,14]$
\testStop
\kluczStart
A
\kluczStop



\zadStart{Zadanie z Wikieł Z 1.62 b) moja wersja nr 542}

Rozwiązać nierówności $(x+16)(14-x)(x+9)\ge0$.
\zadStop
\rozwStart{Patryk Wirkus}{}
Miejsca zerowe naszego wielomianu to: $-16, 14, -9$.\\
Wielomian jest stopnia nieparzystego, ponadto znak współczynnika przy\linebreak najwyższej potędze x jest ujemny.\\ W związku z tym wykres wielomianu zaczyna się od lewej strony powyżej osi OX. A więc $$x \in (-\infty,-16) \cup (-9,14).$$
\rozwStop
\odpStart
$x \in (-\infty,-16) \cup (-9,14)$
\odpStop
\testStart
A.$x \in (-\infty,-16) \cup (-9,14)$\\
B.$x \in (-\infty,-16) \cup (-9,14]$\\
C.$x \in (-\infty,-16) \cup [-9,14)$\\
D.$x \in (-\infty,-16] \cup (-9,14)$\\
E.$x \in (-\infty,-16] \cup (-9,14]$\\
F.$x \in (-\infty,-16] \cup [-9,14)$\\
G.$x \in (-\infty,-16) \cup [-9,14]$\\
H.$x \in (-\infty,-16] \cup [-9,14]$
\testStop
\kluczStart
A
\kluczStop



\zadStart{Zadanie z Wikieł Z 1.62 b) moja wersja nr 543}

Rozwiązać nierówności $(x+16)(14-x)(x+10)\ge0$.
\zadStop
\rozwStart{Patryk Wirkus}{}
Miejsca zerowe naszego wielomianu to: $-16, 14, -10$.\\
Wielomian jest stopnia nieparzystego, ponadto znak współczynnika przy\linebreak najwyższej potędze x jest ujemny.\\ W związku z tym wykres wielomianu zaczyna się od lewej strony powyżej osi OX. A więc $$x \in (-\infty,-16) \cup (-10,14).$$
\rozwStop
\odpStart
$x \in (-\infty,-16) \cup (-10,14)$
\odpStop
\testStart
A.$x \in (-\infty,-16) \cup (-10,14)$\\
B.$x \in (-\infty,-16) \cup (-10,14]$\\
C.$x \in (-\infty,-16) \cup [-10,14)$\\
D.$x \in (-\infty,-16] \cup (-10,14)$\\
E.$x \in (-\infty,-16] \cup (-10,14]$\\
F.$x \in (-\infty,-16] \cup [-10,14)$\\
G.$x \in (-\infty,-16) \cup [-10,14]$\\
H.$x \in (-\infty,-16] \cup [-10,14]$
\testStop
\kluczStart
A
\kluczStop



\zadStart{Zadanie z Wikieł Z 1.62 b) moja wersja nr 544}

Rozwiązać nierówności $(x+16)(14-x)(x+11)\ge0$.
\zadStop
\rozwStart{Patryk Wirkus}{}
Miejsca zerowe naszego wielomianu to: $-16, 14, -11$.\\
Wielomian jest stopnia nieparzystego, ponadto znak współczynnika przy\linebreak najwyższej potędze x jest ujemny.\\ W związku z tym wykres wielomianu zaczyna się od lewej strony powyżej osi OX. A więc $$x \in (-\infty,-16) \cup (-11,14).$$
\rozwStop
\odpStart
$x \in (-\infty,-16) \cup (-11,14)$
\odpStop
\testStart
A.$x \in (-\infty,-16) \cup (-11,14)$\\
B.$x \in (-\infty,-16) \cup (-11,14]$\\
C.$x \in (-\infty,-16) \cup [-11,14)$\\
D.$x \in (-\infty,-16] \cup (-11,14)$\\
E.$x \in (-\infty,-16] \cup (-11,14]$\\
F.$x \in (-\infty,-16] \cup [-11,14)$\\
G.$x \in (-\infty,-16) \cup [-11,14]$\\
H.$x \in (-\infty,-16] \cup [-11,14]$
\testStop
\kluczStart
A
\kluczStop



\zadStart{Zadanie z Wikieł Z 1.62 b) moja wersja nr 545}

Rozwiązać nierówności $(x+16)(14-x)(x+12)\ge0$.
\zadStop
\rozwStart{Patryk Wirkus}{}
Miejsca zerowe naszego wielomianu to: $-16, 14, -12$.\\
Wielomian jest stopnia nieparzystego, ponadto znak współczynnika przy\linebreak najwyższej potędze x jest ujemny.\\ W związku z tym wykres wielomianu zaczyna się od lewej strony powyżej osi OX. A więc $$x \in (-\infty,-16) \cup (-12,14).$$
\rozwStop
\odpStart
$x \in (-\infty,-16) \cup (-12,14)$
\odpStop
\testStart
A.$x \in (-\infty,-16) \cup (-12,14)$\\
B.$x \in (-\infty,-16) \cup (-12,14]$\\
C.$x \in (-\infty,-16) \cup [-12,14)$\\
D.$x \in (-\infty,-16] \cup (-12,14)$\\
E.$x \in (-\infty,-16] \cup (-12,14]$\\
F.$x \in (-\infty,-16] \cup [-12,14)$\\
G.$x \in (-\infty,-16) \cup [-12,14]$\\
H.$x \in (-\infty,-16] \cup [-12,14]$
\testStop
\kluczStart
A
\kluczStop



\zadStart{Zadanie z Wikieł Z 1.62 b) moja wersja nr 546}

Rozwiązać nierówności $(x+16)(14-x)(x+13)\ge0$.
\zadStop
\rozwStart{Patryk Wirkus}{}
Miejsca zerowe naszego wielomianu to: $-16, 14, -13$.\\
Wielomian jest stopnia nieparzystego, ponadto znak współczynnika przy\linebreak najwyższej potędze x jest ujemny.\\ W związku z tym wykres wielomianu zaczyna się od lewej strony powyżej osi OX. A więc $$x \in (-\infty,-16) \cup (-13,14).$$
\rozwStop
\odpStart
$x \in (-\infty,-16) \cup (-13,14)$
\odpStop
\testStart
A.$x \in (-\infty,-16) \cup (-13,14)$\\
B.$x \in (-\infty,-16) \cup (-13,14]$\\
C.$x \in (-\infty,-16) \cup [-13,14)$\\
D.$x \in (-\infty,-16] \cup (-13,14)$\\
E.$x \in (-\infty,-16] \cup (-13,14]$\\
F.$x \in (-\infty,-16] \cup [-13,14)$\\
G.$x \in (-\infty,-16) \cup [-13,14]$\\
H.$x \in (-\infty,-16] \cup [-13,14]$
\testStop
\kluczStart
A
\kluczStop



\zadStart{Zadanie z Wikieł Z 1.62 b) moja wersja nr 547}

Rozwiązać nierówności $(x+16)(15-x)(x+1)\ge0$.
\zadStop
\rozwStart{Patryk Wirkus}{}
Miejsca zerowe naszego wielomianu to: $-16, 15, -1$.\\
Wielomian jest stopnia nieparzystego, ponadto znak współczynnika przy\linebreak najwyższej potędze x jest ujemny.\\ W związku z tym wykres wielomianu zaczyna się od lewej strony powyżej osi OX. A więc $$x \in (-\infty,-16) \cup (-1,15).$$
\rozwStop
\odpStart
$x \in (-\infty,-16) \cup (-1,15)$
\odpStop
\testStart
A.$x \in (-\infty,-16) \cup (-1,15)$\\
B.$x \in (-\infty,-16) \cup (-1,15]$\\
C.$x \in (-\infty,-16) \cup [-1,15)$\\
D.$x \in (-\infty,-16] \cup (-1,15)$\\
E.$x \in (-\infty,-16] \cup (-1,15]$\\
F.$x \in (-\infty,-16] \cup [-1,15)$\\
G.$x \in (-\infty,-16) \cup [-1,15]$\\
H.$x \in (-\infty,-16] \cup [-1,15]$
\testStop
\kluczStart
A
\kluczStop



\zadStart{Zadanie z Wikieł Z 1.62 b) moja wersja nr 548}

Rozwiązać nierówności $(x+16)(15-x)(x+2)\ge0$.
\zadStop
\rozwStart{Patryk Wirkus}{}
Miejsca zerowe naszego wielomianu to: $-16, 15, -2$.\\
Wielomian jest stopnia nieparzystego, ponadto znak współczynnika przy\linebreak najwyższej potędze x jest ujemny.\\ W związku z tym wykres wielomianu zaczyna się od lewej strony powyżej osi OX. A więc $$x \in (-\infty,-16) \cup (-2,15).$$
\rozwStop
\odpStart
$x \in (-\infty,-16) \cup (-2,15)$
\odpStop
\testStart
A.$x \in (-\infty,-16) \cup (-2,15)$\\
B.$x \in (-\infty,-16) \cup (-2,15]$\\
C.$x \in (-\infty,-16) \cup [-2,15)$\\
D.$x \in (-\infty,-16] \cup (-2,15)$\\
E.$x \in (-\infty,-16] \cup (-2,15]$\\
F.$x \in (-\infty,-16] \cup [-2,15)$\\
G.$x \in (-\infty,-16) \cup [-2,15]$\\
H.$x \in (-\infty,-16] \cup [-2,15]$
\testStop
\kluczStart
A
\kluczStop



\zadStart{Zadanie z Wikieł Z 1.62 b) moja wersja nr 549}

Rozwiązać nierówności $(x+16)(15-x)(x+3)\ge0$.
\zadStop
\rozwStart{Patryk Wirkus}{}
Miejsca zerowe naszego wielomianu to: $-16, 15, -3$.\\
Wielomian jest stopnia nieparzystego, ponadto znak współczynnika przy\linebreak najwyższej potędze x jest ujemny.\\ W związku z tym wykres wielomianu zaczyna się od lewej strony powyżej osi OX. A więc $$x \in (-\infty,-16) \cup (-3,15).$$
\rozwStop
\odpStart
$x \in (-\infty,-16) \cup (-3,15)$
\odpStop
\testStart
A.$x \in (-\infty,-16) \cup (-3,15)$\\
B.$x \in (-\infty,-16) \cup (-3,15]$\\
C.$x \in (-\infty,-16) \cup [-3,15)$\\
D.$x \in (-\infty,-16] \cup (-3,15)$\\
E.$x \in (-\infty,-16] \cup (-3,15]$\\
F.$x \in (-\infty,-16] \cup [-3,15)$\\
G.$x \in (-\infty,-16) \cup [-3,15]$\\
H.$x \in (-\infty,-16] \cup [-3,15]$
\testStop
\kluczStart
A
\kluczStop



\zadStart{Zadanie z Wikieł Z 1.62 b) moja wersja nr 550}

Rozwiązać nierówności $(x+16)(15-x)(x+4)\ge0$.
\zadStop
\rozwStart{Patryk Wirkus}{}
Miejsca zerowe naszego wielomianu to: $-16, 15, -4$.\\
Wielomian jest stopnia nieparzystego, ponadto znak współczynnika przy\linebreak najwyższej potędze x jest ujemny.\\ W związku z tym wykres wielomianu zaczyna się od lewej strony powyżej osi OX. A więc $$x \in (-\infty,-16) \cup (-4,15).$$
\rozwStop
\odpStart
$x \in (-\infty,-16) \cup (-4,15)$
\odpStop
\testStart
A.$x \in (-\infty,-16) \cup (-4,15)$\\
B.$x \in (-\infty,-16) \cup (-4,15]$\\
C.$x \in (-\infty,-16) \cup [-4,15)$\\
D.$x \in (-\infty,-16] \cup (-4,15)$\\
E.$x \in (-\infty,-16] \cup (-4,15]$\\
F.$x \in (-\infty,-16] \cup [-4,15)$\\
G.$x \in (-\infty,-16) \cup [-4,15]$\\
H.$x \in (-\infty,-16] \cup [-4,15]$
\testStop
\kluczStart
A
\kluczStop



\zadStart{Zadanie z Wikieł Z 1.62 b) moja wersja nr 551}

Rozwiązać nierówności $(x+16)(15-x)(x+5)\ge0$.
\zadStop
\rozwStart{Patryk Wirkus}{}
Miejsca zerowe naszego wielomianu to: $-16, 15, -5$.\\
Wielomian jest stopnia nieparzystego, ponadto znak współczynnika przy\linebreak najwyższej potędze x jest ujemny.\\ W związku z tym wykres wielomianu zaczyna się od lewej strony powyżej osi OX. A więc $$x \in (-\infty,-16) \cup (-5,15).$$
\rozwStop
\odpStart
$x \in (-\infty,-16) \cup (-5,15)$
\odpStop
\testStart
A.$x \in (-\infty,-16) \cup (-5,15)$\\
B.$x \in (-\infty,-16) \cup (-5,15]$\\
C.$x \in (-\infty,-16) \cup [-5,15)$\\
D.$x \in (-\infty,-16] \cup (-5,15)$\\
E.$x \in (-\infty,-16] \cup (-5,15]$\\
F.$x \in (-\infty,-16] \cup [-5,15)$\\
G.$x \in (-\infty,-16) \cup [-5,15]$\\
H.$x \in (-\infty,-16] \cup [-5,15]$
\testStop
\kluczStart
A
\kluczStop



\zadStart{Zadanie z Wikieł Z 1.62 b) moja wersja nr 552}

Rozwiązać nierówności $(x+16)(15-x)(x+6)\ge0$.
\zadStop
\rozwStart{Patryk Wirkus}{}
Miejsca zerowe naszego wielomianu to: $-16, 15, -6$.\\
Wielomian jest stopnia nieparzystego, ponadto znak współczynnika przy\linebreak najwyższej potędze x jest ujemny.\\ W związku z tym wykres wielomianu zaczyna się od lewej strony powyżej osi OX. A więc $$x \in (-\infty,-16) \cup (-6,15).$$
\rozwStop
\odpStart
$x \in (-\infty,-16) \cup (-6,15)$
\odpStop
\testStart
A.$x \in (-\infty,-16) \cup (-6,15)$\\
B.$x \in (-\infty,-16) \cup (-6,15]$\\
C.$x \in (-\infty,-16) \cup [-6,15)$\\
D.$x \in (-\infty,-16] \cup (-6,15)$\\
E.$x \in (-\infty,-16] \cup (-6,15]$\\
F.$x \in (-\infty,-16] \cup [-6,15)$\\
G.$x \in (-\infty,-16) \cup [-6,15]$\\
H.$x \in (-\infty,-16] \cup [-6,15]$
\testStop
\kluczStart
A
\kluczStop



\zadStart{Zadanie z Wikieł Z 1.62 b) moja wersja nr 553}

Rozwiązać nierówności $(x+16)(15-x)(x+7)\ge0$.
\zadStop
\rozwStart{Patryk Wirkus}{}
Miejsca zerowe naszego wielomianu to: $-16, 15, -7$.\\
Wielomian jest stopnia nieparzystego, ponadto znak współczynnika przy\linebreak najwyższej potędze x jest ujemny.\\ W związku z tym wykres wielomianu zaczyna się od lewej strony powyżej osi OX. A więc $$x \in (-\infty,-16) \cup (-7,15).$$
\rozwStop
\odpStart
$x \in (-\infty,-16) \cup (-7,15)$
\odpStop
\testStart
A.$x \in (-\infty,-16) \cup (-7,15)$\\
B.$x \in (-\infty,-16) \cup (-7,15]$\\
C.$x \in (-\infty,-16) \cup [-7,15)$\\
D.$x \in (-\infty,-16] \cup (-7,15)$\\
E.$x \in (-\infty,-16] \cup (-7,15]$\\
F.$x \in (-\infty,-16] \cup [-7,15)$\\
G.$x \in (-\infty,-16) \cup [-7,15]$\\
H.$x \in (-\infty,-16] \cup [-7,15]$
\testStop
\kluczStart
A
\kluczStop



\zadStart{Zadanie z Wikieł Z 1.62 b) moja wersja nr 554}

Rozwiązać nierówności $(x+16)(15-x)(x+8)\ge0$.
\zadStop
\rozwStart{Patryk Wirkus}{}
Miejsca zerowe naszego wielomianu to: $-16, 15, -8$.\\
Wielomian jest stopnia nieparzystego, ponadto znak współczynnika przy\linebreak najwyższej potędze x jest ujemny.\\ W związku z tym wykres wielomianu zaczyna się od lewej strony powyżej osi OX. A więc $$x \in (-\infty,-16) \cup (-8,15).$$
\rozwStop
\odpStart
$x \in (-\infty,-16) \cup (-8,15)$
\odpStop
\testStart
A.$x \in (-\infty,-16) \cup (-8,15)$\\
B.$x \in (-\infty,-16) \cup (-8,15]$\\
C.$x \in (-\infty,-16) \cup [-8,15)$\\
D.$x \in (-\infty,-16] \cup (-8,15)$\\
E.$x \in (-\infty,-16] \cup (-8,15]$\\
F.$x \in (-\infty,-16] \cup [-8,15)$\\
G.$x \in (-\infty,-16) \cup [-8,15]$\\
H.$x \in (-\infty,-16] \cup [-8,15]$
\testStop
\kluczStart
A
\kluczStop



\zadStart{Zadanie z Wikieł Z 1.62 b) moja wersja nr 555}

Rozwiązać nierówności $(x+16)(15-x)(x+9)\ge0$.
\zadStop
\rozwStart{Patryk Wirkus}{}
Miejsca zerowe naszego wielomianu to: $-16, 15, -9$.\\
Wielomian jest stopnia nieparzystego, ponadto znak współczynnika przy\linebreak najwyższej potędze x jest ujemny.\\ W związku z tym wykres wielomianu zaczyna się od lewej strony powyżej osi OX. A więc $$x \in (-\infty,-16) \cup (-9,15).$$
\rozwStop
\odpStart
$x \in (-\infty,-16) \cup (-9,15)$
\odpStop
\testStart
A.$x \in (-\infty,-16) \cup (-9,15)$\\
B.$x \in (-\infty,-16) \cup (-9,15]$\\
C.$x \in (-\infty,-16) \cup [-9,15)$\\
D.$x \in (-\infty,-16] \cup (-9,15)$\\
E.$x \in (-\infty,-16] \cup (-9,15]$\\
F.$x \in (-\infty,-16] \cup [-9,15)$\\
G.$x \in (-\infty,-16) \cup [-9,15]$\\
H.$x \in (-\infty,-16] \cup [-9,15]$
\testStop
\kluczStart
A
\kluczStop



\zadStart{Zadanie z Wikieł Z 1.62 b) moja wersja nr 556}

Rozwiązać nierówności $(x+16)(15-x)(x+10)\ge0$.
\zadStop
\rozwStart{Patryk Wirkus}{}
Miejsca zerowe naszego wielomianu to: $-16, 15, -10$.\\
Wielomian jest stopnia nieparzystego, ponadto znak współczynnika przy\linebreak najwyższej potędze x jest ujemny.\\ W związku z tym wykres wielomianu zaczyna się od lewej strony powyżej osi OX. A więc $$x \in (-\infty,-16) \cup (-10,15).$$
\rozwStop
\odpStart
$x \in (-\infty,-16) \cup (-10,15)$
\odpStop
\testStart
A.$x \in (-\infty,-16) \cup (-10,15)$\\
B.$x \in (-\infty,-16) \cup (-10,15]$\\
C.$x \in (-\infty,-16) \cup [-10,15)$\\
D.$x \in (-\infty,-16] \cup (-10,15)$\\
E.$x \in (-\infty,-16] \cup (-10,15]$\\
F.$x \in (-\infty,-16] \cup [-10,15)$\\
G.$x \in (-\infty,-16) \cup [-10,15]$\\
H.$x \in (-\infty,-16] \cup [-10,15]$
\testStop
\kluczStart
A
\kluczStop



\zadStart{Zadanie z Wikieł Z 1.62 b) moja wersja nr 557}

Rozwiązać nierówności $(x+16)(15-x)(x+11)\ge0$.
\zadStop
\rozwStart{Patryk Wirkus}{}
Miejsca zerowe naszego wielomianu to: $-16, 15, -11$.\\
Wielomian jest stopnia nieparzystego, ponadto znak współczynnika przy\linebreak najwyższej potędze x jest ujemny.\\ W związku z tym wykres wielomianu zaczyna się od lewej strony powyżej osi OX. A więc $$x \in (-\infty,-16) \cup (-11,15).$$
\rozwStop
\odpStart
$x \in (-\infty,-16) \cup (-11,15)$
\odpStop
\testStart
A.$x \in (-\infty,-16) \cup (-11,15)$\\
B.$x \in (-\infty,-16) \cup (-11,15]$\\
C.$x \in (-\infty,-16) \cup [-11,15)$\\
D.$x \in (-\infty,-16] \cup (-11,15)$\\
E.$x \in (-\infty,-16] \cup (-11,15]$\\
F.$x \in (-\infty,-16] \cup [-11,15)$\\
G.$x \in (-\infty,-16) \cup [-11,15]$\\
H.$x \in (-\infty,-16] \cup [-11,15]$
\testStop
\kluczStart
A
\kluczStop



\zadStart{Zadanie z Wikieł Z 1.62 b) moja wersja nr 558}

Rozwiązać nierówności $(x+16)(15-x)(x+12)\ge0$.
\zadStop
\rozwStart{Patryk Wirkus}{}
Miejsca zerowe naszego wielomianu to: $-16, 15, -12$.\\
Wielomian jest stopnia nieparzystego, ponadto znak współczynnika przy\linebreak najwyższej potędze x jest ujemny.\\ W związku z tym wykres wielomianu zaczyna się od lewej strony powyżej osi OX. A więc $$x \in (-\infty,-16) \cup (-12,15).$$
\rozwStop
\odpStart
$x \in (-\infty,-16) \cup (-12,15)$
\odpStop
\testStart
A.$x \in (-\infty,-16) \cup (-12,15)$\\
B.$x \in (-\infty,-16) \cup (-12,15]$\\
C.$x \in (-\infty,-16) \cup [-12,15)$\\
D.$x \in (-\infty,-16] \cup (-12,15)$\\
E.$x \in (-\infty,-16] \cup (-12,15]$\\
F.$x \in (-\infty,-16] \cup [-12,15)$\\
G.$x \in (-\infty,-16) \cup [-12,15]$\\
H.$x \in (-\infty,-16] \cup [-12,15]$
\testStop
\kluczStart
A
\kluczStop



\zadStart{Zadanie z Wikieł Z 1.62 b) moja wersja nr 559}

Rozwiązać nierówności $(x+16)(15-x)(x+13)\ge0$.
\zadStop
\rozwStart{Patryk Wirkus}{}
Miejsca zerowe naszego wielomianu to: $-16, 15, -13$.\\
Wielomian jest stopnia nieparzystego, ponadto znak współczynnika przy\linebreak najwyższej potędze x jest ujemny.\\ W związku z tym wykres wielomianu zaczyna się od lewej strony powyżej osi OX. A więc $$x \in (-\infty,-16) \cup (-13,15).$$
\rozwStop
\odpStart
$x \in (-\infty,-16) \cup (-13,15)$
\odpStop
\testStart
A.$x \in (-\infty,-16) \cup (-13,15)$\\
B.$x \in (-\infty,-16) \cup (-13,15]$\\
C.$x \in (-\infty,-16) \cup [-13,15)$\\
D.$x \in (-\infty,-16] \cup (-13,15)$\\
E.$x \in (-\infty,-16] \cup (-13,15]$\\
F.$x \in (-\infty,-16] \cup [-13,15)$\\
G.$x \in (-\infty,-16) \cup [-13,15]$\\
H.$x \in (-\infty,-16] \cup [-13,15]$
\testStop
\kluczStart
A
\kluczStop



\zadStart{Zadanie z Wikieł Z 1.62 b) moja wersja nr 560}

Rozwiązać nierówności $(x+16)(15-x)(x+14)\ge0$.
\zadStop
\rozwStart{Patryk Wirkus}{}
Miejsca zerowe naszego wielomianu to: $-16, 15, -14$.\\
Wielomian jest stopnia nieparzystego, ponadto znak współczynnika przy\linebreak najwyższej potędze x jest ujemny.\\ W związku z tym wykres wielomianu zaczyna się od lewej strony powyżej osi OX. A więc $$x \in (-\infty,-16) \cup (-14,15).$$
\rozwStop
\odpStart
$x \in (-\infty,-16) \cup (-14,15)$
\odpStop
\testStart
A.$x \in (-\infty,-16) \cup (-14,15)$\\
B.$x \in (-\infty,-16) \cup (-14,15]$\\
C.$x \in (-\infty,-16) \cup [-14,15)$\\
D.$x \in (-\infty,-16] \cup (-14,15)$\\
E.$x \in (-\infty,-16] \cup (-14,15]$\\
F.$x \in (-\infty,-16] \cup [-14,15)$\\
G.$x \in (-\infty,-16) \cup [-14,15]$\\
H.$x \in (-\infty,-16] \cup [-14,15]$
\testStop
\kluczStart
A
\kluczStop



\zadStart{Zadanie z Wikieł Z 1.62 b) moja wersja nr 561}

Rozwiązać nierówności $(x+17)(2-x)(x+1)\ge0$.
\zadStop
\rozwStart{Patryk Wirkus}{}
Miejsca zerowe naszego wielomianu to: $-17, 2, -1$.\\
Wielomian jest stopnia nieparzystego, ponadto znak współczynnika przy\linebreak najwyższej potędze x jest ujemny.\\ W związku z tym wykres wielomianu zaczyna się od lewej strony powyżej osi OX. A więc $$x \in (-\infty,-17) \cup (-1,2).$$
\rozwStop
\odpStart
$x \in (-\infty,-17) \cup (-1,2)$
\odpStop
\testStart
A.$x \in (-\infty,-17) \cup (-1,2)$\\
B.$x \in (-\infty,-17) \cup (-1,2]$\\
C.$x \in (-\infty,-17) \cup [-1,2)$\\
D.$x \in (-\infty,-17] \cup (-1,2)$\\
E.$x \in (-\infty,-17] \cup (-1,2]$\\
F.$x \in (-\infty,-17] \cup [-1,2)$\\
G.$x \in (-\infty,-17) \cup [-1,2]$\\
H.$x \in (-\infty,-17] \cup [-1,2]$
\testStop
\kluczStart
A
\kluczStop



\zadStart{Zadanie z Wikieł Z 1.62 b) moja wersja nr 562}

Rozwiązać nierówności $(x+17)(3-x)(x+1)\ge0$.
\zadStop
\rozwStart{Patryk Wirkus}{}
Miejsca zerowe naszego wielomianu to: $-17, 3, -1$.\\
Wielomian jest stopnia nieparzystego, ponadto znak współczynnika przy\linebreak najwyższej potędze x jest ujemny.\\ W związku z tym wykres wielomianu zaczyna się od lewej strony powyżej osi OX. A więc $$x \in (-\infty,-17) \cup (-1,3).$$
\rozwStop
\odpStart
$x \in (-\infty,-17) \cup (-1,3)$
\odpStop
\testStart
A.$x \in (-\infty,-17) \cup (-1,3)$\\
B.$x \in (-\infty,-17) \cup (-1,3]$\\
C.$x \in (-\infty,-17) \cup [-1,3)$\\
D.$x \in (-\infty,-17] \cup (-1,3)$\\
E.$x \in (-\infty,-17] \cup (-1,3]$\\
F.$x \in (-\infty,-17] \cup [-1,3)$\\
G.$x \in (-\infty,-17) \cup [-1,3]$\\
H.$x \in (-\infty,-17] \cup [-1,3]$
\testStop
\kluczStart
A
\kluczStop



\zadStart{Zadanie z Wikieł Z 1.62 b) moja wersja nr 563}

Rozwiązać nierówności $(x+17)(3-x)(x+2)\ge0$.
\zadStop
\rozwStart{Patryk Wirkus}{}
Miejsca zerowe naszego wielomianu to: $-17, 3, -2$.\\
Wielomian jest stopnia nieparzystego, ponadto znak współczynnika przy\linebreak najwyższej potędze x jest ujemny.\\ W związku z tym wykres wielomianu zaczyna się od lewej strony powyżej osi OX. A więc $$x \in (-\infty,-17) \cup (-2,3).$$
\rozwStop
\odpStart
$x \in (-\infty,-17) \cup (-2,3)$
\odpStop
\testStart
A.$x \in (-\infty,-17) \cup (-2,3)$\\
B.$x \in (-\infty,-17) \cup (-2,3]$\\
C.$x \in (-\infty,-17) \cup [-2,3)$\\
D.$x \in (-\infty,-17] \cup (-2,3)$\\
E.$x \in (-\infty,-17] \cup (-2,3]$\\
F.$x \in (-\infty,-17] \cup [-2,3)$\\
G.$x \in (-\infty,-17) \cup [-2,3]$\\
H.$x \in (-\infty,-17] \cup [-2,3]$
\testStop
\kluczStart
A
\kluczStop



\zadStart{Zadanie z Wikieł Z 1.62 b) moja wersja nr 564}

Rozwiązać nierówności $(x+17)(4-x)(x+1)\ge0$.
\zadStop
\rozwStart{Patryk Wirkus}{}
Miejsca zerowe naszego wielomianu to: $-17, 4, -1$.\\
Wielomian jest stopnia nieparzystego, ponadto znak współczynnika przy\linebreak najwyższej potędze x jest ujemny.\\ W związku z tym wykres wielomianu zaczyna się od lewej strony powyżej osi OX. A więc $$x \in (-\infty,-17) \cup (-1,4).$$
\rozwStop
\odpStart
$x \in (-\infty,-17) \cup (-1,4)$
\odpStop
\testStart
A.$x \in (-\infty,-17) \cup (-1,4)$\\
B.$x \in (-\infty,-17) \cup (-1,4]$\\
C.$x \in (-\infty,-17) \cup [-1,4)$\\
D.$x \in (-\infty,-17] \cup (-1,4)$\\
E.$x \in (-\infty,-17] \cup (-1,4]$\\
F.$x \in (-\infty,-17] \cup [-1,4)$\\
G.$x \in (-\infty,-17) \cup [-1,4]$\\
H.$x \in (-\infty,-17] \cup [-1,4]$
\testStop
\kluczStart
A
\kluczStop



\zadStart{Zadanie z Wikieł Z 1.62 b) moja wersja nr 565}

Rozwiązać nierówności $(x+17)(4-x)(x+2)\ge0$.
\zadStop
\rozwStart{Patryk Wirkus}{}
Miejsca zerowe naszego wielomianu to: $-17, 4, -2$.\\
Wielomian jest stopnia nieparzystego, ponadto znak współczynnika przy\linebreak najwyższej potędze x jest ujemny.\\ W związku z tym wykres wielomianu zaczyna się od lewej strony powyżej osi OX. A więc $$x \in (-\infty,-17) \cup (-2,4).$$
\rozwStop
\odpStart
$x \in (-\infty,-17) \cup (-2,4)$
\odpStop
\testStart
A.$x \in (-\infty,-17) \cup (-2,4)$\\
B.$x \in (-\infty,-17) \cup (-2,4]$\\
C.$x \in (-\infty,-17) \cup [-2,4)$\\
D.$x \in (-\infty,-17] \cup (-2,4)$\\
E.$x \in (-\infty,-17] \cup (-2,4]$\\
F.$x \in (-\infty,-17] \cup [-2,4)$\\
G.$x \in (-\infty,-17) \cup [-2,4]$\\
H.$x \in (-\infty,-17] \cup [-2,4]$
\testStop
\kluczStart
A
\kluczStop



\zadStart{Zadanie z Wikieł Z 1.62 b) moja wersja nr 566}

Rozwiązać nierówności $(x+17)(4-x)(x+3)\ge0$.
\zadStop
\rozwStart{Patryk Wirkus}{}
Miejsca zerowe naszego wielomianu to: $-17, 4, -3$.\\
Wielomian jest stopnia nieparzystego, ponadto znak współczynnika przy\linebreak najwyższej potędze x jest ujemny.\\ W związku z tym wykres wielomianu zaczyna się od lewej strony powyżej osi OX. A więc $$x \in (-\infty,-17) \cup (-3,4).$$
\rozwStop
\odpStart
$x \in (-\infty,-17) \cup (-3,4)$
\odpStop
\testStart
A.$x \in (-\infty,-17) \cup (-3,4)$\\
B.$x \in (-\infty,-17) \cup (-3,4]$\\
C.$x \in (-\infty,-17) \cup [-3,4)$\\
D.$x \in (-\infty,-17] \cup (-3,4)$\\
E.$x \in (-\infty,-17] \cup (-3,4]$\\
F.$x \in (-\infty,-17] \cup [-3,4)$\\
G.$x \in (-\infty,-17) \cup [-3,4]$\\
H.$x \in (-\infty,-17] \cup [-3,4]$
\testStop
\kluczStart
A
\kluczStop



\zadStart{Zadanie z Wikieł Z 1.62 b) moja wersja nr 567}

Rozwiązać nierówności $(x+17)(5-x)(x+1)\ge0$.
\zadStop
\rozwStart{Patryk Wirkus}{}
Miejsca zerowe naszego wielomianu to: $-17, 5, -1$.\\
Wielomian jest stopnia nieparzystego, ponadto znak współczynnika przy\linebreak najwyższej potędze x jest ujemny.\\ W związku z tym wykres wielomianu zaczyna się od lewej strony powyżej osi OX. A więc $$x \in (-\infty,-17) \cup (-1,5).$$
\rozwStop
\odpStart
$x \in (-\infty,-17) \cup (-1,5)$
\odpStop
\testStart
A.$x \in (-\infty,-17) \cup (-1,5)$\\
B.$x \in (-\infty,-17) \cup (-1,5]$\\
C.$x \in (-\infty,-17) \cup [-1,5)$\\
D.$x \in (-\infty,-17] \cup (-1,5)$\\
E.$x \in (-\infty,-17] \cup (-1,5]$\\
F.$x \in (-\infty,-17] \cup [-1,5)$\\
G.$x \in (-\infty,-17) \cup [-1,5]$\\
H.$x \in (-\infty,-17] \cup [-1,5]$
\testStop
\kluczStart
A
\kluczStop



\zadStart{Zadanie z Wikieł Z 1.62 b) moja wersja nr 568}

Rozwiązać nierówności $(x+17)(5-x)(x+2)\ge0$.
\zadStop
\rozwStart{Patryk Wirkus}{}
Miejsca zerowe naszego wielomianu to: $-17, 5, -2$.\\
Wielomian jest stopnia nieparzystego, ponadto znak współczynnika przy\linebreak najwyższej potędze x jest ujemny.\\ W związku z tym wykres wielomianu zaczyna się od lewej strony powyżej osi OX. A więc $$x \in (-\infty,-17) \cup (-2,5).$$
\rozwStop
\odpStart
$x \in (-\infty,-17) \cup (-2,5)$
\odpStop
\testStart
A.$x \in (-\infty,-17) \cup (-2,5)$\\
B.$x \in (-\infty,-17) \cup (-2,5]$\\
C.$x \in (-\infty,-17) \cup [-2,5)$\\
D.$x \in (-\infty,-17] \cup (-2,5)$\\
E.$x \in (-\infty,-17] \cup (-2,5]$\\
F.$x \in (-\infty,-17] \cup [-2,5)$\\
G.$x \in (-\infty,-17) \cup [-2,5]$\\
H.$x \in (-\infty,-17] \cup [-2,5]$
\testStop
\kluczStart
A
\kluczStop



\zadStart{Zadanie z Wikieł Z 1.62 b) moja wersja nr 569}

Rozwiązać nierówności $(x+17)(5-x)(x+3)\ge0$.
\zadStop
\rozwStart{Patryk Wirkus}{}
Miejsca zerowe naszego wielomianu to: $-17, 5, -3$.\\
Wielomian jest stopnia nieparzystego, ponadto znak współczynnika przy\linebreak najwyższej potędze x jest ujemny.\\ W związku z tym wykres wielomianu zaczyna się od lewej strony powyżej osi OX. A więc $$x \in (-\infty,-17) \cup (-3,5).$$
\rozwStop
\odpStart
$x \in (-\infty,-17) \cup (-3,5)$
\odpStop
\testStart
A.$x \in (-\infty,-17) \cup (-3,5)$\\
B.$x \in (-\infty,-17) \cup (-3,5]$\\
C.$x \in (-\infty,-17) \cup [-3,5)$\\
D.$x \in (-\infty,-17] \cup (-3,5)$\\
E.$x \in (-\infty,-17] \cup (-3,5]$\\
F.$x \in (-\infty,-17] \cup [-3,5)$\\
G.$x \in (-\infty,-17) \cup [-3,5]$\\
H.$x \in (-\infty,-17] \cup [-3,5]$
\testStop
\kluczStart
A
\kluczStop



\zadStart{Zadanie z Wikieł Z 1.62 b) moja wersja nr 570}

Rozwiązać nierówności $(x+17)(5-x)(x+4)\ge0$.
\zadStop
\rozwStart{Patryk Wirkus}{}
Miejsca zerowe naszego wielomianu to: $-17, 5, -4$.\\
Wielomian jest stopnia nieparzystego, ponadto znak współczynnika przy\linebreak najwyższej potędze x jest ujemny.\\ W związku z tym wykres wielomianu zaczyna się od lewej strony powyżej osi OX. A więc $$x \in (-\infty,-17) \cup (-4,5).$$
\rozwStop
\odpStart
$x \in (-\infty,-17) \cup (-4,5)$
\odpStop
\testStart
A.$x \in (-\infty,-17) \cup (-4,5)$\\
B.$x \in (-\infty,-17) \cup (-4,5]$\\
C.$x \in (-\infty,-17) \cup [-4,5)$\\
D.$x \in (-\infty,-17] \cup (-4,5)$\\
E.$x \in (-\infty,-17] \cup (-4,5]$\\
F.$x \in (-\infty,-17] \cup [-4,5)$\\
G.$x \in (-\infty,-17) \cup [-4,5]$\\
H.$x \in (-\infty,-17] \cup [-4,5]$
\testStop
\kluczStart
A
\kluczStop



\zadStart{Zadanie z Wikieł Z 1.62 b) moja wersja nr 571}

Rozwiązać nierówności $(x+17)(6-x)(x+1)\ge0$.
\zadStop
\rozwStart{Patryk Wirkus}{}
Miejsca zerowe naszego wielomianu to: $-17, 6, -1$.\\
Wielomian jest stopnia nieparzystego, ponadto znak współczynnika przy\linebreak najwyższej potędze x jest ujemny.\\ W związku z tym wykres wielomianu zaczyna się od lewej strony powyżej osi OX. A więc $$x \in (-\infty,-17) \cup (-1,6).$$
\rozwStop
\odpStart
$x \in (-\infty,-17) \cup (-1,6)$
\odpStop
\testStart
A.$x \in (-\infty,-17) \cup (-1,6)$\\
B.$x \in (-\infty,-17) \cup (-1,6]$\\
C.$x \in (-\infty,-17) \cup [-1,6)$\\
D.$x \in (-\infty,-17] \cup (-1,6)$\\
E.$x \in (-\infty,-17] \cup (-1,6]$\\
F.$x \in (-\infty,-17] \cup [-1,6)$\\
G.$x \in (-\infty,-17) \cup [-1,6]$\\
H.$x \in (-\infty,-17] \cup [-1,6]$
\testStop
\kluczStart
A
\kluczStop



\zadStart{Zadanie z Wikieł Z 1.62 b) moja wersja nr 572}

Rozwiązać nierówności $(x+17)(6-x)(x+2)\ge0$.
\zadStop
\rozwStart{Patryk Wirkus}{}
Miejsca zerowe naszego wielomianu to: $-17, 6, -2$.\\
Wielomian jest stopnia nieparzystego, ponadto znak współczynnika przy\linebreak najwyższej potędze x jest ujemny.\\ W związku z tym wykres wielomianu zaczyna się od lewej strony powyżej osi OX. A więc $$x \in (-\infty,-17) \cup (-2,6).$$
\rozwStop
\odpStart
$x \in (-\infty,-17) \cup (-2,6)$
\odpStop
\testStart
A.$x \in (-\infty,-17) \cup (-2,6)$\\
B.$x \in (-\infty,-17) \cup (-2,6]$\\
C.$x \in (-\infty,-17) \cup [-2,6)$\\
D.$x \in (-\infty,-17] \cup (-2,6)$\\
E.$x \in (-\infty,-17] \cup (-2,6]$\\
F.$x \in (-\infty,-17] \cup [-2,6)$\\
G.$x \in (-\infty,-17) \cup [-2,6]$\\
H.$x \in (-\infty,-17] \cup [-2,6]$
\testStop
\kluczStart
A
\kluczStop



\zadStart{Zadanie z Wikieł Z 1.62 b) moja wersja nr 573}

Rozwiązać nierówności $(x+17)(6-x)(x+3)\ge0$.
\zadStop
\rozwStart{Patryk Wirkus}{}
Miejsca zerowe naszego wielomianu to: $-17, 6, -3$.\\
Wielomian jest stopnia nieparzystego, ponadto znak współczynnika przy\linebreak najwyższej potędze x jest ujemny.\\ W związku z tym wykres wielomianu zaczyna się od lewej strony powyżej osi OX. A więc $$x \in (-\infty,-17) \cup (-3,6).$$
\rozwStop
\odpStart
$x \in (-\infty,-17) \cup (-3,6)$
\odpStop
\testStart
A.$x \in (-\infty,-17) \cup (-3,6)$\\
B.$x \in (-\infty,-17) \cup (-3,6]$\\
C.$x \in (-\infty,-17) \cup [-3,6)$\\
D.$x \in (-\infty,-17] \cup (-3,6)$\\
E.$x \in (-\infty,-17] \cup (-3,6]$\\
F.$x \in (-\infty,-17] \cup [-3,6)$\\
G.$x \in (-\infty,-17) \cup [-3,6]$\\
H.$x \in (-\infty,-17] \cup [-3,6]$
\testStop
\kluczStart
A
\kluczStop



\zadStart{Zadanie z Wikieł Z 1.62 b) moja wersja nr 574}

Rozwiązać nierówności $(x+17)(6-x)(x+4)\ge0$.
\zadStop
\rozwStart{Patryk Wirkus}{}
Miejsca zerowe naszego wielomianu to: $-17, 6, -4$.\\
Wielomian jest stopnia nieparzystego, ponadto znak współczynnika przy\linebreak najwyższej potędze x jest ujemny.\\ W związku z tym wykres wielomianu zaczyna się od lewej strony powyżej osi OX. A więc $$x \in (-\infty,-17) \cup (-4,6).$$
\rozwStop
\odpStart
$x \in (-\infty,-17) \cup (-4,6)$
\odpStop
\testStart
A.$x \in (-\infty,-17) \cup (-4,6)$\\
B.$x \in (-\infty,-17) \cup (-4,6]$\\
C.$x \in (-\infty,-17) \cup [-4,6)$\\
D.$x \in (-\infty,-17] \cup (-4,6)$\\
E.$x \in (-\infty,-17] \cup (-4,6]$\\
F.$x \in (-\infty,-17] \cup [-4,6)$\\
G.$x \in (-\infty,-17) \cup [-4,6]$\\
H.$x \in (-\infty,-17] \cup [-4,6]$
\testStop
\kluczStart
A
\kluczStop



\zadStart{Zadanie z Wikieł Z 1.62 b) moja wersja nr 575}

Rozwiązać nierówności $(x+17)(6-x)(x+5)\ge0$.
\zadStop
\rozwStart{Patryk Wirkus}{}
Miejsca zerowe naszego wielomianu to: $-17, 6, -5$.\\
Wielomian jest stopnia nieparzystego, ponadto znak współczynnika przy\linebreak najwyższej potędze x jest ujemny.\\ W związku z tym wykres wielomianu zaczyna się od lewej strony powyżej osi OX. A więc $$x \in (-\infty,-17) \cup (-5,6).$$
\rozwStop
\odpStart
$x \in (-\infty,-17) \cup (-5,6)$
\odpStop
\testStart
A.$x \in (-\infty,-17) \cup (-5,6)$\\
B.$x \in (-\infty,-17) \cup (-5,6]$\\
C.$x \in (-\infty,-17) \cup [-5,6)$\\
D.$x \in (-\infty,-17] \cup (-5,6)$\\
E.$x \in (-\infty,-17] \cup (-5,6]$\\
F.$x \in (-\infty,-17] \cup [-5,6)$\\
G.$x \in (-\infty,-17) \cup [-5,6]$\\
H.$x \in (-\infty,-17] \cup [-5,6]$
\testStop
\kluczStart
A
\kluczStop



\zadStart{Zadanie z Wikieł Z 1.62 b) moja wersja nr 576}

Rozwiązać nierówności $(x+17)(7-x)(x+1)\ge0$.
\zadStop
\rozwStart{Patryk Wirkus}{}
Miejsca zerowe naszego wielomianu to: $-17, 7, -1$.\\
Wielomian jest stopnia nieparzystego, ponadto znak współczynnika przy\linebreak najwyższej potędze x jest ujemny.\\ W związku z tym wykres wielomianu zaczyna się od lewej strony powyżej osi OX. A więc $$x \in (-\infty,-17) \cup (-1,7).$$
\rozwStop
\odpStart
$x \in (-\infty,-17) \cup (-1,7)$
\odpStop
\testStart
A.$x \in (-\infty,-17) \cup (-1,7)$\\
B.$x \in (-\infty,-17) \cup (-1,7]$\\
C.$x \in (-\infty,-17) \cup [-1,7)$\\
D.$x \in (-\infty,-17] \cup (-1,7)$\\
E.$x \in (-\infty,-17] \cup (-1,7]$\\
F.$x \in (-\infty,-17] \cup [-1,7)$\\
G.$x \in (-\infty,-17) \cup [-1,7]$\\
H.$x \in (-\infty,-17] \cup [-1,7]$
\testStop
\kluczStart
A
\kluczStop



\zadStart{Zadanie z Wikieł Z 1.62 b) moja wersja nr 577}

Rozwiązać nierówności $(x+17)(7-x)(x+2)\ge0$.
\zadStop
\rozwStart{Patryk Wirkus}{}
Miejsca zerowe naszego wielomianu to: $-17, 7, -2$.\\
Wielomian jest stopnia nieparzystego, ponadto znak współczynnika przy\linebreak najwyższej potędze x jest ujemny.\\ W związku z tym wykres wielomianu zaczyna się od lewej strony powyżej osi OX. A więc $$x \in (-\infty,-17) \cup (-2,7).$$
\rozwStop
\odpStart
$x \in (-\infty,-17) \cup (-2,7)$
\odpStop
\testStart
A.$x \in (-\infty,-17) \cup (-2,7)$\\
B.$x \in (-\infty,-17) \cup (-2,7]$\\
C.$x \in (-\infty,-17) \cup [-2,7)$\\
D.$x \in (-\infty,-17] \cup (-2,7)$\\
E.$x \in (-\infty,-17] \cup (-2,7]$\\
F.$x \in (-\infty,-17] \cup [-2,7)$\\
G.$x \in (-\infty,-17) \cup [-2,7]$\\
H.$x \in (-\infty,-17] \cup [-2,7]$
\testStop
\kluczStart
A
\kluczStop



\zadStart{Zadanie z Wikieł Z 1.62 b) moja wersja nr 578}

Rozwiązać nierówności $(x+17)(7-x)(x+3)\ge0$.
\zadStop
\rozwStart{Patryk Wirkus}{}
Miejsca zerowe naszego wielomianu to: $-17, 7, -3$.\\
Wielomian jest stopnia nieparzystego, ponadto znak współczynnika przy\linebreak najwyższej potędze x jest ujemny.\\ W związku z tym wykres wielomianu zaczyna się od lewej strony powyżej osi OX. A więc $$x \in (-\infty,-17) \cup (-3,7).$$
\rozwStop
\odpStart
$x \in (-\infty,-17) \cup (-3,7)$
\odpStop
\testStart
A.$x \in (-\infty,-17) \cup (-3,7)$\\
B.$x \in (-\infty,-17) \cup (-3,7]$\\
C.$x \in (-\infty,-17) \cup [-3,7)$\\
D.$x \in (-\infty,-17] \cup (-3,7)$\\
E.$x \in (-\infty,-17] \cup (-3,7]$\\
F.$x \in (-\infty,-17] \cup [-3,7)$\\
G.$x \in (-\infty,-17) \cup [-3,7]$\\
H.$x \in (-\infty,-17] \cup [-3,7]$
\testStop
\kluczStart
A
\kluczStop



\zadStart{Zadanie z Wikieł Z 1.62 b) moja wersja nr 579}

Rozwiązać nierówności $(x+17)(7-x)(x+4)\ge0$.
\zadStop
\rozwStart{Patryk Wirkus}{}
Miejsca zerowe naszego wielomianu to: $-17, 7, -4$.\\
Wielomian jest stopnia nieparzystego, ponadto znak współczynnika przy\linebreak najwyższej potędze x jest ujemny.\\ W związku z tym wykres wielomianu zaczyna się od lewej strony powyżej osi OX. A więc $$x \in (-\infty,-17) \cup (-4,7).$$
\rozwStop
\odpStart
$x \in (-\infty,-17) \cup (-4,7)$
\odpStop
\testStart
A.$x \in (-\infty,-17) \cup (-4,7)$\\
B.$x \in (-\infty,-17) \cup (-4,7]$\\
C.$x \in (-\infty,-17) \cup [-4,7)$\\
D.$x \in (-\infty,-17] \cup (-4,7)$\\
E.$x \in (-\infty,-17] \cup (-4,7]$\\
F.$x \in (-\infty,-17] \cup [-4,7)$\\
G.$x \in (-\infty,-17) \cup [-4,7]$\\
H.$x \in (-\infty,-17] \cup [-4,7]$
\testStop
\kluczStart
A
\kluczStop



\zadStart{Zadanie z Wikieł Z 1.62 b) moja wersja nr 580}

Rozwiązać nierówności $(x+17)(7-x)(x+5)\ge0$.
\zadStop
\rozwStart{Patryk Wirkus}{}
Miejsca zerowe naszego wielomianu to: $-17, 7, -5$.\\
Wielomian jest stopnia nieparzystego, ponadto znak współczynnika przy\linebreak najwyższej potędze x jest ujemny.\\ W związku z tym wykres wielomianu zaczyna się od lewej strony powyżej osi OX. A więc $$x \in (-\infty,-17) \cup (-5,7).$$
\rozwStop
\odpStart
$x \in (-\infty,-17) \cup (-5,7)$
\odpStop
\testStart
A.$x \in (-\infty,-17) \cup (-5,7)$\\
B.$x \in (-\infty,-17) \cup (-5,7]$\\
C.$x \in (-\infty,-17) \cup [-5,7)$\\
D.$x \in (-\infty,-17] \cup (-5,7)$\\
E.$x \in (-\infty,-17] \cup (-5,7]$\\
F.$x \in (-\infty,-17] \cup [-5,7)$\\
G.$x \in (-\infty,-17) \cup [-5,7]$\\
H.$x \in (-\infty,-17] \cup [-5,7]$
\testStop
\kluczStart
A
\kluczStop



\zadStart{Zadanie z Wikieł Z 1.62 b) moja wersja nr 581}

Rozwiązać nierówności $(x+17)(7-x)(x+6)\ge0$.
\zadStop
\rozwStart{Patryk Wirkus}{}
Miejsca zerowe naszego wielomianu to: $-17, 7, -6$.\\
Wielomian jest stopnia nieparzystego, ponadto znak współczynnika przy\linebreak najwyższej potędze x jest ujemny.\\ W związku z tym wykres wielomianu zaczyna się od lewej strony powyżej osi OX. A więc $$x \in (-\infty,-17) \cup (-6,7).$$
\rozwStop
\odpStart
$x \in (-\infty,-17) \cup (-6,7)$
\odpStop
\testStart
A.$x \in (-\infty,-17) \cup (-6,7)$\\
B.$x \in (-\infty,-17) \cup (-6,7]$\\
C.$x \in (-\infty,-17) \cup [-6,7)$\\
D.$x \in (-\infty,-17] \cup (-6,7)$\\
E.$x \in (-\infty,-17] \cup (-6,7]$\\
F.$x \in (-\infty,-17] \cup [-6,7)$\\
G.$x \in (-\infty,-17) \cup [-6,7]$\\
H.$x \in (-\infty,-17] \cup [-6,7]$
\testStop
\kluczStart
A
\kluczStop



\zadStart{Zadanie z Wikieł Z 1.62 b) moja wersja nr 582}

Rozwiązać nierówności $(x+17)(8-x)(x+1)\ge0$.
\zadStop
\rozwStart{Patryk Wirkus}{}
Miejsca zerowe naszego wielomianu to: $-17, 8, -1$.\\
Wielomian jest stopnia nieparzystego, ponadto znak współczynnika przy\linebreak najwyższej potędze x jest ujemny.\\ W związku z tym wykres wielomianu zaczyna się od lewej strony powyżej osi OX. A więc $$x \in (-\infty,-17) \cup (-1,8).$$
\rozwStop
\odpStart
$x \in (-\infty,-17) \cup (-1,8)$
\odpStop
\testStart
A.$x \in (-\infty,-17) \cup (-1,8)$\\
B.$x \in (-\infty,-17) \cup (-1,8]$\\
C.$x \in (-\infty,-17) \cup [-1,8)$\\
D.$x \in (-\infty,-17] \cup (-1,8)$\\
E.$x \in (-\infty,-17] \cup (-1,8]$\\
F.$x \in (-\infty,-17] \cup [-1,8)$\\
G.$x \in (-\infty,-17) \cup [-1,8]$\\
H.$x \in (-\infty,-17] \cup [-1,8]$
\testStop
\kluczStart
A
\kluczStop



\zadStart{Zadanie z Wikieł Z 1.62 b) moja wersja nr 583}

Rozwiązać nierówności $(x+17)(8-x)(x+2)\ge0$.
\zadStop
\rozwStart{Patryk Wirkus}{}
Miejsca zerowe naszego wielomianu to: $-17, 8, -2$.\\
Wielomian jest stopnia nieparzystego, ponadto znak współczynnika przy\linebreak najwyższej potędze x jest ujemny.\\ W związku z tym wykres wielomianu zaczyna się od lewej strony powyżej osi OX. A więc $$x \in (-\infty,-17) \cup (-2,8).$$
\rozwStop
\odpStart
$x \in (-\infty,-17) \cup (-2,8)$
\odpStop
\testStart
A.$x \in (-\infty,-17) \cup (-2,8)$\\
B.$x \in (-\infty,-17) \cup (-2,8]$\\
C.$x \in (-\infty,-17) \cup [-2,8)$\\
D.$x \in (-\infty,-17] \cup (-2,8)$\\
E.$x \in (-\infty,-17] \cup (-2,8]$\\
F.$x \in (-\infty,-17] \cup [-2,8)$\\
G.$x \in (-\infty,-17) \cup [-2,8]$\\
H.$x \in (-\infty,-17] \cup [-2,8]$
\testStop
\kluczStart
A
\kluczStop



\zadStart{Zadanie z Wikieł Z 1.62 b) moja wersja nr 584}

Rozwiązać nierówności $(x+17)(8-x)(x+3)\ge0$.
\zadStop
\rozwStart{Patryk Wirkus}{}
Miejsca zerowe naszego wielomianu to: $-17, 8, -3$.\\
Wielomian jest stopnia nieparzystego, ponadto znak współczynnika przy\linebreak najwyższej potędze x jest ujemny.\\ W związku z tym wykres wielomianu zaczyna się od lewej strony powyżej osi OX. A więc $$x \in (-\infty,-17) \cup (-3,8).$$
\rozwStop
\odpStart
$x \in (-\infty,-17) \cup (-3,8)$
\odpStop
\testStart
A.$x \in (-\infty,-17) \cup (-3,8)$\\
B.$x \in (-\infty,-17) \cup (-3,8]$\\
C.$x \in (-\infty,-17) \cup [-3,8)$\\
D.$x \in (-\infty,-17] \cup (-3,8)$\\
E.$x \in (-\infty,-17] \cup (-3,8]$\\
F.$x \in (-\infty,-17] \cup [-3,8)$\\
G.$x \in (-\infty,-17) \cup [-3,8]$\\
H.$x \in (-\infty,-17] \cup [-3,8]$
\testStop
\kluczStart
A
\kluczStop



\zadStart{Zadanie z Wikieł Z 1.62 b) moja wersja nr 585}

Rozwiązać nierówności $(x+17)(8-x)(x+4)\ge0$.
\zadStop
\rozwStart{Patryk Wirkus}{}
Miejsca zerowe naszego wielomianu to: $-17, 8, -4$.\\
Wielomian jest stopnia nieparzystego, ponadto znak współczynnika przy\linebreak najwyższej potędze x jest ujemny.\\ W związku z tym wykres wielomianu zaczyna się od lewej strony powyżej osi OX. A więc $$x \in (-\infty,-17) \cup (-4,8).$$
\rozwStop
\odpStart
$x \in (-\infty,-17) \cup (-4,8)$
\odpStop
\testStart
A.$x \in (-\infty,-17) \cup (-4,8)$\\
B.$x \in (-\infty,-17) \cup (-4,8]$\\
C.$x \in (-\infty,-17) \cup [-4,8)$\\
D.$x \in (-\infty,-17] \cup (-4,8)$\\
E.$x \in (-\infty,-17] \cup (-4,8]$\\
F.$x \in (-\infty,-17] \cup [-4,8)$\\
G.$x \in (-\infty,-17) \cup [-4,8]$\\
H.$x \in (-\infty,-17] \cup [-4,8]$
\testStop
\kluczStart
A
\kluczStop



\zadStart{Zadanie z Wikieł Z 1.62 b) moja wersja nr 586}

Rozwiązać nierówności $(x+17)(8-x)(x+5)\ge0$.
\zadStop
\rozwStart{Patryk Wirkus}{}
Miejsca zerowe naszego wielomianu to: $-17, 8, -5$.\\
Wielomian jest stopnia nieparzystego, ponadto znak współczynnika przy\linebreak najwyższej potędze x jest ujemny.\\ W związku z tym wykres wielomianu zaczyna się od lewej strony powyżej osi OX. A więc $$x \in (-\infty,-17) \cup (-5,8).$$
\rozwStop
\odpStart
$x \in (-\infty,-17) \cup (-5,8)$
\odpStop
\testStart
A.$x \in (-\infty,-17) \cup (-5,8)$\\
B.$x \in (-\infty,-17) \cup (-5,8]$\\
C.$x \in (-\infty,-17) \cup [-5,8)$\\
D.$x \in (-\infty,-17] \cup (-5,8)$\\
E.$x \in (-\infty,-17] \cup (-5,8]$\\
F.$x \in (-\infty,-17] \cup [-5,8)$\\
G.$x \in (-\infty,-17) \cup [-5,8]$\\
H.$x \in (-\infty,-17] \cup [-5,8]$
\testStop
\kluczStart
A
\kluczStop



\zadStart{Zadanie z Wikieł Z 1.62 b) moja wersja nr 587}

Rozwiązać nierówności $(x+17)(8-x)(x+6)\ge0$.
\zadStop
\rozwStart{Patryk Wirkus}{}
Miejsca zerowe naszego wielomianu to: $-17, 8, -6$.\\
Wielomian jest stopnia nieparzystego, ponadto znak współczynnika przy\linebreak najwyższej potędze x jest ujemny.\\ W związku z tym wykres wielomianu zaczyna się od lewej strony powyżej osi OX. A więc $$x \in (-\infty,-17) \cup (-6,8).$$
\rozwStop
\odpStart
$x \in (-\infty,-17) \cup (-6,8)$
\odpStop
\testStart
A.$x \in (-\infty,-17) \cup (-6,8)$\\
B.$x \in (-\infty,-17) \cup (-6,8]$\\
C.$x \in (-\infty,-17) \cup [-6,8)$\\
D.$x \in (-\infty,-17] \cup (-6,8)$\\
E.$x \in (-\infty,-17] \cup (-6,8]$\\
F.$x \in (-\infty,-17] \cup [-6,8)$\\
G.$x \in (-\infty,-17) \cup [-6,8]$\\
H.$x \in (-\infty,-17] \cup [-6,8]$
\testStop
\kluczStart
A
\kluczStop



\zadStart{Zadanie z Wikieł Z 1.62 b) moja wersja nr 588}

Rozwiązać nierówności $(x+17)(8-x)(x+7)\ge0$.
\zadStop
\rozwStart{Patryk Wirkus}{}
Miejsca zerowe naszego wielomianu to: $-17, 8, -7$.\\
Wielomian jest stopnia nieparzystego, ponadto znak współczynnika przy\linebreak najwyższej potędze x jest ujemny.\\ W związku z tym wykres wielomianu zaczyna się od lewej strony powyżej osi OX. A więc $$x \in (-\infty,-17) \cup (-7,8).$$
\rozwStop
\odpStart
$x \in (-\infty,-17) \cup (-7,8)$
\odpStop
\testStart
A.$x \in (-\infty,-17) \cup (-7,8)$\\
B.$x \in (-\infty,-17) \cup (-7,8]$\\
C.$x \in (-\infty,-17) \cup [-7,8)$\\
D.$x \in (-\infty,-17] \cup (-7,8)$\\
E.$x \in (-\infty,-17] \cup (-7,8]$\\
F.$x \in (-\infty,-17] \cup [-7,8)$\\
G.$x \in (-\infty,-17) \cup [-7,8]$\\
H.$x \in (-\infty,-17] \cup [-7,8]$
\testStop
\kluczStart
A
\kluczStop



\zadStart{Zadanie z Wikieł Z 1.62 b) moja wersja nr 589}

Rozwiązać nierówności $(x+17)(9-x)(x+1)\ge0$.
\zadStop
\rozwStart{Patryk Wirkus}{}
Miejsca zerowe naszego wielomianu to: $-17, 9, -1$.\\
Wielomian jest stopnia nieparzystego, ponadto znak współczynnika przy\linebreak najwyższej potędze x jest ujemny.\\ W związku z tym wykres wielomianu zaczyna się od lewej strony powyżej osi OX. A więc $$x \in (-\infty,-17) \cup (-1,9).$$
\rozwStop
\odpStart
$x \in (-\infty,-17) \cup (-1,9)$
\odpStop
\testStart
A.$x \in (-\infty,-17) \cup (-1,9)$\\
B.$x \in (-\infty,-17) \cup (-1,9]$\\
C.$x \in (-\infty,-17) \cup [-1,9)$\\
D.$x \in (-\infty,-17] \cup (-1,9)$\\
E.$x \in (-\infty,-17] \cup (-1,9]$\\
F.$x \in (-\infty,-17] \cup [-1,9)$\\
G.$x \in (-\infty,-17) \cup [-1,9]$\\
H.$x \in (-\infty,-17] \cup [-1,9]$
\testStop
\kluczStart
A
\kluczStop



\zadStart{Zadanie z Wikieł Z 1.62 b) moja wersja nr 590}

Rozwiązać nierówności $(x+17)(9-x)(x+2)\ge0$.
\zadStop
\rozwStart{Patryk Wirkus}{}
Miejsca zerowe naszego wielomianu to: $-17, 9, -2$.\\
Wielomian jest stopnia nieparzystego, ponadto znak współczynnika przy\linebreak najwyższej potędze x jest ujemny.\\ W związku z tym wykres wielomianu zaczyna się od lewej strony powyżej osi OX. A więc $$x \in (-\infty,-17) \cup (-2,9).$$
\rozwStop
\odpStart
$x \in (-\infty,-17) \cup (-2,9)$
\odpStop
\testStart
A.$x \in (-\infty,-17) \cup (-2,9)$\\
B.$x \in (-\infty,-17) \cup (-2,9]$\\
C.$x \in (-\infty,-17) \cup [-2,9)$\\
D.$x \in (-\infty,-17] \cup (-2,9)$\\
E.$x \in (-\infty,-17] \cup (-2,9]$\\
F.$x \in (-\infty,-17] \cup [-2,9)$\\
G.$x \in (-\infty,-17) \cup [-2,9]$\\
H.$x \in (-\infty,-17] \cup [-2,9]$
\testStop
\kluczStart
A
\kluczStop



\zadStart{Zadanie z Wikieł Z 1.62 b) moja wersja nr 591}

Rozwiązać nierówności $(x+17)(9-x)(x+3)\ge0$.
\zadStop
\rozwStart{Patryk Wirkus}{}
Miejsca zerowe naszego wielomianu to: $-17, 9, -3$.\\
Wielomian jest stopnia nieparzystego, ponadto znak współczynnika przy\linebreak najwyższej potędze x jest ujemny.\\ W związku z tym wykres wielomianu zaczyna się od lewej strony powyżej osi OX. A więc $$x \in (-\infty,-17) \cup (-3,9).$$
\rozwStop
\odpStart
$x \in (-\infty,-17) \cup (-3,9)$
\odpStop
\testStart
A.$x \in (-\infty,-17) \cup (-3,9)$\\
B.$x \in (-\infty,-17) \cup (-3,9]$\\
C.$x \in (-\infty,-17) \cup [-3,9)$\\
D.$x \in (-\infty,-17] \cup (-3,9)$\\
E.$x \in (-\infty,-17] \cup (-3,9]$\\
F.$x \in (-\infty,-17] \cup [-3,9)$\\
G.$x \in (-\infty,-17) \cup [-3,9]$\\
H.$x \in (-\infty,-17] \cup [-3,9]$
\testStop
\kluczStart
A
\kluczStop



\zadStart{Zadanie z Wikieł Z 1.62 b) moja wersja nr 592}

Rozwiązać nierówności $(x+17)(9-x)(x+4)\ge0$.
\zadStop
\rozwStart{Patryk Wirkus}{}
Miejsca zerowe naszego wielomianu to: $-17, 9, -4$.\\
Wielomian jest stopnia nieparzystego, ponadto znak współczynnika przy\linebreak najwyższej potędze x jest ujemny.\\ W związku z tym wykres wielomianu zaczyna się od lewej strony powyżej osi OX. A więc $$x \in (-\infty,-17) \cup (-4,9).$$
\rozwStop
\odpStart
$x \in (-\infty,-17) \cup (-4,9)$
\odpStop
\testStart
A.$x \in (-\infty,-17) \cup (-4,9)$\\
B.$x \in (-\infty,-17) \cup (-4,9]$\\
C.$x \in (-\infty,-17) \cup [-4,9)$\\
D.$x \in (-\infty,-17] \cup (-4,9)$\\
E.$x \in (-\infty,-17] \cup (-4,9]$\\
F.$x \in (-\infty,-17] \cup [-4,9)$\\
G.$x \in (-\infty,-17) \cup [-4,9]$\\
H.$x \in (-\infty,-17] \cup [-4,9]$
\testStop
\kluczStart
A
\kluczStop



\zadStart{Zadanie z Wikieł Z 1.62 b) moja wersja nr 593}

Rozwiązać nierówności $(x+17)(9-x)(x+5)\ge0$.
\zadStop
\rozwStart{Patryk Wirkus}{}
Miejsca zerowe naszego wielomianu to: $-17, 9, -5$.\\
Wielomian jest stopnia nieparzystego, ponadto znak współczynnika przy\linebreak najwyższej potędze x jest ujemny.\\ W związku z tym wykres wielomianu zaczyna się od lewej strony powyżej osi OX. A więc $$x \in (-\infty,-17) \cup (-5,9).$$
\rozwStop
\odpStart
$x \in (-\infty,-17) \cup (-5,9)$
\odpStop
\testStart
A.$x \in (-\infty,-17) \cup (-5,9)$\\
B.$x \in (-\infty,-17) \cup (-5,9]$\\
C.$x \in (-\infty,-17) \cup [-5,9)$\\
D.$x \in (-\infty,-17] \cup (-5,9)$\\
E.$x \in (-\infty,-17] \cup (-5,9]$\\
F.$x \in (-\infty,-17] \cup [-5,9)$\\
G.$x \in (-\infty,-17) \cup [-5,9]$\\
H.$x \in (-\infty,-17] \cup [-5,9]$
\testStop
\kluczStart
A
\kluczStop



\zadStart{Zadanie z Wikieł Z 1.62 b) moja wersja nr 594}

Rozwiązać nierówności $(x+17)(9-x)(x+6)\ge0$.
\zadStop
\rozwStart{Patryk Wirkus}{}
Miejsca zerowe naszego wielomianu to: $-17, 9, -6$.\\
Wielomian jest stopnia nieparzystego, ponadto znak współczynnika przy\linebreak najwyższej potędze x jest ujemny.\\ W związku z tym wykres wielomianu zaczyna się od lewej strony powyżej osi OX. A więc $$x \in (-\infty,-17) \cup (-6,9).$$
\rozwStop
\odpStart
$x \in (-\infty,-17) \cup (-6,9)$
\odpStop
\testStart
A.$x \in (-\infty,-17) \cup (-6,9)$\\
B.$x \in (-\infty,-17) \cup (-6,9]$\\
C.$x \in (-\infty,-17) \cup [-6,9)$\\
D.$x \in (-\infty,-17] \cup (-6,9)$\\
E.$x \in (-\infty,-17] \cup (-6,9]$\\
F.$x \in (-\infty,-17] \cup [-6,9)$\\
G.$x \in (-\infty,-17) \cup [-6,9]$\\
H.$x \in (-\infty,-17] \cup [-6,9]$
\testStop
\kluczStart
A
\kluczStop



\zadStart{Zadanie z Wikieł Z 1.62 b) moja wersja nr 595}

Rozwiązać nierówności $(x+17)(9-x)(x+7)\ge0$.
\zadStop
\rozwStart{Patryk Wirkus}{}
Miejsca zerowe naszego wielomianu to: $-17, 9, -7$.\\
Wielomian jest stopnia nieparzystego, ponadto znak współczynnika przy\linebreak najwyższej potędze x jest ujemny.\\ W związku z tym wykres wielomianu zaczyna się od lewej strony powyżej osi OX. A więc $$x \in (-\infty,-17) \cup (-7,9).$$
\rozwStop
\odpStart
$x \in (-\infty,-17) \cup (-7,9)$
\odpStop
\testStart
A.$x \in (-\infty,-17) \cup (-7,9)$\\
B.$x \in (-\infty,-17) \cup (-7,9]$\\
C.$x \in (-\infty,-17) \cup [-7,9)$\\
D.$x \in (-\infty,-17] \cup (-7,9)$\\
E.$x \in (-\infty,-17] \cup (-7,9]$\\
F.$x \in (-\infty,-17] \cup [-7,9)$\\
G.$x \in (-\infty,-17) \cup [-7,9]$\\
H.$x \in (-\infty,-17] \cup [-7,9]$
\testStop
\kluczStart
A
\kluczStop



\zadStart{Zadanie z Wikieł Z 1.62 b) moja wersja nr 596}

Rozwiązać nierówności $(x+17)(9-x)(x+8)\ge0$.
\zadStop
\rozwStart{Patryk Wirkus}{}
Miejsca zerowe naszego wielomianu to: $-17, 9, -8$.\\
Wielomian jest stopnia nieparzystego, ponadto znak współczynnika przy\linebreak najwyższej potędze x jest ujemny.\\ W związku z tym wykres wielomianu zaczyna się od lewej strony powyżej osi OX. A więc $$x \in (-\infty,-17) \cup (-8,9).$$
\rozwStop
\odpStart
$x \in (-\infty,-17) \cup (-8,9)$
\odpStop
\testStart
A.$x \in (-\infty,-17) \cup (-8,9)$\\
B.$x \in (-\infty,-17) \cup (-8,9]$\\
C.$x \in (-\infty,-17) \cup [-8,9)$\\
D.$x \in (-\infty,-17] \cup (-8,9)$\\
E.$x \in (-\infty,-17] \cup (-8,9]$\\
F.$x \in (-\infty,-17] \cup [-8,9)$\\
G.$x \in (-\infty,-17) \cup [-8,9]$\\
H.$x \in (-\infty,-17] \cup [-8,9]$
\testStop
\kluczStart
A
\kluczStop



\zadStart{Zadanie z Wikieł Z 1.62 b) moja wersja nr 597}

Rozwiązać nierówności $(x+17)(10-x)(x+1)\ge0$.
\zadStop
\rozwStart{Patryk Wirkus}{}
Miejsca zerowe naszego wielomianu to: $-17, 10, -1$.\\
Wielomian jest stopnia nieparzystego, ponadto znak współczynnika przy\linebreak najwyższej potędze x jest ujemny.\\ W związku z tym wykres wielomianu zaczyna się od lewej strony powyżej osi OX. A więc $$x \in (-\infty,-17) \cup (-1,10).$$
\rozwStop
\odpStart
$x \in (-\infty,-17) \cup (-1,10)$
\odpStop
\testStart
A.$x \in (-\infty,-17) \cup (-1,10)$\\
B.$x \in (-\infty,-17) \cup (-1,10]$\\
C.$x \in (-\infty,-17) \cup [-1,10)$\\
D.$x \in (-\infty,-17] \cup (-1,10)$\\
E.$x \in (-\infty,-17] \cup (-1,10]$\\
F.$x \in (-\infty,-17] \cup [-1,10)$\\
G.$x \in (-\infty,-17) \cup [-1,10]$\\
H.$x \in (-\infty,-17] \cup [-1,10]$
\testStop
\kluczStart
A
\kluczStop



\zadStart{Zadanie z Wikieł Z 1.62 b) moja wersja nr 598}

Rozwiązać nierówności $(x+17)(10-x)(x+2)\ge0$.
\zadStop
\rozwStart{Patryk Wirkus}{}
Miejsca zerowe naszego wielomianu to: $-17, 10, -2$.\\
Wielomian jest stopnia nieparzystego, ponadto znak współczynnika przy\linebreak najwyższej potędze x jest ujemny.\\ W związku z tym wykres wielomianu zaczyna się od lewej strony powyżej osi OX. A więc $$x \in (-\infty,-17) \cup (-2,10).$$
\rozwStop
\odpStart
$x \in (-\infty,-17) \cup (-2,10)$
\odpStop
\testStart
A.$x \in (-\infty,-17) \cup (-2,10)$\\
B.$x \in (-\infty,-17) \cup (-2,10]$\\
C.$x \in (-\infty,-17) \cup [-2,10)$\\
D.$x \in (-\infty,-17] \cup (-2,10)$\\
E.$x \in (-\infty,-17] \cup (-2,10]$\\
F.$x \in (-\infty,-17] \cup [-2,10)$\\
G.$x \in (-\infty,-17) \cup [-2,10]$\\
H.$x \in (-\infty,-17] \cup [-2,10]$
\testStop
\kluczStart
A
\kluczStop



\zadStart{Zadanie z Wikieł Z 1.62 b) moja wersja nr 599}

Rozwiązać nierówności $(x+17)(10-x)(x+3)\ge0$.
\zadStop
\rozwStart{Patryk Wirkus}{}
Miejsca zerowe naszego wielomianu to: $-17, 10, -3$.\\
Wielomian jest stopnia nieparzystego, ponadto znak współczynnika przy\linebreak najwyższej potędze x jest ujemny.\\ W związku z tym wykres wielomianu zaczyna się od lewej strony powyżej osi OX. A więc $$x \in (-\infty,-17) \cup (-3,10).$$
\rozwStop
\odpStart
$x \in (-\infty,-17) \cup (-3,10)$
\odpStop
\testStart
A.$x \in (-\infty,-17) \cup (-3,10)$\\
B.$x \in (-\infty,-17) \cup (-3,10]$\\
C.$x \in (-\infty,-17) \cup [-3,10)$\\
D.$x \in (-\infty,-17] \cup (-3,10)$\\
E.$x \in (-\infty,-17] \cup (-3,10]$\\
F.$x \in (-\infty,-17] \cup [-3,10)$\\
G.$x \in (-\infty,-17) \cup [-3,10]$\\
H.$x \in (-\infty,-17] \cup [-3,10]$
\testStop
\kluczStart
A
\kluczStop



\zadStart{Zadanie z Wikieł Z 1.62 b) moja wersja nr 600}

Rozwiązać nierówności $(x+17)(10-x)(x+4)\ge0$.
\zadStop
\rozwStart{Patryk Wirkus}{}
Miejsca zerowe naszego wielomianu to: $-17, 10, -4$.\\
Wielomian jest stopnia nieparzystego, ponadto znak współczynnika przy\linebreak najwyższej potędze x jest ujemny.\\ W związku z tym wykres wielomianu zaczyna się od lewej strony powyżej osi OX. A więc $$x \in (-\infty,-17) \cup (-4,10).$$
\rozwStop
\odpStart
$x \in (-\infty,-17) \cup (-4,10)$
\odpStop
\testStart
A.$x \in (-\infty,-17) \cup (-4,10)$\\
B.$x \in (-\infty,-17) \cup (-4,10]$\\
C.$x \in (-\infty,-17) \cup [-4,10)$\\
D.$x \in (-\infty,-17] \cup (-4,10)$\\
E.$x \in (-\infty,-17] \cup (-4,10]$\\
F.$x \in (-\infty,-17] \cup [-4,10)$\\
G.$x \in (-\infty,-17) \cup [-4,10]$\\
H.$x \in (-\infty,-17] \cup [-4,10]$
\testStop
\kluczStart
A
\kluczStop



\zadStart{Zadanie z Wikieł Z 1.62 b) moja wersja nr 601}

Rozwiązać nierówności $(x+17)(10-x)(x+5)\ge0$.
\zadStop
\rozwStart{Patryk Wirkus}{}
Miejsca zerowe naszego wielomianu to: $-17, 10, -5$.\\
Wielomian jest stopnia nieparzystego, ponadto znak współczynnika przy\linebreak najwyższej potędze x jest ujemny.\\ W związku z tym wykres wielomianu zaczyna się od lewej strony powyżej osi OX. A więc $$x \in (-\infty,-17) \cup (-5,10).$$
\rozwStop
\odpStart
$x \in (-\infty,-17) \cup (-5,10)$
\odpStop
\testStart
A.$x \in (-\infty,-17) \cup (-5,10)$\\
B.$x \in (-\infty,-17) \cup (-5,10]$\\
C.$x \in (-\infty,-17) \cup [-5,10)$\\
D.$x \in (-\infty,-17] \cup (-5,10)$\\
E.$x \in (-\infty,-17] \cup (-5,10]$\\
F.$x \in (-\infty,-17] \cup [-5,10)$\\
G.$x \in (-\infty,-17) \cup [-5,10]$\\
H.$x \in (-\infty,-17] \cup [-5,10]$
\testStop
\kluczStart
A
\kluczStop



\zadStart{Zadanie z Wikieł Z 1.62 b) moja wersja nr 602}

Rozwiązać nierówności $(x+17)(10-x)(x+6)\ge0$.
\zadStop
\rozwStart{Patryk Wirkus}{}
Miejsca zerowe naszego wielomianu to: $-17, 10, -6$.\\
Wielomian jest stopnia nieparzystego, ponadto znak współczynnika przy\linebreak najwyższej potędze x jest ujemny.\\ W związku z tym wykres wielomianu zaczyna się od lewej strony powyżej osi OX. A więc $$x \in (-\infty,-17) \cup (-6,10).$$
\rozwStop
\odpStart
$x \in (-\infty,-17) \cup (-6,10)$
\odpStop
\testStart
A.$x \in (-\infty,-17) \cup (-6,10)$\\
B.$x \in (-\infty,-17) \cup (-6,10]$\\
C.$x \in (-\infty,-17) \cup [-6,10)$\\
D.$x \in (-\infty,-17] \cup (-6,10)$\\
E.$x \in (-\infty,-17] \cup (-6,10]$\\
F.$x \in (-\infty,-17] \cup [-6,10)$\\
G.$x \in (-\infty,-17) \cup [-6,10]$\\
H.$x \in (-\infty,-17] \cup [-6,10]$
\testStop
\kluczStart
A
\kluczStop



\zadStart{Zadanie z Wikieł Z 1.62 b) moja wersja nr 603}

Rozwiązać nierówności $(x+17)(10-x)(x+7)\ge0$.
\zadStop
\rozwStart{Patryk Wirkus}{}
Miejsca zerowe naszego wielomianu to: $-17, 10, -7$.\\
Wielomian jest stopnia nieparzystego, ponadto znak współczynnika przy\linebreak najwyższej potędze x jest ujemny.\\ W związku z tym wykres wielomianu zaczyna się od lewej strony powyżej osi OX. A więc $$x \in (-\infty,-17) \cup (-7,10).$$
\rozwStop
\odpStart
$x \in (-\infty,-17) \cup (-7,10)$
\odpStop
\testStart
A.$x \in (-\infty,-17) \cup (-7,10)$\\
B.$x \in (-\infty,-17) \cup (-7,10]$\\
C.$x \in (-\infty,-17) \cup [-7,10)$\\
D.$x \in (-\infty,-17] \cup (-7,10)$\\
E.$x \in (-\infty,-17] \cup (-7,10]$\\
F.$x \in (-\infty,-17] \cup [-7,10)$\\
G.$x \in (-\infty,-17) \cup [-7,10]$\\
H.$x \in (-\infty,-17] \cup [-7,10]$
\testStop
\kluczStart
A
\kluczStop



\zadStart{Zadanie z Wikieł Z 1.62 b) moja wersja nr 604}

Rozwiązać nierówności $(x+17)(10-x)(x+8)\ge0$.
\zadStop
\rozwStart{Patryk Wirkus}{}
Miejsca zerowe naszego wielomianu to: $-17, 10, -8$.\\
Wielomian jest stopnia nieparzystego, ponadto znak współczynnika przy\linebreak najwyższej potędze x jest ujemny.\\ W związku z tym wykres wielomianu zaczyna się od lewej strony powyżej osi OX. A więc $$x \in (-\infty,-17) \cup (-8,10).$$
\rozwStop
\odpStart
$x \in (-\infty,-17) \cup (-8,10)$
\odpStop
\testStart
A.$x \in (-\infty,-17) \cup (-8,10)$\\
B.$x \in (-\infty,-17) \cup (-8,10]$\\
C.$x \in (-\infty,-17) \cup [-8,10)$\\
D.$x \in (-\infty,-17] \cup (-8,10)$\\
E.$x \in (-\infty,-17] \cup (-8,10]$\\
F.$x \in (-\infty,-17] \cup [-8,10)$\\
G.$x \in (-\infty,-17) \cup [-8,10]$\\
H.$x \in (-\infty,-17] \cup [-8,10]$
\testStop
\kluczStart
A
\kluczStop



\zadStart{Zadanie z Wikieł Z 1.62 b) moja wersja nr 605}

Rozwiązać nierówności $(x+17)(10-x)(x+9)\ge0$.
\zadStop
\rozwStart{Patryk Wirkus}{}
Miejsca zerowe naszego wielomianu to: $-17, 10, -9$.\\
Wielomian jest stopnia nieparzystego, ponadto znak współczynnika przy\linebreak najwyższej potędze x jest ujemny.\\ W związku z tym wykres wielomianu zaczyna się od lewej strony powyżej osi OX. A więc $$x \in (-\infty,-17) \cup (-9,10).$$
\rozwStop
\odpStart
$x \in (-\infty,-17) \cup (-9,10)$
\odpStop
\testStart
A.$x \in (-\infty,-17) \cup (-9,10)$\\
B.$x \in (-\infty,-17) \cup (-9,10]$\\
C.$x \in (-\infty,-17) \cup [-9,10)$\\
D.$x \in (-\infty,-17] \cup (-9,10)$\\
E.$x \in (-\infty,-17] \cup (-9,10]$\\
F.$x \in (-\infty,-17] \cup [-9,10)$\\
G.$x \in (-\infty,-17) \cup [-9,10]$\\
H.$x \in (-\infty,-17] \cup [-9,10]$
\testStop
\kluczStart
A
\kluczStop



\zadStart{Zadanie z Wikieł Z 1.62 b) moja wersja nr 606}

Rozwiązać nierówności $(x+17)(11-x)(x+1)\ge0$.
\zadStop
\rozwStart{Patryk Wirkus}{}
Miejsca zerowe naszego wielomianu to: $-17, 11, -1$.\\
Wielomian jest stopnia nieparzystego, ponadto znak współczynnika przy\linebreak najwyższej potędze x jest ujemny.\\ W związku z tym wykres wielomianu zaczyna się od lewej strony powyżej osi OX. A więc $$x \in (-\infty,-17) \cup (-1,11).$$
\rozwStop
\odpStart
$x \in (-\infty,-17) \cup (-1,11)$
\odpStop
\testStart
A.$x \in (-\infty,-17) \cup (-1,11)$\\
B.$x \in (-\infty,-17) \cup (-1,11]$\\
C.$x \in (-\infty,-17) \cup [-1,11)$\\
D.$x \in (-\infty,-17] \cup (-1,11)$\\
E.$x \in (-\infty,-17] \cup (-1,11]$\\
F.$x \in (-\infty,-17] \cup [-1,11)$\\
G.$x \in (-\infty,-17) \cup [-1,11]$\\
H.$x \in (-\infty,-17] \cup [-1,11]$
\testStop
\kluczStart
A
\kluczStop



\zadStart{Zadanie z Wikieł Z 1.62 b) moja wersja nr 607}

Rozwiązać nierówności $(x+17)(11-x)(x+2)\ge0$.
\zadStop
\rozwStart{Patryk Wirkus}{}
Miejsca zerowe naszego wielomianu to: $-17, 11, -2$.\\
Wielomian jest stopnia nieparzystego, ponadto znak współczynnika przy\linebreak najwyższej potędze x jest ujemny.\\ W związku z tym wykres wielomianu zaczyna się od lewej strony powyżej osi OX. A więc $$x \in (-\infty,-17) \cup (-2,11).$$
\rozwStop
\odpStart
$x \in (-\infty,-17) \cup (-2,11)$
\odpStop
\testStart
A.$x \in (-\infty,-17) \cup (-2,11)$\\
B.$x \in (-\infty,-17) \cup (-2,11]$\\
C.$x \in (-\infty,-17) \cup [-2,11)$\\
D.$x \in (-\infty,-17] \cup (-2,11)$\\
E.$x \in (-\infty,-17] \cup (-2,11]$\\
F.$x \in (-\infty,-17] \cup [-2,11)$\\
G.$x \in (-\infty,-17) \cup [-2,11]$\\
H.$x \in (-\infty,-17] \cup [-2,11]$
\testStop
\kluczStart
A
\kluczStop



\zadStart{Zadanie z Wikieł Z 1.62 b) moja wersja nr 608}

Rozwiązać nierówności $(x+17)(11-x)(x+3)\ge0$.
\zadStop
\rozwStart{Patryk Wirkus}{}
Miejsca zerowe naszego wielomianu to: $-17, 11, -3$.\\
Wielomian jest stopnia nieparzystego, ponadto znak współczynnika przy\linebreak najwyższej potędze x jest ujemny.\\ W związku z tym wykres wielomianu zaczyna się od lewej strony powyżej osi OX. A więc $$x \in (-\infty,-17) \cup (-3,11).$$
\rozwStop
\odpStart
$x \in (-\infty,-17) \cup (-3,11)$
\odpStop
\testStart
A.$x \in (-\infty,-17) \cup (-3,11)$\\
B.$x \in (-\infty,-17) \cup (-3,11]$\\
C.$x \in (-\infty,-17) \cup [-3,11)$\\
D.$x \in (-\infty,-17] \cup (-3,11)$\\
E.$x \in (-\infty,-17] \cup (-3,11]$\\
F.$x \in (-\infty,-17] \cup [-3,11)$\\
G.$x \in (-\infty,-17) \cup [-3,11]$\\
H.$x \in (-\infty,-17] \cup [-3,11]$
\testStop
\kluczStart
A
\kluczStop



\zadStart{Zadanie z Wikieł Z 1.62 b) moja wersja nr 609}

Rozwiązać nierówności $(x+17)(11-x)(x+4)\ge0$.
\zadStop
\rozwStart{Patryk Wirkus}{}
Miejsca zerowe naszego wielomianu to: $-17, 11, -4$.\\
Wielomian jest stopnia nieparzystego, ponadto znak współczynnika przy\linebreak najwyższej potędze x jest ujemny.\\ W związku z tym wykres wielomianu zaczyna się od lewej strony powyżej osi OX. A więc $$x \in (-\infty,-17) \cup (-4,11).$$
\rozwStop
\odpStart
$x \in (-\infty,-17) \cup (-4,11)$
\odpStop
\testStart
A.$x \in (-\infty,-17) \cup (-4,11)$\\
B.$x \in (-\infty,-17) \cup (-4,11]$\\
C.$x \in (-\infty,-17) \cup [-4,11)$\\
D.$x \in (-\infty,-17] \cup (-4,11)$\\
E.$x \in (-\infty,-17] \cup (-4,11]$\\
F.$x \in (-\infty,-17] \cup [-4,11)$\\
G.$x \in (-\infty,-17) \cup [-4,11]$\\
H.$x \in (-\infty,-17] \cup [-4,11]$
\testStop
\kluczStart
A
\kluczStop



\zadStart{Zadanie z Wikieł Z 1.62 b) moja wersja nr 610}

Rozwiązać nierówności $(x+17)(11-x)(x+5)\ge0$.
\zadStop
\rozwStart{Patryk Wirkus}{}
Miejsca zerowe naszego wielomianu to: $-17, 11, -5$.\\
Wielomian jest stopnia nieparzystego, ponadto znak współczynnika przy\linebreak najwyższej potędze x jest ujemny.\\ W związku z tym wykres wielomianu zaczyna się od lewej strony powyżej osi OX. A więc $$x \in (-\infty,-17) \cup (-5,11).$$
\rozwStop
\odpStart
$x \in (-\infty,-17) \cup (-5,11)$
\odpStop
\testStart
A.$x \in (-\infty,-17) \cup (-5,11)$\\
B.$x \in (-\infty,-17) \cup (-5,11]$\\
C.$x \in (-\infty,-17) \cup [-5,11)$\\
D.$x \in (-\infty,-17] \cup (-5,11)$\\
E.$x \in (-\infty,-17] \cup (-5,11]$\\
F.$x \in (-\infty,-17] \cup [-5,11)$\\
G.$x \in (-\infty,-17) \cup [-5,11]$\\
H.$x \in (-\infty,-17] \cup [-5,11]$
\testStop
\kluczStart
A
\kluczStop



\zadStart{Zadanie z Wikieł Z 1.62 b) moja wersja nr 611}

Rozwiązać nierówności $(x+17)(11-x)(x+6)\ge0$.
\zadStop
\rozwStart{Patryk Wirkus}{}
Miejsca zerowe naszego wielomianu to: $-17, 11, -6$.\\
Wielomian jest stopnia nieparzystego, ponadto znak współczynnika przy\linebreak najwyższej potędze x jest ujemny.\\ W związku z tym wykres wielomianu zaczyna się od lewej strony powyżej osi OX. A więc $$x \in (-\infty,-17) \cup (-6,11).$$
\rozwStop
\odpStart
$x \in (-\infty,-17) \cup (-6,11)$
\odpStop
\testStart
A.$x \in (-\infty,-17) \cup (-6,11)$\\
B.$x \in (-\infty,-17) \cup (-6,11]$\\
C.$x \in (-\infty,-17) \cup [-6,11)$\\
D.$x \in (-\infty,-17] \cup (-6,11)$\\
E.$x \in (-\infty,-17] \cup (-6,11]$\\
F.$x \in (-\infty,-17] \cup [-6,11)$\\
G.$x \in (-\infty,-17) \cup [-6,11]$\\
H.$x \in (-\infty,-17] \cup [-6,11]$
\testStop
\kluczStart
A
\kluczStop



\zadStart{Zadanie z Wikieł Z 1.62 b) moja wersja nr 612}

Rozwiązać nierówności $(x+17)(11-x)(x+7)\ge0$.
\zadStop
\rozwStart{Patryk Wirkus}{}
Miejsca zerowe naszego wielomianu to: $-17, 11, -7$.\\
Wielomian jest stopnia nieparzystego, ponadto znak współczynnika przy\linebreak najwyższej potędze x jest ujemny.\\ W związku z tym wykres wielomianu zaczyna się od lewej strony powyżej osi OX. A więc $$x \in (-\infty,-17) \cup (-7,11).$$
\rozwStop
\odpStart
$x \in (-\infty,-17) \cup (-7,11)$
\odpStop
\testStart
A.$x \in (-\infty,-17) \cup (-7,11)$\\
B.$x \in (-\infty,-17) \cup (-7,11]$\\
C.$x \in (-\infty,-17) \cup [-7,11)$\\
D.$x \in (-\infty,-17] \cup (-7,11)$\\
E.$x \in (-\infty,-17] \cup (-7,11]$\\
F.$x \in (-\infty,-17] \cup [-7,11)$\\
G.$x \in (-\infty,-17) \cup [-7,11]$\\
H.$x \in (-\infty,-17] \cup [-7,11]$
\testStop
\kluczStart
A
\kluczStop



\zadStart{Zadanie z Wikieł Z 1.62 b) moja wersja nr 613}

Rozwiązać nierówności $(x+17)(11-x)(x+8)\ge0$.
\zadStop
\rozwStart{Patryk Wirkus}{}
Miejsca zerowe naszego wielomianu to: $-17, 11, -8$.\\
Wielomian jest stopnia nieparzystego, ponadto znak współczynnika przy\linebreak najwyższej potędze x jest ujemny.\\ W związku z tym wykres wielomianu zaczyna się od lewej strony powyżej osi OX. A więc $$x \in (-\infty,-17) \cup (-8,11).$$
\rozwStop
\odpStart
$x \in (-\infty,-17) \cup (-8,11)$
\odpStop
\testStart
A.$x \in (-\infty,-17) \cup (-8,11)$\\
B.$x \in (-\infty,-17) \cup (-8,11]$\\
C.$x \in (-\infty,-17) \cup [-8,11)$\\
D.$x \in (-\infty,-17] \cup (-8,11)$\\
E.$x \in (-\infty,-17] \cup (-8,11]$\\
F.$x \in (-\infty,-17] \cup [-8,11)$\\
G.$x \in (-\infty,-17) \cup [-8,11]$\\
H.$x \in (-\infty,-17] \cup [-8,11]$
\testStop
\kluczStart
A
\kluczStop



\zadStart{Zadanie z Wikieł Z 1.62 b) moja wersja nr 614}

Rozwiązać nierówności $(x+17)(11-x)(x+9)\ge0$.
\zadStop
\rozwStart{Patryk Wirkus}{}
Miejsca zerowe naszego wielomianu to: $-17, 11, -9$.\\
Wielomian jest stopnia nieparzystego, ponadto znak współczynnika przy\linebreak najwyższej potędze x jest ujemny.\\ W związku z tym wykres wielomianu zaczyna się od lewej strony powyżej osi OX. A więc $$x \in (-\infty,-17) \cup (-9,11).$$
\rozwStop
\odpStart
$x \in (-\infty,-17) \cup (-9,11)$
\odpStop
\testStart
A.$x \in (-\infty,-17) \cup (-9,11)$\\
B.$x \in (-\infty,-17) \cup (-9,11]$\\
C.$x \in (-\infty,-17) \cup [-9,11)$\\
D.$x \in (-\infty,-17] \cup (-9,11)$\\
E.$x \in (-\infty,-17] \cup (-9,11]$\\
F.$x \in (-\infty,-17] \cup [-9,11)$\\
G.$x \in (-\infty,-17) \cup [-9,11]$\\
H.$x \in (-\infty,-17] \cup [-9,11]$
\testStop
\kluczStart
A
\kluczStop



\zadStart{Zadanie z Wikieł Z 1.62 b) moja wersja nr 615}

Rozwiązać nierówności $(x+17)(11-x)(x+10)\ge0$.
\zadStop
\rozwStart{Patryk Wirkus}{}
Miejsca zerowe naszego wielomianu to: $-17, 11, -10$.\\
Wielomian jest stopnia nieparzystego, ponadto znak współczynnika przy\linebreak najwyższej potędze x jest ujemny.\\ W związku z tym wykres wielomianu zaczyna się od lewej strony powyżej osi OX. A więc $$x \in (-\infty,-17) \cup (-10,11).$$
\rozwStop
\odpStart
$x \in (-\infty,-17) \cup (-10,11)$
\odpStop
\testStart
A.$x \in (-\infty,-17) \cup (-10,11)$\\
B.$x \in (-\infty,-17) \cup (-10,11]$\\
C.$x \in (-\infty,-17) \cup [-10,11)$\\
D.$x \in (-\infty,-17] \cup (-10,11)$\\
E.$x \in (-\infty,-17] \cup (-10,11]$\\
F.$x \in (-\infty,-17] \cup [-10,11)$\\
G.$x \in (-\infty,-17) \cup [-10,11]$\\
H.$x \in (-\infty,-17] \cup [-10,11]$
\testStop
\kluczStart
A
\kluczStop



\zadStart{Zadanie z Wikieł Z 1.62 b) moja wersja nr 616}

Rozwiązać nierówności $(x+17)(12-x)(x+1)\ge0$.
\zadStop
\rozwStart{Patryk Wirkus}{}
Miejsca zerowe naszego wielomianu to: $-17, 12, -1$.\\
Wielomian jest stopnia nieparzystego, ponadto znak współczynnika przy\linebreak najwyższej potędze x jest ujemny.\\ W związku z tym wykres wielomianu zaczyna się od lewej strony powyżej osi OX. A więc $$x \in (-\infty,-17) \cup (-1,12).$$
\rozwStop
\odpStart
$x \in (-\infty,-17) \cup (-1,12)$
\odpStop
\testStart
A.$x \in (-\infty,-17) \cup (-1,12)$\\
B.$x \in (-\infty,-17) \cup (-1,12]$\\
C.$x \in (-\infty,-17) \cup [-1,12)$\\
D.$x \in (-\infty,-17] \cup (-1,12)$\\
E.$x \in (-\infty,-17] \cup (-1,12]$\\
F.$x \in (-\infty,-17] \cup [-1,12)$\\
G.$x \in (-\infty,-17) \cup [-1,12]$\\
H.$x \in (-\infty,-17] \cup [-1,12]$
\testStop
\kluczStart
A
\kluczStop



\zadStart{Zadanie z Wikieł Z 1.62 b) moja wersja nr 617}

Rozwiązać nierówności $(x+17)(12-x)(x+2)\ge0$.
\zadStop
\rozwStart{Patryk Wirkus}{}
Miejsca zerowe naszego wielomianu to: $-17, 12, -2$.\\
Wielomian jest stopnia nieparzystego, ponadto znak współczynnika przy\linebreak najwyższej potędze x jest ujemny.\\ W związku z tym wykres wielomianu zaczyna się od lewej strony powyżej osi OX. A więc $$x \in (-\infty,-17) \cup (-2,12).$$
\rozwStop
\odpStart
$x \in (-\infty,-17) \cup (-2,12)$
\odpStop
\testStart
A.$x \in (-\infty,-17) \cup (-2,12)$\\
B.$x \in (-\infty,-17) \cup (-2,12]$\\
C.$x \in (-\infty,-17) \cup [-2,12)$\\
D.$x \in (-\infty,-17] \cup (-2,12)$\\
E.$x \in (-\infty,-17] \cup (-2,12]$\\
F.$x \in (-\infty,-17] \cup [-2,12)$\\
G.$x \in (-\infty,-17) \cup [-2,12]$\\
H.$x \in (-\infty,-17] \cup [-2,12]$
\testStop
\kluczStart
A
\kluczStop



\zadStart{Zadanie z Wikieł Z 1.62 b) moja wersja nr 618}

Rozwiązać nierówności $(x+17)(12-x)(x+3)\ge0$.
\zadStop
\rozwStart{Patryk Wirkus}{}
Miejsca zerowe naszego wielomianu to: $-17, 12, -3$.\\
Wielomian jest stopnia nieparzystego, ponadto znak współczynnika przy\linebreak najwyższej potędze x jest ujemny.\\ W związku z tym wykres wielomianu zaczyna się od lewej strony powyżej osi OX. A więc $$x \in (-\infty,-17) \cup (-3,12).$$
\rozwStop
\odpStart
$x \in (-\infty,-17) \cup (-3,12)$
\odpStop
\testStart
A.$x \in (-\infty,-17) \cup (-3,12)$\\
B.$x \in (-\infty,-17) \cup (-3,12]$\\
C.$x \in (-\infty,-17) \cup [-3,12)$\\
D.$x \in (-\infty,-17] \cup (-3,12)$\\
E.$x \in (-\infty,-17] \cup (-3,12]$\\
F.$x \in (-\infty,-17] \cup [-3,12)$\\
G.$x \in (-\infty,-17) \cup [-3,12]$\\
H.$x \in (-\infty,-17] \cup [-3,12]$
\testStop
\kluczStart
A
\kluczStop



\zadStart{Zadanie z Wikieł Z 1.62 b) moja wersja nr 619}

Rozwiązać nierówności $(x+17)(12-x)(x+4)\ge0$.
\zadStop
\rozwStart{Patryk Wirkus}{}
Miejsca zerowe naszego wielomianu to: $-17, 12, -4$.\\
Wielomian jest stopnia nieparzystego, ponadto znak współczynnika przy\linebreak najwyższej potędze x jest ujemny.\\ W związku z tym wykres wielomianu zaczyna się od lewej strony powyżej osi OX. A więc $$x \in (-\infty,-17) \cup (-4,12).$$
\rozwStop
\odpStart
$x \in (-\infty,-17) \cup (-4,12)$
\odpStop
\testStart
A.$x \in (-\infty,-17) \cup (-4,12)$\\
B.$x \in (-\infty,-17) \cup (-4,12]$\\
C.$x \in (-\infty,-17) \cup [-4,12)$\\
D.$x \in (-\infty,-17] \cup (-4,12)$\\
E.$x \in (-\infty,-17] \cup (-4,12]$\\
F.$x \in (-\infty,-17] \cup [-4,12)$\\
G.$x \in (-\infty,-17) \cup [-4,12]$\\
H.$x \in (-\infty,-17] \cup [-4,12]$
\testStop
\kluczStart
A
\kluczStop



\zadStart{Zadanie z Wikieł Z 1.62 b) moja wersja nr 620}

Rozwiązać nierówności $(x+17)(12-x)(x+5)\ge0$.
\zadStop
\rozwStart{Patryk Wirkus}{}
Miejsca zerowe naszego wielomianu to: $-17, 12, -5$.\\
Wielomian jest stopnia nieparzystego, ponadto znak współczynnika przy\linebreak najwyższej potędze x jest ujemny.\\ W związku z tym wykres wielomianu zaczyna się od lewej strony powyżej osi OX. A więc $$x \in (-\infty,-17) \cup (-5,12).$$
\rozwStop
\odpStart
$x \in (-\infty,-17) \cup (-5,12)$
\odpStop
\testStart
A.$x \in (-\infty,-17) \cup (-5,12)$\\
B.$x \in (-\infty,-17) \cup (-5,12]$\\
C.$x \in (-\infty,-17) \cup [-5,12)$\\
D.$x \in (-\infty,-17] \cup (-5,12)$\\
E.$x \in (-\infty,-17] \cup (-5,12]$\\
F.$x \in (-\infty,-17] \cup [-5,12)$\\
G.$x \in (-\infty,-17) \cup [-5,12]$\\
H.$x \in (-\infty,-17] \cup [-5,12]$
\testStop
\kluczStart
A
\kluczStop



\zadStart{Zadanie z Wikieł Z 1.62 b) moja wersja nr 621}

Rozwiązać nierówności $(x+17)(12-x)(x+6)\ge0$.
\zadStop
\rozwStart{Patryk Wirkus}{}
Miejsca zerowe naszego wielomianu to: $-17, 12, -6$.\\
Wielomian jest stopnia nieparzystego, ponadto znak współczynnika przy\linebreak najwyższej potędze x jest ujemny.\\ W związku z tym wykres wielomianu zaczyna się od lewej strony powyżej osi OX. A więc $$x \in (-\infty,-17) \cup (-6,12).$$
\rozwStop
\odpStart
$x \in (-\infty,-17) \cup (-6,12)$
\odpStop
\testStart
A.$x \in (-\infty,-17) \cup (-6,12)$\\
B.$x \in (-\infty,-17) \cup (-6,12]$\\
C.$x \in (-\infty,-17) \cup [-6,12)$\\
D.$x \in (-\infty,-17] \cup (-6,12)$\\
E.$x \in (-\infty,-17] \cup (-6,12]$\\
F.$x \in (-\infty,-17] \cup [-6,12)$\\
G.$x \in (-\infty,-17) \cup [-6,12]$\\
H.$x \in (-\infty,-17] \cup [-6,12]$
\testStop
\kluczStart
A
\kluczStop



\zadStart{Zadanie z Wikieł Z 1.62 b) moja wersja nr 622}

Rozwiązać nierówności $(x+17)(12-x)(x+7)\ge0$.
\zadStop
\rozwStart{Patryk Wirkus}{}
Miejsca zerowe naszego wielomianu to: $-17, 12, -7$.\\
Wielomian jest stopnia nieparzystego, ponadto znak współczynnika przy\linebreak najwyższej potędze x jest ujemny.\\ W związku z tym wykres wielomianu zaczyna się od lewej strony powyżej osi OX. A więc $$x \in (-\infty,-17) \cup (-7,12).$$
\rozwStop
\odpStart
$x \in (-\infty,-17) \cup (-7,12)$
\odpStop
\testStart
A.$x \in (-\infty,-17) \cup (-7,12)$\\
B.$x \in (-\infty,-17) \cup (-7,12]$\\
C.$x \in (-\infty,-17) \cup [-7,12)$\\
D.$x \in (-\infty,-17] \cup (-7,12)$\\
E.$x \in (-\infty,-17] \cup (-7,12]$\\
F.$x \in (-\infty,-17] \cup [-7,12)$\\
G.$x \in (-\infty,-17) \cup [-7,12]$\\
H.$x \in (-\infty,-17] \cup [-7,12]$
\testStop
\kluczStart
A
\kluczStop



\zadStart{Zadanie z Wikieł Z 1.62 b) moja wersja nr 623}

Rozwiązać nierówności $(x+17)(12-x)(x+8)\ge0$.
\zadStop
\rozwStart{Patryk Wirkus}{}
Miejsca zerowe naszego wielomianu to: $-17, 12, -8$.\\
Wielomian jest stopnia nieparzystego, ponadto znak współczynnika przy\linebreak najwyższej potędze x jest ujemny.\\ W związku z tym wykres wielomianu zaczyna się od lewej strony powyżej osi OX. A więc $$x \in (-\infty,-17) \cup (-8,12).$$
\rozwStop
\odpStart
$x \in (-\infty,-17) \cup (-8,12)$
\odpStop
\testStart
A.$x \in (-\infty,-17) \cup (-8,12)$\\
B.$x \in (-\infty,-17) \cup (-8,12]$\\
C.$x \in (-\infty,-17) \cup [-8,12)$\\
D.$x \in (-\infty,-17] \cup (-8,12)$\\
E.$x \in (-\infty,-17] \cup (-8,12]$\\
F.$x \in (-\infty,-17] \cup [-8,12)$\\
G.$x \in (-\infty,-17) \cup [-8,12]$\\
H.$x \in (-\infty,-17] \cup [-8,12]$
\testStop
\kluczStart
A
\kluczStop



\zadStart{Zadanie z Wikieł Z 1.62 b) moja wersja nr 624}

Rozwiązać nierówności $(x+17)(12-x)(x+9)\ge0$.
\zadStop
\rozwStart{Patryk Wirkus}{}
Miejsca zerowe naszego wielomianu to: $-17, 12, -9$.\\
Wielomian jest stopnia nieparzystego, ponadto znak współczynnika przy\linebreak najwyższej potędze x jest ujemny.\\ W związku z tym wykres wielomianu zaczyna się od lewej strony powyżej osi OX. A więc $$x \in (-\infty,-17) \cup (-9,12).$$
\rozwStop
\odpStart
$x \in (-\infty,-17) \cup (-9,12)$
\odpStop
\testStart
A.$x \in (-\infty,-17) \cup (-9,12)$\\
B.$x \in (-\infty,-17) \cup (-9,12]$\\
C.$x \in (-\infty,-17) \cup [-9,12)$\\
D.$x \in (-\infty,-17] \cup (-9,12)$\\
E.$x \in (-\infty,-17] \cup (-9,12]$\\
F.$x \in (-\infty,-17] \cup [-9,12)$\\
G.$x \in (-\infty,-17) \cup [-9,12]$\\
H.$x \in (-\infty,-17] \cup [-9,12]$
\testStop
\kluczStart
A
\kluczStop



\zadStart{Zadanie z Wikieł Z 1.62 b) moja wersja nr 625}

Rozwiązać nierówności $(x+17)(12-x)(x+10)\ge0$.
\zadStop
\rozwStart{Patryk Wirkus}{}
Miejsca zerowe naszego wielomianu to: $-17, 12, -10$.\\
Wielomian jest stopnia nieparzystego, ponadto znak współczynnika przy\linebreak najwyższej potędze x jest ujemny.\\ W związku z tym wykres wielomianu zaczyna się od lewej strony powyżej osi OX. A więc $$x \in (-\infty,-17) \cup (-10,12).$$
\rozwStop
\odpStart
$x \in (-\infty,-17) \cup (-10,12)$
\odpStop
\testStart
A.$x \in (-\infty,-17) \cup (-10,12)$\\
B.$x \in (-\infty,-17) \cup (-10,12]$\\
C.$x \in (-\infty,-17) \cup [-10,12)$\\
D.$x \in (-\infty,-17] \cup (-10,12)$\\
E.$x \in (-\infty,-17] \cup (-10,12]$\\
F.$x \in (-\infty,-17] \cup [-10,12)$\\
G.$x \in (-\infty,-17) \cup [-10,12]$\\
H.$x \in (-\infty,-17] \cup [-10,12]$
\testStop
\kluczStart
A
\kluczStop



\zadStart{Zadanie z Wikieł Z 1.62 b) moja wersja nr 626}

Rozwiązać nierówności $(x+17)(12-x)(x+11)\ge0$.
\zadStop
\rozwStart{Patryk Wirkus}{}
Miejsca zerowe naszego wielomianu to: $-17, 12, -11$.\\
Wielomian jest stopnia nieparzystego, ponadto znak współczynnika przy\linebreak najwyższej potędze x jest ujemny.\\ W związku z tym wykres wielomianu zaczyna się od lewej strony powyżej osi OX. A więc $$x \in (-\infty,-17) \cup (-11,12).$$
\rozwStop
\odpStart
$x \in (-\infty,-17) \cup (-11,12)$
\odpStop
\testStart
A.$x \in (-\infty,-17) \cup (-11,12)$\\
B.$x \in (-\infty,-17) \cup (-11,12]$\\
C.$x \in (-\infty,-17) \cup [-11,12)$\\
D.$x \in (-\infty,-17] \cup (-11,12)$\\
E.$x \in (-\infty,-17] \cup (-11,12]$\\
F.$x \in (-\infty,-17] \cup [-11,12)$\\
G.$x \in (-\infty,-17) \cup [-11,12]$\\
H.$x \in (-\infty,-17] \cup [-11,12]$
\testStop
\kluczStart
A
\kluczStop



\zadStart{Zadanie z Wikieł Z 1.62 b) moja wersja nr 627}

Rozwiązać nierówności $(x+17)(13-x)(x+1)\ge0$.
\zadStop
\rozwStart{Patryk Wirkus}{}
Miejsca zerowe naszego wielomianu to: $-17, 13, -1$.\\
Wielomian jest stopnia nieparzystego, ponadto znak współczynnika przy\linebreak najwyższej potędze x jest ujemny.\\ W związku z tym wykres wielomianu zaczyna się od lewej strony powyżej osi OX. A więc $$x \in (-\infty,-17) \cup (-1,13).$$
\rozwStop
\odpStart
$x \in (-\infty,-17) \cup (-1,13)$
\odpStop
\testStart
A.$x \in (-\infty,-17) \cup (-1,13)$\\
B.$x \in (-\infty,-17) \cup (-1,13]$\\
C.$x \in (-\infty,-17) \cup [-1,13)$\\
D.$x \in (-\infty,-17] \cup (-1,13)$\\
E.$x \in (-\infty,-17] \cup (-1,13]$\\
F.$x \in (-\infty,-17] \cup [-1,13)$\\
G.$x \in (-\infty,-17) \cup [-1,13]$\\
H.$x \in (-\infty,-17] \cup [-1,13]$
\testStop
\kluczStart
A
\kluczStop



\zadStart{Zadanie z Wikieł Z 1.62 b) moja wersja nr 628}

Rozwiązać nierówności $(x+17)(13-x)(x+2)\ge0$.
\zadStop
\rozwStart{Patryk Wirkus}{}
Miejsca zerowe naszego wielomianu to: $-17, 13, -2$.\\
Wielomian jest stopnia nieparzystego, ponadto znak współczynnika przy\linebreak najwyższej potędze x jest ujemny.\\ W związku z tym wykres wielomianu zaczyna się od lewej strony powyżej osi OX. A więc $$x \in (-\infty,-17) \cup (-2,13).$$
\rozwStop
\odpStart
$x \in (-\infty,-17) \cup (-2,13)$
\odpStop
\testStart
A.$x \in (-\infty,-17) \cup (-2,13)$\\
B.$x \in (-\infty,-17) \cup (-2,13]$\\
C.$x \in (-\infty,-17) \cup [-2,13)$\\
D.$x \in (-\infty,-17] \cup (-2,13)$\\
E.$x \in (-\infty,-17] \cup (-2,13]$\\
F.$x \in (-\infty,-17] \cup [-2,13)$\\
G.$x \in (-\infty,-17) \cup [-2,13]$\\
H.$x \in (-\infty,-17] \cup [-2,13]$
\testStop
\kluczStart
A
\kluczStop



\zadStart{Zadanie z Wikieł Z 1.62 b) moja wersja nr 629}

Rozwiązać nierówności $(x+17)(13-x)(x+3)\ge0$.
\zadStop
\rozwStart{Patryk Wirkus}{}
Miejsca zerowe naszego wielomianu to: $-17, 13, -3$.\\
Wielomian jest stopnia nieparzystego, ponadto znak współczynnika przy\linebreak najwyższej potędze x jest ujemny.\\ W związku z tym wykres wielomianu zaczyna się od lewej strony powyżej osi OX. A więc $$x \in (-\infty,-17) \cup (-3,13).$$
\rozwStop
\odpStart
$x \in (-\infty,-17) \cup (-3,13)$
\odpStop
\testStart
A.$x \in (-\infty,-17) \cup (-3,13)$\\
B.$x \in (-\infty,-17) \cup (-3,13]$\\
C.$x \in (-\infty,-17) \cup [-3,13)$\\
D.$x \in (-\infty,-17] \cup (-3,13)$\\
E.$x \in (-\infty,-17] \cup (-3,13]$\\
F.$x \in (-\infty,-17] \cup [-3,13)$\\
G.$x \in (-\infty,-17) \cup [-3,13]$\\
H.$x \in (-\infty,-17] \cup [-3,13]$
\testStop
\kluczStart
A
\kluczStop



\zadStart{Zadanie z Wikieł Z 1.62 b) moja wersja nr 630}

Rozwiązać nierówności $(x+17)(13-x)(x+4)\ge0$.
\zadStop
\rozwStart{Patryk Wirkus}{}
Miejsca zerowe naszego wielomianu to: $-17, 13, -4$.\\
Wielomian jest stopnia nieparzystego, ponadto znak współczynnika przy\linebreak najwyższej potędze x jest ujemny.\\ W związku z tym wykres wielomianu zaczyna się od lewej strony powyżej osi OX. A więc $$x \in (-\infty,-17) \cup (-4,13).$$
\rozwStop
\odpStart
$x \in (-\infty,-17) \cup (-4,13)$
\odpStop
\testStart
A.$x \in (-\infty,-17) \cup (-4,13)$\\
B.$x \in (-\infty,-17) \cup (-4,13]$\\
C.$x \in (-\infty,-17) \cup [-4,13)$\\
D.$x \in (-\infty,-17] \cup (-4,13)$\\
E.$x \in (-\infty,-17] \cup (-4,13]$\\
F.$x \in (-\infty,-17] \cup [-4,13)$\\
G.$x \in (-\infty,-17) \cup [-4,13]$\\
H.$x \in (-\infty,-17] \cup [-4,13]$
\testStop
\kluczStart
A
\kluczStop



\zadStart{Zadanie z Wikieł Z 1.62 b) moja wersja nr 631}

Rozwiązać nierówności $(x+17)(13-x)(x+5)\ge0$.
\zadStop
\rozwStart{Patryk Wirkus}{}
Miejsca zerowe naszego wielomianu to: $-17, 13, -5$.\\
Wielomian jest stopnia nieparzystego, ponadto znak współczynnika przy\linebreak najwyższej potędze x jest ujemny.\\ W związku z tym wykres wielomianu zaczyna się od lewej strony powyżej osi OX. A więc $$x \in (-\infty,-17) \cup (-5,13).$$
\rozwStop
\odpStart
$x \in (-\infty,-17) \cup (-5,13)$
\odpStop
\testStart
A.$x \in (-\infty,-17) \cup (-5,13)$\\
B.$x \in (-\infty,-17) \cup (-5,13]$\\
C.$x \in (-\infty,-17) \cup [-5,13)$\\
D.$x \in (-\infty,-17] \cup (-5,13)$\\
E.$x \in (-\infty,-17] \cup (-5,13]$\\
F.$x \in (-\infty,-17] \cup [-5,13)$\\
G.$x \in (-\infty,-17) \cup [-5,13]$\\
H.$x \in (-\infty,-17] \cup [-5,13]$
\testStop
\kluczStart
A
\kluczStop



\zadStart{Zadanie z Wikieł Z 1.62 b) moja wersja nr 632}

Rozwiązać nierówności $(x+17)(13-x)(x+6)\ge0$.
\zadStop
\rozwStart{Patryk Wirkus}{}
Miejsca zerowe naszego wielomianu to: $-17, 13, -6$.\\
Wielomian jest stopnia nieparzystego, ponadto znak współczynnika przy\linebreak najwyższej potędze x jest ujemny.\\ W związku z tym wykres wielomianu zaczyna się od lewej strony powyżej osi OX. A więc $$x \in (-\infty,-17) \cup (-6,13).$$
\rozwStop
\odpStart
$x \in (-\infty,-17) \cup (-6,13)$
\odpStop
\testStart
A.$x \in (-\infty,-17) \cup (-6,13)$\\
B.$x \in (-\infty,-17) \cup (-6,13]$\\
C.$x \in (-\infty,-17) \cup [-6,13)$\\
D.$x \in (-\infty,-17] \cup (-6,13)$\\
E.$x \in (-\infty,-17] \cup (-6,13]$\\
F.$x \in (-\infty,-17] \cup [-6,13)$\\
G.$x \in (-\infty,-17) \cup [-6,13]$\\
H.$x \in (-\infty,-17] \cup [-6,13]$
\testStop
\kluczStart
A
\kluczStop



\zadStart{Zadanie z Wikieł Z 1.62 b) moja wersja nr 633}

Rozwiązać nierówności $(x+17)(13-x)(x+7)\ge0$.
\zadStop
\rozwStart{Patryk Wirkus}{}
Miejsca zerowe naszego wielomianu to: $-17, 13, -7$.\\
Wielomian jest stopnia nieparzystego, ponadto znak współczynnika przy\linebreak najwyższej potędze x jest ujemny.\\ W związku z tym wykres wielomianu zaczyna się od lewej strony powyżej osi OX. A więc $$x \in (-\infty,-17) \cup (-7,13).$$
\rozwStop
\odpStart
$x \in (-\infty,-17) \cup (-7,13)$
\odpStop
\testStart
A.$x \in (-\infty,-17) \cup (-7,13)$\\
B.$x \in (-\infty,-17) \cup (-7,13]$\\
C.$x \in (-\infty,-17) \cup [-7,13)$\\
D.$x \in (-\infty,-17] \cup (-7,13)$\\
E.$x \in (-\infty,-17] \cup (-7,13]$\\
F.$x \in (-\infty,-17] \cup [-7,13)$\\
G.$x \in (-\infty,-17) \cup [-7,13]$\\
H.$x \in (-\infty,-17] \cup [-7,13]$
\testStop
\kluczStart
A
\kluczStop



\zadStart{Zadanie z Wikieł Z 1.62 b) moja wersja nr 634}

Rozwiązać nierówności $(x+17)(13-x)(x+8)\ge0$.
\zadStop
\rozwStart{Patryk Wirkus}{}
Miejsca zerowe naszego wielomianu to: $-17, 13, -8$.\\
Wielomian jest stopnia nieparzystego, ponadto znak współczynnika przy\linebreak najwyższej potędze x jest ujemny.\\ W związku z tym wykres wielomianu zaczyna się od lewej strony powyżej osi OX. A więc $$x \in (-\infty,-17) \cup (-8,13).$$
\rozwStop
\odpStart
$x \in (-\infty,-17) \cup (-8,13)$
\odpStop
\testStart
A.$x \in (-\infty,-17) \cup (-8,13)$\\
B.$x \in (-\infty,-17) \cup (-8,13]$\\
C.$x \in (-\infty,-17) \cup [-8,13)$\\
D.$x \in (-\infty,-17] \cup (-8,13)$\\
E.$x \in (-\infty,-17] \cup (-8,13]$\\
F.$x \in (-\infty,-17] \cup [-8,13)$\\
G.$x \in (-\infty,-17) \cup [-8,13]$\\
H.$x \in (-\infty,-17] \cup [-8,13]$
\testStop
\kluczStart
A
\kluczStop



\zadStart{Zadanie z Wikieł Z 1.62 b) moja wersja nr 635}

Rozwiązać nierówności $(x+17)(13-x)(x+9)\ge0$.
\zadStop
\rozwStart{Patryk Wirkus}{}
Miejsca zerowe naszego wielomianu to: $-17, 13, -9$.\\
Wielomian jest stopnia nieparzystego, ponadto znak współczynnika przy\linebreak najwyższej potędze x jest ujemny.\\ W związku z tym wykres wielomianu zaczyna się od lewej strony powyżej osi OX. A więc $$x \in (-\infty,-17) \cup (-9,13).$$
\rozwStop
\odpStart
$x \in (-\infty,-17) \cup (-9,13)$
\odpStop
\testStart
A.$x \in (-\infty,-17) \cup (-9,13)$\\
B.$x \in (-\infty,-17) \cup (-9,13]$\\
C.$x \in (-\infty,-17) \cup [-9,13)$\\
D.$x \in (-\infty,-17] \cup (-9,13)$\\
E.$x \in (-\infty,-17] \cup (-9,13]$\\
F.$x \in (-\infty,-17] \cup [-9,13)$\\
G.$x \in (-\infty,-17) \cup [-9,13]$\\
H.$x \in (-\infty,-17] \cup [-9,13]$
\testStop
\kluczStart
A
\kluczStop



\zadStart{Zadanie z Wikieł Z 1.62 b) moja wersja nr 636}

Rozwiązać nierówności $(x+17)(13-x)(x+10)\ge0$.
\zadStop
\rozwStart{Patryk Wirkus}{}
Miejsca zerowe naszego wielomianu to: $-17, 13, -10$.\\
Wielomian jest stopnia nieparzystego, ponadto znak współczynnika przy\linebreak najwyższej potędze x jest ujemny.\\ W związku z tym wykres wielomianu zaczyna się od lewej strony powyżej osi OX. A więc $$x \in (-\infty,-17) \cup (-10,13).$$
\rozwStop
\odpStart
$x \in (-\infty,-17) \cup (-10,13)$
\odpStop
\testStart
A.$x \in (-\infty,-17) \cup (-10,13)$\\
B.$x \in (-\infty,-17) \cup (-10,13]$\\
C.$x \in (-\infty,-17) \cup [-10,13)$\\
D.$x \in (-\infty,-17] \cup (-10,13)$\\
E.$x \in (-\infty,-17] \cup (-10,13]$\\
F.$x \in (-\infty,-17] \cup [-10,13)$\\
G.$x \in (-\infty,-17) \cup [-10,13]$\\
H.$x \in (-\infty,-17] \cup [-10,13]$
\testStop
\kluczStart
A
\kluczStop



\zadStart{Zadanie z Wikieł Z 1.62 b) moja wersja nr 637}

Rozwiązać nierówności $(x+17)(13-x)(x+11)\ge0$.
\zadStop
\rozwStart{Patryk Wirkus}{}
Miejsca zerowe naszego wielomianu to: $-17, 13, -11$.\\
Wielomian jest stopnia nieparzystego, ponadto znak współczynnika przy\linebreak najwyższej potędze x jest ujemny.\\ W związku z tym wykres wielomianu zaczyna się od lewej strony powyżej osi OX. A więc $$x \in (-\infty,-17) \cup (-11,13).$$
\rozwStop
\odpStart
$x \in (-\infty,-17) \cup (-11,13)$
\odpStop
\testStart
A.$x \in (-\infty,-17) \cup (-11,13)$\\
B.$x \in (-\infty,-17) \cup (-11,13]$\\
C.$x \in (-\infty,-17) \cup [-11,13)$\\
D.$x \in (-\infty,-17] \cup (-11,13)$\\
E.$x \in (-\infty,-17] \cup (-11,13]$\\
F.$x \in (-\infty,-17] \cup [-11,13)$\\
G.$x \in (-\infty,-17) \cup [-11,13]$\\
H.$x \in (-\infty,-17] \cup [-11,13]$
\testStop
\kluczStart
A
\kluczStop



\zadStart{Zadanie z Wikieł Z 1.62 b) moja wersja nr 638}

Rozwiązać nierówności $(x+17)(13-x)(x+12)\ge0$.
\zadStop
\rozwStart{Patryk Wirkus}{}
Miejsca zerowe naszego wielomianu to: $-17, 13, -12$.\\
Wielomian jest stopnia nieparzystego, ponadto znak współczynnika przy\linebreak najwyższej potędze x jest ujemny.\\ W związku z tym wykres wielomianu zaczyna się od lewej strony powyżej osi OX. A więc $$x \in (-\infty,-17) \cup (-12,13).$$
\rozwStop
\odpStart
$x \in (-\infty,-17) \cup (-12,13)$
\odpStop
\testStart
A.$x \in (-\infty,-17) \cup (-12,13)$\\
B.$x \in (-\infty,-17) \cup (-12,13]$\\
C.$x \in (-\infty,-17) \cup [-12,13)$\\
D.$x \in (-\infty,-17] \cup (-12,13)$\\
E.$x \in (-\infty,-17] \cup (-12,13]$\\
F.$x \in (-\infty,-17] \cup [-12,13)$\\
G.$x \in (-\infty,-17) \cup [-12,13]$\\
H.$x \in (-\infty,-17] \cup [-12,13]$
\testStop
\kluczStart
A
\kluczStop



\zadStart{Zadanie z Wikieł Z 1.62 b) moja wersja nr 639}

Rozwiązać nierówności $(x+17)(14-x)(x+1)\ge0$.
\zadStop
\rozwStart{Patryk Wirkus}{}
Miejsca zerowe naszego wielomianu to: $-17, 14, -1$.\\
Wielomian jest stopnia nieparzystego, ponadto znak współczynnika przy\linebreak najwyższej potędze x jest ujemny.\\ W związku z tym wykres wielomianu zaczyna się od lewej strony powyżej osi OX. A więc $$x \in (-\infty,-17) \cup (-1,14).$$
\rozwStop
\odpStart
$x \in (-\infty,-17) \cup (-1,14)$
\odpStop
\testStart
A.$x \in (-\infty,-17) \cup (-1,14)$\\
B.$x \in (-\infty,-17) \cup (-1,14]$\\
C.$x \in (-\infty,-17) \cup [-1,14)$\\
D.$x \in (-\infty,-17] \cup (-1,14)$\\
E.$x \in (-\infty,-17] \cup (-1,14]$\\
F.$x \in (-\infty,-17] \cup [-1,14)$\\
G.$x \in (-\infty,-17) \cup [-1,14]$\\
H.$x \in (-\infty,-17] \cup [-1,14]$
\testStop
\kluczStart
A
\kluczStop



\zadStart{Zadanie z Wikieł Z 1.62 b) moja wersja nr 640}

Rozwiązać nierówności $(x+17)(14-x)(x+2)\ge0$.
\zadStop
\rozwStart{Patryk Wirkus}{}
Miejsca zerowe naszego wielomianu to: $-17, 14, -2$.\\
Wielomian jest stopnia nieparzystego, ponadto znak współczynnika przy\linebreak najwyższej potędze x jest ujemny.\\ W związku z tym wykres wielomianu zaczyna się od lewej strony powyżej osi OX. A więc $$x \in (-\infty,-17) \cup (-2,14).$$
\rozwStop
\odpStart
$x \in (-\infty,-17) \cup (-2,14)$
\odpStop
\testStart
A.$x \in (-\infty,-17) \cup (-2,14)$\\
B.$x \in (-\infty,-17) \cup (-2,14]$\\
C.$x \in (-\infty,-17) \cup [-2,14)$\\
D.$x \in (-\infty,-17] \cup (-2,14)$\\
E.$x \in (-\infty,-17] \cup (-2,14]$\\
F.$x \in (-\infty,-17] \cup [-2,14)$\\
G.$x \in (-\infty,-17) \cup [-2,14]$\\
H.$x \in (-\infty,-17] \cup [-2,14]$
\testStop
\kluczStart
A
\kluczStop



\zadStart{Zadanie z Wikieł Z 1.62 b) moja wersja nr 641}

Rozwiązać nierówności $(x+17)(14-x)(x+3)\ge0$.
\zadStop
\rozwStart{Patryk Wirkus}{}
Miejsca zerowe naszego wielomianu to: $-17, 14, -3$.\\
Wielomian jest stopnia nieparzystego, ponadto znak współczynnika przy\linebreak najwyższej potędze x jest ujemny.\\ W związku z tym wykres wielomianu zaczyna się od lewej strony powyżej osi OX. A więc $$x \in (-\infty,-17) \cup (-3,14).$$
\rozwStop
\odpStart
$x \in (-\infty,-17) \cup (-3,14)$
\odpStop
\testStart
A.$x \in (-\infty,-17) \cup (-3,14)$\\
B.$x \in (-\infty,-17) \cup (-3,14]$\\
C.$x \in (-\infty,-17) \cup [-3,14)$\\
D.$x \in (-\infty,-17] \cup (-3,14)$\\
E.$x \in (-\infty,-17] \cup (-3,14]$\\
F.$x \in (-\infty,-17] \cup [-3,14)$\\
G.$x \in (-\infty,-17) \cup [-3,14]$\\
H.$x \in (-\infty,-17] \cup [-3,14]$
\testStop
\kluczStart
A
\kluczStop



\zadStart{Zadanie z Wikieł Z 1.62 b) moja wersja nr 642}

Rozwiązać nierówności $(x+17)(14-x)(x+4)\ge0$.
\zadStop
\rozwStart{Patryk Wirkus}{}
Miejsca zerowe naszego wielomianu to: $-17, 14, -4$.\\
Wielomian jest stopnia nieparzystego, ponadto znak współczynnika przy\linebreak najwyższej potędze x jest ujemny.\\ W związku z tym wykres wielomianu zaczyna się od lewej strony powyżej osi OX. A więc $$x \in (-\infty,-17) \cup (-4,14).$$
\rozwStop
\odpStart
$x \in (-\infty,-17) \cup (-4,14)$
\odpStop
\testStart
A.$x \in (-\infty,-17) \cup (-4,14)$\\
B.$x \in (-\infty,-17) \cup (-4,14]$\\
C.$x \in (-\infty,-17) \cup [-4,14)$\\
D.$x \in (-\infty,-17] \cup (-4,14)$\\
E.$x \in (-\infty,-17] \cup (-4,14]$\\
F.$x \in (-\infty,-17] \cup [-4,14)$\\
G.$x \in (-\infty,-17) \cup [-4,14]$\\
H.$x \in (-\infty,-17] \cup [-4,14]$
\testStop
\kluczStart
A
\kluczStop



\zadStart{Zadanie z Wikieł Z 1.62 b) moja wersja nr 643}

Rozwiązać nierówności $(x+17)(14-x)(x+5)\ge0$.
\zadStop
\rozwStart{Patryk Wirkus}{}
Miejsca zerowe naszego wielomianu to: $-17, 14, -5$.\\
Wielomian jest stopnia nieparzystego, ponadto znak współczynnika przy\linebreak najwyższej potędze x jest ujemny.\\ W związku z tym wykres wielomianu zaczyna się od lewej strony powyżej osi OX. A więc $$x \in (-\infty,-17) \cup (-5,14).$$
\rozwStop
\odpStart
$x \in (-\infty,-17) \cup (-5,14)$
\odpStop
\testStart
A.$x \in (-\infty,-17) \cup (-5,14)$\\
B.$x \in (-\infty,-17) \cup (-5,14]$\\
C.$x \in (-\infty,-17) \cup [-5,14)$\\
D.$x \in (-\infty,-17] \cup (-5,14)$\\
E.$x \in (-\infty,-17] \cup (-5,14]$\\
F.$x \in (-\infty,-17] \cup [-5,14)$\\
G.$x \in (-\infty,-17) \cup [-5,14]$\\
H.$x \in (-\infty,-17] \cup [-5,14]$
\testStop
\kluczStart
A
\kluczStop



\zadStart{Zadanie z Wikieł Z 1.62 b) moja wersja nr 644}

Rozwiązać nierówności $(x+17)(14-x)(x+6)\ge0$.
\zadStop
\rozwStart{Patryk Wirkus}{}
Miejsca zerowe naszego wielomianu to: $-17, 14, -6$.\\
Wielomian jest stopnia nieparzystego, ponadto znak współczynnika przy\linebreak najwyższej potędze x jest ujemny.\\ W związku z tym wykres wielomianu zaczyna się od lewej strony powyżej osi OX. A więc $$x \in (-\infty,-17) \cup (-6,14).$$
\rozwStop
\odpStart
$x \in (-\infty,-17) \cup (-6,14)$
\odpStop
\testStart
A.$x \in (-\infty,-17) \cup (-6,14)$\\
B.$x \in (-\infty,-17) \cup (-6,14]$\\
C.$x \in (-\infty,-17) \cup [-6,14)$\\
D.$x \in (-\infty,-17] \cup (-6,14)$\\
E.$x \in (-\infty,-17] \cup (-6,14]$\\
F.$x \in (-\infty,-17] \cup [-6,14)$\\
G.$x \in (-\infty,-17) \cup [-6,14]$\\
H.$x \in (-\infty,-17] \cup [-6,14]$
\testStop
\kluczStart
A
\kluczStop



\zadStart{Zadanie z Wikieł Z 1.62 b) moja wersja nr 645}

Rozwiązać nierówności $(x+17)(14-x)(x+7)\ge0$.
\zadStop
\rozwStart{Patryk Wirkus}{}
Miejsca zerowe naszego wielomianu to: $-17, 14, -7$.\\
Wielomian jest stopnia nieparzystego, ponadto znak współczynnika przy\linebreak najwyższej potędze x jest ujemny.\\ W związku z tym wykres wielomianu zaczyna się od lewej strony powyżej osi OX. A więc $$x \in (-\infty,-17) \cup (-7,14).$$
\rozwStop
\odpStart
$x \in (-\infty,-17) \cup (-7,14)$
\odpStop
\testStart
A.$x \in (-\infty,-17) \cup (-7,14)$\\
B.$x \in (-\infty,-17) \cup (-7,14]$\\
C.$x \in (-\infty,-17) \cup [-7,14)$\\
D.$x \in (-\infty,-17] \cup (-7,14)$\\
E.$x \in (-\infty,-17] \cup (-7,14]$\\
F.$x \in (-\infty,-17] \cup [-7,14)$\\
G.$x \in (-\infty,-17) \cup [-7,14]$\\
H.$x \in (-\infty,-17] \cup [-7,14]$
\testStop
\kluczStart
A
\kluczStop



\zadStart{Zadanie z Wikieł Z 1.62 b) moja wersja nr 646}

Rozwiązać nierówności $(x+17)(14-x)(x+8)\ge0$.
\zadStop
\rozwStart{Patryk Wirkus}{}
Miejsca zerowe naszego wielomianu to: $-17, 14, -8$.\\
Wielomian jest stopnia nieparzystego, ponadto znak współczynnika przy\linebreak najwyższej potędze x jest ujemny.\\ W związku z tym wykres wielomianu zaczyna się od lewej strony powyżej osi OX. A więc $$x \in (-\infty,-17) \cup (-8,14).$$
\rozwStop
\odpStart
$x \in (-\infty,-17) \cup (-8,14)$
\odpStop
\testStart
A.$x \in (-\infty,-17) \cup (-8,14)$\\
B.$x \in (-\infty,-17) \cup (-8,14]$\\
C.$x \in (-\infty,-17) \cup [-8,14)$\\
D.$x \in (-\infty,-17] \cup (-8,14)$\\
E.$x \in (-\infty,-17] \cup (-8,14]$\\
F.$x \in (-\infty,-17] \cup [-8,14)$\\
G.$x \in (-\infty,-17) \cup [-8,14]$\\
H.$x \in (-\infty,-17] \cup [-8,14]$
\testStop
\kluczStart
A
\kluczStop



\zadStart{Zadanie z Wikieł Z 1.62 b) moja wersja nr 647}

Rozwiązać nierówności $(x+17)(14-x)(x+9)\ge0$.
\zadStop
\rozwStart{Patryk Wirkus}{}
Miejsca zerowe naszego wielomianu to: $-17, 14, -9$.\\
Wielomian jest stopnia nieparzystego, ponadto znak współczynnika przy\linebreak najwyższej potędze x jest ujemny.\\ W związku z tym wykres wielomianu zaczyna się od lewej strony powyżej osi OX. A więc $$x \in (-\infty,-17) \cup (-9,14).$$
\rozwStop
\odpStart
$x \in (-\infty,-17) \cup (-9,14)$
\odpStop
\testStart
A.$x \in (-\infty,-17) \cup (-9,14)$\\
B.$x \in (-\infty,-17) \cup (-9,14]$\\
C.$x \in (-\infty,-17) \cup [-9,14)$\\
D.$x \in (-\infty,-17] \cup (-9,14)$\\
E.$x \in (-\infty,-17] \cup (-9,14]$\\
F.$x \in (-\infty,-17] \cup [-9,14)$\\
G.$x \in (-\infty,-17) \cup [-9,14]$\\
H.$x \in (-\infty,-17] \cup [-9,14]$
\testStop
\kluczStart
A
\kluczStop



\zadStart{Zadanie z Wikieł Z 1.62 b) moja wersja nr 648}

Rozwiązać nierówności $(x+17)(14-x)(x+10)\ge0$.
\zadStop
\rozwStart{Patryk Wirkus}{}
Miejsca zerowe naszego wielomianu to: $-17, 14, -10$.\\
Wielomian jest stopnia nieparzystego, ponadto znak współczynnika przy\linebreak najwyższej potędze x jest ujemny.\\ W związku z tym wykres wielomianu zaczyna się od lewej strony powyżej osi OX. A więc $$x \in (-\infty,-17) \cup (-10,14).$$
\rozwStop
\odpStart
$x \in (-\infty,-17) \cup (-10,14)$
\odpStop
\testStart
A.$x \in (-\infty,-17) \cup (-10,14)$\\
B.$x \in (-\infty,-17) \cup (-10,14]$\\
C.$x \in (-\infty,-17) \cup [-10,14)$\\
D.$x \in (-\infty,-17] \cup (-10,14)$\\
E.$x \in (-\infty,-17] \cup (-10,14]$\\
F.$x \in (-\infty,-17] \cup [-10,14)$\\
G.$x \in (-\infty,-17) \cup [-10,14]$\\
H.$x \in (-\infty,-17] \cup [-10,14]$
\testStop
\kluczStart
A
\kluczStop



\zadStart{Zadanie z Wikieł Z 1.62 b) moja wersja nr 649}

Rozwiązać nierówności $(x+17)(14-x)(x+11)\ge0$.
\zadStop
\rozwStart{Patryk Wirkus}{}
Miejsca zerowe naszego wielomianu to: $-17, 14, -11$.\\
Wielomian jest stopnia nieparzystego, ponadto znak współczynnika przy\linebreak najwyższej potędze x jest ujemny.\\ W związku z tym wykres wielomianu zaczyna się od lewej strony powyżej osi OX. A więc $$x \in (-\infty,-17) \cup (-11,14).$$
\rozwStop
\odpStart
$x \in (-\infty,-17) \cup (-11,14)$
\odpStop
\testStart
A.$x \in (-\infty,-17) \cup (-11,14)$\\
B.$x \in (-\infty,-17) \cup (-11,14]$\\
C.$x \in (-\infty,-17) \cup [-11,14)$\\
D.$x \in (-\infty,-17] \cup (-11,14)$\\
E.$x \in (-\infty,-17] \cup (-11,14]$\\
F.$x \in (-\infty,-17] \cup [-11,14)$\\
G.$x \in (-\infty,-17) \cup [-11,14]$\\
H.$x \in (-\infty,-17] \cup [-11,14]$
\testStop
\kluczStart
A
\kluczStop



\zadStart{Zadanie z Wikieł Z 1.62 b) moja wersja nr 650}

Rozwiązać nierówności $(x+17)(14-x)(x+12)\ge0$.
\zadStop
\rozwStart{Patryk Wirkus}{}
Miejsca zerowe naszego wielomianu to: $-17, 14, -12$.\\
Wielomian jest stopnia nieparzystego, ponadto znak współczynnika przy\linebreak najwyższej potędze x jest ujemny.\\ W związku z tym wykres wielomianu zaczyna się od lewej strony powyżej osi OX. A więc $$x \in (-\infty,-17) \cup (-12,14).$$
\rozwStop
\odpStart
$x \in (-\infty,-17) \cup (-12,14)$
\odpStop
\testStart
A.$x \in (-\infty,-17) \cup (-12,14)$\\
B.$x \in (-\infty,-17) \cup (-12,14]$\\
C.$x \in (-\infty,-17) \cup [-12,14)$\\
D.$x \in (-\infty,-17] \cup (-12,14)$\\
E.$x \in (-\infty,-17] \cup (-12,14]$\\
F.$x \in (-\infty,-17] \cup [-12,14)$\\
G.$x \in (-\infty,-17) \cup [-12,14]$\\
H.$x \in (-\infty,-17] \cup [-12,14]$
\testStop
\kluczStart
A
\kluczStop



\zadStart{Zadanie z Wikieł Z 1.62 b) moja wersja nr 651}

Rozwiązać nierówności $(x+17)(14-x)(x+13)\ge0$.
\zadStop
\rozwStart{Patryk Wirkus}{}
Miejsca zerowe naszego wielomianu to: $-17, 14, -13$.\\
Wielomian jest stopnia nieparzystego, ponadto znak współczynnika przy\linebreak najwyższej potędze x jest ujemny.\\ W związku z tym wykres wielomianu zaczyna się od lewej strony powyżej osi OX. A więc $$x \in (-\infty,-17) \cup (-13,14).$$
\rozwStop
\odpStart
$x \in (-\infty,-17) \cup (-13,14)$
\odpStop
\testStart
A.$x \in (-\infty,-17) \cup (-13,14)$\\
B.$x \in (-\infty,-17) \cup (-13,14]$\\
C.$x \in (-\infty,-17) \cup [-13,14)$\\
D.$x \in (-\infty,-17] \cup (-13,14)$\\
E.$x \in (-\infty,-17] \cup (-13,14]$\\
F.$x \in (-\infty,-17] \cup [-13,14)$\\
G.$x \in (-\infty,-17) \cup [-13,14]$\\
H.$x \in (-\infty,-17] \cup [-13,14]$
\testStop
\kluczStart
A
\kluczStop



\zadStart{Zadanie z Wikieł Z 1.62 b) moja wersja nr 652}

Rozwiązać nierówności $(x+17)(15-x)(x+1)\ge0$.
\zadStop
\rozwStart{Patryk Wirkus}{}
Miejsca zerowe naszego wielomianu to: $-17, 15, -1$.\\
Wielomian jest stopnia nieparzystego, ponadto znak współczynnika przy\linebreak najwyższej potędze x jest ujemny.\\ W związku z tym wykres wielomianu zaczyna się od lewej strony powyżej osi OX. A więc $$x \in (-\infty,-17) \cup (-1,15).$$
\rozwStop
\odpStart
$x \in (-\infty,-17) \cup (-1,15)$
\odpStop
\testStart
A.$x \in (-\infty,-17) \cup (-1,15)$\\
B.$x \in (-\infty,-17) \cup (-1,15]$\\
C.$x \in (-\infty,-17) \cup [-1,15)$\\
D.$x \in (-\infty,-17] \cup (-1,15)$\\
E.$x \in (-\infty,-17] \cup (-1,15]$\\
F.$x \in (-\infty,-17] \cup [-1,15)$\\
G.$x \in (-\infty,-17) \cup [-1,15]$\\
H.$x \in (-\infty,-17] \cup [-1,15]$
\testStop
\kluczStart
A
\kluczStop



\zadStart{Zadanie z Wikieł Z 1.62 b) moja wersja nr 653}

Rozwiązać nierówności $(x+17)(15-x)(x+2)\ge0$.
\zadStop
\rozwStart{Patryk Wirkus}{}
Miejsca zerowe naszego wielomianu to: $-17, 15, -2$.\\
Wielomian jest stopnia nieparzystego, ponadto znak współczynnika przy\linebreak najwyższej potędze x jest ujemny.\\ W związku z tym wykres wielomianu zaczyna się od lewej strony powyżej osi OX. A więc $$x \in (-\infty,-17) \cup (-2,15).$$
\rozwStop
\odpStart
$x \in (-\infty,-17) \cup (-2,15)$
\odpStop
\testStart
A.$x \in (-\infty,-17) \cup (-2,15)$\\
B.$x \in (-\infty,-17) \cup (-2,15]$\\
C.$x \in (-\infty,-17) \cup [-2,15)$\\
D.$x \in (-\infty,-17] \cup (-2,15)$\\
E.$x \in (-\infty,-17] \cup (-2,15]$\\
F.$x \in (-\infty,-17] \cup [-2,15)$\\
G.$x \in (-\infty,-17) \cup [-2,15]$\\
H.$x \in (-\infty,-17] \cup [-2,15]$
\testStop
\kluczStart
A
\kluczStop



\zadStart{Zadanie z Wikieł Z 1.62 b) moja wersja nr 654}

Rozwiązać nierówności $(x+17)(15-x)(x+3)\ge0$.
\zadStop
\rozwStart{Patryk Wirkus}{}
Miejsca zerowe naszego wielomianu to: $-17, 15, -3$.\\
Wielomian jest stopnia nieparzystego, ponadto znak współczynnika przy\linebreak najwyższej potędze x jest ujemny.\\ W związku z tym wykres wielomianu zaczyna się od lewej strony powyżej osi OX. A więc $$x \in (-\infty,-17) \cup (-3,15).$$
\rozwStop
\odpStart
$x \in (-\infty,-17) \cup (-3,15)$
\odpStop
\testStart
A.$x \in (-\infty,-17) \cup (-3,15)$\\
B.$x \in (-\infty,-17) \cup (-3,15]$\\
C.$x \in (-\infty,-17) \cup [-3,15)$\\
D.$x \in (-\infty,-17] \cup (-3,15)$\\
E.$x \in (-\infty,-17] \cup (-3,15]$\\
F.$x \in (-\infty,-17] \cup [-3,15)$\\
G.$x \in (-\infty,-17) \cup [-3,15]$\\
H.$x \in (-\infty,-17] \cup [-3,15]$
\testStop
\kluczStart
A
\kluczStop



\zadStart{Zadanie z Wikieł Z 1.62 b) moja wersja nr 655}

Rozwiązać nierówności $(x+17)(15-x)(x+4)\ge0$.
\zadStop
\rozwStart{Patryk Wirkus}{}
Miejsca zerowe naszego wielomianu to: $-17, 15, -4$.\\
Wielomian jest stopnia nieparzystego, ponadto znak współczynnika przy\linebreak najwyższej potędze x jest ujemny.\\ W związku z tym wykres wielomianu zaczyna się od lewej strony powyżej osi OX. A więc $$x \in (-\infty,-17) \cup (-4,15).$$
\rozwStop
\odpStart
$x \in (-\infty,-17) \cup (-4,15)$
\odpStop
\testStart
A.$x \in (-\infty,-17) \cup (-4,15)$\\
B.$x \in (-\infty,-17) \cup (-4,15]$\\
C.$x \in (-\infty,-17) \cup [-4,15)$\\
D.$x \in (-\infty,-17] \cup (-4,15)$\\
E.$x \in (-\infty,-17] \cup (-4,15]$\\
F.$x \in (-\infty,-17] \cup [-4,15)$\\
G.$x \in (-\infty,-17) \cup [-4,15]$\\
H.$x \in (-\infty,-17] \cup [-4,15]$
\testStop
\kluczStart
A
\kluczStop



\zadStart{Zadanie z Wikieł Z 1.62 b) moja wersja nr 656}

Rozwiązać nierówności $(x+17)(15-x)(x+5)\ge0$.
\zadStop
\rozwStart{Patryk Wirkus}{}
Miejsca zerowe naszego wielomianu to: $-17, 15, -5$.\\
Wielomian jest stopnia nieparzystego, ponadto znak współczynnika przy\linebreak najwyższej potędze x jest ujemny.\\ W związku z tym wykres wielomianu zaczyna się od lewej strony powyżej osi OX. A więc $$x \in (-\infty,-17) \cup (-5,15).$$
\rozwStop
\odpStart
$x \in (-\infty,-17) \cup (-5,15)$
\odpStop
\testStart
A.$x \in (-\infty,-17) \cup (-5,15)$\\
B.$x \in (-\infty,-17) \cup (-5,15]$\\
C.$x \in (-\infty,-17) \cup [-5,15)$\\
D.$x \in (-\infty,-17] \cup (-5,15)$\\
E.$x \in (-\infty,-17] \cup (-5,15]$\\
F.$x \in (-\infty,-17] \cup [-5,15)$\\
G.$x \in (-\infty,-17) \cup [-5,15]$\\
H.$x \in (-\infty,-17] \cup [-5,15]$
\testStop
\kluczStart
A
\kluczStop



\zadStart{Zadanie z Wikieł Z 1.62 b) moja wersja nr 657}

Rozwiązać nierówności $(x+17)(15-x)(x+6)\ge0$.
\zadStop
\rozwStart{Patryk Wirkus}{}
Miejsca zerowe naszego wielomianu to: $-17, 15, -6$.\\
Wielomian jest stopnia nieparzystego, ponadto znak współczynnika przy\linebreak najwyższej potędze x jest ujemny.\\ W związku z tym wykres wielomianu zaczyna się od lewej strony powyżej osi OX. A więc $$x \in (-\infty,-17) \cup (-6,15).$$
\rozwStop
\odpStart
$x \in (-\infty,-17) \cup (-6,15)$
\odpStop
\testStart
A.$x \in (-\infty,-17) \cup (-6,15)$\\
B.$x \in (-\infty,-17) \cup (-6,15]$\\
C.$x \in (-\infty,-17) \cup [-6,15)$\\
D.$x \in (-\infty,-17] \cup (-6,15)$\\
E.$x \in (-\infty,-17] \cup (-6,15]$\\
F.$x \in (-\infty,-17] \cup [-6,15)$\\
G.$x \in (-\infty,-17) \cup [-6,15]$\\
H.$x \in (-\infty,-17] \cup [-6,15]$
\testStop
\kluczStart
A
\kluczStop



\zadStart{Zadanie z Wikieł Z 1.62 b) moja wersja nr 658}

Rozwiązać nierówności $(x+17)(15-x)(x+7)\ge0$.
\zadStop
\rozwStart{Patryk Wirkus}{}
Miejsca zerowe naszego wielomianu to: $-17, 15, -7$.\\
Wielomian jest stopnia nieparzystego, ponadto znak współczynnika przy\linebreak najwyższej potędze x jest ujemny.\\ W związku z tym wykres wielomianu zaczyna się od lewej strony powyżej osi OX. A więc $$x \in (-\infty,-17) \cup (-7,15).$$
\rozwStop
\odpStart
$x \in (-\infty,-17) \cup (-7,15)$
\odpStop
\testStart
A.$x \in (-\infty,-17) \cup (-7,15)$\\
B.$x \in (-\infty,-17) \cup (-7,15]$\\
C.$x \in (-\infty,-17) \cup [-7,15)$\\
D.$x \in (-\infty,-17] \cup (-7,15)$\\
E.$x \in (-\infty,-17] \cup (-7,15]$\\
F.$x \in (-\infty,-17] \cup [-7,15)$\\
G.$x \in (-\infty,-17) \cup [-7,15]$\\
H.$x \in (-\infty,-17] \cup [-7,15]$
\testStop
\kluczStart
A
\kluczStop



\zadStart{Zadanie z Wikieł Z 1.62 b) moja wersja nr 659}

Rozwiązać nierówności $(x+17)(15-x)(x+8)\ge0$.
\zadStop
\rozwStart{Patryk Wirkus}{}
Miejsca zerowe naszego wielomianu to: $-17, 15, -8$.\\
Wielomian jest stopnia nieparzystego, ponadto znak współczynnika przy\linebreak najwyższej potędze x jest ujemny.\\ W związku z tym wykres wielomianu zaczyna się od lewej strony powyżej osi OX. A więc $$x \in (-\infty,-17) \cup (-8,15).$$
\rozwStop
\odpStart
$x \in (-\infty,-17) \cup (-8,15)$
\odpStop
\testStart
A.$x \in (-\infty,-17) \cup (-8,15)$\\
B.$x \in (-\infty,-17) \cup (-8,15]$\\
C.$x \in (-\infty,-17) \cup [-8,15)$\\
D.$x \in (-\infty,-17] \cup (-8,15)$\\
E.$x \in (-\infty,-17] \cup (-8,15]$\\
F.$x \in (-\infty,-17] \cup [-8,15)$\\
G.$x \in (-\infty,-17) \cup [-8,15]$\\
H.$x \in (-\infty,-17] \cup [-8,15]$
\testStop
\kluczStart
A
\kluczStop



\zadStart{Zadanie z Wikieł Z 1.62 b) moja wersja nr 660}

Rozwiązać nierówności $(x+17)(15-x)(x+9)\ge0$.
\zadStop
\rozwStart{Patryk Wirkus}{}
Miejsca zerowe naszego wielomianu to: $-17, 15, -9$.\\
Wielomian jest stopnia nieparzystego, ponadto znak współczynnika przy\linebreak najwyższej potędze x jest ujemny.\\ W związku z tym wykres wielomianu zaczyna się od lewej strony powyżej osi OX. A więc $$x \in (-\infty,-17) \cup (-9,15).$$
\rozwStop
\odpStart
$x \in (-\infty,-17) \cup (-9,15)$
\odpStop
\testStart
A.$x \in (-\infty,-17) \cup (-9,15)$\\
B.$x \in (-\infty,-17) \cup (-9,15]$\\
C.$x \in (-\infty,-17) \cup [-9,15)$\\
D.$x \in (-\infty,-17] \cup (-9,15)$\\
E.$x \in (-\infty,-17] \cup (-9,15]$\\
F.$x \in (-\infty,-17] \cup [-9,15)$\\
G.$x \in (-\infty,-17) \cup [-9,15]$\\
H.$x \in (-\infty,-17] \cup [-9,15]$
\testStop
\kluczStart
A
\kluczStop



\zadStart{Zadanie z Wikieł Z 1.62 b) moja wersja nr 661}

Rozwiązać nierówności $(x+17)(15-x)(x+10)\ge0$.
\zadStop
\rozwStart{Patryk Wirkus}{}
Miejsca zerowe naszego wielomianu to: $-17, 15, -10$.\\
Wielomian jest stopnia nieparzystego, ponadto znak współczynnika przy\linebreak najwyższej potędze x jest ujemny.\\ W związku z tym wykres wielomianu zaczyna się od lewej strony powyżej osi OX. A więc $$x \in (-\infty,-17) \cup (-10,15).$$
\rozwStop
\odpStart
$x \in (-\infty,-17) \cup (-10,15)$
\odpStop
\testStart
A.$x \in (-\infty,-17) \cup (-10,15)$\\
B.$x \in (-\infty,-17) \cup (-10,15]$\\
C.$x \in (-\infty,-17) \cup [-10,15)$\\
D.$x \in (-\infty,-17] \cup (-10,15)$\\
E.$x \in (-\infty,-17] \cup (-10,15]$\\
F.$x \in (-\infty,-17] \cup [-10,15)$\\
G.$x \in (-\infty,-17) \cup [-10,15]$\\
H.$x \in (-\infty,-17] \cup [-10,15]$
\testStop
\kluczStart
A
\kluczStop



\zadStart{Zadanie z Wikieł Z 1.62 b) moja wersja nr 662}

Rozwiązać nierówności $(x+17)(15-x)(x+11)\ge0$.
\zadStop
\rozwStart{Patryk Wirkus}{}
Miejsca zerowe naszego wielomianu to: $-17, 15, -11$.\\
Wielomian jest stopnia nieparzystego, ponadto znak współczynnika przy\linebreak najwyższej potędze x jest ujemny.\\ W związku z tym wykres wielomianu zaczyna się od lewej strony powyżej osi OX. A więc $$x \in (-\infty,-17) \cup (-11,15).$$
\rozwStop
\odpStart
$x \in (-\infty,-17) \cup (-11,15)$
\odpStop
\testStart
A.$x \in (-\infty,-17) \cup (-11,15)$\\
B.$x \in (-\infty,-17) \cup (-11,15]$\\
C.$x \in (-\infty,-17) \cup [-11,15)$\\
D.$x \in (-\infty,-17] \cup (-11,15)$\\
E.$x \in (-\infty,-17] \cup (-11,15]$\\
F.$x \in (-\infty,-17] \cup [-11,15)$\\
G.$x \in (-\infty,-17) \cup [-11,15]$\\
H.$x \in (-\infty,-17] \cup [-11,15]$
\testStop
\kluczStart
A
\kluczStop



\zadStart{Zadanie z Wikieł Z 1.62 b) moja wersja nr 663}

Rozwiązać nierówności $(x+17)(15-x)(x+12)\ge0$.
\zadStop
\rozwStart{Patryk Wirkus}{}
Miejsca zerowe naszego wielomianu to: $-17, 15, -12$.\\
Wielomian jest stopnia nieparzystego, ponadto znak współczynnika przy\linebreak najwyższej potędze x jest ujemny.\\ W związku z tym wykres wielomianu zaczyna się od lewej strony powyżej osi OX. A więc $$x \in (-\infty,-17) \cup (-12,15).$$
\rozwStop
\odpStart
$x \in (-\infty,-17) \cup (-12,15)$
\odpStop
\testStart
A.$x \in (-\infty,-17) \cup (-12,15)$\\
B.$x \in (-\infty,-17) \cup (-12,15]$\\
C.$x \in (-\infty,-17) \cup [-12,15)$\\
D.$x \in (-\infty,-17] \cup (-12,15)$\\
E.$x \in (-\infty,-17] \cup (-12,15]$\\
F.$x \in (-\infty,-17] \cup [-12,15)$\\
G.$x \in (-\infty,-17) \cup [-12,15]$\\
H.$x \in (-\infty,-17] \cup [-12,15]$
\testStop
\kluczStart
A
\kluczStop



\zadStart{Zadanie z Wikieł Z 1.62 b) moja wersja nr 664}

Rozwiązać nierówności $(x+17)(15-x)(x+13)\ge0$.
\zadStop
\rozwStart{Patryk Wirkus}{}
Miejsca zerowe naszego wielomianu to: $-17, 15, -13$.\\
Wielomian jest stopnia nieparzystego, ponadto znak współczynnika przy\linebreak najwyższej potędze x jest ujemny.\\ W związku z tym wykres wielomianu zaczyna się od lewej strony powyżej osi OX. A więc $$x \in (-\infty,-17) \cup (-13,15).$$
\rozwStop
\odpStart
$x \in (-\infty,-17) \cup (-13,15)$
\odpStop
\testStart
A.$x \in (-\infty,-17) \cup (-13,15)$\\
B.$x \in (-\infty,-17) \cup (-13,15]$\\
C.$x \in (-\infty,-17) \cup [-13,15)$\\
D.$x \in (-\infty,-17] \cup (-13,15)$\\
E.$x \in (-\infty,-17] \cup (-13,15]$\\
F.$x \in (-\infty,-17] \cup [-13,15)$\\
G.$x \in (-\infty,-17) \cup [-13,15]$\\
H.$x \in (-\infty,-17] \cup [-13,15]$
\testStop
\kluczStart
A
\kluczStop



\zadStart{Zadanie z Wikieł Z 1.62 b) moja wersja nr 665}

Rozwiązać nierówności $(x+17)(15-x)(x+14)\ge0$.
\zadStop
\rozwStart{Patryk Wirkus}{}
Miejsca zerowe naszego wielomianu to: $-17, 15, -14$.\\
Wielomian jest stopnia nieparzystego, ponadto znak współczynnika przy\linebreak najwyższej potędze x jest ujemny.\\ W związku z tym wykres wielomianu zaczyna się od lewej strony powyżej osi OX. A więc $$x \in (-\infty,-17) \cup (-14,15).$$
\rozwStop
\odpStart
$x \in (-\infty,-17) \cup (-14,15)$
\odpStop
\testStart
A.$x \in (-\infty,-17) \cup (-14,15)$\\
B.$x \in (-\infty,-17) \cup (-14,15]$\\
C.$x \in (-\infty,-17) \cup [-14,15)$\\
D.$x \in (-\infty,-17] \cup (-14,15)$\\
E.$x \in (-\infty,-17] \cup (-14,15]$\\
F.$x \in (-\infty,-17] \cup [-14,15)$\\
G.$x \in (-\infty,-17) \cup [-14,15]$\\
H.$x \in (-\infty,-17] \cup [-14,15]$
\testStop
\kluczStart
A
\kluczStop



\zadStart{Zadanie z Wikieł Z 1.62 b) moja wersja nr 666}

Rozwiązać nierówności $(x+17)(16-x)(x+1)\ge0$.
\zadStop
\rozwStart{Patryk Wirkus}{}
Miejsca zerowe naszego wielomianu to: $-17, 16, -1$.\\
Wielomian jest stopnia nieparzystego, ponadto znak współczynnika przy\linebreak najwyższej potędze x jest ujemny.\\ W związku z tym wykres wielomianu zaczyna się od lewej strony powyżej osi OX. A więc $$x \in (-\infty,-17) \cup (-1,16).$$
\rozwStop
\odpStart
$x \in (-\infty,-17) \cup (-1,16)$
\odpStop
\testStart
A.$x \in (-\infty,-17) \cup (-1,16)$\\
B.$x \in (-\infty,-17) \cup (-1,16]$\\
C.$x \in (-\infty,-17) \cup [-1,16)$\\
D.$x \in (-\infty,-17] \cup (-1,16)$\\
E.$x \in (-\infty,-17] \cup (-1,16]$\\
F.$x \in (-\infty,-17] \cup [-1,16)$\\
G.$x \in (-\infty,-17) \cup [-1,16]$\\
H.$x \in (-\infty,-17] \cup [-1,16]$
\testStop
\kluczStart
A
\kluczStop



\zadStart{Zadanie z Wikieł Z 1.62 b) moja wersja nr 667}

Rozwiązać nierówności $(x+17)(16-x)(x+2)\ge0$.
\zadStop
\rozwStart{Patryk Wirkus}{}
Miejsca zerowe naszego wielomianu to: $-17, 16, -2$.\\
Wielomian jest stopnia nieparzystego, ponadto znak współczynnika przy\linebreak najwyższej potędze x jest ujemny.\\ W związku z tym wykres wielomianu zaczyna się od lewej strony powyżej osi OX. A więc $$x \in (-\infty,-17) \cup (-2,16).$$
\rozwStop
\odpStart
$x \in (-\infty,-17) \cup (-2,16)$
\odpStop
\testStart
A.$x \in (-\infty,-17) \cup (-2,16)$\\
B.$x \in (-\infty,-17) \cup (-2,16]$\\
C.$x \in (-\infty,-17) \cup [-2,16)$\\
D.$x \in (-\infty,-17] \cup (-2,16)$\\
E.$x \in (-\infty,-17] \cup (-2,16]$\\
F.$x \in (-\infty,-17] \cup [-2,16)$\\
G.$x \in (-\infty,-17) \cup [-2,16]$\\
H.$x \in (-\infty,-17] \cup [-2,16]$
\testStop
\kluczStart
A
\kluczStop



\zadStart{Zadanie z Wikieł Z 1.62 b) moja wersja nr 668}

Rozwiązać nierówności $(x+17)(16-x)(x+3)\ge0$.
\zadStop
\rozwStart{Patryk Wirkus}{}
Miejsca zerowe naszego wielomianu to: $-17, 16, -3$.\\
Wielomian jest stopnia nieparzystego, ponadto znak współczynnika przy\linebreak najwyższej potędze x jest ujemny.\\ W związku z tym wykres wielomianu zaczyna się od lewej strony powyżej osi OX. A więc $$x \in (-\infty,-17) \cup (-3,16).$$
\rozwStop
\odpStart
$x \in (-\infty,-17) \cup (-3,16)$
\odpStop
\testStart
A.$x \in (-\infty,-17) \cup (-3,16)$\\
B.$x \in (-\infty,-17) \cup (-3,16]$\\
C.$x \in (-\infty,-17) \cup [-3,16)$\\
D.$x \in (-\infty,-17] \cup (-3,16)$\\
E.$x \in (-\infty,-17] \cup (-3,16]$\\
F.$x \in (-\infty,-17] \cup [-3,16)$\\
G.$x \in (-\infty,-17) \cup [-3,16]$\\
H.$x \in (-\infty,-17] \cup [-3,16]$
\testStop
\kluczStart
A
\kluczStop



\zadStart{Zadanie z Wikieł Z 1.62 b) moja wersja nr 669}

Rozwiązać nierówności $(x+17)(16-x)(x+4)\ge0$.
\zadStop
\rozwStart{Patryk Wirkus}{}
Miejsca zerowe naszego wielomianu to: $-17, 16, -4$.\\
Wielomian jest stopnia nieparzystego, ponadto znak współczynnika przy\linebreak najwyższej potędze x jest ujemny.\\ W związku z tym wykres wielomianu zaczyna się od lewej strony powyżej osi OX. A więc $$x \in (-\infty,-17) \cup (-4,16).$$
\rozwStop
\odpStart
$x \in (-\infty,-17) \cup (-4,16)$
\odpStop
\testStart
A.$x \in (-\infty,-17) \cup (-4,16)$\\
B.$x \in (-\infty,-17) \cup (-4,16]$\\
C.$x \in (-\infty,-17) \cup [-4,16)$\\
D.$x \in (-\infty,-17] \cup (-4,16)$\\
E.$x \in (-\infty,-17] \cup (-4,16]$\\
F.$x \in (-\infty,-17] \cup [-4,16)$\\
G.$x \in (-\infty,-17) \cup [-4,16]$\\
H.$x \in (-\infty,-17] \cup [-4,16]$
\testStop
\kluczStart
A
\kluczStop



\zadStart{Zadanie z Wikieł Z 1.62 b) moja wersja nr 670}

Rozwiązać nierówności $(x+17)(16-x)(x+5)\ge0$.
\zadStop
\rozwStart{Patryk Wirkus}{}
Miejsca zerowe naszego wielomianu to: $-17, 16, -5$.\\
Wielomian jest stopnia nieparzystego, ponadto znak współczynnika przy\linebreak najwyższej potędze x jest ujemny.\\ W związku z tym wykres wielomianu zaczyna się od lewej strony powyżej osi OX. A więc $$x \in (-\infty,-17) \cup (-5,16).$$
\rozwStop
\odpStart
$x \in (-\infty,-17) \cup (-5,16)$
\odpStop
\testStart
A.$x \in (-\infty,-17) \cup (-5,16)$\\
B.$x \in (-\infty,-17) \cup (-5,16]$\\
C.$x \in (-\infty,-17) \cup [-5,16)$\\
D.$x \in (-\infty,-17] \cup (-5,16)$\\
E.$x \in (-\infty,-17] \cup (-5,16]$\\
F.$x \in (-\infty,-17] \cup [-5,16)$\\
G.$x \in (-\infty,-17) \cup [-5,16]$\\
H.$x \in (-\infty,-17] \cup [-5,16]$
\testStop
\kluczStart
A
\kluczStop



\zadStart{Zadanie z Wikieł Z 1.62 b) moja wersja nr 671}

Rozwiązać nierówności $(x+17)(16-x)(x+6)\ge0$.
\zadStop
\rozwStart{Patryk Wirkus}{}
Miejsca zerowe naszego wielomianu to: $-17, 16, -6$.\\
Wielomian jest stopnia nieparzystego, ponadto znak współczynnika przy\linebreak najwyższej potędze x jest ujemny.\\ W związku z tym wykres wielomianu zaczyna się od lewej strony powyżej osi OX. A więc $$x \in (-\infty,-17) \cup (-6,16).$$
\rozwStop
\odpStart
$x \in (-\infty,-17) \cup (-6,16)$
\odpStop
\testStart
A.$x \in (-\infty,-17) \cup (-6,16)$\\
B.$x \in (-\infty,-17) \cup (-6,16]$\\
C.$x \in (-\infty,-17) \cup [-6,16)$\\
D.$x \in (-\infty,-17] \cup (-6,16)$\\
E.$x \in (-\infty,-17] \cup (-6,16]$\\
F.$x \in (-\infty,-17] \cup [-6,16)$\\
G.$x \in (-\infty,-17) \cup [-6,16]$\\
H.$x \in (-\infty,-17] \cup [-6,16]$
\testStop
\kluczStart
A
\kluczStop



\zadStart{Zadanie z Wikieł Z 1.62 b) moja wersja nr 672}

Rozwiązać nierówności $(x+17)(16-x)(x+7)\ge0$.
\zadStop
\rozwStart{Patryk Wirkus}{}
Miejsca zerowe naszego wielomianu to: $-17, 16, -7$.\\
Wielomian jest stopnia nieparzystego, ponadto znak współczynnika przy\linebreak najwyższej potędze x jest ujemny.\\ W związku z tym wykres wielomianu zaczyna się od lewej strony powyżej osi OX. A więc $$x \in (-\infty,-17) \cup (-7,16).$$
\rozwStop
\odpStart
$x \in (-\infty,-17) \cup (-7,16)$
\odpStop
\testStart
A.$x \in (-\infty,-17) \cup (-7,16)$\\
B.$x \in (-\infty,-17) \cup (-7,16]$\\
C.$x \in (-\infty,-17) \cup [-7,16)$\\
D.$x \in (-\infty,-17] \cup (-7,16)$\\
E.$x \in (-\infty,-17] \cup (-7,16]$\\
F.$x \in (-\infty,-17] \cup [-7,16)$\\
G.$x \in (-\infty,-17) \cup [-7,16]$\\
H.$x \in (-\infty,-17] \cup [-7,16]$
\testStop
\kluczStart
A
\kluczStop



\zadStart{Zadanie z Wikieł Z 1.62 b) moja wersja nr 673}

Rozwiązać nierówności $(x+17)(16-x)(x+8)\ge0$.
\zadStop
\rozwStart{Patryk Wirkus}{}
Miejsca zerowe naszego wielomianu to: $-17, 16, -8$.\\
Wielomian jest stopnia nieparzystego, ponadto znak współczynnika przy\linebreak najwyższej potędze x jest ujemny.\\ W związku z tym wykres wielomianu zaczyna się od lewej strony powyżej osi OX. A więc $$x \in (-\infty,-17) \cup (-8,16).$$
\rozwStop
\odpStart
$x \in (-\infty,-17) \cup (-8,16)$
\odpStop
\testStart
A.$x \in (-\infty,-17) \cup (-8,16)$\\
B.$x \in (-\infty,-17) \cup (-8,16]$\\
C.$x \in (-\infty,-17) \cup [-8,16)$\\
D.$x \in (-\infty,-17] \cup (-8,16)$\\
E.$x \in (-\infty,-17] \cup (-8,16]$\\
F.$x \in (-\infty,-17] \cup [-8,16)$\\
G.$x \in (-\infty,-17) \cup [-8,16]$\\
H.$x \in (-\infty,-17] \cup [-8,16]$
\testStop
\kluczStart
A
\kluczStop



\zadStart{Zadanie z Wikieł Z 1.62 b) moja wersja nr 674}

Rozwiązać nierówności $(x+17)(16-x)(x+9)\ge0$.
\zadStop
\rozwStart{Patryk Wirkus}{}
Miejsca zerowe naszego wielomianu to: $-17, 16, -9$.\\
Wielomian jest stopnia nieparzystego, ponadto znak współczynnika przy\linebreak najwyższej potędze x jest ujemny.\\ W związku z tym wykres wielomianu zaczyna się od lewej strony powyżej osi OX. A więc $$x \in (-\infty,-17) \cup (-9,16).$$
\rozwStop
\odpStart
$x \in (-\infty,-17) \cup (-9,16)$
\odpStop
\testStart
A.$x \in (-\infty,-17) \cup (-9,16)$\\
B.$x \in (-\infty,-17) \cup (-9,16]$\\
C.$x \in (-\infty,-17) \cup [-9,16)$\\
D.$x \in (-\infty,-17] \cup (-9,16)$\\
E.$x \in (-\infty,-17] \cup (-9,16]$\\
F.$x \in (-\infty,-17] \cup [-9,16)$\\
G.$x \in (-\infty,-17) \cup [-9,16]$\\
H.$x \in (-\infty,-17] \cup [-9,16]$
\testStop
\kluczStart
A
\kluczStop



\zadStart{Zadanie z Wikieł Z 1.62 b) moja wersja nr 675}

Rozwiązać nierówności $(x+17)(16-x)(x+10)\ge0$.
\zadStop
\rozwStart{Patryk Wirkus}{}
Miejsca zerowe naszego wielomianu to: $-17, 16, -10$.\\
Wielomian jest stopnia nieparzystego, ponadto znak współczynnika przy\linebreak najwyższej potędze x jest ujemny.\\ W związku z tym wykres wielomianu zaczyna się od lewej strony powyżej osi OX. A więc $$x \in (-\infty,-17) \cup (-10,16).$$
\rozwStop
\odpStart
$x \in (-\infty,-17) \cup (-10,16)$
\odpStop
\testStart
A.$x \in (-\infty,-17) \cup (-10,16)$\\
B.$x \in (-\infty,-17) \cup (-10,16]$\\
C.$x \in (-\infty,-17) \cup [-10,16)$\\
D.$x \in (-\infty,-17] \cup (-10,16)$\\
E.$x \in (-\infty,-17] \cup (-10,16]$\\
F.$x \in (-\infty,-17] \cup [-10,16)$\\
G.$x \in (-\infty,-17) \cup [-10,16]$\\
H.$x \in (-\infty,-17] \cup [-10,16]$
\testStop
\kluczStart
A
\kluczStop



\zadStart{Zadanie z Wikieł Z 1.62 b) moja wersja nr 676}

Rozwiązać nierówności $(x+17)(16-x)(x+11)\ge0$.
\zadStop
\rozwStart{Patryk Wirkus}{}
Miejsca zerowe naszego wielomianu to: $-17, 16, -11$.\\
Wielomian jest stopnia nieparzystego, ponadto znak współczynnika przy\linebreak najwyższej potędze x jest ujemny.\\ W związku z tym wykres wielomianu zaczyna się od lewej strony powyżej osi OX. A więc $$x \in (-\infty,-17) \cup (-11,16).$$
\rozwStop
\odpStart
$x \in (-\infty,-17) \cup (-11,16)$
\odpStop
\testStart
A.$x \in (-\infty,-17) \cup (-11,16)$\\
B.$x \in (-\infty,-17) \cup (-11,16]$\\
C.$x \in (-\infty,-17) \cup [-11,16)$\\
D.$x \in (-\infty,-17] \cup (-11,16)$\\
E.$x \in (-\infty,-17] \cup (-11,16]$\\
F.$x \in (-\infty,-17] \cup [-11,16)$\\
G.$x \in (-\infty,-17) \cup [-11,16]$\\
H.$x \in (-\infty,-17] \cup [-11,16]$
\testStop
\kluczStart
A
\kluczStop



\zadStart{Zadanie z Wikieł Z 1.62 b) moja wersja nr 677}

Rozwiązać nierówności $(x+17)(16-x)(x+12)\ge0$.
\zadStop
\rozwStart{Patryk Wirkus}{}
Miejsca zerowe naszego wielomianu to: $-17, 16, -12$.\\
Wielomian jest stopnia nieparzystego, ponadto znak współczynnika przy\linebreak najwyższej potędze x jest ujemny.\\ W związku z tym wykres wielomianu zaczyna się od lewej strony powyżej osi OX. A więc $$x \in (-\infty,-17) \cup (-12,16).$$
\rozwStop
\odpStart
$x \in (-\infty,-17) \cup (-12,16)$
\odpStop
\testStart
A.$x \in (-\infty,-17) \cup (-12,16)$\\
B.$x \in (-\infty,-17) \cup (-12,16]$\\
C.$x \in (-\infty,-17) \cup [-12,16)$\\
D.$x \in (-\infty,-17] \cup (-12,16)$\\
E.$x \in (-\infty,-17] \cup (-12,16]$\\
F.$x \in (-\infty,-17] \cup [-12,16)$\\
G.$x \in (-\infty,-17) \cup [-12,16]$\\
H.$x \in (-\infty,-17] \cup [-12,16]$
\testStop
\kluczStart
A
\kluczStop



\zadStart{Zadanie z Wikieł Z 1.62 b) moja wersja nr 678}

Rozwiązać nierówności $(x+17)(16-x)(x+13)\ge0$.
\zadStop
\rozwStart{Patryk Wirkus}{}
Miejsca zerowe naszego wielomianu to: $-17, 16, -13$.\\
Wielomian jest stopnia nieparzystego, ponadto znak współczynnika przy\linebreak najwyższej potędze x jest ujemny.\\ W związku z tym wykres wielomianu zaczyna się od lewej strony powyżej osi OX. A więc $$x \in (-\infty,-17) \cup (-13,16).$$
\rozwStop
\odpStart
$x \in (-\infty,-17) \cup (-13,16)$
\odpStop
\testStart
A.$x \in (-\infty,-17) \cup (-13,16)$\\
B.$x \in (-\infty,-17) \cup (-13,16]$\\
C.$x \in (-\infty,-17) \cup [-13,16)$\\
D.$x \in (-\infty,-17] \cup (-13,16)$\\
E.$x \in (-\infty,-17] \cup (-13,16]$\\
F.$x \in (-\infty,-17] \cup [-13,16)$\\
G.$x \in (-\infty,-17) \cup [-13,16]$\\
H.$x \in (-\infty,-17] \cup [-13,16]$
\testStop
\kluczStart
A
\kluczStop



\zadStart{Zadanie z Wikieł Z 1.62 b) moja wersja nr 679}

Rozwiązać nierówności $(x+17)(16-x)(x+14)\ge0$.
\zadStop
\rozwStart{Patryk Wirkus}{}
Miejsca zerowe naszego wielomianu to: $-17, 16, -14$.\\
Wielomian jest stopnia nieparzystego, ponadto znak współczynnika przy\linebreak najwyższej potędze x jest ujemny.\\ W związku z tym wykres wielomianu zaczyna się od lewej strony powyżej osi OX. A więc $$x \in (-\infty,-17) \cup (-14,16).$$
\rozwStop
\odpStart
$x \in (-\infty,-17) \cup (-14,16)$
\odpStop
\testStart
A.$x \in (-\infty,-17) \cup (-14,16)$\\
B.$x \in (-\infty,-17) \cup (-14,16]$\\
C.$x \in (-\infty,-17) \cup [-14,16)$\\
D.$x \in (-\infty,-17] \cup (-14,16)$\\
E.$x \in (-\infty,-17] \cup (-14,16]$\\
F.$x \in (-\infty,-17] \cup [-14,16)$\\
G.$x \in (-\infty,-17) \cup [-14,16]$\\
H.$x \in (-\infty,-17] \cup [-14,16]$
\testStop
\kluczStart
A
\kluczStop



\zadStart{Zadanie z Wikieł Z 1.62 b) moja wersja nr 680}

Rozwiązać nierówności $(x+17)(16-x)(x+15)\ge0$.
\zadStop
\rozwStart{Patryk Wirkus}{}
Miejsca zerowe naszego wielomianu to: $-17, 16, -15$.\\
Wielomian jest stopnia nieparzystego, ponadto znak współczynnika przy\linebreak najwyższej potędze x jest ujemny.\\ W związku z tym wykres wielomianu zaczyna się od lewej strony powyżej osi OX. A więc $$x \in (-\infty,-17) \cup (-15,16).$$
\rozwStop
\odpStart
$x \in (-\infty,-17) \cup (-15,16)$
\odpStop
\testStart
A.$x \in (-\infty,-17) \cup (-15,16)$\\
B.$x \in (-\infty,-17) \cup (-15,16]$\\
C.$x \in (-\infty,-17) \cup [-15,16)$\\
D.$x \in (-\infty,-17] \cup (-15,16)$\\
E.$x \in (-\infty,-17] \cup (-15,16]$\\
F.$x \in (-\infty,-17] \cup [-15,16)$\\
G.$x \in (-\infty,-17) \cup [-15,16]$\\
H.$x \in (-\infty,-17] \cup [-15,16]$
\testStop
\kluczStart
A
\kluczStop



\zadStart{Zadanie z Wikieł Z 1.62 b) moja wersja nr 681}

Rozwiązać nierówności $(x+18)(2-x)(x+1)\ge0$.
\zadStop
\rozwStart{Patryk Wirkus}{}
Miejsca zerowe naszego wielomianu to: $-18, 2, -1$.\\
Wielomian jest stopnia nieparzystego, ponadto znak współczynnika przy\linebreak najwyższej potędze x jest ujemny.\\ W związku z tym wykres wielomianu zaczyna się od lewej strony powyżej osi OX. A więc $$x \in (-\infty,-18) \cup (-1,2).$$
\rozwStop
\odpStart
$x \in (-\infty,-18) \cup (-1,2)$
\odpStop
\testStart
A.$x \in (-\infty,-18) \cup (-1,2)$\\
B.$x \in (-\infty,-18) \cup (-1,2]$\\
C.$x \in (-\infty,-18) \cup [-1,2)$\\
D.$x \in (-\infty,-18] \cup (-1,2)$\\
E.$x \in (-\infty,-18] \cup (-1,2]$\\
F.$x \in (-\infty,-18] \cup [-1,2)$\\
G.$x \in (-\infty,-18) \cup [-1,2]$\\
H.$x \in (-\infty,-18] \cup [-1,2]$
\testStop
\kluczStart
A
\kluczStop



\zadStart{Zadanie z Wikieł Z 1.62 b) moja wersja nr 682}

Rozwiązać nierówności $(x+18)(3-x)(x+1)\ge0$.
\zadStop
\rozwStart{Patryk Wirkus}{}
Miejsca zerowe naszego wielomianu to: $-18, 3, -1$.\\
Wielomian jest stopnia nieparzystego, ponadto znak współczynnika przy\linebreak najwyższej potędze x jest ujemny.\\ W związku z tym wykres wielomianu zaczyna się od lewej strony powyżej osi OX. A więc $$x \in (-\infty,-18) \cup (-1,3).$$
\rozwStop
\odpStart
$x \in (-\infty,-18) \cup (-1,3)$
\odpStop
\testStart
A.$x \in (-\infty,-18) \cup (-1,3)$\\
B.$x \in (-\infty,-18) \cup (-1,3]$\\
C.$x \in (-\infty,-18) \cup [-1,3)$\\
D.$x \in (-\infty,-18] \cup (-1,3)$\\
E.$x \in (-\infty,-18] \cup (-1,3]$\\
F.$x \in (-\infty,-18] \cup [-1,3)$\\
G.$x \in (-\infty,-18) \cup [-1,3]$\\
H.$x \in (-\infty,-18] \cup [-1,3]$
\testStop
\kluczStart
A
\kluczStop



\zadStart{Zadanie z Wikieł Z 1.62 b) moja wersja nr 683}

Rozwiązać nierówności $(x+18)(3-x)(x+2)\ge0$.
\zadStop
\rozwStart{Patryk Wirkus}{}
Miejsca zerowe naszego wielomianu to: $-18, 3, -2$.\\
Wielomian jest stopnia nieparzystego, ponadto znak współczynnika przy\linebreak najwyższej potędze x jest ujemny.\\ W związku z tym wykres wielomianu zaczyna się od lewej strony powyżej osi OX. A więc $$x \in (-\infty,-18) \cup (-2,3).$$
\rozwStop
\odpStart
$x \in (-\infty,-18) \cup (-2,3)$
\odpStop
\testStart
A.$x \in (-\infty,-18) \cup (-2,3)$\\
B.$x \in (-\infty,-18) \cup (-2,3]$\\
C.$x \in (-\infty,-18) \cup [-2,3)$\\
D.$x \in (-\infty,-18] \cup (-2,3)$\\
E.$x \in (-\infty,-18] \cup (-2,3]$\\
F.$x \in (-\infty,-18] \cup [-2,3)$\\
G.$x \in (-\infty,-18) \cup [-2,3]$\\
H.$x \in (-\infty,-18] \cup [-2,3]$
\testStop
\kluczStart
A
\kluczStop



\zadStart{Zadanie z Wikieł Z 1.62 b) moja wersja nr 684}

Rozwiązać nierówności $(x+18)(4-x)(x+1)\ge0$.
\zadStop
\rozwStart{Patryk Wirkus}{}
Miejsca zerowe naszego wielomianu to: $-18, 4, -1$.\\
Wielomian jest stopnia nieparzystego, ponadto znak współczynnika przy\linebreak najwyższej potędze x jest ujemny.\\ W związku z tym wykres wielomianu zaczyna się od lewej strony powyżej osi OX. A więc $$x \in (-\infty,-18) \cup (-1,4).$$
\rozwStop
\odpStart
$x \in (-\infty,-18) \cup (-1,4)$
\odpStop
\testStart
A.$x \in (-\infty,-18) \cup (-1,4)$\\
B.$x \in (-\infty,-18) \cup (-1,4]$\\
C.$x \in (-\infty,-18) \cup [-1,4)$\\
D.$x \in (-\infty,-18] \cup (-1,4)$\\
E.$x \in (-\infty,-18] \cup (-1,4]$\\
F.$x \in (-\infty,-18] \cup [-1,4)$\\
G.$x \in (-\infty,-18) \cup [-1,4]$\\
H.$x \in (-\infty,-18] \cup [-1,4]$
\testStop
\kluczStart
A
\kluczStop



\zadStart{Zadanie z Wikieł Z 1.62 b) moja wersja nr 685}

Rozwiązać nierówności $(x+18)(4-x)(x+2)\ge0$.
\zadStop
\rozwStart{Patryk Wirkus}{}
Miejsca zerowe naszego wielomianu to: $-18, 4, -2$.\\
Wielomian jest stopnia nieparzystego, ponadto znak współczynnika przy\linebreak najwyższej potędze x jest ujemny.\\ W związku z tym wykres wielomianu zaczyna się od lewej strony powyżej osi OX. A więc $$x \in (-\infty,-18) \cup (-2,4).$$
\rozwStop
\odpStart
$x \in (-\infty,-18) \cup (-2,4)$
\odpStop
\testStart
A.$x \in (-\infty,-18) \cup (-2,4)$\\
B.$x \in (-\infty,-18) \cup (-2,4]$\\
C.$x \in (-\infty,-18) \cup [-2,4)$\\
D.$x \in (-\infty,-18] \cup (-2,4)$\\
E.$x \in (-\infty,-18] \cup (-2,4]$\\
F.$x \in (-\infty,-18] \cup [-2,4)$\\
G.$x \in (-\infty,-18) \cup [-2,4]$\\
H.$x \in (-\infty,-18] \cup [-2,4]$
\testStop
\kluczStart
A
\kluczStop



\zadStart{Zadanie z Wikieł Z 1.62 b) moja wersja nr 686}

Rozwiązać nierówności $(x+18)(4-x)(x+3)\ge0$.
\zadStop
\rozwStart{Patryk Wirkus}{}
Miejsca zerowe naszego wielomianu to: $-18, 4, -3$.\\
Wielomian jest stopnia nieparzystego, ponadto znak współczynnika przy\linebreak najwyższej potędze x jest ujemny.\\ W związku z tym wykres wielomianu zaczyna się od lewej strony powyżej osi OX. A więc $$x \in (-\infty,-18) \cup (-3,4).$$
\rozwStop
\odpStart
$x \in (-\infty,-18) \cup (-3,4)$
\odpStop
\testStart
A.$x \in (-\infty,-18) \cup (-3,4)$\\
B.$x \in (-\infty,-18) \cup (-3,4]$\\
C.$x \in (-\infty,-18) \cup [-3,4)$\\
D.$x \in (-\infty,-18] \cup (-3,4)$\\
E.$x \in (-\infty,-18] \cup (-3,4]$\\
F.$x \in (-\infty,-18] \cup [-3,4)$\\
G.$x \in (-\infty,-18) \cup [-3,4]$\\
H.$x \in (-\infty,-18] \cup [-3,4]$
\testStop
\kluczStart
A
\kluczStop



\zadStart{Zadanie z Wikieł Z 1.62 b) moja wersja nr 687}

Rozwiązać nierówności $(x+18)(5-x)(x+1)\ge0$.
\zadStop
\rozwStart{Patryk Wirkus}{}
Miejsca zerowe naszego wielomianu to: $-18, 5, -1$.\\
Wielomian jest stopnia nieparzystego, ponadto znak współczynnika przy\linebreak najwyższej potędze x jest ujemny.\\ W związku z tym wykres wielomianu zaczyna się od lewej strony powyżej osi OX. A więc $$x \in (-\infty,-18) \cup (-1,5).$$
\rozwStop
\odpStart
$x \in (-\infty,-18) \cup (-1,5)$
\odpStop
\testStart
A.$x \in (-\infty,-18) \cup (-1,5)$\\
B.$x \in (-\infty,-18) \cup (-1,5]$\\
C.$x \in (-\infty,-18) \cup [-1,5)$\\
D.$x \in (-\infty,-18] \cup (-1,5)$\\
E.$x \in (-\infty,-18] \cup (-1,5]$\\
F.$x \in (-\infty,-18] \cup [-1,5)$\\
G.$x \in (-\infty,-18) \cup [-1,5]$\\
H.$x \in (-\infty,-18] \cup [-1,5]$
\testStop
\kluczStart
A
\kluczStop



\zadStart{Zadanie z Wikieł Z 1.62 b) moja wersja nr 688}

Rozwiązać nierówności $(x+18)(5-x)(x+2)\ge0$.
\zadStop
\rozwStart{Patryk Wirkus}{}
Miejsca zerowe naszego wielomianu to: $-18, 5, -2$.\\
Wielomian jest stopnia nieparzystego, ponadto znak współczynnika przy\linebreak najwyższej potędze x jest ujemny.\\ W związku z tym wykres wielomianu zaczyna się od lewej strony powyżej osi OX. A więc $$x \in (-\infty,-18) \cup (-2,5).$$
\rozwStop
\odpStart
$x \in (-\infty,-18) \cup (-2,5)$
\odpStop
\testStart
A.$x \in (-\infty,-18) \cup (-2,5)$\\
B.$x \in (-\infty,-18) \cup (-2,5]$\\
C.$x \in (-\infty,-18) \cup [-2,5)$\\
D.$x \in (-\infty,-18] \cup (-2,5)$\\
E.$x \in (-\infty,-18] \cup (-2,5]$\\
F.$x \in (-\infty,-18] \cup [-2,5)$\\
G.$x \in (-\infty,-18) \cup [-2,5]$\\
H.$x \in (-\infty,-18] \cup [-2,5]$
\testStop
\kluczStart
A
\kluczStop



\zadStart{Zadanie z Wikieł Z 1.62 b) moja wersja nr 689}

Rozwiązać nierówności $(x+18)(5-x)(x+3)\ge0$.
\zadStop
\rozwStart{Patryk Wirkus}{}
Miejsca zerowe naszego wielomianu to: $-18, 5, -3$.\\
Wielomian jest stopnia nieparzystego, ponadto znak współczynnika przy\linebreak najwyższej potędze x jest ujemny.\\ W związku z tym wykres wielomianu zaczyna się od lewej strony powyżej osi OX. A więc $$x \in (-\infty,-18) \cup (-3,5).$$
\rozwStop
\odpStart
$x \in (-\infty,-18) \cup (-3,5)$
\odpStop
\testStart
A.$x \in (-\infty,-18) \cup (-3,5)$\\
B.$x \in (-\infty,-18) \cup (-3,5]$\\
C.$x \in (-\infty,-18) \cup [-3,5)$\\
D.$x \in (-\infty,-18] \cup (-3,5)$\\
E.$x \in (-\infty,-18] \cup (-3,5]$\\
F.$x \in (-\infty,-18] \cup [-3,5)$\\
G.$x \in (-\infty,-18) \cup [-3,5]$\\
H.$x \in (-\infty,-18] \cup [-3,5]$
\testStop
\kluczStart
A
\kluczStop



\zadStart{Zadanie z Wikieł Z 1.62 b) moja wersja nr 690}

Rozwiązać nierówności $(x+18)(5-x)(x+4)\ge0$.
\zadStop
\rozwStart{Patryk Wirkus}{}
Miejsca zerowe naszego wielomianu to: $-18, 5, -4$.\\
Wielomian jest stopnia nieparzystego, ponadto znak współczynnika przy\linebreak najwyższej potędze x jest ujemny.\\ W związku z tym wykres wielomianu zaczyna się od lewej strony powyżej osi OX. A więc $$x \in (-\infty,-18) \cup (-4,5).$$
\rozwStop
\odpStart
$x \in (-\infty,-18) \cup (-4,5)$
\odpStop
\testStart
A.$x \in (-\infty,-18) \cup (-4,5)$\\
B.$x \in (-\infty,-18) \cup (-4,5]$\\
C.$x \in (-\infty,-18) \cup [-4,5)$\\
D.$x \in (-\infty,-18] \cup (-4,5)$\\
E.$x \in (-\infty,-18] \cup (-4,5]$\\
F.$x \in (-\infty,-18] \cup [-4,5)$\\
G.$x \in (-\infty,-18) \cup [-4,5]$\\
H.$x \in (-\infty,-18] \cup [-4,5]$
\testStop
\kluczStart
A
\kluczStop



\zadStart{Zadanie z Wikieł Z 1.62 b) moja wersja nr 691}

Rozwiązać nierówności $(x+18)(6-x)(x+1)\ge0$.
\zadStop
\rozwStart{Patryk Wirkus}{}
Miejsca zerowe naszego wielomianu to: $-18, 6, -1$.\\
Wielomian jest stopnia nieparzystego, ponadto znak współczynnika przy\linebreak najwyższej potędze x jest ujemny.\\ W związku z tym wykres wielomianu zaczyna się od lewej strony powyżej osi OX. A więc $$x \in (-\infty,-18) \cup (-1,6).$$
\rozwStop
\odpStart
$x \in (-\infty,-18) \cup (-1,6)$
\odpStop
\testStart
A.$x \in (-\infty,-18) \cup (-1,6)$\\
B.$x \in (-\infty,-18) \cup (-1,6]$\\
C.$x \in (-\infty,-18) \cup [-1,6)$\\
D.$x \in (-\infty,-18] \cup (-1,6)$\\
E.$x \in (-\infty,-18] \cup (-1,6]$\\
F.$x \in (-\infty,-18] \cup [-1,6)$\\
G.$x \in (-\infty,-18) \cup [-1,6]$\\
H.$x \in (-\infty,-18] \cup [-1,6]$
\testStop
\kluczStart
A
\kluczStop



\zadStart{Zadanie z Wikieł Z 1.62 b) moja wersja nr 692}

Rozwiązać nierówności $(x+18)(6-x)(x+2)\ge0$.
\zadStop
\rozwStart{Patryk Wirkus}{}
Miejsca zerowe naszego wielomianu to: $-18, 6, -2$.\\
Wielomian jest stopnia nieparzystego, ponadto znak współczynnika przy\linebreak najwyższej potędze x jest ujemny.\\ W związku z tym wykres wielomianu zaczyna się od lewej strony powyżej osi OX. A więc $$x \in (-\infty,-18) \cup (-2,6).$$
\rozwStop
\odpStart
$x \in (-\infty,-18) \cup (-2,6)$
\odpStop
\testStart
A.$x \in (-\infty,-18) \cup (-2,6)$\\
B.$x \in (-\infty,-18) \cup (-2,6]$\\
C.$x \in (-\infty,-18) \cup [-2,6)$\\
D.$x \in (-\infty,-18] \cup (-2,6)$\\
E.$x \in (-\infty,-18] \cup (-2,6]$\\
F.$x \in (-\infty,-18] \cup [-2,6)$\\
G.$x \in (-\infty,-18) \cup [-2,6]$\\
H.$x \in (-\infty,-18] \cup [-2,6]$
\testStop
\kluczStart
A
\kluczStop



\zadStart{Zadanie z Wikieł Z 1.62 b) moja wersja nr 693}

Rozwiązać nierówności $(x+18)(6-x)(x+3)\ge0$.
\zadStop
\rozwStart{Patryk Wirkus}{}
Miejsca zerowe naszego wielomianu to: $-18, 6, -3$.\\
Wielomian jest stopnia nieparzystego, ponadto znak współczynnika przy\linebreak najwyższej potędze x jest ujemny.\\ W związku z tym wykres wielomianu zaczyna się od lewej strony powyżej osi OX. A więc $$x \in (-\infty,-18) \cup (-3,6).$$
\rozwStop
\odpStart
$x \in (-\infty,-18) \cup (-3,6)$
\odpStop
\testStart
A.$x \in (-\infty,-18) \cup (-3,6)$\\
B.$x \in (-\infty,-18) \cup (-3,6]$\\
C.$x \in (-\infty,-18) \cup [-3,6)$\\
D.$x \in (-\infty,-18] \cup (-3,6)$\\
E.$x \in (-\infty,-18] \cup (-3,6]$\\
F.$x \in (-\infty,-18] \cup [-3,6)$\\
G.$x \in (-\infty,-18) \cup [-3,6]$\\
H.$x \in (-\infty,-18] \cup [-3,6]$
\testStop
\kluczStart
A
\kluczStop



\zadStart{Zadanie z Wikieł Z 1.62 b) moja wersja nr 694}

Rozwiązać nierówności $(x+18)(6-x)(x+4)\ge0$.
\zadStop
\rozwStart{Patryk Wirkus}{}
Miejsca zerowe naszego wielomianu to: $-18, 6, -4$.\\
Wielomian jest stopnia nieparzystego, ponadto znak współczynnika przy\linebreak najwyższej potędze x jest ujemny.\\ W związku z tym wykres wielomianu zaczyna się od lewej strony powyżej osi OX. A więc $$x \in (-\infty,-18) \cup (-4,6).$$
\rozwStop
\odpStart
$x \in (-\infty,-18) \cup (-4,6)$
\odpStop
\testStart
A.$x \in (-\infty,-18) \cup (-4,6)$\\
B.$x \in (-\infty,-18) \cup (-4,6]$\\
C.$x \in (-\infty,-18) \cup [-4,6)$\\
D.$x \in (-\infty,-18] \cup (-4,6)$\\
E.$x \in (-\infty,-18] \cup (-4,6]$\\
F.$x \in (-\infty,-18] \cup [-4,6)$\\
G.$x \in (-\infty,-18) \cup [-4,6]$\\
H.$x \in (-\infty,-18] \cup [-4,6]$
\testStop
\kluczStart
A
\kluczStop



\zadStart{Zadanie z Wikieł Z 1.62 b) moja wersja nr 695}

Rozwiązać nierówności $(x+18)(6-x)(x+5)\ge0$.
\zadStop
\rozwStart{Patryk Wirkus}{}
Miejsca zerowe naszego wielomianu to: $-18, 6, -5$.\\
Wielomian jest stopnia nieparzystego, ponadto znak współczynnika przy\linebreak najwyższej potędze x jest ujemny.\\ W związku z tym wykres wielomianu zaczyna się od lewej strony powyżej osi OX. A więc $$x \in (-\infty,-18) \cup (-5,6).$$
\rozwStop
\odpStart
$x \in (-\infty,-18) \cup (-5,6)$
\odpStop
\testStart
A.$x \in (-\infty,-18) \cup (-5,6)$\\
B.$x \in (-\infty,-18) \cup (-5,6]$\\
C.$x \in (-\infty,-18) \cup [-5,6)$\\
D.$x \in (-\infty,-18] \cup (-5,6)$\\
E.$x \in (-\infty,-18] \cup (-5,6]$\\
F.$x \in (-\infty,-18] \cup [-5,6)$\\
G.$x \in (-\infty,-18) \cup [-5,6]$\\
H.$x \in (-\infty,-18] \cup [-5,6]$
\testStop
\kluczStart
A
\kluczStop



\zadStart{Zadanie z Wikieł Z 1.62 b) moja wersja nr 696}

Rozwiązać nierówności $(x+18)(7-x)(x+1)\ge0$.
\zadStop
\rozwStart{Patryk Wirkus}{}
Miejsca zerowe naszego wielomianu to: $-18, 7, -1$.\\
Wielomian jest stopnia nieparzystego, ponadto znak współczynnika przy\linebreak najwyższej potędze x jest ujemny.\\ W związku z tym wykres wielomianu zaczyna się od lewej strony powyżej osi OX. A więc $$x \in (-\infty,-18) \cup (-1,7).$$
\rozwStop
\odpStart
$x \in (-\infty,-18) \cup (-1,7)$
\odpStop
\testStart
A.$x \in (-\infty,-18) \cup (-1,7)$\\
B.$x \in (-\infty,-18) \cup (-1,7]$\\
C.$x \in (-\infty,-18) \cup [-1,7)$\\
D.$x \in (-\infty,-18] \cup (-1,7)$\\
E.$x \in (-\infty,-18] \cup (-1,7]$\\
F.$x \in (-\infty,-18] \cup [-1,7)$\\
G.$x \in (-\infty,-18) \cup [-1,7]$\\
H.$x \in (-\infty,-18] \cup [-1,7]$
\testStop
\kluczStart
A
\kluczStop



\zadStart{Zadanie z Wikieł Z 1.62 b) moja wersja nr 697}

Rozwiązać nierówności $(x+18)(7-x)(x+2)\ge0$.
\zadStop
\rozwStart{Patryk Wirkus}{}
Miejsca zerowe naszego wielomianu to: $-18, 7, -2$.\\
Wielomian jest stopnia nieparzystego, ponadto znak współczynnika przy\linebreak najwyższej potędze x jest ujemny.\\ W związku z tym wykres wielomianu zaczyna się od lewej strony powyżej osi OX. A więc $$x \in (-\infty,-18) \cup (-2,7).$$
\rozwStop
\odpStart
$x \in (-\infty,-18) \cup (-2,7)$
\odpStop
\testStart
A.$x \in (-\infty,-18) \cup (-2,7)$\\
B.$x \in (-\infty,-18) \cup (-2,7]$\\
C.$x \in (-\infty,-18) \cup [-2,7)$\\
D.$x \in (-\infty,-18] \cup (-2,7)$\\
E.$x \in (-\infty,-18] \cup (-2,7]$\\
F.$x \in (-\infty,-18] \cup [-2,7)$\\
G.$x \in (-\infty,-18) \cup [-2,7]$\\
H.$x \in (-\infty,-18] \cup [-2,7]$
\testStop
\kluczStart
A
\kluczStop



\zadStart{Zadanie z Wikieł Z 1.62 b) moja wersja nr 698}

Rozwiązać nierówności $(x+18)(7-x)(x+3)\ge0$.
\zadStop
\rozwStart{Patryk Wirkus}{}
Miejsca zerowe naszego wielomianu to: $-18, 7, -3$.\\
Wielomian jest stopnia nieparzystego, ponadto znak współczynnika przy\linebreak najwyższej potędze x jest ujemny.\\ W związku z tym wykres wielomianu zaczyna się od lewej strony powyżej osi OX. A więc $$x \in (-\infty,-18) \cup (-3,7).$$
\rozwStop
\odpStart
$x \in (-\infty,-18) \cup (-3,7)$
\odpStop
\testStart
A.$x \in (-\infty,-18) \cup (-3,7)$\\
B.$x \in (-\infty,-18) \cup (-3,7]$\\
C.$x \in (-\infty,-18) \cup [-3,7)$\\
D.$x \in (-\infty,-18] \cup (-3,7)$\\
E.$x \in (-\infty,-18] \cup (-3,7]$\\
F.$x \in (-\infty,-18] \cup [-3,7)$\\
G.$x \in (-\infty,-18) \cup [-3,7]$\\
H.$x \in (-\infty,-18] \cup [-3,7]$
\testStop
\kluczStart
A
\kluczStop



\zadStart{Zadanie z Wikieł Z 1.62 b) moja wersja nr 699}

Rozwiązać nierówności $(x+18)(7-x)(x+4)\ge0$.
\zadStop
\rozwStart{Patryk Wirkus}{}
Miejsca zerowe naszego wielomianu to: $-18, 7, -4$.\\
Wielomian jest stopnia nieparzystego, ponadto znak współczynnika przy\linebreak najwyższej potędze x jest ujemny.\\ W związku z tym wykres wielomianu zaczyna się od lewej strony powyżej osi OX. A więc $$x \in (-\infty,-18) \cup (-4,7).$$
\rozwStop
\odpStart
$x \in (-\infty,-18) \cup (-4,7)$
\odpStop
\testStart
A.$x \in (-\infty,-18) \cup (-4,7)$\\
B.$x \in (-\infty,-18) \cup (-4,7]$\\
C.$x \in (-\infty,-18) \cup [-4,7)$\\
D.$x \in (-\infty,-18] \cup (-4,7)$\\
E.$x \in (-\infty,-18] \cup (-4,7]$\\
F.$x \in (-\infty,-18] \cup [-4,7)$\\
G.$x \in (-\infty,-18) \cup [-4,7]$\\
H.$x \in (-\infty,-18] \cup [-4,7]$
\testStop
\kluczStart
A
\kluczStop



\zadStart{Zadanie z Wikieł Z 1.62 b) moja wersja nr 700}

Rozwiązać nierówności $(x+18)(7-x)(x+5)\ge0$.
\zadStop
\rozwStart{Patryk Wirkus}{}
Miejsca zerowe naszego wielomianu to: $-18, 7, -5$.\\
Wielomian jest stopnia nieparzystego, ponadto znak współczynnika przy\linebreak najwyższej potędze x jest ujemny.\\ W związku z tym wykres wielomianu zaczyna się od lewej strony powyżej osi OX. A więc $$x \in (-\infty,-18) \cup (-5,7).$$
\rozwStop
\odpStart
$x \in (-\infty,-18) \cup (-5,7)$
\odpStop
\testStart
A.$x \in (-\infty,-18) \cup (-5,7)$\\
B.$x \in (-\infty,-18) \cup (-5,7]$\\
C.$x \in (-\infty,-18) \cup [-5,7)$\\
D.$x \in (-\infty,-18] \cup (-5,7)$\\
E.$x \in (-\infty,-18] \cup (-5,7]$\\
F.$x \in (-\infty,-18] \cup [-5,7)$\\
G.$x \in (-\infty,-18) \cup [-5,7]$\\
H.$x \in (-\infty,-18] \cup [-5,7]$
\testStop
\kluczStart
A
\kluczStop



\zadStart{Zadanie z Wikieł Z 1.62 b) moja wersja nr 701}

Rozwiązać nierówności $(x+18)(7-x)(x+6)\ge0$.
\zadStop
\rozwStart{Patryk Wirkus}{}
Miejsca zerowe naszego wielomianu to: $-18, 7, -6$.\\
Wielomian jest stopnia nieparzystego, ponadto znak współczynnika przy\linebreak najwyższej potędze x jest ujemny.\\ W związku z tym wykres wielomianu zaczyna się od lewej strony powyżej osi OX. A więc $$x \in (-\infty,-18) \cup (-6,7).$$
\rozwStop
\odpStart
$x \in (-\infty,-18) \cup (-6,7)$
\odpStop
\testStart
A.$x \in (-\infty,-18) \cup (-6,7)$\\
B.$x \in (-\infty,-18) \cup (-6,7]$\\
C.$x \in (-\infty,-18) \cup [-6,7)$\\
D.$x \in (-\infty,-18] \cup (-6,7)$\\
E.$x \in (-\infty,-18] \cup (-6,7]$\\
F.$x \in (-\infty,-18] \cup [-6,7)$\\
G.$x \in (-\infty,-18) \cup [-6,7]$\\
H.$x \in (-\infty,-18] \cup [-6,7]$
\testStop
\kluczStart
A
\kluczStop



\zadStart{Zadanie z Wikieł Z 1.62 b) moja wersja nr 702}

Rozwiązać nierówności $(x+18)(8-x)(x+1)\ge0$.
\zadStop
\rozwStart{Patryk Wirkus}{}
Miejsca zerowe naszego wielomianu to: $-18, 8, -1$.\\
Wielomian jest stopnia nieparzystego, ponadto znak współczynnika przy\linebreak najwyższej potędze x jest ujemny.\\ W związku z tym wykres wielomianu zaczyna się od lewej strony powyżej osi OX. A więc $$x \in (-\infty,-18) \cup (-1,8).$$
\rozwStop
\odpStart
$x \in (-\infty,-18) \cup (-1,8)$
\odpStop
\testStart
A.$x \in (-\infty,-18) \cup (-1,8)$\\
B.$x \in (-\infty,-18) \cup (-1,8]$\\
C.$x \in (-\infty,-18) \cup [-1,8)$\\
D.$x \in (-\infty,-18] \cup (-1,8)$\\
E.$x \in (-\infty,-18] \cup (-1,8]$\\
F.$x \in (-\infty,-18] \cup [-1,8)$\\
G.$x \in (-\infty,-18) \cup [-1,8]$\\
H.$x \in (-\infty,-18] \cup [-1,8]$
\testStop
\kluczStart
A
\kluczStop



\zadStart{Zadanie z Wikieł Z 1.62 b) moja wersja nr 703}

Rozwiązać nierówności $(x+18)(8-x)(x+2)\ge0$.
\zadStop
\rozwStart{Patryk Wirkus}{}
Miejsca zerowe naszego wielomianu to: $-18, 8, -2$.\\
Wielomian jest stopnia nieparzystego, ponadto znak współczynnika przy\linebreak najwyższej potędze x jest ujemny.\\ W związku z tym wykres wielomianu zaczyna się od lewej strony powyżej osi OX. A więc $$x \in (-\infty,-18) \cup (-2,8).$$
\rozwStop
\odpStart
$x \in (-\infty,-18) \cup (-2,8)$
\odpStop
\testStart
A.$x \in (-\infty,-18) \cup (-2,8)$\\
B.$x \in (-\infty,-18) \cup (-2,8]$\\
C.$x \in (-\infty,-18) \cup [-2,8)$\\
D.$x \in (-\infty,-18] \cup (-2,8)$\\
E.$x \in (-\infty,-18] \cup (-2,8]$\\
F.$x \in (-\infty,-18] \cup [-2,8)$\\
G.$x \in (-\infty,-18) \cup [-2,8]$\\
H.$x \in (-\infty,-18] \cup [-2,8]$
\testStop
\kluczStart
A
\kluczStop



\zadStart{Zadanie z Wikieł Z 1.62 b) moja wersja nr 704}

Rozwiązać nierówności $(x+18)(8-x)(x+3)\ge0$.
\zadStop
\rozwStart{Patryk Wirkus}{}
Miejsca zerowe naszego wielomianu to: $-18, 8, -3$.\\
Wielomian jest stopnia nieparzystego, ponadto znak współczynnika przy\linebreak najwyższej potędze x jest ujemny.\\ W związku z tym wykres wielomianu zaczyna się od lewej strony powyżej osi OX. A więc $$x \in (-\infty,-18) \cup (-3,8).$$
\rozwStop
\odpStart
$x \in (-\infty,-18) \cup (-3,8)$
\odpStop
\testStart
A.$x \in (-\infty,-18) \cup (-3,8)$\\
B.$x \in (-\infty,-18) \cup (-3,8]$\\
C.$x \in (-\infty,-18) \cup [-3,8)$\\
D.$x \in (-\infty,-18] \cup (-3,8)$\\
E.$x \in (-\infty,-18] \cup (-3,8]$\\
F.$x \in (-\infty,-18] \cup [-3,8)$\\
G.$x \in (-\infty,-18) \cup [-3,8]$\\
H.$x \in (-\infty,-18] \cup [-3,8]$
\testStop
\kluczStart
A
\kluczStop



\zadStart{Zadanie z Wikieł Z 1.62 b) moja wersja nr 705}

Rozwiązać nierówności $(x+18)(8-x)(x+4)\ge0$.
\zadStop
\rozwStart{Patryk Wirkus}{}
Miejsca zerowe naszego wielomianu to: $-18, 8, -4$.\\
Wielomian jest stopnia nieparzystego, ponadto znak współczynnika przy\linebreak najwyższej potędze x jest ujemny.\\ W związku z tym wykres wielomianu zaczyna się od lewej strony powyżej osi OX. A więc $$x \in (-\infty,-18) \cup (-4,8).$$
\rozwStop
\odpStart
$x \in (-\infty,-18) \cup (-4,8)$
\odpStop
\testStart
A.$x \in (-\infty,-18) \cup (-4,8)$\\
B.$x \in (-\infty,-18) \cup (-4,8]$\\
C.$x \in (-\infty,-18) \cup [-4,8)$\\
D.$x \in (-\infty,-18] \cup (-4,8)$\\
E.$x \in (-\infty,-18] \cup (-4,8]$\\
F.$x \in (-\infty,-18] \cup [-4,8)$\\
G.$x \in (-\infty,-18) \cup [-4,8]$\\
H.$x \in (-\infty,-18] \cup [-4,8]$
\testStop
\kluczStart
A
\kluczStop



\zadStart{Zadanie z Wikieł Z 1.62 b) moja wersja nr 706}

Rozwiązać nierówności $(x+18)(8-x)(x+5)\ge0$.
\zadStop
\rozwStart{Patryk Wirkus}{}
Miejsca zerowe naszego wielomianu to: $-18, 8, -5$.\\
Wielomian jest stopnia nieparzystego, ponadto znak współczynnika przy\linebreak najwyższej potędze x jest ujemny.\\ W związku z tym wykres wielomianu zaczyna się od lewej strony powyżej osi OX. A więc $$x \in (-\infty,-18) \cup (-5,8).$$
\rozwStop
\odpStart
$x \in (-\infty,-18) \cup (-5,8)$
\odpStop
\testStart
A.$x \in (-\infty,-18) \cup (-5,8)$\\
B.$x \in (-\infty,-18) \cup (-5,8]$\\
C.$x \in (-\infty,-18) \cup [-5,8)$\\
D.$x \in (-\infty,-18] \cup (-5,8)$\\
E.$x \in (-\infty,-18] \cup (-5,8]$\\
F.$x \in (-\infty,-18] \cup [-5,8)$\\
G.$x \in (-\infty,-18) \cup [-5,8]$\\
H.$x \in (-\infty,-18] \cup [-5,8]$
\testStop
\kluczStart
A
\kluczStop



\zadStart{Zadanie z Wikieł Z 1.62 b) moja wersja nr 707}

Rozwiązać nierówności $(x+18)(8-x)(x+6)\ge0$.
\zadStop
\rozwStart{Patryk Wirkus}{}
Miejsca zerowe naszego wielomianu to: $-18, 8, -6$.\\
Wielomian jest stopnia nieparzystego, ponadto znak współczynnika przy\linebreak najwyższej potędze x jest ujemny.\\ W związku z tym wykres wielomianu zaczyna się od lewej strony powyżej osi OX. A więc $$x \in (-\infty,-18) \cup (-6,8).$$
\rozwStop
\odpStart
$x \in (-\infty,-18) \cup (-6,8)$
\odpStop
\testStart
A.$x \in (-\infty,-18) \cup (-6,8)$\\
B.$x \in (-\infty,-18) \cup (-6,8]$\\
C.$x \in (-\infty,-18) \cup [-6,8)$\\
D.$x \in (-\infty,-18] \cup (-6,8)$\\
E.$x \in (-\infty,-18] \cup (-6,8]$\\
F.$x \in (-\infty,-18] \cup [-6,8)$\\
G.$x \in (-\infty,-18) \cup [-6,8]$\\
H.$x \in (-\infty,-18] \cup [-6,8]$
\testStop
\kluczStart
A
\kluczStop



\zadStart{Zadanie z Wikieł Z 1.62 b) moja wersja nr 708}

Rozwiązać nierówności $(x+18)(8-x)(x+7)\ge0$.
\zadStop
\rozwStart{Patryk Wirkus}{}
Miejsca zerowe naszego wielomianu to: $-18, 8, -7$.\\
Wielomian jest stopnia nieparzystego, ponadto znak współczynnika przy\linebreak najwyższej potędze x jest ujemny.\\ W związku z tym wykres wielomianu zaczyna się od lewej strony powyżej osi OX. A więc $$x \in (-\infty,-18) \cup (-7,8).$$
\rozwStop
\odpStart
$x \in (-\infty,-18) \cup (-7,8)$
\odpStop
\testStart
A.$x \in (-\infty,-18) \cup (-7,8)$\\
B.$x \in (-\infty,-18) \cup (-7,8]$\\
C.$x \in (-\infty,-18) \cup [-7,8)$\\
D.$x \in (-\infty,-18] \cup (-7,8)$\\
E.$x \in (-\infty,-18] \cup (-7,8]$\\
F.$x \in (-\infty,-18] \cup [-7,8)$\\
G.$x \in (-\infty,-18) \cup [-7,8]$\\
H.$x \in (-\infty,-18] \cup [-7,8]$
\testStop
\kluczStart
A
\kluczStop



\zadStart{Zadanie z Wikieł Z 1.62 b) moja wersja nr 709}

Rozwiązać nierówności $(x+18)(9-x)(x+1)\ge0$.
\zadStop
\rozwStart{Patryk Wirkus}{}
Miejsca zerowe naszego wielomianu to: $-18, 9, -1$.\\
Wielomian jest stopnia nieparzystego, ponadto znak współczynnika przy\linebreak najwyższej potędze x jest ujemny.\\ W związku z tym wykres wielomianu zaczyna się od lewej strony powyżej osi OX. A więc $$x \in (-\infty,-18) \cup (-1,9).$$
\rozwStop
\odpStart
$x \in (-\infty,-18) \cup (-1,9)$
\odpStop
\testStart
A.$x \in (-\infty,-18) \cup (-1,9)$\\
B.$x \in (-\infty,-18) \cup (-1,9]$\\
C.$x \in (-\infty,-18) \cup [-1,9)$\\
D.$x \in (-\infty,-18] \cup (-1,9)$\\
E.$x \in (-\infty,-18] \cup (-1,9]$\\
F.$x \in (-\infty,-18] \cup [-1,9)$\\
G.$x \in (-\infty,-18) \cup [-1,9]$\\
H.$x \in (-\infty,-18] \cup [-1,9]$
\testStop
\kluczStart
A
\kluczStop



\zadStart{Zadanie z Wikieł Z 1.62 b) moja wersja nr 710}

Rozwiązać nierówności $(x+18)(9-x)(x+2)\ge0$.
\zadStop
\rozwStart{Patryk Wirkus}{}
Miejsca zerowe naszego wielomianu to: $-18, 9, -2$.\\
Wielomian jest stopnia nieparzystego, ponadto znak współczynnika przy\linebreak najwyższej potędze x jest ujemny.\\ W związku z tym wykres wielomianu zaczyna się od lewej strony powyżej osi OX. A więc $$x \in (-\infty,-18) \cup (-2,9).$$
\rozwStop
\odpStart
$x \in (-\infty,-18) \cup (-2,9)$
\odpStop
\testStart
A.$x \in (-\infty,-18) \cup (-2,9)$\\
B.$x \in (-\infty,-18) \cup (-2,9]$\\
C.$x \in (-\infty,-18) \cup [-2,9)$\\
D.$x \in (-\infty,-18] \cup (-2,9)$\\
E.$x \in (-\infty,-18] \cup (-2,9]$\\
F.$x \in (-\infty,-18] \cup [-2,9)$\\
G.$x \in (-\infty,-18) \cup [-2,9]$\\
H.$x \in (-\infty,-18] \cup [-2,9]$
\testStop
\kluczStart
A
\kluczStop



\zadStart{Zadanie z Wikieł Z 1.62 b) moja wersja nr 711}

Rozwiązać nierówności $(x+18)(9-x)(x+3)\ge0$.
\zadStop
\rozwStart{Patryk Wirkus}{}
Miejsca zerowe naszego wielomianu to: $-18, 9, -3$.\\
Wielomian jest stopnia nieparzystego, ponadto znak współczynnika przy\linebreak najwyższej potędze x jest ujemny.\\ W związku z tym wykres wielomianu zaczyna się od lewej strony powyżej osi OX. A więc $$x \in (-\infty,-18) \cup (-3,9).$$
\rozwStop
\odpStart
$x \in (-\infty,-18) \cup (-3,9)$
\odpStop
\testStart
A.$x \in (-\infty,-18) \cup (-3,9)$\\
B.$x \in (-\infty,-18) \cup (-3,9]$\\
C.$x \in (-\infty,-18) \cup [-3,9)$\\
D.$x \in (-\infty,-18] \cup (-3,9)$\\
E.$x \in (-\infty,-18] \cup (-3,9]$\\
F.$x \in (-\infty,-18] \cup [-3,9)$\\
G.$x \in (-\infty,-18) \cup [-3,9]$\\
H.$x \in (-\infty,-18] \cup [-3,9]$
\testStop
\kluczStart
A
\kluczStop



\zadStart{Zadanie z Wikieł Z 1.62 b) moja wersja nr 712}

Rozwiązać nierówności $(x+18)(9-x)(x+4)\ge0$.
\zadStop
\rozwStart{Patryk Wirkus}{}
Miejsca zerowe naszego wielomianu to: $-18, 9, -4$.\\
Wielomian jest stopnia nieparzystego, ponadto znak współczynnika przy\linebreak najwyższej potędze x jest ujemny.\\ W związku z tym wykres wielomianu zaczyna się od lewej strony powyżej osi OX. A więc $$x \in (-\infty,-18) \cup (-4,9).$$
\rozwStop
\odpStart
$x \in (-\infty,-18) \cup (-4,9)$
\odpStop
\testStart
A.$x \in (-\infty,-18) \cup (-4,9)$\\
B.$x \in (-\infty,-18) \cup (-4,9]$\\
C.$x \in (-\infty,-18) \cup [-4,9)$\\
D.$x \in (-\infty,-18] \cup (-4,9)$\\
E.$x \in (-\infty,-18] \cup (-4,9]$\\
F.$x \in (-\infty,-18] \cup [-4,9)$\\
G.$x \in (-\infty,-18) \cup [-4,9]$\\
H.$x \in (-\infty,-18] \cup [-4,9]$
\testStop
\kluczStart
A
\kluczStop



\zadStart{Zadanie z Wikieł Z 1.62 b) moja wersja nr 713}

Rozwiązać nierówności $(x+18)(9-x)(x+5)\ge0$.
\zadStop
\rozwStart{Patryk Wirkus}{}
Miejsca zerowe naszego wielomianu to: $-18, 9, -5$.\\
Wielomian jest stopnia nieparzystego, ponadto znak współczynnika przy\linebreak najwyższej potędze x jest ujemny.\\ W związku z tym wykres wielomianu zaczyna się od lewej strony powyżej osi OX. A więc $$x \in (-\infty,-18) \cup (-5,9).$$
\rozwStop
\odpStart
$x \in (-\infty,-18) \cup (-5,9)$
\odpStop
\testStart
A.$x \in (-\infty,-18) \cup (-5,9)$\\
B.$x \in (-\infty,-18) \cup (-5,9]$\\
C.$x \in (-\infty,-18) \cup [-5,9)$\\
D.$x \in (-\infty,-18] \cup (-5,9)$\\
E.$x \in (-\infty,-18] \cup (-5,9]$\\
F.$x \in (-\infty,-18] \cup [-5,9)$\\
G.$x \in (-\infty,-18) \cup [-5,9]$\\
H.$x \in (-\infty,-18] \cup [-5,9]$
\testStop
\kluczStart
A
\kluczStop



\zadStart{Zadanie z Wikieł Z 1.62 b) moja wersja nr 714}

Rozwiązać nierówności $(x+18)(9-x)(x+6)\ge0$.
\zadStop
\rozwStart{Patryk Wirkus}{}
Miejsca zerowe naszego wielomianu to: $-18, 9, -6$.\\
Wielomian jest stopnia nieparzystego, ponadto znak współczynnika przy\linebreak najwyższej potędze x jest ujemny.\\ W związku z tym wykres wielomianu zaczyna się od lewej strony powyżej osi OX. A więc $$x \in (-\infty,-18) \cup (-6,9).$$
\rozwStop
\odpStart
$x \in (-\infty,-18) \cup (-6,9)$
\odpStop
\testStart
A.$x \in (-\infty,-18) \cup (-6,9)$\\
B.$x \in (-\infty,-18) \cup (-6,9]$\\
C.$x \in (-\infty,-18) \cup [-6,9)$\\
D.$x \in (-\infty,-18] \cup (-6,9)$\\
E.$x \in (-\infty,-18] \cup (-6,9]$\\
F.$x \in (-\infty,-18] \cup [-6,9)$\\
G.$x \in (-\infty,-18) \cup [-6,9]$\\
H.$x \in (-\infty,-18] \cup [-6,9]$
\testStop
\kluczStart
A
\kluczStop



\zadStart{Zadanie z Wikieł Z 1.62 b) moja wersja nr 715}

Rozwiązać nierówności $(x+18)(9-x)(x+7)\ge0$.
\zadStop
\rozwStart{Patryk Wirkus}{}
Miejsca zerowe naszego wielomianu to: $-18, 9, -7$.\\
Wielomian jest stopnia nieparzystego, ponadto znak współczynnika przy\linebreak najwyższej potędze x jest ujemny.\\ W związku z tym wykres wielomianu zaczyna się od lewej strony powyżej osi OX. A więc $$x \in (-\infty,-18) \cup (-7,9).$$
\rozwStop
\odpStart
$x \in (-\infty,-18) \cup (-7,9)$
\odpStop
\testStart
A.$x \in (-\infty,-18) \cup (-7,9)$\\
B.$x \in (-\infty,-18) \cup (-7,9]$\\
C.$x \in (-\infty,-18) \cup [-7,9)$\\
D.$x \in (-\infty,-18] \cup (-7,9)$\\
E.$x \in (-\infty,-18] \cup (-7,9]$\\
F.$x \in (-\infty,-18] \cup [-7,9)$\\
G.$x \in (-\infty,-18) \cup [-7,9]$\\
H.$x \in (-\infty,-18] \cup [-7,9]$
\testStop
\kluczStart
A
\kluczStop



\zadStart{Zadanie z Wikieł Z 1.62 b) moja wersja nr 716}

Rozwiązać nierówności $(x+18)(9-x)(x+8)\ge0$.
\zadStop
\rozwStart{Patryk Wirkus}{}
Miejsca zerowe naszego wielomianu to: $-18, 9, -8$.\\
Wielomian jest stopnia nieparzystego, ponadto znak współczynnika przy\linebreak najwyższej potędze x jest ujemny.\\ W związku z tym wykres wielomianu zaczyna się od lewej strony powyżej osi OX. A więc $$x \in (-\infty,-18) \cup (-8,9).$$
\rozwStop
\odpStart
$x \in (-\infty,-18) \cup (-8,9)$
\odpStop
\testStart
A.$x \in (-\infty,-18) \cup (-8,9)$\\
B.$x \in (-\infty,-18) \cup (-8,9]$\\
C.$x \in (-\infty,-18) \cup [-8,9)$\\
D.$x \in (-\infty,-18] \cup (-8,9)$\\
E.$x \in (-\infty,-18] \cup (-8,9]$\\
F.$x \in (-\infty,-18] \cup [-8,9)$\\
G.$x \in (-\infty,-18) \cup [-8,9]$\\
H.$x \in (-\infty,-18] \cup [-8,9]$
\testStop
\kluczStart
A
\kluczStop



\zadStart{Zadanie z Wikieł Z 1.62 b) moja wersja nr 717}

Rozwiązać nierówności $(x+18)(10-x)(x+1)\ge0$.
\zadStop
\rozwStart{Patryk Wirkus}{}
Miejsca zerowe naszego wielomianu to: $-18, 10, -1$.\\
Wielomian jest stopnia nieparzystego, ponadto znak współczynnika przy\linebreak najwyższej potędze x jest ujemny.\\ W związku z tym wykres wielomianu zaczyna się od lewej strony powyżej osi OX. A więc $$x \in (-\infty,-18) \cup (-1,10).$$
\rozwStop
\odpStart
$x \in (-\infty,-18) \cup (-1,10)$
\odpStop
\testStart
A.$x \in (-\infty,-18) \cup (-1,10)$\\
B.$x \in (-\infty,-18) \cup (-1,10]$\\
C.$x \in (-\infty,-18) \cup [-1,10)$\\
D.$x \in (-\infty,-18] \cup (-1,10)$\\
E.$x \in (-\infty,-18] \cup (-1,10]$\\
F.$x \in (-\infty,-18] \cup [-1,10)$\\
G.$x \in (-\infty,-18) \cup [-1,10]$\\
H.$x \in (-\infty,-18] \cup [-1,10]$
\testStop
\kluczStart
A
\kluczStop



\zadStart{Zadanie z Wikieł Z 1.62 b) moja wersja nr 718}

Rozwiązać nierówności $(x+18)(10-x)(x+2)\ge0$.
\zadStop
\rozwStart{Patryk Wirkus}{}
Miejsca zerowe naszego wielomianu to: $-18, 10, -2$.\\
Wielomian jest stopnia nieparzystego, ponadto znak współczynnika przy\linebreak najwyższej potędze x jest ujemny.\\ W związku z tym wykres wielomianu zaczyna się od lewej strony powyżej osi OX. A więc $$x \in (-\infty,-18) \cup (-2,10).$$
\rozwStop
\odpStart
$x \in (-\infty,-18) \cup (-2,10)$
\odpStop
\testStart
A.$x \in (-\infty,-18) \cup (-2,10)$\\
B.$x \in (-\infty,-18) \cup (-2,10]$\\
C.$x \in (-\infty,-18) \cup [-2,10)$\\
D.$x \in (-\infty,-18] \cup (-2,10)$\\
E.$x \in (-\infty,-18] \cup (-2,10]$\\
F.$x \in (-\infty,-18] \cup [-2,10)$\\
G.$x \in (-\infty,-18) \cup [-2,10]$\\
H.$x \in (-\infty,-18] \cup [-2,10]$
\testStop
\kluczStart
A
\kluczStop



\zadStart{Zadanie z Wikieł Z 1.62 b) moja wersja nr 719}

Rozwiązać nierówności $(x+18)(10-x)(x+3)\ge0$.
\zadStop
\rozwStart{Patryk Wirkus}{}
Miejsca zerowe naszego wielomianu to: $-18, 10, -3$.\\
Wielomian jest stopnia nieparzystego, ponadto znak współczynnika przy\linebreak najwyższej potędze x jest ujemny.\\ W związku z tym wykres wielomianu zaczyna się od lewej strony powyżej osi OX. A więc $$x \in (-\infty,-18) \cup (-3,10).$$
\rozwStop
\odpStart
$x \in (-\infty,-18) \cup (-3,10)$
\odpStop
\testStart
A.$x \in (-\infty,-18) \cup (-3,10)$\\
B.$x \in (-\infty,-18) \cup (-3,10]$\\
C.$x \in (-\infty,-18) \cup [-3,10)$\\
D.$x \in (-\infty,-18] \cup (-3,10)$\\
E.$x \in (-\infty,-18] \cup (-3,10]$\\
F.$x \in (-\infty,-18] \cup [-3,10)$\\
G.$x \in (-\infty,-18) \cup [-3,10]$\\
H.$x \in (-\infty,-18] \cup [-3,10]$
\testStop
\kluczStart
A
\kluczStop



\zadStart{Zadanie z Wikieł Z 1.62 b) moja wersja nr 720}

Rozwiązać nierówności $(x+18)(10-x)(x+4)\ge0$.
\zadStop
\rozwStart{Patryk Wirkus}{}
Miejsca zerowe naszego wielomianu to: $-18, 10, -4$.\\
Wielomian jest stopnia nieparzystego, ponadto znak współczynnika przy\linebreak najwyższej potędze x jest ujemny.\\ W związku z tym wykres wielomianu zaczyna się od lewej strony powyżej osi OX. A więc $$x \in (-\infty,-18) \cup (-4,10).$$
\rozwStop
\odpStart
$x \in (-\infty,-18) \cup (-4,10)$
\odpStop
\testStart
A.$x \in (-\infty,-18) \cup (-4,10)$\\
B.$x \in (-\infty,-18) \cup (-4,10]$\\
C.$x \in (-\infty,-18) \cup [-4,10)$\\
D.$x \in (-\infty,-18] \cup (-4,10)$\\
E.$x \in (-\infty,-18] \cup (-4,10]$\\
F.$x \in (-\infty,-18] \cup [-4,10)$\\
G.$x \in (-\infty,-18) \cup [-4,10]$\\
H.$x \in (-\infty,-18] \cup [-4,10]$
\testStop
\kluczStart
A
\kluczStop



\zadStart{Zadanie z Wikieł Z 1.62 b) moja wersja nr 721}

Rozwiązać nierówności $(x+18)(10-x)(x+5)\ge0$.
\zadStop
\rozwStart{Patryk Wirkus}{}
Miejsca zerowe naszego wielomianu to: $-18, 10, -5$.\\
Wielomian jest stopnia nieparzystego, ponadto znak współczynnika przy\linebreak najwyższej potędze x jest ujemny.\\ W związku z tym wykres wielomianu zaczyna się od lewej strony powyżej osi OX. A więc $$x \in (-\infty,-18) \cup (-5,10).$$
\rozwStop
\odpStart
$x \in (-\infty,-18) \cup (-5,10)$
\odpStop
\testStart
A.$x \in (-\infty,-18) \cup (-5,10)$\\
B.$x \in (-\infty,-18) \cup (-5,10]$\\
C.$x \in (-\infty,-18) \cup [-5,10)$\\
D.$x \in (-\infty,-18] \cup (-5,10)$\\
E.$x \in (-\infty,-18] \cup (-5,10]$\\
F.$x \in (-\infty,-18] \cup [-5,10)$\\
G.$x \in (-\infty,-18) \cup [-5,10]$\\
H.$x \in (-\infty,-18] \cup [-5,10]$
\testStop
\kluczStart
A
\kluczStop



\zadStart{Zadanie z Wikieł Z 1.62 b) moja wersja nr 722}

Rozwiązać nierówności $(x+18)(10-x)(x+6)\ge0$.
\zadStop
\rozwStart{Patryk Wirkus}{}
Miejsca zerowe naszego wielomianu to: $-18, 10, -6$.\\
Wielomian jest stopnia nieparzystego, ponadto znak współczynnika przy\linebreak najwyższej potędze x jest ujemny.\\ W związku z tym wykres wielomianu zaczyna się od lewej strony powyżej osi OX. A więc $$x \in (-\infty,-18) \cup (-6,10).$$
\rozwStop
\odpStart
$x \in (-\infty,-18) \cup (-6,10)$
\odpStop
\testStart
A.$x \in (-\infty,-18) \cup (-6,10)$\\
B.$x \in (-\infty,-18) \cup (-6,10]$\\
C.$x \in (-\infty,-18) \cup [-6,10)$\\
D.$x \in (-\infty,-18] \cup (-6,10)$\\
E.$x \in (-\infty,-18] \cup (-6,10]$\\
F.$x \in (-\infty,-18] \cup [-6,10)$\\
G.$x \in (-\infty,-18) \cup [-6,10]$\\
H.$x \in (-\infty,-18] \cup [-6,10]$
\testStop
\kluczStart
A
\kluczStop



\zadStart{Zadanie z Wikieł Z 1.62 b) moja wersja nr 723}

Rozwiązać nierówności $(x+18)(10-x)(x+7)\ge0$.
\zadStop
\rozwStart{Patryk Wirkus}{}
Miejsca zerowe naszego wielomianu to: $-18, 10, -7$.\\
Wielomian jest stopnia nieparzystego, ponadto znak współczynnika przy\linebreak najwyższej potędze x jest ujemny.\\ W związku z tym wykres wielomianu zaczyna się od lewej strony powyżej osi OX. A więc $$x \in (-\infty,-18) \cup (-7,10).$$
\rozwStop
\odpStart
$x \in (-\infty,-18) \cup (-7,10)$
\odpStop
\testStart
A.$x \in (-\infty,-18) \cup (-7,10)$\\
B.$x \in (-\infty,-18) \cup (-7,10]$\\
C.$x \in (-\infty,-18) \cup [-7,10)$\\
D.$x \in (-\infty,-18] \cup (-7,10)$\\
E.$x \in (-\infty,-18] \cup (-7,10]$\\
F.$x \in (-\infty,-18] \cup [-7,10)$\\
G.$x \in (-\infty,-18) \cup [-7,10]$\\
H.$x \in (-\infty,-18] \cup [-7,10]$
\testStop
\kluczStart
A
\kluczStop



\zadStart{Zadanie z Wikieł Z 1.62 b) moja wersja nr 724}

Rozwiązać nierówności $(x+18)(10-x)(x+8)\ge0$.
\zadStop
\rozwStart{Patryk Wirkus}{}
Miejsca zerowe naszego wielomianu to: $-18, 10, -8$.\\
Wielomian jest stopnia nieparzystego, ponadto znak współczynnika przy\linebreak najwyższej potędze x jest ujemny.\\ W związku z tym wykres wielomianu zaczyna się od lewej strony powyżej osi OX. A więc $$x \in (-\infty,-18) \cup (-8,10).$$
\rozwStop
\odpStart
$x \in (-\infty,-18) \cup (-8,10)$
\odpStop
\testStart
A.$x \in (-\infty,-18) \cup (-8,10)$\\
B.$x \in (-\infty,-18) \cup (-8,10]$\\
C.$x \in (-\infty,-18) \cup [-8,10)$\\
D.$x \in (-\infty,-18] \cup (-8,10)$\\
E.$x \in (-\infty,-18] \cup (-8,10]$\\
F.$x \in (-\infty,-18] \cup [-8,10)$\\
G.$x \in (-\infty,-18) \cup [-8,10]$\\
H.$x \in (-\infty,-18] \cup [-8,10]$
\testStop
\kluczStart
A
\kluczStop



\zadStart{Zadanie z Wikieł Z 1.62 b) moja wersja nr 725}

Rozwiązać nierówności $(x+18)(10-x)(x+9)\ge0$.
\zadStop
\rozwStart{Patryk Wirkus}{}
Miejsca zerowe naszego wielomianu to: $-18, 10, -9$.\\
Wielomian jest stopnia nieparzystego, ponadto znak współczynnika przy\linebreak najwyższej potędze x jest ujemny.\\ W związku z tym wykres wielomianu zaczyna się od lewej strony powyżej osi OX. A więc $$x \in (-\infty,-18) \cup (-9,10).$$
\rozwStop
\odpStart
$x \in (-\infty,-18) \cup (-9,10)$
\odpStop
\testStart
A.$x \in (-\infty,-18) \cup (-9,10)$\\
B.$x \in (-\infty,-18) \cup (-9,10]$\\
C.$x \in (-\infty,-18) \cup [-9,10)$\\
D.$x \in (-\infty,-18] \cup (-9,10)$\\
E.$x \in (-\infty,-18] \cup (-9,10]$\\
F.$x \in (-\infty,-18] \cup [-9,10)$\\
G.$x \in (-\infty,-18) \cup [-9,10]$\\
H.$x \in (-\infty,-18] \cup [-9,10]$
\testStop
\kluczStart
A
\kluczStop



\zadStart{Zadanie z Wikieł Z 1.62 b) moja wersja nr 726}

Rozwiązać nierówności $(x+18)(11-x)(x+1)\ge0$.
\zadStop
\rozwStart{Patryk Wirkus}{}
Miejsca zerowe naszego wielomianu to: $-18, 11, -1$.\\
Wielomian jest stopnia nieparzystego, ponadto znak współczynnika przy\linebreak najwyższej potędze x jest ujemny.\\ W związku z tym wykres wielomianu zaczyna się od lewej strony powyżej osi OX. A więc $$x \in (-\infty,-18) \cup (-1,11).$$
\rozwStop
\odpStart
$x \in (-\infty,-18) \cup (-1,11)$
\odpStop
\testStart
A.$x \in (-\infty,-18) \cup (-1,11)$\\
B.$x \in (-\infty,-18) \cup (-1,11]$\\
C.$x \in (-\infty,-18) \cup [-1,11)$\\
D.$x \in (-\infty,-18] \cup (-1,11)$\\
E.$x \in (-\infty,-18] \cup (-1,11]$\\
F.$x \in (-\infty,-18] \cup [-1,11)$\\
G.$x \in (-\infty,-18) \cup [-1,11]$\\
H.$x \in (-\infty,-18] \cup [-1,11]$
\testStop
\kluczStart
A
\kluczStop



\zadStart{Zadanie z Wikieł Z 1.62 b) moja wersja nr 727}

Rozwiązać nierówności $(x+18)(11-x)(x+2)\ge0$.
\zadStop
\rozwStart{Patryk Wirkus}{}
Miejsca zerowe naszego wielomianu to: $-18, 11, -2$.\\
Wielomian jest stopnia nieparzystego, ponadto znak współczynnika przy\linebreak najwyższej potędze x jest ujemny.\\ W związku z tym wykres wielomianu zaczyna się od lewej strony powyżej osi OX. A więc $$x \in (-\infty,-18) \cup (-2,11).$$
\rozwStop
\odpStart
$x \in (-\infty,-18) \cup (-2,11)$
\odpStop
\testStart
A.$x \in (-\infty,-18) \cup (-2,11)$\\
B.$x \in (-\infty,-18) \cup (-2,11]$\\
C.$x \in (-\infty,-18) \cup [-2,11)$\\
D.$x \in (-\infty,-18] \cup (-2,11)$\\
E.$x \in (-\infty,-18] \cup (-2,11]$\\
F.$x \in (-\infty,-18] \cup [-2,11)$\\
G.$x \in (-\infty,-18) \cup [-2,11]$\\
H.$x \in (-\infty,-18] \cup [-2,11]$
\testStop
\kluczStart
A
\kluczStop



\zadStart{Zadanie z Wikieł Z 1.62 b) moja wersja nr 728}

Rozwiązać nierówności $(x+18)(11-x)(x+3)\ge0$.
\zadStop
\rozwStart{Patryk Wirkus}{}
Miejsca zerowe naszego wielomianu to: $-18, 11, -3$.\\
Wielomian jest stopnia nieparzystego, ponadto znak współczynnika przy\linebreak najwyższej potędze x jest ujemny.\\ W związku z tym wykres wielomianu zaczyna się od lewej strony powyżej osi OX. A więc $$x \in (-\infty,-18) \cup (-3,11).$$
\rozwStop
\odpStart
$x \in (-\infty,-18) \cup (-3,11)$
\odpStop
\testStart
A.$x \in (-\infty,-18) \cup (-3,11)$\\
B.$x \in (-\infty,-18) \cup (-3,11]$\\
C.$x \in (-\infty,-18) \cup [-3,11)$\\
D.$x \in (-\infty,-18] \cup (-3,11)$\\
E.$x \in (-\infty,-18] \cup (-3,11]$\\
F.$x \in (-\infty,-18] \cup [-3,11)$\\
G.$x \in (-\infty,-18) \cup [-3,11]$\\
H.$x \in (-\infty,-18] \cup [-3,11]$
\testStop
\kluczStart
A
\kluczStop



\zadStart{Zadanie z Wikieł Z 1.62 b) moja wersja nr 729}

Rozwiązać nierówności $(x+18)(11-x)(x+4)\ge0$.
\zadStop
\rozwStart{Patryk Wirkus}{}
Miejsca zerowe naszego wielomianu to: $-18, 11, -4$.\\
Wielomian jest stopnia nieparzystego, ponadto znak współczynnika przy\linebreak najwyższej potędze x jest ujemny.\\ W związku z tym wykres wielomianu zaczyna się od lewej strony powyżej osi OX. A więc $$x \in (-\infty,-18) \cup (-4,11).$$
\rozwStop
\odpStart
$x \in (-\infty,-18) \cup (-4,11)$
\odpStop
\testStart
A.$x \in (-\infty,-18) \cup (-4,11)$\\
B.$x \in (-\infty,-18) \cup (-4,11]$\\
C.$x \in (-\infty,-18) \cup [-4,11)$\\
D.$x \in (-\infty,-18] \cup (-4,11)$\\
E.$x \in (-\infty,-18] \cup (-4,11]$\\
F.$x \in (-\infty,-18] \cup [-4,11)$\\
G.$x \in (-\infty,-18) \cup [-4,11]$\\
H.$x \in (-\infty,-18] \cup [-4,11]$
\testStop
\kluczStart
A
\kluczStop



\zadStart{Zadanie z Wikieł Z 1.62 b) moja wersja nr 730}

Rozwiązać nierówności $(x+18)(11-x)(x+5)\ge0$.
\zadStop
\rozwStart{Patryk Wirkus}{}
Miejsca zerowe naszego wielomianu to: $-18, 11, -5$.\\
Wielomian jest stopnia nieparzystego, ponadto znak współczynnika przy\linebreak najwyższej potędze x jest ujemny.\\ W związku z tym wykres wielomianu zaczyna się od lewej strony powyżej osi OX. A więc $$x \in (-\infty,-18) \cup (-5,11).$$
\rozwStop
\odpStart
$x \in (-\infty,-18) \cup (-5,11)$
\odpStop
\testStart
A.$x \in (-\infty,-18) \cup (-5,11)$\\
B.$x \in (-\infty,-18) \cup (-5,11]$\\
C.$x \in (-\infty,-18) \cup [-5,11)$\\
D.$x \in (-\infty,-18] \cup (-5,11)$\\
E.$x \in (-\infty,-18] \cup (-5,11]$\\
F.$x \in (-\infty,-18] \cup [-5,11)$\\
G.$x \in (-\infty,-18) \cup [-5,11]$\\
H.$x \in (-\infty,-18] \cup [-5,11]$
\testStop
\kluczStart
A
\kluczStop



\zadStart{Zadanie z Wikieł Z 1.62 b) moja wersja nr 731}

Rozwiązać nierówności $(x+18)(11-x)(x+6)\ge0$.
\zadStop
\rozwStart{Patryk Wirkus}{}
Miejsca zerowe naszego wielomianu to: $-18, 11, -6$.\\
Wielomian jest stopnia nieparzystego, ponadto znak współczynnika przy\linebreak najwyższej potędze x jest ujemny.\\ W związku z tym wykres wielomianu zaczyna się od lewej strony powyżej osi OX. A więc $$x \in (-\infty,-18) \cup (-6,11).$$
\rozwStop
\odpStart
$x \in (-\infty,-18) \cup (-6,11)$
\odpStop
\testStart
A.$x \in (-\infty,-18) \cup (-6,11)$\\
B.$x \in (-\infty,-18) \cup (-6,11]$\\
C.$x \in (-\infty,-18) \cup [-6,11)$\\
D.$x \in (-\infty,-18] \cup (-6,11)$\\
E.$x \in (-\infty,-18] \cup (-6,11]$\\
F.$x \in (-\infty,-18] \cup [-6,11)$\\
G.$x \in (-\infty,-18) \cup [-6,11]$\\
H.$x \in (-\infty,-18] \cup [-6,11]$
\testStop
\kluczStart
A
\kluczStop



\zadStart{Zadanie z Wikieł Z 1.62 b) moja wersja nr 732}

Rozwiązać nierówności $(x+18)(11-x)(x+7)\ge0$.
\zadStop
\rozwStart{Patryk Wirkus}{}
Miejsca zerowe naszego wielomianu to: $-18, 11, -7$.\\
Wielomian jest stopnia nieparzystego, ponadto znak współczynnika przy\linebreak najwyższej potędze x jest ujemny.\\ W związku z tym wykres wielomianu zaczyna się od lewej strony powyżej osi OX. A więc $$x \in (-\infty,-18) \cup (-7,11).$$
\rozwStop
\odpStart
$x \in (-\infty,-18) \cup (-7,11)$
\odpStop
\testStart
A.$x \in (-\infty,-18) \cup (-7,11)$\\
B.$x \in (-\infty,-18) \cup (-7,11]$\\
C.$x \in (-\infty,-18) \cup [-7,11)$\\
D.$x \in (-\infty,-18] \cup (-7,11)$\\
E.$x \in (-\infty,-18] \cup (-7,11]$\\
F.$x \in (-\infty,-18] \cup [-7,11)$\\
G.$x \in (-\infty,-18) \cup [-7,11]$\\
H.$x \in (-\infty,-18] \cup [-7,11]$
\testStop
\kluczStart
A
\kluczStop



\zadStart{Zadanie z Wikieł Z 1.62 b) moja wersja nr 733}

Rozwiązać nierówności $(x+18)(11-x)(x+8)\ge0$.
\zadStop
\rozwStart{Patryk Wirkus}{}
Miejsca zerowe naszego wielomianu to: $-18, 11, -8$.\\
Wielomian jest stopnia nieparzystego, ponadto znak współczynnika przy\linebreak najwyższej potędze x jest ujemny.\\ W związku z tym wykres wielomianu zaczyna się od lewej strony powyżej osi OX. A więc $$x \in (-\infty,-18) \cup (-8,11).$$
\rozwStop
\odpStart
$x \in (-\infty,-18) \cup (-8,11)$
\odpStop
\testStart
A.$x \in (-\infty,-18) \cup (-8,11)$\\
B.$x \in (-\infty,-18) \cup (-8,11]$\\
C.$x \in (-\infty,-18) \cup [-8,11)$\\
D.$x \in (-\infty,-18] \cup (-8,11)$\\
E.$x \in (-\infty,-18] \cup (-8,11]$\\
F.$x \in (-\infty,-18] \cup [-8,11)$\\
G.$x \in (-\infty,-18) \cup [-8,11]$\\
H.$x \in (-\infty,-18] \cup [-8,11]$
\testStop
\kluczStart
A
\kluczStop



\zadStart{Zadanie z Wikieł Z 1.62 b) moja wersja nr 734}

Rozwiązać nierówności $(x+18)(11-x)(x+9)\ge0$.
\zadStop
\rozwStart{Patryk Wirkus}{}
Miejsca zerowe naszego wielomianu to: $-18, 11, -9$.\\
Wielomian jest stopnia nieparzystego, ponadto znak współczynnika przy\linebreak najwyższej potędze x jest ujemny.\\ W związku z tym wykres wielomianu zaczyna się od lewej strony powyżej osi OX. A więc $$x \in (-\infty,-18) \cup (-9,11).$$
\rozwStop
\odpStart
$x \in (-\infty,-18) \cup (-9,11)$
\odpStop
\testStart
A.$x \in (-\infty,-18) \cup (-9,11)$\\
B.$x \in (-\infty,-18) \cup (-9,11]$\\
C.$x \in (-\infty,-18) \cup [-9,11)$\\
D.$x \in (-\infty,-18] \cup (-9,11)$\\
E.$x \in (-\infty,-18] \cup (-9,11]$\\
F.$x \in (-\infty,-18] \cup [-9,11)$\\
G.$x \in (-\infty,-18) \cup [-9,11]$\\
H.$x \in (-\infty,-18] \cup [-9,11]$
\testStop
\kluczStart
A
\kluczStop



\zadStart{Zadanie z Wikieł Z 1.62 b) moja wersja nr 735}

Rozwiązać nierówności $(x+18)(11-x)(x+10)\ge0$.
\zadStop
\rozwStart{Patryk Wirkus}{}
Miejsca zerowe naszego wielomianu to: $-18, 11, -10$.\\
Wielomian jest stopnia nieparzystego, ponadto znak współczynnika przy\linebreak najwyższej potędze x jest ujemny.\\ W związku z tym wykres wielomianu zaczyna się od lewej strony powyżej osi OX. A więc $$x \in (-\infty,-18) \cup (-10,11).$$
\rozwStop
\odpStart
$x \in (-\infty,-18) \cup (-10,11)$
\odpStop
\testStart
A.$x \in (-\infty,-18) \cup (-10,11)$\\
B.$x \in (-\infty,-18) \cup (-10,11]$\\
C.$x \in (-\infty,-18) \cup [-10,11)$\\
D.$x \in (-\infty,-18] \cup (-10,11)$\\
E.$x \in (-\infty,-18] \cup (-10,11]$\\
F.$x \in (-\infty,-18] \cup [-10,11)$\\
G.$x \in (-\infty,-18) \cup [-10,11]$\\
H.$x \in (-\infty,-18] \cup [-10,11]$
\testStop
\kluczStart
A
\kluczStop



\zadStart{Zadanie z Wikieł Z 1.62 b) moja wersja nr 736}

Rozwiązać nierówności $(x+18)(12-x)(x+1)\ge0$.
\zadStop
\rozwStart{Patryk Wirkus}{}
Miejsca zerowe naszego wielomianu to: $-18, 12, -1$.\\
Wielomian jest stopnia nieparzystego, ponadto znak współczynnika przy\linebreak najwyższej potędze x jest ujemny.\\ W związku z tym wykres wielomianu zaczyna się od lewej strony powyżej osi OX. A więc $$x \in (-\infty,-18) \cup (-1,12).$$
\rozwStop
\odpStart
$x \in (-\infty,-18) \cup (-1,12)$
\odpStop
\testStart
A.$x \in (-\infty,-18) \cup (-1,12)$\\
B.$x \in (-\infty,-18) \cup (-1,12]$\\
C.$x \in (-\infty,-18) \cup [-1,12)$\\
D.$x \in (-\infty,-18] \cup (-1,12)$\\
E.$x \in (-\infty,-18] \cup (-1,12]$\\
F.$x \in (-\infty,-18] \cup [-1,12)$\\
G.$x \in (-\infty,-18) \cup [-1,12]$\\
H.$x \in (-\infty,-18] \cup [-1,12]$
\testStop
\kluczStart
A
\kluczStop



\zadStart{Zadanie z Wikieł Z 1.62 b) moja wersja nr 737}

Rozwiązać nierówności $(x+18)(12-x)(x+2)\ge0$.
\zadStop
\rozwStart{Patryk Wirkus}{}
Miejsca zerowe naszego wielomianu to: $-18, 12, -2$.\\
Wielomian jest stopnia nieparzystego, ponadto znak współczynnika przy\linebreak najwyższej potędze x jest ujemny.\\ W związku z tym wykres wielomianu zaczyna się od lewej strony powyżej osi OX. A więc $$x \in (-\infty,-18) \cup (-2,12).$$
\rozwStop
\odpStart
$x \in (-\infty,-18) \cup (-2,12)$
\odpStop
\testStart
A.$x \in (-\infty,-18) \cup (-2,12)$\\
B.$x \in (-\infty,-18) \cup (-2,12]$\\
C.$x \in (-\infty,-18) \cup [-2,12)$\\
D.$x \in (-\infty,-18] \cup (-2,12)$\\
E.$x \in (-\infty,-18] \cup (-2,12]$\\
F.$x \in (-\infty,-18] \cup [-2,12)$\\
G.$x \in (-\infty,-18) \cup [-2,12]$\\
H.$x \in (-\infty,-18] \cup [-2,12]$
\testStop
\kluczStart
A
\kluczStop



\zadStart{Zadanie z Wikieł Z 1.62 b) moja wersja nr 738}

Rozwiązać nierówności $(x+18)(12-x)(x+3)\ge0$.
\zadStop
\rozwStart{Patryk Wirkus}{}
Miejsca zerowe naszego wielomianu to: $-18, 12, -3$.\\
Wielomian jest stopnia nieparzystego, ponadto znak współczynnika przy\linebreak najwyższej potędze x jest ujemny.\\ W związku z tym wykres wielomianu zaczyna się od lewej strony powyżej osi OX. A więc $$x \in (-\infty,-18) \cup (-3,12).$$
\rozwStop
\odpStart
$x \in (-\infty,-18) \cup (-3,12)$
\odpStop
\testStart
A.$x \in (-\infty,-18) \cup (-3,12)$\\
B.$x \in (-\infty,-18) \cup (-3,12]$\\
C.$x \in (-\infty,-18) \cup [-3,12)$\\
D.$x \in (-\infty,-18] \cup (-3,12)$\\
E.$x \in (-\infty,-18] \cup (-3,12]$\\
F.$x \in (-\infty,-18] \cup [-3,12)$\\
G.$x \in (-\infty,-18) \cup [-3,12]$\\
H.$x \in (-\infty,-18] \cup [-3,12]$
\testStop
\kluczStart
A
\kluczStop



\zadStart{Zadanie z Wikieł Z 1.62 b) moja wersja nr 739}

Rozwiązać nierówności $(x+18)(12-x)(x+4)\ge0$.
\zadStop
\rozwStart{Patryk Wirkus}{}
Miejsca zerowe naszego wielomianu to: $-18, 12, -4$.\\
Wielomian jest stopnia nieparzystego, ponadto znak współczynnika przy\linebreak najwyższej potędze x jest ujemny.\\ W związku z tym wykres wielomianu zaczyna się od lewej strony powyżej osi OX. A więc $$x \in (-\infty,-18) \cup (-4,12).$$
\rozwStop
\odpStart
$x \in (-\infty,-18) \cup (-4,12)$
\odpStop
\testStart
A.$x \in (-\infty,-18) \cup (-4,12)$\\
B.$x \in (-\infty,-18) \cup (-4,12]$\\
C.$x \in (-\infty,-18) \cup [-4,12)$\\
D.$x \in (-\infty,-18] \cup (-4,12)$\\
E.$x \in (-\infty,-18] \cup (-4,12]$\\
F.$x \in (-\infty,-18] \cup [-4,12)$\\
G.$x \in (-\infty,-18) \cup [-4,12]$\\
H.$x \in (-\infty,-18] \cup [-4,12]$
\testStop
\kluczStart
A
\kluczStop



\zadStart{Zadanie z Wikieł Z 1.62 b) moja wersja nr 740}

Rozwiązać nierówności $(x+18)(12-x)(x+5)\ge0$.
\zadStop
\rozwStart{Patryk Wirkus}{}
Miejsca zerowe naszego wielomianu to: $-18, 12, -5$.\\
Wielomian jest stopnia nieparzystego, ponadto znak współczynnika przy\linebreak najwyższej potędze x jest ujemny.\\ W związku z tym wykres wielomianu zaczyna się od lewej strony powyżej osi OX. A więc $$x \in (-\infty,-18) \cup (-5,12).$$
\rozwStop
\odpStart
$x \in (-\infty,-18) \cup (-5,12)$
\odpStop
\testStart
A.$x \in (-\infty,-18) \cup (-5,12)$\\
B.$x \in (-\infty,-18) \cup (-5,12]$\\
C.$x \in (-\infty,-18) \cup [-5,12)$\\
D.$x \in (-\infty,-18] \cup (-5,12)$\\
E.$x \in (-\infty,-18] \cup (-5,12]$\\
F.$x \in (-\infty,-18] \cup [-5,12)$\\
G.$x \in (-\infty,-18) \cup [-5,12]$\\
H.$x \in (-\infty,-18] \cup [-5,12]$
\testStop
\kluczStart
A
\kluczStop



\zadStart{Zadanie z Wikieł Z 1.62 b) moja wersja nr 741}

Rozwiązać nierówności $(x+18)(12-x)(x+6)\ge0$.
\zadStop
\rozwStart{Patryk Wirkus}{}
Miejsca zerowe naszego wielomianu to: $-18, 12, -6$.\\
Wielomian jest stopnia nieparzystego, ponadto znak współczynnika przy\linebreak najwyższej potędze x jest ujemny.\\ W związku z tym wykres wielomianu zaczyna się od lewej strony powyżej osi OX. A więc $$x \in (-\infty,-18) \cup (-6,12).$$
\rozwStop
\odpStart
$x \in (-\infty,-18) \cup (-6,12)$
\odpStop
\testStart
A.$x \in (-\infty,-18) \cup (-6,12)$\\
B.$x \in (-\infty,-18) \cup (-6,12]$\\
C.$x \in (-\infty,-18) \cup [-6,12)$\\
D.$x \in (-\infty,-18] \cup (-6,12)$\\
E.$x \in (-\infty,-18] \cup (-6,12]$\\
F.$x \in (-\infty,-18] \cup [-6,12)$\\
G.$x \in (-\infty,-18) \cup [-6,12]$\\
H.$x \in (-\infty,-18] \cup [-6,12]$
\testStop
\kluczStart
A
\kluczStop



\zadStart{Zadanie z Wikieł Z 1.62 b) moja wersja nr 742}

Rozwiązać nierówności $(x+18)(12-x)(x+7)\ge0$.
\zadStop
\rozwStart{Patryk Wirkus}{}
Miejsca zerowe naszego wielomianu to: $-18, 12, -7$.\\
Wielomian jest stopnia nieparzystego, ponadto znak współczynnika przy\linebreak najwyższej potędze x jest ujemny.\\ W związku z tym wykres wielomianu zaczyna się od lewej strony powyżej osi OX. A więc $$x \in (-\infty,-18) \cup (-7,12).$$
\rozwStop
\odpStart
$x \in (-\infty,-18) \cup (-7,12)$
\odpStop
\testStart
A.$x \in (-\infty,-18) \cup (-7,12)$\\
B.$x \in (-\infty,-18) \cup (-7,12]$\\
C.$x \in (-\infty,-18) \cup [-7,12)$\\
D.$x \in (-\infty,-18] \cup (-7,12)$\\
E.$x \in (-\infty,-18] \cup (-7,12]$\\
F.$x \in (-\infty,-18] \cup [-7,12)$\\
G.$x \in (-\infty,-18) \cup [-7,12]$\\
H.$x \in (-\infty,-18] \cup [-7,12]$
\testStop
\kluczStart
A
\kluczStop



\zadStart{Zadanie z Wikieł Z 1.62 b) moja wersja nr 743}

Rozwiązać nierówności $(x+18)(12-x)(x+8)\ge0$.
\zadStop
\rozwStart{Patryk Wirkus}{}
Miejsca zerowe naszego wielomianu to: $-18, 12, -8$.\\
Wielomian jest stopnia nieparzystego, ponadto znak współczynnika przy\linebreak najwyższej potędze x jest ujemny.\\ W związku z tym wykres wielomianu zaczyna się od lewej strony powyżej osi OX. A więc $$x \in (-\infty,-18) \cup (-8,12).$$
\rozwStop
\odpStart
$x \in (-\infty,-18) \cup (-8,12)$
\odpStop
\testStart
A.$x \in (-\infty,-18) \cup (-8,12)$\\
B.$x \in (-\infty,-18) \cup (-8,12]$\\
C.$x \in (-\infty,-18) \cup [-8,12)$\\
D.$x \in (-\infty,-18] \cup (-8,12)$\\
E.$x \in (-\infty,-18] \cup (-8,12]$\\
F.$x \in (-\infty,-18] \cup [-8,12)$\\
G.$x \in (-\infty,-18) \cup [-8,12]$\\
H.$x \in (-\infty,-18] \cup [-8,12]$
\testStop
\kluczStart
A
\kluczStop



\zadStart{Zadanie z Wikieł Z 1.62 b) moja wersja nr 744}

Rozwiązać nierówności $(x+18)(12-x)(x+9)\ge0$.
\zadStop
\rozwStart{Patryk Wirkus}{}
Miejsca zerowe naszego wielomianu to: $-18, 12, -9$.\\
Wielomian jest stopnia nieparzystego, ponadto znak współczynnika przy\linebreak najwyższej potędze x jest ujemny.\\ W związku z tym wykres wielomianu zaczyna się od lewej strony powyżej osi OX. A więc $$x \in (-\infty,-18) \cup (-9,12).$$
\rozwStop
\odpStart
$x \in (-\infty,-18) \cup (-9,12)$
\odpStop
\testStart
A.$x \in (-\infty,-18) \cup (-9,12)$\\
B.$x \in (-\infty,-18) \cup (-9,12]$\\
C.$x \in (-\infty,-18) \cup [-9,12)$\\
D.$x \in (-\infty,-18] \cup (-9,12)$\\
E.$x \in (-\infty,-18] \cup (-9,12]$\\
F.$x \in (-\infty,-18] \cup [-9,12)$\\
G.$x \in (-\infty,-18) \cup [-9,12]$\\
H.$x \in (-\infty,-18] \cup [-9,12]$
\testStop
\kluczStart
A
\kluczStop



\zadStart{Zadanie z Wikieł Z 1.62 b) moja wersja nr 745}

Rozwiązać nierówności $(x+18)(12-x)(x+10)\ge0$.
\zadStop
\rozwStart{Patryk Wirkus}{}
Miejsca zerowe naszego wielomianu to: $-18, 12, -10$.\\
Wielomian jest stopnia nieparzystego, ponadto znak współczynnika przy\linebreak najwyższej potędze x jest ujemny.\\ W związku z tym wykres wielomianu zaczyna się od lewej strony powyżej osi OX. A więc $$x \in (-\infty,-18) \cup (-10,12).$$
\rozwStop
\odpStart
$x \in (-\infty,-18) \cup (-10,12)$
\odpStop
\testStart
A.$x \in (-\infty,-18) \cup (-10,12)$\\
B.$x \in (-\infty,-18) \cup (-10,12]$\\
C.$x \in (-\infty,-18) \cup [-10,12)$\\
D.$x \in (-\infty,-18] \cup (-10,12)$\\
E.$x \in (-\infty,-18] \cup (-10,12]$\\
F.$x \in (-\infty,-18] \cup [-10,12)$\\
G.$x \in (-\infty,-18) \cup [-10,12]$\\
H.$x \in (-\infty,-18] \cup [-10,12]$
\testStop
\kluczStart
A
\kluczStop



\zadStart{Zadanie z Wikieł Z 1.62 b) moja wersja nr 746}

Rozwiązać nierówności $(x+18)(12-x)(x+11)\ge0$.
\zadStop
\rozwStart{Patryk Wirkus}{}
Miejsca zerowe naszego wielomianu to: $-18, 12, -11$.\\
Wielomian jest stopnia nieparzystego, ponadto znak współczynnika przy\linebreak najwyższej potędze x jest ujemny.\\ W związku z tym wykres wielomianu zaczyna się od lewej strony powyżej osi OX. A więc $$x \in (-\infty,-18) \cup (-11,12).$$
\rozwStop
\odpStart
$x \in (-\infty,-18) \cup (-11,12)$
\odpStop
\testStart
A.$x \in (-\infty,-18) \cup (-11,12)$\\
B.$x \in (-\infty,-18) \cup (-11,12]$\\
C.$x \in (-\infty,-18) \cup [-11,12)$\\
D.$x \in (-\infty,-18] \cup (-11,12)$\\
E.$x \in (-\infty,-18] \cup (-11,12]$\\
F.$x \in (-\infty,-18] \cup [-11,12)$\\
G.$x \in (-\infty,-18) \cup [-11,12]$\\
H.$x \in (-\infty,-18] \cup [-11,12]$
\testStop
\kluczStart
A
\kluczStop



\zadStart{Zadanie z Wikieł Z 1.62 b) moja wersja nr 747}

Rozwiązać nierówności $(x+18)(13-x)(x+1)\ge0$.
\zadStop
\rozwStart{Patryk Wirkus}{}
Miejsca zerowe naszego wielomianu to: $-18, 13, -1$.\\
Wielomian jest stopnia nieparzystego, ponadto znak współczynnika przy\linebreak najwyższej potędze x jest ujemny.\\ W związku z tym wykres wielomianu zaczyna się od lewej strony powyżej osi OX. A więc $$x \in (-\infty,-18) \cup (-1,13).$$
\rozwStop
\odpStart
$x \in (-\infty,-18) \cup (-1,13)$
\odpStop
\testStart
A.$x \in (-\infty,-18) \cup (-1,13)$\\
B.$x \in (-\infty,-18) \cup (-1,13]$\\
C.$x \in (-\infty,-18) \cup [-1,13)$\\
D.$x \in (-\infty,-18] \cup (-1,13)$\\
E.$x \in (-\infty,-18] \cup (-1,13]$\\
F.$x \in (-\infty,-18] \cup [-1,13)$\\
G.$x \in (-\infty,-18) \cup [-1,13]$\\
H.$x \in (-\infty,-18] \cup [-1,13]$
\testStop
\kluczStart
A
\kluczStop



\zadStart{Zadanie z Wikieł Z 1.62 b) moja wersja nr 748}

Rozwiązać nierówności $(x+18)(13-x)(x+2)\ge0$.
\zadStop
\rozwStart{Patryk Wirkus}{}
Miejsca zerowe naszego wielomianu to: $-18, 13, -2$.\\
Wielomian jest stopnia nieparzystego, ponadto znak współczynnika przy\linebreak najwyższej potędze x jest ujemny.\\ W związku z tym wykres wielomianu zaczyna się od lewej strony powyżej osi OX. A więc $$x \in (-\infty,-18) \cup (-2,13).$$
\rozwStop
\odpStart
$x \in (-\infty,-18) \cup (-2,13)$
\odpStop
\testStart
A.$x \in (-\infty,-18) \cup (-2,13)$\\
B.$x \in (-\infty,-18) \cup (-2,13]$\\
C.$x \in (-\infty,-18) \cup [-2,13)$\\
D.$x \in (-\infty,-18] \cup (-2,13)$\\
E.$x \in (-\infty,-18] \cup (-2,13]$\\
F.$x \in (-\infty,-18] \cup [-2,13)$\\
G.$x \in (-\infty,-18) \cup [-2,13]$\\
H.$x \in (-\infty,-18] \cup [-2,13]$
\testStop
\kluczStart
A
\kluczStop



\zadStart{Zadanie z Wikieł Z 1.62 b) moja wersja nr 749}

Rozwiązać nierówności $(x+18)(13-x)(x+3)\ge0$.
\zadStop
\rozwStart{Patryk Wirkus}{}
Miejsca zerowe naszego wielomianu to: $-18, 13, -3$.\\
Wielomian jest stopnia nieparzystego, ponadto znak współczynnika przy\linebreak najwyższej potędze x jest ujemny.\\ W związku z tym wykres wielomianu zaczyna się od lewej strony powyżej osi OX. A więc $$x \in (-\infty,-18) \cup (-3,13).$$
\rozwStop
\odpStart
$x \in (-\infty,-18) \cup (-3,13)$
\odpStop
\testStart
A.$x \in (-\infty,-18) \cup (-3,13)$\\
B.$x \in (-\infty,-18) \cup (-3,13]$\\
C.$x \in (-\infty,-18) \cup [-3,13)$\\
D.$x \in (-\infty,-18] \cup (-3,13)$\\
E.$x \in (-\infty,-18] \cup (-3,13]$\\
F.$x \in (-\infty,-18] \cup [-3,13)$\\
G.$x \in (-\infty,-18) \cup [-3,13]$\\
H.$x \in (-\infty,-18] \cup [-3,13]$
\testStop
\kluczStart
A
\kluczStop



\zadStart{Zadanie z Wikieł Z 1.62 b) moja wersja nr 750}

Rozwiązać nierówności $(x+18)(13-x)(x+4)\ge0$.
\zadStop
\rozwStart{Patryk Wirkus}{}
Miejsca zerowe naszego wielomianu to: $-18, 13, -4$.\\
Wielomian jest stopnia nieparzystego, ponadto znak współczynnika przy\linebreak najwyższej potędze x jest ujemny.\\ W związku z tym wykres wielomianu zaczyna się od lewej strony powyżej osi OX. A więc $$x \in (-\infty,-18) \cup (-4,13).$$
\rozwStop
\odpStart
$x \in (-\infty,-18) \cup (-4,13)$
\odpStop
\testStart
A.$x \in (-\infty,-18) \cup (-4,13)$\\
B.$x \in (-\infty,-18) \cup (-4,13]$\\
C.$x \in (-\infty,-18) \cup [-4,13)$\\
D.$x \in (-\infty,-18] \cup (-4,13)$\\
E.$x \in (-\infty,-18] \cup (-4,13]$\\
F.$x \in (-\infty,-18] \cup [-4,13)$\\
G.$x \in (-\infty,-18) \cup [-4,13]$\\
H.$x \in (-\infty,-18] \cup [-4,13]$
\testStop
\kluczStart
A
\kluczStop



\zadStart{Zadanie z Wikieł Z 1.62 b) moja wersja nr 751}

Rozwiązać nierówności $(x+18)(13-x)(x+5)\ge0$.
\zadStop
\rozwStart{Patryk Wirkus}{}
Miejsca zerowe naszego wielomianu to: $-18, 13, -5$.\\
Wielomian jest stopnia nieparzystego, ponadto znak współczynnika przy\linebreak najwyższej potędze x jest ujemny.\\ W związku z tym wykres wielomianu zaczyna się od lewej strony powyżej osi OX. A więc $$x \in (-\infty,-18) \cup (-5,13).$$
\rozwStop
\odpStart
$x \in (-\infty,-18) \cup (-5,13)$
\odpStop
\testStart
A.$x \in (-\infty,-18) \cup (-5,13)$\\
B.$x \in (-\infty,-18) \cup (-5,13]$\\
C.$x \in (-\infty,-18) \cup [-5,13)$\\
D.$x \in (-\infty,-18] \cup (-5,13)$\\
E.$x \in (-\infty,-18] \cup (-5,13]$\\
F.$x \in (-\infty,-18] \cup [-5,13)$\\
G.$x \in (-\infty,-18) \cup [-5,13]$\\
H.$x \in (-\infty,-18] \cup [-5,13]$
\testStop
\kluczStart
A
\kluczStop



\zadStart{Zadanie z Wikieł Z 1.62 b) moja wersja nr 752}

Rozwiązać nierówności $(x+18)(13-x)(x+6)\ge0$.
\zadStop
\rozwStart{Patryk Wirkus}{}
Miejsca zerowe naszego wielomianu to: $-18, 13, -6$.\\
Wielomian jest stopnia nieparzystego, ponadto znak współczynnika przy\linebreak najwyższej potędze x jest ujemny.\\ W związku z tym wykres wielomianu zaczyna się od lewej strony powyżej osi OX. A więc $$x \in (-\infty,-18) \cup (-6,13).$$
\rozwStop
\odpStart
$x \in (-\infty,-18) \cup (-6,13)$
\odpStop
\testStart
A.$x \in (-\infty,-18) \cup (-6,13)$\\
B.$x \in (-\infty,-18) \cup (-6,13]$\\
C.$x \in (-\infty,-18) \cup [-6,13)$\\
D.$x \in (-\infty,-18] \cup (-6,13)$\\
E.$x \in (-\infty,-18] \cup (-6,13]$\\
F.$x \in (-\infty,-18] \cup [-6,13)$\\
G.$x \in (-\infty,-18) \cup [-6,13]$\\
H.$x \in (-\infty,-18] \cup [-6,13]$
\testStop
\kluczStart
A
\kluczStop



\zadStart{Zadanie z Wikieł Z 1.62 b) moja wersja nr 753}

Rozwiązać nierówności $(x+18)(13-x)(x+7)\ge0$.
\zadStop
\rozwStart{Patryk Wirkus}{}
Miejsca zerowe naszego wielomianu to: $-18, 13, -7$.\\
Wielomian jest stopnia nieparzystego, ponadto znak współczynnika przy\linebreak najwyższej potędze x jest ujemny.\\ W związku z tym wykres wielomianu zaczyna się od lewej strony powyżej osi OX. A więc $$x \in (-\infty,-18) \cup (-7,13).$$
\rozwStop
\odpStart
$x \in (-\infty,-18) \cup (-7,13)$
\odpStop
\testStart
A.$x \in (-\infty,-18) \cup (-7,13)$\\
B.$x \in (-\infty,-18) \cup (-7,13]$\\
C.$x \in (-\infty,-18) \cup [-7,13)$\\
D.$x \in (-\infty,-18] \cup (-7,13)$\\
E.$x \in (-\infty,-18] \cup (-7,13]$\\
F.$x \in (-\infty,-18] \cup [-7,13)$\\
G.$x \in (-\infty,-18) \cup [-7,13]$\\
H.$x \in (-\infty,-18] \cup [-7,13]$
\testStop
\kluczStart
A
\kluczStop



\zadStart{Zadanie z Wikieł Z 1.62 b) moja wersja nr 754}

Rozwiązać nierówności $(x+18)(13-x)(x+8)\ge0$.
\zadStop
\rozwStart{Patryk Wirkus}{}
Miejsca zerowe naszego wielomianu to: $-18, 13, -8$.\\
Wielomian jest stopnia nieparzystego, ponadto znak współczynnika przy\linebreak najwyższej potędze x jest ujemny.\\ W związku z tym wykres wielomianu zaczyna się od lewej strony powyżej osi OX. A więc $$x \in (-\infty,-18) \cup (-8,13).$$
\rozwStop
\odpStart
$x \in (-\infty,-18) \cup (-8,13)$
\odpStop
\testStart
A.$x \in (-\infty,-18) \cup (-8,13)$\\
B.$x \in (-\infty,-18) \cup (-8,13]$\\
C.$x \in (-\infty,-18) \cup [-8,13)$\\
D.$x \in (-\infty,-18] \cup (-8,13)$\\
E.$x \in (-\infty,-18] \cup (-8,13]$\\
F.$x \in (-\infty,-18] \cup [-8,13)$\\
G.$x \in (-\infty,-18) \cup [-8,13]$\\
H.$x \in (-\infty,-18] \cup [-8,13]$
\testStop
\kluczStart
A
\kluczStop



\zadStart{Zadanie z Wikieł Z 1.62 b) moja wersja nr 755}

Rozwiązać nierówności $(x+18)(13-x)(x+9)\ge0$.
\zadStop
\rozwStart{Patryk Wirkus}{}
Miejsca zerowe naszego wielomianu to: $-18, 13, -9$.\\
Wielomian jest stopnia nieparzystego, ponadto znak współczynnika przy\linebreak najwyższej potędze x jest ujemny.\\ W związku z tym wykres wielomianu zaczyna się od lewej strony powyżej osi OX. A więc $$x \in (-\infty,-18) \cup (-9,13).$$
\rozwStop
\odpStart
$x \in (-\infty,-18) \cup (-9,13)$
\odpStop
\testStart
A.$x \in (-\infty,-18) \cup (-9,13)$\\
B.$x \in (-\infty,-18) \cup (-9,13]$\\
C.$x \in (-\infty,-18) \cup [-9,13)$\\
D.$x \in (-\infty,-18] \cup (-9,13)$\\
E.$x \in (-\infty,-18] \cup (-9,13]$\\
F.$x \in (-\infty,-18] \cup [-9,13)$\\
G.$x \in (-\infty,-18) \cup [-9,13]$\\
H.$x \in (-\infty,-18] \cup [-9,13]$
\testStop
\kluczStart
A
\kluczStop



\zadStart{Zadanie z Wikieł Z 1.62 b) moja wersja nr 756}

Rozwiązać nierówności $(x+18)(13-x)(x+10)\ge0$.
\zadStop
\rozwStart{Patryk Wirkus}{}
Miejsca zerowe naszego wielomianu to: $-18, 13, -10$.\\
Wielomian jest stopnia nieparzystego, ponadto znak współczynnika przy\linebreak najwyższej potędze x jest ujemny.\\ W związku z tym wykres wielomianu zaczyna się od lewej strony powyżej osi OX. A więc $$x \in (-\infty,-18) \cup (-10,13).$$
\rozwStop
\odpStart
$x \in (-\infty,-18) \cup (-10,13)$
\odpStop
\testStart
A.$x \in (-\infty,-18) \cup (-10,13)$\\
B.$x \in (-\infty,-18) \cup (-10,13]$\\
C.$x \in (-\infty,-18) \cup [-10,13)$\\
D.$x \in (-\infty,-18] \cup (-10,13)$\\
E.$x \in (-\infty,-18] \cup (-10,13]$\\
F.$x \in (-\infty,-18] \cup [-10,13)$\\
G.$x \in (-\infty,-18) \cup [-10,13]$\\
H.$x \in (-\infty,-18] \cup [-10,13]$
\testStop
\kluczStart
A
\kluczStop



\zadStart{Zadanie z Wikieł Z 1.62 b) moja wersja nr 757}

Rozwiązać nierówności $(x+18)(13-x)(x+11)\ge0$.
\zadStop
\rozwStart{Patryk Wirkus}{}
Miejsca zerowe naszego wielomianu to: $-18, 13, -11$.\\
Wielomian jest stopnia nieparzystego, ponadto znak współczynnika przy\linebreak najwyższej potędze x jest ujemny.\\ W związku z tym wykres wielomianu zaczyna się od lewej strony powyżej osi OX. A więc $$x \in (-\infty,-18) \cup (-11,13).$$
\rozwStop
\odpStart
$x \in (-\infty,-18) \cup (-11,13)$
\odpStop
\testStart
A.$x \in (-\infty,-18) \cup (-11,13)$\\
B.$x \in (-\infty,-18) \cup (-11,13]$\\
C.$x \in (-\infty,-18) \cup [-11,13)$\\
D.$x \in (-\infty,-18] \cup (-11,13)$\\
E.$x \in (-\infty,-18] \cup (-11,13]$\\
F.$x \in (-\infty,-18] \cup [-11,13)$\\
G.$x \in (-\infty,-18) \cup [-11,13]$\\
H.$x \in (-\infty,-18] \cup [-11,13]$
\testStop
\kluczStart
A
\kluczStop



\zadStart{Zadanie z Wikieł Z 1.62 b) moja wersja nr 758}

Rozwiązać nierówności $(x+18)(13-x)(x+12)\ge0$.
\zadStop
\rozwStart{Patryk Wirkus}{}
Miejsca zerowe naszego wielomianu to: $-18, 13, -12$.\\
Wielomian jest stopnia nieparzystego, ponadto znak współczynnika przy\linebreak najwyższej potędze x jest ujemny.\\ W związku z tym wykres wielomianu zaczyna się od lewej strony powyżej osi OX. A więc $$x \in (-\infty,-18) \cup (-12,13).$$
\rozwStop
\odpStart
$x \in (-\infty,-18) \cup (-12,13)$
\odpStop
\testStart
A.$x \in (-\infty,-18) \cup (-12,13)$\\
B.$x \in (-\infty,-18) \cup (-12,13]$\\
C.$x \in (-\infty,-18) \cup [-12,13)$\\
D.$x \in (-\infty,-18] \cup (-12,13)$\\
E.$x \in (-\infty,-18] \cup (-12,13]$\\
F.$x \in (-\infty,-18] \cup [-12,13)$\\
G.$x \in (-\infty,-18) \cup [-12,13]$\\
H.$x \in (-\infty,-18] \cup [-12,13]$
\testStop
\kluczStart
A
\kluczStop



\zadStart{Zadanie z Wikieł Z 1.62 b) moja wersja nr 759}

Rozwiązać nierówności $(x+18)(14-x)(x+1)\ge0$.
\zadStop
\rozwStart{Patryk Wirkus}{}
Miejsca zerowe naszego wielomianu to: $-18, 14, -1$.\\
Wielomian jest stopnia nieparzystego, ponadto znak współczynnika przy\linebreak najwyższej potędze x jest ujemny.\\ W związku z tym wykres wielomianu zaczyna się od lewej strony powyżej osi OX. A więc $$x \in (-\infty,-18) \cup (-1,14).$$
\rozwStop
\odpStart
$x \in (-\infty,-18) \cup (-1,14)$
\odpStop
\testStart
A.$x \in (-\infty,-18) \cup (-1,14)$\\
B.$x \in (-\infty,-18) \cup (-1,14]$\\
C.$x \in (-\infty,-18) \cup [-1,14)$\\
D.$x \in (-\infty,-18] \cup (-1,14)$\\
E.$x \in (-\infty,-18] \cup (-1,14]$\\
F.$x \in (-\infty,-18] \cup [-1,14)$\\
G.$x \in (-\infty,-18) \cup [-1,14]$\\
H.$x \in (-\infty,-18] \cup [-1,14]$
\testStop
\kluczStart
A
\kluczStop



\zadStart{Zadanie z Wikieł Z 1.62 b) moja wersja nr 760}

Rozwiązać nierówności $(x+18)(14-x)(x+2)\ge0$.
\zadStop
\rozwStart{Patryk Wirkus}{}
Miejsca zerowe naszego wielomianu to: $-18, 14, -2$.\\
Wielomian jest stopnia nieparzystego, ponadto znak współczynnika przy\linebreak najwyższej potędze x jest ujemny.\\ W związku z tym wykres wielomianu zaczyna się od lewej strony powyżej osi OX. A więc $$x \in (-\infty,-18) \cup (-2,14).$$
\rozwStop
\odpStart
$x \in (-\infty,-18) \cup (-2,14)$
\odpStop
\testStart
A.$x \in (-\infty,-18) \cup (-2,14)$\\
B.$x \in (-\infty,-18) \cup (-2,14]$\\
C.$x \in (-\infty,-18) \cup [-2,14)$\\
D.$x \in (-\infty,-18] \cup (-2,14)$\\
E.$x \in (-\infty,-18] \cup (-2,14]$\\
F.$x \in (-\infty,-18] \cup [-2,14)$\\
G.$x \in (-\infty,-18) \cup [-2,14]$\\
H.$x \in (-\infty,-18] \cup [-2,14]$
\testStop
\kluczStart
A
\kluczStop



\zadStart{Zadanie z Wikieł Z 1.62 b) moja wersja nr 761}

Rozwiązać nierówności $(x+18)(14-x)(x+3)\ge0$.
\zadStop
\rozwStart{Patryk Wirkus}{}
Miejsca zerowe naszego wielomianu to: $-18, 14, -3$.\\
Wielomian jest stopnia nieparzystego, ponadto znak współczynnika przy\linebreak najwyższej potędze x jest ujemny.\\ W związku z tym wykres wielomianu zaczyna się od lewej strony powyżej osi OX. A więc $$x \in (-\infty,-18) \cup (-3,14).$$
\rozwStop
\odpStart
$x \in (-\infty,-18) \cup (-3,14)$
\odpStop
\testStart
A.$x \in (-\infty,-18) \cup (-3,14)$\\
B.$x \in (-\infty,-18) \cup (-3,14]$\\
C.$x \in (-\infty,-18) \cup [-3,14)$\\
D.$x \in (-\infty,-18] \cup (-3,14)$\\
E.$x \in (-\infty,-18] \cup (-3,14]$\\
F.$x \in (-\infty,-18] \cup [-3,14)$\\
G.$x \in (-\infty,-18) \cup [-3,14]$\\
H.$x \in (-\infty,-18] \cup [-3,14]$
\testStop
\kluczStart
A
\kluczStop



\zadStart{Zadanie z Wikieł Z 1.62 b) moja wersja nr 762}

Rozwiązać nierówności $(x+18)(14-x)(x+4)\ge0$.
\zadStop
\rozwStart{Patryk Wirkus}{}
Miejsca zerowe naszego wielomianu to: $-18, 14, -4$.\\
Wielomian jest stopnia nieparzystego, ponadto znak współczynnika przy\linebreak najwyższej potędze x jest ujemny.\\ W związku z tym wykres wielomianu zaczyna się od lewej strony powyżej osi OX. A więc $$x \in (-\infty,-18) \cup (-4,14).$$
\rozwStop
\odpStart
$x \in (-\infty,-18) \cup (-4,14)$
\odpStop
\testStart
A.$x \in (-\infty,-18) \cup (-4,14)$\\
B.$x \in (-\infty,-18) \cup (-4,14]$\\
C.$x \in (-\infty,-18) \cup [-4,14)$\\
D.$x \in (-\infty,-18] \cup (-4,14)$\\
E.$x \in (-\infty,-18] \cup (-4,14]$\\
F.$x \in (-\infty,-18] \cup [-4,14)$\\
G.$x \in (-\infty,-18) \cup [-4,14]$\\
H.$x \in (-\infty,-18] \cup [-4,14]$
\testStop
\kluczStart
A
\kluczStop



\zadStart{Zadanie z Wikieł Z 1.62 b) moja wersja nr 763}

Rozwiązać nierówności $(x+18)(14-x)(x+5)\ge0$.
\zadStop
\rozwStart{Patryk Wirkus}{}
Miejsca zerowe naszego wielomianu to: $-18, 14, -5$.\\
Wielomian jest stopnia nieparzystego, ponadto znak współczynnika przy\linebreak najwyższej potędze x jest ujemny.\\ W związku z tym wykres wielomianu zaczyna się od lewej strony powyżej osi OX. A więc $$x \in (-\infty,-18) \cup (-5,14).$$
\rozwStop
\odpStart
$x \in (-\infty,-18) \cup (-5,14)$
\odpStop
\testStart
A.$x \in (-\infty,-18) \cup (-5,14)$\\
B.$x \in (-\infty,-18) \cup (-5,14]$\\
C.$x \in (-\infty,-18) \cup [-5,14)$\\
D.$x \in (-\infty,-18] \cup (-5,14)$\\
E.$x \in (-\infty,-18] \cup (-5,14]$\\
F.$x \in (-\infty,-18] \cup [-5,14)$\\
G.$x \in (-\infty,-18) \cup [-5,14]$\\
H.$x \in (-\infty,-18] \cup [-5,14]$
\testStop
\kluczStart
A
\kluczStop



\zadStart{Zadanie z Wikieł Z 1.62 b) moja wersja nr 764}

Rozwiązać nierówności $(x+18)(14-x)(x+6)\ge0$.
\zadStop
\rozwStart{Patryk Wirkus}{}
Miejsca zerowe naszego wielomianu to: $-18, 14, -6$.\\
Wielomian jest stopnia nieparzystego, ponadto znak współczynnika przy\linebreak najwyższej potędze x jest ujemny.\\ W związku z tym wykres wielomianu zaczyna się od lewej strony powyżej osi OX. A więc $$x \in (-\infty,-18) \cup (-6,14).$$
\rozwStop
\odpStart
$x \in (-\infty,-18) \cup (-6,14)$
\odpStop
\testStart
A.$x \in (-\infty,-18) \cup (-6,14)$\\
B.$x \in (-\infty,-18) \cup (-6,14]$\\
C.$x \in (-\infty,-18) \cup [-6,14)$\\
D.$x \in (-\infty,-18] \cup (-6,14)$\\
E.$x \in (-\infty,-18] \cup (-6,14]$\\
F.$x \in (-\infty,-18] \cup [-6,14)$\\
G.$x \in (-\infty,-18) \cup [-6,14]$\\
H.$x \in (-\infty,-18] \cup [-6,14]$
\testStop
\kluczStart
A
\kluczStop



\zadStart{Zadanie z Wikieł Z 1.62 b) moja wersja nr 765}

Rozwiązać nierówności $(x+18)(14-x)(x+7)\ge0$.
\zadStop
\rozwStart{Patryk Wirkus}{}
Miejsca zerowe naszego wielomianu to: $-18, 14, -7$.\\
Wielomian jest stopnia nieparzystego, ponadto znak współczynnika przy\linebreak najwyższej potędze x jest ujemny.\\ W związku z tym wykres wielomianu zaczyna się od lewej strony powyżej osi OX. A więc $$x \in (-\infty,-18) \cup (-7,14).$$
\rozwStop
\odpStart
$x \in (-\infty,-18) \cup (-7,14)$
\odpStop
\testStart
A.$x \in (-\infty,-18) \cup (-7,14)$\\
B.$x \in (-\infty,-18) \cup (-7,14]$\\
C.$x \in (-\infty,-18) \cup [-7,14)$\\
D.$x \in (-\infty,-18] \cup (-7,14)$\\
E.$x \in (-\infty,-18] \cup (-7,14]$\\
F.$x \in (-\infty,-18] \cup [-7,14)$\\
G.$x \in (-\infty,-18) \cup [-7,14]$\\
H.$x \in (-\infty,-18] \cup [-7,14]$
\testStop
\kluczStart
A
\kluczStop



\zadStart{Zadanie z Wikieł Z 1.62 b) moja wersja nr 766}

Rozwiązać nierówności $(x+18)(14-x)(x+8)\ge0$.
\zadStop
\rozwStart{Patryk Wirkus}{}
Miejsca zerowe naszego wielomianu to: $-18, 14, -8$.\\
Wielomian jest stopnia nieparzystego, ponadto znak współczynnika przy\linebreak najwyższej potędze x jest ujemny.\\ W związku z tym wykres wielomianu zaczyna się od lewej strony powyżej osi OX. A więc $$x \in (-\infty,-18) \cup (-8,14).$$
\rozwStop
\odpStart
$x \in (-\infty,-18) \cup (-8,14)$
\odpStop
\testStart
A.$x \in (-\infty,-18) \cup (-8,14)$\\
B.$x \in (-\infty,-18) \cup (-8,14]$\\
C.$x \in (-\infty,-18) \cup [-8,14)$\\
D.$x \in (-\infty,-18] \cup (-8,14)$\\
E.$x \in (-\infty,-18] \cup (-8,14]$\\
F.$x \in (-\infty,-18] \cup [-8,14)$\\
G.$x \in (-\infty,-18) \cup [-8,14]$\\
H.$x \in (-\infty,-18] \cup [-8,14]$
\testStop
\kluczStart
A
\kluczStop



\zadStart{Zadanie z Wikieł Z 1.62 b) moja wersja nr 767}

Rozwiązać nierówności $(x+18)(14-x)(x+9)\ge0$.
\zadStop
\rozwStart{Patryk Wirkus}{}
Miejsca zerowe naszego wielomianu to: $-18, 14, -9$.\\
Wielomian jest stopnia nieparzystego, ponadto znak współczynnika przy\linebreak najwyższej potędze x jest ujemny.\\ W związku z tym wykres wielomianu zaczyna się od lewej strony powyżej osi OX. A więc $$x \in (-\infty,-18) \cup (-9,14).$$
\rozwStop
\odpStart
$x \in (-\infty,-18) \cup (-9,14)$
\odpStop
\testStart
A.$x \in (-\infty,-18) \cup (-9,14)$\\
B.$x \in (-\infty,-18) \cup (-9,14]$\\
C.$x \in (-\infty,-18) \cup [-9,14)$\\
D.$x \in (-\infty,-18] \cup (-9,14)$\\
E.$x \in (-\infty,-18] \cup (-9,14]$\\
F.$x \in (-\infty,-18] \cup [-9,14)$\\
G.$x \in (-\infty,-18) \cup [-9,14]$\\
H.$x \in (-\infty,-18] \cup [-9,14]$
\testStop
\kluczStart
A
\kluczStop



\zadStart{Zadanie z Wikieł Z 1.62 b) moja wersja nr 768}

Rozwiązać nierówności $(x+18)(14-x)(x+10)\ge0$.
\zadStop
\rozwStart{Patryk Wirkus}{}
Miejsca zerowe naszego wielomianu to: $-18, 14, -10$.\\
Wielomian jest stopnia nieparzystego, ponadto znak współczynnika przy\linebreak najwyższej potędze x jest ujemny.\\ W związku z tym wykres wielomianu zaczyna się od lewej strony powyżej osi OX. A więc $$x \in (-\infty,-18) \cup (-10,14).$$
\rozwStop
\odpStart
$x \in (-\infty,-18) \cup (-10,14)$
\odpStop
\testStart
A.$x \in (-\infty,-18) \cup (-10,14)$\\
B.$x \in (-\infty,-18) \cup (-10,14]$\\
C.$x \in (-\infty,-18) \cup [-10,14)$\\
D.$x \in (-\infty,-18] \cup (-10,14)$\\
E.$x \in (-\infty,-18] \cup (-10,14]$\\
F.$x \in (-\infty,-18] \cup [-10,14)$\\
G.$x \in (-\infty,-18) \cup [-10,14]$\\
H.$x \in (-\infty,-18] \cup [-10,14]$
\testStop
\kluczStart
A
\kluczStop



\zadStart{Zadanie z Wikieł Z 1.62 b) moja wersja nr 769}

Rozwiązać nierówności $(x+18)(14-x)(x+11)\ge0$.
\zadStop
\rozwStart{Patryk Wirkus}{}
Miejsca zerowe naszego wielomianu to: $-18, 14, -11$.\\
Wielomian jest stopnia nieparzystego, ponadto znak współczynnika przy\linebreak najwyższej potędze x jest ujemny.\\ W związku z tym wykres wielomianu zaczyna się od lewej strony powyżej osi OX. A więc $$x \in (-\infty,-18) \cup (-11,14).$$
\rozwStop
\odpStart
$x \in (-\infty,-18) \cup (-11,14)$
\odpStop
\testStart
A.$x \in (-\infty,-18) \cup (-11,14)$\\
B.$x \in (-\infty,-18) \cup (-11,14]$\\
C.$x \in (-\infty,-18) \cup [-11,14)$\\
D.$x \in (-\infty,-18] \cup (-11,14)$\\
E.$x \in (-\infty,-18] \cup (-11,14]$\\
F.$x \in (-\infty,-18] \cup [-11,14)$\\
G.$x \in (-\infty,-18) \cup [-11,14]$\\
H.$x \in (-\infty,-18] \cup [-11,14]$
\testStop
\kluczStart
A
\kluczStop



\zadStart{Zadanie z Wikieł Z 1.62 b) moja wersja nr 770}

Rozwiązać nierówności $(x+18)(14-x)(x+12)\ge0$.
\zadStop
\rozwStart{Patryk Wirkus}{}
Miejsca zerowe naszego wielomianu to: $-18, 14, -12$.\\
Wielomian jest stopnia nieparzystego, ponadto znak współczynnika przy\linebreak najwyższej potędze x jest ujemny.\\ W związku z tym wykres wielomianu zaczyna się od lewej strony powyżej osi OX. A więc $$x \in (-\infty,-18) \cup (-12,14).$$
\rozwStop
\odpStart
$x \in (-\infty,-18) \cup (-12,14)$
\odpStop
\testStart
A.$x \in (-\infty,-18) \cup (-12,14)$\\
B.$x \in (-\infty,-18) \cup (-12,14]$\\
C.$x \in (-\infty,-18) \cup [-12,14)$\\
D.$x \in (-\infty,-18] \cup (-12,14)$\\
E.$x \in (-\infty,-18] \cup (-12,14]$\\
F.$x \in (-\infty,-18] \cup [-12,14)$\\
G.$x \in (-\infty,-18) \cup [-12,14]$\\
H.$x \in (-\infty,-18] \cup [-12,14]$
\testStop
\kluczStart
A
\kluczStop



\zadStart{Zadanie z Wikieł Z 1.62 b) moja wersja nr 771}

Rozwiązać nierówności $(x+18)(14-x)(x+13)\ge0$.
\zadStop
\rozwStart{Patryk Wirkus}{}
Miejsca zerowe naszego wielomianu to: $-18, 14, -13$.\\
Wielomian jest stopnia nieparzystego, ponadto znak współczynnika przy\linebreak najwyższej potędze x jest ujemny.\\ W związku z tym wykres wielomianu zaczyna się od lewej strony powyżej osi OX. A więc $$x \in (-\infty,-18) \cup (-13,14).$$
\rozwStop
\odpStart
$x \in (-\infty,-18) \cup (-13,14)$
\odpStop
\testStart
A.$x \in (-\infty,-18) \cup (-13,14)$\\
B.$x \in (-\infty,-18) \cup (-13,14]$\\
C.$x \in (-\infty,-18) \cup [-13,14)$\\
D.$x \in (-\infty,-18] \cup (-13,14)$\\
E.$x \in (-\infty,-18] \cup (-13,14]$\\
F.$x \in (-\infty,-18] \cup [-13,14)$\\
G.$x \in (-\infty,-18) \cup [-13,14]$\\
H.$x \in (-\infty,-18] \cup [-13,14]$
\testStop
\kluczStart
A
\kluczStop



\zadStart{Zadanie z Wikieł Z 1.62 b) moja wersja nr 772}

Rozwiązać nierówności $(x+18)(15-x)(x+1)\ge0$.
\zadStop
\rozwStart{Patryk Wirkus}{}
Miejsca zerowe naszego wielomianu to: $-18, 15, -1$.\\
Wielomian jest stopnia nieparzystego, ponadto znak współczynnika przy\linebreak najwyższej potędze x jest ujemny.\\ W związku z tym wykres wielomianu zaczyna się od lewej strony powyżej osi OX. A więc $$x \in (-\infty,-18) \cup (-1,15).$$
\rozwStop
\odpStart
$x \in (-\infty,-18) \cup (-1,15)$
\odpStop
\testStart
A.$x \in (-\infty,-18) \cup (-1,15)$\\
B.$x \in (-\infty,-18) \cup (-1,15]$\\
C.$x \in (-\infty,-18) \cup [-1,15)$\\
D.$x \in (-\infty,-18] \cup (-1,15)$\\
E.$x \in (-\infty,-18] \cup (-1,15]$\\
F.$x \in (-\infty,-18] \cup [-1,15)$\\
G.$x \in (-\infty,-18) \cup [-1,15]$\\
H.$x \in (-\infty,-18] \cup [-1,15]$
\testStop
\kluczStart
A
\kluczStop



\zadStart{Zadanie z Wikieł Z 1.62 b) moja wersja nr 773}

Rozwiązać nierówności $(x+18)(15-x)(x+2)\ge0$.
\zadStop
\rozwStart{Patryk Wirkus}{}
Miejsca zerowe naszego wielomianu to: $-18, 15, -2$.\\
Wielomian jest stopnia nieparzystego, ponadto znak współczynnika przy\linebreak najwyższej potędze x jest ujemny.\\ W związku z tym wykres wielomianu zaczyna się od lewej strony powyżej osi OX. A więc $$x \in (-\infty,-18) \cup (-2,15).$$
\rozwStop
\odpStart
$x \in (-\infty,-18) \cup (-2,15)$
\odpStop
\testStart
A.$x \in (-\infty,-18) \cup (-2,15)$\\
B.$x \in (-\infty,-18) \cup (-2,15]$\\
C.$x \in (-\infty,-18) \cup [-2,15)$\\
D.$x \in (-\infty,-18] \cup (-2,15)$\\
E.$x \in (-\infty,-18] \cup (-2,15]$\\
F.$x \in (-\infty,-18] \cup [-2,15)$\\
G.$x \in (-\infty,-18) \cup [-2,15]$\\
H.$x \in (-\infty,-18] \cup [-2,15]$
\testStop
\kluczStart
A
\kluczStop



\zadStart{Zadanie z Wikieł Z 1.62 b) moja wersja nr 774}

Rozwiązać nierówności $(x+18)(15-x)(x+3)\ge0$.
\zadStop
\rozwStart{Patryk Wirkus}{}
Miejsca zerowe naszego wielomianu to: $-18, 15, -3$.\\
Wielomian jest stopnia nieparzystego, ponadto znak współczynnika przy\linebreak najwyższej potędze x jest ujemny.\\ W związku z tym wykres wielomianu zaczyna się od lewej strony powyżej osi OX. A więc $$x \in (-\infty,-18) \cup (-3,15).$$
\rozwStop
\odpStart
$x \in (-\infty,-18) \cup (-3,15)$
\odpStop
\testStart
A.$x \in (-\infty,-18) \cup (-3,15)$\\
B.$x \in (-\infty,-18) \cup (-3,15]$\\
C.$x \in (-\infty,-18) \cup [-3,15)$\\
D.$x \in (-\infty,-18] \cup (-3,15)$\\
E.$x \in (-\infty,-18] \cup (-3,15]$\\
F.$x \in (-\infty,-18] \cup [-3,15)$\\
G.$x \in (-\infty,-18) \cup [-3,15]$\\
H.$x \in (-\infty,-18] \cup [-3,15]$
\testStop
\kluczStart
A
\kluczStop



\zadStart{Zadanie z Wikieł Z 1.62 b) moja wersja nr 775}

Rozwiązać nierówności $(x+18)(15-x)(x+4)\ge0$.
\zadStop
\rozwStart{Patryk Wirkus}{}
Miejsca zerowe naszego wielomianu to: $-18, 15, -4$.\\
Wielomian jest stopnia nieparzystego, ponadto znak współczynnika przy\linebreak najwyższej potędze x jest ujemny.\\ W związku z tym wykres wielomianu zaczyna się od lewej strony powyżej osi OX. A więc $$x \in (-\infty,-18) \cup (-4,15).$$
\rozwStop
\odpStart
$x \in (-\infty,-18) \cup (-4,15)$
\odpStop
\testStart
A.$x \in (-\infty,-18) \cup (-4,15)$\\
B.$x \in (-\infty,-18) \cup (-4,15]$\\
C.$x \in (-\infty,-18) \cup [-4,15)$\\
D.$x \in (-\infty,-18] \cup (-4,15)$\\
E.$x \in (-\infty,-18] \cup (-4,15]$\\
F.$x \in (-\infty,-18] \cup [-4,15)$\\
G.$x \in (-\infty,-18) \cup [-4,15]$\\
H.$x \in (-\infty,-18] \cup [-4,15]$
\testStop
\kluczStart
A
\kluczStop



\zadStart{Zadanie z Wikieł Z 1.62 b) moja wersja nr 776}

Rozwiązać nierówności $(x+18)(15-x)(x+5)\ge0$.
\zadStop
\rozwStart{Patryk Wirkus}{}
Miejsca zerowe naszego wielomianu to: $-18, 15, -5$.\\
Wielomian jest stopnia nieparzystego, ponadto znak współczynnika przy\linebreak najwyższej potędze x jest ujemny.\\ W związku z tym wykres wielomianu zaczyna się od lewej strony powyżej osi OX. A więc $$x \in (-\infty,-18) \cup (-5,15).$$
\rozwStop
\odpStart
$x \in (-\infty,-18) \cup (-5,15)$
\odpStop
\testStart
A.$x \in (-\infty,-18) \cup (-5,15)$\\
B.$x \in (-\infty,-18) \cup (-5,15]$\\
C.$x \in (-\infty,-18) \cup [-5,15)$\\
D.$x \in (-\infty,-18] \cup (-5,15)$\\
E.$x \in (-\infty,-18] \cup (-5,15]$\\
F.$x \in (-\infty,-18] \cup [-5,15)$\\
G.$x \in (-\infty,-18) \cup [-5,15]$\\
H.$x \in (-\infty,-18] \cup [-5,15]$
\testStop
\kluczStart
A
\kluczStop



\zadStart{Zadanie z Wikieł Z 1.62 b) moja wersja nr 777}

Rozwiązać nierówności $(x+18)(15-x)(x+6)\ge0$.
\zadStop
\rozwStart{Patryk Wirkus}{}
Miejsca zerowe naszego wielomianu to: $-18, 15, -6$.\\
Wielomian jest stopnia nieparzystego, ponadto znak współczynnika przy\linebreak najwyższej potędze x jest ujemny.\\ W związku z tym wykres wielomianu zaczyna się od lewej strony powyżej osi OX. A więc $$x \in (-\infty,-18) \cup (-6,15).$$
\rozwStop
\odpStart
$x \in (-\infty,-18) \cup (-6,15)$
\odpStop
\testStart
A.$x \in (-\infty,-18) \cup (-6,15)$\\
B.$x \in (-\infty,-18) \cup (-6,15]$\\
C.$x \in (-\infty,-18) \cup [-6,15)$\\
D.$x \in (-\infty,-18] \cup (-6,15)$\\
E.$x \in (-\infty,-18] \cup (-6,15]$\\
F.$x \in (-\infty,-18] \cup [-6,15)$\\
G.$x \in (-\infty,-18) \cup [-6,15]$\\
H.$x \in (-\infty,-18] \cup [-6,15]$
\testStop
\kluczStart
A
\kluczStop



\zadStart{Zadanie z Wikieł Z 1.62 b) moja wersja nr 778}

Rozwiązać nierówności $(x+18)(15-x)(x+7)\ge0$.
\zadStop
\rozwStart{Patryk Wirkus}{}
Miejsca zerowe naszego wielomianu to: $-18, 15, -7$.\\
Wielomian jest stopnia nieparzystego, ponadto znak współczynnika przy\linebreak najwyższej potędze x jest ujemny.\\ W związku z tym wykres wielomianu zaczyna się od lewej strony powyżej osi OX. A więc $$x \in (-\infty,-18) \cup (-7,15).$$
\rozwStop
\odpStart
$x \in (-\infty,-18) \cup (-7,15)$
\odpStop
\testStart
A.$x \in (-\infty,-18) \cup (-7,15)$\\
B.$x \in (-\infty,-18) \cup (-7,15]$\\
C.$x \in (-\infty,-18) \cup [-7,15)$\\
D.$x \in (-\infty,-18] \cup (-7,15)$\\
E.$x \in (-\infty,-18] \cup (-7,15]$\\
F.$x \in (-\infty,-18] \cup [-7,15)$\\
G.$x \in (-\infty,-18) \cup [-7,15]$\\
H.$x \in (-\infty,-18] \cup [-7,15]$
\testStop
\kluczStart
A
\kluczStop



\zadStart{Zadanie z Wikieł Z 1.62 b) moja wersja nr 779}

Rozwiązać nierówności $(x+18)(15-x)(x+8)\ge0$.
\zadStop
\rozwStart{Patryk Wirkus}{}
Miejsca zerowe naszego wielomianu to: $-18, 15, -8$.\\
Wielomian jest stopnia nieparzystego, ponadto znak współczynnika przy\linebreak najwyższej potędze x jest ujemny.\\ W związku z tym wykres wielomianu zaczyna się od lewej strony powyżej osi OX. A więc $$x \in (-\infty,-18) \cup (-8,15).$$
\rozwStop
\odpStart
$x \in (-\infty,-18) \cup (-8,15)$
\odpStop
\testStart
A.$x \in (-\infty,-18) \cup (-8,15)$\\
B.$x \in (-\infty,-18) \cup (-8,15]$\\
C.$x \in (-\infty,-18) \cup [-8,15)$\\
D.$x \in (-\infty,-18] \cup (-8,15)$\\
E.$x \in (-\infty,-18] \cup (-8,15]$\\
F.$x \in (-\infty,-18] \cup [-8,15)$\\
G.$x \in (-\infty,-18) \cup [-8,15]$\\
H.$x \in (-\infty,-18] \cup [-8,15]$
\testStop
\kluczStart
A
\kluczStop



\zadStart{Zadanie z Wikieł Z 1.62 b) moja wersja nr 780}

Rozwiązać nierówności $(x+18)(15-x)(x+9)\ge0$.
\zadStop
\rozwStart{Patryk Wirkus}{}
Miejsca zerowe naszego wielomianu to: $-18, 15, -9$.\\
Wielomian jest stopnia nieparzystego, ponadto znak współczynnika przy\linebreak najwyższej potędze x jest ujemny.\\ W związku z tym wykres wielomianu zaczyna się od lewej strony powyżej osi OX. A więc $$x \in (-\infty,-18) \cup (-9,15).$$
\rozwStop
\odpStart
$x \in (-\infty,-18) \cup (-9,15)$
\odpStop
\testStart
A.$x \in (-\infty,-18) \cup (-9,15)$\\
B.$x \in (-\infty,-18) \cup (-9,15]$\\
C.$x \in (-\infty,-18) \cup [-9,15)$\\
D.$x \in (-\infty,-18] \cup (-9,15)$\\
E.$x \in (-\infty,-18] \cup (-9,15]$\\
F.$x \in (-\infty,-18] \cup [-9,15)$\\
G.$x \in (-\infty,-18) \cup [-9,15]$\\
H.$x \in (-\infty,-18] \cup [-9,15]$
\testStop
\kluczStart
A
\kluczStop



\zadStart{Zadanie z Wikieł Z 1.62 b) moja wersja nr 781}

Rozwiązać nierówności $(x+18)(15-x)(x+10)\ge0$.
\zadStop
\rozwStart{Patryk Wirkus}{}
Miejsca zerowe naszego wielomianu to: $-18, 15, -10$.\\
Wielomian jest stopnia nieparzystego, ponadto znak współczynnika przy\linebreak najwyższej potędze x jest ujemny.\\ W związku z tym wykres wielomianu zaczyna się od lewej strony powyżej osi OX. A więc $$x \in (-\infty,-18) \cup (-10,15).$$
\rozwStop
\odpStart
$x \in (-\infty,-18) \cup (-10,15)$
\odpStop
\testStart
A.$x \in (-\infty,-18) \cup (-10,15)$\\
B.$x \in (-\infty,-18) \cup (-10,15]$\\
C.$x \in (-\infty,-18) \cup [-10,15)$\\
D.$x \in (-\infty,-18] \cup (-10,15)$\\
E.$x \in (-\infty,-18] \cup (-10,15]$\\
F.$x \in (-\infty,-18] \cup [-10,15)$\\
G.$x \in (-\infty,-18) \cup [-10,15]$\\
H.$x \in (-\infty,-18] \cup [-10,15]$
\testStop
\kluczStart
A
\kluczStop



\zadStart{Zadanie z Wikieł Z 1.62 b) moja wersja nr 782}

Rozwiązać nierówności $(x+18)(15-x)(x+11)\ge0$.
\zadStop
\rozwStart{Patryk Wirkus}{}
Miejsca zerowe naszego wielomianu to: $-18, 15, -11$.\\
Wielomian jest stopnia nieparzystego, ponadto znak współczynnika przy\linebreak najwyższej potędze x jest ujemny.\\ W związku z tym wykres wielomianu zaczyna się od lewej strony powyżej osi OX. A więc $$x \in (-\infty,-18) \cup (-11,15).$$
\rozwStop
\odpStart
$x \in (-\infty,-18) \cup (-11,15)$
\odpStop
\testStart
A.$x \in (-\infty,-18) \cup (-11,15)$\\
B.$x \in (-\infty,-18) \cup (-11,15]$\\
C.$x \in (-\infty,-18) \cup [-11,15)$\\
D.$x \in (-\infty,-18] \cup (-11,15)$\\
E.$x \in (-\infty,-18] \cup (-11,15]$\\
F.$x \in (-\infty,-18] \cup [-11,15)$\\
G.$x \in (-\infty,-18) \cup [-11,15]$\\
H.$x \in (-\infty,-18] \cup [-11,15]$
\testStop
\kluczStart
A
\kluczStop



\zadStart{Zadanie z Wikieł Z 1.62 b) moja wersja nr 783}

Rozwiązać nierówności $(x+18)(15-x)(x+12)\ge0$.
\zadStop
\rozwStart{Patryk Wirkus}{}
Miejsca zerowe naszego wielomianu to: $-18, 15, -12$.\\
Wielomian jest stopnia nieparzystego, ponadto znak współczynnika przy\linebreak najwyższej potędze x jest ujemny.\\ W związku z tym wykres wielomianu zaczyna się od lewej strony powyżej osi OX. A więc $$x \in (-\infty,-18) \cup (-12,15).$$
\rozwStop
\odpStart
$x \in (-\infty,-18) \cup (-12,15)$
\odpStop
\testStart
A.$x \in (-\infty,-18) \cup (-12,15)$\\
B.$x \in (-\infty,-18) \cup (-12,15]$\\
C.$x \in (-\infty,-18) \cup [-12,15)$\\
D.$x \in (-\infty,-18] \cup (-12,15)$\\
E.$x \in (-\infty,-18] \cup (-12,15]$\\
F.$x \in (-\infty,-18] \cup [-12,15)$\\
G.$x \in (-\infty,-18) \cup [-12,15]$\\
H.$x \in (-\infty,-18] \cup [-12,15]$
\testStop
\kluczStart
A
\kluczStop



\zadStart{Zadanie z Wikieł Z 1.62 b) moja wersja nr 784}

Rozwiązać nierówności $(x+18)(15-x)(x+13)\ge0$.
\zadStop
\rozwStart{Patryk Wirkus}{}
Miejsca zerowe naszego wielomianu to: $-18, 15, -13$.\\
Wielomian jest stopnia nieparzystego, ponadto znak współczynnika przy\linebreak najwyższej potędze x jest ujemny.\\ W związku z tym wykres wielomianu zaczyna się od lewej strony powyżej osi OX. A więc $$x \in (-\infty,-18) \cup (-13,15).$$
\rozwStop
\odpStart
$x \in (-\infty,-18) \cup (-13,15)$
\odpStop
\testStart
A.$x \in (-\infty,-18) \cup (-13,15)$\\
B.$x \in (-\infty,-18) \cup (-13,15]$\\
C.$x \in (-\infty,-18) \cup [-13,15)$\\
D.$x \in (-\infty,-18] \cup (-13,15)$\\
E.$x \in (-\infty,-18] \cup (-13,15]$\\
F.$x \in (-\infty,-18] \cup [-13,15)$\\
G.$x \in (-\infty,-18) \cup [-13,15]$\\
H.$x \in (-\infty,-18] \cup [-13,15]$
\testStop
\kluczStart
A
\kluczStop



\zadStart{Zadanie z Wikieł Z 1.62 b) moja wersja nr 785}

Rozwiązać nierówności $(x+18)(15-x)(x+14)\ge0$.
\zadStop
\rozwStart{Patryk Wirkus}{}
Miejsca zerowe naszego wielomianu to: $-18, 15, -14$.\\
Wielomian jest stopnia nieparzystego, ponadto znak współczynnika przy\linebreak najwyższej potędze x jest ujemny.\\ W związku z tym wykres wielomianu zaczyna się od lewej strony powyżej osi OX. A więc $$x \in (-\infty,-18) \cup (-14,15).$$
\rozwStop
\odpStart
$x \in (-\infty,-18) \cup (-14,15)$
\odpStop
\testStart
A.$x \in (-\infty,-18) \cup (-14,15)$\\
B.$x \in (-\infty,-18) \cup (-14,15]$\\
C.$x \in (-\infty,-18) \cup [-14,15)$\\
D.$x \in (-\infty,-18] \cup (-14,15)$\\
E.$x \in (-\infty,-18] \cup (-14,15]$\\
F.$x \in (-\infty,-18] \cup [-14,15)$\\
G.$x \in (-\infty,-18) \cup [-14,15]$\\
H.$x \in (-\infty,-18] \cup [-14,15]$
\testStop
\kluczStart
A
\kluczStop



\zadStart{Zadanie z Wikieł Z 1.62 b) moja wersja nr 786}

Rozwiązać nierówności $(x+18)(16-x)(x+1)\ge0$.
\zadStop
\rozwStart{Patryk Wirkus}{}
Miejsca zerowe naszego wielomianu to: $-18, 16, -1$.\\
Wielomian jest stopnia nieparzystego, ponadto znak współczynnika przy\linebreak najwyższej potędze x jest ujemny.\\ W związku z tym wykres wielomianu zaczyna się od lewej strony powyżej osi OX. A więc $$x \in (-\infty,-18) \cup (-1,16).$$
\rozwStop
\odpStart
$x \in (-\infty,-18) \cup (-1,16)$
\odpStop
\testStart
A.$x \in (-\infty,-18) \cup (-1,16)$\\
B.$x \in (-\infty,-18) \cup (-1,16]$\\
C.$x \in (-\infty,-18) \cup [-1,16)$\\
D.$x \in (-\infty,-18] \cup (-1,16)$\\
E.$x \in (-\infty,-18] \cup (-1,16]$\\
F.$x \in (-\infty,-18] \cup [-1,16)$\\
G.$x \in (-\infty,-18) \cup [-1,16]$\\
H.$x \in (-\infty,-18] \cup [-1,16]$
\testStop
\kluczStart
A
\kluczStop



\zadStart{Zadanie z Wikieł Z 1.62 b) moja wersja nr 787}

Rozwiązać nierówności $(x+18)(16-x)(x+2)\ge0$.
\zadStop
\rozwStart{Patryk Wirkus}{}
Miejsca zerowe naszego wielomianu to: $-18, 16, -2$.\\
Wielomian jest stopnia nieparzystego, ponadto znak współczynnika przy\linebreak najwyższej potędze x jest ujemny.\\ W związku z tym wykres wielomianu zaczyna się od lewej strony powyżej osi OX. A więc $$x \in (-\infty,-18) \cup (-2,16).$$
\rozwStop
\odpStart
$x \in (-\infty,-18) \cup (-2,16)$
\odpStop
\testStart
A.$x \in (-\infty,-18) \cup (-2,16)$\\
B.$x \in (-\infty,-18) \cup (-2,16]$\\
C.$x \in (-\infty,-18) \cup [-2,16)$\\
D.$x \in (-\infty,-18] \cup (-2,16)$\\
E.$x \in (-\infty,-18] \cup (-2,16]$\\
F.$x \in (-\infty,-18] \cup [-2,16)$\\
G.$x \in (-\infty,-18) \cup [-2,16]$\\
H.$x \in (-\infty,-18] \cup [-2,16]$
\testStop
\kluczStart
A
\kluczStop



\zadStart{Zadanie z Wikieł Z 1.62 b) moja wersja nr 788}

Rozwiązać nierówności $(x+18)(16-x)(x+3)\ge0$.
\zadStop
\rozwStart{Patryk Wirkus}{}
Miejsca zerowe naszego wielomianu to: $-18, 16, -3$.\\
Wielomian jest stopnia nieparzystego, ponadto znak współczynnika przy\linebreak najwyższej potędze x jest ujemny.\\ W związku z tym wykres wielomianu zaczyna się od lewej strony powyżej osi OX. A więc $$x \in (-\infty,-18) \cup (-3,16).$$
\rozwStop
\odpStart
$x \in (-\infty,-18) \cup (-3,16)$
\odpStop
\testStart
A.$x \in (-\infty,-18) \cup (-3,16)$\\
B.$x \in (-\infty,-18) \cup (-3,16]$\\
C.$x \in (-\infty,-18) \cup [-3,16)$\\
D.$x \in (-\infty,-18] \cup (-3,16)$\\
E.$x \in (-\infty,-18] \cup (-3,16]$\\
F.$x \in (-\infty,-18] \cup [-3,16)$\\
G.$x \in (-\infty,-18) \cup [-3,16]$\\
H.$x \in (-\infty,-18] \cup [-3,16]$
\testStop
\kluczStart
A
\kluczStop



\zadStart{Zadanie z Wikieł Z 1.62 b) moja wersja nr 789}

Rozwiązać nierówności $(x+18)(16-x)(x+4)\ge0$.
\zadStop
\rozwStart{Patryk Wirkus}{}
Miejsca zerowe naszego wielomianu to: $-18, 16, -4$.\\
Wielomian jest stopnia nieparzystego, ponadto znak współczynnika przy\linebreak najwyższej potędze x jest ujemny.\\ W związku z tym wykres wielomianu zaczyna się od lewej strony powyżej osi OX. A więc $$x \in (-\infty,-18) \cup (-4,16).$$
\rozwStop
\odpStart
$x \in (-\infty,-18) \cup (-4,16)$
\odpStop
\testStart
A.$x \in (-\infty,-18) \cup (-4,16)$\\
B.$x \in (-\infty,-18) \cup (-4,16]$\\
C.$x \in (-\infty,-18) \cup [-4,16)$\\
D.$x \in (-\infty,-18] \cup (-4,16)$\\
E.$x \in (-\infty,-18] \cup (-4,16]$\\
F.$x \in (-\infty,-18] \cup [-4,16)$\\
G.$x \in (-\infty,-18) \cup [-4,16]$\\
H.$x \in (-\infty,-18] \cup [-4,16]$
\testStop
\kluczStart
A
\kluczStop



\zadStart{Zadanie z Wikieł Z 1.62 b) moja wersja nr 790}

Rozwiązać nierówności $(x+18)(16-x)(x+5)\ge0$.
\zadStop
\rozwStart{Patryk Wirkus}{}
Miejsca zerowe naszego wielomianu to: $-18, 16, -5$.\\
Wielomian jest stopnia nieparzystego, ponadto znak współczynnika przy\linebreak najwyższej potędze x jest ujemny.\\ W związku z tym wykres wielomianu zaczyna się od lewej strony powyżej osi OX. A więc $$x \in (-\infty,-18) \cup (-5,16).$$
\rozwStop
\odpStart
$x \in (-\infty,-18) \cup (-5,16)$
\odpStop
\testStart
A.$x \in (-\infty,-18) \cup (-5,16)$\\
B.$x \in (-\infty,-18) \cup (-5,16]$\\
C.$x \in (-\infty,-18) \cup [-5,16)$\\
D.$x \in (-\infty,-18] \cup (-5,16)$\\
E.$x \in (-\infty,-18] \cup (-5,16]$\\
F.$x \in (-\infty,-18] \cup [-5,16)$\\
G.$x \in (-\infty,-18) \cup [-5,16]$\\
H.$x \in (-\infty,-18] \cup [-5,16]$
\testStop
\kluczStart
A
\kluczStop



\zadStart{Zadanie z Wikieł Z 1.62 b) moja wersja nr 791}

Rozwiązać nierówności $(x+18)(16-x)(x+6)\ge0$.
\zadStop
\rozwStart{Patryk Wirkus}{}
Miejsca zerowe naszego wielomianu to: $-18, 16, -6$.\\
Wielomian jest stopnia nieparzystego, ponadto znak współczynnika przy\linebreak najwyższej potędze x jest ujemny.\\ W związku z tym wykres wielomianu zaczyna się od lewej strony powyżej osi OX. A więc $$x \in (-\infty,-18) \cup (-6,16).$$
\rozwStop
\odpStart
$x \in (-\infty,-18) \cup (-6,16)$
\odpStop
\testStart
A.$x \in (-\infty,-18) \cup (-6,16)$\\
B.$x \in (-\infty,-18) \cup (-6,16]$\\
C.$x \in (-\infty,-18) \cup [-6,16)$\\
D.$x \in (-\infty,-18] \cup (-6,16)$\\
E.$x \in (-\infty,-18] \cup (-6,16]$\\
F.$x \in (-\infty,-18] \cup [-6,16)$\\
G.$x \in (-\infty,-18) \cup [-6,16]$\\
H.$x \in (-\infty,-18] \cup [-6,16]$
\testStop
\kluczStart
A
\kluczStop



\zadStart{Zadanie z Wikieł Z 1.62 b) moja wersja nr 792}

Rozwiązać nierówności $(x+18)(16-x)(x+7)\ge0$.
\zadStop
\rozwStart{Patryk Wirkus}{}
Miejsca zerowe naszego wielomianu to: $-18, 16, -7$.\\
Wielomian jest stopnia nieparzystego, ponadto znak współczynnika przy\linebreak najwyższej potędze x jest ujemny.\\ W związku z tym wykres wielomianu zaczyna się od lewej strony powyżej osi OX. A więc $$x \in (-\infty,-18) \cup (-7,16).$$
\rozwStop
\odpStart
$x \in (-\infty,-18) \cup (-7,16)$
\odpStop
\testStart
A.$x \in (-\infty,-18) \cup (-7,16)$\\
B.$x \in (-\infty,-18) \cup (-7,16]$\\
C.$x \in (-\infty,-18) \cup [-7,16)$\\
D.$x \in (-\infty,-18] \cup (-7,16)$\\
E.$x \in (-\infty,-18] \cup (-7,16]$\\
F.$x \in (-\infty,-18] \cup [-7,16)$\\
G.$x \in (-\infty,-18) \cup [-7,16]$\\
H.$x \in (-\infty,-18] \cup [-7,16]$
\testStop
\kluczStart
A
\kluczStop



\zadStart{Zadanie z Wikieł Z 1.62 b) moja wersja nr 793}

Rozwiązać nierówności $(x+18)(16-x)(x+8)\ge0$.
\zadStop
\rozwStart{Patryk Wirkus}{}
Miejsca zerowe naszego wielomianu to: $-18, 16, -8$.\\
Wielomian jest stopnia nieparzystego, ponadto znak współczynnika przy\linebreak najwyższej potędze x jest ujemny.\\ W związku z tym wykres wielomianu zaczyna się od lewej strony powyżej osi OX. A więc $$x \in (-\infty,-18) \cup (-8,16).$$
\rozwStop
\odpStart
$x \in (-\infty,-18) \cup (-8,16)$
\odpStop
\testStart
A.$x \in (-\infty,-18) \cup (-8,16)$\\
B.$x \in (-\infty,-18) \cup (-8,16]$\\
C.$x \in (-\infty,-18) \cup [-8,16)$\\
D.$x \in (-\infty,-18] \cup (-8,16)$\\
E.$x \in (-\infty,-18] \cup (-8,16]$\\
F.$x \in (-\infty,-18] \cup [-8,16)$\\
G.$x \in (-\infty,-18) \cup [-8,16]$\\
H.$x \in (-\infty,-18] \cup [-8,16]$
\testStop
\kluczStart
A
\kluczStop



\zadStart{Zadanie z Wikieł Z 1.62 b) moja wersja nr 794}

Rozwiązać nierówności $(x+18)(16-x)(x+9)\ge0$.
\zadStop
\rozwStart{Patryk Wirkus}{}
Miejsca zerowe naszego wielomianu to: $-18, 16, -9$.\\
Wielomian jest stopnia nieparzystego, ponadto znak współczynnika przy\linebreak najwyższej potędze x jest ujemny.\\ W związku z tym wykres wielomianu zaczyna się od lewej strony powyżej osi OX. A więc $$x \in (-\infty,-18) \cup (-9,16).$$
\rozwStop
\odpStart
$x \in (-\infty,-18) \cup (-9,16)$
\odpStop
\testStart
A.$x \in (-\infty,-18) \cup (-9,16)$\\
B.$x \in (-\infty,-18) \cup (-9,16]$\\
C.$x \in (-\infty,-18) \cup [-9,16)$\\
D.$x \in (-\infty,-18] \cup (-9,16)$\\
E.$x \in (-\infty,-18] \cup (-9,16]$\\
F.$x \in (-\infty,-18] \cup [-9,16)$\\
G.$x \in (-\infty,-18) \cup [-9,16]$\\
H.$x \in (-\infty,-18] \cup [-9,16]$
\testStop
\kluczStart
A
\kluczStop



\zadStart{Zadanie z Wikieł Z 1.62 b) moja wersja nr 795}

Rozwiązać nierówności $(x+18)(16-x)(x+10)\ge0$.
\zadStop
\rozwStart{Patryk Wirkus}{}
Miejsca zerowe naszego wielomianu to: $-18, 16, -10$.\\
Wielomian jest stopnia nieparzystego, ponadto znak współczynnika przy\linebreak najwyższej potędze x jest ujemny.\\ W związku z tym wykres wielomianu zaczyna się od lewej strony powyżej osi OX. A więc $$x \in (-\infty,-18) \cup (-10,16).$$
\rozwStop
\odpStart
$x \in (-\infty,-18) \cup (-10,16)$
\odpStop
\testStart
A.$x \in (-\infty,-18) \cup (-10,16)$\\
B.$x \in (-\infty,-18) \cup (-10,16]$\\
C.$x \in (-\infty,-18) \cup [-10,16)$\\
D.$x \in (-\infty,-18] \cup (-10,16)$\\
E.$x \in (-\infty,-18] \cup (-10,16]$\\
F.$x \in (-\infty,-18] \cup [-10,16)$\\
G.$x \in (-\infty,-18) \cup [-10,16]$\\
H.$x \in (-\infty,-18] \cup [-10,16]$
\testStop
\kluczStart
A
\kluczStop



\zadStart{Zadanie z Wikieł Z 1.62 b) moja wersja nr 796}

Rozwiązać nierówności $(x+18)(16-x)(x+11)\ge0$.
\zadStop
\rozwStart{Patryk Wirkus}{}
Miejsca zerowe naszego wielomianu to: $-18, 16, -11$.\\
Wielomian jest stopnia nieparzystego, ponadto znak współczynnika przy\linebreak najwyższej potędze x jest ujemny.\\ W związku z tym wykres wielomianu zaczyna się od lewej strony powyżej osi OX. A więc $$x \in (-\infty,-18) \cup (-11,16).$$
\rozwStop
\odpStart
$x \in (-\infty,-18) \cup (-11,16)$
\odpStop
\testStart
A.$x \in (-\infty,-18) \cup (-11,16)$\\
B.$x \in (-\infty,-18) \cup (-11,16]$\\
C.$x \in (-\infty,-18) \cup [-11,16)$\\
D.$x \in (-\infty,-18] \cup (-11,16)$\\
E.$x \in (-\infty,-18] \cup (-11,16]$\\
F.$x \in (-\infty,-18] \cup [-11,16)$\\
G.$x \in (-\infty,-18) \cup [-11,16]$\\
H.$x \in (-\infty,-18] \cup [-11,16]$
\testStop
\kluczStart
A
\kluczStop



\zadStart{Zadanie z Wikieł Z 1.62 b) moja wersja nr 797}

Rozwiązać nierówności $(x+18)(16-x)(x+12)\ge0$.
\zadStop
\rozwStart{Patryk Wirkus}{}
Miejsca zerowe naszego wielomianu to: $-18, 16, -12$.\\
Wielomian jest stopnia nieparzystego, ponadto znak współczynnika przy\linebreak najwyższej potędze x jest ujemny.\\ W związku z tym wykres wielomianu zaczyna się od lewej strony powyżej osi OX. A więc $$x \in (-\infty,-18) \cup (-12,16).$$
\rozwStop
\odpStart
$x \in (-\infty,-18) \cup (-12,16)$
\odpStop
\testStart
A.$x \in (-\infty,-18) \cup (-12,16)$\\
B.$x \in (-\infty,-18) \cup (-12,16]$\\
C.$x \in (-\infty,-18) \cup [-12,16)$\\
D.$x \in (-\infty,-18] \cup (-12,16)$\\
E.$x \in (-\infty,-18] \cup (-12,16]$\\
F.$x \in (-\infty,-18] \cup [-12,16)$\\
G.$x \in (-\infty,-18) \cup [-12,16]$\\
H.$x \in (-\infty,-18] \cup [-12,16]$
\testStop
\kluczStart
A
\kluczStop



\zadStart{Zadanie z Wikieł Z 1.62 b) moja wersja nr 798}

Rozwiązać nierówności $(x+18)(16-x)(x+13)\ge0$.
\zadStop
\rozwStart{Patryk Wirkus}{}
Miejsca zerowe naszego wielomianu to: $-18, 16, -13$.\\
Wielomian jest stopnia nieparzystego, ponadto znak współczynnika przy\linebreak najwyższej potędze x jest ujemny.\\ W związku z tym wykres wielomianu zaczyna się od lewej strony powyżej osi OX. A więc $$x \in (-\infty,-18) \cup (-13,16).$$
\rozwStop
\odpStart
$x \in (-\infty,-18) \cup (-13,16)$
\odpStop
\testStart
A.$x \in (-\infty,-18) \cup (-13,16)$\\
B.$x \in (-\infty,-18) \cup (-13,16]$\\
C.$x \in (-\infty,-18) \cup [-13,16)$\\
D.$x \in (-\infty,-18] \cup (-13,16)$\\
E.$x \in (-\infty,-18] \cup (-13,16]$\\
F.$x \in (-\infty,-18] \cup [-13,16)$\\
G.$x \in (-\infty,-18) \cup [-13,16]$\\
H.$x \in (-\infty,-18] \cup [-13,16]$
\testStop
\kluczStart
A
\kluczStop



\zadStart{Zadanie z Wikieł Z 1.62 b) moja wersja nr 799}

Rozwiązać nierówności $(x+18)(16-x)(x+14)\ge0$.
\zadStop
\rozwStart{Patryk Wirkus}{}
Miejsca zerowe naszego wielomianu to: $-18, 16, -14$.\\
Wielomian jest stopnia nieparzystego, ponadto znak współczynnika przy\linebreak najwyższej potędze x jest ujemny.\\ W związku z tym wykres wielomianu zaczyna się od lewej strony powyżej osi OX. A więc $$x \in (-\infty,-18) \cup (-14,16).$$
\rozwStop
\odpStart
$x \in (-\infty,-18) \cup (-14,16)$
\odpStop
\testStart
A.$x \in (-\infty,-18) \cup (-14,16)$\\
B.$x \in (-\infty,-18) \cup (-14,16]$\\
C.$x \in (-\infty,-18) \cup [-14,16)$\\
D.$x \in (-\infty,-18] \cup (-14,16)$\\
E.$x \in (-\infty,-18] \cup (-14,16]$\\
F.$x \in (-\infty,-18] \cup [-14,16)$\\
G.$x \in (-\infty,-18) \cup [-14,16]$\\
H.$x \in (-\infty,-18] \cup [-14,16]$
\testStop
\kluczStart
A
\kluczStop



\zadStart{Zadanie z Wikieł Z 1.62 b) moja wersja nr 800}

Rozwiązać nierówności $(x+18)(16-x)(x+15)\ge0$.
\zadStop
\rozwStart{Patryk Wirkus}{}
Miejsca zerowe naszego wielomianu to: $-18, 16, -15$.\\
Wielomian jest stopnia nieparzystego, ponadto znak współczynnika przy\linebreak najwyższej potędze x jest ujemny.\\ W związku z tym wykres wielomianu zaczyna się od lewej strony powyżej osi OX. A więc $$x \in (-\infty,-18) \cup (-15,16).$$
\rozwStop
\odpStart
$x \in (-\infty,-18) \cup (-15,16)$
\odpStop
\testStart
A.$x \in (-\infty,-18) \cup (-15,16)$\\
B.$x \in (-\infty,-18) \cup (-15,16]$\\
C.$x \in (-\infty,-18) \cup [-15,16)$\\
D.$x \in (-\infty,-18] \cup (-15,16)$\\
E.$x \in (-\infty,-18] \cup (-15,16]$\\
F.$x \in (-\infty,-18] \cup [-15,16)$\\
G.$x \in (-\infty,-18) \cup [-15,16]$\\
H.$x \in (-\infty,-18] \cup [-15,16]$
\testStop
\kluczStart
A
\kluczStop



\zadStart{Zadanie z Wikieł Z 1.62 b) moja wersja nr 801}

Rozwiązać nierówności $(x+18)(17-x)(x+1)\ge0$.
\zadStop
\rozwStart{Patryk Wirkus}{}
Miejsca zerowe naszego wielomianu to: $-18, 17, -1$.\\
Wielomian jest stopnia nieparzystego, ponadto znak współczynnika przy\linebreak najwyższej potędze x jest ujemny.\\ W związku z tym wykres wielomianu zaczyna się od lewej strony powyżej osi OX. A więc $$x \in (-\infty,-18) \cup (-1,17).$$
\rozwStop
\odpStart
$x \in (-\infty,-18) \cup (-1,17)$
\odpStop
\testStart
A.$x \in (-\infty,-18) \cup (-1,17)$\\
B.$x \in (-\infty,-18) \cup (-1,17]$\\
C.$x \in (-\infty,-18) \cup [-1,17)$\\
D.$x \in (-\infty,-18] \cup (-1,17)$\\
E.$x \in (-\infty,-18] \cup (-1,17]$\\
F.$x \in (-\infty,-18] \cup [-1,17)$\\
G.$x \in (-\infty,-18) \cup [-1,17]$\\
H.$x \in (-\infty,-18] \cup [-1,17]$
\testStop
\kluczStart
A
\kluczStop



\zadStart{Zadanie z Wikieł Z 1.62 b) moja wersja nr 802}

Rozwiązać nierówności $(x+18)(17-x)(x+2)\ge0$.
\zadStop
\rozwStart{Patryk Wirkus}{}
Miejsca zerowe naszego wielomianu to: $-18, 17, -2$.\\
Wielomian jest stopnia nieparzystego, ponadto znak współczynnika przy\linebreak najwyższej potędze x jest ujemny.\\ W związku z tym wykres wielomianu zaczyna się od lewej strony powyżej osi OX. A więc $$x \in (-\infty,-18) \cup (-2,17).$$
\rozwStop
\odpStart
$x \in (-\infty,-18) \cup (-2,17)$
\odpStop
\testStart
A.$x \in (-\infty,-18) \cup (-2,17)$\\
B.$x \in (-\infty,-18) \cup (-2,17]$\\
C.$x \in (-\infty,-18) \cup [-2,17)$\\
D.$x \in (-\infty,-18] \cup (-2,17)$\\
E.$x \in (-\infty,-18] \cup (-2,17]$\\
F.$x \in (-\infty,-18] \cup [-2,17)$\\
G.$x \in (-\infty,-18) \cup [-2,17]$\\
H.$x \in (-\infty,-18] \cup [-2,17]$
\testStop
\kluczStart
A
\kluczStop



\zadStart{Zadanie z Wikieł Z 1.62 b) moja wersja nr 803}

Rozwiązać nierówności $(x+18)(17-x)(x+3)\ge0$.
\zadStop
\rozwStart{Patryk Wirkus}{}
Miejsca zerowe naszego wielomianu to: $-18, 17, -3$.\\
Wielomian jest stopnia nieparzystego, ponadto znak współczynnika przy\linebreak najwyższej potędze x jest ujemny.\\ W związku z tym wykres wielomianu zaczyna się od lewej strony powyżej osi OX. A więc $$x \in (-\infty,-18) \cup (-3,17).$$
\rozwStop
\odpStart
$x \in (-\infty,-18) \cup (-3,17)$
\odpStop
\testStart
A.$x \in (-\infty,-18) \cup (-3,17)$\\
B.$x \in (-\infty,-18) \cup (-3,17]$\\
C.$x \in (-\infty,-18) \cup [-3,17)$\\
D.$x \in (-\infty,-18] \cup (-3,17)$\\
E.$x \in (-\infty,-18] \cup (-3,17]$\\
F.$x \in (-\infty,-18] \cup [-3,17)$\\
G.$x \in (-\infty,-18) \cup [-3,17]$\\
H.$x \in (-\infty,-18] \cup [-3,17]$
\testStop
\kluczStart
A
\kluczStop



\zadStart{Zadanie z Wikieł Z 1.62 b) moja wersja nr 804}

Rozwiązać nierówności $(x+18)(17-x)(x+4)\ge0$.
\zadStop
\rozwStart{Patryk Wirkus}{}
Miejsca zerowe naszego wielomianu to: $-18, 17, -4$.\\
Wielomian jest stopnia nieparzystego, ponadto znak współczynnika przy\linebreak najwyższej potędze x jest ujemny.\\ W związku z tym wykres wielomianu zaczyna się od lewej strony powyżej osi OX. A więc $$x \in (-\infty,-18) \cup (-4,17).$$
\rozwStop
\odpStart
$x \in (-\infty,-18) \cup (-4,17)$
\odpStop
\testStart
A.$x \in (-\infty,-18) \cup (-4,17)$\\
B.$x \in (-\infty,-18) \cup (-4,17]$\\
C.$x \in (-\infty,-18) \cup [-4,17)$\\
D.$x \in (-\infty,-18] \cup (-4,17)$\\
E.$x \in (-\infty,-18] \cup (-4,17]$\\
F.$x \in (-\infty,-18] \cup [-4,17)$\\
G.$x \in (-\infty,-18) \cup [-4,17]$\\
H.$x \in (-\infty,-18] \cup [-4,17]$
\testStop
\kluczStart
A
\kluczStop



\zadStart{Zadanie z Wikieł Z 1.62 b) moja wersja nr 805}

Rozwiązać nierówności $(x+18)(17-x)(x+5)\ge0$.
\zadStop
\rozwStart{Patryk Wirkus}{}
Miejsca zerowe naszego wielomianu to: $-18, 17, -5$.\\
Wielomian jest stopnia nieparzystego, ponadto znak współczynnika przy\linebreak najwyższej potędze x jest ujemny.\\ W związku z tym wykres wielomianu zaczyna się od lewej strony powyżej osi OX. A więc $$x \in (-\infty,-18) \cup (-5,17).$$
\rozwStop
\odpStart
$x \in (-\infty,-18) \cup (-5,17)$
\odpStop
\testStart
A.$x \in (-\infty,-18) \cup (-5,17)$\\
B.$x \in (-\infty,-18) \cup (-5,17]$\\
C.$x \in (-\infty,-18) \cup [-5,17)$\\
D.$x \in (-\infty,-18] \cup (-5,17)$\\
E.$x \in (-\infty,-18] \cup (-5,17]$\\
F.$x \in (-\infty,-18] \cup [-5,17)$\\
G.$x \in (-\infty,-18) \cup [-5,17]$\\
H.$x \in (-\infty,-18] \cup [-5,17]$
\testStop
\kluczStart
A
\kluczStop



\zadStart{Zadanie z Wikieł Z 1.62 b) moja wersja nr 806}

Rozwiązać nierówności $(x+18)(17-x)(x+6)\ge0$.
\zadStop
\rozwStart{Patryk Wirkus}{}
Miejsca zerowe naszego wielomianu to: $-18, 17, -6$.\\
Wielomian jest stopnia nieparzystego, ponadto znak współczynnika przy\linebreak najwyższej potędze x jest ujemny.\\ W związku z tym wykres wielomianu zaczyna się od lewej strony powyżej osi OX. A więc $$x \in (-\infty,-18) \cup (-6,17).$$
\rozwStop
\odpStart
$x \in (-\infty,-18) \cup (-6,17)$
\odpStop
\testStart
A.$x \in (-\infty,-18) \cup (-6,17)$\\
B.$x \in (-\infty,-18) \cup (-6,17]$\\
C.$x \in (-\infty,-18) \cup [-6,17)$\\
D.$x \in (-\infty,-18] \cup (-6,17)$\\
E.$x \in (-\infty,-18] \cup (-6,17]$\\
F.$x \in (-\infty,-18] \cup [-6,17)$\\
G.$x \in (-\infty,-18) \cup [-6,17]$\\
H.$x \in (-\infty,-18] \cup [-6,17]$
\testStop
\kluczStart
A
\kluczStop



\zadStart{Zadanie z Wikieł Z 1.62 b) moja wersja nr 807}

Rozwiązać nierówności $(x+18)(17-x)(x+7)\ge0$.
\zadStop
\rozwStart{Patryk Wirkus}{}
Miejsca zerowe naszego wielomianu to: $-18, 17, -7$.\\
Wielomian jest stopnia nieparzystego, ponadto znak współczynnika przy\linebreak najwyższej potędze x jest ujemny.\\ W związku z tym wykres wielomianu zaczyna się od lewej strony powyżej osi OX. A więc $$x \in (-\infty,-18) \cup (-7,17).$$
\rozwStop
\odpStart
$x \in (-\infty,-18) \cup (-7,17)$
\odpStop
\testStart
A.$x \in (-\infty,-18) \cup (-7,17)$\\
B.$x \in (-\infty,-18) \cup (-7,17]$\\
C.$x \in (-\infty,-18) \cup [-7,17)$\\
D.$x \in (-\infty,-18] \cup (-7,17)$\\
E.$x \in (-\infty,-18] \cup (-7,17]$\\
F.$x \in (-\infty,-18] \cup [-7,17)$\\
G.$x \in (-\infty,-18) \cup [-7,17]$\\
H.$x \in (-\infty,-18] \cup [-7,17]$
\testStop
\kluczStart
A
\kluczStop



\zadStart{Zadanie z Wikieł Z 1.62 b) moja wersja nr 808}

Rozwiązać nierówności $(x+18)(17-x)(x+8)\ge0$.
\zadStop
\rozwStart{Patryk Wirkus}{}
Miejsca zerowe naszego wielomianu to: $-18, 17, -8$.\\
Wielomian jest stopnia nieparzystego, ponadto znak współczynnika przy\linebreak najwyższej potędze x jest ujemny.\\ W związku z tym wykres wielomianu zaczyna się od lewej strony powyżej osi OX. A więc $$x \in (-\infty,-18) \cup (-8,17).$$
\rozwStop
\odpStart
$x \in (-\infty,-18) \cup (-8,17)$
\odpStop
\testStart
A.$x \in (-\infty,-18) \cup (-8,17)$\\
B.$x \in (-\infty,-18) \cup (-8,17]$\\
C.$x \in (-\infty,-18) \cup [-8,17)$\\
D.$x \in (-\infty,-18] \cup (-8,17)$\\
E.$x \in (-\infty,-18] \cup (-8,17]$\\
F.$x \in (-\infty,-18] \cup [-8,17)$\\
G.$x \in (-\infty,-18) \cup [-8,17]$\\
H.$x \in (-\infty,-18] \cup [-8,17]$
\testStop
\kluczStart
A
\kluczStop



\zadStart{Zadanie z Wikieł Z 1.62 b) moja wersja nr 809}

Rozwiązać nierówności $(x+18)(17-x)(x+9)\ge0$.
\zadStop
\rozwStart{Patryk Wirkus}{}
Miejsca zerowe naszego wielomianu to: $-18, 17, -9$.\\
Wielomian jest stopnia nieparzystego, ponadto znak współczynnika przy\linebreak najwyższej potędze x jest ujemny.\\ W związku z tym wykres wielomianu zaczyna się od lewej strony powyżej osi OX. A więc $$x \in (-\infty,-18) \cup (-9,17).$$
\rozwStop
\odpStart
$x \in (-\infty,-18) \cup (-9,17)$
\odpStop
\testStart
A.$x \in (-\infty,-18) \cup (-9,17)$\\
B.$x \in (-\infty,-18) \cup (-9,17]$\\
C.$x \in (-\infty,-18) \cup [-9,17)$\\
D.$x \in (-\infty,-18] \cup (-9,17)$\\
E.$x \in (-\infty,-18] \cup (-9,17]$\\
F.$x \in (-\infty,-18] \cup [-9,17)$\\
G.$x \in (-\infty,-18) \cup [-9,17]$\\
H.$x \in (-\infty,-18] \cup [-9,17]$
\testStop
\kluczStart
A
\kluczStop



\zadStart{Zadanie z Wikieł Z 1.62 b) moja wersja nr 810}

Rozwiązać nierówności $(x+18)(17-x)(x+10)\ge0$.
\zadStop
\rozwStart{Patryk Wirkus}{}
Miejsca zerowe naszego wielomianu to: $-18, 17, -10$.\\
Wielomian jest stopnia nieparzystego, ponadto znak współczynnika przy\linebreak najwyższej potędze x jest ujemny.\\ W związku z tym wykres wielomianu zaczyna się od lewej strony powyżej osi OX. A więc $$x \in (-\infty,-18) \cup (-10,17).$$
\rozwStop
\odpStart
$x \in (-\infty,-18) \cup (-10,17)$
\odpStop
\testStart
A.$x \in (-\infty,-18) \cup (-10,17)$\\
B.$x \in (-\infty,-18) \cup (-10,17]$\\
C.$x \in (-\infty,-18) \cup [-10,17)$\\
D.$x \in (-\infty,-18] \cup (-10,17)$\\
E.$x \in (-\infty,-18] \cup (-10,17]$\\
F.$x \in (-\infty,-18] \cup [-10,17)$\\
G.$x \in (-\infty,-18) \cup [-10,17]$\\
H.$x \in (-\infty,-18] \cup [-10,17]$
\testStop
\kluczStart
A
\kluczStop



\zadStart{Zadanie z Wikieł Z 1.62 b) moja wersja nr 811}

Rozwiązać nierówności $(x+18)(17-x)(x+11)\ge0$.
\zadStop
\rozwStart{Patryk Wirkus}{}
Miejsca zerowe naszego wielomianu to: $-18, 17, -11$.\\
Wielomian jest stopnia nieparzystego, ponadto znak współczynnika przy\linebreak najwyższej potędze x jest ujemny.\\ W związku z tym wykres wielomianu zaczyna się od lewej strony powyżej osi OX. A więc $$x \in (-\infty,-18) \cup (-11,17).$$
\rozwStop
\odpStart
$x \in (-\infty,-18) \cup (-11,17)$
\odpStop
\testStart
A.$x \in (-\infty,-18) \cup (-11,17)$\\
B.$x \in (-\infty,-18) \cup (-11,17]$\\
C.$x \in (-\infty,-18) \cup [-11,17)$\\
D.$x \in (-\infty,-18] \cup (-11,17)$\\
E.$x \in (-\infty,-18] \cup (-11,17]$\\
F.$x \in (-\infty,-18] \cup [-11,17)$\\
G.$x \in (-\infty,-18) \cup [-11,17]$\\
H.$x \in (-\infty,-18] \cup [-11,17]$
\testStop
\kluczStart
A
\kluczStop



\zadStart{Zadanie z Wikieł Z 1.62 b) moja wersja nr 812}

Rozwiązać nierówności $(x+18)(17-x)(x+12)\ge0$.
\zadStop
\rozwStart{Patryk Wirkus}{}
Miejsca zerowe naszego wielomianu to: $-18, 17, -12$.\\
Wielomian jest stopnia nieparzystego, ponadto znak współczynnika przy\linebreak najwyższej potędze x jest ujemny.\\ W związku z tym wykres wielomianu zaczyna się od lewej strony powyżej osi OX. A więc $$x \in (-\infty,-18) \cup (-12,17).$$
\rozwStop
\odpStart
$x \in (-\infty,-18) \cup (-12,17)$
\odpStop
\testStart
A.$x \in (-\infty,-18) \cup (-12,17)$\\
B.$x \in (-\infty,-18) \cup (-12,17]$\\
C.$x \in (-\infty,-18) \cup [-12,17)$\\
D.$x \in (-\infty,-18] \cup (-12,17)$\\
E.$x \in (-\infty,-18] \cup (-12,17]$\\
F.$x \in (-\infty,-18] \cup [-12,17)$\\
G.$x \in (-\infty,-18) \cup [-12,17]$\\
H.$x \in (-\infty,-18] \cup [-12,17]$
\testStop
\kluczStart
A
\kluczStop



\zadStart{Zadanie z Wikieł Z 1.62 b) moja wersja nr 813}

Rozwiązać nierówności $(x+18)(17-x)(x+13)\ge0$.
\zadStop
\rozwStart{Patryk Wirkus}{}
Miejsca zerowe naszego wielomianu to: $-18, 17, -13$.\\
Wielomian jest stopnia nieparzystego, ponadto znak współczynnika przy\linebreak najwyższej potędze x jest ujemny.\\ W związku z tym wykres wielomianu zaczyna się od lewej strony powyżej osi OX. A więc $$x \in (-\infty,-18) \cup (-13,17).$$
\rozwStop
\odpStart
$x \in (-\infty,-18) \cup (-13,17)$
\odpStop
\testStart
A.$x \in (-\infty,-18) \cup (-13,17)$\\
B.$x \in (-\infty,-18) \cup (-13,17]$\\
C.$x \in (-\infty,-18) \cup [-13,17)$\\
D.$x \in (-\infty,-18] \cup (-13,17)$\\
E.$x \in (-\infty,-18] \cup (-13,17]$\\
F.$x \in (-\infty,-18] \cup [-13,17)$\\
G.$x \in (-\infty,-18) \cup [-13,17]$\\
H.$x \in (-\infty,-18] \cup [-13,17]$
\testStop
\kluczStart
A
\kluczStop



\zadStart{Zadanie z Wikieł Z 1.62 b) moja wersja nr 814}

Rozwiązać nierówności $(x+18)(17-x)(x+14)\ge0$.
\zadStop
\rozwStart{Patryk Wirkus}{}
Miejsca zerowe naszego wielomianu to: $-18, 17, -14$.\\
Wielomian jest stopnia nieparzystego, ponadto znak współczynnika przy\linebreak najwyższej potędze x jest ujemny.\\ W związku z tym wykres wielomianu zaczyna się od lewej strony powyżej osi OX. A więc $$x \in (-\infty,-18) \cup (-14,17).$$
\rozwStop
\odpStart
$x \in (-\infty,-18) \cup (-14,17)$
\odpStop
\testStart
A.$x \in (-\infty,-18) \cup (-14,17)$\\
B.$x \in (-\infty,-18) \cup (-14,17]$\\
C.$x \in (-\infty,-18) \cup [-14,17)$\\
D.$x \in (-\infty,-18] \cup (-14,17)$\\
E.$x \in (-\infty,-18] \cup (-14,17]$\\
F.$x \in (-\infty,-18] \cup [-14,17)$\\
G.$x \in (-\infty,-18) \cup [-14,17]$\\
H.$x \in (-\infty,-18] \cup [-14,17]$
\testStop
\kluczStart
A
\kluczStop



\zadStart{Zadanie z Wikieł Z 1.62 b) moja wersja nr 815}

Rozwiązać nierówności $(x+18)(17-x)(x+15)\ge0$.
\zadStop
\rozwStart{Patryk Wirkus}{}
Miejsca zerowe naszego wielomianu to: $-18, 17, -15$.\\
Wielomian jest stopnia nieparzystego, ponadto znak współczynnika przy\linebreak najwyższej potędze x jest ujemny.\\ W związku z tym wykres wielomianu zaczyna się od lewej strony powyżej osi OX. A więc $$x \in (-\infty,-18) \cup (-15,17).$$
\rozwStop
\odpStart
$x \in (-\infty,-18) \cup (-15,17)$
\odpStop
\testStart
A.$x \in (-\infty,-18) \cup (-15,17)$\\
B.$x \in (-\infty,-18) \cup (-15,17]$\\
C.$x \in (-\infty,-18) \cup [-15,17)$\\
D.$x \in (-\infty,-18] \cup (-15,17)$\\
E.$x \in (-\infty,-18] \cup (-15,17]$\\
F.$x \in (-\infty,-18] \cup [-15,17)$\\
G.$x \in (-\infty,-18) \cup [-15,17]$\\
H.$x \in (-\infty,-18] \cup [-15,17]$
\testStop
\kluczStart
A
\kluczStop



\zadStart{Zadanie z Wikieł Z 1.62 b) moja wersja nr 816}

Rozwiązać nierówności $(x+18)(17-x)(x+16)\ge0$.
\zadStop
\rozwStart{Patryk Wirkus}{}
Miejsca zerowe naszego wielomianu to: $-18, 17, -16$.\\
Wielomian jest stopnia nieparzystego, ponadto znak współczynnika przy\linebreak najwyższej potędze x jest ujemny.\\ W związku z tym wykres wielomianu zaczyna się od lewej strony powyżej osi OX. A więc $$x \in (-\infty,-18) \cup (-16,17).$$
\rozwStop
\odpStart
$x \in (-\infty,-18) \cup (-16,17)$
\odpStop
\testStart
A.$x \in (-\infty,-18) \cup (-16,17)$\\
B.$x \in (-\infty,-18) \cup (-16,17]$\\
C.$x \in (-\infty,-18) \cup [-16,17)$\\
D.$x \in (-\infty,-18] \cup (-16,17)$\\
E.$x \in (-\infty,-18] \cup (-16,17]$\\
F.$x \in (-\infty,-18] \cup [-16,17)$\\
G.$x \in (-\infty,-18) \cup [-16,17]$\\
H.$x \in (-\infty,-18] \cup [-16,17]$
\testStop
\kluczStart
A
\kluczStop



\zadStart{Zadanie z Wikieł Z 1.62 b) moja wersja nr 817}

Rozwiązać nierówności $(x+19)(2-x)(x+1)\ge0$.
\zadStop
\rozwStart{Patryk Wirkus}{}
Miejsca zerowe naszego wielomianu to: $-19, 2, -1$.\\
Wielomian jest stopnia nieparzystego, ponadto znak współczynnika przy\linebreak najwyższej potędze x jest ujemny.\\ W związku z tym wykres wielomianu zaczyna się od lewej strony powyżej osi OX. A więc $$x \in (-\infty,-19) \cup (-1,2).$$
\rozwStop
\odpStart
$x \in (-\infty,-19) \cup (-1,2)$
\odpStop
\testStart
A.$x \in (-\infty,-19) \cup (-1,2)$\\
B.$x \in (-\infty,-19) \cup (-1,2]$\\
C.$x \in (-\infty,-19) \cup [-1,2)$\\
D.$x \in (-\infty,-19] \cup (-1,2)$\\
E.$x \in (-\infty,-19] \cup (-1,2]$\\
F.$x \in (-\infty,-19] \cup [-1,2)$\\
G.$x \in (-\infty,-19) \cup [-1,2]$\\
H.$x \in (-\infty,-19] \cup [-1,2]$
\testStop
\kluczStart
A
\kluczStop



\zadStart{Zadanie z Wikieł Z 1.62 b) moja wersja nr 818}

Rozwiązać nierówności $(x+19)(3-x)(x+1)\ge0$.
\zadStop
\rozwStart{Patryk Wirkus}{}
Miejsca zerowe naszego wielomianu to: $-19, 3, -1$.\\
Wielomian jest stopnia nieparzystego, ponadto znak współczynnika przy\linebreak najwyższej potędze x jest ujemny.\\ W związku z tym wykres wielomianu zaczyna się od lewej strony powyżej osi OX. A więc $$x \in (-\infty,-19) \cup (-1,3).$$
\rozwStop
\odpStart
$x \in (-\infty,-19) \cup (-1,3)$
\odpStop
\testStart
A.$x \in (-\infty,-19) \cup (-1,3)$\\
B.$x \in (-\infty,-19) \cup (-1,3]$\\
C.$x \in (-\infty,-19) \cup [-1,3)$\\
D.$x \in (-\infty,-19] \cup (-1,3)$\\
E.$x \in (-\infty,-19] \cup (-1,3]$\\
F.$x \in (-\infty,-19] \cup [-1,3)$\\
G.$x \in (-\infty,-19) \cup [-1,3]$\\
H.$x \in (-\infty,-19] \cup [-1,3]$
\testStop
\kluczStart
A
\kluczStop



\zadStart{Zadanie z Wikieł Z 1.62 b) moja wersja nr 819}

Rozwiązać nierówności $(x+19)(3-x)(x+2)\ge0$.
\zadStop
\rozwStart{Patryk Wirkus}{}
Miejsca zerowe naszego wielomianu to: $-19, 3, -2$.\\
Wielomian jest stopnia nieparzystego, ponadto znak współczynnika przy\linebreak najwyższej potędze x jest ujemny.\\ W związku z tym wykres wielomianu zaczyna się od lewej strony powyżej osi OX. A więc $$x \in (-\infty,-19) \cup (-2,3).$$
\rozwStop
\odpStart
$x \in (-\infty,-19) \cup (-2,3)$
\odpStop
\testStart
A.$x \in (-\infty,-19) \cup (-2,3)$\\
B.$x \in (-\infty,-19) \cup (-2,3]$\\
C.$x \in (-\infty,-19) \cup [-2,3)$\\
D.$x \in (-\infty,-19] \cup (-2,3)$\\
E.$x \in (-\infty,-19] \cup (-2,3]$\\
F.$x \in (-\infty,-19] \cup [-2,3)$\\
G.$x \in (-\infty,-19) \cup [-2,3]$\\
H.$x \in (-\infty,-19] \cup [-2,3]$
\testStop
\kluczStart
A
\kluczStop



\zadStart{Zadanie z Wikieł Z 1.62 b) moja wersja nr 820}

Rozwiązać nierówności $(x+19)(4-x)(x+1)\ge0$.
\zadStop
\rozwStart{Patryk Wirkus}{}
Miejsca zerowe naszego wielomianu to: $-19, 4, -1$.\\
Wielomian jest stopnia nieparzystego, ponadto znak współczynnika przy\linebreak najwyższej potędze x jest ujemny.\\ W związku z tym wykres wielomianu zaczyna się od lewej strony powyżej osi OX. A więc $$x \in (-\infty,-19) \cup (-1,4).$$
\rozwStop
\odpStart
$x \in (-\infty,-19) \cup (-1,4)$
\odpStop
\testStart
A.$x \in (-\infty,-19) \cup (-1,4)$\\
B.$x \in (-\infty,-19) \cup (-1,4]$\\
C.$x \in (-\infty,-19) \cup [-1,4)$\\
D.$x \in (-\infty,-19] \cup (-1,4)$\\
E.$x \in (-\infty,-19] \cup (-1,4]$\\
F.$x \in (-\infty,-19] \cup [-1,4)$\\
G.$x \in (-\infty,-19) \cup [-1,4]$\\
H.$x \in (-\infty,-19] \cup [-1,4]$
\testStop
\kluczStart
A
\kluczStop



\zadStart{Zadanie z Wikieł Z 1.62 b) moja wersja nr 821}

Rozwiązać nierówności $(x+19)(4-x)(x+2)\ge0$.
\zadStop
\rozwStart{Patryk Wirkus}{}
Miejsca zerowe naszego wielomianu to: $-19, 4, -2$.\\
Wielomian jest stopnia nieparzystego, ponadto znak współczynnika przy\linebreak najwyższej potędze x jest ujemny.\\ W związku z tym wykres wielomianu zaczyna się od lewej strony powyżej osi OX. A więc $$x \in (-\infty,-19) \cup (-2,4).$$
\rozwStop
\odpStart
$x \in (-\infty,-19) \cup (-2,4)$
\odpStop
\testStart
A.$x \in (-\infty,-19) \cup (-2,4)$\\
B.$x \in (-\infty,-19) \cup (-2,4]$\\
C.$x \in (-\infty,-19) \cup [-2,4)$\\
D.$x \in (-\infty,-19] \cup (-2,4)$\\
E.$x \in (-\infty,-19] \cup (-2,4]$\\
F.$x \in (-\infty,-19] \cup [-2,4)$\\
G.$x \in (-\infty,-19) \cup [-2,4]$\\
H.$x \in (-\infty,-19] \cup [-2,4]$
\testStop
\kluczStart
A
\kluczStop



\zadStart{Zadanie z Wikieł Z 1.62 b) moja wersja nr 822}

Rozwiązać nierówności $(x+19)(4-x)(x+3)\ge0$.
\zadStop
\rozwStart{Patryk Wirkus}{}
Miejsca zerowe naszego wielomianu to: $-19, 4, -3$.\\
Wielomian jest stopnia nieparzystego, ponadto znak współczynnika przy\linebreak najwyższej potędze x jest ujemny.\\ W związku z tym wykres wielomianu zaczyna się od lewej strony powyżej osi OX. A więc $$x \in (-\infty,-19) \cup (-3,4).$$
\rozwStop
\odpStart
$x \in (-\infty,-19) \cup (-3,4)$
\odpStop
\testStart
A.$x \in (-\infty,-19) \cup (-3,4)$\\
B.$x \in (-\infty,-19) \cup (-3,4]$\\
C.$x \in (-\infty,-19) \cup [-3,4)$\\
D.$x \in (-\infty,-19] \cup (-3,4)$\\
E.$x \in (-\infty,-19] \cup (-3,4]$\\
F.$x \in (-\infty,-19] \cup [-3,4)$\\
G.$x \in (-\infty,-19) \cup [-3,4]$\\
H.$x \in (-\infty,-19] \cup [-3,4]$
\testStop
\kluczStart
A
\kluczStop



\zadStart{Zadanie z Wikieł Z 1.62 b) moja wersja nr 823}

Rozwiązać nierówności $(x+19)(5-x)(x+1)\ge0$.
\zadStop
\rozwStart{Patryk Wirkus}{}
Miejsca zerowe naszego wielomianu to: $-19, 5, -1$.\\
Wielomian jest stopnia nieparzystego, ponadto znak współczynnika przy\linebreak najwyższej potędze x jest ujemny.\\ W związku z tym wykres wielomianu zaczyna się od lewej strony powyżej osi OX. A więc $$x \in (-\infty,-19) \cup (-1,5).$$
\rozwStop
\odpStart
$x \in (-\infty,-19) \cup (-1,5)$
\odpStop
\testStart
A.$x \in (-\infty,-19) \cup (-1,5)$\\
B.$x \in (-\infty,-19) \cup (-1,5]$\\
C.$x \in (-\infty,-19) \cup [-1,5)$\\
D.$x \in (-\infty,-19] \cup (-1,5)$\\
E.$x \in (-\infty,-19] \cup (-1,5]$\\
F.$x \in (-\infty,-19] \cup [-1,5)$\\
G.$x \in (-\infty,-19) \cup [-1,5]$\\
H.$x \in (-\infty,-19] \cup [-1,5]$
\testStop
\kluczStart
A
\kluczStop



\zadStart{Zadanie z Wikieł Z 1.62 b) moja wersja nr 824}

Rozwiązać nierówności $(x+19)(5-x)(x+2)\ge0$.
\zadStop
\rozwStart{Patryk Wirkus}{}
Miejsca zerowe naszego wielomianu to: $-19, 5, -2$.\\
Wielomian jest stopnia nieparzystego, ponadto znak współczynnika przy\linebreak najwyższej potędze x jest ujemny.\\ W związku z tym wykres wielomianu zaczyna się od lewej strony powyżej osi OX. A więc $$x \in (-\infty,-19) \cup (-2,5).$$
\rozwStop
\odpStart
$x \in (-\infty,-19) \cup (-2,5)$
\odpStop
\testStart
A.$x \in (-\infty,-19) \cup (-2,5)$\\
B.$x \in (-\infty,-19) \cup (-2,5]$\\
C.$x \in (-\infty,-19) \cup [-2,5)$\\
D.$x \in (-\infty,-19] \cup (-2,5)$\\
E.$x \in (-\infty,-19] \cup (-2,5]$\\
F.$x \in (-\infty,-19] \cup [-2,5)$\\
G.$x \in (-\infty,-19) \cup [-2,5]$\\
H.$x \in (-\infty,-19] \cup [-2,5]$
\testStop
\kluczStart
A
\kluczStop



\zadStart{Zadanie z Wikieł Z 1.62 b) moja wersja nr 825}

Rozwiązać nierówności $(x+19)(5-x)(x+3)\ge0$.
\zadStop
\rozwStart{Patryk Wirkus}{}
Miejsca zerowe naszego wielomianu to: $-19, 5, -3$.\\
Wielomian jest stopnia nieparzystego, ponadto znak współczynnika przy\linebreak najwyższej potędze x jest ujemny.\\ W związku z tym wykres wielomianu zaczyna się od lewej strony powyżej osi OX. A więc $$x \in (-\infty,-19) \cup (-3,5).$$
\rozwStop
\odpStart
$x \in (-\infty,-19) \cup (-3,5)$
\odpStop
\testStart
A.$x \in (-\infty,-19) \cup (-3,5)$\\
B.$x \in (-\infty,-19) \cup (-3,5]$\\
C.$x \in (-\infty,-19) \cup [-3,5)$\\
D.$x \in (-\infty,-19] \cup (-3,5)$\\
E.$x \in (-\infty,-19] \cup (-3,5]$\\
F.$x \in (-\infty,-19] \cup [-3,5)$\\
G.$x \in (-\infty,-19) \cup [-3,5]$\\
H.$x \in (-\infty,-19] \cup [-3,5]$
\testStop
\kluczStart
A
\kluczStop



\zadStart{Zadanie z Wikieł Z 1.62 b) moja wersja nr 826}

Rozwiązać nierówności $(x+19)(5-x)(x+4)\ge0$.
\zadStop
\rozwStart{Patryk Wirkus}{}
Miejsca zerowe naszego wielomianu to: $-19, 5, -4$.\\
Wielomian jest stopnia nieparzystego, ponadto znak współczynnika przy\linebreak najwyższej potędze x jest ujemny.\\ W związku z tym wykres wielomianu zaczyna się od lewej strony powyżej osi OX. A więc $$x \in (-\infty,-19) \cup (-4,5).$$
\rozwStop
\odpStart
$x \in (-\infty,-19) \cup (-4,5)$
\odpStop
\testStart
A.$x \in (-\infty,-19) \cup (-4,5)$\\
B.$x \in (-\infty,-19) \cup (-4,5]$\\
C.$x \in (-\infty,-19) \cup [-4,5)$\\
D.$x \in (-\infty,-19] \cup (-4,5)$\\
E.$x \in (-\infty,-19] \cup (-4,5]$\\
F.$x \in (-\infty,-19] \cup [-4,5)$\\
G.$x \in (-\infty,-19) \cup [-4,5]$\\
H.$x \in (-\infty,-19] \cup [-4,5]$
\testStop
\kluczStart
A
\kluczStop



\zadStart{Zadanie z Wikieł Z 1.62 b) moja wersja nr 827}

Rozwiązać nierówności $(x+19)(6-x)(x+1)\ge0$.
\zadStop
\rozwStart{Patryk Wirkus}{}
Miejsca zerowe naszego wielomianu to: $-19, 6, -1$.\\
Wielomian jest stopnia nieparzystego, ponadto znak współczynnika przy\linebreak najwyższej potędze x jest ujemny.\\ W związku z tym wykres wielomianu zaczyna się od lewej strony powyżej osi OX. A więc $$x \in (-\infty,-19) \cup (-1,6).$$
\rozwStop
\odpStart
$x \in (-\infty,-19) \cup (-1,6)$
\odpStop
\testStart
A.$x \in (-\infty,-19) \cup (-1,6)$\\
B.$x \in (-\infty,-19) \cup (-1,6]$\\
C.$x \in (-\infty,-19) \cup [-1,6)$\\
D.$x \in (-\infty,-19] \cup (-1,6)$\\
E.$x \in (-\infty,-19] \cup (-1,6]$\\
F.$x \in (-\infty,-19] \cup [-1,6)$\\
G.$x \in (-\infty,-19) \cup [-1,6]$\\
H.$x \in (-\infty,-19] \cup [-1,6]$
\testStop
\kluczStart
A
\kluczStop



\zadStart{Zadanie z Wikieł Z 1.62 b) moja wersja nr 828}

Rozwiązać nierówności $(x+19)(6-x)(x+2)\ge0$.
\zadStop
\rozwStart{Patryk Wirkus}{}
Miejsca zerowe naszego wielomianu to: $-19, 6, -2$.\\
Wielomian jest stopnia nieparzystego, ponadto znak współczynnika przy\linebreak najwyższej potędze x jest ujemny.\\ W związku z tym wykres wielomianu zaczyna się od lewej strony powyżej osi OX. A więc $$x \in (-\infty,-19) \cup (-2,6).$$
\rozwStop
\odpStart
$x \in (-\infty,-19) \cup (-2,6)$
\odpStop
\testStart
A.$x \in (-\infty,-19) \cup (-2,6)$\\
B.$x \in (-\infty,-19) \cup (-2,6]$\\
C.$x \in (-\infty,-19) \cup [-2,6)$\\
D.$x \in (-\infty,-19] \cup (-2,6)$\\
E.$x \in (-\infty,-19] \cup (-2,6]$\\
F.$x \in (-\infty,-19] \cup [-2,6)$\\
G.$x \in (-\infty,-19) \cup [-2,6]$\\
H.$x \in (-\infty,-19] \cup [-2,6]$
\testStop
\kluczStart
A
\kluczStop



\zadStart{Zadanie z Wikieł Z 1.62 b) moja wersja nr 829}

Rozwiązać nierówności $(x+19)(6-x)(x+3)\ge0$.
\zadStop
\rozwStart{Patryk Wirkus}{}
Miejsca zerowe naszego wielomianu to: $-19, 6, -3$.\\
Wielomian jest stopnia nieparzystego, ponadto znak współczynnika przy\linebreak najwyższej potędze x jest ujemny.\\ W związku z tym wykres wielomianu zaczyna się od lewej strony powyżej osi OX. A więc $$x \in (-\infty,-19) \cup (-3,6).$$
\rozwStop
\odpStart
$x \in (-\infty,-19) \cup (-3,6)$
\odpStop
\testStart
A.$x \in (-\infty,-19) \cup (-3,6)$\\
B.$x \in (-\infty,-19) \cup (-3,6]$\\
C.$x \in (-\infty,-19) \cup [-3,6)$\\
D.$x \in (-\infty,-19] \cup (-3,6)$\\
E.$x \in (-\infty,-19] \cup (-3,6]$\\
F.$x \in (-\infty,-19] \cup [-3,6)$\\
G.$x \in (-\infty,-19) \cup [-3,6]$\\
H.$x \in (-\infty,-19] \cup [-3,6]$
\testStop
\kluczStart
A
\kluczStop



\zadStart{Zadanie z Wikieł Z 1.62 b) moja wersja nr 830}

Rozwiązać nierówności $(x+19)(6-x)(x+4)\ge0$.
\zadStop
\rozwStart{Patryk Wirkus}{}
Miejsca zerowe naszego wielomianu to: $-19, 6, -4$.\\
Wielomian jest stopnia nieparzystego, ponadto znak współczynnika przy\linebreak najwyższej potędze x jest ujemny.\\ W związku z tym wykres wielomianu zaczyna się od lewej strony powyżej osi OX. A więc $$x \in (-\infty,-19) \cup (-4,6).$$
\rozwStop
\odpStart
$x \in (-\infty,-19) \cup (-4,6)$
\odpStop
\testStart
A.$x \in (-\infty,-19) \cup (-4,6)$\\
B.$x \in (-\infty,-19) \cup (-4,6]$\\
C.$x \in (-\infty,-19) \cup [-4,6)$\\
D.$x \in (-\infty,-19] \cup (-4,6)$\\
E.$x \in (-\infty,-19] \cup (-4,6]$\\
F.$x \in (-\infty,-19] \cup [-4,6)$\\
G.$x \in (-\infty,-19) \cup [-4,6]$\\
H.$x \in (-\infty,-19] \cup [-4,6]$
\testStop
\kluczStart
A
\kluczStop



\zadStart{Zadanie z Wikieł Z 1.62 b) moja wersja nr 831}

Rozwiązać nierówności $(x+19)(6-x)(x+5)\ge0$.
\zadStop
\rozwStart{Patryk Wirkus}{}
Miejsca zerowe naszego wielomianu to: $-19, 6, -5$.\\
Wielomian jest stopnia nieparzystego, ponadto znak współczynnika przy\linebreak najwyższej potędze x jest ujemny.\\ W związku z tym wykres wielomianu zaczyna się od lewej strony powyżej osi OX. A więc $$x \in (-\infty,-19) \cup (-5,6).$$
\rozwStop
\odpStart
$x \in (-\infty,-19) \cup (-5,6)$
\odpStop
\testStart
A.$x \in (-\infty,-19) \cup (-5,6)$\\
B.$x \in (-\infty,-19) \cup (-5,6]$\\
C.$x \in (-\infty,-19) \cup [-5,6)$\\
D.$x \in (-\infty,-19] \cup (-5,6)$\\
E.$x \in (-\infty,-19] \cup (-5,6]$\\
F.$x \in (-\infty,-19] \cup [-5,6)$\\
G.$x \in (-\infty,-19) \cup [-5,6]$\\
H.$x \in (-\infty,-19] \cup [-5,6]$
\testStop
\kluczStart
A
\kluczStop



\zadStart{Zadanie z Wikieł Z 1.62 b) moja wersja nr 832}

Rozwiązać nierówności $(x+19)(7-x)(x+1)\ge0$.
\zadStop
\rozwStart{Patryk Wirkus}{}
Miejsca zerowe naszego wielomianu to: $-19, 7, -1$.\\
Wielomian jest stopnia nieparzystego, ponadto znak współczynnika przy\linebreak najwyższej potędze x jest ujemny.\\ W związku z tym wykres wielomianu zaczyna się od lewej strony powyżej osi OX. A więc $$x \in (-\infty,-19) \cup (-1,7).$$
\rozwStop
\odpStart
$x \in (-\infty,-19) \cup (-1,7)$
\odpStop
\testStart
A.$x \in (-\infty,-19) \cup (-1,7)$\\
B.$x \in (-\infty,-19) \cup (-1,7]$\\
C.$x \in (-\infty,-19) \cup [-1,7)$\\
D.$x \in (-\infty,-19] \cup (-1,7)$\\
E.$x \in (-\infty,-19] \cup (-1,7]$\\
F.$x \in (-\infty,-19] \cup [-1,7)$\\
G.$x \in (-\infty,-19) \cup [-1,7]$\\
H.$x \in (-\infty,-19] \cup [-1,7]$
\testStop
\kluczStart
A
\kluczStop



\zadStart{Zadanie z Wikieł Z 1.62 b) moja wersja nr 833}

Rozwiązać nierówności $(x+19)(7-x)(x+2)\ge0$.
\zadStop
\rozwStart{Patryk Wirkus}{}
Miejsca zerowe naszego wielomianu to: $-19, 7, -2$.\\
Wielomian jest stopnia nieparzystego, ponadto znak współczynnika przy\linebreak najwyższej potędze x jest ujemny.\\ W związku z tym wykres wielomianu zaczyna się od lewej strony powyżej osi OX. A więc $$x \in (-\infty,-19) \cup (-2,7).$$
\rozwStop
\odpStart
$x \in (-\infty,-19) \cup (-2,7)$
\odpStop
\testStart
A.$x \in (-\infty,-19) \cup (-2,7)$\\
B.$x \in (-\infty,-19) \cup (-2,7]$\\
C.$x \in (-\infty,-19) \cup [-2,7)$\\
D.$x \in (-\infty,-19] \cup (-2,7)$\\
E.$x \in (-\infty,-19] \cup (-2,7]$\\
F.$x \in (-\infty,-19] \cup [-2,7)$\\
G.$x \in (-\infty,-19) \cup [-2,7]$\\
H.$x \in (-\infty,-19] \cup [-2,7]$
\testStop
\kluczStart
A
\kluczStop



\zadStart{Zadanie z Wikieł Z 1.62 b) moja wersja nr 834}

Rozwiązać nierówności $(x+19)(7-x)(x+3)\ge0$.
\zadStop
\rozwStart{Patryk Wirkus}{}
Miejsca zerowe naszego wielomianu to: $-19, 7, -3$.\\
Wielomian jest stopnia nieparzystego, ponadto znak współczynnika przy\linebreak najwyższej potędze x jest ujemny.\\ W związku z tym wykres wielomianu zaczyna się od lewej strony powyżej osi OX. A więc $$x \in (-\infty,-19) \cup (-3,7).$$
\rozwStop
\odpStart
$x \in (-\infty,-19) \cup (-3,7)$
\odpStop
\testStart
A.$x \in (-\infty,-19) \cup (-3,7)$\\
B.$x \in (-\infty,-19) \cup (-3,7]$\\
C.$x \in (-\infty,-19) \cup [-3,7)$\\
D.$x \in (-\infty,-19] \cup (-3,7)$\\
E.$x \in (-\infty,-19] \cup (-3,7]$\\
F.$x \in (-\infty,-19] \cup [-3,7)$\\
G.$x \in (-\infty,-19) \cup [-3,7]$\\
H.$x \in (-\infty,-19] \cup [-3,7]$
\testStop
\kluczStart
A
\kluczStop



\zadStart{Zadanie z Wikieł Z 1.62 b) moja wersja nr 835}

Rozwiązać nierówności $(x+19)(7-x)(x+4)\ge0$.
\zadStop
\rozwStart{Patryk Wirkus}{}
Miejsca zerowe naszego wielomianu to: $-19, 7, -4$.\\
Wielomian jest stopnia nieparzystego, ponadto znak współczynnika przy\linebreak najwyższej potędze x jest ujemny.\\ W związku z tym wykres wielomianu zaczyna się od lewej strony powyżej osi OX. A więc $$x \in (-\infty,-19) \cup (-4,7).$$
\rozwStop
\odpStart
$x \in (-\infty,-19) \cup (-4,7)$
\odpStop
\testStart
A.$x \in (-\infty,-19) \cup (-4,7)$\\
B.$x \in (-\infty,-19) \cup (-4,7]$\\
C.$x \in (-\infty,-19) \cup [-4,7)$\\
D.$x \in (-\infty,-19] \cup (-4,7)$\\
E.$x \in (-\infty,-19] \cup (-4,7]$\\
F.$x \in (-\infty,-19] \cup [-4,7)$\\
G.$x \in (-\infty,-19) \cup [-4,7]$\\
H.$x \in (-\infty,-19] \cup [-4,7]$
\testStop
\kluczStart
A
\kluczStop



\zadStart{Zadanie z Wikieł Z 1.62 b) moja wersja nr 836}

Rozwiązać nierówności $(x+19)(7-x)(x+5)\ge0$.
\zadStop
\rozwStart{Patryk Wirkus}{}
Miejsca zerowe naszego wielomianu to: $-19, 7, -5$.\\
Wielomian jest stopnia nieparzystego, ponadto znak współczynnika przy\linebreak najwyższej potędze x jest ujemny.\\ W związku z tym wykres wielomianu zaczyna się od lewej strony powyżej osi OX. A więc $$x \in (-\infty,-19) \cup (-5,7).$$
\rozwStop
\odpStart
$x \in (-\infty,-19) \cup (-5,7)$
\odpStop
\testStart
A.$x \in (-\infty,-19) \cup (-5,7)$\\
B.$x \in (-\infty,-19) \cup (-5,7]$\\
C.$x \in (-\infty,-19) \cup [-5,7)$\\
D.$x \in (-\infty,-19] \cup (-5,7)$\\
E.$x \in (-\infty,-19] \cup (-5,7]$\\
F.$x \in (-\infty,-19] \cup [-5,7)$\\
G.$x \in (-\infty,-19) \cup [-5,7]$\\
H.$x \in (-\infty,-19] \cup [-5,7]$
\testStop
\kluczStart
A
\kluczStop



\zadStart{Zadanie z Wikieł Z 1.62 b) moja wersja nr 837}

Rozwiązać nierówności $(x+19)(7-x)(x+6)\ge0$.
\zadStop
\rozwStart{Patryk Wirkus}{}
Miejsca zerowe naszego wielomianu to: $-19, 7, -6$.\\
Wielomian jest stopnia nieparzystego, ponadto znak współczynnika przy\linebreak najwyższej potędze x jest ujemny.\\ W związku z tym wykres wielomianu zaczyna się od lewej strony powyżej osi OX. A więc $$x \in (-\infty,-19) \cup (-6,7).$$
\rozwStop
\odpStart
$x \in (-\infty,-19) \cup (-6,7)$
\odpStop
\testStart
A.$x \in (-\infty,-19) \cup (-6,7)$\\
B.$x \in (-\infty,-19) \cup (-6,7]$\\
C.$x \in (-\infty,-19) \cup [-6,7)$\\
D.$x \in (-\infty,-19] \cup (-6,7)$\\
E.$x \in (-\infty,-19] \cup (-6,7]$\\
F.$x \in (-\infty,-19] \cup [-6,7)$\\
G.$x \in (-\infty,-19) \cup [-6,7]$\\
H.$x \in (-\infty,-19] \cup [-6,7]$
\testStop
\kluczStart
A
\kluczStop



\zadStart{Zadanie z Wikieł Z 1.62 b) moja wersja nr 838}

Rozwiązać nierówności $(x+19)(8-x)(x+1)\ge0$.
\zadStop
\rozwStart{Patryk Wirkus}{}
Miejsca zerowe naszego wielomianu to: $-19, 8, -1$.\\
Wielomian jest stopnia nieparzystego, ponadto znak współczynnika przy\linebreak najwyższej potędze x jest ujemny.\\ W związku z tym wykres wielomianu zaczyna się od lewej strony powyżej osi OX. A więc $$x \in (-\infty,-19) \cup (-1,8).$$
\rozwStop
\odpStart
$x \in (-\infty,-19) \cup (-1,8)$
\odpStop
\testStart
A.$x \in (-\infty,-19) \cup (-1,8)$\\
B.$x \in (-\infty,-19) \cup (-1,8]$\\
C.$x \in (-\infty,-19) \cup [-1,8)$\\
D.$x \in (-\infty,-19] \cup (-1,8)$\\
E.$x \in (-\infty,-19] \cup (-1,8]$\\
F.$x \in (-\infty,-19] \cup [-1,8)$\\
G.$x \in (-\infty,-19) \cup [-1,8]$\\
H.$x \in (-\infty,-19] \cup [-1,8]$
\testStop
\kluczStart
A
\kluczStop



\zadStart{Zadanie z Wikieł Z 1.62 b) moja wersja nr 839}

Rozwiązać nierówności $(x+19)(8-x)(x+2)\ge0$.
\zadStop
\rozwStart{Patryk Wirkus}{}
Miejsca zerowe naszego wielomianu to: $-19, 8, -2$.\\
Wielomian jest stopnia nieparzystego, ponadto znak współczynnika przy\linebreak najwyższej potędze x jest ujemny.\\ W związku z tym wykres wielomianu zaczyna się od lewej strony powyżej osi OX. A więc $$x \in (-\infty,-19) \cup (-2,8).$$
\rozwStop
\odpStart
$x \in (-\infty,-19) \cup (-2,8)$
\odpStop
\testStart
A.$x \in (-\infty,-19) \cup (-2,8)$\\
B.$x \in (-\infty,-19) \cup (-2,8]$\\
C.$x \in (-\infty,-19) \cup [-2,8)$\\
D.$x \in (-\infty,-19] \cup (-2,8)$\\
E.$x \in (-\infty,-19] \cup (-2,8]$\\
F.$x \in (-\infty,-19] \cup [-2,8)$\\
G.$x \in (-\infty,-19) \cup [-2,8]$\\
H.$x \in (-\infty,-19] \cup [-2,8]$
\testStop
\kluczStart
A
\kluczStop



\zadStart{Zadanie z Wikieł Z 1.62 b) moja wersja nr 840}

Rozwiązać nierówności $(x+19)(8-x)(x+3)\ge0$.
\zadStop
\rozwStart{Patryk Wirkus}{}
Miejsca zerowe naszego wielomianu to: $-19, 8, -3$.\\
Wielomian jest stopnia nieparzystego, ponadto znak współczynnika przy\linebreak najwyższej potędze x jest ujemny.\\ W związku z tym wykres wielomianu zaczyna się od lewej strony powyżej osi OX. A więc $$x \in (-\infty,-19) \cup (-3,8).$$
\rozwStop
\odpStart
$x \in (-\infty,-19) \cup (-3,8)$
\odpStop
\testStart
A.$x \in (-\infty,-19) \cup (-3,8)$\\
B.$x \in (-\infty,-19) \cup (-3,8]$\\
C.$x \in (-\infty,-19) \cup [-3,8)$\\
D.$x \in (-\infty,-19] \cup (-3,8)$\\
E.$x \in (-\infty,-19] \cup (-3,8]$\\
F.$x \in (-\infty,-19] \cup [-3,8)$\\
G.$x \in (-\infty,-19) \cup [-3,8]$\\
H.$x \in (-\infty,-19] \cup [-3,8]$
\testStop
\kluczStart
A
\kluczStop



\zadStart{Zadanie z Wikieł Z 1.62 b) moja wersja nr 841}

Rozwiązać nierówności $(x+19)(8-x)(x+4)\ge0$.
\zadStop
\rozwStart{Patryk Wirkus}{}
Miejsca zerowe naszego wielomianu to: $-19, 8, -4$.\\
Wielomian jest stopnia nieparzystego, ponadto znak współczynnika przy\linebreak najwyższej potędze x jest ujemny.\\ W związku z tym wykres wielomianu zaczyna się od lewej strony powyżej osi OX. A więc $$x \in (-\infty,-19) \cup (-4,8).$$
\rozwStop
\odpStart
$x \in (-\infty,-19) \cup (-4,8)$
\odpStop
\testStart
A.$x \in (-\infty,-19) \cup (-4,8)$\\
B.$x \in (-\infty,-19) \cup (-4,8]$\\
C.$x \in (-\infty,-19) \cup [-4,8)$\\
D.$x \in (-\infty,-19] \cup (-4,8)$\\
E.$x \in (-\infty,-19] \cup (-4,8]$\\
F.$x \in (-\infty,-19] \cup [-4,8)$\\
G.$x \in (-\infty,-19) \cup [-4,8]$\\
H.$x \in (-\infty,-19] \cup [-4,8]$
\testStop
\kluczStart
A
\kluczStop



\zadStart{Zadanie z Wikieł Z 1.62 b) moja wersja nr 842}

Rozwiązać nierówności $(x+19)(8-x)(x+5)\ge0$.
\zadStop
\rozwStart{Patryk Wirkus}{}
Miejsca zerowe naszego wielomianu to: $-19, 8, -5$.\\
Wielomian jest stopnia nieparzystego, ponadto znak współczynnika przy\linebreak najwyższej potędze x jest ujemny.\\ W związku z tym wykres wielomianu zaczyna się od lewej strony powyżej osi OX. A więc $$x \in (-\infty,-19) \cup (-5,8).$$
\rozwStop
\odpStart
$x \in (-\infty,-19) \cup (-5,8)$
\odpStop
\testStart
A.$x \in (-\infty,-19) \cup (-5,8)$\\
B.$x \in (-\infty,-19) \cup (-5,8]$\\
C.$x \in (-\infty,-19) \cup [-5,8)$\\
D.$x \in (-\infty,-19] \cup (-5,8)$\\
E.$x \in (-\infty,-19] \cup (-5,8]$\\
F.$x \in (-\infty,-19] \cup [-5,8)$\\
G.$x \in (-\infty,-19) \cup [-5,8]$\\
H.$x \in (-\infty,-19] \cup [-5,8]$
\testStop
\kluczStart
A
\kluczStop



\zadStart{Zadanie z Wikieł Z 1.62 b) moja wersja nr 843}

Rozwiązać nierówności $(x+19)(8-x)(x+6)\ge0$.
\zadStop
\rozwStart{Patryk Wirkus}{}
Miejsca zerowe naszego wielomianu to: $-19, 8, -6$.\\
Wielomian jest stopnia nieparzystego, ponadto znak współczynnika przy\linebreak najwyższej potędze x jest ujemny.\\ W związku z tym wykres wielomianu zaczyna się od lewej strony powyżej osi OX. A więc $$x \in (-\infty,-19) \cup (-6,8).$$
\rozwStop
\odpStart
$x \in (-\infty,-19) \cup (-6,8)$
\odpStop
\testStart
A.$x \in (-\infty,-19) \cup (-6,8)$\\
B.$x \in (-\infty,-19) \cup (-6,8]$\\
C.$x \in (-\infty,-19) \cup [-6,8)$\\
D.$x \in (-\infty,-19] \cup (-6,8)$\\
E.$x \in (-\infty,-19] \cup (-6,8]$\\
F.$x \in (-\infty,-19] \cup [-6,8)$\\
G.$x \in (-\infty,-19) \cup [-6,8]$\\
H.$x \in (-\infty,-19] \cup [-6,8]$
\testStop
\kluczStart
A
\kluczStop



\zadStart{Zadanie z Wikieł Z 1.62 b) moja wersja nr 844}

Rozwiązać nierówności $(x+19)(8-x)(x+7)\ge0$.
\zadStop
\rozwStart{Patryk Wirkus}{}
Miejsca zerowe naszego wielomianu to: $-19, 8, -7$.\\
Wielomian jest stopnia nieparzystego, ponadto znak współczynnika przy\linebreak najwyższej potędze x jest ujemny.\\ W związku z tym wykres wielomianu zaczyna się od lewej strony powyżej osi OX. A więc $$x \in (-\infty,-19) \cup (-7,8).$$
\rozwStop
\odpStart
$x \in (-\infty,-19) \cup (-7,8)$
\odpStop
\testStart
A.$x \in (-\infty,-19) \cup (-7,8)$\\
B.$x \in (-\infty,-19) \cup (-7,8]$\\
C.$x \in (-\infty,-19) \cup [-7,8)$\\
D.$x \in (-\infty,-19] \cup (-7,8)$\\
E.$x \in (-\infty,-19] \cup (-7,8]$\\
F.$x \in (-\infty,-19] \cup [-7,8)$\\
G.$x \in (-\infty,-19) \cup [-7,8]$\\
H.$x \in (-\infty,-19] \cup [-7,8]$
\testStop
\kluczStart
A
\kluczStop



\zadStart{Zadanie z Wikieł Z 1.62 b) moja wersja nr 845}

Rozwiązać nierówności $(x+19)(9-x)(x+1)\ge0$.
\zadStop
\rozwStart{Patryk Wirkus}{}
Miejsca zerowe naszego wielomianu to: $-19, 9, -1$.\\
Wielomian jest stopnia nieparzystego, ponadto znak współczynnika przy\linebreak najwyższej potędze x jest ujemny.\\ W związku z tym wykres wielomianu zaczyna się od lewej strony powyżej osi OX. A więc $$x \in (-\infty,-19) \cup (-1,9).$$
\rozwStop
\odpStart
$x \in (-\infty,-19) \cup (-1,9)$
\odpStop
\testStart
A.$x \in (-\infty,-19) \cup (-1,9)$\\
B.$x \in (-\infty,-19) \cup (-1,9]$\\
C.$x \in (-\infty,-19) \cup [-1,9)$\\
D.$x \in (-\infty,-19] \cup (-1,9)$\\
E.$x \in (-\infty,-19] \cup (-1,9]$\\
F.$x \in (-\infty,-19] \cup [-1,9)$\\
G.$x \in (-\infty,-19) \cup [-1,9]$\\
H.$x \in (-\infty,-19] \cup [-1,9]$
\testStop
\kluczStart
A
\kluczStop



\zadStart{Zadanie z Wikieł Z 1.62 b) moja wersja nr 846}

Rozwiązać nierówności $(x+19)(9-x)(x+2)\ge0$.
\zadStop
\rozwStart{Patryk Wirkus}{}
Miejsca zerowe naszego wielomianu to: $-19, 9, -2$.\\
Wielomian jest stopnia nieparzystego, ponadto znak współczynnika przy\linebreak najwyższej potędze x jest ujemny.\\ W związku z tym wykres wielomianu zaczyna się od lewej strony powyżej osi OX. A więc $$x \in (-\infty,-19) \cup (-2,9).$$
\rozwStop
\odpStart
$x \in (-\infty,-19) \cup (-2,9)$
\odpStop
\testStart
A.$x \in (-\infty,-19) \cup (-2,9)$\\
B.$x \in (-\infty,-19) \cup (-2,9]$\\
C.$x \in (-\infty,-19) \cup [-2,9)$\\
D.$x \in (-\infty,-19] \cup (-2,9)$\\
E.$x \in (-\infty,-19] \cup (-2,9]$\\
F.$x \in (-\infty,-19] \cup [-2,9)$\\
G.$x \in (-\infty,-19) \cup [-2,9]$\\
H.$x \in (-\infty,-19] \cup [-2,9]$
\testStop
\kluczStart
A
\kluczStop



\zadStart{Zadanie z Wikieł Z 1.62 b) moja wersja nr 847}

Rozwiązać nierówności $(x+19)(9-x)(x+3)\ge0$.
\zadStop
\rozwStart{Patryk Wirkus}{}
Miejsca zerowe naszego wielomianu to: $-19, 9, -3$.\\
Wielomian jest stopnia nieparzystego, ponadto znak współczynnika przy\linebreak najwyższej potędze x jest ujemny.\\ W związku z tym wykres wielomianu zaczyna się od lewej strony powyżej osi OX. A więc $$x \in (-\infty,-19) \cup (-3,9).$$
\rozwStop
\odpStart
$x \in (-\infty,-19) \cup (-3,9)$
\odpStop
\testStart
A.$x \in (-\infty,-19) \cup (-3,9)$\\
B.$x \in (-\infty,-19) \cup (-3,9]$\\
C.$x \in (-\infty,-19) \cup [-3,9)$\\
D.$x \in (-\infty,-19] \cup (-3,9)$\\
E.$x \in (-\infty,-19] \cup (-3,9]$\\
F.$x \in (-\infty,-19] \cup [-3,9)$\\
G.$x \in (-\infty,-19) \cup [-3,9]$\\
H.$x \in (-\infty,-19] \cup [-3,9]$
\testStop
\kluczStart
A
\kluczStop



\zadStart{Zadanie z Wikieł Z 1.62 b) moja wersja nr 848}

Rozwiązać nierówności $(x+19)(9-x)(x+4)\ge0$.
\zadStop
\rozwStart{Patryk Wirkus}{}
Miejsca zerowe naszego wielomianu to: $-19, 9, -4$.\\
Wielomian jest stopnia nieparzystego, ponadto znak współczynnika przy\linebreak najwyższej potędze x jest ujemny.\\ W związku z tym wykres wielomianu zaczyna się od lewej strony powyżej osi OX. A więc $$x \in (-\infty,-19) \cup (-4,9).$$
\rozwStop
\odpStart
$x \in (-\infty,-19) \cup (-4,9)$
\odpStop
\testStart
A.$x \in (-\infty,-19) \cup (-4,9)$\\
B.$x \in (-\infty,-19) \cup (-4,9]$\\
C.$x \in (-\infty,-19) \cup [-4,9)$\\
D.$x \in (-\infty,-19] \cup (-4,9)$\\
E.$x \in (-\infty,-19] \cup (-4,9]$\\
F.$x \in (-\infty,-19] \cup [-4,9)$\\
G.$x \in (-\infty,-19) \cup [-4,9]$\\
H.$x \in (-\infty,-19] \cup [-4,9]$
\testStop
\kluczStart
A
\kluczStop



\zadStart{Zadanie z Wikieł Z 1.62 b) moja wersja nr 849}

Rozwiązać nierówności $(x+19)(9-x)(x+5)\ge0$.
\zadStop
\rozwStart{Patryk Wirkus}{}
Miejsca zerowe naszego wielomianu to: $-19, 9, -5$.\\
Wielomian jest stopnia nieparzystego, ponadto znak współczynnika przy\linebreak najwyższej potędze x jest ujemny.\\ W związku z tym wykres wielomianu zaczyna się od lewej strony powyżej osi OX. A więc $$x \in (-\infty,-19) \cup (-5,9).$$
\rozwStop
\odpStart
$x \in (-\infty,-19) \cup (-5,9)$
\odpStop
\testStart
A.$x \in (-\infty,-19) \cup (-5,9)$\\
B.$x \in (-\infty,-19) \cup (-5,9]$\\
C.$x \in (-\infty,-19) \cup [-5,9)$\\
D.$x \in (-\infty,-19] \cup (-5,9)$\\
E.$x \in (-\infty,-19] \cup (-5,9]$\\
F.$x \in (-\infty,-19] \cup [-5,9)$\\
G.$x \in (-\infty,-19) \cup [-5,9]$\\
H.$x \in (-\infty,-19] \cup [-5,9]$
\testStop
\kluczStart
A
\kluczStop



\zadStart{Zadanie z Wikieł Z 1.62 b) moja wersja nr 850}

Rozwiązać nierówności $(x+19)(9-x)(x+6)\ge0$.
\zadStop
\rozwStart{Patryk Wirkus}{}
Miejsca zerowe naszego wielomianu to: $-19, 9, -6$.\\
Wielomian jest stopnia nieparzystego, ponadto znak współczynnika przy\linebreak najwyższej potędze x jest ujemny.\\ W związku z tym wykres wielomianu zaczyna się od lewej strony powyżej osi OX. A więc $$x \in (-\infty,-19) \cup (-6,9).$$
\rozwStop
\odpStart
$x \in (-\infty,-19) \cup (-6,9)$
\odpStop
\testStart
A.$x \in (-\infty,-19) \cup (-6,9)$\\
B.$x \in (-\infty,-19) \cup (-6,9]$\\
C.$x \in (-\infty,-19) \cup [-6,9)$\\
D.$x \in (-\infty,-19] \cup (-6,9)$\\
E.$x \in (-\infty,-19] \cup (-6,9]$\\
F.$x \in (-\infty,-19] \cup [-6,9)$\\
G.$x \in (-\infty,-19) \cup [-6,9]$\\
H.$x \in (-\infty,-19] \cup [-6,9]$
\testStop
\kluczStart
A
\kluczStop



\zadStart{Zadanie z Wikieł Z 1.62 b) moja wersja nr 851}

Rozwiązać nierówności $(x+19)(9-x)(x+7)\ge0$.
\zadStop
\rozwStart{Patryk Wirkus}{}
Miejsca zerowe naszego wielomianu to: $-19, 9, -7$.\\
Wielomian jest stopnia nieparzystego, ponadto znak współczynnika przy\linebreak najwyższej potędze x jest ujemny.\\ W związku z tym wykres wielomianu zaczyna się od lewej strony powyżej osi OX. A więc $$x \in (-\infty,-19) \cup (-7,9).$$
\rozwStop
\odpStart
$x \in (-\infty,-19) \cup (-7,9)$
\odpStop
\testStart
A.$x \in (-\infty,-19) \cup (-7,9)$\\
B.$x \in (-\infty,-19) \cup (-7,9]$\\
C.$x \in (-\infty,-19) \cup [-7,9)$\\
D.$x \in (-\infty,-19] \cup (-7,9)$\\
E.$x \in (-\infty,-19] \cup (-7,9]$\\
F.$x \in (-\infty,-19] \cup [-7,9)$\\
G.$x \in (-\infty,-19) \cup [-7,9]$\\
H.$x \in (-\infty,-19] \cup [-7,9]$
\testStop
\kluczStart
A
\kluczStop



\zadStart{Zadanie z Wikieł Z 1.62 b) moja wersja nr 852}

Rozwiązać nierówności $(x+19)(9-x)(x+8)\ge0$.
\zadStop
\rozwStart{Patryk Wirkus}{}
Miejsca zerowe naszego wielomianu to: $-19, 9, -8$.\\
Wielomian jest stopnia nieparzystego, ponadto znak współczynnika przy\linebreak najwyższej potędze x jest ujemny.\\ W związku z tym wykres wielomianu zaczyna się od lewej strony powyżej osi OX. A więc $$x \in (-\infty,-19) \cup (-8,9).$$
\rozwStop
\odpStart
$x \in (-\infty,-19) \cup (-8,9)$
\odpStop
\testStart
A.$x \in (-\infty,-19) \cup (-8,9)$\\
B.$x \in (-\infty,-19) \cup (-8,9]$\\
C.$x \in (-\infty,-19) \cup [-8,9)$\\
D.$x \in (-\infty,-19] \cup (-8,9)$\\
E.$x \in (-\infty,-19] \cup (-8,9]$\\
F.$x \in (-\infty,-19] \cup [-8,9)$\\
G.$x \in (-\infty,-19) \cup [-8,9]$\\
H.$x \in (-\infty,-19] \cup [-8,9]$
\testStop
\kluczStart
A
\kluczStop



\zadStart{Zadanie z Wikieł Z 1.62 b) moja wersja nr 853}

Rozwiązać nierówności $(x+19)(10-x)(x+1)\ge0$.
\zadStop
\rozwStart{Patryk Wirkus}{}
Miejsca zerowe naszego wielomianu to: $-19, 10, -1$.\\
Wielomian jest stopnia nieparzystego, ponadto znak współczynnika przy\linebreak najwyższej potędze x jest ujemny.\\ W związku z tym wykres wielomianu zaczyna się od lewej strony powyżej osi OX. A więc $$x \in (-\infty,-19) \cup (-1,10).$$
\rozwStop
\odpStart
$x \in (-\infty,-19) \cup (-1,10)$
\odpStop
\testStart
A.$x \in (-\infty,-19) \cup (-1,10)$\\
B.$x \in (-\infty,-19) \cup (-1,10]$\\
C.$x \in (-\infty,-19) \cup [-1,10)$\\
D.$x \in (-\infty,-19] \cup (-1,10)$\\
E.$x \in (-\infty,-19] \cup (-1,10]$\\
F.$x \in (-\infty,-19] \cup [-1,10)$\\
G.$x \in (-\infty,-19) \cup [-1,10]$\\
H.$x \in (-\infty,-19] \cup [-1,10]$
\testStop
\kluczStart
A
\kluczStop



\zadStart{Zadanie z Wikieł Z 1.62 b) moja wersja nr 854}

Rozwiązać nierówności $(x+19)(10-x)(x+2)\ge0$.
\zadStop
\rozwStart{Patryk Wirkus}{}
Miejsca zerowe naszego wielomianu to: $-19, 10, -2$.\\
Wielomian jest stopnia nieparzystego, ponadto znak współczynnika przy\linebreak najwyższej potędze x jest ujemny.\\ W związku z tym wykres wielomianu zaczyna się od lewej strony powyżej osi OX. A więc $$x \in (-\infty,-19) \cup (-2,10).$$
\rozwStop
\odpStart
$x \in (-\infty,-19) \cup (-2,10)$
\odpStop
\testStart
A.$x \in (-\infty,-19) \cup (-2,10)$\\
B.$x \in (-\infty,-19) \cup (-2,10]$\\
C.$x \in (-\infty,-19) \cup [-2,10)$\\
D.$x \in (-\infty,-19] \cup (-2,10)$\\
E.$x \in (-\infty,-19] \cup (-2,10]$\\
F.$x \in (-\infty,-19] \cup [-2,10)$\\
G.$x \in (-\infty,-19) \cup [-2,10]$\\
H.$x \in (-\infty,-19] \cup [-2,10]$
\testStop
\kluczStart
A
\kluczStop



\zadStart{Zadanie z Wikieł Z 1.62 b) moja wersja nr 855}

Rozwiązać nierówności $(x+19)(10-x)(x+3)\ge0$.
\zadStop
\rozwStart{Patryk Wirkus}{}
Miejsca zerowe naszego wielomianu to: $-19, 10, -3$.\\
Wielomian jest stopnia nieparzystego, ponadto znak współczynnika przy\linebreak najwyższej potędze x jest ujemny.\\ W związku z tym wykres wielomianu zaczyna się od lewej strony powyżej osi OX. A więc $$x \in (-\infty,-19) \cup (-3,10).$$
\rozwStop
\odpStart
$x \in (-\infty,-19) \cup (-3,10)$
\odpStop
\testStart
A.$x \in (-\infty,-19) \cup (-3,10)$\\
B.$x \in (-\infty,-19) \cup (-3,10]$\\
C.$x \in (-\infty,-19) \cup [-3,10)$\\
D.$x \in (-\infty,-19] \cup (-3,10)$\\
E.$x \in (-\infty,-19] \cup (-3,10]$\\
F.$x \in (-\infty,-19] \cup [-3,10)$\\
G.$x \in (-\infty,-19) \cup [-3,10]$\\
H.$x \in (-\infty,-19] \cup [-3,10]$
\testStop
\kluczStart
A
\kluczStop



\zadStart{Zadanie z Wikieł Z 1.62 b) moja wersja nr 856}

Rozwiązać nierówności $(x+19)(10-x)(x+4)\ge0$.
\zadStop
\rozwStart{Patryk Wirkus}{}
Miejsca zerowe naszego wielomianu to: $-19, 10, -4$.\\
Wielomian jest stopnia nieparzystego, ponadto znak współczynnika przy\linebreak najwyższej potędze x jest ujemny.\\ W związku z tym wykres wielomianu zaczyna się od lewej strony powyżej osi OX. A więc $$x \in (-\infty,-19) \cup (-4,10).$$
\rozwStop
\odpStart
$x \in (-\infty,-19) \cup (-4,10)$
\odpStop
\testStart
A.$x \in (-\infty,-19) \cup (-4,10)$\\
B.$x \in (-\infty,-19) \cup (-4,10]$\\
C.$x \in (-\infty,-19) \cup [-4,10)$\\
D.$x \in (-\infty,-19] \cup (-4,10)$\\
E.$x \in (-\infty,-19] \cup (-4,10]$\\
F.$x \in (-\infty,-19] \cup [-4,10)$\\
G.$x \in (-\infty,-19) \cup [-4,10]$\\
H.$x \in (-\infty,-19] \cup [-4,10]$
\testStop
\kluczStart
A
\kluczStop



\zadStart{Zadanie z Wikieł Z 1.62 b) moja wersja nr 857}

Rozwiązać nierówności $(x+19)(10-x)(x+5)\ge0$.
\zadStop
\rozwStart{Patryk Wirkus}{}
Miejsca zerowe naszego wielomianu to: $-19, 10, -5$.\\
Wielomian jest stopnia nieparzystego, ponadto znak współczynnika przy\linebreak najwyższej potędze x jest ujemny.\\ W związku z tym wykres wielomianu zaczyna się od lewej strony powyżej osi OX. A więc $$x \in (-\infty,-19) \cup (-5,10).$$
\rozwStop
\odpStart
$x \in (-\infty,-19) \cup (-5,10)$
\odpStop
\testStart
A.$x \in (-\infty,-19) \cup (-5,10)$\\
B.$x \in (-\infty,-19) \cup (-5,10]$\\
C.$x \in (-\infty,-19) \cup [-5,10)$\\
D.$x \in (-\infty,-19] \cup (-5,10)$\\
E.$x \in (-\infty,-19] \cup (-5,10]$\\
F.$x \in (-\infty,-19] \cup [-5,10)$\\
G.$x \in (-\infty,-19) \cup [-5,10]$\\
H.$x \in (-\infty,-19] \cup [-5,10]$
\testStop
\kluczStart
A
\kluczStop



\zadStart{Zadanie z Wikieł Z 1.62 b) moja wersja nr 858}

Rozwiązać nierówności $(x+19)(10-x)(x+6)\ge0$.
\zadStop
\rozwStart{Patryk Wirkus}{}
Miejsca zerowe naszego wielomianu to: $-19, 10, -6$.\\
Wielomian jest stopnia nieparzystego, ponadto znak współczynnika przy\linebreak najwyższej potędze x jest ujemny.\\ W związku z tym wykres wielomianu zaczyna się od lewej strony powyżej osi OX. A więc $$x \in (-\infty,-19) \cup (-6,10).$$
\rozwStop
\odpStart
$x \in (-\infty,-19) \cup (-6,10)$
\odpStop
\testStart
A.$x \in (-\infty,-19) \cup (-6,10)$\\
B.$x \in (-\infty,-19) \cup (-6,10]$\\
C.$x \in (-\infty,-19) \cup [-6,10)$\\
D.$x \in (-\infty,-19] \cup (-6,10)$\\
E.$x \in (-\infty,-19] \cup (-6,10]$\\
F.$x \in (-\infty,-19] \cup [-6,10)$\\
G.$x \in (-\infty,-19) \cup [-6,10]$\\
H.$x \in (-\infty,-19] \cup [-6,10]$
\testStop
\kluczStart
A
\kluczStop



\zadStart{Zadanie z Wikieł Z 1.62 b) moja wersja nr 859}

Rozwiązać nierówności $(x+19)(10-x)(x+7)\ge0$.
\zadStop
\rozwStart{Patryk Wirkus}{}
Miejsca zerowe naszego wielomianu to: $-19, 10, -7$.\\
Wielomian jest stopnia nieparzystego, ponadto znak współczynnika przy\linebreak najwyższej potędze x jest ujemny.\\ W związku z tym wykres wielomianu zaczyna się od lewej strony powyżej osi OX. A więc $$x \in (-\infty,-19) \cup (-7,10).$$
\rozwStop
\odpStart
$x \in (-\infty,-19) \cup (-7,10)$
\odpStop
\testStart
A.$x \in (-\infty,-19) \cup (-7,10)$\\
B.$x \in (-\infty,-19) \cup (-7,10]$\\
C.$x \in (-\infty,-19) \cup [-7,10)$\\
D.$x \in (-\infty,-19] \cup (-7,10)$\\
E.$x \in (-\infty,-19] \cup (-7,10]$\\
F.$x \in (-\infty,-19] \cup [-7,10)$\\
G.$x \in (-\infty,-19) \cup [-7,10]$\\
H.$x \in (-\infty,-19] \cup [-7,10]$
\testStop
\kluczStart
A
\kluczStop



\zadStart{Zadanie z Wikieł Z 1.62 b) moja wersja nr 860}

Rozwiązać nierówności $(x+19)(10-x)(x+8)\ge0$.
\zadStop
\rozwStart{Patryk Wirkus}{}
Miejsca zerowe naszego wielomianu to: $-19, 10, -8$.\\
Wielomian jest stopnia nieparzystego, ponadto znak współczynnika przy\linebreak najwyższej potędze x jest ujemny.\\ W związku z tym wykres wielomianu zaczyna się od lewej strony powyżej osi OX. A więc $$x \in (-\infty,-19) \cup (-8,10).$$
\rozwStop
\odpStart
$x \in (-\infty,-19) \cup (-8,10)$
\odpStop
\testStart
A.$x \in (-\infty,-19) \cup (-8,10)$\\
B.$x \in (-\infty,-19) \cup (-8,10]$\\
C.$x \in (-\infty,-19) \cup [-8,10)$\\
D.$x \in (-\infty,-19] \cup (-8,10)$\\
E.$x \in (-\infty,-19] \cup (-8,10]$\\
F.$x \in (-\infty,-19] \cup [-8,10)$\\
G.$x \in (-\infty,-19) \cup [-8,10]$\\
H.$x \in (-\infty,-19] \cup [-8,10]$
\testStop
\kluczStart
A
\kluczStop



\zadStart{Zadanie z Wikieł Z 1.62 b) moja wersja nr 861}

Rozwiązać nierówności $(x+19)(10-x)(x+9)\ge0$.
\zadStop
\rozwStart{Patryk Wirkus}{}
Miejsca zerowe naszego wielomianu to: $-19, 10, -9$.\\
Wielomian jest stopnia nieparzystego, ponadto znak współczynnika przy\linebreak najwyższej potędze x jest ujemny.\\ W związku z tym wykres wielomianu zaczyna się od lewej strony powyżej osi OX. A więc $$x \in (-\infty,-19) \cup (-9,10).$$
\rozwStop
\odpStart
$x \in (-\infty,-19) \cup (-9,10)$
\odpStop
\testStart
A.$x \in (-\infty,-19) \cup (-9,10)$\\
B.$x \in (-\infty,-19) \cup (-9,10]$\\
C.$x \in (-\infty,-19) \cup [-9,10)$\\
D.$x \in (-\infty,-19] \cup (-9,10)$\\
E.$x \in (-\infty,-19] \cup (-9,10]$\\
F.$x \in (-\infty,-19] \cup [-9,10)$\\
G.$x \in (-\infty,-19) \cup [-9,10]$\\
H.$x \in (-\infty,-19] \cup [-9,10]$
\testStop
\kluczStart
A
\kluczStop



\zadStart{Zadanie z Wikieł Z 1.62 b) moja wersja nr 862}

Rozwiązać nierówności $(x+19)(11-x)(x+1)\ge0$.
\zadStop
\rozwStart{Patryk Wirkus}{}
Miejsca zerowe naszego wielomianu to: $-19, 11, -1$.\\
Wielomian jest stopnia nieparzystego, ponadto znak współczynnika przy\linebreak najwyższej potędze x jest ujemny.\\ W związku z tym wykres wielomianu zaczyna się od lewej strony powyżej osi OX. A więc $$x \in (-\infty,-19) \cup (-1,11).$$
\rozwStop
\odpStart
$x \in (-\infty,-19) \cup (-1,11)$
\odpStop
\testStart
A.$x \in (-\infty,-19) \cup (-1,11)$\\
B.$x \in (-\infty,-19) \cup (-1,11]$\\
C.$x \in (-\infty,-19) \cup [-1,11)$\\
D.$x \in (-\infty,-19] \cup (-1,11)$\\
E.$x \in (-\infty,-19] \cup (-1,11]$\\
F.$x \in (-\infty,-19] \cup [-1,11)$\\
G.$x \in (-\infty,-19) \cup [-1,11]$\\
H.$x \in (-\infty,-19] \cup [-1,11]$
\testStop
\kluczStart
A
\kluczStop



\zadStart{Zadanie z Wikieł Z 1.62 b) moja wersja nr 863}

Rozwiązać nierówności $(x+19)(11-x)(x+2)\ge0$.
\zadStop
\rozwStart{Patryk Wirkus}{}
Miejsca zerowe naszego wielomianu to: $-19, 11, -2$.\\
Wielomian jest stopnia nieparzystego, ponadto znak współczynnika przy\linebreak najwyższej potędze x jest ujemny.\\ W związku z tym wykres wielomianu zaczyna się od lewej strony powyżej osi OX. A więc $$x \in (-\infty,-19) \cup (-2,11).$$
\rozwStop
\odpStart
$x \in (-\infty,-19) \cup (-2,11)$
\odpStop
\testStart
A.$x \in (-\infty,-19) \cup (-2,11)$\\
B.$x \in (-\infty,-19) \cup (-2,11]$\\
C.$x \in (-\infty,-19) \cup [-2,11)$\\
D.$x \in (-\infty,-19] \cup (-2,11)$\\
E.$x \in (-\infty,-19] \cup (-2,11]$\\
F.$x \in (-\infty,-19] \cup [-2,11)$\\
G.$x \in (-\infty,-19) \cup [-2,11]$\\
H.$x \in (-\infty,-19] \cup [-2,11]$
\testStop
\kluczStart
A
\kluczStop



\zadStart{Zadanie z Wikieł Z 1.62 b) moja wersja nr 864}

Rozwiązać nierówności $(x+19)(11-x)(x+3)\ge0$.
\zadStop
\rozwStart{Patryk Wirkus}{}
Miejsca zerowe naszego wielomianu to: $-19, 11, -3$.\\
Wielomian jest stopnia nieparzystego, ponadto znak współczynnika przy\linebreak najwyższej potędze x jest ujemny.\\ W związku z tym wykres wielomianu zaczyna się od lewej strony powyżej osi OX. A więc $$x \in (-\infty,-19) \cup (-3,11).$$
\rozwStop
\odpStart
$x \in (-\infty,-19) \cup (-3,11)$
\odpStop
\testStart
A.$x \in (-\infty,-19) \cup (-3,11)$\\
B.$x \in (-\infty,-19) \cup (-3,11]$\\
C.$x \in (-\infty,-19) \cup [-3,11)$\\
D.$x \in (-\infty,-19] \cup (-3,11)$\\
E.$x \in (-\infty,-19] \cup (-3,11]$\\
F.$x \in (-\infty,-19] \cup [-3,11)$\\
G.$x \in (-\infty,-19) \cup [-3,11]$\\
H.$x \in (-\infty,-19] \cup [-3,11]$
\testStop
\kluczStart
A
\kluczStop



\zadStart{Zadanie z Wikieł Z 1.62 b) moja wersja nr 865}

Rozwiązać nierówności $(x+19)(11-x)(x+4)\ge0$.
\zadStop
\rozwStart{Patryk Wirkus}{}
Miejsca zerowe naszego wielomianu to: $-19, 11, -4$.\\
Wielomian jest stopnia nieparzystego, ponadto znak współczynnika przy\linebreak najwyższej potędze x jest ujemny.\\ W związku z tym wykres wielomianu zaczyna się od lewej strony powyżej osi OX. A więc $$x \in (-\infty,-19) \cup (-4,11).$$
\rozwStop
\odpStart
$x \in (-\infty,-19) \cup (-4,11)$
\odpStop
\testStart
A.$x \in (-\infty,-19) \cup (-4,11)$\\
B.$x \in (-\infty,-19) \cup (-4,11]$\\
C.$x \in (-\infty,-19) \cup [-4,11)$\\
D.$x \in (-\infty,-19] \cup (-4,11)$\\
E.$x \in (-\infty,-19] \cup (-4,11]$\\
F.$x \in (-\infty,-19] \cup [-4,11)$\\
G.$x \in (-\infty,-19) \cup [-4,11]$\\
H.$x \in (-\infty,-19] \cup [-4,11]$
\testStop
\kluczStart
A
\kluczStop



\zadStart{Zadanie z Wikieł Z 1.62 b) moja wersja nr 866}

Rozwiązać nierówności $(x+19)(11-x)(x+5)\ge0$.
\zadStop
\rozwStart{Patryk Wirkus}{}
Miejsca zerowe naszego wielomianu to: $-19, 11, -5$.\\
Wielomian jest stopnia nieparzystego, ponadto znak współczynnika przy\linebreak najwyższej potędze x jest ujemny.\\ W związku z tym wykres wielomianu zaczyna się od lewej strony powyżej osi OX. A więc $$x \in (-\infty,-19) \cup (-5,11).$$
\rozwStop
\odpStart
$x \in (-\infty,-19) \cup (-5,11)$
\odpStop
\testStart
A.$x \in (-\infty,-19) \cup (-5,11)$\\
B.$x \in (-\infty,-19) \cup (-5,11]$\\
C.$x \in (-\infty,-19) \cup [-5,11)$\\
D.$x \in (-\infty,-19] \cup (-5,11)$\\
E.$x \in (-\infty,-19] \cup (-5,11]$\\
F.$x \in (-\infty,-19] \cup [-5,11)$\\
G.$x \in (-\infty,-19) \cup [-5,11]$\\
H.$x \in (-\infty,-19] \cup [-5,11]$
\testStop
\kluczStart
A
\kluczStop



\zadStart{Zadanie z Wikieł Z 1.62 b) moja wersja nr 867}

Rozwiązać nierówności $(x+19)(11-x)(x+6)\ge0$.
\zadStop
\rozwStart{Patryk Wirkus}{}
Miejsca zerowe naszego wielomianu to: $-19, 11, -6$.\\
Wielomian jest stopnia nieparzystego, ponadto znak współczynnika przy\linebreak najwyższej potędze x jest ujemny.\\ W związku z tym wykres wielomianu zaczyna się od lewej strony powyżej osi OX. A więc $$x \in (-\infty,-19) \cup (-6,11).$$
\rozwStop
\odpStart
$x \in (-\infty,-19) \cup (-6,11)$
\odpStop
\testStart
A.$x \in (-\infty,-19) \cup (-6,11)$\\
B.$x \in (-\infty,-19) \cup (-6,11]$\\
C.$x \in (-\infty,-19) \cup [-6,11)$\\
D.$x \in (-\infty,-19] \cup (-6,11)$\\
E.$x \in (-\infty,-19] \cup (-6,11]$\\
F.$x \in (-\infty,-19] \cup [-6,11)$\\
G.$x \in (-\infty,-19) \cup [-6,11]$\\
H.$x \in (-\infty,-19] \cup [-6,11]$
\testStop
\kluczStart
A
\kluczStop



\zadStart{Zadanie z Wikieł Z 1.62 b) moja wersja nr 868}

Rozwiązać nierówności $(x+19)(11-x)(x+7)\ge0$.
\zadStop
\rozwStart{Patryk Wirkus}{}
Miejsca zerowe naszego wielomianu to: $-19, 11, -7$.\\
Wielomian jest stopnia nieparzystego, ponadto znak współczynnika przy\linebreak najwyższej potędze x jest ujemny.\\ W związku z tym wykres wielomianu zaczyna się od lewej strony powyżej osi OX. A więc $$x \in (-\infty,-19) \cup (-7,11).$$
\rozwStop
\odpStart
$x \in (-\infty,-19) \cup (-7,11)$
\odpStop
\testStart
A.$x \in (-\infty,-19) \cup (-7,11)$\\
B.$x \in (-\infty,-19) \cup (-7,11]$\\
C.$x \in (-\infty,-19) \cup [-7,11)$\\
D.$x \in (-\infty,-19] \cup (-7,11)$\\
E.$x \in (-\infty,-19] \cup (-7,11]$\\
F.$x \in (-\infty,-19] \cup [-7,11)$\\
G.$x \in (-\infty,-19) \cup [-7,11]$\\
H.$x \in (-\infty,-19] \cup [-7,11]$
\testStop
\kluczStart
A
\kluczStop



\zadStart{Zadanie z Wikieł Z 1.62 b) moja wersja nr 869}

Rozwiązać nierówności $(x+19)(11-x)(x+8)\ge0$.
\zadStop
\rozwStart{Patryk Wirkus}{}
Miejsca zerowe naszego wielomianu to: $-19, 11, -8$.\\
Wielomian jest stopnia nieparzystego, ponadto znak współczynnika przy\linebreak najwyższej potędze x jest ujemny.\\ W związku z tym wykres wielomianu zaczyna się od lewej strony powyżej osi OX. A więc $$x \in (-\infty,-19) \cup (-8,11).$$
\rozwStop
\odpStart
$x \in (-\infty,-19) \cup (-8,11)$
\odpStop
\testStart
A.$x \in (-\infty,-19) \cup (-8,11)$\\
B.$x \in (-\infty,-19) \cup (-8,11]$\\
C.$x \in (-\infty,-19) \cup [-8,11)$\\
D.$x \in (-\infty,-19] \cup (-8,11)$\\
E.$x \in (-\infty,-19] \cup (-8,11]$\\
F.$x \in (-\infty,-19] \cup [-8,11)$\\
G.$x \in (-\infty,-19) \cup [-8,11]$\\
H.$x \in (-\infty,-19] \cup [-8,11]$
\testStop
\kluczStart
A
\kluczStop



\zadStart{Zadanie z Wikieł Z 1.62 b) moja wersja nr 870}

Rozwiązać nierówności $(x+19)(11-x)(x+9)\ge0$.
\zadStop
\rozwStart{Patryk Wirkus}{}
Miejsca zerowe naszego wielomianu to: $-19, 11, -9$.\\
Wielomian jest stopnia nieparzystego, ponadto znak współczynnika przy\linebreak najwyższej potędze x jest ujemny.\\ W związku z tym wykres wielomianu zaczyna się od lewej strony powyżej osi OX. A więc $$x \in (-\infty,-19) \cup (-9,11).$$
\rozwStop
\odpStart
$x \in (-\infty,-19) \cup (-9,11)$
\odpStop
\testStart
A.$x \in (-\infty,-19) \cup (-9,11)$\\
B.$x \in (-\infty,-19) \cup (-9,11]$\\
C.$x \in (-\infty,-19) \cup [-9,11)$\\
D.$x \in (-\infty,-19] \cup (-9,11)$\\
E.$x \in (-\infty,-19] \cup (-9,11]$\\
F.$x \in (-\infty,-19] \cup [-9,11)$\\
G.$x \in (-\infty,-19) \cup [-9,11]$\\
H.$x \in (-\infty,-19] \cup [-9,11]$
\testStop
\kluczStart
A
\kluczStop



\zadStart{Zadanie z Wikieł Z 1.62 b) moja wersja nr 871}

Rozwiązać nierówności $(x+19)(11-x)(x+10)\ge0$.
\zadStop
\rozwStart{Patryk Wirkus}{}
Miejsca zerowe naszego wielomianu to: $-19, 11, -10$.\\
Wielomian jest stopnia nieparzystego, ponadto znak współczynnika przy\linebreak najwyższej potędze x jest ujemny.\\ W związku z tym wykres wielomianu zaczyna się od lewej strony powyżej osi OX. A więc $$x \in (-\infty,-19) \cup (-10,11).$$
\rozwStop
\odpStart
$x \in (-\infty,-19) \cup (-10,11)$
\odpStop
\testStart
A.$x \in (-\infty,-19) \cup (-10,11)$\\
B.$x \in (-\infty,-19) \cup (-10,11]$\\
C.$x \in (-\infty,-19) \cup [-10,11)$\\
D.$x \in (-\infty,-19] \cup (-10,11)$\\
E.$x \in (-\infty,-19] \cup (-10,11]$\\
F.$x \in (-\infty,-19] \cup [-10,11)$\\
G.$x \in (-\infty,-19) \cup [-10,11]$\\
H.$x \in (-\infty,-19] \cup [-10,11]$
\testStop
\kluczStart
A
\kluczStop



\zadStart{Zadanie z Wikieł Z 1.62 b) moja wersja nr 872}

Rozwiązać nierówności $(x+19)(12-x)(x+1)\ge0$.
\zadStop
\rozwStart{Patryk Wirkus}{}
Miejsca zerowe naszego wielomianu to: $-19, 12, -1$.\\
Wielomian jest stopnia nieparzystego, ponadto znak współczynnika przy\linebreak najwyższej potędze x jest ujemny.\\ W związku z tym wykres wielomianu zaczyna się od lewej strony powyżej osi OX. A więc $$x \in (-\infty,-19) \cup (-1,12).$$
\rozwStop
\odpStart
$x \in (-\infty,-19) \cup (-1,12)$
\odpStop
\testStart
A.$x \in (-\infty,-19) \cup (-1,12)$\\
B.$x \in (-\infty,-19) \cup (-1,12]$\\
C.$x \in (-\infty,-19) \cup [-1,12)$\\
D.$x \in (-\infty,-19] \cup (-1,12)$\\
E.$x \in (-\infty,-19] \cup (-1,12]$\\
F.$x \in (-\infty,-19] \cup [-1,12)$\\
G.$x \in (-\infty,-19) \cup [-1,12]$\\
H.$x \in (-\infty,-19] \cup [-1,12]$
\testStop
\kluczStart
A
\kluczStop



\zadStart{Zadanie z Wikieł Z 1.62 b) moja wersja nr 873}

Rozwiązać nierówności $(x+19)(12-x)(x+2)\ge0$.
\zadStop
\rozwStart{Patryk Wirkus}{}
Miejsca zerowe naszego wielomianu to: $-19, 12, -2$.\\
Wielomian jest stopnia nieparzystego, ponadto znak współczynnika przy\linebreak najwyższej potędze x jest ujemny.\\ W związku z tym wykres wielomianu zaczyna się od lewej strony powyżej osi OX. A więc $$x \in (-\infty,-19) \cup (-2,12).$$
\rozwStop
\odpStart
$x \in (-\infty,-19) \cup (-2,12)$
\odpStop
\testStart
A.$x \in (-\infty,-19) \cup (-2,12)$\\
B.$x \in (-\infty,-19) \cup (-2,12]$\\
C.$x \in (-\infty,-19) \cup [-2,12)$\\
D.$x \in (-\infty,-19] \cup (-2,12)$\\
E.$x \in (-\infty,-19] \cup (-2,12]$\\
F.$x \in (-\infty,-19] \cup [-2,12)$\\
G.$x \in (-\infty,-19) \cup [-2,12]$\\
H.$x \in (-\infty,-19] \cup [-2,12]$
\testStop
\kluczStart
A
\kluczStop



\zadStart{Zadanie z Wikieł Z 1.62 b) moja wersja nr 874}

Rozwiązać nierówności $(x+19)(12-x)(x+3)\ge0$.
\zadStop
\rozwStart{Patryk Wirkus}{}
Miejsca zerowe naszego wielomianu to: $-19, 12, -3$.\\
Wielomian jest stopnia nieparzystego, ponadto znak współczynnika przy\linebreak najwyższej potędze x jest ujemny.\\ W związku z tym wykres wielomianu zaczyna się od lewej strony powyżej osi OX. A więc $$x \in (-\infty,-19) \cup (-3,12).$$
\rozwStop
\odpStart
$x \in (-\infty,-19) \cup (-3,12)$
\odpStop
\testStart
A.$x \in (-\infty,-19) \cup (-3,12)$\\
B.$x \in (-\infty,-19) \cup (-3,12]$\\
C.$x \in (-\infty,-19) \cup [-3,12)$\\
D.$x \in (-\infty,-19] \cup (-3,12)$\\
E.$x \in (-\infty,-19] \cup (-3,12]$\\
F.$x \in (-\infty,-19] \cup [-3,12)$\\
G.$x \in (-\infty,-19) \cup [-3,12]$\\
H.$x \in (-\infty,-19] \cup [-3,12]$
\testStop
\kluczStart
A
\kluczStop



\zadStart{Zadanie z Wikieł Z 1.62 b) moja wersja nr 875}

Rozwiązać nierówności $(x+19)(12-x)(x+4)\ge0$.
\zadStop
\rozwStart{Patryk Wirkus}{}
Miejsca zerowe naszego wielomianu to: $-19, 12, -4$.\\
Wielomian jest stopnia nieparzystego, ponadto znak współczynnika przy\linebreak najwyższej potędze x jest ujemny.\\ W związku z tym wykres wielomianu zaczyna się od lewej strony powyżej osi OX. A więc $$x \in (-\infty,-19) \cup (-4,12).$$
\rozwStop
\odpStart
$x \in (-\infty,-19) \cup (-4,12)$
\odpStop
\testStart
A.$x \in (-\infty,-19) \cup (-4,12)$\\
B.$x \in (-\infty,-19) \cup (-4,12]$\\
C.$x \in (-\infty,-19) \cup [-4,12)$\\
D.$x \in (-\infty,-19] \cup (-4,12)$\\
E.$x \in (-\infty,-19] \cup (-4,12]$\\
F.$x \in (-\infty,-19] \cup [-4,12)$\\
G.$x \in (-\infty,-19) \cup [-4,12]$\\
H.$x \in (-\infty,-19] \cup [-4,12]$
\testStop
\kluczStart
A
\kluczStop



\zadStart{Zadanie z Wikieł Z 1.62 b) moja wersja nr 876}

Rozwiązać nierówności $(x+19)(12-x)(x+5)\ge0$.
\zadStop
\rozwStart{Patryk Wirkus}{}
Miejsca zerowe naszego wielomianu to: $-19, 12, -5$.\\
Wielomian jest stopnia nieparzystego, ponadto znak współczynnika przy\linebreak najwyższej potędze x jest ujemny.\\ W związku z tym wykres wielomianu zaczyna się od lewej strony powyżej osi OX. A więc $$x \in (-\infty,-19) \cup (-5,12).$$
\rozwStop
\odpStart
$x \in (-\infty,-19) \cup (-5,12)$
\odpStop
\testStart
A.$x \in (-\infty,-19) \cup (-5,12)$\\
B.$x \in (-\infty,-19) \cup (-5,12]$\\
C.$x \in (-\infty,-19) \cup [-5,12)$\\
D.$x \in (-\infty,-19] \cup (-5,12)$\\
E.$x \in (-\infty,-19] \cup (-5,12]$\\
F.$x \in (-\infty,-19] \cup [-5,12)$\\
G.$x \in (-\infty,-19) \cup [-5,12]$\\
H.$x \in (-\infty,-19] \cup [-5,12]$
\testStop
\kluczStart
A
\kluczStop



\zadStart{Zadanie z Wikieł Z 1.62 b) moja wersja nr 877}

Rozwiązać nierówności $(x+19)(12-x)(x+6)\ge0$.
\zadStop
\rozwStart{Patryk Wirkus}{}
Miejsca zerowe naszego wielomianu to: $-19, 12, -6$.\\
Wielomian jest stopnia nieparzystego, ponadto znak współczynnika przy\linebreak najwyższej potędze x jest ujemny.\\ W związku z tym wykres wielomianu zaczyna się od lewej strony powyżej osi OX. A więc $$x \in (-\infty,-19) \cup (-6,12).$$
\rozwStop
\odpStart
$x \in (-\infty,-19) \cup (-6,12)$
\odpStop
\testStart
A.$x \in (-\infty,-19) \cup (-6,12)$\\
B.$x \in (-\infty,-19) \cup (-6,12]$\\
C.$x \in (-\infty,-19) \cup [-6,12)$\\
D.$x \in (-\infty,-19] \cup (-6,12)$\\
E.$x \in (-\infty,-19] \cup (-6,12]$\\
F.$x \in (-\infty,-19] \cup [-6,12)$\\
G.$x \in (-\infty,-19) \cup [-6,12]$\\
H.$x \in (-\infty,-19] \cup [-6,12]$
\testStop
\kluczStart
A
\kluczStop



\zadStart{Zadanie z Wikieł Z 1.62 b) moja wersja nr 878}

Rozwiązać nierówności $(x+19)(12-x)(x+7)\ge0$.
\zadStop
\rozwStart{Patryk Wirkus}{}
Miejsca zerowe naszego wielomianu to: $-19, 12, -7$.\\
Wielomian jest stopnia nieparzystego, ponadto znak współczynnika przy\linebreak najwyższej potędze x jest ujemny.\\ W związku z tym wykres wielomianu zaczyna się od lewej strony powyżej osi OX. A więc $$x \in (-\infty,-19) \cup (-7,12).$$
\rozwStop
\odpStart
$x \in (-\infty,-19) \cup (-7,12)$
\odpStop
\testStart
A.$x \in (-\infty,-19) \cup (-7,12)$\\
B.$x \in (-\infty,-19) \cup (-7,12]$\\
C.$x \in (-\infty,-19) \cup [-7,12)$\\
D.$x \in (-\infty,-19] \cup (-7,12)$\\
E.$x \in (-\infty,-19] \cup (-7,12]$\\
F.$x \in (-\infty,-19] \cup [-7,12)$\\
G.$x \in (-\infty,-19) \cup [-7,12]$\\
H.$x \in (-\infty,-19] \cup [-7,12]$
\testStop
\kluczStart
A
\kluczStop



\zadStart{Zadanie z Wikieł Z 1.62 b) moja wersja nr 879}

Rozwiązać nierówności $(x+19)(12-x)(x+8)\ge0$.
\zadStop
\rozwStart{Patryk Wirkus}{}
Miejsca zerowe naszego wielomianu to: $-19, 12, -8$.\\
Wielomian jest stopnia nieparzystego, ponadto znak współczynnika przy\linebreak najwyższej potędze x jest ujemny.\\ W związku z tym wykres wielomianu zaczyna się od lewej strony powyżej osi OX. A więc $$x \in (-\infty,-19) \cup (-8,12).$$
\rozwStop
\odpStart
$x \in (-\infty,-19) \cup (-8,12)$
\odpStop
\testStart
A.$x \in (-\infty,-19) \cup (-8,12)$\\
B.$x \in (-\infty,-19) \cup (-8,12]$\\
C.$x \in (-\infty,-19) \cup [-8,12)$\\
D.$x \in (-\infty,-19] \cup (-8,12)$\\
E.$x \in (-\infty,-19] \cup (-8,12]$\\
F.$x \in (-\infty,-19] \cup [-8,12)$\\
G.$x \in (-\infty,-19) \cup [-8,12]$\\
H.$x \in (-\infty,-19] \cup [-8,12]$
\testStop
\kluczStart
A
\kluczStop



\zadStart{Zadanie z Wikieł Z 1.62 b) moja wersja nr 880}

Rozwiązać nierówności $(x+19)(12-x)(x+9)\ge0$.
\zadStop
\rozwStart{Patryk Wirkus}{}
Miejsca zerowe naszego wielomianu to: $-19, 12, -9$.\\
Wielomian jest stopnia nieparzystego, ponadto znak współczynnika przy\linebreak najwyższej potędze x jest ujemny.\\ W związku z tym wykres wielomianu zaczyna się od lewej strony powyżej osi OX. A więc $$x \in (-\infty,-19) \cup (-9,12).$$
\rozwStop
\odpStart
$x \in (-\infty,-19) \cup (-9,12)$
\odpStop
\testStart
A.$x \in (-\infty,-19) \cup (-9,12)$\\
B.$x \in (-\infty,-19) \cup (-9,12]$\\
C.$x \in (-\infty,-19) \cup [-9,12)$\\
D.$x \in (-\infty,-19] \cup (-9,12)$\\
E.$x \in (-\infty,-19] \cup (-9,12]$\\
F.$x \in (-\infty,-19] \cup [-9,12)$\\
G.$x \in (-\infty,-19) \cup [-9,12]$\\
H.$x \in (-\infty,-19] \cup [-9,12]$
\testStop
\kluczStart
A
\kluczStop



\zadStart{Zadanie z Wikieł Z 1.62 b) moja wersja nr 881}

Rozwiązać nierówności $(x+19)(12-x)(x+10)\ge0$.
\zadStop
\rozwStart{Patryk Wirkus}{}
Miejsca zerowe naszego wielomianu to: $-19, 12, -10$.\\
Wielomian jest stopnia nieparzystego, ponadto znak współczynnika przy\linebreak najwyższej potędze x jest ujemny.\\ W związku z tym wykres wielomianu zaczyna się od lewej strony powyżej osi OX. A więc $$x \in (-\infty,-19) \cup (-10,12).$$
\rozwStop
\odpStart
$x \in (-\infty,-19) \cup (-10,12)$
\odpStop
\testStart
A.$x \in (-\infty,-19) \cup (-10,12)$\\
B.$x \in (-\infty,-19) \cup (-10,12]$\\
C.$x \in (-\infty,-19) \cup [-10,12)$\\
D.$x \in (-\infty,-19] \cup (-10,12)$\\
E.$x \in (-\infty,-19] \cup (-10,12]$\\
F.$x \in (-\infty,-19] \cup [-10,12)$\\
G.$x \in (-\infty,-19) \cup [-10,12]$\\
H.$x \in (-\infty,-19] \cup [-10,12]$
\testStop
\kluczStart
A
\kluczStop



\zadStart{Zadanie z Wikieł Z 1.62 b) moja wersja nr 882}

Rozwiązać nierówności $(x+19)(12-x)(x+11)\ge0$.
\zadStop
\rozwStart{Patryk Wirkus}{}
Miejsca zerowe naszego wielomianu to: $-19, 12, -11$.\\
Wielomian jest stopnia nieparzystego, ponadto znak współczynnika przy\linebreak najwyższej potędze x jest ujemny.\\ W związku z tym wykres wielomianu zaczyna się od lewej strony powyżej osi OX. A więc $$x \in (-\infty,-19) \cup (-11,12).$$
\rozwStop
\odpStart
$x \in (-\infty,-19) \cup (-11,12)$
\odpStop
\testStart
A.$x \in (-\infty,-19) \cup (-11,12)$\\
B.$x \in (-\infty,-19) \cup (-11,12]$\\
C.$x \in (-\infty,-19) \cup [-11,12)$\\
D.$x \in (-\infty,-19] \cup (-11,12)$\\
E.$x \in (-\infty,-19] \cup (-11,12]$\\
F.$x \in (-\infty,-19] \cup [-11,12)$\\
G.$x \in (-\infty,-19) \cup [-11,12]$\\
H.$x \in (-\infty,-19] \cup [-11,12]$
\testStop
\kluczStart
A
\kluczStop



\zadStart{Zadanie z Wikieł Z 1.62 b) moja wersja nr 883}

Rozwiązać nierówności $(x+19)(13-x)(x+1)\ge0$.
\zadStop
\rozwStart{Patryk Wirkus}{}
Miejsca zerowe naszego wielomianu to: $-19, 13, -1$.\\
Wielomian jest stopnia nieparzystego, ponadto znak współczynnika przy\linebreak najwyższej potędze x jest ujemny.\\ W związku z tym wykres wielomianu zaczyna się od lewej strony powyżej osi OX. A więc $$x \in (-\infty,-19) \cup (-1,13).$$
\rozwStop
\odpStart
$x \in (-\infty,-19) \cup (-1,13)$
\odpStop
\testStart
A.$x \in (-\infty,-19) \cup (-1,13)$\\
B.$x \in (-\infty,-19) \cup (-1,13]$\\
C.$x \in (-\infty,-19) \cup [-1,13)$\\
D.$x \in (-\infty,-19] \cup (-1,13)$\\
E.$x \in (-\infty,-19] \cup (-1,13]$\\
F.$x \in (-\infty,-19] \cup [-1,13)$\\
G.$x \in (-\infty,-19) \cup [-1,13]$\\
H.$x \in (-\infty,-19] \cup [-1,13]$
\testStop
\kluczStart
A
\kluczStop



\zadStart{Zadanie z Wikieł Z 1.62 b) moja wersja nr 884}

Rozwiązać nierówności $(x+19)(13-x)(x+2)\ge0$.
\zadStop
\rozwStart{Patryk Wirkus}{}
Miejsca zerowe naszego wielomianu to: $-19, 13, -2$.\\
Wielomian jest stopnia nieparzystego, ponadto znak współczynnika przy\linebreak najwyższej potędze x jest ujemny.\\ W związku z tym wykres wielomianu zaczyna się od lewej strony powyżej osi OX. A więc $$x \in (-\infty,-19) \cup (-2,13).$$
\rozwStop
\odpStart
$x \in (-\infty,-19) \cup (-2,13)$
\odpStop
\testStart
A.$x \in (-\infty,-19) \cup (-2,13)$\\
B.$x \in (-\infty,-19) \cup (-2,13]$\\
C.$x \in (-\infty,-19) \cup [-2,13)$\\
D.$x \in (-\infty,-19] \cup (-2,13)$\\
E.$x \in (-\infty,-19] \cup (-2,13]$\\
F.$x \in (-\infty,-19] \cup [-2,13)$\\
G.$x \in (-\infty,-19) \cup [-2,13]$\\
H.$x \in (-\infty,-19] \cup [-2,13]$
\testStop
\kluczStart
A
\kluczStop



\zadStart{Zadanie z Wikieł Z 1.62 b) moja wersja nr 885}

Rozwiązać nierówności $(x+19)(13-x)(x+3)\ge0$.
\zadStop
\rozwStart{Patryk Wirkus}{}
Miejsca zerowe naszego wielomianu to: $-19, 13, -3$.\\
Wielomian jest stopnia nieparzystego, ponadto znak współczynnika przy\linebreak najwyższej potędze x jest ujemny.\\ W związku z tym wykres wielomianu zaczyna się od lewej strony powyżej osi OX. A więc $$x \in (-\infty,-19) \cup (-3,13).$$
\rozwStop
\odpStart
$x \in (-\infty,-19) \cup (-3,13)$
\odpStop
\testStart
A.$x \in (-\infty,-19) \cup (-3,13)$\\
B.$x \in (-\infty,-19) \cup (-3,13]$\\
C.$x \in (-\infty,-19) \cup [-3,13)$\\
D.$x \in (-\infty,-19] \cup (-3,13)$\\
E.$x \in (-\infty,-19] \cup (-3,13]$\\
F.$x \in (-\infty,-19] \cup [-3,13)$\\
G.$x \in (-\infty,-19) \cup [-3,13]$\\
H.$x \in (-\infty,-19] \cup [-3,13]$
\testStop
\kluczStart
A
\kluczStop



\zadStart{Zadanie z Wikieł Z 1.62 b) moja wersja nr 886}

Rozwiązać nierówności $(x+19)(13-x)(x+4)\ge0$.
\zadStop
\rozwStart{Patryk Wirkus}{}
Miejsca zerowe naszego wielomianu to: $-19, 13, -4$.\\
Wielomian jest stopnia nieparzystego, ponadto znak współczynnika przy\linebreak najwyższej potędze x jest ujemny.\\ W związku z tym wykres wielomianu zaczyna się od lewej strony powyżej osi OX. A więc $$x \in (-\infty,-19) \cup (-4,13).$$
\rozwStop
\odpStart
$x \in (-\infty,-19) \cup (-4,13)$
\odpStop
\testStart
A.$x \in (-\infty,-19) \cup (-4,13)$\\
B.$x \in (-\infty,-19) \cup (-4,13]$\\
C.$x \in (-\infty,-19) \cup [-4,13)$\\
D.$x \in (-\infty,-19] \cup (-4,13)$\\
E.$x \in (-\infty,-19] \cup (-4,13]$\\
F.$x \in (-\infty,-19] \cup [-4,13)$\\
G.$x \in (-\infty,-19) \cup [-4,13]$\\
H.$x \in (-\infty,-19] \cup [-4,13]$
\testStop
\kluczStart
A
\kluczStop



\zadStart{Zadanie z Wikieł Z 1.62 b) moja wersja nr 887}

Rozwiązać nierówności $(x+19)(13-x)(x+5)\ge0$.
\zadStop
\rozwStart{Patryk Wirkus}{}
Miejsca zerowe naszego wielomianu to: $-19, 13, -5$.\\
Wielomian jest stopnia nieparzystego, ponadto znak współczynnika przy\linebreak najwyższej potędze x jest ujemny.\\ W związku z tym wykres wielomianu zaczyna się od lewej strony powyżej osi OX. A więc $$x \in (-\infty,-19) \cup (-5,13).$$
\rozwStop
\odpStart
$x \in (-\infty,-19) \cup (-5,13)$
\odpStop
\testStart
A.$x \in (-\infty,-19) \cup (-5,13)$\\
B.$x \in (-\infty,-19) \cup (-5,13]$\\
C.$x \in (-\infty,-19) \cup [-5,13)$\\
D.$x \in (-\infty,-19] \cup (-5,13)$\\
E.$x \in (-\infty,-19] \cup (-5,13]$\\
F.$x \in (-\infty,-19] \cup [-5,13)$\\
G.$x \in (-\infty,-19) \cup [-5,13]$\\
H.$x \in (-\infty,-19] \cup [-5,13]$
\testStop
\kluczStart
A
\kluczStop



\zadStart{Zadanie z Wikieł Z 1.62 b) moja wersja nr 888}

Rozwiązać nierówności $(x+19)(13-x)(x+6)\ge0$.
\zadStop
\rozwStart{Patryk Wirkus}{}
Miejsca zerowe naszego wielomianu to: $-19, 13, -6$.\\
Wielomian jest stopnia nieparzystego, ponadto znak współczynnika przy\linebreak najwyższej potędze x jest ujemny.\\ W związku z tym wykres wielomianu zaczyna się od lewej strony powyżej osi OX. A więc $$x \in (-\infty,-19) \cup (-6,13).$$
\rozwStop
\odpStart
$x \in (-\infty,-19) \cup (-6,13)$
\odpStop
\testStart
A.$x \in (-\infty,-19) \cup (-6,13)$\\
B.$x \in (-\infty,-19) \cup (-6,13]$\\
C.$x \in (-\infty,-19) \cup [-6,13)$\\
D.$x \in (-\infty,-19] \cup (-6,13)$\\
E.$x \in (-\infty,-19] \cup (-6,13]$\\
F.$x \in (-\infty,-19] \cup [-6,13)$\\
G.$x \in (-\infty,-19) \cup [-6,13]$\\
H.$x \in (-\infty,-19] \cup [-6,13]$
\testStop
\kluczStart
A
\kluczStop



\zadStart{Zadanie z Wikieł Z 1.62 b) moja wersja nr 889}

Rozwiązać nierówności $(x+19)(13-x)(x+7)\ge0$.
\zadStop
\rozwStart{Patryk Wirkus}{}
Miejsca zerowe naszego wielomianu to: $-19, 13, -7$.\\
Wielomian jest stopnia nieparzystego, ponadto znak współczynnika przy\linebreak najwyższej potędze x jest ujemny.\\ W związku z tym wykres wielomianu zaczyna się od lewej strony powyżej osi OX. A więc $$x \in (-\infty,-19) \cup (-7,13).$$
\rozwStop
\odpStart
$x \in (-\infty,-19) \cup (-7,13)$
\odpStop
\testStart
A.$x \in (-\infty,-19) \cup (-7,13)$\\
B.$x \in (-\infty,-19) \cup (-7,13]$\\
C.$x \in (-\infty,-19) \cup [-7,13)$\\
D.$x \in (-\infty,-19] \cup (-7,13)$\\
E.$x \in (-\infty,-19] \cup (-7,13]$\\
F.$x \in (-\infty,-19] \cup [-7,13)$\\
G.$x \in (-\infty,-19) \cup [-7,13]$\\
H.$x \in (-\infty,-19] \cup [-7,13]$
\testStop
\kluczStart
A
\kluczStop



\zadStart{Zadanie z Wikieł Z 1.62 b) moja wersja nr 890}

Rozwiązać nierówności $(x+19)(13-x)(x+8)\ge0$.
\zadStop
\rozwStart{Patryk Wirkus}{}
Miejsca zerowe naszego wielomianu to: $-19, 13, -8$.\\
Wielomian jest stopnia nieparzystego, ponadto znak współczynnika przy\linebreak najwyższej potędze x jest ujemny.\\ W związku z tym wykres wielomianu zaczyna się od lewej strony powyżej osi OX. A więc $$x \in (-\infty,-19) \cup (-8,13).$$
\rozwStop
\odpStart
$x \in (-\infty,-19) \cup (-8,13)$
\odpStop
\testStart
A.$x \in (-\infty,-19) \cup (-8,13)$\\
B.$x \in (-\infty,-19) \cup (-8,13]$\\
C.$x \in (-\infty,-19) \cup [-8,13)$\\
D.$x \in (-\infty,-19] \cup (-8,13)$\\
E.$x \in (-\infty,-19] \cup (-8,13]$\\
F.$x \in (-\infty,-19] \cup [-8,13)$\\
G.$x \in (-\infty,-19) \cup [-8,13]$\\
H.$x \in (-\infty,-19] \cup [-8,13]$
\testStop
\kluczStart
A
\kluczStop



\zadStart{Zadanie z Wikieł Z 1.62 b) moja wersja nr 891}

Rozwiązać nierówności $(x+19)(13-x)(x+9)\ge0$.
\zadStop
\rozwStart{Patryk Wirkus}{}
Miejsca zerowe naszego wielomianu to: $-19, 13, -9$.\\
Wielomian jest stopnia nieparzystego, ponadto znak współczynnika przy\linebreak najwyższej potędze x jest ujemny.\\ W związku z tym wykres wielomianu zaczyna się od lewej strony powyżej osi OX. A więc $$x \in (-\infty,-19) \cup (-9,13).$$
\rozwStop
\odpStart
$x \in (-\infty,-19) \cup (-9,13)$
\odpStop
\testStart
A.$x \in (-\infty,-19) \cup (-9,13)$\\
B.$x \in (-\infty,-19) \cup (-9,13]$\\
C.$x \in (-\infty,-19) \cup [-9,13)$\\
D.$x \in (-\infty,-19] \cup (-9,13)$\\
E.$x \in (-\infty,-19] \cup (-9,13]$\\
F.$x \in (-\infty,-19] \cup [-9,13)$\\
G.$x \in (-\infty,-19) \cup [-9,13]$\\
H.$x \in (-\infty,-19] \cup [-9,13]$
\testStop
\kluczStart
A
\kluczStop



\zadStart{Zadanie z Wikieł Z 1.62 b) moja wersja nr 892}

Rozwiązać nierówności $(x+19)(13-x)(x+10)\ge0$.
\zadStop
\rozwStart{Patryk Wirkus}{}
Miejsca zerowe naszego wielomianu to: $-19, 13, -10$.\\
Wielomian jest stopnia nieparzystego, ponadto znak współczynnika przy\linebreak najwyższej potędze x jest ujemny.\\ W związku z tym wykres wielomianu zaczyna się od lewej strony powyżej osi OX. A więc $$x \in (-\infty,-19) \cup (-10,13).$$
\rozwStop
\odpStart
$x \in (-\infty,-19) \cup (-10,13)$
\odpStop
\testStart
A.$x \in (-\infty,-19) \cup (-10,13)$\\
B.$x \in (-\infty,-19) \cup (-10,13]$\\
C.$x \in (-\infty,-19) \cup [-10,13)$\\
D.$x \in (-\infty,-19] \cup (-10,13)$\\
E.$x \in (-\infty,-19] \cup (-10,13]$\\
F.$x \in (-\infty,-19] \cup [-10,13)$\\
G.$x \in (-\infty,-19) \cup [-10,13]$\\
H.$x \in (-\infty,-19] \cup [-10,13]$
\testStop
\kluczStart
A
\kluczStop



\zadStart{Zadanie z Wikieł Z 1.62 b) moja wersja nr 893}

Rozwiązać nierówności $(x+19)(13-x)(x+11)\ge0$.
\zadStop
\rozwStart{Patryk Wirkus}{}
Miejsca zerowe naszego wielomianu to: $-19, 13, -11$.\\
Wielomian jest stopnia nieparzystego, ponadto znak współczynnika przy\linebreak najwyższej potędze x jest ujemny.\\ W związku z tym wykres wielomianu zaczyna się od lewej strony powyżej osi OX. A więc $$x \in (-\infty,-19) \cup (-11,13).$$
\rozwStop
\odpStart
$x \in (-\infty,-19) \cup (-11,13)$
\odpStop
\testStart
A.$x \in (-\infty,-19) \cup (-11,13)$\\
B.$x \in (-\infty,-19) \cup (-11,13]$\\
C.$x \in (-\infty,-19) \cup [-11,13)$\\
D.$x \in (-\infty,-19] \cup (-11,13)$\\
E.$x \in (-\infty,-19] \cup (-11,13]$\\
F.$x \in (-\infty,-19] \cup [-11,13)$\\
G.$x \in (-\infty,-19) \cup [-11,13]$\\
H.$x \in (-\infty,-19] \cup [-11,13]$
\testStop
\kluczStart
A
\kluczStop



\zadStart{Zadanie z Wikieł Z 1.62 b) moja wersja nr 894}

Rozwiązać nierówności $(x+19)(13-x)(x+12)\ge0$.
\zadStop
\rozwStart{Patryk Wirkus}{}
Miejsca zerowe naszego wielomianu to: $-19, 13, -12$.\\
Wielomian jest stopnia nieparzystego, ponadto znak współczynnika przy\linebreak najwyższej potędze x jest ujemny.\\ W związku z tym wykres wielomianu zaczyna się od lewej strony powyżej osi OX. A więc $$x \in (-\infty,-19) \cup (-12,13).$$
\rozwStop
\odpStart
$x \in (-\infty,-19) \cup (-12,13)$
\odpStop
\testStart
A.$x \in (-\infty,-19) \cup (-12,13)$\\
B.$x \in (-\infty,-19) \cup (-12,13]$\\
C.$x \in (-\infty,-19) \cup [-12,13)$\\
D.$x \in (-\infty,-19] \cup (-12,13)$\\
E.$x \in (-\infty,-19] \cup (-12,13]$\\
F.$x \in (-\infty,-19] \cup [-12,13)$\\
G.$x \in (-\infty,-19) \cup [-12,13]$\\
H.$x \in (-\infty,-19] \cup [-12,13]$
\testStop
\kluczStart
A
\kluczStop



\zadStart{Zadanie z Wikieł Z 1.62 b) moja wersja nr 895}

Rozwiązać nierówności $(x+19)(14-x)(x+1)\ge0$.
\zadStop
\rozwStart{Patryk Wirkus}{}
Miejsca zerowe naszego wielomianu to: $-19, 14, -1$.\\
Wielomian jest stopnia nieparzystego, ponadto znak współczynnika przy\linebreak najwyższej potędze x jest ujemny.\\ W związku z tym wykres wielomianu zaczyna się od lewej strony powyżej osi OX. A więc $$x \in (-\infty,-19) \cup (-1,14).$$
\rozwStop
\odpStart
$x \in (-\infty,-19) \cup (-1,14)$
\odpStop
\testStart
A.$x \in (-\infty,-19) \cup (-1,14)$\\
B.$x \in (-\infty,-19) \cup (-1,14]$\\
C.$x \in (-\infty,-19) \cup [-1,14)$\\
D.$x \in (-\infty,-19] \cup (-1,14)$\\
E.$x \in (-\infty,-19] \cup (-1,14]$\\
F.$x \in (-\infty,-19] \cup [-1,14)$\\
G.$x \in (-\infty,-19) \cup [-1,14]$\\
H.$x \in (-\infty,-19] \cup [-1,14]$
\testStop
\kluczStart
A
\kluczStop



\zadStart{Zadanie z Wikieł Z 1.62 b) moja wersja nr 896}

Rozwiązać nierówności $(x+19)(14-x)(x+2)\ge0$.
\zadStop
\rozwStart{Patryk Wirkus}{}
Miejsca zerowe naszego wielomianu to: $-19, 14, -2$.\\
Wielomian jest stopnia nieparzystego, ponadto znak współczynnika przy\linebreak najwyższej potędze x jest ujemny.\\ W związku z tym wykres wielomianu zaczyna się od lewej strony powyżej osi OX. A więc $$x \in (-\infty,-19) \cup (-2,14).$$
\rozwStop
\odpStart
$x \in (-\infty,-19) \cup (-2,14)$
\odpStop
\testStart
A.$x \in (-\infty,-19) \cup (-2,14)$\\
B.$x \in (-\infty,-19) \cup (-2,14]$\\
C.$x \in (-\infty,-19) \cup [-2,14)$\\
D.$x \in (-\infty,-19] \cup (-2,14)$\\
E.$x \in (-\infty,-19] \cup (-2,14]$\\
F.$x \in (-\infty,-19] \cup [-2,14)$\\
G.$x \in (-\infty,-19) \cup [-2,14]$\\
H.$x \in (-\infty,-19] \cup [-2,14]$
\testStop
\kluczStart
A
\kluczStop



\zadStart{Zadanie z Wikieł Z 1.62 b) moja wersja nr 897}

Rozwiązać nierówności $(x+19)(14-x)(x+3)\ge0$.
\zadStop
\rozwStart{Patryk Wirkus}{}
Miejsca zerowe naszego wielomianu to: $-19, 14, -3$.\\
Wielomian jest stopnia nieparzystego, ponadto znak współczynnika przy\linebreak najwyższej potędze x jest ujemny.\\ W związku z tym wykres wielomianu zaczyna się od lewej strony powyżej osi OX. A więc $$x \in (-\infty,-19) \cup (-3,14).$$
\rozwStop
\odpStart
$x \in (-\infty,-19) \cup (-3,14)$
\odpStop
\testStart
A.$x \in (-\infty,-19) \cup (-3,14)$\\
B.$x \in (-\infty,-19) \cup (-3,14]$\\
C.$x \in (-\infty,-19) \cup [-3,14)$\\
D.$x \in (-\infty,-19] \cup (-3,14)$\\
E.$x \in (-\infty,-19] \cup (-3,14]$\\
F.$x \in (-\infty,-19] \cup [-3,14)$\\
G.$x \in (-\infty,-19) \cup [-3,14]$\\
H.$x \in (-\infty,-19] \cup [-3,14]$
\testStop
\kluczStart
A
\kluczStop



\zadStart{Zadanie z Wikieł Z 1.62 b) moja wersja nr 898}

Rozwiązać nierówności $(x+19)(14-x)(x+4)\ge0$.
\zadStop
\rozwStart{Patryk Wirkus}{}
Miejsca zerowe naszego wielomianu to: $-19, 14, -4$.\\
Wielomian jest stopnia nieparzystego, ponadto znak współczynnika przy\linebreak najwyższej potędze x jest ujemny.\\ W związku z tym wykres wielomianu zaczyna się od lewej strony powyżej osi OX. A więc $$x \in (-\infty,-19) \cup (-4,14).$$
\rozwStop
\odpStart
$x \in (-\infty,-19) \cup (-4,14)$
\odpStop
\testStart
A.$x \in (-\infty,-19) \cup (-4,14)$\\
B.$x \in (-\infty,-19) \cup (-4,14]$\\
C.$x \in (-\infty,-19) \cup [-4,14)$\\
D.$x \in (-\infty,-19] \cup (-4,14)$\\
E.$x \in (-\infty,-19] \cup (-4,14]$\\
F.$x \in (-\infty,-19] \cup [-4,14)$\\
G.$x \in (-\infty,-19) \cup [-4,14]$\\
H.$x \in (-\infty,-19] \cup [-4,14]$
\testStop
\kluczStart
A
\kluczStop



\zadStart{Zadanie z Wikieł Z 1.62 b) moja wersja nr 899}

Rozwiązać nierówności $(x+19)(14-x)(x+5)\ge0$.
\zadStop
\rozwStart{Patryk Wirkus}{}
Miejsca zerowe naszego wielomianu to: $-19, 14, -5$.\\
Wielomian jest stopnia nieparzystego, ponadto znak współczynnika przy\linebreak najwyższej potędze x jest ujemny.\\ W związku z tym wykres wielomianu zaczyna się od lewej strony powyżej osi OX. A więc $$x \in (-\infty,-19) \cup (-5,14).$$
\rozwStop
\odpStart
$x \in (-\infty,-19) \cup (-5,14)$
\odpStop
\testStart
A.$x \in (-\infty,-19) \cup (-5,14)$\\
B.$x \in (-\infty,-19) \cup (-5,14]$\\
C.$x \in (-\infty,-19) \cup [-5,14)$\\
D.$x \in (-\infty,-19] \cup (-5,14)$\\
E.$x \in (-\infty,-19] \cup (-5,14]$\\
F.$x \in (-\infty,-19] \cup [-5,14)$\\
G.$x \in (-\infty,-19) \cup [-5,14]$\\
H.$x \in (-\infty,-19] \cup [-5,14]$
\testStop
\kluczStart
A
\kluczStop



\zadStart{Zadanie z Wikieł Z 1.62 b) moja wersja nr 900}

Rozwiązać nierówności $(x+19)(14-x)(x+6)\ge0$.
\zadStop
\rozwStart{Patryk Wirkus}{}
Miejsca zerowe naszego wielomianu to: $-19, 14, -6$.\\
Wielomian jest stopnia nieparzystego, ponadto znak współczynnika przy\linebreak najwyższej potędze x jest ujemny.\\ W związku z tym wykres wielomianu zaczyna się od lewej strony powyżej osi OX. A więc $$x \in (-\infty,-19) \cup (-6,14).$$
\rozwStop
\odpStart
$x \in (-\infty,-19) \cup (-6,14)$
\odpStop
\testStart
A.$x \in (-\infty,-19) \cup (-6,14)$\\
B.$x \in (-\infty,-19) \cup (-6,14]$\\
C.$x \in (-\infty,-19) \cup [-6,14)$\\
D.$x \in (-\infty,-19] \cup (-6,14)$\\
E.$x \in (-\infty,-19] \cup (-6,14]$\\
F.$x \in (-\infty,-19] \cup [-6,14)$\\
G.$x \in (-\infty,-19) \cup [-6,14]$\\
H.$x \in (-\infty,-19] \cup [-6,14]$
\testStop
\kluczStart
A
\kluczStop



\zadStart{Zadanie z Wikieł Z 1.62 b) moja wersja nr 901}

Rozwiązać nierówności $(x+19)(14-x)(x+7)\ge0$.
\zadStop
\rozwStart{Patryk Wirkus}{}
Miejsca zerowe naszego wielomianu to: $-19, 14, -7$.\\
Wielomian jest stopnia nieparzystego, ponadto znak współczynnika przy\linebreak najwyższej potędze x jest ujemny.\\ W związku z tym wykres wielomianu zaczyna się od lewej strony powyżej osi OX. A więc $$x \in (-\infty,-19) \cup (-7,14).$$
\rozwStop
\odpStart
$x \in (-\infty,-19) \cup (-7,14)$
\odpStop
\testStart
A.$x \in (-\infty,-19) \cup (-7,14)$\\
B.$x \in (-\infty,-19) \cup (-7,14]$\\
C.$x \in (-\infty,-19) \cup [-7,14)$\\
D.$x \in (-\infty,-19] \cup (-7,14)$\\
E.$x \in (-\infty,-19] \cup (-7,14]$\\
F.$x \in (-\infty,-19] \cup [-7,14)$\\
G.$x \in (-\infty,-19) \cup [-7,14]$\\
H.$x \in (-\infty,-19] \cup [-7,14]$
\testStop
\kluczStart
A
\kluczStop



\zadStart{Zadanie z Wikieł Z 1.62 b) moja wersja nr 902}

Rozwiązać nierówności $(x+19)(14-x)(x+8)\ge0$.
\zadStop
\rozwStart{Patryk Wirkus}{}
Miejsca zerowe naszego wielomianu to: $-19, 14, -8$.\\
Wielomian jest stopnia nieparzystego, ponadto znak współczynnika przy\linebreak najwyższej potędze x jest ujemny.\\ W związku z tym wykres wielomianu zaczyna się od lewej strony powyżej osi OX. A więc $$x \in (-\infty,-19) \cup (-8,14).$$
\rozwStop
\odpStart
$x \in (-\infty,-19) \cup (-8,14)$
\odpStop
\testStart
A.$x \in (-\infty,-19) \cup (-8,14)$\\
B.$x \in (-\infty,-19) \cup (-8,14]$\\
C.$x \in (-\infty,-19) \cup [-8,14)$\\
D.$x \in (-\infty,-19] \cup (-8,14)$\\
E.$x \in (-\infty,-19] \cup (-8,14]$\\
F.$x \in (-\infty,-19] \cup [-8,14)$\\
G.$x \in (-\infty,-19) \cup [-8,14]$\\
H.$x \in (-\infty,-19] \cup [-8,14]$
\testStop
\kluczStart
A
\kluczStop



\zadStart{Zadanie z Wikieł Z 1.62 b) moja wersja nr 903}

Rozwiązać nierówności $(x+19)(14-x)(x+9)\ge0$.
\zadStop
\rozwStart{Patryk Wirkus}{}
Miejsca zerowe naszego wielomianu to: $-19, 14, -9$.\\
Wielomian jest stopnia nieparzystego, ponadto znak współczynnika przy\linebreak najwyższej potędze x jest ujemny.\\ W związku z tym wykres wielomianu zaczyna się od lewej strony powyżej osi OX. A więc $$x \in (-\infty,-19) \cup (-9,14).$$
\rozwStop
\odpStart
$x \in (-\infty,-19) \cup (-9,14)$
\odpStop
\testStart
A.$x \in (-\infty,-19) \cup (-9,14)$\\
B.$x \in (-\infty,-19) \cup (-9,14]$\\
C.$x \in (-\infty,-19) \cup [-9,14)$\\
D.$x \in (-\infty,-19] \cup (-9,14)$\\
E.$x \in (-\infty,-19] \cup (-9,14]$\\
F.$x \in (-\infty,-19] \cup [-9,14)$\\
G.$x \in (-\infty,-19) \cup [-9,14]$\\
H.$x \in (-\infty,-19] \cup [-9,14]$
\testStop
\kluczStart
A
\kluczStop



\zadStart{Zadanie z Wikieł Z 1.62 b) moja wersja nr 904}

Rozwiązać nierówności $(x+19)(14-x)(x+10)\ge0$.
\zadStop
\rozwStart{Patryk Wirkus}{}
Miejsca zerowe naszego wielomianu to: $-19, 14, -10$.\\
Wielomian jest stopnia nieparzystego, ponadto znak współczynnika przy\linebreak najwyższej potędze x jest ujemny.\\ W związku z tym wykres wielomianu zaczyna się od lewej strony powyżej osi OX. A więc $$x \in (-\infty,-19) \cup (-10,14).$$
\rozwStop
\odpStart
$x \in (-\infty,-19) \cup (-10,14)$
\odpStop
\testStart
A.$x \in (-\infty,-19) \cup (-10,14)$\\
B.$x \in (-\infty,-19) \cup (-10,14]$\\
C.$x \in (-\infty,-19) \cup [-10,14)$\\
D.$x \in (-\infty,-19] \cup (-10,14)$\\
E.$x \in (-\infty,-19] \cup (-10,14]$\\
F.$x \in (-\infty,-19] \cup [-10,14)$\\
G.$x \in (-\infty,-19) \cup [-10,14]$\\
H.$x \in (-\infty,-19] \cup [-10,14]$
\testStop
\kluczStart
A
\kluczStop



\zadStart{Zadanie z Wikieł Z 1.62 b) moja wersja nr 905}

Rozwiązać nierówności $(x+19)(14-x)(x+11)\ge0$.
\zadStop
\rozwStart{Patryk Wirkus}{}
Miejsca zerowe naszego wielomianu to: $-19, 14, -11$.\\
Wielomian jest stopnia nieparzystego, ponadto znak współczynnika przy\linebreak najwyższej potędze x jest ujemny.\\ W związku z tym wykres wielomianu zaczyna się od lewej strony powyżej osi OX. A więc $$x \in (-\infty,-19) \cup (-11,14).$$
\rozwStop
\odpStart
$x \in (-\infty,-19) \cup (-11,14)$
\odpStop
\testStart
A.$x \in (-\infty,-19) \cup (-11,14)$\\
B.$x \in (-\infty,-19) \cup (-11,14]$\\
C.$x \in (-\infty,-19) \cup [-11,14)$\\
D.$x \in (-\infty,-19] \cup (-11,14)$\\
E.$x \in (-\infty,-19] \cup (-11,14]$\\
F.$x \in (-\infty,-19] \cup [-11,14)$\\
G.$x \in (-\infty,-19) \cup [-11,14]$\\
H.$x \in (-\infty,-19] \cup [-11,14]$
\testStop
\kluczStart
A
\kluczStop



\zadStart{Zadanie z Wikieł Z 1.62 b) moja wersja nr 906}

Rozwiązać nierówności $(x+19)(14-x)(x+12)\ge0$.
\zadStop
\rozwStart{Patryk Wirkus}{}
Miejsca zerowe naszego wielomianu to: $-19, 14, -12$.\\
Wielomian jest stopnia nieparzystego, ponadto znak współczynnika przy\linebreak najwyższej potędze x jest ujemny.\\ W związku z tym wykres wielomianu zaczyna się od lewej strony powyżej osi OX. A więc $$x \in (-\infty,-19) \cup (-12,14).$$
\rozwStop
\odpStart
$x \in (-\infty,-19) \cup (-12,14)$
\odpStop
\testStart
A.$x \in (-\infty,-19) \cup (-12,14)$\\
B.$x \in (-\infty,-19) \cup (-12,14]$\\
C.$x \in (-\infty,-19) \cup [-12,14)$\\
D.$x \in (-\infty,-19] \cup (-12,14)$\\
E.$x \in (-\infty,-19] \cup (-12,14]$\\
F.$x \in (-\infty,-19] \cup [-12,14)$\\
G.$x \in (-\infty,-19) \cup [-12,14]$\\
H.$x \in (-\infty,-19] \cup [-12,14]$
\testStop
\kluczStart
A
\kluczStop



\zadStart{Zadanie z Wikieł Z 1.62 b) moja wersja nr 907}

Rozwiązać nierówności $(x+19)(14-x)(x+13)\ge0$.
\zadStop
\rozwStart{Patryk Wirkus}{}
Miejsca zerowe naszego wielomianu to: $-19, 14, -13$.\\
Wielomian jest stopnia nieparzystego, ponadto znak współczynnika przy\linebreak najwyższej potędze x jest ujemny.\\ W związku z tym wykres wielomianu zaczyna się od lewej strony powyżej osi OX. A więc $$x \in (-\infty,-19) \cup (-13,14).$$
\rozwStop
\odpStart
$x \in (-\infty,-19) \cup (-13,14)$
\odpStop
\testStart
A.$x \in (-\infty,-19) \cup (-13,14)$\\
B.$x \in (-\infty,-19) \cup (-13,14]$\\
C.$x \in (-\infty,-19) \cup [-13,14)$\\
D.$x \in (-\infty,-19] \cup (-13,14)$\\
E.$x \in (-\infty,-19] \cup (-13,14]$\\
F.$x \in (-\infty,-19] \cup [-13,14)$\\
G.$x \in (-\infty,-19) \cup [-13,14]$\\
H.$x \in (-\infty,-19] \cup [-13,14]$
\testStop
\kluczStart
A
\kluczStop



\zadStart{Zadanie z Wikieł Z 1.62 b) moja wersja nr 908}

Rozwiązać nierówności $(x+19)(15-x)(x+1)\ge0$.
\zadStop
\rozwStart{Patryk Wirkus}{}
Miejsca zerowe naszego wielomianu to: $-19, 15, -1$.\\
Wielomian jest stopnia nieparzystego, ponadto znak współczynnika przy\linebreak najwyższej potędze x jest ujemny.\\ W związku z tym wykres wielomianu zaczyna się od lewej strony powyżej osi OX. A więc $$x \in (-\infty,-19) \cup (-1,15).$$
\rozwStop
\odpStart
$x \in (-\infty,-19) \cup (-1,15)$
\odpStop
\testStart
A.$x \in (-\infty,-19) \cup (-1,15)$\\
B.$x \in (-\infty,-19) \cup (-1,15]$\\
C.$x \in (-\infty,-19) \cup [-1,15)$\\
D.$x \in (-\infty,-19] \cup (-1,15)$\\
E.$x \in (-\infty,-19] \cup (-1,15]$\\
F.$x \in (-\infty,-19] \cup [-1,15)$\\
G.$x \in (-\infty,-19) \cup [-1,15]$\\
H.$x \in (-\infty,-19] \cup [-1,15]$
\testStop
\kluczStart
A
\kluczStop



\zadStart{Zadanie z Wikieł Z 1.62 b) moja wersja nr 909}

Rozwiązać nierówności $(x+19)(15-x)(x+2)\ge0$.
\zadStop
\rozwStart{Patryk Wirkus}{}
Miejsca zerowe naszego wielomianu to: $-19, 15, -2$.\\
Wielomian jest stopnia nieparzystego, ponadto znak współczynnika przy\linebreak najwyższej potędze x jest ujemny.\\ W związku z tym wykres wielomianu zaczyna się od lewej strony powyżej osi OX. A więc $$x \in (-\infty,-19) \cup (-2,15).$$
\rozwStop
\odpStart
$x \in (-\infty,-19) \cup (-2,15)$
\odpStop
\testStart
A.$x \in (-\infty,-19) \cup (-2,15)$\\
B.$x \in (-\infty,-19) \cup (-2,15]$\\
C.$x \in (-\infty,-19) \cup [-2,15)$\\
D.$x \in (-\infty,-19] \cup (-2,15)$\\
E.$x \in (-\infty,-19] \cup (-2,15]$\\
F.$x \in (-\infty,-19] \cup [-2,15)$\\
G.$x \in (-\infty,-19) \cup [-2,15]$\\
H.$x \in (-\infty,-19] \cup [-2,15]$
\testStop
\kluczStart
A
\kluczStop



\zadStart{Zadanie z Wikieł Z 1.62 b) moja wersja nr 910}

Rozwiązać nierówności $(x+19)(15-x)(x+3)\ge0$.
\zadStop
\rozwStart{Patryk Wirkus}{}
Miejsca zerowe naszego wielomianu to: $-19, 15, -3$.\\
Wielomian jest stopnia nieparzystego, ponadto znak współczynnika przy\linebreak najwyższej potędze x jest ujemny.\\ W związku z tym wykres wielomianu zaczyna się od lewej strony powyżej osi OX. A więc $$x \in (-\infty,-19) \cup (-3,15).$$
\rozwStop
\odpStart
$x \in (-\infty,-19) \cup (-3,15)$
\odpStop
\testStart
A.$x \in (-\infty,-19) \cup (-3,15)$\\
B.$x \in (-\infty,-19) \cup (-3,15]$\\
C.$x \in (-\infty,-19) \cup [-3,15)$\\
D.$x \in (-\infty,-19] \cup (-3,15)$\\
E.$x \in (-\infty,-19] \cup (-3,15]$\\
F.$x \in (-\infty,-19] \cup [-3,15)$\\
G.$x \in (-\infty,-19) \cup [-3,15]$\\
H.$x \in (-\infty,-19] \cup [-3,15]$
\testStop
\kluczStart
A
\kluczStop



\zadStart{Zadanie z Wikieł Z 1.62 b) moja wersja nr 911}

Rozwiązać nierówności $(x+19)(15-x)(x+4)\ge0$.
\zadStop
\rozwStart{Patryk Wirkus}{}
Miejsca zerowe naszego wielomianu to: $-19, 15, -4$.\\
Wielomian jest stopnia nieparzystego, ponadto znak współczynnika przy\linebreak najwyższej potędze x jest ujemny.\\ W związku z tym wykres wielomianu zaczyna się od lewej strony powyżej osi OX. A więc $$x \in (-\infty,-19) \cup (-4,15).$$
\rozwStop
\odpStart
$x \in (-\infty,-19) \cup (-4,15)$
\odpStop
\testStart
A.$x \in (-\infty,-19) \cup (-4,15)$\\
B.$x \in (-\infty,-19) \cup (-4,15]$\\
C.$x \in (-\infty,-19) \cup [-4,15)$\\
D.$x \in (-\infty,-19] \cup (-4,15)$\\
E.$x \in (-\infty,-19] \cup (-4,15]$\\
F.$x \in (-\infty,-19] \cup [-4,15)$\\
G.$x \in (-\infty,-19) \cup [-4,15]$\\
H.$x \in (-\infty,-19] \cup [-4,15]$
\testStop
\kluczStart
A
\kluczStop



\zadStart{Zadanie z Wikieł Z 1.62 b) moja wersja nr 912}

Rozwiązać nierówności $(x+19)(15-x)(x+5)\ge0$.
\zadStop
\rozwStart{Patryk Wirkus}{}
Miejsca zerowe naszego wielomianu to: $-19, 15, -5$.\\
Wielomian jest stopnia nieparzystego, ponadto znak współczynnika przy\linebreak najwyższej potędze x jest ujemny.\\ W związku z tym wykres wielomianu zaczyna się od lewej strony powyżej osi OX. A więc $$x \in (-\infty,-19) \cup (-5,15).$$
\rozwStop
\odpStart
$x \in (-\infty,-19) \cup (-5,15)$
\odpStop
\testStart
A.$x \in (-\infty,-19) \cup (-5,15)$\\
B.$x \in (-\infty,-19) \cup (-5,15]$\\
C.$x \in (-\infty,-19) \cup [-5,15)$\\
D.$x \in (-\infty,-19] \cup (-5,15)$\\
E.$x \in (-\infty,-19] \cup (-5,15]$\\
F.$x \in (-\infty,-19] \cup [-5,15)$\\
G.$x \in (-\infty,-19) \cup [-5,15]$\\
H.$x \in (-\infty,-19] \cup [-5,15]$
\testStop
\kluczStart
A
\kluczStop



\zadStart{Zadanie z Wikieł Z 1.62 b) moja wersja nr 913}

Rozwiązać nierówności $(x+19)(15-x)(x+6)\ge0$.
\zadStop
\rozwStart{Patryk Wirkus}{}
Miejsca zerowe naszego wielomianu to: $-19, 15, -6$.\\
Wielomian jest stopnia nieparzystego, ponadto znak współczynnika przy\linebreak najwyższej potędze x jest ujemny.\\ W związku z tym wykres wielomianu zaczyna się od lewej strony powyżej osi OX. A więc $$x \in (-\infty,-19) \cup (-6,15).$$
\rozwStop
\odpStart
$x \in (-\infty,-19) \cup (-6,15)$
\odpStop
\testStart
A.$x \in (-\infty,-19) \cup (-6,15)$\\
B.$x \in (-\infty,-19) \cup (-6,15]$\\
C.$x \in (-\infty,-19) \cup [-6,15)$\\
D.$x \in (-\infty,-19] \cup (-6,15)$\\
E.$x \in (-\infty,-19] \cup (-6,15]$\\
F.$x \in (-\infty,-19] \cup [-6,15)$\\
G.$x \in (-\infty,-19) \cup [-6,15]$\\
H.$x \in (-\infty,-19] \cup [-6,15]$
\testStop
\kluczStart
A
\kluczStop



\zadStart{Zadanie z Wikieł Z 1.62 b) moja wersja nr 914}

Rozwiązać nierówności $(x+19)(15-x)(x+7)\ge0$.
\zadStop
\rozwStart{Patryk Wirkus}{}
Miejsca zerowe naszego wielomianu to: $-19, 15, -7$.\\
Wielomian jest stopnia nieparzystego, ponadto znak współczynnika przy\linebreak najwyższej potędze x jest ujemny.\\ W związku z tym wykres wielomianu zaczyna się od lewej strony powyżej osi OX. A więc $$x \in (-\infty,-19) \cup (-7,15).$$
\rozwStop
\odpStart
$x \in (-\infty,-19) \cup (-7,15)$
\odpStop
\testStart
A.$x \in (-\infty,-19) \cup (-7,15)$\\
B.$x \in (-\infty,-19) \cup (-7,15]$\\
C.$x \in (-\infty,-19) \cup [-7,15)$\\
D.$x \in (-\infty,-19] \cup (-7,15)$\\
E.$x \in (-\infty,-19] \cup (-7,15]$\\
F.$x \in (-\infty,-19] \cup [-7,15)$\\
G.$x \in (-\infty,-19) \cup [-7,15]$\\
H.$x \in (-\infty,-19] \cup [-7,15]$
\testStop
\kluczStart
A
\kluczStop



\zadStart{Zadanie z Wikieł Z 1.62 b) moja wersja nr 915}

Rozwiązać nierówności $(x+19)(15-x)(x+8)\ge0$.
\zadStop
\rozwStart{Patryk Wirkus}{}
Miejsca zerowe naszego wielomianu to: $-19, 15, -8$.\\
Wielomian jest stopnia nieparzystego, ponadto znak współczynnika przy\linebreak najwyższej potędze x jest ujemny.\\ W związku z tym wykres wielomianu zaczyna się od lewej strony powyżej osi OX. A więc $$x \in (-\infty,-19) \cup (-8,15).$$
\rozwStop
\odpStart
$x \in (-\infty,-19) \cup (-8,15)$
\odpStop
\testStart
A.$x \in (-\infty,-19) \cup (-8,15)$\\
B.$x \in (-\infty,-19) \cup (-8,15]$\\
C.$x \in (-\infty,-19) \cup [-8,15)$\\
D.$x \in (-\infty,-19] \cup (-8,15)$\\
E.$x \in (-\infty,-19] \cup (-8,15]$\\
F.$x \in (-\infty,-19] \cup [-8,15)$\\
G.$x \in (-\infty,-19) \cup [-8,15]$\\
H.$x \in (-\infty,-19] \cup [-8,15]$
\testStop
\kluczStart
A
\kluczStop



\zadStart{Zadanie z Wikieł Z 1.62 b) moja wersja nr 916}

Rozwiązać nierówności $(x+19)(15-x)(x+9)\ge0$.
\zadStop
\rozwStart{Patryk Wirkus}{}
Miejsca zerowe naszego wielomianu to: $-19, 15, -9$.\\
Wielomian jest stopnia nieparzystego, ponadto znak współczynnika przy\linebreak najwyższej potędze x jest ujemny.\\ W związku z tym wykres wielomianu zaczyna się od lewej strony powyżej osi OX. A więc $$x \in (-\infty,-19) \cup (-9,15).$$
\rozwStop
\odpStart
$x \in (-\infty,-19) \cup (-9,15)$
\odpStop
\testStart
A.$x \in (-\infty,-19) \cup (-9,15)$\\
B.$x \in (-\infty,-19) \cup (-9,15]$\\
C.$x \in (-\infty,-19) \cup [-9,15)$\\
D.$x \in (-\infty,-19] \cup (-9,15)$\\
E.$x \in (-\infty,-19] \cup (-9,15]$\\
F.$x \in (-\infty,-19] \cup [-9,15)$\\
G.$x \in (-\infty,-19) \cup [-9,15]$\\
H.$x \in (-\infty,-19] \cup [-9,15]$
\testStop
\kluczStart
A
\kluczStop



\zadStart{Zadanie z Wikieł Z 1.62 b) moja wersja nr 917}

Rozwiązać nierówności $(x+19)(15-x)(x+10)\ge0$.
\zadStop
\rozwStart{Patryk Wirkus}{}
Miejsca zerowe naszego wielomianu to: $-19, 15, -10$.\\
Wielomian jest stopnia nieparzystego, ponadto znak współczynnika przy\linebreak najwyższej potędze x jest ujemny.\\ W związku z tym wykres wielomianu zaczyna się od lewej strony powyżej osi OX. A więc $$x \in (-\infty,-19) \cup (-10,15).$$
\rozwStop
\odpStart
$x \in (-\infty,-19) \cup (-10,15)$
\odpStop
\testStart
A.$x \in (-\infty,-19) \cup (-10,15)$\\
B.$x \in (-\infty,-19) \cup (-10,15]$\\
C.$x \in (-\infty,-19) \cup [-10,15)$\\
D.$x \in (-\infty,-19] \cup (-10,15)$\\
E.$x \in (-\infty,-19] \cup (-10,15]$\\
F.$x \in (-\infty,-19] \cup [-10,15)$\\
G.$x \in (-\infty,-19) \cup [-10,15]$\\
H.$x \in (-\infty,-19] \cup [-10,15]$
\testStop
\kluczStart
A
\kluczStop



\zadStart{Zadanie z Wikieł Z 1.62 b) moja wersja nr 918}

Rozwiązać nierówności $(x+19)(15-x)(x+11)\ge0$.
\zadStop
\rozwStart{Patryk Wirkus}{}
Miejsca zerowe naszego wielomianu to: $-19, 15, -11$.\\
Wielomian jest stopnia nieparzystego, ponadto znak współczynnika przy\linebreak najwyższej potędze x jest ujemny.\\ W związku z tym wykres wielomianu zaczyna się od lewej strony powyżej osi OX. A więc $$x \in (-\infty,-19) \cup (-11,15).$$
\rozwStop
\odpStart
$x \in (-\infty,-19) \cup (-11,15)$
\odpStop
\testStart
A.$x \in (-\infty,-19) \cup (-11,15)$\\
B.$x \in (-\infty,-19) \cup (-11,15]$\\
C.$x \in (-\infty,-19) \cup [-11,15)$\\
D.$x \in (-\infty,-19] \cup (-11,15)$\\
E.$x \in (-\infty,-19] \cup (-11,15]$\\
F.$x \in (-\infty,-19] \cup [-11,15)$\\
G.$x \in (-\infty,-19) \cup [-11,15]$\\
H.$x \in (-\infty,-19] \cup [-11,15]$
\testStop
\kluczStart
A
\kluczStop



\zadStart{Zadanie z Wikieł Z 1.62 b) moja wersja nr 919}

Rozwiązać nierówności $(x+19)(15-x)(x+12)\ge0$.
\zadStop
\rozwStart{Patryk Wirkus}{}
Miejsca zerowe naszego wielomianu to: $-19, 15, -12$.\\
Wielomian jest stopnia nieparzystego, ponadto znak współczynnika przy\linebreak najwyższej potędze x jest ujemny.\\ W związku z tym wykres wielomianu zaczyna się od lewej strony powyżej osi OX. A więc $$x \in (-\infty,-19) \cup (-12,15).$$
\rozwStop
\odpStart
$x \in (-\infty,-19) \cup (-12,15)$
\odpStop
\testStart
A.$x \in (-\infty,-19) \cup (-12,15)$\\
B.$x \in (-\infty,-19) \cup (-12,15]$\\
C.$x \in (-\infty,-19) \cup [-12,15)$\\
D.$x \in (-\infty,-19] \cup (-12,15)$\\
E.$x \in (-\infty,-19] \cup (-12,15]$\\
F.$x \in (-\infty,-19] \cup [-12,15)$\\
G.$x \in (-\infty,-19) \cup [-12,15]$\\
H.$x \in (-\infty,-19] \cup [-12,15]$
\testStop
\kluczStart
A
\kluczStop



\zadStart{Zadanie z Wikieł Z 1.62 b) moja wersja nr 920}

Rozwiązać nierówności $(x+19)(15-x)(x+13)\ge0$.
\zadStop
\rozwStart{Patryk Wirkus}{}
Miejsca zerowe naszego wielomianu to: $-19, 15, -13$.\\
Wielomian jest stopnia nieparzystego, ponadto znak współczynnika przy\linebreak najwyższej potędze x jest ujemny.\\ W związku z tym wykres wielomianu zaczyna się od lewej strony powyżej osi OX. A więc $$x \in (-\infty,-19) \cup (-13,15).$$
\rozwStop
\odpStart
$x \in (-\infty,-19) \cup (-13,15)$
\odpStop
\testStart
A.$x \in (-\infty,-19) \cup (-13,15)$\\
B.$x \in (-\infty,-19) \cup (-13,15]$\\
C.$x \in (-\infty,-19) \cup [-13,15)$\\
D.$x \in (-\infty,-19] \cup (-13,15)$\\
E.$x \in (-\infty,-19] \cup (-13,15]$\\
F.$x \in (-\infty,-19] \cup [-13,15)$\\
G.$x \in (-\infty,-19) \cup [-13,15]$\\
H.$x \in (-\infty,-19] \cup [-13,15]$
\testStop
\kluczStart
A
\kluczStop



\zadStart{Zadanie z Wikieł Z 1.62 b) moja wersja nr 921}

Rozwiązać nierówności $(x+19)(15-x)(x+14)\ge0$.
\zadStop
\rozwStart{Patryk Wirkus}{}
Miejsca zerowe naszego wielomianu to: $-19, 15, -14$.\\
Wielomian jest stopnia nieparzystego, ponadto znak współczynnika przy\linebreak najwyższej potędze x jest ujemny.\\ W związku z tym wykres wielomianu zaczyna się od lewej strony powyżej osi OX. A więc $$x \in (-\infty,-19) \cup (-14,15).$$
\rozwStop
\odpStart
$x \in (-\infty,-19) \cup (-14,15)$
\odpStop
\testStart
A.$x \in (-\infty,-19) \cup (-14,15)$\\
B.$x \in (-\infty,-19) \cup (-14,15]$\\
C.$x \in (-\infty,-19) \cup [-14,15)$\\
D.$x \in (-\infty,-19] \cup (-14,15)$\\
E.$x \in (-\infty,-19] \cup (-14,15]$\\
F.$x \in (-\infty,-19] \cup [-14,15)$\\
G.$x \in (-\infty,-19) \cup [-14,15]$\\
H.$x \in (-\infty,-19] \cup [-14,15]$
\testStop
\kluczStart
A
\kluczStop



\zadStart{Zadanie z Wikieł Z 1.62 b) moja wersja nr 922}

Rozwiązać nierówności $(x+19)(16-x)(x+1)\ge0$.
\zadStop
\rozwStart{Patryk Wirkus}{}
Miejsca zerowe naszego wielomianu to: $-19, 16, -1$.\\
Wielomian jest stopnia nieparzystego, ponadto znak współczynnika przy\linebreak najwyższej potędze x jest ujemny.\\ W związku z tym wykres wielomianu zaczyna się od lewej strony powyżej osi OX. A więc $$x \in (-\infty,-19) \cup (-1,16).$$
\rozwStop
\odpStart
$x \in (-\infty,-19) \cup (-1,16)$
\odpStop
\testStart
A.$x \in (-\infty,-19) \cup (-1,16)$\\
B.$x \in (-\infty,-19) \cup (-1,16]$\\
C.$x \in (-\infty,-19) \cup [-1,16)$\\
D.$x \in (-\infty,-19] \cup (-1,16)$\\
E.$x \in (-\infty,-19] \cup (-1,16]$\\
F.$x \in (-\infty,-19] \cup [-1,16)$\\
G.$x \in (-\infty,-19) \cup [-1,16]$\\
H.$x \in (-\infty,-19] \cup [-1,16]$
\testStop
\kluczStart
A
\kluczStop



\zadStart{Zadanie z Wikieł Z 1.62 b) moja wersja nr 923}

Rozwiązać nierówności $(x+19)(16-x)(x+2)\ge0$.
\zadStop
\rozwStart{Patryk Wirkus}{}
Miejsca zerowe naszego wielomianu to: $-19, 16, -2$.\\
Wielomian jest stopnia nieparzystego, ponadto znak współczynnika przy\linebreak najwyższej potędze x jest ujemny.\\ W związku z tym wykres wielomianu zaczyna się od lewej strony powyżej osi OX. A więc $$x \in (-\infty,-19) \cup (-2,16).$$
\rozwStop
\odpStart
$x \in (-\infty,-19) \cup (-2,16)$
\odpStop
\testStart
A.$x \in (-\infty,-19) \cup (-2,16)$\\
B.$x \in (-\infty,-19) \cup (-2,16]$\\
C.$x \in (-\infty,-19) \cup [-2,16)$\\
D.$x \in (-\infty,-19] \cup (-2,16)$\\
E.$x \in (-\infty,-19] \cup (-2,16]$\\
F.$x \in (-\infty,-19] \cup [-2,16)$\\
G.$x \in (-\infty,-19) \cup [-2,16]$\\
H.$x \in (-\infty,-19] \cup [-2,16]$
\testStop
\kluczStart
A
\kluczStop



\zadStart{Zadanie z Wikieł Z 1.62 b) moja wersja nr 924}

Rozwiązać nierówności $(x+19)(16-x)(x+3)\ge0$.
\zadStop
\rozwStart{Patryk Wirkus}{}
Miejsca zerowe naszego wielomianu to: $-19, 16, -3$.\\
Wielomian jest stopnia nieparzystego, ponadto znak współczynnika przy\linebreak najwyższej potędze x jest ujemny.\\ W związku z tym wykres wielomianu zaczyna się od lewej strony powyżej osi OX. A więc $$x \in (-\infty,-19) \cup (-3,16).$$
\rozwStop
\odpStart
$x \in (-\infty,-19) \cup (-3,16)$
\odpStop
\testStart
A.$x \in (-\infty,-19) \cup (-3,16)$\\
B.$x \in (-\infty,-19) \cup (-3,16]$\\
C.$x \in (-\infty,-19) \cup [-3,16)$\\
D.$x \in (-\infty,-19] \cup (-3,16)$\\
E.$x \in (-\infty,-19] \cup (-3,16]$\\
F.$x \in (-\infty,-19] \cup [-3,16)$\\
G.$x \in (-\infty,-19) \cup [-3,16]$\\
H.$x \in (-\infty,-19] \cup [-3,16]$
\testStop
\kluczStart
A
\kluczStop



\zadStart{Zadanie z Wikieł Z 1.62 b) moja wersja nr 925}

Rozwiązać nierówności $(x+19)(16-x)(x+4)\ge0$.
\zadStop
\rozwStart{Patryk Wirkus}{}
Miejsca zerowe naszego wielomianu to: $-19, 16, -4$.\\
Wielomian jest stopnia nieparzystego, ponadto znak współczynnika przy\linebreak najwyższej potędze x jest ujemny.\\ W związku z tym wykres wielomianu zaczyna się od lewej strony powyżej osi OX. A więc $$x \in (-\infty,-19) \cup (-4,16).$$
\rozwStop
\odpStart
$x \in (-\infty,-19) \cup (-4,16)$
\odpStop
\testStart
A.$x \in (-\infty,-19) \cup (-4,16)$\\
B.$x \in (-\infty,-19) \cup (-4,16]$\\
C.$x \in (-\infty,-19) \cup [-4,16)$\\
D.$x \in (-\infty,-19] \cup (-4,16)$\\
E.$x \in (-\infty,-19] \cup (-4,16]$\\
F.$x \in (-\infty,-19] \cup [-4,16)$\\
G.$x \in (-\infty,-19) \cup [-4,16]$\\
H.$x \in (-\infty,-19] \cup [-4,16]$
\testStop
\kluczStart
A
\kluczStop



\zadStart{Zadanie z Wikieł Z 1.62 b) moja wersja nr 926}

Rozwiązać nierówności $(x+19)(16-x)(x+5)\ge0$.
\zadStop
\rozwStart{Patryk Wirkus}{}
Miejsca zerowe naszego wielomianu to: $-19, 16, -5$.\\
Wielomian jest stopnia nieparzystego, ponadto znak współczynnika przy\linebreak najwyższej potędze x jest ujemny.\\ W związku z tym wykres wielomianu zaczyna się od lewej strony powyżej osi OX. A więc $$x \in (-\infty,-19) \cup (-5,16).$$
\rozwStop
\odpStart
$x \in (-\infty,-19) \cup (-5,16)$
\odpStop
\testStart
A.$x \in (-\infty,-19) \cup (-5,16)$\\
B.$x \in (-\infty,-19) \cup (-5,16]$\\
C.$x \in (-\infty,-19) \cup [-5,16)$\\
D.$x \in (-\infty,-19] \cup (-5,16)$\\
E.$x \in (-\infty,-19] \cup (-5,16]$\\
F.$x \in (-\infty,-19] \cup [-5,16)$\\
G.$x \in (-\infty,-19) \cup [-5,16]$\\
H.$x \in (-\infty,-19] \cup [-5,16]$
\testStop
\kluczStart
A
\kluczStop



\zadStart{Zadanie z Wikieł Z 1.62 b) moja wersja nr 927}

Rozwiązać nierówności $(x+19)(16-x)(x+6)\ge0$.
\zadStop
\rozwStart{Patryk Wirkus}{}
Miejsca zerowe naszego wielomianu to: $-19, 16, -6$.\\
Wielomian jest stopnia nieparzystego, ponadto znak współczynnika przy\linebreak najwyższej potędze x jest ujemny.\\ W związku z tym wykres wielomianu zaczyna się od lewej strony powyżej osi OX. A więc $$x \in (-\infty,-19) \cup (-6,16).$$
\rozwStop
\odpStart
$x \in (-\infty,-19) \cup (-6,16)$
\odpStop
\testStart
A.$x \in (-\infty,-19) \cup (-6,16)$\\
B.$x \in (-\infty,-19) \cup (-6,16]$\\
C.$x \in (-\infty,-19) \cup [-6,16)$\\
D.$x \in (-\infty,-19] \cup (-6,16)$\\
E.$x \in (-\infty,-19] \cup (-6,16]$\\
F.$x \in (-\infty,-19] \cup [-6,16)$\\
G.$x \in (-\infty,-19) \cup [-6,16]$\\
H.$x \in (-\infty,-19] \cup [-6,16]$
\testStop
\kluczStart
A
\kluczStop



\zadStart{Zadanie z Wikieł Z 1.62 b) moja wersja nr 928}

Rozwiązać nierówności $(x+19)(16-x)(x+7)\ge0$.
\zadStop
\rozwStart{Patryk Wirkus}{}
Miejsca zerowe naszego wielomianu to: $-19, 16, -7$.\\
Wielomian jest stopnia nieparzystego, ponadto znak współczynnika przy\linebreak najwyższej potędze x jest ujemny.\\ W związku z tym wykres wielomianu zaczyna się od lewej strony powyżej osi OX. A więc $$x \in (-\infty,-19) \cup (-7,16).$$
\rozwStop
\odpStart
$x \in (-\infty,-19) \cup (-7,16)$
\odpStop
\testStart
A.$x \in (-\infty,-19) \cup (-7,16)$\\
B.$x \in (-\infty,-19) \cup (-7,16]$\\
C.$x \in (-\infty,-19) \cup [-7,16)$\\
D.$x \in (-\infty,-19] \cup (-7,16)$\\
E.$x \in (-\infty,-19] \cup (-7,16]$\\
F.$x \in (-\infty,-19] \cup [-7,16)$\\
G.$x \in (-\infty,-19) \cup [-7,16]$\\
H.$x \in (-\infty,-19] \cup [-7,16]$
\testStop
\kluczStart
A
\kluczStop



\zadStart{Zadanie z Wikieł Z 1.62 b) moja wersja nr 929}

Rozwiązać nierówności $(x+19)(16-x)(x+8)\ge0$.
\zadStop
\rozwStart{Patryk Wirkus}{}
Miejsca zerowe naszego wielomianu to: $-19, 16, -8$.\\
Wielomian jest stopnia nieparzystego, ponadto znak współczynnika przy\linebreak najwyższej potędze x jest ujemny.\\ W związku z tym wykres wielomianu zaczyna się od lewej strony powyżej osi OX. A więc $$x \in (-\infty,-19) \cup (-8,16).$$
\rozwStop
\odpStart
$x \in (-\infty,-19) \cup (-8,16)$
\odpStop
\testStart
A.$x \in (-\infty,-19) \cup (-8,16)$\\
B.$x \in (-\infty,-19) \cup (-8,16]$\\
C.$x \in (-\infty,-19) \cup [-8,16)$\\
D.$x \in (-\infty,-19] \cup (-8,16)$\\
E.$x \in (-\infty,-19] \cup (-8,16]$\\
F.$x \in (-\infty,-19] \cup [-8,16)$\\
G.$x \in (-\infty,-19) \cup [-8,16]$\\
H.$x \in (-\infty,-19] \cup [-8,16]$
\testStop
\kluczStart
A
\kluczStop



\zadStart{Zadanie z Wikieł Z 1.62 b) moja wersja nr 930}

Rozwiązać nierówności $(x+19)(16-x)(x+9)\ge0$.
\zadStop
\rozwStart{Patryk Wirkus}{}
Miejsca zerowe naszego wielomianu to: $-19, 16, -9$.\\
Wielomian jest stopnia nieparzystego, ponadto znak współczynnika przy\linebreak najwyższej potędze x jest ujemny.\\ W związku z tym wykres wielomianu zaczyna się od lewej strony powyżej osi OX. A więc $$x \in (-\infty,-19) \cup (-9,16).$$
\rozwStop
\odpStart
$x \in (-\infty,-19) \cup (-9,16)$
\odpStop
\testStart
A.$x \in (-\infty,-19) \cup (-9,16)$\\
B.$x \in (-\infty,-19) \cup (-9,16]$\\
C.$x \in (-\infty,-19) \cup [-9,16)$\\
D.$x \in (-\infty,-19] \cup (-9,16)$\\
E.$x \in (-\infty,-19] \cup (-9,16]$\\
F.$x \in (-\infty,-19] \cup [-9,16)$\\
G.$x \in (-\infty,-19) \cup [-9,16]$\\
H.$x \in (-\infty,-19] \cup [-9,16]$
\testStop
\kluczStart
A
\kluczStop



\zadStart{Zadanie z Wikieł Z 1.62 b) moja wersja nr 931}

Rozwiązać nierówności $(x+19)(16-x)(x+10)\ge0$.
\zadStop
\rozwStart{Patryk Wirkus}{}
Miejsca zerowe naszego wielomianu to: $-19, 16, -10$.\\
Wielomian jest stopnia nieparzystego, ponadto znak współczynnika przy\linebreak najwyższej potędze x jest ujemny.\\ W związku z tym wykres wielomianu zaczyna się od lewej strony powyżej osi OX. A więc $$x \in (-\infty,-19) \cup (-10,16).$$
\rozwStop
\odpStart
$x \in (-\infty,-19) \cup (-10,16)$
\odpStop
\testStart
A.$x \in (-\infty,-19) \cup (-10,16)$\\
B.$x \in (-\infty,-19) \cup (-10,16]$\\
C.$x \in (-\infty,-19) \cup [-10,16)$\\
D.$x \in (-\infty,-19] \cup (-10,16)$\\
E.$x \in (-\infty,-19] \cup (-10,16]$\\
F.$x \in (-\infty,-19] \cup [-10,16)$\\
G.$x \in (-\infty,-19) \cup [-10,16]$\\
H.$x \in (-\infty,-19] \cup [-10,16]$
\testStop
\kluczStart
A
\kluczStop



\zadStart{Zadanie z Wikieł Z 1.62 b) moja wersja nr 932}

Rozwiązać nierówności $(x+19)(16-x)(x+11)\ge0$.
\zadStop
\rozwStart{Patryk Wirkus}{}
Miejsca zerowe naszego wielomianu to: $-19, 16, -11$.\\
Wielomian jest stopnia nieparzystego, ponadto znak współczynnika przy\linebreak najwyższej potędze x jest ujemny.\\ W związku z tym wykres wielomianu zaczyna się od lewej strony powyżej osi OX. A więc $$x \in (-\infty,-19) \cup (-11,16).$$
\rozwStop
\odpStart
$x \in (-\infty,-19) \cup (-11,16)$
\odpStop
\testStart
A.$x \in (-\infty,-19) \cup (-11,16)$\\
B.$x \in (-\infty,-19) \cup (-11,16]$\\
C.$x \in (-\infty,-19) \cup [-11,16)$\\
D.$x \in (-\infty,-19] \cup (-11,16)$\\
E.$x \in (-\infty,-19] \cup (-11,16]$\\
F.$x \in (-\infty,-19] \cup [-11,16)$\\
G.$x \in (-\infty,-19) \cup [-11,16]$\\
H.$x \in (-\infty,-19] \cup [-11,16]$
\testStop
\kluczStart
A
\kluczStop



\zadStart{Zadanie z Wikieł Z 1.62 b) moja wersja nr 933}

Rozwiązać nierówności $(x+19)(16-x)(x+12)\ge0$.
\zadStop
\rozwStart{Patryk Wirkus}{}
Miejsca zerowe naszego wielomianu to: $-19, 16, -12$.\\
Wielomian jest stopnia nieparzystego, ponadto znak współczynnika przy\linebreak najwyższej potędze x jest ujemny.\\ W związku z tym wykres wielomianu zaczyna się od lewej strony powyżej osi OX. A więc $$x \in (-\infty,-19) \cup (-12,16).$$
\rozwStop
\odpStart
$x \in (-\infty,-19) \cup (-12,16)$
\odpStop
\testStart
A.$x \in (-\infty,-19) \cup (-12,16)$\\
B.$x \in (-\infty,-19) \cup (-12,16]$\\
C.$x \in (-\infty,-19) \cup [-12,16)$\\
D.$x \in (-\infty,-19] \cup (-12,16)$\\
E.$x \in (-\infty,-19] \cup (-12,16]$\\
F.$x \in (-\infty,-19] \cup [-12,16)$\\
G.$x \in (-\infty,-19) \cup [-12,16]$\\
H.$x \in (-\infty,-19] \cup [-12,16]$
\testStop
\kluczStart
A
\kluczStop



\zadStart{Zadanie z Wikieł Z 1.62 b) moja wersja nr 934}

Rozwiązać nierówności $(x+19)(16-x)(x+13)\ge0$.
\zadStop
\rozwStart{Patryk Wirkus}{}
Miejsca zerowe naszego wielomianu to: $-19, 16, -13$.\\
Wielomian jest stopnia nieparzystego, ponadto znak współczynnika przy\linebreak najwyższej potędze x jest ujemny.\\ W związku z tym wykres wielomianu zaczyna się od lewej strony powyżej osi OX. A więc $$x \in (-\infty,-19) \cup (-13,16).$$
\rozwStop
\odpStart
$x \in (-\infty,-19) \cup (-13,16)$
\odpStop
\testStart
A.$x \in (-\infty,-19) \cup (-13,16)$\\
B.$x \in (-\infty,-19) \cup (-13,16]$\\
C.$x \in (-\infty,-19) \cup [-13,16)$\\
D.$x \in (-\infty,-19] \cup (-13,16)$\\
E.$x \in (-\infty,-19] \cup (-13,16]$\\
F.$x \in (-\infty,-19] \cup [-13,16)$\\
G.$x \in (-\infty,-19) \cup [-13,16]$\\
H.$x \in (-\infty,-19] \cup [-13,16]$
\testStop
\kluczStart
A
\kluczStop



\zadStart{Zadanie z Wikieł Z 1.62 b) moja wersja nr 935}

Rozwiązać nierówności $(x+19)(16-x)(x+14)\ge0$.
\zadStop
\rozwStart{Patryk Wirkus}{}
Miejsca zerowe naszego wielomianu to: $-19, 16, -14$.\\
Wielomian jest stopnia nieparzystego, ponadto znak współczynnika przy\linebreak najwyższej potędze x jest ujemny.\\ W związku z tym wykres wielomianu zaczyna się od lewej strony powyżej osi OX. A więc $$x \in (-\infty,-19) \cup (-14,16).$$
\rozwStop
\odpStart
$x \in (-\infty,-19) \cup (-14,16)$
\odpStop
\testStart
A.$x \in (-\infty,-19) \cup (-14,16)$\\
B.$x \in (-\infty,-19) \cup (-14,16]$\\
C.$x \in (-\infty,-19) \cup [-14,16)$\\
D.$x \in (-\infty,-19] \cup (-14,16)$\\
E.$x \in (-\infty,-19] \cup (-14,16]$\\
F.$x \in (-\infty,-19] \cup [-14,16)$\\
G.$x \in (-\infty,-19) \cup [-14,16]$\\
H.$x \in (-\infty,-19] \cup [-14,16]$
\testStop
\kluczStart
A
\kluczStop



\zadStart{Zadanie z Wikieł Z 1.62 b) moja wersja nr 936}

Rozwiązać nierówności $(x+19)(16-x)(x+15)\ge0$.
\zadStop
\rozwStart{Patryk Wirkus}{}
Miejsca zerowe naszego wielomianu to: $-19, 16, -15$.\\
Wielomian jest stopnia nieparzystego, ponadto znak współczynnika przy\linebreak najwyższej potędze x jest ujemny.\\ W związku z tym wykres wielomianu zaczyna się od lewej strony powyżej osi OX. A więc $$x \in (-\infty,-19) \cup (-15,16).$$
\rozwStop
\odpStart
$x \in (-\infty,-19) \cup (-15,16)$
\odpStop
\testStart
A.$x \in (-\infty,-19) \cup (-15,16)$\\
B.$x \in (-\infty,-19) \cup (-15,16]$\\
C.$x \in (-\infty,-19) \cup [-15,16)$\\
D.$x \in (-\infty,-19] \cup (-15,16)$\\
E.$x \in (-\infty,-19] \cup (-15,16]$\\
F.$x \in (-\infty,-19] \cup [-15,16)$\\
G.$x \in (-\infty,-19) \cup [-15,16]$\\
H.$x \in (-\infty,-19] \cup [-15,16]$
\testStop
\kluczStart
A
\kluczStop



\zadStart{Zadanie z Wikieł Z 1.62 b) moja wersja nr 937}

Rozwiązać nierówności $(x+19)(17-x)(x+1)\ge0$.
\zadStop
\rozwStart{Patryk Wirkus}{}
Miejsca zerowe naszego wielomianu to: $-19, 17, -1$.\\
Wielomian jest stopnia nieparzystego, ponadto znak współczynnika przy\linebreak najwyższej potędze x jest ujemny.\\ W związku z tym wykres wielomianu zaczyna się od lewej strony powyżej osi OX. A więc $$x \in (-\infty,-19) \cup (-1,17).$$
\rozwStop
\odpStart
$x \in (-\infty,-19) \cup (-1,17)$
\odpStop
\testStart
A.$x \in (-\infty,-19) \cup (-1,17)$\\
B.$x \in (-\infty,-19) \cup (-1,17]$\\
C.$x \in (-\infty,-19) \cup [-1,17)$\\
D.$x \in (-\infty,-19] \cup (-1,17)$\\
E.$x \in (-\infty,-19] \cup (-1,17]$\\
F.$x \in (-\infty,-19] \cup [-1,17)$\\
G.$x \in (-\infty,-19) \cup [-1,17]$\\
H.$x \in (-\infty,-19] \cup [-1,17]$
\testStop
\kluczStart
A
\kluczStop



\zadStart{Zadanie z Wikieł Z 1.62 b) moja wersja nr 938}

Rozwiązać nierówności $(x+19)(17-x)(x+2)\ge0$.
\zadStop
\rozwStart{Patryk Wirkus}{}
Miejsca zerowe naszego wielomianu to: $-19, 17, -2$.\\
Wielomian jest stopnia nieparzystego, ponadto znak współczynnika przy\linebreak najwyższej potędze x jest ujemny.\\ W związku z tym wykres wielomianu zaczyna się od lewej strony powyżej osi OX. A więc $$x \in (-\infty,-19) \cup (-2,17).$$
\rozwStop
\odpStart
$x \in (-\infty,-19) \cup (-2,17)$
\odpStop
\testStart
A.$x \in (-\infty,-19) \cup (-2,17)$\\
B.$x \in (-\infty,-19) \cup (-2,17]$\\
C.$x \in (-\infty,-19) \cup [-2,17)$\\
D.$x \in (-\infty,-19] \cup (-2,17)$\\
E.$x \in (-\infty,-19] \cup (-2,17]$\\
F.$x \in (-\infty,-19] \cup [-2,17)$\\
G.$x \in (-\infty,-19) \cup [-2,17]$\\
H.$x \in (-\infty,-19] \cup [-2,17]$
\testStop
\kluczStart
A
\kluczStop



\zadStart{Zadanie z Wikieł Z 1.62 b) moja wersja nr 939}

Rozwiązać nierówności $(x+19)(17-x)(x+3)\ge0$.
\zadStop
\rozwStart{Patryk Wirkus}{}
Miejsca zerowe naszego wielomianu to: $-19, 17, -3$.\\
Wielomian jest stopnia nieparzystego, ponadto znak współczynnika przy\linebreak najwyższej potędze x jest ujemny.\\ W związku z tym wykres wielomianu zaczyna się od lewej strony powyżej osi OX. A więc $$x \in (-\infty,-19) \cup (-3,17).$$
\rozwStop
\odpStart
$x \in (-\infty,-19) \cup (-3,17)$
\odpStop
\testStart
A.$x \in (-\infty,-19) \cup (-3,17)$\\
B.$x \in (-\infty,-19) \cup (-3,17]$\\
C.$x \in (-\infty,-19) \cup [-3,17)$\\
D.$x \in (-\infty,-19] \cup (-3,17)$\\
E.$x \in (-\infty,-19] \cup (-3,17]$\\
F.$x \in (-\infty,-19] \cup [-3,17)$\\
G.$x \in (-\infty,-19) \cup [-3,17]$\\
H.$x \in (-\infty,-19] \cup [-3,17]$
\testStop
\kluczStart
A
\kluczStop



\zadStart{Zadanie z Wikieł Z 1.62 b) moja wersja nr 940}

Rozwiązać nierówności $(x+19)(17-x)(x+4)\ge0$.
\zadStop
\rozwStart{Patryk Wirkus}{}
Miejsca zerowe naszego wielomianu to: $-19, 17, -4$.\\
Wielomian jest stopnia nieparzystego, ponadto znak współczynnika przy\linebreak najwyższej potędze x jest ujemny.\\ W związku z tym wykres wielomianu zaczyna się od lewej strony powyżej osi OX. A więc $$x \in (-\infty,-19) \cup (-4,17).$$
\rozwStop
\odpStart
$x \in (-\infty,-19) \cup (-4,17)$
\odpStop
\testStart
A.$x \in (-\infty,-19) \cup (-4,17)$\\
B.$x \in (-\infty,-19) \cup (-4,17]$\\
C.$x \in (-\infty,-19) \cup [-4,17)$\\
D.$x \in (-\infty,-19] \cup (-4,17)$\\
E.$x \in (-\infty,-19] \cup (-4,17]$\\
F.$x \in (-\infty,-19] \cup [-4,17)$\\
G.$x \in (-\infty,-19) \cup [-4,17]$\\
H.$x \in (-\infty,-19] \cup [-4,17]$
\testStop
\kluczStart
A
\kluczStop



\zadStart{Zadanie z Wikieł Z 1.62 b) moja wersja nr 941}

Rozwiązać nierówności $(x+19)(17-x)(x+5)\ge0$.
\zadStop
\rozwStart{Patryk Wirkus}{}
Miejsca zerowe naszego wielomianu to: $-19, 17, -5$.\\
Wielomian jest stopnia nieparzystego, ponadto znak współczynnika przy\linebreak najwyższej potędze x jest ujemny.\\ W związku z tym wykres wielomianu zaczyna się od lewej strony powyżej osi OX. A więc $$x \in (-\infty,-19) \cup (-5,17).$$
\rozwStop
\odpStart
$x \in (-\infty,-19) \cup (-5,17)$
\odpStop
\testStart
A.$x \in (-\infty,-19) \cup (-5,17)$\\
B.$x \in (-\infty,-19) \cup (-5,17]$\\
C.$x \in (-\infty,-19) \cup [-5,17)$\\
D.$x \in (-\infty,-19] \cup (-5,17)$\\
E.$x \in (-\infty,-19] \cup (-5,17]$\\
F.$x \in (-\infty,-19] \cup [-5,17)$\\
G.$x \in (-\infty,-19) \cup [-5,17]$\\
H.$x \in (-\infty,-19] \cup [-5,17]$
\testStop
\kluczStart
A
\kluczStop



\zadStart{Zadanie z Wikieł Z 1.62 b) moja wersja nr 942}

Rozwiązać nierówności $(x+19)(17-x)(x+6)\ge0$.
\zadStop
\rozwStart{Patryk Wirkus}{}
Miejsca zerowe naszego wielomianu to: $-19, 17, -6$.\\
Wielomian jest stopnia nieparzystego, ponadto znak współczynnika przy\linebreak najwyższej potędze x jest ujemny.\\ W związku z tym wykres wielomianu zaczyna się od lewej strony powyżej osi OX. A więc $$x \in (-\infty,-19) \cup (-6,17).$$
\rozwStop
\odpStart
$x \in (-\infty,-19) \cup (-6,17)$
\odpStop
\testStart
A.$x \in (-\infty,-19) \cup (-6,17)$\\
B.$x \in (-\infty,-19) \cup (-6,17]$\\
C.$x \in (-\infty,-19) \cup [-6,17)$\\
D.$x \in (-\infty,-19] \cup (-6,17)$\\
E.$x \in (-\infty,-19] \cup (-6,17]$\\
F.$x \in (-\infty,-19] \cup [-6,17)$\\
G.$x \in (-\infty,-19) \cup [-6,17]$\\
H.$x \in (-\infty,-19] \cup [-6,17]$
\testStop
\kluczStart
A
\kluczStop



\zadStart{Zadanie z Wikieł Z 1.62 b) moja wersja nr 943}

Rozwiązać nierówności $(x+19)(17-x)(x+7)\ge0$.
\zadStop
\rozwStart{Patryk Wirkus}{}
Miejsca zerowe naszego wielomianu to: $-19, 17, -7$.\\
Wielomian jest stopnia nieparzystego, ponadto znak współczynnika przy\linebreak najwyższej potędze x jest ujemny.\\ W związku z tym wykres wielomianu zaczyna się od lewej strony powyżej osi OX. A więc $$x \in (-\infty,-19) \cup (-7,17).$$
\rozwStop
\odpStart
$x \in (-\infty,-19) \cup (-7,17)$
\odpStop
\testStart
A.$x \in (-\infty,-19) \cup (-7,17)$\\
B.$x \in (-\infty,-19) \cup (-7,17]$\\
C.$x \in (-\infty,-19) \cup [-7,17)$\\
D.$x \in (-\infty,-19] \cup (-7,17)$\\
E.$x \in (-\infty,-19] \cup (-7,17]$\\
F.$x \in (-\infty,-19] \cup [-7,17)$\\
G.$x \in (-\infty,-19) \cup [-7,17]$\\
H.$x \in (-\infty,-19] \cup [-7,17]$
\testStop
\kluczStart
A
\kluczStop



\zadStart{Zadanie z Wikieł Z 1.62 b) moja wersja nr 944}

Rozwiązać nierówności $(x+19)(17-x)(x+8)\ge0$.
\zadStop
\rozwStart{Patryk Wirkus}{}
Miejsca zerowe naszego wielomianu to: $-19, 17, -8$.\\
Wielomian jest stopnia nieparzystego, ponadto znak współczynnika przy\linebreak najwyższej potędze x jest ujemny.\\ W związku z tym wykres wielomianu zaczyna się od lewej strony powyżej osi OX. A więc $$x \in (-\infty,-19) \cup (-8,17).$$
\rozwStop
\odpStart
$x \in (-\infty,-19) \cup (-8,17)$
\odpStop
\testStart
A.$x \in (-\infty,-19) \cup (-8,17)$\\
B.$x \in (-\infty,-19) \cup (-8,17]$\\
C.$x \in (-\infty,-19) \cup [-8,17)$\\
D.$x \in (-\infty,-19] \cup (-8,17)$\\
E.$x \in (-\infty,-19] \cup (-8,17]$\\
F.$x \in (-\infty,-19] \cup [-8,17)$\\
G.$x \in (-\infty,-19) \cup [-8,17]$\\
H.$x \in (-\infty,-19] \cup [-8,17]$
\testStop
\kluczStart
A
\kluczStop



\zadStart{Zadanie z Wikieł Z 1.62 b) moja wersja nr 945}

Rozwiązać nierówności $(x+19)(17-x)(x+9)\ge0$.
\zadStop
\rozwStart{Patryk Wirkus}{}
Miejsca zerowe naszego wielomianu to: $-19, 17, -9$.\\
Wielomian jest stopnia nieparzystego, ponadto znak współczynnika przy\linebreak najwyższej potędze x jest ujemny.\\ W związku z tym wykres wielomianu zaczyna się od lewej strony powyżej osi OX. A więc $$x \in (-\infty,-19) \cup (-9,17).$$
\rozwStop
\odpStart
$x \in (-\infty,-19) \cup (-9,17)$
\odpStop
\testStart
A.$x \in (-\infty,-19) \cup (-9,17)$\\
B.$x \in (-\infty,-19) \cup (-9,17]$\\
C.$x \in (-\infty,-19) \cup [-9,17)$\\
D.$x \in (-\infty,-19] \cup (-9,17)$\\
E.$x \in (-\infty,-19] \cup (-9,17]$\\
F.$x \in (-\infty,-19] \cup [-9,17)$\\
G.$x \in (-\infty,-19) \cup [-9,17]$\\
H.$x \in (-\infty,-19] \cup [-9,17]$
\testStop
\kluczStart
A
\kluczStop



\zadStart{Zadanie z Wikieł Z 1.62 b) moja wersja nr 946}

Rozwiązać nierówności $(x+19)(17-x)(x+10)\ge0$.
\zadStop
\rozwStart{Patryk Wirkus}{}
Miejsca zerowe naszego wielomianu to: $-19, 17, -10$.\\
Wielomian jest stopnia nieparzystego, ponadto znak współczynnika przy\linebreak najwyższej potędze x jest ujemny.\\ W związku z tym wykres wielomianu zaczyna się od lewej strony powyżej osi OX. A więc $$x \in (-\infty,-19) \cup (-10,17).$$
\rozwStop
\odpStart
$x \in (-\infty,-19) \cup (-10,17)$
\odpStop
\testStart
A.$x \in (-\infty,-19) \cup (-10,17)$\\
B.$x \in (-\infty,-19) \cup (-10,17]$\\
C.$x \in (-\infty,-19) \cup [-10,17)$\\
D.$x \in (-\infty,-19] \cup (-10,17)$\\
E.$x \in (-\infty,-19] \cup (-10,17]$\\
F.$x \in (-\infty,-19] \cup [-10,17)$\\
G.$x \in (-\infty,-19) \cup [-10,17]$\\
H.$x \in (-\infty,-19] \cup [-10,17]$
\testStop
\kluczStart
A
\kluczStop



\zadStart{Zadanie z Wikieł Z 1.62 b) moja wersja nr 947}

Rozwiązać nierówności $(x+19)(17-x)(x+11)\ge0$.
\zadStop
\rozwStart{Patryk Wirkus}{}
Miejsca zerowe naszego wielomianu to: $-19, 17, -11$.\\
Wielomian jest stopnia nieparzystego, ponadto znak współczynnika przy\linebreak najwyższej potędze x jest ujemny.\\ W związku z tym wykres wielomianu zaczyna się od lewej strony powyżej osi OX. A więc $$x \in (-\infty,-19) \cup (-11,17).$$
\rozwStop
\odpStart
$x \in (-\infty,-19) \cup (-11,17)$
\odpStop
\testStart
A.$x \in (-\infty,-19) \cup (-11,17)$\\
B.$x \in (-\infty,-19) \cup (-11,17]$\\
C.$x \in (-\infty,-19) \cup [-11,17)$\\
D.$x \in (-\infty,-19] \cup (-11,17)$\\
E.$x \in (-\infty,-19] \cup (-11,17]$\\
F.$x \in (-\infty,-19] \cup [-11,17)$\\
G.$x \in (-\infty,-19) \cup [-11,17]$\\
H.$x \in (-\infty,-19] \cup [-11,17]$
\testStop
\kluczStart
A
\kluczStop



\zadStart{Zadanie z Wikieł Z 1.62 b) moja wersja nr 948}

Rozwiązać nierówności $(x+19)(17-x)(x+12)\ge0$.
\zadStop
\rozwStart{Patryk Wirkus}{}
Miejsca zerowe naszego wielomianu to: $-19, 17, -12$.\\
Wielomian jest stopnia nieparzystego, ponadto znak współczynnika przy\linebreak najwyższej potędze x jest ujemny.\\ W związku z tym wykres wielomianu zaczyna się od lewej strony powyżej osi OX. A więc $$x \in (-\infty,-19) \cup (-12,17).$$
\rozwStop
\odpStart
$x \in (-\infty,-19) \cup (-12,17)$
\odpStop
\testStart
A.$x \in (-\infty,-19) \cup (-12,17)$\\
B.$x \in (-\infty,-19) \cup (-12,17]$\\
C.$x \in (-\infty,-19) \cup [-12,17)$\\
D.$x \in (-\infty,-19] \cup (-12,17)$\\
E.$x \in (-\infty,-19] \cup (-12,17]$\\
F.$x \in (-\infty,-19] \cup [-12,17)$\\
G.$x \in (-\infty,-19) \cup [-12,17]$\\
H.$x \in (-\infty,-19] \cup [-12,17]$
\testStop
\kluczStart
A
\kluczStop



\zadStart{Zadanie z Wikieł Z 1.62 b) moja wersja nr 949}

Rozwiązać nierówności $(x+19)(17-x)(x+13)\ge0$.
\zadStop
\rozwStart{Patryk Wirkus}{}
Miejsca zerowe naszego wielomianu to: $-19, 17, -13$.\\
Wielomian jest stopnia nieparzystego, ponadto znak współczynnika przy\linebreak najwyższej potędze x jest ujemny.\\ W związku z tym wykres wielomianu zaczyna się od lewej strony powyżej osi OX. A więc $$x \in (-\infty,-19) \cup (-13,17).$$
\rozwStop
\odpStart
$x \in (-\infty,-19) \cup (-13,17)$
\odpStop
\testStart
A.$x \in (-\infty,-19) \cup (-13,17)$\\
B.$x \in (-\infty,-19) \cup (-13,17]$\\
C.$x \in (-\infty,-19) \cup [-13,17)$\\
D.$x \in (-\infty,-19] \cup (-13,17)$\\
E.$x \in (-\infty,-19] \cup (-13,17]$\\
F.$x \in (-\infty,-19] \cup [-13,17)$\\
G.$x \in (-\infty,-19) \cup [-13,17]$\\
H.$x \in (-\infty,-19] \cup [-13,17]$
\testStop
\kluczStart
A
\kluczStop



\zadStart{Zadanie z Wikieł Z 1.62 b) moja wersja nr 950}

Rozwiązać nierówności $(x+19)(17-x)(x+14)\ge0$.
\zadStop
\rozwStart{Patryk Wirkus}{}
Miejsca zerowe naszego wielomianu to: $-19, 17, -14$.\\
Wielomian jest stopnia nieparzystego, ponadto znak współczynnika przy\linebreak najwyższej potędze x jest ujemny.\\ W związku z tym wykres wielomianu zaczyna się od lewej strony powyżej osi OX. A więc $$x \in (-\infty,-19) \cup (-14,17).$$
\rozwStop
\odpStart
$x \in (-\infty,-19) \cup (-14,17)$
\odpStop
\testStart
A.$x \in (-\infty,-19) \cup (-14,17)$\\
B.$x \in (-\infty,-19) \cup (-14,17]$\\
C.$x \in (-\infty,-19) \cup [-14,17)$\\
D.$x \in (-\infty,-19] \cup (-14,17)$\\
E.$x \in (-\infty,-19] \cup (-14,17]$\\
F.$x \in (-\infty,-19] \cup [-14,17)$\\
G.$x \in (-\infty,-19) \cup [-14,17]$\\
H.$x \in (-\infty,-19] \cup [-14,17]$
\testStop
\kluczStart
A
\kluczStop



\zadStart{Zadanie z Wikieł Z 1.62 b) moja wersja nr 951}

Rozwiązać nierówności $(x+19)(17-x)(x+15)\ge0$.
\zadStop
\rozwStart{Patryk Wirkus}{}
Miejsca zerowe naszego wielomianu to: $-19, 17, -15$.\\
Wielomian jest stopnia nieparzystego, ponadto znak współczynnika przy\linebreak najwyższej potędze x jest ujemny.\\ W związku z tym wykres wielomianu zaczyna się od lewej strony powyżej osi OX. A więc $$x \in (-\infty,-19) \cup (-15,17).$$
\rozwStop
\odpStart
$x \in (-\infty,-19) \cup (-15,17)$
\odpStop
\testStart
A.$x \in (-\infty,-19) \cup (-15,17)$\\
B.$x \in (-\infty,-19) \cup (-15,17]$\\
C.$x \in (-\infty,-19) \cup [-15,17)$\\
D.$x \in (-\infty,-19] \cup (-15,17)$\\
E.$x \in (-\infty,-19] \cup (-15,17]$\\
F.$x \in (-\infty,-19] \cup [-15,17)$\\
G.$x \in (-\infty,-19) \cup [-15,17]$\\
H.$x \in (-\infty,-19] \cup [-15,17]$
\testStop
\kluczStart
A
\kluczStop



\zadStart{Zadanie z Wikieł Z 1.62 b) moja wersja nr 952}

Rozwiązać nierówności $(x+19)(17-x)(x+16)\ge0$.
\zadStop
\rozwStart{Patryk Wirkus}{}
Miejsca zerowe naszego wielomianu to: $-19, 17, -16$.\\
Wielomian jest stopnia nieparzystego, ponadto znak współczynnika przy\linebreak najwyższej potędze x jest ujemny.\\ W związku z tym wykres wielomianu zaczyna się od lewej strony powyżej osi OX. A więc $$x \in (-\infty,-19) \cup (-16,17).$$
\rozwStop
\odpStart
$x \in (-\infty,-19) \cup (-16,17)$
\odpStop
\testStart
A.$x \in (-\infty,-19) \cup (-16,17)$\\
B.$x \in (-\infty,-19) \cup (-16,17]$\\
C.$x \in (-\infty,-19) \cup [-16,17)$\\
D.$x \in (-\infty,-19] \cup (-16,17)$\\
E.$x \in (-\infty,-19] \cup (-16,17]$\\
F.$x \in (-\infty,-19] \cup [-16,17)$\\
G.$x \in (-\infty,-19) \cup [-16,17]$\\
H.$x \in (-\infty,-19] \cup [-16,17]$
\testStop
\kluczStart
A
\kluczStop



\zadStart{Zadanie z Wikieł Z 1.62 b) moja wersja nr 953}

Rozwiązać nierówności $(x+19)(18-x)(x+1)\ge0$.
\zadStop
\rozwStart{Patryk Wirkus}{}
Miejsca zerowe naszego wielomianu to: $-19, 18, -1$.\\
Wielomian jest stopnia nieparzystego, ponadto znak współczynnika przy\linebreak najwyższej potędze x jest ujemny.\\ W związku z tym wykres wielomianu zaczyna się od lewej strony powyżej osi OX. A więc $$x \in (-\infty,-19) \cup (-1,18).$$
\rozwStop
\odpStart
$x \in (-\infty,-19) \cup (-1,18)$
\odpStop
\testStart
A.$x \in (-\infty,-19) \cup (-1,18)$\\
B.$x \in (-\infty,-19) \cup (-1,18]$\\
C.$x \in (-\infty,-19) \cup [-1,18)$\\
D.$x \in (-\infty,-19] \cup (-1,18)$\\
E.$x \in (-\infty,-19] \cup (-1,18]$\\
F.$x \in (-\infty,-19] \cup [-1,18)$\\
G.$x \in (-\infty,-19) \cup [-1,18]$\\
H.$x \in (-\infty,-19] \cup [-1,18]$
\testStop
\kluczStart
A
\kluczStop



\zadStart{Zadanie z Wikieł Z 1.62 b) moja wersja nr 954}

Rozwiązać nierówności $(x+19)(18-x)(x+2)\ge0$.
\zadStop
\rozwStart{Patryk Wirkus}{}
Miejsca zerowe naszego wielomianu to: $-19, 18, -2$.\\
Wielomian jest stopnia nieparzystego, ponadto znak współczynnika przy\linebreak najwyższej potędze x jest ujemny.\\ W związku z tym wykres wielomianu zaczyna się od lewej strony powyżej osi OX. A więc $$x \in (-\infty,-19) \cup (-2,18).$$
\rozwStop
\odpStart
$x \in (-\infty,-19) \cup (-2,18)$
\odpStop
\testStart
A.$x \in (-\infty,-19) \cup (-2,18)$\\
B.$x \in (-\infty,-19) \cup (-2,18]$\\
C.$x \in (-\infty,-19) \cup [-2,18)$\\
D.$x \in (-\infty,-19] \cup (-2,18)$\\
E.$x \in (-\infty,-19] \cup (-2,18]$\\
F.$x \in (-\infty,-19] \cup [-2,18)$\\
G.$x \in (-\infty,-19) \cup [-2,18]$\\
H.$x \in (-\infty,-19] \cup [-2,18]$
\testStop
\kluczStart
A
\kluczStop



\zadStart{Zadanie z Wikieł Z 1.62 b) moja wersja nr 955}

Rozwiązać nierówności $(x+19)(18-x)(x+3)\ge0$.
\zadStop
\rozwStart{Patryk Wirkus}{}
Miejsca zerowe naszego wielomianu to: $-19, 18, -3$.\\
Wielomian jest stopnia nieparzystego, ponadto znak współczynnika przy\linebreak najwyższej potędze x jest ujemny.\\ W związku z tym wykres wielomianu zaczyna się od lewej strony powyżej osi OX. A więc $$x \in (-\infty,-19) \cup (-3,18).$$
\rozwStop
\odpStart
$x \in (-\infty,-19) \cup (-3,18)$
\odpStop
\testStart
A.$x \in (-\infty,-19) \cup (-3,18)$\\
B.$x \in (-\infty,-19) \cup (-3,18]$\\
C.$x \in (-\infty,-19) \cup [-3,18)$\\
D.$x \in (-\infty,-19] \cup (-3,18)$\\
E.$x \in (-\infty,-19] \cup (-3,18]$\\
F.$x \in (-\infty,-19] \cup [-3,18)$\\
G.$x \in (-\infty,-19) \cup [-3,18]$\\
H.$x \in (-\infty,-19] \cup [-3,18]$
\testStop
\kluczStart
A
\kluczStop



\zadStart{Zadanie z Wikieł Z 1.62 b) moja wersja nr 956}

Rozwiązać nierówności $(x+19)(18-x)(x+4)\ge0$.
\zadStop
\rozwStart{Patryk Wirkus}{}
Miejsca zerowe naszego wielomianu to: $-19, 18, -4$.\\
Wielomian jest stopnia nieparzystego, ponadto znak współczynnika przy\linebreak najwyższej potędze x jest ujemny.\\ W związku z tym wykres wielomianu zaczyna się od lewej strony powyżej osi OX. A więc $$x \in (-\infty,-19) \cup (-4,18).$$
\rozwStop
\odpStart
$x \in (-\infty,-19) \cup (-4,18)$
\odpStop
\testStart
A.$x \in (-\infty,-19) \cup (-4,18)$\\
B.$x \in (-\infty,-19) \cup (-4,18]$\\
C.$x \in (-\infty,-19) \cup [-4,18)$\\
D.$x \in (-\infty,-19] \cup (-4,18)$\\
E.$x \in (-\infty,-19] \cup (-4,18]$\\
F.$x \in (-\infty,-19] \cup [-4,18)$\\
G.$x \in (-\infty,-19) \cup [-4,18]$\\
H.$x \in (-\infty,-19] \cup [-4,18]$
\testStop
\kluczStart
A
\kluczStop



\zadStart{Zadanie z Wikieł Z 1.62 b) moja wersja nr 957}

Rozwiązać nierówności $(x+19)(18-x)(x+5)\ge0$.
\zadStop
\rozwStart{Patryk Wirkus}{}
Miejsca zerowe naszego wielomianu to: $-19, 18, -5$.\\
Wielomian jest stopnia nieparzystego, ponadto znak współczynnika przy\linebreak najwyższej potędze x jest ujemny.\\ W związku z tym wykres wielomianu zaczyna się od lewej strony powyżej osi OX. A więc $$x \in (-\infty,-19) \cup (-5,18).$$
\rozwStop
\odpStart
$x \in (-\infty,-19) \cup (-5,18)$
\odpStop
\testStart
A.$x \in (-\infty,-19) \cup (-5,18)$\\
B.$x \in (-\infty,-19) \cup (-5,18]$\\
C.$x \in (-\infty,-19) \cup [-5,18)$\\
D.$x \in (-\infty,-19] \cup (-5,18)$\\
E.$x \in (-\infty,-19] \cup (-5,18]$\\
F.$x \in (-\infty,-19] \cup [-5,18)$\\
G.$x \in (-\infty,-19) \cup [-5,18]$\\
H.$x \in (-\infty,-19] \cup [-5,18]$
\testStop
\kluczStart
A
\kluczStop



\zadStart{Zadanie z Wikieł Z 1.62 b) moja wersja nr 958}

Rozwiązać nierówności $(x+19)(18-x)(x+6)\ge0$.
\zadStop
\rozwStart{Patryk Wirkus}{}
Miejsca zerowe naszego wielomianu to: $-19, 18, -6$.\\
Wielomian jest stopnia nieparzystego, ponadto znak współczynnika przy\linebreak najwyższej potędze x jest ujemny.\\ W związku z tym wykres wielomianu zaczyna się od lewej strony powyżej osi OX. A więc $$x \in (-\infty,-19) \cup (-6,18).$$
\rozwStop
\odpStart
$x \in (-\infty,-19) \cup (-6,18)$
\odpStop
\testStart
A.$x \in (-\infty,-19) \cup (-6,18)$\\
B.$x \in (-\infty,-19) \cup (-6,18]$\\
C.$x \in (-\infty,-19) \cup [-6,18)$\\
D.$x \in (-\infty,-19] \cup (-6,18)$\\
E.$x \in (-\infty,-19] \cup (-6,18]$\\
F.$x \in (-\infty,-19] \cup [-6,18)$\\
G.$x \in (-\infty,-19) \cup [-6,18]$\\
H.$x \in (-\infty,-19] \cup [-6,18]$
\testStop
\kluczStart
A
\kluczStop



\zadStart{Zadanie z Wikieł Z 1.62 b) moja wersja nr 959}

Rozwiązać nierówności $(x+19)(18-x)(x+7)\ge0$.
\zadStop
\rozwStart{Patryk Wirkus}{}
Miejsca zerowe naszego wielomianu to: $-19, 18, -7$.\\
Wielomian jest stopnia nieparzystego, ponadto znak współczynnika przy\linebreak najwyższej potędze x jest ujemny.\\ W związku z tym wykres wielomianu zaczyna się od lewej strony powyżej osi OX. A więc $$x \in (-\infty,-19) \cup (-7,18).$$
\rozwStop
\odpStart
$x \in (-\infty,-19) \cup (-7,18)$
\odpStop
\testStart
A.$x \in (-\infty,-19) \cup (-7,18)$\\
B.$x \in (-\infty,-19) \cup (-7,18]$\\
C.$x \in (-\infty,-19) \cup [-7,18)$\\
D.$x \in (-\infty,-19] \cup (-7,18)$\\
E.$x \in (-\infty,-19] \cup (-7,18]$\\
F.$x \in (-\infty,-19] \cup [-7,18)$\\
G.$x \in (-\infty,-19) \cup [-7,18]$\\
H.$x \in (-\infty,-19] \cup [-7,18]$
\testStop
\kluczStart
A
\kluczStop



\zadStart{Zadanie z Wikieł Z 1.62 b) moja wersja nr 960}

Rozwiązać nierówności $(x+19)(18-x)(x+8)\ge0$.
\zadStop
\rozwStart{Patryk Wirkus}{}
Miejsca zerowe naszego wielomianu to: $-19, 18, -8$.\\
Wielomian jest stopnia nieparzystego, ponadto znak współczynnika przy\linebreak najwyższej potędze x jest ujemny.\\ W związku z tym wykres wielomianu zaczyna się od lewej strony powyżej osi OX. A więc $$x \in (-\infty,-19) \cup (-8,18).$$
\rozwStop
\odpStart
$x \in (-\infty,-19) \cup (-8,18)$
\odpStop
\testStart
A.$x \in (-\infty,-19) \cup (-8,18)$\\
B.$x \in (-\infty,-19) \cup (-8,18]$\\
C.$x \in (-\infty,-19) \cup [-8,18)$\\
D.$x \in (-\infty,-19] \cup (-8,18)$\\
E.$x \in (-\infty,-19] \cup (-8,18]$\\
F.$x \in (-\infty,-19] \cup [-8,18)$\\
G.$x \in (-\infty,-19) \cup [-8,18]$\\
H.$x \in (-\infty,-19] \cup [-8,18]$
\testStop
\kluczStart
A
\kluczStop



\zadStart{Zadanie z Wikieł Z 1.62 b) moja wersja nr 961}

Rozwiązać nierówności $(x+19)(18-x)(x+9)\ge0$.
\zadStop
\rozwStart{Patryk Wirkus}{}
Miejsca zerowe naszego wielomianu to: $-19, 18, -9$.\\
Wielomian jest stopnia nieparzystego, ponadto znak współczynnika przy\linebreak najwyższej potędze x jest ujemny.\\ W związku z tym wykres wielomianu zaczyna się od lewej strony powyżej osi OX. A więc $$x \in (-\infty,-19) \cup (-9,18).$$
\rozwStop
\odpStart
$x \in (-\infty,-19) \cup (-9,18)$
\odpStop
\testStart
A.$x \in (-\infty,-19) \cup (-9,18)$\\
B.$x \in (-\infty,-19) \cup (-9,18]$\\
C.$x \in (-\infty,-19) \cup [-9,18)$\\
D.$x \in (-\infty,-19] \cup (-9,18)$\\
E.$x \in (-\infty,-19] \cup (-9,18]$\\
F.$x \in (-\infty,-19] \cup [-9,18)$\\
G.$x \in (-\infty,-19) \cup [-9,18]$\\
H.$x \in (-\infty,-19] \cup [-9,18]$
\testStop
\kluczStart
A
\kluczStop



\zadStart{Zadanie z Wikieł Z 1.62 b) moja wersja nr 962}

Rozwiązać nierówności $(x+19)(18-x)(x+10)\ge0$.
\zadStop
\rozwStart{Patryk Wirkus}{}
Miejsca zerowe naszego wielomianu to: $-19, 18, -10$.\\
Wielomian jest stopnia nieparzystego, ponadto znak współczynnika przy\linebreak najwyższej potędze x jest ujemny.\\ W związku z tym wykres wielomianu zaczyna się od lewej strony powyżej osi OX. A więc $$x \in (-\infty,-19) \cup (-10,18).$$
\rozwStop
\odpStart
$x \in (-\infty,-19) \cup (-10,18)$
\odpStop
\testStart
A.$x \in (-\infty,-19) \cup (-10,18)$\\
B.$x \in (-\infty,-19) \cup (-10,18]$\\
C.$x \in (-\infty,-19) \cup [-10,18)$\\
D.$x \in (-\infty,-19] \cup (-10,18)$\\
E.$x \in (-\infty,-19] \cup (-10,18]$\\
F.$x \in (-\infty,-19] \cup [-10,18)$\\
G.$x \in (-\infty,-19) \cup [-10,18]$\\
H.$x \in (-\infty,-19] \cup [-10,18]$
\testStop
\kluczStart
A
\kluczStop



\zadStart{Zadanie z Wikieł Z 1.62 b) moja wersja nr 963}

Rozwiązać nierówności $(x+19)(18-x)(x+11)\ge0$.
\zadStop
\rozwStart{Patryk Wirkus}{}
Miejsca zerowe naszego wielomianu to: $-19, 18, -11$.\\
Wielomian jest stopnia nieparzystego, ponadto znak współczynnika przy\linebreak najwyższej potędze x jest ujemny.\\ W związku z tym wykres wielomianu zaczyna się od lewej strony powyżej osi OX. A więc $$x \in (-\infty,-19) \cup (-11,18).$$
\rozwStop
\odpStart
$x \in (-\infty,-19) \cup (-11,18)$
\odpStop
\testStart
A.$x \in (-\infty,-19) \cup (-11,18)$\\
B.$x \in (-\infty,-19) \cup (-11,18]$\\
C.$x \in (-\infty,-19) \cup [-11,18)$\\
D.$x \in (-\infty,-19] \cup (-11,18)$\\
E.$x \in (-\infty,-19] \cup (-11,18]$\\
F.$x \in (-\infty,-19] \cup [-11,18)$\\
G.$x \in (-\infty,-19) \cup [-11,18]$\\
H.$x \in (-\infty,-19] \cup [-11,18]$
\testStop
\kluczStart
A
\kluczStop



\zadStart{Zadanie z Wikieł Z 1.62 b) moja wersja nr 964}

Rozwiązać nierówności $(x+19)(18-x)(x+12)\ge0$.
\zadStop
\rozwStart{Patryk Wirkus}{}
Miejsca zerowe naszego wielomianu to: $-19, 18, -12$.\\
Wielomian jest stopnia nieparzystego, ponadto znak współczynnika przy\linebreak najwyższej potędze x jest ujemny.\\ W związku z tym wykres wielomianu zaczyna się od lewej strony powyżej osi OX. A więc $$x \in (-\infty,-19) \cup (-12,18).$$
\rozwStop
\odpStart
$x \in (-\infty,-19) \cup (-12,18)$
\odpStop
\testStart
A.$x \in (-\infty,-19) \cup (-12,18)$\\
B.$x \in (-\infty,-19) \cup (-12,18]$\\
C.$x \in (-\infty,-19) \cup [-12,18)$\\
D.$x \in (-\infty,-19] \cup (-12,18)$\\
E.$x \in (-\infty,-19] \cup (-12,18]$\\
F.$x \in (-\infty,-19] \cup [-12,18)$\\
G.$x \in (-\infty,-19) \cup [-12,18]$\\
H.$x \in (-\infty,-19] \cup [-12,18]$
\testStop
\kluczStart
A
\kluczStop



\zadStart{Zadanie z Wikieł Z 1.62 b) moja wersja nr 965}

Rozwiązać nierówności $(x+19)(18-x)(x+13)\ge0$.
\zadStop
\rozwStart{Patryk Wirkus}{}
Miejsca zerowe naszego wielomianu to: $-19, 18, -13$.\\
Wielomian jest stopnia nieparzystego, ponadto znak współczynnika przy\linebreak najwyższej potędze x jest ujemny.\\ W związku z tym wykres wielomianu zaczyna się od lewej strony powyżej osi OX. A więc $$x \in (-\infty,-19) \cup (-13,18).$$
\rozwStop
\odpStart
$x \in (-\infty,-19) \cup (-13,18)$
\odpStop
\testStart
A.$x \in (-\infty,-19) \cup (-13,18)$\\
B.$x \in (-\infty,-19) \cup (-13,18]$\\
C.$x \in (-\infty,-19) \cup [-13,18)$\\
D.$x \in (-\infty,-19] \cup (-13,18)$\\
E.$x \in (-\infty,-19] \cup (-13,18]$\\
F.$x \in (-\infty,-19] \cup [-13,18)$\\
G.$x \in (-\infty,-19) \cup [-13,18]$\\
H.$x \in (-\infty,-19] \cup [-13,18]$
\testStop
\kluczStart
A
\kluczStop



\zadStart{Zadanie z Wikieł Z 1.62 b) moja wersja nr 966}

Rozwiązać nierówności $(x+19)(18-x)(x+14)\ge0$.
\zadStop
\rozwStart{Patryk Wirkus}{}
Miejsca zerowe naszego wielomianu to: $-19, 18, -14$.\\
Wielomian jest stopnia nieparzystego, ponadto znak współczynnika przy\linebreak najwyższej potędze x jest ujemny.\\ W związku z tym wykres wielomianu zaczyna się od lewej strony powyżej osi OX. A więc $$x \in (-\infty,-19) \cup (-14,18).$$
\rozwStop
\odpStart
$x \in (-\infty,-19) \cup (-14,18)$
\odpStop
\testStart
A.$x \in (-\infty,-19) \cup (-14,18)$\\
B.$x \in (-\infty,-19) \cup (-14,18]$\\
C.$x \in (-\infty,-19) \cup [-14,18)$\\
D.$x \in (-\infty,-19] \cup (-14,18)$\\
E.$x \in (-\infty,-19] \cup (-14,18]$\\
F.$x \in (-\infty,-19] \cup [-14,18)$\\
G.$x \in (-\infty,-19) \cup [-14,18]$\\
H.$x \in (-\infty,-19] \cup [-14,18]$
\testStop
\kluczStart
A
\kluczStop



\zadStart{Zadanie z Wikieł Z 1.62 b) moja wersja nr 967}

Rozwiązać nierówności $(x+19)(18-x)(x+15)\ge0$.
\zadStop
\rozwStart{Patryk Wirkus}{}
Miejsca zerowe naszego wielomianu to: $-19, 18, -15$.\\
Wielomian jest stopnia nieparzystego, ponadto znak współczynnika przy\linebreak najwyższej potędze x jest ujemny.\\ W związku z tym wykres wielomianu zaczyna się od lewej strony powyżej osi OX. A więc $$x \in (-\infty,-19) \cup (-15,18).$$
\rozwStop
\odpStart
$x \in (-\infty,-19) \cup (-15,18)$
\odpStop
\testStart
A.$x \in (-\infty,-19) \cup (-15,18)$\\
B.$x \in (-\infty,-19) \cup (-15,18]$\\
C.$x \in (-\infty,-19) \cup [-15,18)$\\
D.$x \in (-\infty,-19] \cup (-15,18)$\\
E.$x \in (-\infty,-19] \cup (-15,18]$\\
F.$x \in (-\infty,-19] \cup [-15,18)$\\
G.$x \in (-\infty,-19) \cup [-15,18]$\\
H.$x \in (-\infty,-19] \cup [-15,18]$
\testStop
\kluczStart
A
\kluczStop



\zadStart{Zadanie z Wikieł Z 1.62 b) moja wersja nr 968}

Rozwiązać nierówności $(x+19)(18-x)(x+16)\ge0$.
\zadStop
\rozwStart{Patryk Wirkus}{}
Miejsca zerowe naszego wielomianu to: $-19, 18, -16$.\\
Wielomian jest stopnia nieparzystego, ponadto znak współczynnika przy\linebreak najwyższej potędze x jest ujemny.\\ W związku z tym wykres wielomianu zaczyna się od lewej strony powyżej osi OX. A więc $$x \in (-\infty,-19) \cup (-16,18).$$
\rozwStop
\odpStart
$x \in (-\infty,-19) \cup (-16,18)$
\odpStop
\testStart
A.$x \in (-\infty,-19) \cup (-16,18)$\\
B.$x \in (-\infty,-19) \cup (-16,18]$\\
C.$x \in (-\infty,-19) \cup [-16,18)$\\
D.$x \in (-\infty,-19] \cup (-16,18)$\\
E.$x \in (-\infty,-19] \cup (-16,18]$\\
F.$x \in (-\infty,-19] \cup [-16,18)$\\
G.$x \in (-\infty,-19) \cup [-16,18]$\\
H.$x \in (-\infty,-19] \cup [-16,18]$
\testStop
\kluczStart
A
\kluczStop



\zadStart{Zadanie z Wikieł Z 1.62 b) moja wersja nr 969}

Rozwiązać nierówności $(x+19)(18-x)(x+17)\ge0$.
\zadStop
\rozwStart{Patryk Wirkus}{}
Miejsca zerowe naszego wielomianu to: $-19, 18, -17$.\\
Wielomian jest stopnia nieparzystego, ponadto znak współczynnika przy\linebreak najwyższej potędze x jest ujemny.\\ W związku z tym wykres wielomianu zaczyna się od lewej strony powyżej osi OX. A więc $$x \in (-\infty,-19) \cup (-17,18).$$
\rozwStop
\odpStart
$x \in (-\infty,-19) \cup (-17,18)$
\odpStop
\testStart
A.$x \in (-\infty,-19) \cup (-17,18)$\\
B.$x \in (-\infty,-19) \cup (-17,18]$\\
C.$x \in (-\infty,-19) \cup [-17,18)$\\
D.$x \in (-\infty,-19] \cup (-17,18)$\\
E.$x \in (-\infty,-19] \cup (-17,18]$\\
F.$x \in (-\infty,-19] \cup [-17,18)$\\
G.$x \in (-\infty,-19) \cup [-17,18]$\\
H.$x \in (-\infty,-19] \cup [-17,18]$
\testStop
\kluczStart
A
\kluczStop



\zadStart{Zadanie z Wikieł Z 1.62 b) moja wersja nr 970}

Rozwiązać nierówności $(x+20)(2-x)(x+1)\ge0$.
\zadStop
\rozwStart{Patryk Wirkus}{}
Miejsca zerowe naszego wielomianu to: $-20, 2, -1$.\\
Wielomian jest stopnia nieparzystego, ponadto znak współczynnika przy\linebreak najwyższej potędze x jest ujemny.\\ W związku z tym wykres wielomianu zaczyna się od lewej strony powyżej osi OX. A więc $$x \in (-\infty,-20) \cup (-1,2).$$
\rozwStop
\odpStart
$x \in (-\infty,-20) \cup (-1,2)$
\odpStop
\testStart
A.$x \in (-\infty,-20) \cup (-1,2)$\\
B.$x \in (-\infty,-20) \cup (-1,2]$\\
C.$x \in (-\infty,-20) \cup [-1,2)$\\
D.$x \in (-\infty,-20] \cup (-1,2)$\\
E.$x \in (-\infty,-20] \cup (-1,2]$\\
F.$x \in (-\infty,-20] \cup [-1,2)$\\
G.$x \in (-\infty,-20) \cup [-1,2]$\\
H.$x \in (-\infty,-20] \cup [-1,2]$
\testStop
\kluczStart
A
\kluczStop



\zadStart{Zadanie z Wikieł Z 1.62 b) moja wersja nr 971}

Rozwiązać nierówności $(x+20)(3-x)(x+1)\ge0$.
\zadStop
\rozwStart{Patryk Wirkus}{}
Miejsca zerowe naszego wielomianu to: $-20, 3, -1$.\\
Wielomian jest stopnia nieparzystego, ponadto znak współczynnika przy\linebreak najwyższej potędze x jest ujemny.\\ W związku z tym wykres wielomianu zaczyna się od lewej strony powyżej osi OX. A więc $$x \in (-\infty,-20) \cup (-1,3).$$
\rozwStop
\odpStart
$x \in (-\infty,-20) \cup (-1,3)$
\odpStop
\testStart
A.$x \in (-\infty,-20) \cup (-1,3)$\\
B.$x \in (-\infty,-20) \cup (-1,3]$\\
C.$x \in (-\infty,-20) \cup [-1,3)$\\
D.$x \in (-\infty,-20] \cup (-1,3)$\\
E.$x \in (-\infty,-20] \cup (-1,3]$\\
F.$x \in (-\infty,-20] \cup [-1,3)$\\
G.$x \in (-\infty,-20) \cup [-1,3]$\\
H.$x \in (-\infty,-20] \cup [-1,3]$
\testStop
\kluczStart
A
\kluczStop



\zadStart{Zadanie z Wikieł Z 1.62 b) moja wersja nr 972}

Rozwiązać nierówności $(x+20)(3-x)(x+2)\ge0$.
\zadStop
\rozwStart{Patryk Wirkus}{}
Miejsca zerowe naszego wielomianu to: $-20, 3, -2$.\\
Wielomian jest stopnia nieparzystego, ponadto znak współczynnika przy\linebreak najwyższej potędze x jest ujemny.\\ W związku z tym wykres wielomianu zaczyna się od lewej strony powyżej osi OX. A więc $$x \in (-\infty,-20) \cup (-2,3).$$
\rozwStop
\odpStart
$x \in (-\infty,-20) \cup (-2,3)$
\odpStop
\testStart
A.$x \in (-\infty,-20) \cup (-2,3)$\\
B.$x \in (-\infty,-20) \cup (-2,3]$\\
C.$x \in (-\infty,-20) \cup [-2,3)$\\
D.$x \in (-\infty,-20] \cup (-2,3)$\\
E.$x \in (-\infty,-20] \cup (-2,3]$\\
F.$x \in (-\infty,-20] \cup [-2,3)$\\
G.$x \in (-\infty,-20) \cup [-2,3]$\\
H.$x \in (-\infty,-20] \cup [-2,3]$
\testStop
\kluczStart
A
\kluczStop



\zadStart{Zadanie z Wikieł Z 1.62 b) moja wersja nr 973}

Rozwiązać nierówności $(x+20)(4-x)(x+1)\ge0$.
\zadStop
\rozwStart{Patryk Wirkus}{}
Miejsca zerowe naszego wielomianu to: $-20, 4, -1$.\\
Wielomian jest stopnia nieparzystego, ponadto znak współczynnika przy\linebreak najwyższej potędze x jest ujemny.\\ W związku z tym wykres wielomianu zaczyna się od lewej strony powyżej osi OX. A więc $$x \in (-\infty,-20) \cup (-1,4).$$
\rozwStop
\odpStart
$x \in (-\infty,-20) \cup (-1,4)$
\odpStop
\testStart
A.$x \in (-\infty,-20) \cup (-1,4)$\\
B.$x \in (-\infty,-20) \cup (-1,4]$\\
C.$x \in (-\infty,-20) \cup [-1,4)$\\
D.$x \in (-\infty,-20] \cup (-1,4)$\\
E.$x \in (-\infty,-20] \cup (-1,4]$\\
F.$x \in (-\infty,-20] \cup [-1,4)$\\
G.$x \in (-\infty,-20) \cup [-1,4]$\\
H.$x \in (-\infty,-20] \cup [-1,4]$
\testStop
\kluczStart
A
\kluczStop



\zadStart{Zadanie z Wikieł Z 1.62 b) moja wersja nr 974}

Rozwiązać nierówności $(x+20)(4-x)(x+2)\ge0$.
\zadStop
\rozwStart{Patryk Wirkus}{}
Miejsca zerowe naszego wielomianu to: $-20, 4, -2$.\\
Wielomian jest stopnia nieparzystego, ponadto znak współczynnika przy\linebreak najwyższej potędze x jest ujemny.\\ W związku z tym wykres wielomianu zaczyna się od lewej strony powyżej osi OX. A więc $$x \in (-\infty,-20) \cup (-2,4).$$
\rozwStop
\odpStart
$x \in (-\infty,-20) \cup (-2,4)$
\odpStop
\testStart
A.$x \in (-\infty,-20) \cup (-2,4)$\\
B.$x \in (-\infty,-20) \cup (-2,4]$\\
C.$x \in (-\infty,-20) \cup [-2,4)$\\
D.$x \in (-\infty,-20] \cup (-2,4)$\\
E.$x \in (-\infty,-20] \cup (-2,4]$\\
F.$x \in (-\infty,-20] \cup [-2,4)$\\
G.$x \in (-\infty,-20) \cup [-2,4]$\\
H.$x \in (-\infty,-20] \cup [-2,4]$
\testStop
\kluczStart
A
\kluczStop



\zadStart{Zadanie z Wikieł Z 1.62 b) moja wersja nr 975}

Rozwiązać nierówności $(x+20)(4-x)(x+3)\ge0$.
\zadStop
\rozwStart{Patryk Wirkus}{}
Miejsca zerowe naszego wielomianu to: $-20, 4, -3$.\\
Wielomian jest stopnia nieparzystego, ponadto znak współczynnika przy\linebreak najwyższej potędze x jest ujemny.\\ W związku z tym wykres wielomianu zaczyna się od lewej strony powyżej osi OX. A więc $$x \in (-\infty,-20) \cup (-3,4).$$
\rozwStop
\odpStart
$x \in (-\infty,-20) \cup (-3,4)$
\odpStop
\testStart
A.$x \in (-\infty,-20) \cup (-3,4)$\\
B.$x \in (-\infty,-20) \cup (-3,4]$\\
C.$x \in (-\infty,-20) \cup [-3,4)$\\
D.$x \in (-\infty,-20] \cup (-3,4)$\\
E.$x \in (-\infty,-20] \cup (-3,4]$\\
F.$x \in (-\infty,-20] \cup [-3,4)$\\
G.$x \in (-\infty,-20) \cup [-3,4]$\\
H.$x \in (-\infty,-20] \cup [-3,4]$
\testStop
\kluczStart
A
\kluczStop



\zadStart{Zadanie z Wikieł Z 1.62 b) moja wersja nr 976}

Rozwiązać nierówności $(x+20)(5-x)(x+1)\ge0$.
\zadStop
\rozwStart{Patryk Wirkus}{}
Miejsca zerowe naszego wielomianu to: $-20, 5, -1$.\\
Wielomian jest stopnia nieparzystego, ponadto znak współczynnika przy\linebreak najwyższej potędze x jest ujemny.\\ W związku z tym wykres wielomianu zaczyna się od lewej strony powyżej osi OX. A więc $$x \in (-\infty,-20) \cup (-1,5).$$
\rozwStop
\odpStart
$x \in (-\infty,-20) \cup (-1,5)$
\odpStop
\testStart
A.$x \in (-\infty,-20) \cup (-1,5)$\\
B.$x \in (-\infty,-20) \cup (-1,5]$\\
C.$x \in (-\infty,-20) \cup [-1,5)$\\
D.$x \in (-\infty,-20] \cup (-1,5)$\\
E.$x \in (-\infty,-20] \cup (-1,5]$\\
F.$x \in (-\infty,-20] \cup [-1,5)$\\
G.$x \in (-\infty,-20) \cup [-1,5]$\\
H.$x \in (-\infty,-20] \cup [-1,5]$
\testStop
\kluczStart
A
\kluczStop



\zadStart{Zadanie z Wikieł Z 1.62 b) moja wersja nr 977}

Rozwiązać nierówności $(x+20)(5-x)(x+2)\ge0$.
\zadStop
\rozwStart{Patryk Wirkus}{}
Miejsca zerowe naszego wielomianu to: $-20, 5, -2$.\\
Wielomian jest stopnia nieparzystego, ponadto znak współczynnika przy\linebreak najwyższej potędze x jest ujemny.\\ W związku z tym wykres wielomianu zaczyna się od lewej strony powyżej osi OX. A więc $$x \in (-\infty,-20) \cup (-2,5).$$
\rozwStop
\odpStart
$x \in (-\infty,-20) \cup (-2,5)$
\odpStop
\testStart
A.$x \in (-\infty,-20) \cup (-2,5)$\\
B.$x \in (-\infty,-20) \cup (-2,5]$\\
C.$x \in (-\infty,-20) \cup [-2,5)$\\
D.$x \in (-\infty,-20] \cup (-2,5)$\\
E.$x \in (-\infty,-20] \cup (-2,5]$\\
F.$x \in (-\infty,-20] \cup [-2,5)$\\
G.$x \in (-\infty,-20) \cup [-2,5]$\\
H.$x \in (-\infty,-20] \cup [-2,5]$
\testStop
\kluczStart
A
\kluczStop



\zadStart{Zadanie z Wikieł Z 1.62 b) moja wersja nr 978}

Rozwiązać nierówności $(x+20)(5-x)(x+3)\ge0$.
\zadStop
\rozwStart{Patryk Wirkus}{}
Miejsca zerowe naszego wielomianu to: $-20, 5, -3$.\\
Wielomian jest stopnia nieparzystego, ponadto znak współczynnika przy\linebreak najwyższej potędze x jest ujemny.\\ W związku z tym wykres wielomianu zaczyna się od lewej strony powyżej osi OX. A więc $$x \in (-\infty,-20) \cup (-3,5).$$
\rozwStop
\odpStart
$x \in (-\infty,-20) \cup (-3,5)$
\odpStop
\testStart
A.$x \in (-\infty,-20) \cup (-3,5)$\\
B.$x \in (-\infty,-20) \cup (-3,5]$\\
C.$x \in (-\infty,-20) \cup [-3,5)$\\
D.$x \in (-\infty,-20] \cup (-3,5)$\\
E.$x \in (-\infty,-20] \cup (-3,5]$\\
F.$x \in (-\infty,-20] \cup [-3,5)$\\
G.$x \in (-\infty,-20) \cup [-3,5]$\\
H.$x \in (-\infty,-20] \cup [-3,5]$
\testStop
\kluczStart
A
\kluczStop



\zadStart{Zadanie z Wikieł Z 1.62 b) moja wersja nr 979}

Rozwiązać nierówności $(x+20)(5-x)(x+4)\ge0$.
\zadStop
\rozwStart{Patryk Wirkus}{}
Miejsca zerowe naszego wielomianu to: $-20, 5, -4$.\\
Wielomian jest stopnia nieparzystego, ponadto znak współczynnika przy\linebreak najwyższej potędze x jest ujemny.\\ W związku z tym wykres wielomianu zaczyna się od lewej strony powyżej osi OX. A więc $$x \in (-\infty,-20) \cup (-4,5).$$
\rozwStop
\odpStart
$x \in (-\infty,-20) \cup (-4,5)$
\odpStop
\testStart
A.$x \in (-\infty,-20) \cup (-4,5)$\\
B.$x \in (-\infty,-20) \cup (-4,5]$\\
C.$x \in (-\infty,-20) \cup [-4,5)$\\
D.$x \in (-\infty,-20] \cup (-4,5)$\\
E.$x \in (-\infty,-20] \cup (-4,5]$\\
F.$x \in (-\infty,-20] \cup [-4,5)$\\
G.$x \in (-\infty,-20) \cup [-4,5]$\\
H.$x \in (-\infty,-20] \cup [-4,5]$
\testStop
\kluczStart
A
\kluczStop



\zadStart{Zadanie z Wikieł Z 1.62 b) moja wersja nr 980}

Rozwiązać nierówności $(x+20)(6-x)(x+1)\ge0$.
\zadStop
\rozwStart{Patryk Wirkus}{}
Miejsca zerowe naszego wielomianu to: $-20, 6, -1$.\\
Wielomian jest stopnia nieparzystego, ponadto znak współczynnika przy\linebreak najwyższej potędze x jest ujemny.\\ W związku z tym wykres wielomianu zaczyna się od lewej strony powyżej osi OX. A więc $$x \in (-\infty,-20) \cup (-1,6).$$
\rozwStop
\odpStart
$x \in (-\infty,-20) \cup (-1,6)$
\odpStop
\testStart
A.$x \in (-\infty,-20) \cup (-1,6)$\\
B.$x \in (-\infty,-20) \cup (-1,6]$\\
C.$x \in (-\infty,-20) \cup [-1,6)$\\
D.$x \in (-\infty,-20] \cup (-1,6)$\\
E.$x \in (-\infty,-20] \cup (-1,6]$\\
F.$x \in (-\infty,-20] \cup [-1,6)$\\
G.$x \in (-\infty,-20) \cup [-1,6]$\\
H.$x \in (-\infty,-20] \cup [-1,6]$
\testStop
\kluczStart
A
\kluczStop



\zadStart{Zadanie z Wikieł Z 1.62 b) moja wersja nr 981}

Rozwiązać nierówności $(x+20)(6-x)(x+2)\ge0$.
\zadStop
\rozwStart{Patryk Wirkus}{}
Miejsca zerowe naszego wielomianu to: $-20, 6, -2$.\\
Wielomian jest stopnia nieparzystego, ponadto znak współczynnika przy\linebreak najwyższej potędze x jest ujemny.\\ W związku z tym wykres wielomianu zaczyna się od lewej strony powyżej osi OX. A więc $$x \in (-\infty,-20) \cup (-2,6).$$
\rozwStop
\odpStart
$x \in (-\infty,-20) \cup (-2,6)$
\odpStop
\testStart
A.$x \in (-\infty,-20) \cup (-2,6)$\\
B.$x \in (-\infty,-20) \cup (-2,6]$\\
C.$x \in (-\infty,-20) \cup [-2,6)$\\
D.$x \in (-\infty,-20] \cup (-2,6)$\\
E.$x \in (-\infty,-20] \cup (-2,6]$\\
F.$x \in (-\infty,-20] \cup [-2,6)$\\
G.$x \in (-\infty,-20) \cup [-2,6]$\\
H.$x \in (-\infty,-20] \cup [-2,6]$
\testStop
\kluczStart
A
\kluczStop



\zadStart{Zadanie z Wikieł Z 1.62 b) moja wersja nr 982}

Rozwiązać nierówności $(x+20)(6-x)(x+3)\ge0$.
\zadStop
\rozwStart{Patryk Wirkus}{}
Miejsca zerowe naszego wielomianu to: $-20, 6, -3$.\\
Wielomian jest stopnia nieparzystego, ponadto znak współczynnika przy\linebreak najwyższej potędze x jest ujemny.\\ W związku z tym wykres wielomianu zaczyna się od lewej strony powyżej osi OX. A więc $$x \in (-\infty,-20) \cup (-3,6).$$
\rozwStop
\odpStart
$x \in (-\infty,-20) \cup (-3,6)$
\odpStop
\testStart
A.$x \in (-\infty,-20) \cup (-3,6)$\\
B.$x \in (-\infty,-20) \cup (-3,6]$\\
C.$x \in (-\infty,-20) \cup [-3,6)$\\
D.$x \in (-\infty,-20] \cup (-3,6)$\\
E.$x \in (-\infty,-20] \cup (-3,6]$\\
F.$x \in (-\infty,-20] \cup [-3,6)$\\
G.$x \in (-\infty,-20) \cup [-3,6]$\\
H.$x \in (-\infty,-20] \cup [-3,6]$
\testStop
\kluczStart
A
\kluczStop



\zadStart{Zadanie z Wikieł Z 1.62 b) moja wersja nr 983}

Rozwiązać nierówności $(x+20)(6-x)(x+4)\ge0$.
\zadStop
\rozwStart{Patryk Wirkus}{}
Miejsca zerowe naszego wielomianu to: $-20, 6, -4$.\\
Wielomian jest stopnia nieparzystego, ponadto znak współczynnika przy\linebreak najwyższej potędze x jest ujemny.\\ W związku z tym wykres wielomianu zaczyna się od lewej strony powyżej osi OX. A więc $$x \in (-\infty,-20) \cup (-4,6).$$
\rozwStop
\odpStart
$x \in (-\infty,-20) \cup (-4,6)$
\odpStop
\testStart
A.$x \in (-\infty,-20) \cup (-4,6)$\\
B.$x \in (-\infty,-20) \cup (-4,6]$\\
C.$x \in (-\infty,-20) \cup [-4,6)$\\
D.$x \in (-\infty,-20] \cup (-4,6)$\\
E.$x \in (-\infty,-20] \cup (-4,6]$\\
F.$x \in (-\infty,-20] \cup [-4,6)$\\
G.$x \in (-\infty,-20) \cup [-4,6]$\\
H.$x \in (-\infty,-20] \cup [-4,6]$
\testStop
\kluczStart
A
\kluczStop



\zadStart{Zadanie z Wikieł Z 1.62 b) moja wersja nr 984}

Rozwiązać nierówności $(x+20)(6-x)(x+5)\ge0$.
\zadStop
\rozwStart{Patryk Wirkus}{}
Miejsca zerowe naszego wielomianu to: $-20, 6, -5$.\\
Wielomian jest stopnia nieparzystego, ponadto znak współczynnika przy\linebreak najwyższej potędze x jest ujemny.\\ W związku z tym wykres wielomianu zaczyna się od lewej strony powyżej osi OX. A więc $$x \in (-\infty,-20) \cup (-5,6).$$
\rozwStop
\odpStart
$x \in (-\infty,-20) \cup (-5,6)$
\odpStop
\testStart
A.$x \in (-\infty,-20) \cup (-5,6)$\\
B.$x \in (-\infty,-20) \cup (-5,6]$\\
C.$x \in (-\infty,-20) \cup [-5,6)$\\
D.$x \in (-\infty,-20] \cup (-5,6)$\\
E.$x \in (-\infty,-20] \cup (-5,6]$\\
F.$x \in (-\infty,-20] \cup [-5,6)$\\
G.$x \in (-\infty,-20) \cup [-5,6]$\\
H.$x \in (-\infty,-20] \cup [-5,6]$
\testStop
\kluczStart
A
\kluczStop



\zadStart{Zadanie z Wikieł Z 1.62 b) moja wersja nr 985}

Rozwiązać nierówności $(x+20)(7-x)(x+1)\ge0$.
\zadStop
\rozwStart{Patryk Wirkus}{}
Miejsca zerowe naszego wielomianu to: $-20, 7, -1$.\\
Wielomian jest stopnia nieparzystego, ponadto znak współczynnika przy\linebreak najwyższej potędze x jest ujemny.\\ W związku z tym wykres wielomianu zaczyna się od lewej strony powyżej osi OX. A więc $$x \in (-\infty,-20) \cup (-1,7).$$
\rozwStop
\odpStart
$x \in (-\infty,-20) \cup (-1,7)$
\odpStop
\testStart
A.$x \in (-\infty,-20) \cup (-1,7)$\\
B.$x \in (-\infty,-20) \cup (-1,7]$\\
C.$x \in (-\infty,-20) \cup [-1,7)$\\
D.$x \in (-\infty,-20] \cup (-1,7)$\\
E.$x \in (-\infty,-20] \cup (-1,7]$\\
F.$x \in (-\infty,-20] \cup [-1,7)$\\
G.$x \in (-\infty,-20) \cup [-1,7]$\\
H.$x \in (-\infty,-20] \cup [-1,7]$
\testStop
\kluczStart
A
\kluczStop



\zadStart{Zadanie z Wikieł Z 1.62 b) moja wersja nr 986}

Rozwiązać nierówności $(x+20)(7-x)(x+2)\ge0$.
\zadStop
\rozwStart{Patryk Wirkus}{}
Miejsca zerowe naszego wielomianu to: $-20, 7, -2$.\\
Wielomian jest stopnia nieparzystego, ponadto znak współczynnika przy\linebreak najwyższej potędze x jest ujemny.\\ W związku z tym wykres wielomianu zaczyna się od lewej strony powyżej osi OX. A więc $$x \in (-\infty,-20) \cup (-2,7).$$
\rozwStop
\odpStart
$x \in (-\infty,-20) \cup (-2,7)$
\odpStop
\testStart
A.$x \in (-\infty,-20) \cup (-2,7)$\\
B.$x \in (-\infty,-20) \cup (-2,7]$\\
C.$x \in (-\infty,-20) \cup [-2,7)$\\
D.$x \in (-\infty,-20] \cup (-2,7)$\\
E.$x \in (-\infty,-20] \cup (-2,7]$\\
F.$x \in (-\infty,-20] \cup [-2,7)$\\
G.$x \in (-\infty,-20) \cup [-2,7]$\\
H.$x \in (-\infty,-20] \cup [-2,7]$
\testStop
\kluczStart
A
\kluczStop



\zadStart{Zadanie z Wikieł Z 1.62 b) moja wersja nr 987}

Rozwiązać nierówności $(x+20)(7-x)(x+3)\ge0$.
\zadStop
\rozwStart{Patryk Wirkus}{}
Miejsca zerowe naszego wielomianu to: $-20, 7, -3$.\\
Wielomian jest stopnia nieparzystego, ponadto znak współczynnika przy\linebreak najwyższej potędze x jest ujemny.\\ W związku z tym wykres wielomianu zaczyna się od lewej strony powyżej osi OX. A więc $$x \in (-\infty,-20) \cup (-3,7).$$
\rozwStop
\odpStart
$x \in (-\infty,-20) \cup (-3,7)$
\odpStop
\testStart
A.$x \in (-\infty,-20) \cup (-3,7)$\\
B.$x \in (-\infty,-20) \cup (-3,7]$\\
C.$x \in (-\infty,-20) \cup [-3,7)$\\
D.$x \in (-\infty,-20] \cup (-3,7)$\\
E.$x \in (-\infty,-20] \cup (-3,7]$\\
F.$x \in (-\infty,-20] \cup [-3,7)$\\
G.$x \in (-\infty,-20) \cup [-3,7]$\\
H.$x \in (-\infty,-20] \cup [-3,7]$
\testStop
\kluczStart
A
\kluczStop



\zadStart{Zadanie z Wikieł Z 1.62 b) moja wersja nr 988}

Rozwiązać nierówności $(x+20)(7-x)(x+4)\ge0$.
\zadStop
\rozwStart{Patryk Wirkus}{}
Miejsca zerowe naszego wielomianu to: $-20, 7, -4$.\\
Wielomian jest stopnia nieparzystego, ponadto znak współczynnika przy\linebreak najwyższej potędze x jest ujemny.\\ W związku z tym wykres wielomianu zaczyna się od lewej strony powyżej osi OX. A więc $$x \in (-\infty,-20) \cup (-4,7).$$
\rozwStop
\odpStart
$x \in (-\infty,-20) \cup (-4,7)$
\odpStop
\testStart
A.$x \in (-\infty,-20) \cup (-4,7)$\\
B.$x \in (-\infty,-20) \cup (-4,7]$\\
C.$x \in (-\infty,-20) \cup [-4,7)$\\
D.$x \in (-\infty,-20] \cup (-4,7)$\\
E.$x \in (-\infty,-20] \cup (-4,7]$\\
F.$x \in (-\infty,-20] \cup [-4,7)$\\
G.$x \in (-\infty,-20) \cup [-4,7]$\\
H.$x \in (-\infty,-20] \cup [-4,7]$
\testStop
\kluczStart
A
\kluczStop



\zadStart{Zadanie z Wikieł Z 1.62 b) moja wersja nr 989}

Rozwiązać nierówności $(x+20)(7-x)(x+5)\ge0$.
\zadStop
\rozwStart{Patryk Wirkus}{}
Miejsca zerowe naszego wielomianu to: $-20, 7, -5$.\\
Wielomian jest stopnia nieparzystego, ponadto znak współczynnika przy\linebreak najwyższej potędze x jest ujemny.\\ W związku z tym wykres wielomianu zaczyna się od lewej strony powyżej osi OX. A więc $$x \in (-\infty,-20) \cup (-5,7).$$
\rozwStop
\odpStart
$x \in (-\infty,-20) \cup (-5,7)$
\odpStop
\testStart
A.$x \in (-\infty,-20) \cup (-5,7)$\\
B.$x \in (-\infty,-20) \cup (-5,7]$\\
C.$x \in (-\infty,-20) \cup [-5,7)$\\
D.$x \in (-\infty,-20] \cup (-5,7)$\\
E.$x \in (-\infty,-20] \cup (-5,7]$\\
F.$x \in (-\infty,-20] \cup [-5,7)$\\
G.$x \in (-\infty,-20) \cup [-5,7]$\\
H.$x \in (-\infty,-20] \cup [-5,7]$
\testStop
\kluczStart
A
\kluczStop



\zadStart{Zadanie z Wikieł Z 1.62 b) moja wersja nr 990}

Rozwiązać nierówności $(x+20)(7-x)(x+6)\ge0$.
\zadStop
\rozwStart{Patryk Wirkus}{}
Miejsca zerowe naszego wielomianu to: $-20, 7, -6$.\\
Wielomian jest stopnia nieparzystego, ponadto znak współczynnika przy\linebreak najwyższej potędze x jest ujemny.\\ W związku z tym wykres wielomianu zaczyna się od lewej strony powyżej osi OX. A więc $$x \in (-\infty,-20) \cup (-6,7).$$
\rozwStop
\odpStart
$x \in (-\infty,-20) \cup (-6,7)$
\odpStop
\testStart
A.$x \in (-\infty,-20) \cup (-6,7)$\\
B.$x \in (-\infty,-20) \cup (-6,7]$\\
C.$x \in (-\infty,-20) \cup [-6,7)$\\
D.$x \in (-\infty,-20] \cup (-6,7)$\\
E.$x \in (-\infty,-20] \cup (-6,7]$\\
F.$x \in (-\infty,-20] \cup [-6,7)$\\
G.$x \in (-\infty,-20) \cup [-6,7]$\\
H.$x \in (-\infty,-20] \cup [-6,7]$
\testStop
\kluczStart
A
\kluczStop



\zadStart{Zadanie z Wikieł Z 1.62 b) moja wersja nr 991}

Rozwiązać nierówności $(x+20)(8-x)(x+1)\ge0$.
\zadStop
\rozwStart{Patryk Wirkus}{}
Miejsca zerowe naszego wielomianu to: $-20, 8, -1$.\\
Wielomian jest stopnia nieparzystego, ponadto znak współczynnika przy\linebreak najwyższej potędze x jest ujemny.\\ W związku z tym wykres wielomianu zaczyna się od lewej strony powyżej osi OX. A więc $$x \in (-\infty,-20) \cup (-1,8).$$
\rozwStop
\odpStart
$x \in (-\infty,-20) \cup (-1,8)$
\odpStop
\testStart
A.$x \in (-\infty,-20) \cup (-1,8)$\\
B.$x \in (-\infty,-20) \cup (-1,8]$\\
C.$x \in (-\infty,-20) \cup [-1,8)$\\
D.$x \in (-\infty,-20] \cup (-1,8)$\\
E.$x \in (-\infty,-20] \cup (-1,8]$\\
F.$x \in (-\infty,-20] \cup [-1,8)$\\
G.$x \in (-\infty,-20) \cup [-1,8]$\\
H.$x \in (-\infty,-20] \cup [-1,8]$
\testStop
\kluczStart
A
\kluczStop



\zadStart{Zadanie z Wikieł Z 1.62 b) moja wersja nr 992}

Rozwiązać nierówności $(x+20)(8-x)(x+2)\ge0$.
\zadStop
\rozwStart{Patryk Wirkus}{}
Miejsca zerowe naszego wielomianu to: $-20, 8, -2$.\\
Wielomian jest stopnia nieparzystego, ponadto znak współczynnika przy\linebreak najwyższej potędze x jest ujemny.\\ W związku z tym wykres wielomianu zaczyna się od lewej strony powyżej osi OX. A więc $$x \in (-\infty,-20) \cup (-2,8).$$
\rozwStop
\odpStart
$x \in (-\infty,-20) \cup (-2,8)$
\odpStop
\testStart
A.$x \in (-\infty,-20) \cup (-2,8)$\\
B.$x \in (-\infty,-20) \cup (-2,8]$\\
C.$x \in (-\infty,-20) \cup [-2,8)$\\
D.$x \in (-\infty,-20] \cup (-2,8)$\\
E.$x \in (-\infty,-20] \cup (-2,8]$\\
F.$x \in (-\infty,-20] \cup [-2,8)$\\
G.$x \in (-\infty,-20) \cup [-2,8]$\\
H.$x \in (-\infty,-20] \cup [-2,8]$
\testStop
\kluczStart
A
\kluczStop



\zadStart{Zadanie z Wikieł Z 1.62 b) moja wersja nr 993}

Rozwiązać nierówności $(x+20)(8-x)(x+3)\ge0$.
\zadStop
\rozwStart{Patryk Wirkus}{}
Miejsca zerowe naszego wielomianu to: $-20, 8, -3$.\\
Wielomian jest stopnia nieparzystego, ponadto znak współczynnika przy\linebreak najwyższej potędze x jest ujemny.\\ W związku z tym wykres wielomianu zaczyna się od lewej strony powyżej osi OX. A więc $$x \in (-\infty,-20) \cup (-3,8).$$
\rozwStop
\odpStart
$x \in (-\infty,-20) \cup (-3,8)$
\odpStop
\testStart
A.$x \in (-\infty,-20) \cup (-3,8)$\\
B.$x \in (-\infty,-20) \cup (-3,8]$\\
C.$x \in (-\infty,-20) \cup [-3,8)$\\
D.$x \in (-\infty,-20] \cup (-3,8)$\\
E.$x \in (-\infty,-20] \cup (-3,8]$\\
F.$x \in (-\infty,-20] \cup [-3,8)$\\
G.$x \in (-\infty,-20) \cup [-3,8]$\\
H.$x \in (-\infty,-20] \cup [-3,8]$
\testStop
\kluczStart
A
\kluczStop



\zadStart{Zadanie z Wikieł Z 1.62 b) moja wersja nr 994}

Rozwiązać nierówności $(x+20)(8-x)(x+4)\ge0$.
\zadStop
\rozwStart{Patryk Wirkus}{}
Miejsca zerowe naszego wielomianu to: $-20, 8, -4$.\\
Wielomian jest stopnia nieparzystego, ponadto znak współczynnika przy\linebreak najwyższej potędze x jest ujemny.\\ W związku z tym wykres wielomianu zaczyna się od lewej strony powyżej osi OX. A więc $$x \in (-\infty,-20) \cup (-4,8).$$
\rozwStop
\odpStart
$x \in (-\infty,-20) \cup (-4,8)$
\odpStop
\testStart
A.$x \in (-\infty,-20) \cup (-4,8)$\\
B.$x \in (-\infty,-20) \cup (-4,8]$\\
C.$x \in (-\infty,-20) \cup [-4,8)$\\
D.$x \in (-\infty,-20] \cup (-4,8)$\\
E.$x \in (-\infty,-20] \cup (-4,8]$\\
F.$x \in (-\infty,-20] \cup [-4,8)$\\
G.$x \in (-\infty,-20) \cup [-4,8]$\\
H.$x \in (-\infty,-20] \cup [-4,8]$
\testStop
\kluczStart
A
\kluczStop



\zadStart{Zadanie z Wikieł Z 1.62 b) moja wersja nr 995}

Rozwiązać nierówności $(x+20)(8-x)(x+5)\ge0$.
\zadStop
\rozwStart{Patryk Wirkus}{}
Miejsca zerowe naszego wielomianu to: $-20, 8, -5$.\\
Wielomian jest stopnia nieparzystego, ponadto znak współczynnika przy\linebreak najwyższej potędze x jest ujemny.\\ W związku z tym wykres wielomianu zaczyna się od lewej strony powyżej osi OX. A więc $$x \in (-\infty,-20) \cup (-5,8).$$
\rozwStop
\odpStart
$x \in (-\infty,-20) \cup (-5,8)$
\odpStop
\testStart
A.$x \in (-\infty,-20) \cup (-5,8)$\\
B.$x \in (-\infty,-20) \cup (-5,8]$\\
C.$x \in (-\infty,-20) \cup [-5,8)$\\
D.$x \in (-\infty,-20] \cup (-5,8)$\\
E.$x \in (-\infty,-20] \cup (-5,8]$\\
F.$x \in (-\infty,-20] \cup [-5,8)$\\
G.$x \in (-\infty,-20) \cup [-5,8]$\\
H.$x \in (-\infty,-20] \cup [-5,8]$
\testStop
\kluczStart
A
\kluczStop



\zadStart{Zadanie z Wikieł Z 1.62 b) moja wersja nr 996}

Rozwiązać nierówności $(x+20)(8-x)(x+6)\ge0$.
\zadStop
\rozwStart{Patryk Wirkus}{}
Miejsca zerowe naszego wielomianu to: $-20, 8, -6$.\\
Wielomian jest stopnia nieparzystego, ponadto znak współczynnika przy\linebreak najwyższej potędze x jest ujemny.\\ W związku z tym wykres wielomianu zaczyna się od lewej strony powyżej osi OX. A więc $$x \in (-\infty,-20) \cup (-6,8).$$
\rozwStop
\odpStart
$x \in (-\infty,-20) \cup (-6,8)$
\odpStop
\testStart
A.$x \in (-\infty,-20) \cup (-6,8)$\\
B.$x \in (-\infty,-20) \cup (-6,8]$\\
C.$x \in (-\infty,-20) \cup [-6,8)$\\
D.$x \in (-\infty,-20] \cup (-6,8)$\\
E.$x \in (-\infty,-20] \cup (-6,8]$\\
F.$x \in (-\infty,-20] \cup [-6,8)$\\
G.$x \in (-\infty,-20) \cup [-6,8]$\\
H.$x \in (-\infty,-20] \cup [-6,8]$
\testStop
\kluczStart
A
\kluczStop



\zadStart{Zadanie z Wikieł Z 1.62 b) moja wersja nr 997}

Rozwiązać nierówności $(x+20)(8-x)(x+7)\ge0$.
\zadStop
\rozwStart{Patryk Wirkus}{}
Miejsca zerowe naszego wielomianu to: $-20, 8, -7$.\\
Wielomian jest stopnia nieparzystego, ponadto znak współczynnika przy\linebreak najwyższej potędze x jest ujemny.\\ W związku z tym wykres wielomianu zaczyna się od lewej strony powyżej osi OX. A więc $$x \in (-\infty,-20) \cup (-7,8).$$
\rozwStop
\odpStart
$x \in (-\infty,-20) \cup (-7,8)$
\odpStop
\testStart
A.$x \in (-\infty,-20) \cup (-7,8)$\\
B.$x \in (-\infty,-20) \cup (-7,8]$\\
C.$x \in (-\infty,-20) \cup [-7,8)$\\
D.$x \in (-\infty,-20] \cup (-7,8)$\\
E.$x \in (-\infty,-20] \cup (-7,8]$\\
F.$x \in (-\infty,-20] \cup [-7,8)$\\
G.$x \in (-\infty,-20) \cup [-7,8]$\\
H.$x \in (-\infty,-20] \cup [-7,8]$
\testStop
\kluczStart
A
\kluczStop



\zadStart{Zadanie z Wikieł Z 1.62 b) moja wersja nr 998}

Rozwiązać nierówności $(x+20)(9-x)(x+1)\ge0$.
\zadStop
\rozwStart{Patryk Wirkus}{}
Miejsca zerowe naszego wielomianu to: $-20, 9, -1$.\\
Wielomian jest stopnia nieparzystego, ponadto znak współczynnika przy\linebreak najwyższej potędze x jest ujemny.\\ W związku z tym wykres wielomianu zaczyna się od lewej strony powyżej osi OX. A więc $$x \in (-\infty,-20) \cup (-1,9).$$
\rozwStop
\odpStart
$x \in (-\infty,-20) \cup (-1,9)$
\odpStop
\testStart
A.$x \in (-\infty,-20) \cup (-1,9)$\\
B.$x \in (-\infty,-20) \cup (-1,9]$\\
C.$x \in (-\infty,-20) \cup [-1,9)$\\
D.$x \in (-\infty,-20] \cup (-1,9)$\\
E.$x \in (-\infty,-20] \cup (-1,9]$\\
F.$x \in (-\infty,-20] \cup [-1,9)$\\
G.$x \in (-\infty,-20) \cup [-1,9]$\\
H.$x \in (-\infty,-20] \cup [-1,9]$
\testStop
\kluczStart
A
\kluczStop



\zadStart{Zadanie z Wikieł Z 1.62 b) moja wersja nr 999}

Rozwiązać nierówności $(x+20)(9-x)(x+2)\ge0$.
\zadStop
\rozwStart{Patryk Wirkus}{}
Miejsca zerowe naszego wielomianu to: $-20, 9, -2$.\\
Wielomian jest stopnia nieparzystego, ponadto znak współczynnika przy\linebreak najwyższej potędze x jest ujemny.\\ W związku z tym wykres wielomianu zaczyna się od lewej strony powyżej osi OX. A więc $$x \in (-\infty,-20) \cup (-2,9).$$
\rozwStop
\odpStart
$x \in (-\infty,-20) \cup (-2,9)$
\odpStop
\testStart
A.$x \in (-\infty,-20) \cup (-2,9)$\\
B.$x \in (-\infty,-20) \cup (-2,9]$\\
C.$x \in (-\infty,-20) \cup [-2,9)$\\
D.$x \in (-\infty,-20] \cup (-2,9)$\\
E.$x \in (-\infty,-20] \cup (-2,9]$\\
F.$x \in (-\infty,-20] \cup [-2,9)$\\
G.$x \in (-\infty,-20) \cup [-2,9]$\\
H.$x \in (-\infty,-20] \cup [-2,9]$
\testStop
\kluczStart
A
\kluczStop



\zadStart{Zadanie z Wikieł Z 1.62 b) moja wersja nr 1000}

Rozwiązać nierówności $(x+20)(9-x)(x+3)\ge0$.
\zadStop
\rozwStart{Patryk Wirkus}{}
Miejsca zerowe naszego wielomianu to: $-20, 9, -3$.\\
Wielomian jest stopnia nieparzystego, ponadto znak współczynnika przy\linebreak najwyższej potędze x jest ujemny.\\ W związku z tym wykres wielomianu zaczyna się od lewej strony powyżej osi OX. A więc $$x \in (-\infty,-20) \cup (-3,9).$$
\rozwStop
\odpStart
$x \in (-\infty,-20) \cup (-3,9)$
\odpStop
\testStart
A.$x \in (-\infty,-20) \cup (-3,9)$\\
B.$x \in (-\infty,-20) \cup (-3,9]$\\
C.$x \in (-\infty,-20) \cup [-3,9)$\\
D.$x \in (-\infty,-20] \cup (-3,9)$\\
E.$x \in (-\infty,-20] \cup (-3,9]$\\
F.$x \in (-\infty,-20] \cup [-3,9)$\\
G.$x \in (-\infty,-20) \cup [-3,9]$\\
H.$x \in (-\infty,-20] \cup [-3,9]$
\testStop
\kluczStart
A
\kluczStop



\zadStart{Zadanie z Wikieł Z 1.62 b) moja wersja nr 1001}

Rozwiązać nierówności $(x+20)(9-x)(x+4)\ge0$.
\zadStop
\rozwStart{Patryk Wirkus}{}
Miejsca zerowe naszego wielomianu to: $-20, 9, -4$.\\
Wielomian jest stopnia nieparzystego, ponadto znak współczynnika przy\linebreak najwyższej potędze x jest ujemny.\\ W związku z tym wykres wielomianu zaczyna się od lewej strony powyżej osi OX. A więc $$x \in (-\infty,-20) \cup (-4,9).$$
\rozwStop
\odpStart
$x \in (-\infty,-20) \cup (-4,9)$
\odpStop
\testStart
A.$x \in (-\infty,-20) \cup (-4,9)$\\
B.$x \in (-\infty,-20) \cup (-4,9]$\\
C.$x \in (-\infty,-20) \cup [-4,9)$\\
D.$x \in (-\infty,-20] \cup (-4,9)$\\
E.$x \in (-\infty,-20] \cup (-4,9]$\\
F.$x \in (-\infty,-20] \cup [-4,9)$\\
G.$x \in (-\infty,-20) \cup [-4,9]$\\
H.$x \in (-\infty,-20] \cup [-4,9]$
\testStop
\kluczStart
A
\kluczStop



\zadStart{Zadanie z Wikieł Z 1.62 b) moja wersja nr 1002}

Rozwiązać nierówności $(x+20)(9-x)(x+5)\ge0$.
\zadStop
\rozwStart{Patryk Wirkus}{}
Miejsca zerowe naszego wielomianu to: $-20, 9, -5$.\\
Wielomian jest stopnia nieparzystego, ponadto znak współczynnika przy\linebreak najwyższej potędze x jest ujemny.\\ W związku z tym wykres wielomianu zaczyna się od lewej strony powyżej osi OX. A więc $$x \in (-\infty,-20) \cup (-5,9).$$
\rozwStop
\odpStart
$x \in (-\infty,-20) \cup (-5,9)$
\odpStop
\testStart
A.$x \in (-\infty,-20) \cup (-5,9)$\\
B.$x \in (-\infty,-20) \cup (-5,9]$\\
C.$x \in (-\infty,-20) \cup [-5,9)$\\
D.$x \in (-\infty,-20] \cup (-5,9)$\\
E.$x \in (-\infty,-20] \cup (-5,9]$\\
F.$x \in (-\infty,-20] \cup [-5,9)$\\
G.$x \in (-\infty,-20) \cup [-5,9]$\\
H.$x \in (-\infty,-20] \cup [-5,9]$
\testStop
\kluczStart
A
\kluczStop



\zadStart{Zadanie z Wikieł Z 1.62 b) moja wersja nr 1003}

Rozwiązać nierówności $(x+20)(9-x)(x+6)\ge0$.
\zadStop
\rozwStart{Patryk Wirkus}{}
Miejsca zerowe naszego wielomianu to: $-20, 9, -6$.\\
Wielomian jest stopnia nieparzystego, ponadto znak współczynnika przy\linebreak najwyższej potędze x jest ujemny.\\ W związku z tym wykres wielomianu zaczyna się od lewej strony powyżej osi OX. A więc $$x \in (-\infty,-20) \cup (-6,9).$$
\rozwStop
\odpStart
$x \in (-\infty,-20) \cup (-6,9)$
\odpStop
\testStart
A.$x \in (-\infty,-20) \cup (-6,9)$\\
B.$x \in (-\infty,-20) \cup (-6,9]$\\
C.$x \in (-\infty,-20) \cup [-6,9)$\\
D.$x \in (-\infty,-20] \cup (-6,9)$\\
E.$x \in (-\infty,-20] \cup (-6,9]$\\
F.$x \in (-\infty,-20] \cup [-6,9)$\\
G.$x \in (-\infty,-20) \cup [-6,9]$\\
H.$x \in (-\infty,-20] \cup [-6,9]$
\testStop
\kluczStart
A
\kluczStop



\zadStart{Zadanie z Wikieł Z 1.62 b) moja wersja nr 1004}

Rozwiązać nierówności $(x+20)(9-x)(x+7)\ge0$.
\zadStop
\rozwStart{Patryk Wirkus}{}
Miejsca zerowe naszego wielomianu to: $-20, 9, -7$.\\
Wielomian jest stopnia nieparzystego, ponadto znak współczynnika przy\linebreak najwyższej potędze x jest ujemny.\\ W związku z tym wykres wielomianu zaczyna się od lewej strony powyżej osi OX. A więc $$x \in (-\infty,-20) \cup (-7,9).$$
\rozwStop
\odpStart
$x \in (-\infty,-20) \cup (-7,9)$
\odpStop
\testStart
A.$x \in (-\infty,-20) \cup (-7,9)$\\
B.$x \in (-\infty,-20) \cup (-7,9]$\\
C.$x \in (-\infty,-20) \cup [-7,9)$\\
D.$x \in (-\infty,-20] \cup (-7,9)$\\
E.$x \in (-\infty,-20] \cup (-7,9]$\\
F.$x \in (-\infty,-20] \cup [-7,9)$\\
G.$x \in (-\infty,-20) \cup [-7,9]$\\
H.$x \in (-\infty,-20] \cup [-7,9]$
\testStop
\kluczStart
A
\kluczStop



\zadStart{Zadanie z Wikieł Z 1.62 b) moja wersja nr 1005}

Rozwiązać nierówności $(x+20)(9-x)(x+8)\ge0$.
\zadStop
\rozwStart{Patryk Wirkus}{}
Miejsca zerowe naszego wielomianu to: $-20, 9, -8$.\\
Wielomian jest stopnia nieparzystego, ponadto znak współczynnika przy\linebreak najwyższej potędze x jest ujemny.\\ W związku z tym wykres wielomianu zaczyna się od lewej strony powyżej osi OX. A więc $$x \in (-\infty,-20) \cup (-8,9).$$
\rozwStop
\odpStart
$x \in (-\infty,-20) \cup (-8,9)$
\odpStop
\testStart
A.$x \in (-\infty,-20) \cup (-8,9)$\\
B.$x \in (-\infty,-20) \cup (-8,9]$\\
C.$x \in (-\infty,-20) \cup [-8,9)$\\
D.$x \in (-\infty,-20] \cup (-8,9)$\\
E.$x \in (-\infty,-20] \cup (-8,9]$\\
F.$x \in (-\infty,-20] \cup [-8,9)$\\
G.$x \in (-\infty,-20) \cup [-8,9]$\\
H.$x \in (-\infty,-20] \cup [-8,9]$
\testStop
\kluczStart
A
\kluczStop



\zadStart{Zadanie z Wikieł Z 1.62 b) moja wersja nr 1006}

Rozwiązać nierówności $(x+20)(10-x)(x+1)\ge0$.
\zadStop
\rozwStart{Patryk Wirkus}{}
Miejsca zerowe naszego wielomianu to: $-20, 10, -1$.\\
Wielomian jest stopnia nieparzystego, ponadto znak współczynnika przy\linebreak najwyższej potędze x jest ujemny.\\ W związku z tym wykres wielomianu zaczyna się od lewej strony powyżej osi OX. A więc $$x \in (-\infty,-20) \cup (-1,10).$$
\rozwStop
\odpStart
$x \in (-\infty,-20) \cup (-1,10)$
\odpStop
\testStart
A.$x \in (-\infty,-20) \cup (-1,10)$\\
B.$x \in (-\infty,-20) \cup (-1,10]$\\
C.$x \in (-\infty,-20) \cup [-1,10)$\\
D.$x \in (-\infty,-20] \cup (-1,10)$\\
E.$x \in (-\infty,-20] \cup (-1,10]$\\
F.$x \in (-\infty,-20] \cup [-1,10)$\\
G.$x \in (-\infty,-20) \cup [-1,10]$\\
H.$x \in (-\infty,-20] \cup [-1,10]$
\testStop
\kluczStart
A
\kluczStop



\zadStart{Zadanie z Wikieł Z 1.62 b) moja wersja nr 1007}

Rozwiązać nierówności $(x+20)(10-x)(x+2)\ge0$.
\zadStop
\rozwStart{Patryk Wirkus}{}
Miejsca zerowe naszego wielomianu to: $-20, 10, -2$.\\
Wielomian jest stopnia nieparzystego, ponadto znak współczynnika przy\linebreak najwyższej potędze x jest ujemny.\\ W związku z tym wykres wielomianu zaczyna się od lewej strony powyżej osi OX. A więc $$x \in (-\infty,-20) \cup (-2,10).$$
\rozwStop
\odpStart
$x \in (-\infty,-20) \cup (-2,10)$
\odpStop
\testStart
A.$x \in (-\infty,-20) \cup (-2,10)$\\
B.$x \in (-\infty,-20) \cup (-2,10]$\\
C.$x \in (-\infty,-20) \cup [-2,10)$\\
D.$x \in (-\infty,-20] \cup (-2,10)$\\
E.$x \in (-\infty,-20] \cup (-2,10]$\\
F.$x \in (-\infty,-20] \cup [-2,10)$\\
G.$x \in (-\infty,-20) \cup [-2,10]$\\
H.$x \in (-\infty,-20] \cup [-2,10]$
\testStop
\kluczStart
A
\kluczStop



\zadStart{Zadanie z Wikieł Z 1.62 b) moja wersja nr 1008}

Rozwiązać nierówności $(x+20)(10-x)(x+3)\ge0$.
\zadStop
\rozwStart{Patryk Wirkus}{}
Miejsca zerowe naszego wielomianu to: $-20, 10, -3$.\\
Wielomian jest stopnia nieparzystego, ponadto znak współczynnika przy\linebreak najwyższej potędze x jest ujemny.\\ W związku z tym wykres wielomianu zaczyna się od lewej strony powyżej osi OX. A więc $$x \in (-\infty,-20) \cup (-3,10).$$
\rozwStop
\odpStart
$x \in (-\infty,-20) \cup (-3,10)$
\odpStop
\testStart
A.$x \in (-\infty,-20) \cup (-3,10)$\\
B.$x \in (-\infty,-20) \cup (-3,10]$\\
C.$x \in (-\infty,-20) \cup [-3,10)$\\
D.$x \in (-\infty,-20] \cup (-3,10)$\\
E.$x \in (-\infty,-20] \cup (-3,10]$\\
F.$x \in (-\infty,-20] \cup [-3,10)$\\
G.$x \in (-\infty,-20) \cup [-3,10]$\\
H.$x \in (-\infty,-20] \cup [-3,10]$
\testStop
\kluczStart
A
\kluczStop



\zadStart{Zadanie z Wikieł Z 1.62 b) moja wersja nr 1009}

Rozwiązać nierówności $(x+20)(10-x)(x+4)\ge0$.
\zadStop
\rozwStart{Patryk Wirkus}{}
Miejsca zerowe naszego wielomianu to: $-20, 10, -4$.\\
Wielomian jest stopnia nieparzystego, ponadto znak współczynnika przy\linebreak najwyższej potędze x jest ujemny.\\ W związku z tym wykres wielomianu zaczyna się od lewej strony powyżej osi OX. A więc $$x \in (-\infty,-20) \cup (-4,10).$$
\rozwStop
\odpStart
$x \in (-\infty,-20) \cup (-4,10)$
\odpStop
\testStart
A.$x \in (-\infty,-20) \cup (-4,10)$\\
B.$x \in (-\infty,-20) \cup (-4,10]$\\
C.$x \in (-\infty,-20) \cup [-4,10)$\\
D.$x \in (-\infty,-20] \cup (-4,10)$\\
E.$x \in (-\infty,-20] \cup (-4,10]$\\
F.$x \in (-\infty,-20] \cup [-4,10)$\\
G.$x \in (-\infty,-20) \cup [-4,10]$\\
H.$x \in (-\infty,-20] \cup [-4,10]$
\testStop
\kluczStart
A
\kluczStop



\zadStart{Zadanie z Wikieł Z 1.62 b) moja wersja nr 1010}

Rozwiązać nierówności $(x+20)(10-x)(x+5)\ge0$.
\zadStop
\rozwStart{Patryk Wirkus}{}
Miejsca zerowe naszego wielomianu to: $-20, 10, -5$.\\
Wielomian jest stopnia nieparzystego, ponadto znak współczynnika przy\linebreak najwyższej potędze x jest ujemny.\\ W związku z tym wykres wielomianu zaczyna się od lewej strony powyżej osi OX. A więc $$x \in (-\infty,-20) \cup (-5,10).$$
\rozwStop
\odpStart
$x \in (-\infty,-20) \cup (-5,10)$
\odpStop
\testStart
A.$x \in (-\infty,-20) \cup (-5,10)$\\
B.$x \in (-\infty,-20) \cup (-5,10]$\\
C.$x \in (-\infty,-20) \cup [-5,10)$\\
D.$x \in (-\infty,-20] \cup (-5,10)$\\
E.$x \in (-\infty,-20] \cup (-5,10]$\\
F.$x \in (-\infty,-20] \cup [-5,10)$\\
G.$x \in (-\infty,-20) \cup [-5,10]$\\
H.$x \in (-\infty,-20] \cup [-5,10]$
\testStop
\kluczStart
A
\kluczStop



\zadStart{Zadanie z Wikieł Z 1.62 b) moja wersja nr 1011}

Rozwiązać nierówności $(x+20)(10-x)(x+6)\ge0$.
\zadStop
\rozwStart{Patryk Wirkus}{}
Miejsca zerowe naszego wielomianu to: $-20, 10, -6$.\\
Wielomian jest stopnia nieparzystego, ponadto znak współczynnika przy\linebreak najwyższej potędze x jest ujemny.\\ W związku z tym wykres wielomianu zaczyna się od lewej strony powyżej osi OX. A więc $$x \in (-\infty,-20) \cup (-6,10).$$
\rozwStop
\odpStart
$x \in (-\infty,-20) \cup (-6,10)$
\odpStop
\testStart
A.$x \in (-\infty,-20) \cup (-6,10)$\\
B.$x \in (-\infty,-20) \cup (-6,10]$\\
C.$x \in (-\infty,-20) \cup [-6,10)$\\
D.$x \in (-\infty,-20] \cup (-6,10)$\\
E.$x \in (-\infty,-20] \cup (-6,10]$\\
F.$x \in (-\infty,-20] \cup [-6,10)$\\
G.$x \in (-\infty,-20) \cup [-6,10]$\\
H.$x \in (-\infty,-20] \cup [-6,10]$
\testStop
\kluczStart
A
\kluczStop



\zadStart{Zadanie z Wikieł Z 1.62 b) moja wersja nr 1012}

Rozwiązać nierówności $(x+20)(10-x)(x+7)\ge0$.
\zadStop
\rozwStart{Patryk Wirkus}{}
Miejsca zerowe naszego wielomianu to: $-20, 10, -7$.\\
Wielomian jest stopnia nieparzystego, ponadto znak współczynnika przy\linebreak najwyższej potędze x jest ujemny.\\ W związku z tym wykres wielomianu zaczyna się od lewej strony powyżej osi OX. A więc $$x \in (-\infty,-20) \cup (-7,10).$$
\rozwStop
\odpStart
$x \in (-\infty,-20) \cup (-7,10)$
\odpStop
\testStart
A.$x \in (-\infty,-20) \cup (-7,10)$\\
B.$x \in (-\infty,-20) \cup (-7,10]$\\
C.$x \in (-\infty,-20) \cup [-7,10)$\\
D.$x \in (-\infty,-20] \cup (-7,10)$\\
E.$x \in (-\infty,-20] \cup (-7,10]$\\
F.$x \in (-\infty,-20] \cup [-7,10)$\\
G.$x \in (-\infty,-20) \cup [-7,10]$\\
H.$x \in (-\infty,-20] \cup [-7,10]$
\testStop
\kluczStart
A
\kluczStop



\zadStart{Zadanie z Wikieł Z 1.62 b) moja wersja nr 1013}

Rozwiązać nierówności $(x+20)(10-x)(x+8)\ge0$.
\zadStop
\rozwStart{Patryk Wirkus}{}
Miejsca zerowe naszego wielomianu to: $-20, 10, -8$.\\
Wielomian jest stopnia nieparzystego, ponadto znak współczynnika przy\linebreak najwyższej potędze x jest ujemny.\\ W związku z tym wykres wielomianu zaczyna się od lewej strony powyżej osi OX. A więc $$x \in (-\infty,-20) \cup (-8,10).$$
\rozwStop
\odpStart
$x \in (-\infty,-20) \cup (-8,10)$
\odpStop
\testStart
A.$x \in (-\infty,-20) \cup (-8,10)$\\
B.$x \in (-\infty,-20) \cup (-8,10]$\\
C.$x \in (-\infty,-20) \cup [-8,10)$\\
D.$x \in (-\infty,-20] \cup (-8,10)$\\
E.$x \in (-\infty,-20] \cup (-8,10]$\\
F.$x \in (-\infty,-20] \cup [-8,10)$\\
G.$x \in (-\infty,-20) \cup [-8,10]$\\
H.$x \in (-\infty,-20] \cup [-8,10]$
\testStop
\kluczStart
A
\kluczStop



\zadStart{Zadanie z Wikieł Z 1.62 b) moja wersja nr 1014}

Rozwiązać nierówności $(x+20)(10-x)(x+9)\ge0$.
\zadStop
\rozwStart{Patryk Wirkus}{}
Miejsca zerowe naszego wielomianu to: $-20, 10, -9$.\\
Wielomian jest stopnia nieparzystego, ponadto znak współczynnika przy\linebreak najwyższej potędze x jest ujemny.\\ W związku z tym wykres wielomianu zaczyna się od lewej strony powyżej osi OX. A więc $$x \in (-\infty,-20) \cup (-9,10).$$
\rozwStop
\odpStart
$x \in (-\infty,-20) \cup (-9,10)$
\odpStop
\testStart
A.$x \in (-\infty,-20) \cup (-9,10)$\\
B.$x \in (-\infty,-20) \cup (-9,10]$\\
C.$x \in (-\infty,-20) \cup [-9,10)$\\
D.$x \in (-\infty,-20] \cup (-9,10)$\\
E.$x \in (-\infty,-20] \cup (-9,10]$\\
F.$x \in (-\infty,-20] \cup [-9,10)$\\
G.$x \in (-\infty,-20) \cup [-9,10]$\\
H.$x \in (-\infty,-20] \cup [-9,10]$
\testStop
\kluczStart
A
\kluczStop



\zadStart{Zadanie z Wikieł Z 1.62 b) moja wersja nr 1015}

Rozwiązać nierówności $(x+20)(11-x)(x+1)\ge0$.
\zadStop
\rozwStart{Patryk Wirkus}{}
Miejsca zerowe naszego wielomianu to: $-20, 11, -1$.\\
Wielomian jest stopnia nieparzystego, ponadto znak współczynnika przy\linebreak najwyższej potędze x jest ujemny.\\ W związku z tym wykres wielomianu zaczyna się od lewej strony powyżej osi OX. A więc $$x \in (-\infty,-20) \cup (-1,11).$$
\rozwStop
\odpStart
$x \in (-\infty,-20) \cup (-1,11)$
\odpStop
\testStart
A.$x \in (-\infty,-20) \cup (-1,11)$\\
B.$x \in (-\infty,-20) \cup (-1,11]$\\
C.$x \in (-\infty,-20) \cup [-1,11)$\\
D.$x \in (-\infty,-20] \cup (-1,11)$\\
E.$x \in (-\infty,-20] \cup (-1,11]$\\
F.$x \in (-\infty,-20] \cup [-1,11)$\\
G.$x \in (-\infty,-20) \cup [-1,11]$\\
H.$x \in (-\infty,-20] \cup [-1,11]$
\testStop
\kluczStart
A
\kluczStop



\zadStart{Zadanie z Wikieł Z 1.62 b) moja wersja nr 1016}

Rozwiązać nierówności $(x+20)(11-x)(x+2)\ge0$.
\zadStop
\rozwStart{Patryk Wirkus}{}
Miejsca zerowe naszego wielomianu to: $-20, 11, -2$.\\
Wielomian jest stopnia nieparzystego, ponadto znak współczynnika przy\linebreak najwyższej potędze x jest ujemny.\\ W związku z tym wykres wielomianu zaczyna się od lewej strony powyżej osi OX. A więc $$x \in (-\infty,-20) \cup (-2,11).$$
\rozwStop
\odpStart
$x \in (-\infty,-20) \cup (-2,11)$
\odpStop
\testStart
A.$x \in (-\infty,-20) \cup (-2,11)$\\
B.$x \in (-\infty,-20) \cup (-2,11]$\\
C.$x \in (-\infty,-20) \cup [-2,11)$\\
D.$x \in (-\infty,-20] \cup (-2,11)$\\
E.$x \in (-\infty,-20] \cup (-2,11]$\\
F.$x \in (-\infty,-20] \cup [-2,11)$\\
G.$x \in (-\infty,-20) \cup [-2,11]$\\
H.$x \in (-\infty,-20] \cup [-2,11]$
\testStop
\kluczStart
A
\kluczStop



\zadStart{Zadanie z Wikieł Z 1.62 b) moja wersja nr 1017}

Rozwiązać nierówności $(x+20)(11-x)(x+3)\ge0$.
\zadStop
\rozwStart{Patryk Wirkus}{}
Miejsca zerowe naszego wielomianu to: $-20, 11, -3$.\\
Wielomian jest stopnia nieparzystego, ponadto znak współczynnika przy\linebreak najwyższej potędze x jest ujemny.\\ W związku z tym wykres wielomianu zaczyna się od lewej strony powyżej osi OX. A więc $$x \in (-\infty,-20) \cup (-3,11).$$
\rozwStop
\odpStart
$x \in (-\infty,-20) \cup (-3,11)$
\odpStop
\testStart
A.$x \in (-\infty,-20) \cup (-3,11)$\\
B.$x \in (-\infty,-20) \cup (-3,11]$\\
C.$x \in (-\infty,-20) \cup [-3,11)$\\
D.$x \in (-\infty,-20] \cup (-3,11)$\\
E.$x \in (-\infty,-20] \cup (-3,11]$\\
F.$x \in (-\infty,-20] \cup [-3,11)$\\
G.$x \in (-\infty,-20) \cup [-3,11]$\\
H.$x \in (-\infty,-20] \cup [-3,11]$
\testStop
\kluczStart
A
\kluczStop



\zadStart{Zadanie z Wikieł Z 1.62 b) moja wersja nr 1018}

Rozwiązać nierówności $(x+20)(11-x)(x+4)\ge0$.
\zadStop
\rozwStart{Patryk Wirkus}{}
Miejsca zerowe naszego wielomianu to: $-20, 11, -4$.\\
Wielomian jest stopnia nieparzystego, ponadto znak współczynnika przy\linebreak najwyższej potędze x jest ujemny.\\ W związku z tym wykres wielomianu zaczyna się od lewej strony powyżej osi OX. A więc $$x \in (-\infty,-20) \cup (-4,11).$$
\rozwStop
\odpStart
$x \in (-\infty,-20) \cup (-4,11)$
\odpStop
\testStart
A.$x \in (-\infty,-20) \cup (-4,11)$\\
B.$x \in (-\infty,-20) \cup (-4,11]$\\
C.$x \in (-\infty,-20) \cup [-4,11)$\\
D.$x \in (-\infty,-20] \cup (-4,11)$\\
E.$x \in (-\infty,-20] \cup (-4,11]$\\
F.$x \in (-\infty,-20] \cup [-4,11)$\\
G.$x \in (-\infty,-20) \cup [-4,11]$\\
H.$x \in (-\infty,-20] \cup [-4,11]$
\testStop
\kluczStart
A
\kluczStop



\zadStart{Zadanie z Wikieł Z 1.62 b) moja wersja nr 1019}

Rozwiązać nierówności $(x+20)(11-x)(x+5)\ge0$.
\zadStop
\rozwStart{Patryk Wirkus}{}
Miejsca zerowe naszego wielomianu to: $-20, 11, -5$.\\
Wielomian jest stopnia nieparzystego, ponadto znak współczynnika przy\linebreak najwyższej potędze x jest ujemny.\\ W związku z tym wykres wielomianu zaczyna się od lewej strony powyżej osi OX. A więc $$x \in (-\infty,-20) \cup (-5,11).$$
\rozwStop
\odpStart
$x \in (-\infty,-20) \cup (-5,11)$
\odpStop
\testStart
A.$x \in (-\infty,-20) \cup (-5,11)$\\
B.$x \in (-\infty,-20) \cup (-5,11]$\\
C.$x \in (-\infty,-20) \cup [-5,11)$\\
D.$x \in (-\infty,-20] \cup (-5,11)$\\
E.$x \in (-\infty,-20] \cup (-5,11]$\\
F.$x \in (-\infty,-20] \cup [-5,11)$\\
G.$x \in (-\infty,-20) \cup [-5,11]$\\
H.$x \in (-\infty,-20] \cup [-5,11]$
\testStop
\kluczStart
A
\kluczStop



\zadStart{Zadanie z Wikieł Z 1.62 b) moja wersja nr 1020}

Rozwiązać nierówności $(x+20)(11-x)(x+6)\ge0$.
\zadStop
\rozwStart{Patryk Wirkus}{}
Miejsca zerowe naszego wielomianu to: $-20, 11, -6$.\\
Wielomian jest stopnia nieparzystego, ponadto znak współczynnika przy\linebreak najwyższej potędze x jest ujemny.\\ W związku z tym wykres wielomianu zaczyna się od lewej strony powyżej osi OX. A więc $$x \in (-\infty,-20) \cup (-6,11).$$
\rozwStop
\odpStart
$x \in (-\infty,-20) \cup (-6,11)$
\odpStop
\testStart
A.$x \in (-\infty,-20) \cup (-6,11)$\\
B.$x \in (-\infty,-20) \cup (-6,11]$\\
C.$x \in (-\infty,-20) \cup [-6,11)$\\
D.$x \in (-\infty,-20] \cup (-6,11)$\\
E.$x \in (-\infty,-20] \cup (-6,11]$\\
F.$x \in (-\infty,-20] \cup [-6,11)$\\
G.$x \in (-\infty,-20) \cup [-6,11]$\\
H.$x \in (-\infty,-20] \cup [-6,11]$
\testStop
\kluczStart
A
\kluczStop



\zadStart{Zadanie z Wikieł Z 1.62 b) moja wersja nr 1021}

Rozwiązać nierówności $(x+20)(11-x)(x+7)\ge0$.
\zadStop
\rozwStart{Patryk Wirkus}{}
Miejsca zerowe naszego wielomianu to: $-20, 11, -7$.\\
Wielomian jest stopnia nieparzystego, ponadto znak współczynnika przy\linebreak najwyższej potędze x jest ujemny.\\ W związku z tym wykres wielomianu zaczyna się od lewej strony powyżej osi OX. A więc $$x \in (-\infty,-20) \cup (-7,11).$$
\rozwStop
\odpStart
$x \in (-\infty,-20) \cup (-7,11)$
\odpStop
\testStart
A.$x \in (-\infty,-20) \cup (-7,11)$\\
B.$x \in (-\infty,-20) \cup (-7,11]$\\
C.$x \in (-\infty,-20) \cup [-7,11)$\\
D.$x \in (-\infty,-20] \cup (-7,11)$\\
E.$x \in (-\infty,-20] \cup (-7,11]$\\
F.$x \in (-\infty,-20] \cup [-7,11)$\\
G.$x \in (-\infty,-20) \cup [-7,11]$\\
H.$x \in (-\infty,-20] \cup [-7,11]$
\testStop
\kluczStart
A
\kluczStop



\zadStart{Zadanie z Wikieł Z 1.62 b) moja wersja nr 1022}

Rozwiązać nierówności $(x+20)(11-x)(x+8)\ge0$.
\zadStop
\rozwStart{Patryk Wirkus}{}
Miejsca zerowe naszego wielomianu to: $-20, 11, -8$.\\
Wielomian jest stopnia nieparzystego, ponadto znak współczynnika przy\linebreak najwyższej potędze x jest ujemny.\\ W związku z tym wykres wielomianu zaczyna się od lewej strony powyżej osi OX. A więc $$x \in (-\infty,-20) \cup (-8,11).$$
\rozwStop
\odpStart
$x \in (-\infty,-20) \cup (-8,11)$
\odpStop
\testStart
A.$x \in (-\infty,-20) \cup (-8,11)$\\
B.$x \in (-\infty,-20) \cup (-8,11]$\\
C.$x \in (-\infty,-20) \cup [-8,11)$\\
D.$x \in (-\infty,-20] \cup (-8,11)$\\
E.$x \in (-\infty,-20] \cup (-8,11]$\\
F.$x \in (-\infty,-20] \cup [-8,11)$\\
G.$x \in (-\infty,-20) \cup [-8,11]$\\
H.$x \in (-\infty,-20] \cup [-8,11]$
\testStop
\kluczStart
A
\kluczStop



\zadStart{Zadanie z Wikieł Z 1.62 b) moja wersja nr 1023}

Rozwiązać nierówności $(x+20)(11-x)(x+9)\ge0$.
\zadStop
\rozwStart{Patryk Wirkus}{}
Miejsca zerowe naszego wielomianu to: $-20, 11, -9$.\\
Wielomian jest stopnia nieparzystego, ponadto znak współczynnika przy\linebreak najwyższej potędze x jest ujemny.\\ W związku z tym wykres wielomianu zaczyna się od lewej strony powyżej osi OX. A więc $$x \in (-\infty,-20) \cup (-9,11).$$
\rozwStop
\odpStart
$x \in (-\infty,-20) \cup (-9,11)$
\odpStop
\testStart
A.$x \in (-\infty,-20) \cup (-9,11)$\\
B.$x \in (-\infty,-20) \cup (-9,11]$\\
C.$x \in (-\infty,-20) \cup [-9,11)$\\
D.$x \in (-\infty,-20] \cup (-9,11)$\\
E.$x \in (-\infty,-20] \cup (-9,11]$\\
F.$x \in (-\infty,-20] \cup [-9,11)$\\
G.$x \in (-\infty,-20) \cup [-9,11]$\\
H.$x \in (-\infty,-20] \cup [-9,11]$
\testStop
\kluczStart
A
\kluczStop



\zadStart{Zadanie z Wikieł Z 1.62 b) moja wersja nr 1024}

Rozwiązać nierówności $(x+20)(11-x)(x+10)\ge0$.
\zadStop
\rozwStart{Patryk Wirkus}{}
Miejsca zerowe naszego wielomianu to: $-20, 11, -10$.\\
Wielomian jest stopnia nieparzystego, ponadto znak współczynnika przy\linebreak najwyższej potędze x jest ujemny.\\ W związku z tym wykres wielomianu zaczyna się od lewej strony powyżej osi OX. A więc $$x \in (-\infty,-20) \cup (-10,11).$$
\rozwStop
\odpStart
$x \in (-\infty,-20) \cup (-10,11)$
\odpStop
\testStart
A.$x \in (-\infty,-20) \cup (-10,11)$\\
B.$x \in (-\infty,-20) \cup (-10,11]$\\
C.$x \in (-\infty,-20) \cup [-10,11)$\\
D.$x \in (-\infty,-20] \cup (-10,11)$\\
E.$x \in (-\infty,-20] \cup (-10,11]$\\
F.$x \in (-\infty,-20] \cup [-10,11)$\\
G.$x \in (-\infty,-20) \cup [-10,11]$\\
H.$x \in (-\infty,-20] \cup [-10,11]$
\testStop
\kluczStart
A
\kluczStop



\zadStart{Zadanie z Wikieł Z 1.62 b) moja wersja nr 1025}

Rozwiązać nierówności $(x+20)(12-x)(x+1)\ge0$.
\zadStop
\rozwStart{Patryk Wirkus}{}
Miejsca zerowe naszego wielomianu to: $-20, 12, -1$.\\
Wielomian jest stopnia nieparzystego, ponadto znak współczynnika przy\linebreak najwyższej potędze x jest ujemny.\\ W związku z tym wykres wielomianu zaczyna się od lewej strony powyżej osi OX. A więc $$x \in (-\infty,-20) \cup (-1,12).$$
\rozwStop
\odpStart
$x \in (-\infty,-20) \cup (-1,12)$
\odpStop
\testStart
A.$x \in (-\infty,-20) \cup (-1,12)$\\
B.$x \in (-\infty,-20) \cup (-1,12]$\\
C.$x \in (-\infty,-20) \cup [-1,12)$\\
D.$x \in (-\infty,-20] \cup (-1,12)$\\
E.$x \in (-\infty,-20] \cup (-1,12]$\\
F.$x \in (-\infty,-20] \cup [-1,12)$\\
G.$x \in (-\infty,-20) \cup [-1,12]$\\
H.$x \in (-\infty,-20] \cup [-1,12]$
\testStop
\kluczStart
A
\kluczStop



\zadStart{Zadanie z Wikieł Z 1.62 b) moja wersja nr 1026}

Rozwiązać nierówności $(x+20)(12-x)(x+2)\ge0$.
\zadStop
\rozwStart{Patryk Wirkus}{}
Miejsca zerowe naszego wielomianu to: $-20, 12, -2$.\\
Wielomian jest stopnia nieparzystego, ponadto znak współczynnika przy\linebreak najwyższej potędze x jest ujemny.\\ W związku z tym wykres wielomianu zaczyna się od lewej strony powyżej osi OX. A więc $$x \in (-\infty,-20) \cup (-2,12).$$
\rozwStop
\odpStart
$x \in (-\infty,-20) \cup (-2,12)$
\odpStop
\testStart
A.$x \in (-\infty,-20) \cup (-2,12)$\\
B.$x \in (-\infty,-20) \cup (-2,12]$\\
C.$x \in (-\infty,-20) \cup [-2,12)$\\
D.$x \in (-\infty,-20] \cup (-2,12)$\\
E.$x \in (-\infty,-20] \cup (-2,12]$\\
F.$x \in (-\infty,-20] \cup [-2,12)$\\
G.$x \in (-\infty,-20) \cup [-2,12]$\\
H.$x \in (-\infty,-20] \cup [-2,12]$
\testStop
\kluczStart
A
\kluczStop



\zadStart{Zadanie z Wikieł Z 1.62 b) moja wersja nr 1027}

Rozwiązać nierówności $(x+20)(12-x)(x+3)\ge0$.
\zadStop
\rozwStart{Patryk Wirkus}{}
Miejsca zerowe naszego wielomianu to: $-20, 12, -3$.\\
Wielomian jest stopnia nieparzystego, ponadto znak współczynnika przy\linebreak najwyższej potędze x jest ujemny.\\ W związku z tym wykres wielomianu zaczyna się od lewej strony powyżej osi OX. A więc $$x \in (-\infty,-20) \cup (-3,12).$$
\rozwStop
\odpStart
$x \in (-\infty,-20) \cup (-3,12)$
\odpStop
\testStart
A.$x \in (-\infty,-20) \cup (-3,12)$\\
B.$x \in (-\infty,-20) \cup (-3,12]$\\
C.$x \in (-\infty,-20) \cup [-3,12)$\\
D.$x \in (-\infty,-20] \cup (-3,12)$\\
E.$x \in (-\infty,-20] \cup (-3,12]$\\
F.$x \in (-\infty,-20] \cup [-3,12)$\\
G.$x \in (-\infty,-20) \cup [-3,12]$\\
H.$x \in (-\infty,-20] \cup [-3,12]$
\testStop
\kluczStart
A
\kluczStop



\zadStart{Zadanie z Wikieł Z 1.62 b) moja wersja nr 1028}

Rozwiązać nierówności $(x+20)(12-x)(x+4)\ge0$.
\zadStop
\rozwStart{Patryk Wirkus}{}
Miejsca zerowe naszego wielomianu to: $-20, 12, -4$.\\
Wielomian jest stopnia nieparzystego, ponadto znak współczynnika przy\linebreak najwyższej potędze x jest ujemny.\\ W związku z tym wykres wielomianu zaczyna się od lewej strony powyżej osi OX. A więc $$x \in (-\infty,-20) \cup (-4,12).$$
\rozwStop
\odpStart
$x \in (-\infty,-20) \cup (-4,12)$
\odpStop
\testStart
A.$x \in (-\infty,-20) \cup (-4,12)$\\
B.$x \in (-\infty,-20) \cup (-4,12]$\\
C.$x \in (-\infty,-20) \cup [-4,12)$\\
D.$x \in (-\infty,-20] \cup (-4,12)$\\
E.$x \in (-\infty,-20] \cup (-4,12]$\\
F.$x \in (-\infty,-20] \cup [-4,12)$\\
G.$x \in (-\infty,-20) \cup [-4,12]$\\
H.$x \in (-\infty,-20] \cup [-4,12]$
\testStop
\kluczStart
A
\kluczStop



\zadStart{Zadanie z Wikieł Z 1.62 b) moja wersja nr 1029}

Rozwiązać nierówności $(x+20)(12-x)(x+5)\ge0$.
\zadStop
\rozwStart{Patryk Wirkus}{}
Miejsca zerowe naszego wielomianu to: $-20, 12, -5$.\\
Wielomian jest stopnia nieparzystego, ponadto znak współczynnika przy\linebreak najwyższej potędze x jest ujemny.\\ W związku z tym wykres wielomianu zaczyna się od lewej strony powyżej osi OX. A więc $$x \in (-\infty,-20) \cup (-5,12).$$
\rozwStop
\odpStart
$x \in (-\infty,-20) \cup (-5,12)$
\odpStop
\testStart
A.$x \in (-\infty,-20) \cup (-5,12)$\\
B.$x \in (-\infty,-20) \cup (-5,12]$\\
C.$x \in (-\infty,-20) \cup [-5,12)$\\
D.$x \in (-\infty,-20] \cup (-5,12)$\\
E.$x \in (-\infty,-20] \cup (-5,12]$\\
F.$x \in (-\infty,-20] \cup [-5,12)$\\
G.$x \in (-\infty,-20) \cup [-5,12]$\\
H.$x \in (-\infty,-20] \cup [-5,12]$
\testStop
\kluczStart
A
\kluczStop



\zadStart{Zadanie z Wikieł Z 1.62 b) moja wersja nr 1030}

Rozwiązać nierówności $(x+20)(12-x)(x+6)\ge0$.
\zadStop
\rozwStart{Patryk Wirkus}{}
Miejsca zerowe naszego wielomianu to: $-20, 12, -6$.\\
Wielomian jest stopnia nieparzystego, ponadto znak współczynnika przy\linebreak najwyższej potędze x jest ujemny.\\ W związku z tym wykres wielomianu zaczyna się od lewej strony powyżej osi OX. A więc $$x \in (-\infty,-20) \cup (-6,12).$$
\rozwStop
\odpStart
$x \in (-\infty,-20) \cup (-6,12)$
\odpStop
\testStart
A.$x \in (-\infty,-20) \cup (-6,12)$\\
B.$x \in (-\infty,-20) \cup (-6,12]$\\
C.$x \in (-\infty,-20) \cup [-6,12)$\\
D.$x \in (-\infty,-20] \cup (-6,12)$\\
E.$x \in (-\infty,-20] \cup (-6,12]$\\
F.$x \in (-\infty,-20] \cup [-6,12)$\\
G.$x \in (-\infty,-20) \cup [-6,12]$\\
H.$x \in (-\infty,-20] \cup [-6,12]$
\testStop
\kluczStart
A
\kluczStop



\zadStart{Zadanie z Wikieł Z 1.62 b) moja wersja nr 1031}

Rozwiązać nierówności $(x+20)(12-x)(x+7)\ge0$.
\zadStop
\rozwStart{Patryk Wirkus}{}
Miejsca zerowe naszego wielomianu to: $-20, 12, -7$.\\
Wielomian jest stopnia nieparzystego, ponadto znak współczynnika przy\linebreak najwyższej potędze x jest ujemny.\\ W związku z tym wykres wielomianu zaczyna się od lewej strony powyżej osi OX. A więc $$x \in (-\infty,-20) \cup (-7,12).$$
\rozwStop
\odpStart
$x \in (-\infty,-20) \cup (-7,12)$
\odpStop
\testStart
A.$x \in (-\infty,-20) \cup (-7,12)$\\
B.$x \in (-\infty,-20) \cup (-7,12]$\\
C.$x \in (-\infty,-20) \cup [-7,12)$\\
D.$x \in (-\infty,-20] \cup (-7,12)$\\
E.$x \in (-\infty,-20] \cup (-7,12]$\\
F.$x \in (-\infty,-20] \cup [-7,12)$\\
G.$x \in (-\infty,-20) \cup [-7,12]$\\
H.$x \in (-\infty,-20] \cup [-7,12]$
\testStop
\kluczStart
A
\kluczStop



\zadStart{Zadanie z Wikieł Z 1.62 b) moja wersja nr 1032}

Rozwiązać nierówności $(x+20)(12-x)(x+8)\ge0$.
\zadStop
\rozwStart{Patryk Wirkus}{}
Miejsca zerowe naszego wielomianu to: $-20, 12, -8$.\\
Wielomian jest stopnia nieparzystego, ponadto znak współczynnika przy\linebreak najwyższej potędze x jest ujemny.\\ W związku z tym wykres wielomianu zaczyna się od lewej strony powyżej osi OX. A więc $$x \in (-\infty,-20) \cup (-8,12).$$
\rozwStop
\odpStart
$x \in (-\infty,-20) \cup (-8,12)$
\odpStop
\testStart
A.$x \in (-\infty,-20) \cup (-8,12)$\\
B.$x \in (-\infty,-20) \cup (-8,12]$\\
C.$x \in (-\infty,-20) \cup [-8,12)$\\
D.$x \in (-\infty,-20] \cup (-8,12)$\\
E.$x \in (-\infty,-20] \cup (-8,12]$\\
F.$x \in (-\infty,-20] \cup [-8,12)$\\
G.$x \in (-\infty,-20) \cup [-8,12]$\\
H.$x \in (-\infty,-20] \cup [-8,12]$
\testStop
\kluczStart
A
\kluczStop



\zadStart{Zadanie z Wikieł Z 1.62 b) moja wersja nr 1033}

Rozwiązać nierówności $(x+20)(12-x)(x+9)\ge0$.
\zadStop
\rozwStart{Patryk Wirkus}{}
Miejsca zerowe naszego wielomianu to: $-20, 12, -9$.\\
Wielomian jest stopnia nieparzystego, ponadto znak współczynnika przy\linebreak najwyższej potędze x jest ujemny.\\ W związku z tym wykres wielomianu zaczyna się od lewej strony powyżej osi OX. A więc $$x \in (-\infty,-20) \cup (-9,12).$$
\rozwStop
\odpStart
$x \in (-\infty,-20) \cup (-9,12)$
\odpStop
\testStart
A.$x \in (-\infty,-20) \cup (-9,12)$\\
B.$x \in (-\infty,-20) \cup (-9,12]$\\
C.$x \in (-\infty,-20) \cup [-9,12)$\\
D.$x \in (-\infty,-20] \cup (-9,12)$\\
E.$x \in (-\infty,-20] \cup (-9,12]$\\
F.$x \in (-\infty,-20] \cup [-9,12)$\\
G.$x \in (-\infty,-20) \cup [-9,12]$\\
H.$x \in (-\infty,-20] \cup [-9,12]$
\testStop
\kluczStart
A
\kluczStop



\zadStart{Zadanie z Wikieł Z 1.62 b) moja wersja nr 1034}

Rozwiązać nierówności $(x+20)(12-x)(x+10)\ge0$.
\zadStop
\rozwStart{Patryk Wirkus}{}
Miejsca zerowe naszego wielomianu to: $-20, 12, -10$.\\
Wielomian jest stopnia nieparzystego, ponadto znak współczynnika przy\linebreak najwyższej potędze x jest ujemny.\\ W związku z tym wykres wielomianu zaczyna się od lewej strony powyżej osi OX. A więc $$x \in (-\infty,-20) \cup (-10,12).$$
\rozwStop
\odpStart
$x \in (-\infty,-20) \cup (-10,12)$
\odpStop
\testStart
A.$x \in (-\infty,-20) \cup (-10,12)$\\
B.$x \in (-\infty,-20) \cup (-10,12]$\\
C.$x \in (-\infty,-20) \cup [-10,12)$\\
D.$x \in (-\infty,-20] \cup (-10,12)$\\
E.$x \in (-\infty,-20] \cup (-10,12]$\\
F.$x \in (-\infty,-20] \cup [-10,12)$\\
G.$x \in (-\infty,-20) \cup [-10,12]$\\
H.$x \in (-\infty,-20] \cup [-10,12]$
\testStop
\kluczStart
A
\kluczStop



\zadStart{Zadanie z Wikieł Z 1.62 b) moja wersja nr 1035}

Rozwiązać nierówności $(x+20)(12-x)(x+11)\ge0$.
\zadStop
\rozwStart{Patryk Wirkus}{}
Miejsca zerowe naszego wielomianu to: $-20, 12, -11$.\\
Wielomian jest stopnia nieparzystego, ponadto znak współczynnika przy\linebreak najwyższej potędze x jest ujemny.\\ W związku z tym wykres wielomianu zaczyna się od lewej strony powyżej osi OX. A więc $$x \in (-\infty,-20) \cup (-11,12).$$
\rozwStop
\odpStart
$x \in (-\infty,-20) \cup (-11,12)$
\odpStop
\testStart
A.$x \in (-\infty,-20) \cup (-11,12)$\\
B.$x \in (-\infty,-20) \cup (-11,12]$\\
C.$x \in (-\infty,-20) \cup [-11,12)$\\
D.$x \in (-\infty,-20] \cup (-11,12)$\\
E.$x \in (-\infty,-20] \cup (-11,12]$\\
F.$x \in (-\infty,-20] \cup [-11,12)$\\
G.$x \in (-\infty,-20) \cup [-11,12]$\\
H.$x \in (-\infty,-20] \cup [-11,12]$
\testStop
\kluczStart
A
\kluczStop



\zadStart{Zadanie z Wikieł Z 1.62 b) moja wersja nr 1036}

Rozwiązać nierówności $(x+20)(13-x)(x+1)\ge0$.
\zadStop
\rozwStart{Patryk Wirkus}{}
Miejsca zerowe naszego wielomianu to: $-20, 13, -1$.\\
Wielomian jest stopnia nieparzystego, ponadto znak współczynnika przy\linebreak najwyższej potędze x jest ujemny.\\ W związku z tym wykres wielomianu zaczyna się od lewej strony powyżej osi OX. A więc $$x \in (-\infty,-20) \cup (-1,13).$$
\rozwStop
\odpStart
$x \in (-\infty,-20) \cup (-1,13)$
\odpStop
\testStart
A.$x \in (-\infty,-20) \cup (-1,13)$\\
B.$x \in (-\infty,-20) \cup (-1,13]$\\
C.$x \in (-\infty,-20) \cup [-1,13)$\\
D.$x \in (-\infty,-20] \cup (-1,13)$\\
E.$x \in (-\infty,-20] \cup (-1,13]$\\
F.$x \in (-\infty,-20] \cup [-1,13)$\\
G.$x \in (-\infty,-20) \cup [-1,13]$\\
H.$x \in (-\infty,-20] \cup [-1,13]$
\testStop
\kluczStart
A
\kluczStop



\zadStart{Zadanie z Wikieł Z 1.62 b) moja wersja nr 1037}

Rozwiązać nierówności $(x+20)(13-x)(x+2)\ge0$.
\zadStop
\rozwStart{Patryk Wirkus}{}
Miejsca zerowe naszego wielomianu to: $-20, 13, -2$.\\
Wielomian jest stopnia nieparzystego, ponadto znak współczynnika przy\linebreak najwyższej potędze x jest ujemny.\\ W związku z tym wykres wielomianu zaczyna się od lewej strony powyżej osi OX. A więc $$x \in (-\infty,-20) \cup (-2,13).$$
\rozwStop
\odpStart
$x \in (-\infty,-20) \cup (-2,13)$
\odpStop
\testStart
A.$x \in (-\infty,-20) \cup (-2,13)$\\
B.$x \in (-\infty,-20) \cup (-2,13]$\\
C.$x \in (-\infty,-20) \cup [-2,13)$\\
D.$x \in (-\infty,-20] \cup (-2,13)$\\
E.$x \in (-\infty,-20] \cup (-2,13]$\\
F.$x \in (-\infty,-20] \cup [-2,13)$\\
G.$x \in (-\infty,-20) \cup [-2,13]$\\
H.$x \in (-\infty,-20] \cup [-2,13]$
\testStop
\kluczStart
A
\kluczStop



\zadStart{Zadanie z Wikieł Z 1.62 b) moja wersja nr 1038}

Rozwiązać nierówności $(x+20)(13-x)(x+3)\ge0$.
\zadStop
\rozwStart{Patryk Wirkus}{}
Miejsca zerowe naszego wielomianu to: $-20, 13, -3$.\\
Wielomian jest stopnia nieparzystego, ponadto znak współczynnika przy\linebreak najwyższej potędze x jest ujemny.\\ W związku z tym wykres wielomianu zaczyna się od lewej strony powyżej osi OX. A więc $$x \in (-\infty,-20) \cup (-3,13).$$
\rozwStop
\odpStart
$x \in (-\infty,-20) \cup (-3,13)$
\odpStop
\testStart
A.$x \in (-\infty,-20) \cup (-3,13)$\\
B.$x \in (-\infty,-20) \cup (-3,13]$\\
C.$x \in (-\infty,-20) \cup [-3,13)$\\
D.$x \in (-\infty,-20] \cup (-3,13)$\\
E.$x \in (-\infty,-20] \cup (-3,13]$\\
F.$x \in (-\infty,-20] \cup [-3,13)$\\
G.$x \in (-\infty,-20) \cup [-3,13]$\\
H.$x \in (-\infty,-20] \cup [-3,13]$
\testStop
\kluczStart
A
\kluczStop



\zadStart{Zadanie z Wikieł Z 1.62 b) moja wersja nr 1039}

Rozwiązać nierówności $(x+20)(13-x)(x+4)\ge0$.
\zadStop
\rozwStart{Patryk Wirkus}{}
Miejsca zerowe naszego wielomianu to: $-20, 13, -4$.\\
Wielomian jest stopnia nieparzystego, ponadto znak współczynnika przy\linebreak najwyższej potędze x jest ujemny.\\ W związku z tym wykres wielomianu zaczyna się od lewej strony powyżej osi OX. A więc $$x \in (-\infty,-20) \cup (-4,13).$$
\rozwStop
\odpStart
$x \in (-\infty,-20) \cup (-4,13)$
\odpStop
\testStart
A.$x \in (-\infty,-20) \cup (-4,13)$\\
B.$x \in (-\infty,-20) \cup (-4,13]$\\
C.$x \in (-\infty,-20) \cup [-4,13)$\\
D.$x \in (-\infty,-20] \cup (-4,13)$\\
E.$x \in (-\infty,-20] \cup (-4,13]$\\
F.$x \in (-\infty,-20] \cup [-4,13)$\\
G.$x \in (-\infty,-20) \cup [-4,13]$\\
H.$x \in (-\infty,-20] \cup [-4,13]$
\testStop
\kluczStart
A
\kluczStop



\zadStart{Zadanie z Wikieł Z 1.62 b) moja wersja nr 1040}

Rozwiązać nierówności $(x+20)(13-x)(x+5)\ge0$.
\zadStop
\rozwStart{Patryk Wirkus}{}
Miejsca zerowe naszego wielomianu to: $-20, 13, -5$.\\
Wielomian jest stopnia nieparzystego, ponadto znak współczynnika przy\linebreak najwyższej potędze x jest ujemny.\\ W związku z tym wykres wielomianu zaczyna się od lewej strony powyżej osi OX. A więc $$x \in (-\infty,-20) \cup (-5,13).$$
\rozwStop
\odpStart
$x \in (-\infty,-20) \cup (-5,13)$
\odpStop
\testStart
A.$x \in (-\infty,-20) \cup (-5,13)$\\
B.$x \in (-\infty,-20) \cup (-5,13]$\\
C.$x \in (-\infty,-20) \cup [-5,13)$\\
D.$x \in (-\infty,-20] \cup (-5,13)$\\
E.$x \in (-\infty,-20] \cup (-5,13]$\\
F.$x \in (-\infty,-20] \cup [-5,13)$\\
G.$x \in (-\infty,-20) \cup [-5,13]$\\
H.$x \in (-\infty,-20] \cup [-5,13]$
\testStop
\kluczStart
A
\kluczStop



\zadStart{Zadanie z Wikieł Z 1.62 b) moja wersja nr 1041}

Rozwiązać nierówności $(x+20)(13-x)(x+6)\ge0$.
\zadStop
\rozwStart{Patryk Wirkus}{}
Miejsca zerowe naszego wielomianu to: $-20, 13, -6$.\\
Wielomian jest stopnia nieparzystego, ponadto znak współczynnika przy\linebreak najwyższej potędze x jest ujemny.\\ W związku z tym wykres wielomianu zaczyna się od lewej strony powyżej osi OX. A więc $$x \in (-\infty,-20) \cup (-6,13).$$
\rozwStop
\odpStart
$x \in (-\infty,-20) \cup (-6,13)$
\odpStop
\testStart
A.$x \in (-\infty,-20) \cup (-6,13)$\\
B.$x \in (-\infty,-20) \cup (-6,13]$\\
C.$x \in (-\infty,-20) \cup [-6,13)$\\
D.$x \in (-\infty,-20] \cup (-6,13)$\\
E.$x \in (-\infty,-20] \cup (-6,13]$\\
F.$x \in (-\infty,-20] \cup [-6,13)$\\
G.$x \in (-\infty,-20) \cup [-6,13]$\\
H.$x \in (-\infty,-20] \cup [-6,13]$
\testStop
\kluczStart
A
\kluczStop



\zadStart{Zadanie z Wikieł Z 1.62 b) moja wersja nr 1042}

Rozwiązać nierówności $(x+20)(13-x)(x+7)\ge0$.
\zadStop
\rozwStart{Patryk Wirkus}{}
Miejsca zerowe naszego wielomianu to: $-20, 13, -7$.\\
Wielomian jest stopnia nieparzystego, ponadto znak współczynnika przy\linebreak najwyższej potędze x jest ujemny.\\ W związku z tym wykres wielomianu zaczyna się od lewej strony powyżej osi OX. A więc $$x \in (-\infty,-20) \cup (-7,13).$$
\rozwStop
\odpStart
$x \in (-\infty,-20) \cup (-7,13)$
\odpStop
\testStart
A.$x \in (-\infty,-20) \cup (-7,13)$\\
B.$x \in (-\infty,-20) \cup (-7,13]$\\
C.$x \in (-\infty,-20) \cup [-7,13)$\\
D.$x \in (-\infty,-20] \cup (-7,13)$\\
E.$x \in (-\infty,-20] \cup (-7,13]$\\
F.$x \in (-\infty,-20] \cup [-7,13)$\\
G.$x \in (-\infty,-20) \cup [-7,13]$\\
H.$x \in (-\infty,-20] \cup [-7,13]$
\testStop
\kluczStart
A
\kluczStop



\zadStart{Zadanie z Wikieł Z 1.62 b) moja wersja nr 1043}

Rozwiązać nierówności $(x+20)(13-x)(x+8)\ge0$.
\zadStop
\rozwStart{Patryk Wirkus}{}
Miejsca zerowe naszego wielomianu to: $-20, 13, -8$.\\
Wielomian jest stopnia nieparzystego, ponadto znak współczynnika przy\linebreak najwyższej potędze x jest ujemny.\\ W związku z tym wykres wielomianu zaczyna się od lewej strony powyżej osi OX. A więc $$x \in (-\infty,-20) \cup (-8,13).$$
\rozwStop
\odpStart
$x \in (-\infty,-20) \cup (-8,13)$
\odpStop
\testStart
A.$x \in (-\infty,-20) \cup (-8,13)$\\
B.$x \in (-\infty,-20) \cup (-8,13]$\\
C.$x \in (-\infty,-20) \cup [-8,13)$\\
D.$x \in (-\infty,-20] \cup (-8,13)$\\
E.$x \in (-\infty,-20] \cup (-8,13]$\\
F.$x \in (-\infty,-20] \cup [-8,13)$\\
G.$x \in (-\infty,-20) \cup [-8,13]$\\
H.$x \in (-\infty,-20] \cup [-8,13]$
\testStop
\kluczStart
A
\kluczStop



\zadStart{Zadanie z Wikieł Z 1.62 b) moja wersja nr 1044}

Rozwiązać nierówności $(x+20)(13-x)(x+9)\ge0$.
\zadStop
\rozwStart{Patryk Wirkus}{}
Miejsca zerowe naszego wielomianu to: $-20, 13, -9$.\\
Wielomian jest stopnia nieparzystego, ponadto znak współczynnika przy\linebreak najwyższej potędze x jest ujemny.\\ W związku z tym wykres wielomianu zaczyna się od lewej strony powyżej osi OX. A więc $$x \in (-\infty,-20) \cup (-9,13).$$
\rozwStop
\odpStart
$x \in (-\infty,-20) \cup (-9,13)$
\odpStop
\testStart
A.$x \in (-\infty,-20) \cup (-9,13)$\\
B.$x \in (-\infty,-20) \cup (-9,13]$\\
C.$x \in (-\infty,-20) \cup [-9,13)$\\
D.$x \in (-\infty,-20] \cup (-9,13)$\\
E.$x \in (-\infty,-20] \cup (-9,13]$\\
F.$x \in (-\infty,-20] \cup [-9,13)$\\
G.$x \in (-\infty,-20) \cup [-9,13]$\\
H.$x \in (-\infty,-20] \cup [-9,13]$
\testStop
\kluczStart
A
\kluczStop



\zadStart{Zadanie z Wikieł Z 1.62 b) moja wersja nr 1045}

Rozwiązać nierówności $(x+20)(13-x)(x+10)\ge0$.
\zadStop
\rozwStart{Patryk Wirkus}{}
Miejsca zerowe naszego wielomianu to: $-20, 13, -10$.\\
Wielomian jest stopnia nieparzystego, ponadto znak współczynnika przy\linebreak najwyższej potędze x jest ujemny.\\ W związku z tym wykres wielomianu zaczyna się od lewej strony powyżej osi OX. A więc $$x \in (-\infty,-20) \cup (-10,13).$$
\rozwStop
\odpStart
$x \in (-\infty,-20) \cup (-10,13)$
\odpStop
\testStart
A.$x \in (-\infty,-20) \cup (-10,13)$\\
B.$x \in (-\infty,-20) \cup (-10,13]$\\
C.$x \in (-\infty,-20) \cup [-10,13)$\\
D.$x \in (-\infty,-20] \cup (-10,13)$\\
E.$x \in (-\infty,-20] \cup (-10,13]$\\
F.$x \in (-\infty,-20] \cup [-10,13)$\\
G.$x \in (-\infty,-20) \cup [-10,13]$\\
H.$x \in (-\infty,-20] \cup [-10,13]$
\testStop
\kluczStart
A
\kluczStop



\zadStart{Zadanie z Wikieł Z 1.62 b) moja wersja nr 1046}

Rozwiązać nierówności $(x+20)(13-x)(x+11)\ge0$.
\zadStop
\rozwStart{Patryk Wirkus}{}
Miejsca zerowe naszego wielomianu to: $-20, 13, -11$.\\
Wielomian jest stopnia nieparzystego, ponadto znak współczynnika przy\linebreak najwyższej potędze x jest ujemny.\\ W związku z tym wykres wielomianu zaczyna się od lewej strony powyżej osi OX. A więc $$x \in (-\infty,-20) \cup (-11,13).$$
\rozwStop
\odpStart
$x \in (-\infty,-20) \cup (-11,13)$
\odpStop
\testStart
A.$x \in (-\infty,-20) \cup (-11,13)$\\
B.$x \in (-\infty,-20) \cup (-11,13]$\\
C.$x \in (-\infty,-20) \cup [-11,13)$\\
D.$x \in (-\infty,-20] \cup (-11,13)$\\
E.$x \in (-\infty,-20] \cup (-11,13]$\\
F.$x \in (-\infty,-20] \cup [-11,13)$\\
G.$x \in (-\infty,-20) \cup [-11,13]$\\
H.$x \in (-\infty,-20] \cup [-11,13]$
\testStop
\kluczStart
A
\kluczStop



\zadStart{Zadanie z Wikieł Z 1.62 b) moja wersja nr 1047}

Rozwiązać nierówności $(x+20)(13-x)(x+12)\ge0$.
\zadStop
\rozwStart{Patryk Wirkus}{}
Miejsca zerowe naszego wielomianu to: $-20, 13, -12$.\\
Wielomian jest stopnia nieparzystego, ponadto znak współczynnika przy\linebreak najwyższej potędze x jest ujemny.\\ W związku z tym wykres wielomianu zaczyna się od lewej strony powyżej osi OX. A więc $$x \in (-\infty,-20) \cup (-12,13).$$
\rozwStop
\odpStart
$x \in (-\infty,-20) \cup (-12,13)$
\odpStop
\testStart
A.$x \in (-\infty,-20) \cup (-12,13)$\\
B.$x \in (-\infty,-20) \cup (-12,13]$\\
C.$x \in (-\infty,-20) \cup [-12,13)$\\
D.$x \in (-\infty,-20] \cup (-12,13)$\\
E.$x \in (-\infty,-20] \cup (-12,13]$\\
F.$x \in (-\infty,-20] \cup [-12,13)$\\
G.$x \in (-\infty,-20) \cup [-12,13]$\\
H.$x \in (-\infty,-20] \cup [-12,13]$
\testStop
\kluczStart
A
\kluczStop



\zadStart{Zadanie z Wikieł Z 1.62 b) moja wersja nr 1048}

Rozwiązać nierówności $(x+20)(14-x)(x+1)\ge0$.
\zadStop
\rozwStart{Patryk Wirkus}{}
Miejsca zerowe naszego wielomianu to: $-20, 14, -1$.\\
Wielomian jest stopnia nieparzystego, ponadto znak współczynnika przy\linebreak najwyższej potędze x jest ujemny.\\ W związku z tym wykres wielomianu zaczyna się od lewej strony powyżej osi OX. A więc $$x \in (-\infty,-20) \cup (-1,14).$$
\rozwStop
\odpStart
$x \in (-\infty,-20) \cup (-1,14)$
\odpStop
\testStart
A.$x \in (-\infty,-20) \cup (-1,14)$\\
B.$x \in (-\infty,-20) \cup (-1,14]$\\
C.$x \in (-\infty,-20) \cup [-1,14)$\\
D.$x \in (-\infty,-20] \cup (-1,14)$\\
E.$x \in (-\infty,-20] \cup (-1,14]$\\
F.$x \in (-\infty,-20] \cup [-1,14)$\\
G.$x \in (-\infty,-20) \cup [-1,14]$\\
H.$x \in (-\infty,-20] \cup [-1,14]$
\testStop
\kluczStart
A
\kluczStop



\zadStart{Zadanie z Wikieł Z 1.62 b) moja wersja nr 1049}

Rozwiązać nierówności $(x+20)(14-x)(x+2)\ge0$.
\zadStop
\rozwStart{Patryk Wirkus}{}
Miejsca zerowe naszego wielomianu to: $-20, 14, -2$.\\
Wielomian jest stopnia nieparzystego, ponadto znak współczynnika przy\linebreak najwyższej potędze x jest ujemny.\\ W związku z tym wykres wielomianu zaczyna się od lewej strony powyżej osi OX. A więc $$x \in (-\infty,-20) \cup (-2,14).$$
\rozwStop
\odpStart
$x \in (-\infty,-20) \cup (-2,14)$
\odpStop
\testStart
A.$x \in (-\infty,-20) \cup (-2,14)$\\
B.$x \in (-\infty,-20) \cup (-2,14]$\\
C.$x \in (-\infty,-20) \cup [-2,14)$\\
D.$x \in (-\infty,-20] \cup (-2,14)$\\
E.$x \in (-\infty,-20] \cup (-2,14]$\\
F.$x \in (-\infty,-20] \cup [-2,14)$\\
G.$x \in (-\infty,-20) \cup [-2,14]$\\
H.$x \in (-\infty,-20] \cup [-2,14]$
\testStop
\kluczStart
A
\kluczStop



\zadStart{Zadanie z Wikieł Z 1.62 b) moja wersja nr 1050}

Rozwiązać nierówności $(x+20)(14-x)(x+3)\ge0$.
\zadStop
\rozwStart{Patryk Wirkus}{}
Miejsca zerowe naszego wielomianu to: $-20, 14, -3$.\\
Wielomian jest stopnia nieparzystego, ponadto znak współczynnika przy\linebreak najwyższej potędze x jest ujemny.\\ W związku z tym wykres wielomianu zaczyna się od lewej strony powyżej osi OX. A więc $$x \in (-\infty,-20) \cup (-3,14).$$
\rozwStop
\odpStart
$x \in (-\infty,-20) \cup (-3,14)$
\odpStop
\testStart
A.$x \in (-\infty,-20) \cup (-3,14)$\\
B.$x \in (-\infty,-20) \cup (-3,14]$\\
C.$x \in (-\infty,-20) \cup [-3,14)$\\
D.$x \in (-\infty,-20] \cup (-3,14)$\\
E.$x \in (-\infty,-20] \cup (-3,14]$\\
F.$x \in (-\infty,-20] \cup [-3,14)$\\
G.$x \in (-\infty,-20) \cup [-3,14]$\\
H.$x \in (-\infty,-20] \cup [-3,14]$
\testStop
\kluczStart
A
\kluczStop



\zadStart{Zadanie z Wikieł Z 1.62 b) moja wersja nr 1051}

Rozwiązać nierówności $(x+20)(14-x)(x+4)\ge0$.
\zadStop
\rozwStart{Patryk Wirkus}{}
Miejsca zerowe naszego wielomianu to: $-20, 14, -4$.\\
Wielomian jest stopnia nieparzystego, ponadto znak współczynnika przy\linebreak najwyższej potędze x jest ujemny.\\ W związku z tym wykres wielomianu zaczyna się od lewej strony powyżej osi OX. A więc $$x \in (-\infty,-20) \cup (-4,14).$$
\rozwStop
\odpStart
$x \in (-\infty,-20) \cup (-4,14)$
\odpStop
\testStart
A.$x \in (-\infty,-20) \cup (-4,14)$\\
B.$x \in (-\infty,-20) \cup (-4,14]$\\
C.$x \in (-\infty,-20) \cup [-4,14)$\\
D.$x \in (-\infty,-20] \cup (-4,14)$\\
E.$x \in (-\infty,-20] \cup (-4,14]$\\
F.$x \in (-\infty,-20] \cup [-4,14)$\\
G.$x \in (-\infty,-20) \cup [-4,14]$\\
H.$x \in (-\infty,-20] \cup [-4,14]$
\testStop
\kluczStart
A
\kluczStop



\zadStart{Zadanie z Wikieł Z 1.62 b) moja wersja nr 1052}

Rozwiązać nierówności $(x+20)(14-x)(x+5)\ge0$.
\zadStop
\rozwStart{Patryk Wirkus}{}
Miejsca zerowe naszego wielomianu to: $-20, 14, -5$.\\
Wielomian jest stopnia nieparzystego, ponadto znak współczynnika przy\linebreak najwyższej potędze x jest ujemny.\\ W związku z tym wykres wielomianu zaczyna się od lewej strony powyżej osi OX. A więc $$x \in (-\infty,-20) \cup (-5,14).$$
\rozwStop
\odpStart
$x \in (-\infty,-20) \cup (-5,14)$
\odpStop
\testStart
A.$x \in (-\infty,-20) \cup (-5,14)$\\
B.$x \in (-\infty,-20) \cup (-5,14]$\\
C.$x \in (-\infty,-20) \cup [-5,14)$\\
D.$x \in (-\infty,-20] \cup (-5,14)$\\
E.$x \in (-\infty,-20] \cup (-5,14]$\\
F.$x \in (-\infty,-20] \cup [-5,14)$\\
G.$x \in (-\infty,-20) \cup [-5,14]$\\
H.$x \in (-\infty,-20] \cup [-5,14]$
\testStop
\kluczStart
A
\kluczStop



\zadStart{Zadanie z Wikieł Z 1.62 b) moja wersja nr 1053}

Rozwiązać nierówności $(x+20)(14-x)(x+6)\ge0$.
\zadStop
\rozwStart{Patryk Wirkus}{}
Miejsca zerowe naszego wielomianu to: $-20, 14, -6$.\\
Wielomian jest stopnia nieparzystego, ponadto znak współczynnika przy\linebreak najwyższej potędze x jest ujemny.\\ W związku z tym wykres wielomianu zaczyna się od lewej strony powyżej osi OX. A więc $$x \in (-\infty,-20) \cup (-6,14).$$
\rozwStop
\odpStart
$x \in (-\infty,-20) \cup (-6,14)$
\odpStop
\testStart
A.$x \in (-\infty,-20) \cup (-6,14)$\\
B.$x \in (-\infty,-20) \cup (-6,14]$\\
C.$x \in (-\infty,-20) \cup [-6,14)$\\
D.$x \in (-\infty,-20] \cup (-6,14)$\\
E.$x \in (-\infty,-20] \cup (-6,14]$\\
F.$x \in (-\infty,-20] \cup [-6,14)$\\
G.$x \in (-\infty,-20) \cup [-6,14]$\\
H.$x \in (-\infty,-20] \cup [-6,14]$
\testStop
\kluczStart
A
\kluczStop



\zadStart{Zadanie z Wikieł Z 1.62 b) moja wersja nr 1054}

Rozwiązać nierówności $(x+20)(14-x)(x+7)\ge0$.
\zadStop
\rozwStart{Patryk Wirkus}{}
Miejsca zerowe naszego wielomianu to: $-20, 14, -7$.\\
Wielomian jest stopnia nieparzystego, ponadto znak współczynnika przy\linebreak najwyższej potędze x jest ujemny.\\ W związku z tym wykres wielomianu zaczyna się od lewej strony powyżej osi OX. A więc $$x \in (-\infty,-20) \cup (-7,14).$$
\rozwStop
\odpStart
$x \in (-\infty,-20) \cup (-7,14)$
\odpStop
\testStart
A.$x \in (-\infty,-20) \cup (-7,14)$\\
B.$x \in (-\infty,-20) \cup (-7,14]$\\
C.$x \in (-\infty,-20) \cup [-7,14)$\\
D.$x \in (-\infty,-20] \cup (-7,14)$\\
E.$x \in (-\infty,-20] \cup (-7,14]$\\
F.$x \in (-\infty,-20] \cup [-7,14)$\\
G.$x \in (-\infty,-20) \cup [-7,14]$\\
H.$x \in (-\infty,-20] \cup [-7,14]$
\testStop
\kluczStart
A
\kluczStop



\zadStart{Zadanie z Wikieł Z 1.62 b) moja wersja nr 1055}

Rozwiązać nierówności $(x+20)(14-x)(x+8)\ge0$.
\zadStop
\rozwStart{Patryk Wirkus}{}
Miejsca zerowe naszego wielomianu to: $-20, 14, -8$.\\
Wielomian jest stopnia nieparzystego, ponadto znak współczynnika przy\linebreak najwyższej potędze x jest ujemny.\\ W związku z tym wykres wielomianu zaczyna się od lewej strony powyżej osi OX. A więc $$x \in (-\infty,-20) \cup (-8,14).$$
\rozwStop
\odpStart
$x \in (-\infty,-20) \cup (-8,14)$
\odpStop
\testStart
A.$x \in (-\infty,-20) \cup (-8,14)$\\
B.$x \in (-\infty,-20) \cup (-8,14]$\\
C.$x \in (-\infty,-20) \cup [-8,14)$\\
D.$x \in (-\infty,-20] \cup (-8,14)$\\
E.$x \in (-\infty,-20] \cup (-8,14]$\\
F.$x \in (-\infty,-20] \cup [-8,14)$\\
G.$x \in (-\infty,-20) \cup [-8,14]$\\
H.$x \in (-\infty,-20] \cup [-8,14]$
\testStop
\kluczStart
A
\kluczStop



\zadStart{Zadanie z Wikieł Z 1.62 b) moja wersja nr 1056}

Rozwiązać nierówności $(x+20)(14-x)(x+9)\ge0$.
\zadStop
\rozwStart{Patryk Wirkus}{}
Miejsca zerowe naszego wielomianu to: $-20, 14, -9$.\\
Wielomian jest stopnia nieparzystego, ponadto znak współczynnika przy\linebreak najwyższej potędze x jest ujemny.\\ W związku z tym wykres wielomianu zaczyna się od lewej strony powyżej osi OX. A więc $$x \in (-\infty,-20) \cup (-9,14).$$
\rozwStop
\odpStart
$x \in (-\infty,-20) \cup (-9,14)$
\odpStop
\testStart
A.$x \in (-\infty,-20) \cup (-9,14)$\\
B.$x \in (-\infty,-20) \cup (-9,14]$\\
C.$x \in (-\infty,-20) \cup [-9,14)$\\
D.$x \in (-\infty,-20] \cup (-9,14)$\\
E.$x \in (-\infty,-20] \cup (-9,14]$\\
F.$x \in (-\infty,-20] \cup [-9,14)$\\
G.$x \in (-\infty,-20) \cup [-9,14]$\\
H.$x \in (-\infty,-20] \cup [-9,14]$
\testStop
\kluczStart
A
\kluczStop



\zadStart{Zadanie z Wikieł Z 1.62 b) moja wersja nr 1057}

Rozwiązać nierówności $(x+20)(14-x)(x+10)\ge0$.
\zadStop
\rozwStart{Patryk Wirkus}{}
Miejsca zerowe naszego wielomianu to: $-20, 14, -10$.\\
Wielomian jest stopnia nieparzystego, ponadto znak współczynnika przy\linebreak najwyższej potędze x jest ujemny.\\ W związku z tym wykres wielomianu zaczyna się od lewej strony powyżej osi OX. A więc $$x \in (-\infty,-20) \cup (-10,14).$$
\rozwStop
\odpStart
$x \in (-\infty,-20) \cup (-10,14)$
\odpStop
\testStart
A.$x \in (-\infty,-20) \cup (-10,14)$\\
B.$x \in (-\infty,-20) \cup (-10,14]$\\
C.$x \in (-\infty,-20) \cup [-10,14)$\\
D.$x \in (-\infty,-20] \cup (-10,14)$\\
E.$x \in (-\infty,-20] \cup (-10,14]$\\
F.$x \in (-\infty,-20] \cup [-10,14)$\\
G.$x \in (-\infty,-20) \cup [-10,14]$\\
H.$x \in (-\infty,-20] \cup [-10,14]$
\testStop
\kluczStart
A
\kluczStop



\zadStart{Zadanie z Wikieł Z 1.62 b) moja wersja nr 1058}

Rozwiązać nierówności $(x+20)(14-x)(x+11)\ge0$.
\zadStop
\rozwStart{Patryk Wirkus}{}
Miejsca zerowe naszego wielomianu to: $-20, 14, -11$.\\
Wielomian jest stopnia nieparzystego, ponadto znak współczynnika przy\linebreak najwyższej potędze x jest ujemny.\\ W związku z tym wykres wielomianu zaczyna się od lewej strony powyżej osi OX. A więc $$x \in (-\infty,-20) \cup (-11,14).$$
\rozwStop
\odpStart
$x \in (-\infty,-20) \cup (-11,14)$
\odpStop
\testStart
A.$x \in (-\infty,-20) \cup (-11,14)$\\
B.$x \in (-\infty,-20) \cup (-11,14]$\\
C.$x \in (-\infty,-20) \cup [-11,14)$\\
D.$x \in (-\infty,-20] \cup (-11,14)$\\
E.$x \in (-\infty,-20] \cup (-11,14]$\\
F.$x \in (-\infty,-20] \cup [-11,14)$\\
G.$x \in (-\infty,-20) \cup [-11,14]$\\
H.$x \in (-\infty,-20] \cup [-11,14]$
\testStop
\kluczStart
A
\kluczStop



\zadStart{Zadanie z Wikieł Z 1.62 b) moja wersja nr 1059}

Rozwiązać nierówności $(x+20)(14-x)(x+12)\ge0$.
\zadStop
\rozwStart{Patryk Wirkus}{}
Miejsca zerowe naszego wielomianu to: $-20, 14, -12$.\\
Wielomian jest stopnia nieparzystego, ponadto znak współczynnika przy\linebreak najwyższej potędze x jest ujemny.\\ W związku z tym wykres wielomianu zaczyna się od lewej strony powyżej osi OX. A więc $$x \in (-\infty,-20) \cup (-12,14).$$
\rozwStop
\odpStart
$x \in (-\infty,-20) \cup (-12,14)$
\odpStop
\testStart
A.$x \in (-\infty,-20) \cup (-12,14)$\\
B.$x \in (-\infty,-20) \cup (-12,14]$\\
C.$x \in (-\infty,-20) \cup [-12,14)$\\
D.$x \in (-\infty,-20] \cup (-12,14)$\\
E.$x \in (-\infty,-20] \cup (-12,14]$\\
F.$x \in (-\infty,-20] \cup [-12,14)$\\
G.$x \in (-\infty,-20) \cup [-12,14]$\\
H.$x \in (-\infty,-20] \cup [-12,14]$
\testStop
\kluczStart
A
\kluczStop



\zadStart{Zadanie z Wikieł Z 1.62 b) moja wersja nr 1060}

Rozwiązać nierówności $(x+20)(14-x)(x+13)\ge0$.
\zadStop
\rozwStart{Patryk Wirkus}{}
Miejsca zerowe naszego wielomianu to: $-20, 14, -13$.\\
Wielomian jest stopnia nieparzystego, ponadto znak współczynnika przy\linebreak najwyższej potędze x jest ujemny.\\ W związku z tym wykres wielomianu zaczyna się od lewej strony powyżej osi OX. A więc $$x \in (-\infty,-20) \cup (-13,14).$$
\rozwStop
\odpStart
$x \in (-\infty,-20) \cup (-13,14)$
\odpStop
\testStart
A.$x \in (-\infty,-20) \cup (-13,14)$\\
B.$x \in (-\infty,-20) \cup (-13,14]$\\
C.$x \in (-\infty,-20) \cup [-13,14)$\\
D.$x \in (-\infty,-20] \cup (-13,14)$\\
E.$x \in (-\infty,-20] \cup (-13,14]$\\
F.$x \in (-\infty,-20] \cup [-13,14)$\\
G.$x \in (-\infty,-20) \cup [-13,14]$\\
H.$x \in (-\infty,-20] \cup [-13,14]$
\testStop
\kluczStart
A
\kluczStop



\zadStart{Zadanie z Wikieł Z 1.62 b) moja wersja nr 1061}

Rozwiązać nierówności $(x+20)(15-x)(x+1)\ge0$.
\zadStop
\rozwStart{Patryk Wirkus}{}
Miejsca zerowe naszego wielomianu to: $-20, 15, -1$.\\
Wielomian jest stopnia nieparzystego, ponadto znak współczynnika przy\linebreak najwyższej potędze x jest ujemny.\\ W związku z tym wykres wielomianu zaczyna się od lewej strony powyżej osi OX. A więc $$x \in (-\infty,-20) \cup (-1,15).$$
\rozwStop
\odpStart
$x \in (-\infty,-20) \cup (-1,15)$
\odpStop
\testStart
A.$x \in (-\infty,-20) \cup (-1,15)$\\
B.$x \in (-\infty,-20) \cup (-1,15]$\\
C.$x \in (-\infty,-20) \cup [-1,15)$\\
D.$x \in (-\infty,-20] \cup (-1,15)$\\
E.$x \in (-\infty,-20] \cup (-1,15]$\\
F.$x \in (-\infty,-20] \cup [-1,15)$\\
G.$x \in (-\infty,-20) \cup [-1,15]$\\
H.$x \in (-\infty,-20] \cup [-1,15]$
\testStop
\kluczStart
A
\kluczStop



\zadStart{Zadanie z Wikieł Z 1.62 b) moja wersja nr 1062}

Rozwiązać nierówności $(x+20)(15-x)(x+2)\ge0$.
\zadStop
\rozwStart{Patryk Wirkus}{}
Miejsca zerowe naszego wielomianu to: $-20, 15, -2$.\\
Wielomian jest stopnia nieparzystego, ponadto znak współczynnika przy\linebreak najwyższej potędze x jest ujemny.\\ W związku z tym wykres wielomianu zaczyna się od lewej strony powyżej osi OX. A więc $$x \in (-\infty,-20) \cup (-2,15).$$
\rozwStop
\odpStart
$x \in (-\infty,-20) \cup (-2,15)$
\odpStop
\testStart
A.$x \in (-\infty,-20) \cup (-2,15)$\\
B.$x \in (-\infty,-20) \cup (-2,15]$\\
C.$x \in (-\infty,-20) \cup [-2,15)$\\
D.$x \in (-\infty,-20] \cup (-2,15)$\\
E.$x \in (-\infty,-20] \cup (-2,15]$\\
F.$x \in (-\infty,-20] \cup [-2,15)$\\
G.$x \in (-\infty,-20) \cup [-2,15]$\\
H.$x \in (-\infty,-20] \cup [-2,15]$
\testStop
\kluczStart
A
\kluczStop



\zadStart{Zadanie z Wikieł Z 1.62 b) moja wersja nr 1063}

Rozwiązać nierówności $(x+20)(15-x)(x+3)\ge0$.
\zadStop
\rozwStart{Patryk Wirkus}{}
Miejsca zerowe naszego wielomianu to: $-20, 15, -3$.\\
Wielomian jest stopnia nieparzystego, ponadto znak współczynnika przy\linebreak najwyższej potędze x jest ujemny.\\ W związku z tym wykres wielomianu zaczyna się od lewej strony powyżej osi OX. A więc $$x \in (-\infty,-20) \cup (-3,15).$$
\rozwStop
\odpStart
$x \in (-\infty,-20) \cup (-3,15)$
\odpStop
\testStart
A.$x \in (-\infty,-20) \cup (-3,15)$\\
B.$x \in (-\infty,-20) \cup (-3,15]$\\
C.$x \in (-\infty,-20) \cup [-3,15)$\\
D.$x \in (-\infty,-20] \cup (-3,15)$\\
E.$x \in (-\infty,-20] \cup (-3,15]$\\
F.$x \in (-\infty,-20] \cup [-3,15)$\\
G.$x \in (-\infty,-20) \cup [-3,15]$\\
H.$x \in (-\infty,-20] \cup [-3,15]$
\testStop
\kluczStart
A
\kluczStop



\zadStart{Zadanie z Wikieł Z 1.62 b) moja wersja nr 1064}

Rozwiązać nierówności $(x+20)(15-x)(x+4)\ge0$.
\zadStop
\rozwStart{Patryk Wirkus}{}
Miejsca zerowe naszego wielomianu to: $-20, 15, -4$.\\
Wielomian jest stopnia nieparzystego, ponadto znak współczynnika przy\linebreak najwyższej potędze x jest ujemny.\\ W związku z tym wykres wielomianu zaczyna się od lewej strony powyżej osi OX. A więc $$x \in (-\infty,-20) \cup (-4,15).$$
\rozwStop
\odpStart
$x \in (-\infty,-20) \cup (-4,15)$
\odpStop
\testStart
A.$x \in (-\infty,-20) \cup (-4,15)$\\
B.$x \in (-\infty,-20) \cup (-4,15]$\\
C.$x \in (-\infty,-20) \cup [-4,15)$\\
D.$x \in (-\infty,-20] \cup (-4,15)$\\
E.$x \in (-\infty,-20] \cup (-4,15]$\\
F.$x \in (-\infty,-20] \cup [-4,15)$\\
G.$x \in (-\infty,-20) \cup [-4,15]$\\
H.$x \in (-\infty,-20] \cup [-4,15]$
\testStop
\kluczStart
A
\kluczStop



\zadStart{Zadanie z Wikieł Z 1.62 b) moja wersja nr 1065}

Rozwiązać nierówności $(x+20)(15-x)(x+5)\ge0$.
\zadStop
\rozwStart{Patryk Wirkus}{}
Miejsca zerowe naszego wielomianu to: $-20, 15, -5$.\\
Wielomian jest stopnia nieparzystego, ponadto znak współczynnika przy\linebreak najwyższej potędze x jest ujemny.\\ W związku z tym wykres wielomianu zaczyna się od lewej strony powyżej osi OX. A więc $$x \in (-\infty,-20) \cup (-5,15).$$
\rozwStop
\odpStart
$x \in (-\infty,-20) \cup (-5,15)$
\odpStop
\testStart
A.$x \in (-\infty,-20) \cup (-5,15)$\\
B.$x \in (-\infty,-20) \cup (-5,15]$\\
C.$x \in (-\infty,-20) \cup [-5,15)$\\
D.$x \in (-\infty,-20] \cup (-5,15)$\\
E.$x \in (-\infty,-20] \cup (-5,15]$\\
F.$x \in (-\infty,-20] \cup [-5,15)$\\
G.$x \in (-\infty,-20) \cup [-5,15]$\\
H.$x \in (-\infty,-20] \cup [-5,15]$
\testStop
\kluczStart
A
\kluczStop



\zadStart{Zadanie z Wikieł Z 1.62 b) moja wersja nr 1066}

Rozwiązać nierówności $(x+20)(15-x)(x+6)\ge0$.
\zadStop
\rozwStart{Patryk Wirkus}{}
Miejsca zerowe naszego wielomianu to: $-20, 15, -6$.\\
Wielomian jest stopnia nieparzystego, ponadto znak współczynnika przy\linebreak najwyższej potędze x jest ujemny.\\ W związku z tym wykres wielomianu zaczyna się od lewej strony powyżej osi OX. A więc $$x \in (-\infty,-20) \cup (-6,15).$$
\rozwStop
\odpStart
$x \in (-\infty,-20) \cup (-6,15)$
\odpStop
\testStart
A.$x \in (-\infty,-20) \cup (-6,15)$\\
B.$x \in (-\infty,-20) \cup (-6,15]$\\
C.$x \in (-\infty,-20) \cup [-6,15)$\\
D.$x \in (-\infty,-20] \cup (-6,15)$\\
E.$x \in (-\infty,-20] \cup (-6,15]$\\
F.$x \in (-\infty,-20] \cup [-6,15)$\\
G.$x \in (-\infty,-20) \cup [-6,15]$\\
H.$x \in (-\infty,-20] \cup [-6,15]$
\testStop
\kluczStart
A
\kluczStop



\zadStart{Zadanie z Wikieł Z 1.62 b) moja wersja nr 1067}

Rozwiązać nierówności $(x+20)(15-x)(x+7)\ge0$.
\zadStop
\rozwStart{Patryk Wirkus}{}
Miejsca zerowe naszego wielomianu to: $-20, 15, -7$.\\
Wielomian jest stopnia nieparzystego, ponadto znak współczynnika przy\linebreak najwyższej potędze x jest ujemny.\\ W związku z tym wykres wielomianu zaczyna się od lewej strony powyżej osi OX. A więc $$x \in (-\infty,-20) \cup (-7,15).$$
\rozwStop
\odpStart
$x \in (-\infty,-20) \cup (-7,15)$
\odpStop
\testStart
A.$x \in (-\infty,-20) \cup (-7,15)$\\
B.$x \in (-\infty,-20) \cup (-7,15]$\\
C.$x \in (-\infty,-20) \cup [-7,15)$\\
D.$x \in (-\infty,-20] \cup (-7,15)$\\
E.$x \in (-\infty,-20] \cup (-7,15]$\\
F.$x \in (-\infty,-20] \cup [-7,15)$\\
G.$x \in (-\infty,-20) \cup [-7,15]$\\
H.$x \in (-\infty,-20] \cup [-7,15]$
\testStop
\kluczStart
A
\kluczStop



\zadStart{Zadanie z Wikieł Z 1.62 b) moja wersja nr 1068}

Rozwiązać nierówności $(x+20)(15-x)(x+8)\ge0$.
\zadStop
\rozwStart{Patryk Wirkus}{}
Miejsca zerowe naszego wielomianu to: $-20, 15, -8$.\\
Wielomian jest stopnia nieparzystego, ponadto znak współczynnika przy\linebreak najwyższej potędze x jest ujemny.\\ W związku z tym wykres wielomianu zaczyna się od lewej strony powyżej osi OX. A więc $$x \in (-\infty,-20) \cup (-8,15).$$
\rozwStop
\odpStart
$x \in (-\infty,-20) \cup (-8,15)$
\odpStop
\testStart
A.$x \in (-\infty,-20) \cup (-8,15)$\\
B.$x \in (-\infty,-20) \cup (-8,15]$\\
C.$x \in (-\infty,-20) \cup [-8,15)$\\
D.$x \in (-\infty,-20] \cup (-8,15)$\\
E.$x \in (-\infty,-20] \cup (-8,15]$\\
F.$x \in (-\infty,-20] \cup [-8,15)$\\
G.$x \in (-\infty,-20) \cup [-8,15]$\\
H.$x \in (-\infty,-20] \cup [-8,15]$
\testStop
\kluczStart
A
\kluczStop



\zadStart{Zadanie z Wikieł Z 1.62 b) moja wersja nr 1069}

Rozwiązać nierówności $(x+20)(15-x)(x+9)\ge0$.
\zadStop
\rozwStart{Patryk Wirkus}{}
Miejsca zerowe naszego wielomianu to: $-20, 15, -9$.\\
Wielomian jest stopnia nieparzystego, ponadto znak współczynnika przy\linebreak najwyższej potędze x jest ujemny.\\ W związku z tym wykres wielomianu zaczyna się od lewej strony powyżej osi OX. A więc $$x \in (-\infty,-20) \cup (-9,15).$$
\rozwStop
\odpStart
$x \in (-\infty,-20) \cup (-9,15)$
\odpStop
\testStart
A.$x \in (-\infty,-20) \cup (-9,15)$\\
B.$x \in (-\infty,-20) \cup (-9,15]$\\
C.$x \in (-\infty,-20) \cup [-9,15)$\\
D.$x \in (-\infty,-20] \cup (-9,15)$\\
E.$x \in (-\infty,-20] \cup (-9,15]$\\
F.$x \in (-\infty,-20] \cup [-9,15)$\\
G.$x \in (-\infty,-20) \cup [-9,15]$\\
H.$x \in (-\infty,-20] \cup [-9,15]$
\testStop
\kluczStart
A
\kluczStop



\zadStart{Zadanie z Wikieł Z 1.62 b) moja wersja nr 1070}

Rozwiązać nierówności $(x+20)(15-x)(x+10)\ge0$.
\zadStop
\rozwStart{Patryk Wirkus}{}
Miejsca zerowe naszego wielomianu to: $-20, 15, -10$.\\
Wielomian jest stopnia nieparzystego, ponadto znak współczynnika przy\linebreak najwyższej potędze x jest ujemny.\\ W związku z tym wykres wielomianu zaczyna się od lewej strony powyżej osi OX. A więc $$x \in (-\infty,-20) \cup (-10,15).$$
\rozwStop
\odpStart
$x \in (-\infty,-20) \cup (-10,15)$
\odpStop
\testStart
A.$x \in (-\infty,-20) \cup (-10,15)$\\
B.$x \in (-\infty,-20) \cup (-10,15]$\\
C.$x \in (-\infty,-20) \cup [-10,15)$\\
D.$x \in (-\infty,-20] \cup (-10,15)$\\
E.$x \in (-\infty,-20] \cup (-10,15]$\\
F.$x \in (-\infty,-20] \cup [-10,15)$\\
G.$x \in (-\infty,-20) \cup [-10,15]$\\
H.$x \in (-\infty,-20] \cup [-10,15]$
\testStop
\kluczStart
A
\kluczStop



\zadStart{Zadanie z Wikieł Z 1.62 b) moja wersja nr 1071}

Rozwiązać nierówności $(x+20)(15-x)(x+11)\ge0$.
\zadStop
\rozwStart{Patryk Wirkus}{}
Miejsca zerowe naszego wielomianu to: $-20, 15, -11$.\\
Wielomian jest stopnia nieparzystego, ponadto znak współczynnika przy\linebreak najwyższej potędze x jest ujemny.\\ W związku z tym wykres wielomianu zaczyna się od lewej strony powyżej osi OX. A więc $$x \in (-\infty,-20) \cup (-11,15).$$
\rozwStop
\odpStart
$x \in (-\infty,-20) \cup (-11,15)$
\odpStop
\testStart
A.$x \in (-\infty,-20) \cup (-11,15)$\\
B.$x \in (-\infty,-20) \cup (-11,15]$\\
C.$x \in (-\infty,-20) \cup [-11,15)$\\
D.$x \in (-\infty,-20] \cup (-11,15)$\\
E.$x \in (-\infty,-20] \cup (-11,15]$\\
F.$x \in (-\infty,-20] \cup [-11,15)$\\
G.$x \in (-\infty,-20) \cup [-11,15]$\\
H.$x \in (-\infty,-20] \cup [-11,15]$
\testStop
\kluczStart
A
\kluczStop



\zadStart{Zadanie z Wikieł Z 1.62 b) moja wersja nr 1072}

Rozwiązać nierówności $(x+20)(15-x)(x+12)\ge0$.
\zadStop
\rozwStart{Patryk Wirkus}{}
Miejsca zerowe naszego wielomianu to: $-20, 15, -12$.\\
Wielomian jest stopnia nieparzystego, ponadto znak współczynnika przy\linebreak najwyższej potędze x jest ujemny.\\ W związku z tym wykres wielomianu zaczyna się od lewej strony powyżej osi OX. A więc $$x \in (-\infty,-20) \cup (-12,15).$$
\rozwStop
\odpStart
$x \in (-\infty,-20) \cup (-12,15)$
\odpStop
\testStart
A.$x \in (-\infty,-20) \cup (-12,15)$\\
B.$x \in (-\infty,-20) \cup (-12,15]$\\
C.$x \in (-\infty,-20) \cup [-12,15)$\\
D.$x \in (-\infty,-20] \cup (-12,15)$\\
E.$x \in (-\infty,-20] \cup (-12,15]$\\
F.$x \in (-\infty,-20] \cup [-12,15)$\\
G.$x \in (-\infty,-20) \cup [-12,15]$\\
H.$x \in (-\infty,-20] \cup [-12,15]$
\testStop
\kluczStart
A
\kluczStop



\zadStart{Zadanie z Wikieł Z 1.62 b) moja wersja nr 1073}

Rozwiązać nierówności $(x+20)(15-x)(x+13)\ge0$.
\zadStop
\rozwStart{Patryk Wirkus}{}
Miejsca zerowe naszego wielomianu to: $-20, 15, -13$.\\
Wielomian jest stopnia nieparzystego, ponadto znak współczynnika przy\linebreak najwyższej potędze x jest ujemny.\\ W związku z tym wykres wielomianu zaczyna się od lewej strony powyżej osi OX. A więc $$x \in (-\infty,-20) \cup (-13,15).$$
\rozwStop
\odpStart
$x \in (-\infty,-20) \cup (-13,15)$
\odpStop
\testStart
A.$x \in (-\infty,-20) \cup (-13,15)$\\
B.$x \in (-\infty,-20) \cup (-13,15]$\\
C.$x \in (-\infty,-20) \cup [-13,15)$\\
D.$x \in (-\infty,-20] \cup (-13,15)$\\
E.$x \in (-\infty,-20] \cup (-13,15]$\\
F.$x \in (-\infty,-20] \cup [-13,15)$\\
G.$x \in (-\infty,-20) \cup [-13,15]$\\
H.$x \in (-\infty,-20] \cup [-13,15]$
\testStop
\kluczStart
A
\kluczStop



\zadStart{Zadanie z Wikieł Z 1.62 b) moja wersja nr 1074}

Rozwiązać nierówności $(x+20)(15-x)(x+14)\ge0$.
\zadStop
\rozwStart{Patryk Wirkus}{}
Miejsca zerowe naszego wielomianu to: $-20, 15, -14$.\\
Wielomian jest stopnia nieparzystego, ponadto znak współczynnika przy\linebreak najwyższej potędze x jest ujemny.\\ W związku z tym wykres wielomianu zaczyna się od lewej strony powyżej osi OX. A więc $$x \in (-\infty,-20) \cup (-14,15).$$
\rozwStop
\odpStart
$x \in (-\infty,-20) \cup (-14,15)$
\odpStop
\testStart
A.$x \in (-\infty,-20) \cup (-14,15)$\\
B.$x \in (-\infty,-20) \cup (-14,15]$\\
C.$x \in (-\infty,-20) \cup [-14,15)$\\
D.$x \in (-\infty,-20] \cup (-14,15)$\\
E.$x \in (-\infty,-20] \cup (-14,15]$\\
F.$x \in (-\infty,-20] \cup [-14,15)$\\
G.$x \in (-\infty,-20) \cup [-14,15]$\\
H.$x \in (-\infty,-20] \cup [-14,15]$
\testStop
\kluczStart
A
\kluczStop



\zadStart{Zadanie z Wikieł Z 1.62 b) moja wersja nr 1075}

Rozwiązać nierówności $(x+20)(16-x)(x+1)\ge0$.
\zadStop
\rozwStart{Patryk Wirkus}{}
Miejsca zerowe naszego wielomianu to: $-20, 16, -1$.\\
Wielomian jest stopnia nieparzystego, ponadto znak współczynnika przy\linebreak najwyższej potędze x jest ujemny.\\ W związku z tym wykres wielomianu zaczyna się od lewej strony powyżej osi OX. A więc $$x \in (-\infty,-20) \cup (-1,16).$$
\rozwStop
\odpStart
$x \in (-\infty,-20) \cup (-1,16)$
\odpStop
\testStart
A.$x \in (-\infty,-20) \cup (-1,16)$\\
B.$x \in (-\infty,-20) \cup (-1,16]$\\
C.$x \in (-\infty,-20) \cup [-1,16)$\\
D.$x \in (-\infty,-20] \cup (-1,16)$\\
E.$x \in (-\infty,-20] \cup (-1,16]$\\
F.$x \in (-\infty,-20] \cup [-1,16)$\\
G.$x \in (-\infty,-20) \cup [-1,16]$\\
H.$x \in (-\infty,-20] \cup [-1,16]$
\testStop
\kluczStart
A
\kluczStop



\zadStart{Zadanie z Wikieł Z 1.62 b) moja wersja nr 1076}

Rozwiązać nierówności $(x+20)(16-x)(x+2)\ge0$.
\zadStop
\rozwStart{Patryk Wirkus}{}
Miejsca zerowe naszego wielomianu to: $-20, 16, -2$.\\
Wielomian jest stopnia nieparzystego, ponadto znak współczynnika przy\linebreak najwyższej potędze x jest ujemny.\\ W związku z tym wykres wielomianu zaczyna się od lewej strony powyżej osi OX. A więc $$x \in (-\infty,-20) \cup (-2,16).$$
\rozwStop
\odpStart
$x \in (-\infty,-20) \cup (-2,16)$
\odpStop
\testStart
A.$x \in (-\infty,-20) \cup (-2,16)$\\
B.$x \in (-\infty,-20) \cup (-2,16]$\\
C.$x \in (-\infty,-20) \cup [-2,16)$\\
D.$x \in (-\infty,-20] \cup (-2,16)$\\
E.$x \in (-\infty,-20] \cup (-2,16]$\\
F.$x \in (-\infty,-20] \cup [-2,16)$\\
G.$x \in (-\infty,-20) \cup [-2,16]$\\
H.$x \in (-\infty,-20] \cup [-2,16]$
\testStop
\kluczStart
A
\kluczStop



\zadStart{Zadanie z Wikieł Z 1.62 b) moja wersja nr 1077}

Rozwiązać nierówności $(x+20)(16-x)(x+3)\ge0$.
\zadStop
\rozwStart{Patryk Wirkus}{}
Miejsca zerowe naszego wielomianu to: $-20, 16, -3$.\\
Wielomian jest stopnia nieparzystego, ponadto znak współczynnika przy\linebreak najwyższej potędze x jest ujemny.\\ W związku z tym wykres wielomianu zaczyna się od lewej strony powyżej osi OX. A więc $$x \in (-\infty,-20) \cup (-3,16).$$
\rozwStop
\odpStart
$x \in (-\infty,-20) \cup (-3,16)$
\odpStop
\testStart
A.$x \in (-\infty,-20) \cup (-3,16)$\\
B.$x \in (-\infty,-20) \cup (-3,16]$\\
C.$x \in (-\infty,-20) \cup [-3,16)$\\
D.$x \in (-\infty,-20] \cup (-3,16)$\\
E.$x \in (-\infty,-20] \cup (-3,16]$\\
F.$x \in (-\infty,-20] \cup [-3,16)$\\
G.$x \in (-\infty,-20) \cup [-3,16]$\\
H.$x \in (-\infty,-20] \cup [-3,16]$
\testStop
\kluczStart
A
\kluczStop



\zadStart{Zadanie z Wikieł Z 1.62 b) moja wersja nr 1078}

Rozwiązać nierówności $(x+20)(16-x)(x+4)\ge0$.
\zadStop
\rozwStart{Patryk Wirkus}{}
Miejsca zerowe naszego wielomianu to: $-20, 16, -4$.\\
Wielomian jest stopnia nieparzystego, ponadto znak współczynnika przy\linebreak najwyższej potędze x jest ujemny.\\ W związku z tym wykres wielomianu zaczyna się od lewej strony powyżej osi OX. A więc $$x \in (-\infty,-20) \cup (-4,16).$$
\rozwStop
\odpStart
$x \in (-\infty,-20) \cup (-4,16)$
\odpStop
\testStart
A.$x \in (-\infty,-20) \cup (-4,16)$\\
B.$x \in (-\infty,-20) \cup (-4,16]$\\
C.$x \in (-\infty,-20) \cup [-4,16)$\\
D.$x \in (-\infty,-20] \cup (-4,16)$\\
E.$x \in (-\infty,-20] \cup (-4,16]$\\
F.$x \in (-\infty,-20] \cup [-4,16)$\\
G.$x \in (-\infty,-20) \cup [-4,16]$\\
H.$x \in (-\infty,-20] \cup [-4,16]$
\testStop
\kluczStart
A
\kluczStop



\zadStart{Zadanie z Wikieł Z 1.62 b) moja wersja nr 1079}

Rozwiązać nierówności $(x+20)(16-x)(x+5)\ge0$.
\zadStop
\rozwStart{Patryk Wirkus}{}
Miejsca zerowe naszego wielomianu to: $-20, 16, -5$.\\
Wielomian jest stopnia nieparzystego, ponadto znak współczynnika przy\linebreak najwyższej potędze x jest ujemny.\\ W związku z tym wykres wielomianu zaczyna się od lewej strony powyżej osi OX. A więc $$x \in (-\infty,-20) \cup (-5,16).$$
\rozwStop
\odpStart
$x \in (-\infty,-20) \cup (-5,16)$
\odpStop
\testStart
A.$x \in (-\infty,-20) \cup (-5,16)$\\
B.$x \in (-\infty,-20) \cup (-5,16]$\\
C.$x \in (-\infty,-20) \cup [-5,16)$\\
D.$x \in (-\infty,-20] \cup (-5,16)$\\
E.$x \in (-\infty,-20] \cup (-5,16]$\\
F.$x \in (-\infty,-20] \cup [-5,16)$\\
G.$x \in (-\infty,-20) \cup [-5,16]$\\
H.$x \in (-\infty,-20] \cup [-5,16]$
\testStop
\kluczStart
A
\kluczStop



\zadStart{Zadanie z Wikieł Z 1.62 b) moja wersja nr 1080}

Rozwiązać nierówności $(x+20)(16-x)(x+6)\ge0$.
\zadStop
\rozwStart{Patryk Wirkus}{}
Miejsca zerowe naszego wielomianu to: $-20, 16, -6$.\\
Wielomian jest stopnia nieparzystego, ponadto znak współczynnika przy\linebreak najwyższej potędze x jest ujemny.\\ W związku z tym wykres wielomianu zaczyna się od lewej strony powyżej osi OX. A więc $$x \in (-\infty,-20) \cup (-6,16).$$
\rozwStop
\odpStart
$x \in (-\infty,-20) \cup (-6,16)$
\odpStop
\testStart
A.$x \in (-\infty,-20) \cup (-6,16)$\\
B.$x \in (-\infty,-20) \cup (-6,16]$\\
C.$x \in (-\infty,-20) \cup [-6,16)$\\
D.$x \in (-\infty,-20] \cup (-6,16)$\\
E.$x \in (-\infty,-20] \cup (-6,16]$\\
F.$x \in (-\infty,-20] \cup [-6,16)$\\
G.$x \in (-\infty,-20) \cup [-6,16]$\\
H.$x \in (-\infty,-20] \cup [-6,16]$
\testStop
\kluczStart
A
\kluczStop



\zadStart{Zadanie z Wikieł Z 1.62 b) moja wersja nr 1081}

Rozwiązać nierówności $(x+20)(16-x)(x+7)\ge0$.
\zadStop
\rozwStart{Patryk Wirkus}{}
Miejsca zerowe naszego wielomianu to: $-20, 16, -7$.\\
Wielomian jest stopnia nieparzystego, ponadto znak współczynnika przy\linebreak najwyższej potędze x jest ujemny.\\ W związku z tym wykres wielomianu zaczyna się od lewej strony powyżej osi OX. A więc $$x \in (-\infty,-20) \cup (-7,16).$$
\rozwStop
\odpStart
$x \in (-\infty,-20) \cup (-7,16)$
\odpStop
\testStart
A.$x \in (-\infty,-20) \cup (-7,16)$\\
B.$x \in (-\infty,-20) \cup (-7,16]$\\
C.$x \in (-\infty,-20) \cup [-7,16)$\\
D.$x \in (-\infty,-20] \cup (-7,16)$\\
E.$x \in (-\infty,-20] \cup (-7,16]$\\
F.$x \in (-\infty,-20] \cup [-7,16)$\\
G.$x \in (-\infty,-20) \cup [-7,16]$\\
H.$x \in (-\infty,-20] \cup [-7,16]$
\testStop
\kluczStart
A
\kluczStop



\zadStart{Zadanie z Wikieł Z 1.62 b) moja wersja nr 1082}

Rozwiązać nierówności $(x+20)(16-x)(x+8)\ge0$.
\zadStop
\rozwStart{Patryk Wirkus}{}
Miejsca zerowe naszego wielomianu to: $-20, 16, -8$.\\
Wielomian jest stopnia nieparzystego, ponadto znak współczynnika przy\linebreak najwyższej potędze x jest ujemny.\\ W związku z tym wykres wielomianu zaczyna się od lewej strony powyżej osi OX. A więc $$x \in (-\infty,-20) \cup (-8,16).$$
\rozwStop
\odpStart
$x \in (-\infty,-20) \cup (-8,16)$
\odpStop
\testStart
A.$x \in (-\infty,-20) \cup (-8,16)$\\
B.$x \in (-\infty,-20) \cup (-8,16]$\\
C.$x \in (-\infty,-20) \cup [-8,16)$\\
D.$x \in (-\infty,-20] \cup (-8,16)$\\
E.$x \in (-\infty,-20] \cup (-8,16]$\\
F.$x \in (-\infty,-20] \cup [-8,16)$\\
G.$x \in (-\infty,-20) \cup [-8,16]$\\
H.$x \in (-\infty,-20] \cup [-8,16]$
\testStop
\kluczStart
A
\kluczStop



\zadStart{Zadanie z Wikieł Z 1.62 b) moja wersja nr 1083}

Rozwiązać nierówności $(x+20)(16-x)(x+9)\ge0$.
\zadStop
\rozwStart{Patryk Wirkus}{}
Miejsca zerowe naszego wielomianu to: $-20, 16, -9$.\\
Wielomian jest stopnia nieparzystego, ponadto znak współczynnika przy\linebreak najwyższej potędze x jest ujemny.\\ W związku z tym wykres wielomianu zaczyna się od lewej strony powyżej osi OX. A więc $$x \in (-\infty,-20) \cup (-9,16).$$
\rozwStop
\odpStart
$x \in (-\infty,-20) \cup (-9,16)$
\odpStop
\testStart
A.$x \in (-\infty,-20) \cup (-9,16)$\\
B.$x \in (-\infty,-20) \cup (-9,16]$\\
C.$x \in (-\infty,-20) \cup [-9,16)$\\
D.$x \in (-\infty,-20] \cup (-9,16)$\\
E.$x \in (-\infty,-20] \cup (-9,16]$\\
F.$x \in (-\infty,-20] \cup [-9,16)$\\
G.$x \in (-\infty,-20) \cup [-9,16]$\\
H.$x \in (-\infty,-20] \cup [-9,16]$
\testStop
\kluczStart
A
\kluczStop



\zadStart{Zadanie z Wikieł Z 1.62 b) moja wersja nr 1084}

Rozwiązać nierówności $(x+20)(16-x)(x+10)\ge0$.
\zadStop
\rozwStart{Patryk Wirkus}{}
Miejsca zerowe naszego wielomianu to: $-20, 16, -10$.\\
Wielomian jest stopnia nieparzystego, ponadto znak współczynnika przy\linebreak najwyższej potędze x jest ujemny.\\ W związku z tym wykres wielomianu zaczyna się od lewej strony powyżej osi OX. A więc $$x \in (-\infty,-20) \cup (-10,16).$$
\rozwStop
\odpStart
$x \in (-\infty,-20) \cup (-10,16)$
\odpStop
\testStart
A.$x \in (-\infty,-20) \cup (-10,16)$\\
B.$x \in (-\infty,-20) \cup (-10,16]$\\
C.$x \in (-\infty,-20) \cup [-10,16)$\\
D.$x \in (-\infty,-20] \cup (-10,16)$\\
E.$x \in (-\infty,-20] \cup (-10,16]$\\
F.$x \in (-\infty,-20] \cup [-10,16)$\\
G.$x \in (-\infty,-20) \cup [-10,16]$\\
H.$x \in (-\infty,-20] \cup [-10,16]$
\testStop
\kluczStart
A
\kluczStop



\zadStart{Zadanie z Wikieł Z 1.62 b) moja wersja nr 1085}

Rozwiązać nierówności $(x+20)(16-x)(x+11)\ge0$.
\zadStop
\rozwStart{Patryk Wirkus}{}
Miejsca zerowe naszego wielomianu to: $-20, 16, -11$.\\
Wielomian jest stopnia nieparzystego, ponadto znak współczynnika przy\linebreak najwyższej potędze x jest ujemny.\\ W związku z tym wykres wielomianu zaczyna się od lewej strony powyżej osi OX. A więc $$x \in (-\infty,-20) \cup (-11,16).$$
\rozwStop
\odpStart
$x \in (-\infty,-20) \cup (-11,16)$
\odpStop
\testStart
A.$x \in (-\infty,-20) \cup (-11,16)$\\
B.$x \in (-\infty,-20) \cup (-11,16]$\\
C.$x \in (-\infty,-20) \cup [-11,16)$\\
D.$x \in (-\infty,-20] \cup (-11,16)$\\
E.$x \in (-\infty,-20] \cup (-11,16]$\\
F.$x \in (-\infty,-20] \cup [-11,16)$\\
G.$x \in (-\infty,-20) \cup [-11,16]$\\
H.$x \in (-\infty,-20] \cup [-11,16]$
\testStop
\kluczStart
A
\kluczStop



\zadStart{Zadanie z Wikieł Z 1.62 b) moja wersja nr 1086}

Rozwiązać nierówności $(x+20)(16-x)(x+12)\ge0$.
\zadStop
\rozwStart{Patryk Wirkus}{}
Miejsca zerowe naszego wielomianu to: $-20, 16, -12$.\\
Wielomian jest stopnia nieparzystego, ponadto znak współczynnika przy\linebreak najwyższej potędze x jest ujemny.\\ W związku z tym wykres wielomianu zaczyna się od lewej strony powyżej osi OX. A więc $$x \in (-\infty,-20) \cup (-12,16).$$
\rozwStop
\odpStart
$x \in (-\infty,-20) \cup (-12,16)$
\odpStop
\testStart
A.$x \in (-\infty,-20) \cup (-12,16)$\\
B.$x \in (-\infty,-20) \cup (-12,16]$\\
C.$x \in (-\infty,-20) \cup [-12,16)$\\
D.$x \in (-\infty,-20] \cup (-12,16)$\\
E.$x \in (-\infty,-20] \cup (-12,16]$\\
F.$x \in (-\infty,-20] \cup [-12,16)$\\
G.$x \in (-\infty,-20) \cup [-12,16]$\\
H.$x \in (-\infty,-20] \cup [-12,16]$
\testStop
\kluczStart
A
\kluczStop



\zadStart{Zadanie z Wikieł Z 1.62 b) moja wersja nr 1087}

Rozwiązać nierówności $(x+20)(16-x)(x+13)\ge0$.
\zadStop
\rozwStart{Patryk Wirkus}{}
Miejsca zerowe naszego wielomianu to: $-20, 16, -13$.\\
Wielomian jest stopnia nieparzystego, ponadto znak współczynnika przy\linebreak najwyższej potędze x jest ujemny.\\ W związku z tym wykres wielomianu zaczyna się od lewej strony powyżej osi OX. A więc $$x \in (-\infty,-20) \cup (-13,16).$$
\rozwStop
\odpStart
$x \in (-\infty,-20) \cup (-13,16)$
\odpStop
\testStart
A.$x \in (-\infty,-20) \cup (-13,16)$\\
B.$x \in (-\infty,-20) \cup (-13,16]$\\
C.$x \in (-\infty,-20) \cup [-13,16)$\\
D.$x \in (-\infty,-20] \cup (-13,16)$\\
E.$x \in (-\infty,-20] \cup (-13,16]$\\
F.$x \in (-\infty,-20] \cup [-13,16)$\\
G.$x \in (-\infty,-20) \cup [-13,16]$\\
H.$x \in (-\infty,-20] \cup [-13,16]$
\testStop
\kluczStart
A
\kluczStop



\zadStart{Zadanie z Wikieł Z 1.62 b) moja wersja nr 1088}

Rozwiązać nierówności $(x+20)(16-x)(x+14)\ge0$.
\zadStop
\rozwStart{Patryk Wirkus}{}
Miejsca zerowe naszego wielomianu to: $-20, 16, -14$.\\
Wielomian jest stopnia nieparzystego, ponadto znak współczynnika przy\linebreak najwyższej potędze x jest ujemny.\\ W związku z tym wykres wielomianu zaczyna się od lewej strony powyżej osi OX. A więc $$x \in (-\infty,-20) \cup (-14,16).$$
\rozwStop
\odpStart
$x \in (-\infty,-20) \cup (-14,16)$
\odpStop
\testStart
A.$x \in (-\infty,-20) \cup (-14,16)$\\
B.$x \in (-\infty,-20) \cup (-14,16]$\\
C.$x \in (-\infty,-20) \cup [-14,16)$\\
D.$x \in (-\infty,-20] \cup (-14,16)$\\
E.$x \in (-\infty,-20] \cup (-14,16]$\\
F.$x \in (-\infty,-20] \cup [-14,16)$\\
G.$x \in (-\infty,-20) \cup [-14,16]$\\
H.$x \in (-\infty,-20] \cup [-14,16]$
\testStop
\kluczStart
A
\kluczStop



\zadStart{Zadanie z Wikieł Z 1.62 b) moja wersja nr 1089}

Rozwiązać nierówności $(x+20)(16-x)(x+15)\ge0$.
\zadStop
\rozwStart{Patryk Wirkus}{}
Miejsca zerowe naszego wielomianu to: $-20, 16, -15$.\\
Wielomian jest stopnia nieparzystego, ponadto znak współczynnika przy\linebreak najwyższej potędze x jest ujemny.\\ W związku z tym wykres wielomianu zaczyna się od lewej strony powyżej osi OX. A więc $$x \in (-\infty,-20) \cup (-15,16).$$
\rozwStop
\odpStart
$x \in (-\infty,-20) \cup (-15,16)$
\odpStop
\testStart
A.$x \in (-\infty,-20) \cup (-15,16)$\\
B.$x \in (-\infty,-20) \cup (-15,16]$\\
C.$x \in (-\infty,-20) \cup [-15,16)$\\
D.$x \in (-\infty,-20] \cup (-15,16)$\\
E.$x \in (-\infty,-20] \cup (-15,16]$\\
F.$x \in (-\infty,-20] \cup [-15,16)$\\
G.$x \in (-\infty,-20) \cup [-15,16]$\\
H.$x \in (-\infty,-20] \cup [-15,16]$
\testStop
\kluczStart
A
\kluczStop



\zadStart{Zadanie z Wikieł Z 1.62 b) moja wersja nr 1090}

Rozwiązać nierówności $(x+20)(17-x)(x+1)\ge0$.
\zadStop
\rozwStart{Patryk Wirkus}{}
Miejsca zerowe naszego wielomianu to: $-20, 17, -1$.\\
Wielomian jest stopnia nieparzystego, ponadto znak współczynnika przy\linebreak najwyższej potędze x jest ujemny.\\ W związku z tym wykres wielomianu zaczyna się od lewej strony powyżej osi OX. A więc $$x \in (-\infty,-20) \cup (-1,17).$$
\rozwStop
\odpStart
$x \in (-\infty,-20) \cup (-1,17)$
\odpStop
\testStart
A.$x \in (-\infty,-20) \cup (-1,17)$\\
B.$x \in (-\infty,-20) \cup (-1,17]$\\
C.$x \in (-\infty,-20) \cup [-1,17)$\\
D.$x \in (-\infty,-20] \cup (-1,17)$\\
E.$x \in (-\infty,-20] \cup (-1,17]$\\
F.$x \in (-\infty,-20] \cup [-1,17)$\\
G.$x \in (-\infty,-20) \cup [-1,17]$\\
H.$x \in (-\infty,-20] \cup [-1,17]$
\testStop
\kluczStart
A
\kluczStop



\zadStart{Zadanie z Wikieł Z 1.62 b) moja wersja nr 1091}

Rozwiązać nierówności $(x+20)(17-x)(x+2)\ge0$.
\zadStop
\rozwStart{Patryk Wirkus}{}
Miejsca zerowe naszego wielomianu to: $-20, 17, -2$.\\
Wielomian jest stopnia nieparzystego, ponadto znak współczynnika przy\linebreak najwyższej potędze x jest ujemny.\\ W związku z tym wykres wielomianu zaczyna się od lewej strony powyżej osi OX. A więc $$x \in (-\infty,-20) \cup (-2,17).$$
\rozwStop
\odpStart
$x \in (-\infty,-20) \cup (-2,17)$
\odpStop
\testStart
A.$x \in (-\infty,-20) \cup (-2,17)$\\
B.$x \in (-\infty,-20) \cup (-2,17]$\\
C.$x \in (-\infty,-20) \cup [-2,17)$\\
D.$x \in (-\infty,-20] \cup (-2,17)$\\
E.$x \in (-\infty,-20] \cup (-2,17]$\\
F.$x \in (-\infty,-20] \cup [-2,17)$\\
G.$x \in (-\infty,-20) \cup [-2,17]$\\
H.$x \in (-\infty,-20] \cup [-2,17]$
\testStop
\kluczStart
A
\kluczStop



\zadStart{Zadanie z Wikieł Z 1.62 b) moja wersja nr 1092}

Rozwiązać nierówności $(x+20)(17-x)(x+3)\ge0$.
\zadStop
\rozwStart{Patryk Wirkus}{}
Miejsca zerowe naszego wielomianu to: $-20, 17, -3$.\\
Wielomian jest stopnia nieparzystego, ponadto znak współczynnika przy\linebreak najwyższej potędze x jest ujemny.\\ W związku z tym wykres wielomianu zaczyna się od lewej strony powyżej osi OX. A więc $$x \in (-\infty,-20) \cup (-3,17).$$
\rozwStop
\odpStart
$x \in (-\infty,-20) \cup (-3,17)$
\odpStop
\testStart
A.$x \in (-\infty,-20) \cup (-3,17)$\\
B.$x \in (-\infty,-20) \cup (-3,17]$\\
C.$x \in (-\infty,-20) \cup [-3,17)$\\
D.$x \in (-\infty,-20] \cup (-3,17)$\\
E.$x \in (-\infty,-20] \cup (-3,17]$\\
F.$x \in (-\infty,-20] \cup [-3,17)$\\
G.$x \in (-\infty,-20) \cup [-3,17]$\\
H.$x \in (-\infty,-20] \cup [-3,17]$
\testStop
\kluczStart
A
\kluczStop



\zadStart{Zadanie z Wikieł Z 1.62 b) moja wersja nr 1093}

Rozwiązać nierówności $(x+20)(17-x)(x+4)\ge0$.
\zadStop
\rozwStart{Patryk Wirkus}{}
Miejsca zerowe naszego wielomianu to: $-20, 17, -4$.\\
Wielomian jest stopnia nieparzystego, ponadto znak współczynnika przy\linebreak najwyższej potędze x jest ujemny.\\ W związku z tym wykres wielomianu zaczyna się od lewej strony powyżej osi OX. A więc $$x \in (-\infty,-20) \cup (-4,17).$$
\rozwStop
\odpStart
$x \in (-\infty,-20) \cup (-4,17)$
\odpStop
\testStart
A.$x \in (-\infty,-20) \cup (-4,17)$\\
B.$x \in (-\infty,-20) \cup (-4,17]$\\
C.$x \in (-\infty,-20) \cup [-4,17)$\\
D.$x \in (-\infty,-20] \cup (-4,17)$\\
E.$x \in (-\infty,-20] \cup (-4,17]$\\
F.$x \in (-\infty,-20] \cup [-4,17)$\\
G.$x \in (-\infty,-20) \cup [-4,17]$\\
H.$x \in (-\infty,-20] \cup [-4,17]$
\testStop
\kluczStart
A
\kluczStop



\zadStart{Zadanie z Wikieł Z 1.62 b) moja wersja nr 1094}

Rozwiązać nierówności $(x+20)(17-x)(x+5)\ge0$.
\zadStop
\rozwStart{Patryk Wirkus}{}
Miejsca zerowe naszego wielomianu to: $-20, 17, -5$.\\
Wielomian jest stopnia nieparzystego, ponadto znak współczynnika przy\linebreak najwyższej potędze x jest ujemny.\\ W związku z tym wykres wielomianu zaczyna się od lewej strony powyżej osi OX. A więc $$x \in (-\infty,-20) \cup (-5,17).$$
\rozwStop
\odpStart
$x \in (-\infty,-20) \cup (-5,17)$
\odpStop
\testStart
A.$x \in (-\infty,-20) \cup (-5,17)$\\
B.$x \in (-\infty,-20) \cup (-5,17]$\\
C.$x \in (-\infty,-20) \cup [-5,17)$\\
D.$x \in (-\infty,-20] \cup (-5,17)$\\
E.$x \in (-\infty,-20] \cup (-5,17]$\\
F.$x \in (-\infty,-20] \cup [-5,17)$\\
G.$x \in (-\infty,-20) \cup [-5,17]$\\
H.$x \in (-\infty,-20] \cup [-5,17]$
\testStop
\kluczStart
A
\kluczStop



\zadStart{Zadanie z Wikieł Z 1.62 b) moja wersja nr 1095}

Rozwiązać nierówności $(x+20)(17-x)(x+6)\ge0$.
\zadStop
\rozwStart{Patryk Wirkus}{}
Miejsca zerowe naszego wielomianu to: $-20, 17, -6$.\\
Wielomian jest stopnia nieparzystego, ponadto znak współczynnika przy\linebreak najwyższej potędze x jest ujemny.\\ W związku z tym wykres wielomianu zaczyna się od lewej strony powyżej osi OX. A więc $$x \in (-\infty,-20) \cup (-6,17).$$
\rozwStop
\odpStart
$x \in (-\infty,-20) \cup (-6,17)$
\odpStop
\testStart
A.$x \in (-\infty,-20) \cup (-6,17)$\\
B.$x \in (-\infty,-20) \cup (-6,17]$\\
C.$x \in (-\infty,-20) \cup [-6,17)$\\
D.$x \in (-\infty,-20] \cup (-6,17)$\\
E.$x \in (-\infty,-20] \cup (-6,17]$\\
F.$x \in (-\infty,-20] \cup [-6,17)$\\
G.$x \in (-\infty,-20) \cup [-6,17]$\\
H.$x \in (-\infty,-20] \cup [-6,17]$
\testStop
\kluczStart
A
\kluczStop



\zadStart{Zadanie z Wikieł Z 1.62 b) moja wersja nr 1096}

Rozwiązać nierówności $(x+20)(17-x)(x+7)\ge0$.
\zadStop
\rozwStart{Patryk Wirkus}{}
Miejsca zerowe naszego wielomianu to: $-20, 17, -7$.\\
Wielomian jest stopnia nieparzystego, ponadto znak współczynnika przy\linebreak najwyższej potędze x jest ujemny.\\ W związku z tym wykres wielomianu zaczyna się od lewej strony powyżej osi OX. A więc $$x \in (-\infty,-20) \cup (-7,17).$$
\rozwStop
\odpStart
$x \in (-\infty,-20) \cup (-7,17)$
\odpStop
\testStart
A.$x \in (-\infty,-20) \cup (-7,17)$\\
B.$x \in (-\infty,-20) \cup (-7,17]$\\
C.$x \in (-\infty,-20) \cup [-7,17)$\\
D.$x \in (-\infty,-20] \cup (-7,17)$\\
E.$x \in (-\infty,-20] \cup (-7,17]$\\
F.$x \in (-\infty,-20] \cup [-7,17)$\\
G.$x \in (-\infty,-20) \cup [-7,17]$\\
H.$x \in (-\infty,-20] \cup [-7,17]$
\testStop
\kluczStart
A
\kluczStop



\zadStart{Zadanie z Wikieł Z 1.62 b) moja wersja nr 1097}

Rozwiązać nierówności $(x+20)(17-x)(x+8)\ge0$.
\zadStop
\rozwStart{Patryk Wirkus}{}
Miejsca zerowe naszego wielomianu to: $-20, 17, -8$.\\
Wielomian jest stopnia nieparzystego, ponadto znak współczynnika przy\linebreak najwyższej potędze x jest ujemny.\\ W związku z tym wykres wielomianu zaczyna się od lewej strony powyżej osi OX. A więc $$x \in (-\infty,-20) \cup (-8,17).$$
\rozwStop
\odpStart
$x \in (-\infty,-20) \cup (-8,17)$
\odpStop
\testStart
A.$x \in (-\infty,-20) \cup (-8,17)$\\
B.$x \in (-\infty,-20) \cup (-8,17]$\\
C.$x \in (-\infty,-20) \cup [-8,17)$\\
D.$x \in (-\infty,-20] \cup (-8,17)$\\
E.$x \in (-\infty,-20] \cup (-8,17]$\\
F.$x \in (-\infty,-20] \cup [-8,17)$\\
G.$x \in (-\infty,-20) \cup [-8,17]$\\
H.$x \in (-\infty,-20] \cup [-8,17]$
\testStop
\kluczStart
A
\kluczStop



\zadStart{Zadanie z Wikieł Z 1.62 b) moja wersja nr 1098}

Rozwiązać nierówności $(x+20)(17-x)(x+9)\ge0$.
\zadStop
\rozwStart{Patryk Wirkus}{}
Miejsca zerowe naszego wielomianu to: $-20, 17, -9$.\\
Wielomian jest stopnia nieparzystego, ponadto znak współczynnika przy\linebreak najwyższej potędze x jest ujemny.\\ W związku z tym wykres wielomianu zaczyna się od lewej strony powyżej osi OX. A więc $$x \in (-\infty,-20) \cup (-9,17).$$
\rozwStop
\odpStart
$x \in (-\infty,-20) \cup (-9,17)$
\odpStop
\testStart
A.$x \in (-\infty,-20) \cup (-9,17)$\\
B.$x \in (-\infty,-20) \cup (-9,17]$\\
C.$x \in (-\infty,-20) \cup [-9,17)$\\
D.$x \in (-\infty,-20] \cup (-9,17)$\\
E.$x \in (-\infty,-20] \cup (-9,17]$\\
F.$x \in (-\infty,-20] \cup [-9,17)$\\
G.$x \in (-\infty,-20) \cup [-9,17]$\\
H.$x \in (-\infty,-20] \cup [-9,17]$
\testStop
\kluczStart
A
\kluczStop



\zadStart{Zadanie z Wikieł Z 1.62 b) moja wersja nr 1099}

Rozwiązać nierówności $(x+20)(17-x)(x+10)\ge0$.
\zadStop
\rozwStart{Patryk Wirkus}{}
Miejsca zerowe naszego wielomianu to: $-20, 17, -10$.\\
Wielomian jest stopnia nieparzystego, ponadto znak współczynnika przy\linebreak najwyższej potędze x jest ujemny.\\ W związku z tym wykres wielomianu zaczyna się od lewej strony powyżej osi OX. A więc $$x \in (-\infty,-20) \cup (-10,17).$$
\rozwStop
\odpStart
$x \in (-\infty,-20) \cup (-10,17)$
\odpStop
\testStart
A.$x \in (-\infty,-20) \cup (-10,17)$\\
B.$x \in (-\infty,-20) \cup (-10,17]$\\
C.$x \in (-\infty,-20) \cup [-10,17)$\\
D.$x \in (-\infty,-20] \cup (-10,17)$\\
E.$x \in (-\infty,-20] \cup (-10,17]$\\
F.$x \in (-\infty,-20] \cup [-10,17)$\\
G.$x \in (-\infty,-20) \cup [-10,17]$\\
H.$x \in (-\infty,-20] \cup [-10,17]$
\testStop
\kluczStart
A
\kluczStop



\zadStart{Zadanie z Wikieł Z 1.62 b) moja wersja nr 1100}

Rozwiązać nierówności $(x+20)(17-x)(x+11)\ge0$.
\zadStop
\rozwStart{Patryk Wirkus}{}
Miejsca zerowe naszego wielomianu to: $-20, 17, -11$.\\
Wielomian jest stopnia nieparzystego, ponadto znak współczynnika przy\linebreak najwyższej potędze x jest ujemny.\\ W związku z tym wykres wielomianu zaczyna się od lewej strony powyżej osi OX. A więc $$x \in (-\infty,-20) \cup (-11,17).$$
\rozwStop
\odpStart
$x \in (-\infty,-20) \cup (-11,17)$
\odpStop
\testStart
A.$x \in (-\infty,-20) \cup (-11,17)$\\
B.$x \in (-\infty,-20) \cup (-11,17]$\\
C.$x \in (-\infty,-20) \cup [-11,17)$\\
D.$x \in (-\infty,-20] \cup (-11,17)$\\
E.$x \in (-\infty,-20] \cup (-11,17]$\\
F.$x \in (-\infty,-20] \cup [-11,17)$\\
G.$x \in (-\infty,-20) \cup [-11,17]$\\
H.$x \in (-\infty,-20] \cup [-11,17]$
\testStop
\kluczStart
A
\kluczStop



\zadStart{Zadanie z Wikieł Z 1.62 b) moja wersja nr 1101}

Rozwiązać nierówności $(x+20)(17-x)(x+12)\ge0$.
\zadStop
\rozwStart{Patryk Wirkus}{}
Miejsca zerowe naszego wielomianu to: $-20, 17, -12$.\\
Wielomian jest stopnia nieparzystego, ponadto znak współczynnika przy\linebreak najwyższej potędze x jest ujemny.\\ W związku z tym wykres wielomianu zaczyna się od lewej strony powyżej osi OX. A więc $$x \in (-\infty,-20) \cup (-12,17).$$
\rozwStop
\odpStart
$x \in (-\infty,-20) \cup (-12,17)$
\odpStop
\testStart
A.$x \in (-\infty,-20) \cup (-12,17)$\\
B.$x \in (-\infty,-20) \cup (-12,17]$\\
C.$x \in (-\infty,-20) \cup [-12,17)$\\
D.$x \in (-\infty,-20] \cup (-12,17)$\\
E.$x \in (-\infty,-20] \cup (-12,17]$\\
F.$x \in (-\infty,-20] \cup [-12,17)$\\
G.$x \in (-\infty,-20) \cup [-12,17]$\\
H.$x \in (-\infty,-20] \cup [-12,17]$
\testStop
\kluczStart
A
\kluczStop



\zadStart{Zadanie z Wikieł Z 1.62 b) moja wersja nr 1102}

Rozwiązać nierówności $(x+20)(17-x)(x+13)\ge0$.
\zadStop
\rozwStart{Patryk Wirkus}{}
Miejsca zerowe naszego wielomianu to: $-20, 17, -13$.\\
Wielomian jest stopnia nieparzystego, ponadto znak współczynnika przy\linebreak najwyższej potędze x jest ujemny.\\ W związku z tym wykres wielomianu zaczyna się od lewej strony powyżej osi OX. A więc $$x \in (-\infty,-20) \cup (-13,17).$$
\rozwStop
\odpStart
$x \in (-\infty,-20) \cup (-13,17)$
\odpStop
\testStart
A.$x \in (-\infty,-20) \cup (-13,17)$\\
B.$x \in (-\infty,-20) \cup (-13,17]$\\
C.$x \in (-\infty,-20) \cup [-13,17)$\\
D.$x \in (-\infty,-20] \cup (-13,17)$\\
E.$x \in (-\infty,-20] \cup (-13,17]$\\
F.$x \in (-\infty,-20] \cup [-13,17)$\\
G.$x \in (-\infty,-20) \cup [-13,17]$\\
H.$x \in (-\infty,-20] \cup [-13,17]$
\testStop
\kluczStart
A
\kluczStop



\zadStart{Zadanie z Wikieł Z 1.62 b) moja wersja nr 1103}

Rozwiązać nierówności $(x+20)(17-x)(x+14)\ge0$.
\zadStop
\rozwStart{Patryk Wirkus}{}
Miejsca zerowe naszego wielomianu to: $-20, 17, -14$.\\
Wielomian jest stopnia nieparzystego, ponadto znak współczynnika przy\linebreak najwyższej potędze x jest ujemny.\\ W związku z tym wykres wielomianu zaczyna się od lewej strony powyżej osi OX. A więc $$x \in (-\infty,-20) \cup (-14,17).$$
\rozwStop
\odpStart
$x \in (-\infty,-20) \cup (-14,17)$
\odpStop
\testStart
A.$x \in (-\infty,-20) \cup (-14,17)$\\
B.$x \in (-\infty,-20) \cup (-14,17]$\\
C.$x \in (-\infty,-20) \cup [-14,17)$\\
D.$x \in (-\infty,-20] \cup (-14,17)$\\
E.$x \in (-\infty,-20] \cup (-14,17]$\\
F.$x \in (-\infty,-20] \cup [-14,17)$\\
G.$x \in (-\infty,-20) \cup [-14,17]$\\
H.$x \in (-\infty,-20] \cup [-14,17]$
\testStop
\kluczStart
A
\kluczStop



\zadStart{Zadanie z Wikieł Z 1.62 b) moja wersja nr 1104}

Rozwiązać nierówności $(x+20)(17-x)(x+15)\ge0$.
\zadStop
\rozwStart{Patryk Wirkus}{}
Miejsca zerowe naszego wielomianu to: $-20, 17, -15$.\\
Wielomian jest stopnia nieparzystego, ponadto znak współczynnika przy\linebreak najwyższej potędze x jest ujemny.\\ W związku z tym wykres wielomianu zaczyna się od lewej strony powyżej osi OX. A więc $$x \in (-\infty,-20) \cup (-15,17).$$
\rozwStop
\odpStart
$x \in (-\infty,-20) \cup (-15,17)$
\odpStop
\testStart
A.$x \in (-\infty,-20) \cup (-15,17)$\\
B.$x \in (-\infty,-20) \cup (-15,17]$\\
C.$x \in (-\infty,-20) \cup [-15,17)$\\
D.$x \in (-\infty,-20] \cup (-15,17)$\\
E.$x \in (-\infty,-20] \cup (-15,17]$\\
F.$x \in (-\infty,-20] \cup [-15,17)$\\
G.$x \in (-\infty,-20) \cup [-15,17]$\\
H.$x \in (-\infty,-20] \cup [-15,17]$
\testStop
\kluczStart
A
\kluczStop



\zadStart{Zadanie z Wikieł Z 1.62 b) moja wersja nr 1105}

Rozwiązać nierówności $(x+20)(17-x)(x+16)\ge0$.
\zadStop
\rozwStart{Patryk Wirkus}{}
Miejsca zerowe naszego wielomianu to: $-20, 17, -16$.\\
Wielomian jest stopnia nieparzystego, ponadto znak współczynnika przy\linebreak najwyższej potędze x jest ujemny.\\ W związku z tym wykres wielomianu zaczyna się od lewej strony powyżej osi OX. A więc $$x \in (-\infty,-20) \cup (-16,17).$$
\rozwStop
\odpStart
$x \in (-\infty,-20) \cup (-16,17)$
\odpStop
\testStart
A.$x \in (-\infty,-20) \cup (-16,17)$\\
B.$x \in (-\infty,-20) \cup (-16,17]$\\
C.$x \in (-\infty,-20) \cup [-16,17)$\\
D.$x \in (-\infty,-20] \cup (-16,17)$\\
E.$x \in (-\infty,-20] \cup (-16,17]$\\
F.$x \in (-\infty,-20] \cup [-16,17)$\\
G.$x \in (-\infty,-20) \cup [-16,17]$\\
H.$x \in (-\infty,-20] \cup [-16,17]$
\testStop
\kluczStart
A
\kluczStop



\zadStart{Zadanie z Wikieł Z 1.62 b) moja wersja nr 1106}

Rozwiązać nierówności $(x+20)(18-x)(x+1)\ge0$.
\zadStop
\rozwStart{Patryk Wirkus}{}
Miejsca zerowe naszego wielomianu to: $-20, 18, -1$.\\
Wielomian jest stopnia nieparzystego, ponadto znak współczynnika przy\linebreak najwyższej potędze x jest ujemny.\\ W związku z tym wykres wielomianu zaczyna się od lewej strony powyżej osi OX. A więc $$x \in (-\infty,-20) \cup (-1,18).$$
\rozwStop
\odpStart
$x \in (-\infty,-20) \cup (-1,18)$
\odpStop
\testStart
A.$x \in (-\infty,-20) \cup (-1,18)$\\
B.$x \in (-\infty,-20) \cup (-1,18]$\\
C.$x \in (-\infty,-20) \cup [-1,18)$\\
D.$x \in (-\infty,-20] \cup (-1,18)$\\
E.$x \in (-\infty,-20] \cup (-1,18]$\\
F.$x \in (-\infty,-20] \cup [-1,18)$\\
G.$x \in (-\infty,-20) \cup [-1,18]$\\
H.$x \in (-\infty,-20] \cup [-1,18]$
\testStop
\kluczStart
A
\kluczStop



\zadStart{Zadanie z Wikieł Z 1.62 b) moja wersja nr 1107}

Rozwiązać nierówności $(x+20)(18-x)(x+2)\ge0$.
\zadStop
\rozwStart{Patryk Wirkus}{}
Miejsca zerowe naszego wielomianu to: $-20, 18, -2$.\\
Wielomian jest stopnia nieparzystego, ponadto znak współczynnika przy\linebreak najwyższej potędze x jest ujemny.\\ W związku z tym wykres wielomianu zaczyna się od lewej strony powyżej osi OX. A więc $$x \in (-\infty,-20) \cup (-2,18).$$
\rozwStop
\odpStart
$x \in (-\infty,-20) \cup (-2,18)$
\odpStop
\testStart
A.$x \in (-\infty,-20) \cup (-2,18)$\\
B.$x \in (-\infty,-20) \cup (-2,18]$\\
C.$x \in (-\infty,-20) \cup [-2,18)$\\
D.$x \in (-\infty,-20] \cup (-2,18)$\\
E.$x \in (-\infty,-20] \cup (-2,18]$\\
F.$x \in (-\infty,-20] \cup [-2,18)$\\
G.$x \in (-\infty,-20) \cup [-2,18]$\\
H.$x \in (-\infty,-20] \cup [-2,18]$
\testStop
\kluczStart
A
\kluczStop



\zadStart{Zadanie z Wikieł Z 1.62 b) moja wersja nr 1108}

Rozwiązać nierówności $(x+20)(18-x)(x+3)\ge0$.
\zadStop
\rozwStart{Patryk Wirkus}{}
Miejsca zerowe naszego wielomianu to: $-20, 18, -3$.\\
Wielomian jest stopnia nieparzystego, ponadto znak współczynnika przy\linebreak najwyższej potędze x jest ujemny.\\ W związku z tym wykres wielomianu zaczyna się od lewej strony powyżej osi OX. A więc $$x \in (-\infty,-20) \cup (-3,18).$$
\rozwStop
\odpStart
$x \in (-\infty,-20) \cup (-3,18)$
\odpStop
\testStart
A.$x \in (-\infty,-20) \cup (-3,18)$\\
B.$x \in (-\infty,-20) \cup (-3,18]$\\
C.$x \in (-\infty,-20) \cup [-3,18)$\\
D.$x \in (-\infty,-20] \cup (-3,18)$\\
E.$x \in (-\infty,-20] \cup (-3,18]$\\
F.$x \in (-\infty,-20] \cup [-3,18)$\\
G.$x \in (-\infty,-20) \cup [-3,18]$\\
H.$x \in (-\infty,-20] \cup [-3,18]$
\testStop
\kluczStart
A
\kluczStop



\zadStart{Zadanie z Wikieł Z 1.62 b) moja wersja nr 1109}

Rozwiązać nierówności $(x+20)(18-x)(x+4)\ge0$.
\zadStop
\rozwStart{Patryk Wirkus}{}
Miejsca zerowe naszego wielomianu to: $-20, 18, -4$.\\
Wielomian jest stopnia nieparzystego, ponadto znak współczynnika przy\linebreak najwyższej potędze x jest ujemny.\\ W związku z tym wykres wielomianu zaczyna się od lewej strony powyżej osi OX. A więc $$x \in (-\infty,-20) \cup (-4,18).$$
\rozwStop
\odpStart
$x \in (-\infty,-20) \cup (-4,18)$
\odpStop
\testStart
A.$x \in (-\infty,-20) \cup (-4,18)$\\
B.$x \in (-\infty,-20) \cup (-4,18]$\\
C.$x \in (-\infty,-20) \cup [-4,18)$\\
D.$x \in (-\infty,-20] \cup (-4,18)$\\
E.$x \in (-\infty,-20] \cup (-4,18]$\\
F.$x \in (-\infty,-20] \cup [-4,18)$\\
G.$x \in (-\infty,-20) \cup [-4,18]$\\
H.$x \in (-\infty,-20] \cup [-4,18]$
\testStop
\kluczStart
A
\kluczStop



\zadStart{Zadanie z Wikieł Z 1.62 b) moja wersja nr 1110}

Rozwiązać nierówności $(x+20)(18-x)(x+5)\ge0$.
\zadStop
\rozwStart{Patryk Wirkus}{}
Miejsca zerowe naszego wielomianu to: $-20, 18, -5$.\\
Wielomian jest stopnia nieparzystego, ponadto znak współczynnika przy\linebreak najwyższej potędze x jest ujemny.\\ W związku z tym wykres wielomianu zaczyna się od lewej strony powyżej osi OX. A więc $$x \in (-\infty,-20) \cup (-5,18).$$
\rozwStop
\odpStart
$x \in (-\infty,-20) \cup (-5,18)$
\odpStop
\testStart
A.$x \in (-\infty,-20) \cup (-5,18)$\\
B.$x \in (-\infty,-20) \cup (-5,18]$\\
C.$x \in (-\infty,-20) \cup [-5,18)$\\
D.$x \in (-\infty,-20] \cup (-5,18)$\\
E.$x \in (-\infty,-20] \cup (-5,18]$\\
F.$x \in (-\infty,-20] \cup [-5,18)$\\
G.$x \in (-\infty,-20) \cup [-5,18]$\\
H.$x \in (-\infty,-20] \cup [-5,18]$
\testStop
\kluczStart
A
\kluczStop



\zadStart{Zadanie z Wikieł Z 1.62 b) moja wersja nr 1111}

Rozwiązać nierówności $(x+20)(18-x)(x+6)\ge0$.
\zadStop
\rozwStart{Patryk Wirkus}{}
Miejsca zerowe naszego wielomianu to: $-20, 18, -6$.\\
Wielomian jest stopnia nieparzystego, ponadto znak współczynnika przy\linebreak najwyższej potędze x jest ujemny.\\ W związku z tym wykres wielomianu zaczyna się od lewej strony powyżej osi OX. A więc $$x \in (-\infty,-20) \cup (-6,18).$$
\rozwStop
\odpStart
$x \in (-\infty,-20) \cup (-6,18)$
\odpStop
\testStart
A.$x \in (-\infty,-20) \cup (-6,18)$\\
B.$x \in (-\infty,-20) \cup (-6,18]$\\
C.$x \in (-\infty,-20) \cup [-6,18)$\\
D.$x \in (-\infty,-20] \cup (-6,18)$\\
E.$x \in (-\infty,-20] \cup (-6,18]$\\
F.$x \in (-\infty,-20] \cup [-6,18)$\\
G.$x \in (-\infty,-20) \cup [-6,18]$\\
H.$x \in (-\infty,-20] \cup [-6,18]$
\testStop
\kluczStart
A
\kluczStop



\zadStart{Zadanie z Wikieł Z 1.62 b) moja wersja nr 1112}

Rozwiązać nierówności $(x+20)(18-x)(x+7)\ge0$.
\zadStop
\rozwStart{Patryk Wirkus}{}
Miejsca zerowe naszego wielomianu to: $-20, 18, -7$.\\
Wielomian jest stopnia nieparzystego, ponadto znak współczynnika przy\linebreak najwyższej potędze x jest ujemny.\\ W związku z tym wykres wielomianu zaczyna się od lewej strony powyżej osi OX. A więc $$x \in (-\infty,-20) \cup (-7,18).$$
\rozwStop
\odpStart
$x \in (-\infty,-20) \cup (-7,18)$
\odpStop
\testStart
A.$x \in (-\infty,-20) \cup (-7,18)$\\
B.$x \in (-\infty,-20) \cup (-7,18]$\\
C.$x \in (-\infty,-20) \cup [-7,18)$\\
D.$x \in (-\infty,-20] \cup (-7,18)$\\
E.$x \in (-\infty,-20] \cup (-7,18]$\\
F.$x \in (-\infty,-20] \cup [-7,18)$\\
G.$x \in (-\infty,-20) \cup [-7,18]$\\
H.$x \in (-\infty,-20] \cup [-7,18]$
\testStop
\kluczStart
A
\kluczStop



\zadStart{Zadanie z Wikieł Z 1.62 b) moja wersja nr 1113}

Rozwiązać nierówności $(x+20)(18-x)(x+8)\ge0$.
\zadStop
\rozwStart{Patryk Wirkus}{}
Miejsca zerowe naszego wielomianu to: $-20, 18, -8$.\\
Wielomian jest stopnia nieparzystego, ponadto znak współczynnika przy\linebreak najwyższej potędze x jest ujemny.\\ W związku z tym wykres wielomianu zaczyna się od lewej strony powyżej osi OX. A więc $$x \in (-\infty,-20) \cup (-8,18).$$
\rozwStop
\odpStart
$x \in (-\infty,-20) \cup (-8,18)$
\odpStop
\testStart
A.$x \in (-\infty,-20) \cup (-8,18)$\\
B.$x \in (-\infty,-20) \cup (-8,18]$\\
C.$x \in (-\infty,-20) \cup [-8,18)$\\
D.$x \in (-\infty,-20] \cup (-8,18)$\\
E.$x \in (-\infty,-20] \cup (-8,18]$\\
F.$x \in (-\infty,-20] \cup [-8,18)$\\
G.$x \in (-\infty,-20) \cup [-8,18]$\\
H.$x \in (-\infty,-20] \cup [-8,18]$
\testStop
\kluczStart
A
\kluczStop



\zadStart{Zadanie z Wikieł Z 1.62 b) moja wersja nr 1114}

Rozwiązać nierówności $(x+20)(18-x)(x+9)\ge0$.
\zadStop
\rozwStart{Patryk Wirkus}{}
Miejsca zerowe naszego wielomianu to: $-20, 18, -9$.\\
Wielomian jest stopnia nieparzystego, ponadto znak współczynnika przy\linebreak najwyższej potędze x jest ujemny.\\ W związku z tym wykres wielomianu zaczyna się od lewej strony powyżej osi OX. A więc $$x \in (-\infty,-20) \cup (-9,18).$$
\rozwStop
\odpStart
$x \in (-\infty,-20) \cup (-9,18)$
\odpStop
\testStart
A.$x \in (-\infty,-20) \cup (-9,18)$\\
B.$x \in (-\infty,-20) \cup (-9,18]$\\
C.$x \in (-\infty,-20) \cup [-9,18)$\\
D.$x \in (-\infty,-20] \cup (-9,18)$\\
E.$x \in (-\infty,-20] \cup (-9,18]$\\
F.$x \in (-\infty,-20] \cup [-9,18)$\\
G.$x \in (-\infty,-20) \cup [-9,18]$\\
H.$x \in (-\infty,-20] \cup [-9,18]$
\testStop
\kluczStart
A
\kluczStop



\zadStart{Zadanie z Wikieł Z 1.62 b) moja wersja nr 1115}

Rozwiązać nierówności $(x+20)(18-x)(x+10)\ge0$.
\zadStop
\rozwStart{Patryk Wirkus}{}
Miejsca zerowe naszego wielomianu to: $-20, 18, -10$.\\
Wielomian jest stopnia nieparzystego, ponadto znak współczynnika przy\linebreak najwyższej potędze x jest ujemny.\\ W związku z tym wykres wielomianu zaczyna się od lewej strony powyżej osi OX. A więc $$x \in (-\infty,-20) \cup (-10,18).$$
\rozwStop
\odpStart
$x \in (-\infty,-20) \cup (-10,18)$
\odpStop
\testStart
A.$x \in (-\infty,-20) \cup (-10,18)$\\
B.$x \in (-\infty,-20) \cup (-10,18]$\\
C.$x \in (-\infty,-20) \cup [-10,18)$\\
D.$x \in (-\infty,-20] \cup (-10,18)$\\
E.$x \in (-\infty,-20] \cup (-10,18]$\\
F.$x \in (-\infty,-20] \cup [-10,18)$\\
G.$x \in (-\infty,-20) \cup [-10,18]$\\
H.$x \in (-\infty,-20] \cup [-10,18]$
\testStop
\kluczStart
A
\kluczStop



\zadStart{Zadanie z Wikieł Z 1.62 b) moja wersja nr 1116}

Rozwiązać nierówności $(x+20)(18-x)(x+11)\ge0$.
\zadStop
\rozwStart{Patryk Wirkus}{}
Miejsca zerowe naszego wielomianu to: $-20, 18, -11$.\\
Wielomian jest stopnia nieparzystego, ponadto znak współczynnika przy\linebreak najwyższej potędze x jest ujemny.\\ W związku z tym wykres wielomianu zaczyna się od lewej strony powyżej osi OX. A więc $$x \in (-\infty,-20) \cup (-11,18).$$
\rozwStop
\odpStart
$x \in (-\infty,-20) \cup (-11,18)$
\odpStop
\testStart
A.$x \in (-\infty,-20) \cup (-11,18)$\\
B.$x \in (-\infty,-20) \cup (-11,18]$\\
C.$x \in (-\infty,-20) \cup [-11,18)$\\
D.$x \in (-\infty,-20] \cup (-11,18)$\\
E.$x \in (-\infty,-20] \cup (-11,18]$\\
F.$x \in (-\infty,-20] \cup [-11,18)$\\
G.$x \in (-\infty,-20) \cup [-11,18]$\\
H.$x \in (-\infty,-20] \cup [-11,18]$
\testStop
\kluczStart
A
\kluczStop



\zadStart{Zadanie z Wikieł Z 1.62 b) moja wersja nr 1117}

Rozwiązać nierówności $(x+20)(18-x)(x+12)\ge0$.
\zadStop
\rozwStart{Patryk Wirkus}{}
Miejsca zerowe naszego wielomianu to: $-20, 18, -12$.\\
Wielomian jest stopnia nieparzystego, ponadto znak współczynnika przy\linebreak najwyższej potędze x jest ujemny.\\ W związku z tym wykres wielomianu zaczyna się od lewej strony powyżej osi OX. A więc $$x \in (-\infty,-20) \cup (-12,18).$$
\rozwStop
\odpStart
$x \in (-\infty,-20) \cup (-12,18)$
\odpStop
\testStart
A.$x \in (-\infty,-20) \cup (-12,18)$\\
B.$x \in (-\infty,-20) \cup (-12,18]$\\
C.$x \in (-\infty,-20) \cup [-12,18)$\\
D.$x \in (-\infty,-20] \cup (-12,18)$\\
E.$x \in (-\infty,-20] \cup (-12,18]$\\
F.$x \in (-\infty,-20] \cup [-12,18)$\\
G.$x \in (-\infty,-20) \cup [-12,18]$\\
H.$x \in (-\infty,-20] \cup [-12,18]$
\testStop
\kluczStart
A
\kluczStop



\zadStart{Zadanie z Wikieł Z 1.62 b) moja wersja nr 1118}

Rozwiązać nierówności $(x+20)(18-x)(x+13)\ge0$.
\zadStop
\rozwStart{Patryk Wirkus}{}
Miejsca zerowe naszego wielomianu to: $-20, 18, -13$.\\
Wielomian jest stopnia nieparzystego, ponadto znak współczynnika przy\linebreak najwyższej potędze x jest ujemny.\\ W związku z tym wykres wielomianu zaczyna się od lewej strony powyżej osi OX. A więc $$x \in (-\infty,-20) \cup (-13,18).$$
\rozwStop
\odpStart
$x \in (-\infty,-20) \cup (-13,18)$
\odpStop
\testStart
A.$x \in (-\infty,-20) \cup (-13,18)$\\
B.$x \in (-\infty,-20) \cup (-13,18]$\\
C.$x \in (-\infty,-20) \cup [-13,18)$\\
D.$x \in (-\infty,-20] \cup (-13,18)$\\
E.$x \in (-\infty,-20] \cup (-13,18]$\\
F.$x \in (-\infty,-20] \cup [-13,18)$\\
G.$x \in (-\infty,-20) \cup [-13,18]$\\
H.$x \in (-\infty,-20] \cup [-13,18]$
\testStop
\kluczStart
A
\kluczStop



\zadStart{Zadanie z Wikieł Z 1.62 b) moja wersja nr 1119}

Rozwiązać nierówności $(x+20)(18-x)(x+14)\ge0$.
\zadStop
\rozwStart{Patryk Wirkus}{}
Miejsca zerowe naszego wielomianu to: $-20, 18, -14$.\\
Wielomian jest stopnia nieparzystego, ponadto znak współczynnika przy\linebreak najwyższej potędze x jest ujemny.\\ W związku z tym wykres wielomianu zaczyna się od lewej strony powyżej osi OX. A więc $$x \in (-\infty,-20) \cup (-14,18).$$
\rozwStop
\odpStart
$x \in (-\infty,-20) \cup (-14,18)$
\odpStop
\testStart
A.$x \in (-\infty,-20) \cup (-14,18)$\\
B.$x \in (-\infty,-20) \cup (-14,18]$\\
C.$x \in (-\infty,-20) \cup [-14,18)$\\
D.$x \in (-\infty,-20] \cup (-14,18)$\\
E.$x \in (-\infty,-20] \cup (-14,18]$\\
F.$x \in (-\infty,-20] \cup [-14,18)$\\
G.$x \in (-\infty,-20) \cup [-14,18]$\\
H.$x \in (-\infty,-20] \cup [-14,18]$
\testStop
\kluczStart
A
\kluczStop



\zadStart{Zadanie z Wikieł Z 1.62 b) moja wersja nr 1120}

Rozwiązać nierówności $(x+20)(18-x)(x+15)\ge0$.
\zadStop
\rozwStart{Patryk Wirkus}{}
Miejsca zerowe naszego wielomianu to: $-20, 18, -15$.\\
Wielomian jest stopnia nieparzystego, ponadto znak współczynnika przy\linebreak najwyższej potędze x jest ujemny.\\ W związku z tym wykres wielomianu zaczyna się od lewej strony powyżej osi OX. A więc $$x \in (-\infty,-20) \cup (-15,18).$$
\rozwStop
\odpStart
$x \in (-\infty,-20) \cup (-15,18)$
\odpStop
\testStart
A.$x \in (-\infty,-20) \cup (-15,18)$\\
B.$x \in (-\infty,-20) \cup (-15,18]$\\
C.$x \in (-\infty,-20) \cup [-15,18)$\\
D.$x \in (-\infty,-20] \cup (-15,18)$\\
E.$x \in (-\infty,-20] \cup (-15,18]$\\
F.$x \in (-\infty,-20] \cup [-15,18)$\\
G.$x \in (-\infty,-20) \cup [-15,18]$\\
H.$x \in (-\infty,-20] \cup [-15,18]$
\testStop
\kluczStart
A
\kluczStop



\zadStart{Zadanie z Wikieł Z 1.62 b) moja wersja nr 1121}

Rozwiązać nierówności $(x+20)(18-x)(x+16)\ge0$.
\zadStop
\rozwStart{Patryk Wirkus}{}
Miejsca zerowe naszego wielomianu to: $-20, 18, -16$.\\
Wielomian jest stopnia nieparzystego, ponadto znak współczynnika przy\linebreak najwyższej potędze x jest ujemny.\\ W związku z tym wykres wielomianu zaczyna się od lewej strony powyżej osi OX. A więc $$x \in (-\infty,-20) \cup (-16,18).$$
\rozwStop
\odpStart
$x \in (-\infty,-20) \cup (-16,18)$
\odpStop
\testStart
A.$x \in (-\infty,-20) \cup (-16,18)$\\
B.$x \in (-\infty,-20) \cup (-16,18]$\\
C.$x \in (-\infty,-20) \cup [-16,18)$\\
D.$x \in (-\infty,-20] \cup (-16,18)$\\
E.$x \in (-\infty,-20] \cup (-16,18]$\\
F.$x \in (-\infty,-20] \cup [-16,18)$\\
G.$x \in (-\infty,-20) \cup [-16,18]$\\
H.$x \in (-\infty,-20] \cup [-16,18]$
\testStop
\kluczStart
A
\kluczStop



\zadStart{Zadanie z Wikieł Z 1.62 b) moja wersja nr 1122}

Rozwiązać nierówności $(x+20)(18-x)(x+17)\ge0$.
\zadStop
\rozwStart{Patryk Wirkus}{}
Miejsca zerowe naszego wielomianu to: $-20, 18, -17$.\\
Wielomian jest stopnia nieparzystego, ponadto znak współczynnika przy\linebreak najwyższej potędze x jest ujemny.\\ W związku z tym wykres wielomianu zaczyna się od lewej strony powyżej osi OX. A więc $$x \in (-\infty,-20) \cup (-17,18).$$
\rozwStop
\odpStart
$x \in (-\infty,-20) \cup (-17,18)$
\odpStop
\testStart
A.$x \in (-\infty,-20) \cup (-17,18)$\\
B.$x \in (-\infty,-20) \cup (-17,18]$\\
C.$x \in (-\infty,-20) \cup [-17,18)$\\
D.$x \in (-\infty,-20] \cup (-17,18)$\\
E.$x \in (-\infty,-20] \cup (-17,18]$\\
F.$x \in (-\infty,-20] \cup [-17,18)$\\
G.$x \in (-\infty,-20) \cup [-17,18]$\\
H.$x \in (-\infty,-20] \cup [-17,18]$
\testStop
\kluczStart
A
\kluczStop



\zadStart{Zadanie z Wikieł Z 1.62 b) moja wersja nr 1123}

Rozwiązać nierówności $(x+20)(19-x)(x+1)\ge0$.
\zadStop
\rozwStart{Patryk Wirkus}{}
Miejsca zerowe naszego wielomianu to: $-20, 19, -1$.\\
Wielomian jest stopnia nieparzystego, ponadto znak współczynnika przy\linebreak najwyższej potędze x jest ujemny.\\ W związku z tym wykres wielomianu zaczyna się od lewej strony powyżej osi OX. A więc $$x \in (-\infty,-20) \cup (-1,19).$$
\rozwStop
\odpStart
$x \in (-\infty,-20) \cup (-1,19)$
\odpStop
\testStart
A.$x \in (-\infty,-20) \cup (-1,19)$\\
B.$x \in (-\infty,-20) \cup (-1,19]$\\
C.$x \in (-\infty,-20) \cup [-1,19)$\\
D.$x \in (-\infty,-20] \cup (-1,19)$\\
E.$x \in (-\infty,-20] \cup (-1,19]$\\
F.$x \in (-\infty,-20] \cup [-1,19)$\\
G.$x \in (-\infty,-20) \cup [-1,19]$\\
H.$x \in (-\infty,-20] \cup [-1,19]$
\testStop
\kluczStart
A
\kluczStop



\zadStart{Zadanie z Wikieł Z 1.62 b) moja wersja nr 1124}

Rozwiązać nierówności $(x+20)(19-x)(x+2)\ge0$.
\zadStop
\rozwStart{Patryk Wirkus}{}
Miejsca zerowe naszego wielomianu to: $-20, 19, -2$.\\
Wielomian jest stopnia nieparzystego, ponadto znak współczynnika przy\linebreak najwyższej potędze x jest ujemny.\\ W związku z tym wykres wielomianu zaczyna się od lewej strony powyżej osi OX. A więc $$x \in (-\infty,-20) \cup (-2,19).$$
\rozwStop
\odpStart
$x \in (-\infty,-20) \cup (-2,19)$
\odpStop
\testStart
A.$x \in (-\infty,-20) \cup (-2,19)$\\
B.$x \in (-\infty,-20) \cup (-2,19]$\\
C.$x \in (-\infty,-20) \cup [-2,19)$\\
D.$x \in (-\infty,-20] \cup (-2,19)$\\
E.$x \in (-\infty,-20] \cup (-2,19]$\\
F.$x \in (-\infty,-20] \cup [-2,19)$\\
G.$x \in (-\infty,-20) \cup [-2,19]$\\
H.$x \in (-\infty,-20] \cup [-2,19]$
\testStop
\kluczStart
A
\kluczStop



\zadStart{Zadanie z Wikieł Z 1.62 b) moja wersja nr 1125}

Rozwiązać nierówności $(x+20)(19-x)(x+3)\ge0$.
\zadStop
\rozwStart{Patryk Wirkus}{}
Miejsca zerowe naszego wielomianu to: $-20, 19, -3$.\\
Wielomian jest stopnia nieparzystego, ponadto znak współczynnika przy\linebreak najwyższej potędze x jest ujemny.\\ W związku z tym wykres wielomianu zaczyna się od lewej strony powyżej osi OX. A więc $$x \in (-\infty,-20) \cup (-3,19).$$
\rozwStop
\odpStart
$x \in (-\infty,-20) \cup (-3,19)$
\odpStop
\testStart
A.$x \in (-\infty,-20) \cup (-3,19)$\\
B.$x \in (-\infty,-20) \cup (-3,19]$\\
C.$x \in (-\infty,-20) \cup [-3,19)$\\
D.$x \in (-\infty,-20] \cup (-3,19)$\\
E.$x \in (-\infty,-20] \cup (-3,19]$\\
F.$x \in (-\infty,-20] \cup [-3,19)$\\
G.$x \in (-\infty,-20) \cup [-3,19]$\\
H.$x \in (-\infty,-20] \cup [-3,19]$
\testStop
\kluczStart
A
\kluczStop



\zadStart{Zadanie z Wikieł Z 1.62 b) moja wersja nr 1126}

Rozwiązać nierówności $(x+20)(19-x)(x+4)\ge0$.
\zadStop
\rozwStart{Patryk Wirkus}{}
Miejsca zerowe naszego wielomianu to: $-20, 19, -4$.\\
Wielomian jest stopnia nieparzystego, ponadto znak współczynnika przy\linebreak najwyższej potędze x jest ujemny.\\ W związku z tym wykres wielomianu zaczyna się od lewej strony powyżej osi OX. A więc $$x \in (-\infty,-20) \cup (-4,19).$$
\rozwStop
\odpStart
$x \in (-\infty,-20) \cup (-4,19)$
\odpStop
\testStart
A.$x \in (-\infty,-20) \cup (-4,19)$\\
B.$x \in (-\infty,-20) \cup (-4,19]$\\
C.$x \in (-\infty,-20) \cup [-4,19)$\\
D.$x \in (-\infty,-20] \cup (-4,19)$\\
E.$x \in (-\infty,-20] \cup (-4,19]$\\
F.$x \in (-\infty,-20] \cup [-4,19)$\\
G.$x \in (-\infty,-20) \cup [-4,19]$\\
H.$x \in (-\infty,-20] \cup [-4,19]$
\testStop
\kluczStart
A
\kluczStop



\zadStart{Zadanie z Wikieł Z 1.62 b) moja wersja nr 1127}

Rozwiązać nierówności $(x+20)(19-x)(x+5)\ge0$.
\zadStop
\rozwStart{Patryk Wirkus}{}
Miejsca zerowe naszego wielomianu to: $-20, 19, -5$.\\
Wielomian jest stopnia nieparzystego, ponadto znak współczynnika przy\linebreak najwyższej potędze x jest ujemny.\\ W związku z tym wykres wielomianu zaczyna się od lewej strony powyżej osi OX. A więc $$x \in (-\infty,-20) \cup (-5,19).$$
\rozwStop
\odpStart
$x \in (-\infty,-20) \cup (-5,19)$
\odpStop
\testStart
A.$x \in (-\infty,-20) \cup (-5,19)$\\
B.$x \in (-\infty,-20) \cup (-5,19]$\\
C.$x \in (-\infty,-20) \cup [-5,19)$\\
D.$x \in (-\infty,-20] \cup (-5,19)$\\
E.$x \in (-\infty,-20] \cup (-5,19]$\\
F.$x \in (-\infty,-20] \cup [-5,19)$\\
G.$x \in (-\infty,-20) \cup [-5,19]$\\
H.$x \in (-\infty,-20] \cup [-5,19]$
\testStop
\kluczStart
A
\kluczStop



\zadStart{Zadanie z Wikieł Z 1.62 b) moja wersja nr 1128}

Rozwiązać nierówności $(x+20)(19-x)(x+6)\ge0$.
\zadStop
\rozwStart{Patryk Wirkus}{}
Miejsca zerowe naszego wielomianu to: $-20, 19, -6$.\\
Wielomian jest stopnia nieparzystego, ponadto znak współczynnika przy\linebreak najwyższej potędze x jest ujemny.\\ W związku z tym wykres wielomianu zaczyna się od lewej strony powyżej osi OX. A więc $$x \in (-\infty,-20) \cup (-6,19).$$
\rozwStop
\odpStart
$x \in (-\infty,-20) \cup (-6,19)$
\odpStop
\testStart
A.$x \in (-\infty,-20) \cup (-6,19)$\\
B.$x \in (-\infty,-20) \cup (-6,19]$\\
C.$x \in (-\infty,-20) \cup [-6,19)$\\
D.$x \in (-\infty,-20] \cup (-6,19)$\\
E.$x \in (-\infty,-20] \cup (-6,19]$\\
F.$x \in (-\infty,-20] \cup [-6,19)$\\
G.$x \in (-\infty,-20) \cup [-6,19]$\\
H.$x \in (-\infty,-20] \cup [-6,19]$
\testStop
\kluczStart
A
\kluczStop



\zadStart{Zadanie z Wikieł Z 1.62 b) moja wersja nr 1129}

Rozwiązać nierówności $(x+20)(19-x)(x+7)\ge0$.
\zadStop
\rozwStart{Patryk Wirkus}{}
Miejsca zerowe naszego wielomianu to: $-20, 19, -7$.\\
Wielomian jest stopnia nieparzystego, ponadto znak współczynnika przy\linebreak najwyższej potędze x jest ujemny.\\ W związku z tym wykres wielomianu zaczyna się od lewej strony powyżej osi OX. A więc $$x \in (-\infty,-20) \cup (-7,19).$$
\rozwStop
\odpStart
$x \in (-\infty,-20) \cup (-7,19)$
\odpStop
\testStart
A.$x \in (-\infty,-20) \cup (-7,19)$\\
B.$x \in (-\infty,-20) \cup (-7,19]$\\
C.$x \in (-\infty,-20) \cup [-7,19)$\\
D.$x \in (-\infty,-20] \cup (-7,19)$\\
E.$x \in (-\infty,-20] \cup (-7,19]$\\
F.$x \in (-\infty,-20] \cup [-7,19)$\\
G.$x \in (-\infty,-20) \cup [-7,19]$\\
H.$x \in (-\infty,-20] \cup [-7,19]$
\testStop
\kluczStart
A
\kluczStop



\zadStart{Zadanie z Wikieł Z 1.62 b) moja wersja nr 1130}

Rozwiązać nierówności $(x+20)(19-x)(x+8)\ge0$.
\zadStop
\rozwStart{Patryk Wirkus}{}
Miejsca zerowe naszego wielomianu to: $-20, 19, -8$.\\
Wielomian jest stopnia nieparzystego, ponadto znak współczynnika przy\linebreak najwyższej potędze x jest ujemny.\\ W związku z tym wykres wielomianu zaczyna się od lewej strony powyżej osi OX. A więc $$x \in (-\infty,-20) \cup (-8,19).$$
\rozwStop
\odpStart
$x \in (-\infty,-20) \cup (-8,19)$
\odpStop
\testStart
A.$x \in (-\infty,-20) \cup (-8,19)$\\
B.$x \in (-\infty,-20) \cup (-8,19]$\\
C.$x \in (-\infty,-20) \cup [-8,19)$\\
D.$x \in (-\infty,-20] \cup (-8,19)$\\
E.$x \in (-\infty,-20] \cup (-8,19]$\\
F.$x \in (-\infty,-20] \cup [-8,19)$\\
G.$x \in (-\infty,-20) \cup [-8,19]$\\
H.$x \in (-\infty,-20] \cup [-8,19]$
\testStop
\kluczStart
A
\kluczStop



\zadStart{Zadanie z Wikieł Z 1.62 b) moja wersja nr 1131}

Rozwiązać nierówności $(x+20)(19-x)(x+9)\ge0$.
\zadStop
\rozwStart{Patryk Wirkus}{}
Miejsca zerowe naszego wielomianu to: $-20, 19, -9$.\\
Wielomian jest stopnia nieparzystego, ponadto znak współczynnika przy\linebreak najwyższej potędze x jest ujemny.\\ W związku z tym wykres wielomianu zaczyna się od lewej strony powyżej osi OX. A więc $$x \in (-\infty,-20) \cup (-9,19).$$
\rozwStop
\odpStart
$x \in (-\infty,-20) \cup (-9,19)$
\odpStop
\testStart
A.$x \in (-\infty,-20) \cup (-9,19)$\\
B.$x \in (-\infty,-20) \cup (-9,19]$\\
C.$x \in (-\infty,-20) \cup [-9,19)$\\
D.$x \in (-\infty,-20] \cup (-9,19)$\\
E.$x \in (-\infty,-20] \cup (-9,19]$\\
F.$x \in (-\infty,-20] \cup [-9,19)$\\
G.$x \in (-\infty,-20) \cup [-9,19]$\\
H.$x \in (-\infty,-20] \cup [-9,19]$
\testStop
\kluczStart
A
\kluczStop



\zadStart{Zadanie z Wikieł Z 1.62 b) moja wersja nr 1132}

Rozwiązać nierówności $(x+20)(19-x)(x+10)\ge0$.
\zadStop
\rozwStart{Patryk Wirkus}{}
Miejsca zerowe naszego wielomianu to: $-20, 19, -10$.\\
Wielomian jest stopnia nieparzystego, ponadto znak współczynnika przy\linebreak najwyższej potędze x jest ujemny.\\ W związku z tym wykres wielomianu zaczyna się od lewej strony powyżej osi OX. A więc $$x \in (-\infty,-20) \cup (-10,19).$$
\rozwStop
\odpStart
$x \in (-\infty,-20) \cup (-10,19)$
\odpStop
\testStart
A.$x \in (-\infty,-20) \cup (-10,19)$\\
B.$x \in (-\infty,-20) \cup (-10,19]$\\
C.$x \in (-\infty,-20) \cup [-10,19)$\\
D.$x \in (-\infty,-20] \cup (-10,19)$\\
E.$x \in (-\infty,-20] \cup (-10,19]$\\
F.$x \in (-\infty,-20] \cup [-10,19)$\\
G.$x \in (-\infty,-20) \cup [-10,19]$\\
H.$x \in (-\infty,-20] \cup [-10,19]$
\testStop
\kluczStart
A
\kluczStop



\zadStart{Zadanie z Wikieł Z 1.62 b) moja wersja nr 1133}

Rozwiązać nierówności $(x+20)(19-x)(x+11)\ge0$.
\zadStop
\rozwStart{Patryk Wirkus}{}
Miejsca zerowe naszego wielomianu to: $-20, 19, -11$.\\
Wielomian jest stopnia nieparzystego, ponadto znak współczynnika przy\linebreak najwyższej potędze x jest ujemny.\\ W związku z tym wykres wielomianu zaczyna się od lewej strony powyżej osi OX. A więc $$x \in (-\infty,-20) \cup (-11,19).$$
\rozwStop
\odpStart
$x \in (-\infty,-20) \cup (-11,19)$
\odpStop
\testStart
A.$x \in (-\infty,-20) \cup (-11,19)$\\
B.$x \in (-\infty,-20) \cup (-11,19]$\\
C.$x \in (-\infty,-20) \cup [-11,19)$\\
D.$x \in (-\infty,-20] \cup (-11,19)$\\
E.$x \in (-\infty,-20] \cup (-11,19]$\\
F.$x \in (-\infty,-20] \cup [-11,19)$\\
G.$x \in (-\infty,-20) \cup [-11,19]$\\
H.$x \in (-\infty,-20] \cup [-11,19]$
\testStop
\kluczStart
A
\kluczStop



\zadStart{Zadanie z Wikieł Z 1.62 b) moja wersja nr 1134}

Rozwiązać nierówności $(x+20)(19-x)(x+12)\ge0$.
\zadStop
\rozwStart{Patryk Wirkus}{}
Miejsca zerowe naszego wielomianu to: $-20, 19, -12$.\\
Wielomian jest stopnia nieparzystego, ponadto znak współczynnika przy\linebreak najwyższej potędze x jest ujemny.\\ W związku z tym wykres wielomianu zaczyna się od lewej strony powyżej osi OX. A więc $$x \in (-\infty,-20) \cup (-12,19).$$
\rozwStop
\odpStart
$x \in (-\infty,-20) \cup (-12,19)$
\odpStop
\testStart
A.$x \in (-\infty,-20) \cup (-12,19)$\\
B.$x \in (-\infty,-20) \cup (-12,19]$\\
C.$x \in (-\infty,-20) \cup [-12,19)$\\
D.$x \in (-\infty,-20] \cup (-12,19)$\\
E.$x \in (-\infty,-20] \cup (-12,19]$\\
F.$x \in (-\infty,-20] \cup [-12,19)$\\
G.$x \in (-\infty,-20) \cup [-12,19]$\\
H.$x \in (-\infty,-20] \cup [-12,19]$
\testStop
\kluczStart
A
\kluczStop



\zadStart{Zadanie z Wikieł Z 1.62 b) moja wersja nr 1135}

Rozwiązać nierówności $(x+20)(19-x)(x+13)\ge0$.
\zadStop
\rozwStart{Patryk Wirkus}{}
Miejsca zerowe naszego wielomianu to: $-20, 19, -13$.\\
Wielomian jest stopnia nieparzystego, ponadto znak współczynnika przy\linebreak najwyższej potędze x jest ujemny.\\ W związku z tym wykres wielomianu zaczyna się od lewej strony powyżej osi OX. A więc $$x \in (-\infty,-20) \cup (-13,19).$$
\rozwStop
\odpStart
$x \in (-\infty,-20) \cup (-13,19)$
\odpStop
\testStart
A.$x \in (-\infty,-20) \cup (-13,19)$\\
B.$x \in (-\infty,-20) \cup (-13,19]$\\
C.$x \in (-\infty,-20) \cup [-13,19)$\\
D.$x \in (-\infty,-20] \cup (-13,19)$\\
E.$x \in (-\infty,-20] \cup (-13,19]$\\
F.$x \in (-\infty,-20] \cup [-13,19)$\\
G.$x \in (-\infty,-20) \cup [-13,19]$\\
H.$x \in (-\infty,-20] \cup [-13,19]$
\testStop
\kluczStart
A
\kluczStop



\zadStart{Zadanie z Wikieł Z 1.62 b) moja wersja nr 1136}

Rozwiązać nierówności $(x+20)(19-x)(x+14)\ge0$.
\zadStop
\rozwStart{Patryk Wirkus}{}
Miejsca zerowe naszego wielomianu to: $-20, 19, -14$.\\
Wielomian jest stopnia nieparzystego, ponadto znak współczynnika przy\linebreak najwyższej potędze x jest ujemny.\\ W związku z tym wykres wielomianu zaczyna się od lewej strony powyżej osi OX. A więc $$x \in (-\infty,-20) \cup (-14,19).$$
\rozwStop
\odpStart
$x \in (-\infty,-20) \cup (-14,19)$
\odpStop
\testStart
A.$x \in (-\infty,-20) \cup (-14,19)$\\
B.$x \in (-\infty,-20) \cup (-14,19]$\\
C.$x \in (-\infty,-20) \cup [-14,19)$\\
D.$x \in (-\infty,-20] \cup (-14,19)$\\
E.$x \in (-\infty,-20] \cup (-14,19]$\\
F.$x \in (-\infty,-20] \cup [-14,19)$\\
G.$x \in (-\infty,-20) \cup [-14,19]$\\
H.$x \in (-\infty,-20] \cup [-14,19]$
\testStop
\kluczStart
A
\kluczStop



\zadStart{Zadanie z Wikieł Z 1.62 b) moja wersja nr 1137}

Rozwiązać nierówności $(x+20)(19-x)(x+15)\ge0$.
\zadStop
\rozwStart{Patryk Wirkus}{}
Miejsca zerowe naszego wielomianu to: $-20, 19, -15$.\\
Wielomian jest stopnia nieparzystego, ponadto znak współczynnika przy\linebreak najwyższej potędze x jest ujemny.\\ W związku z tym wykres wielomianu zaczyna się od lewej strony powyżej osi OX. A więc $$x \in (-\infty,-20) \cup (-15,19).$$
\rozwStop
\odpStart
$x \in (-\infty,-20) \cup (-15,19)$
\odpStop
\testStart
A.$x \in (-\infty,-20) \cup (-15,19)$\\
B.$x \in (-\infty,-20) \cup (-15,19]$\\
C.$x \in (-\infty,-20) \cup [-15,19)$\\
D.$x \in (-\infty,-20] \cup (-15,19)$\\
E.$x \in (-\infty,-20] \cup (-15,19]$\\
F.$x \in (-\infty,-20] \cup [-15,19)$\\
G.$x \in (-\infty,-20) \cup [-15,19]$\\
H.$x \in (-\infty,-20] \cup [-15,19]$
\testStop
\kluczStart
A
\kluczStop



\zadStart{Zadanie z Wikieł Z 1.62 b) moja wersja nr 1138}

Rozwiązać nierówności $(x+20)(19-x)(x+16)\ge0$.
\zadStop
\rozwStart{Patryk Wirkus}{}
Miejsca zerowe naszego wielomianu to: $-20, 19, -16$.\\
Wielomian jest stopnia nieparzystego, ponadto znak współczynnika przy\linebreak najwyższej potędze x jest ujemny.\\ W związku z tym wykres wielomianu zaczyna się od lewej strony powyżej osi OX. A więc $$x \in (-\infty,-20) \cup (-16,19).$$
\rozwStop
\odpStart
$x \in (-\infty,-20) \cup (-16,19)$
\odpStop
\testStart
A.$x \in (-\infty,-20) \cup (-16,19)$\\
B.$x \in (-\infty,-20) \cup (-16,19]$\\
C.$x \in (-\infty,-20) \cup [-16,19)$\\
D.$x \in (-\infty,-20] \cup (-16,19)$\\
E.$x \in (-\infty,-20] \cup (-16,19]$\\
F.$x \in (-\infty,-20] \cup [-16,19)$\\
G.$x \in (-\infty,-20) \cup [-16,19]$\\
H.$x \in (-\infty,-20] \cup [-16,19]$
\testStop
\kluczStart
A
\kluczStop



\zadStart{Zadanie z Wikieł Z 1.62 b) moja wersja nr 1139}

Rozwiązać nierówności $(x+20)(19-x)(x+17)\ge0$.
\zadStop
\rozwStart{Patryk Wirkus}{}
Miejsca zerowe naszego wielomianu to: $-20, 19, -17$.\\
Wielomian jest stopnia nieparzystego, ponadto znak współczynnika przy\linebreak najwyższej potędze x jest ujemny.\\ W związku z tym wykres wielomianu zaczyna się od lewej strony powyżej osi OX. A więc $$x \in (-\infty,-20) \cup (-17,19).$$
\rozwStop
\odpStart
$x \in (-\infty,-20) \cup (-17,19)$
\odpStop
\testStart
A.$x \in (-\infty,-20) \cup (-17,19)$\\
B.$x \in (-\infty,-20) \cup (-17,19]$\\
C.$x \in (-\infty,-20) \cup [-17,19)$\\
D.$x \in (-\infty,-20] \cup (-17,19)$\\
E.$x \in (-\infty,-20] \cup (-17,19]$\\
F.$x \in (-\infty,-20] \cup [-17,19)$\\
G.$x \in (-\infty,-20) \cup [-17,19]$\\
H.$x \in (-\infty,-20] \cup [-17,19]$
\testStop
\kluczStart
A
\kluczStop



\zadStart{Zadanie z Wikieł Z 1.62 b) moja wersja nr 1140}

Rozwiązać nierówności $(x+20)(19-x)(x+18)\ge0$.
\zadStop
\rozwStart{Patryk Wirkus}{}
Miejsca zerowe naszego wielomianu to: $-20, 19, -18$.\\
Wielomian jest stopnia nieparzystego, ponadto znak współczynnika przy\linebreak najwyższej potędze x jest ujemny.\\ W związku z tym wykres wielomianu zaczyna się od lewej strony powyżej osi OX. A więc $$x \in (-\infty,-20) \cup (-18,19).$$
\rozwStop
\odpStart
$x \in (-\infty,-20) \cup (-18,19)$
\odpStop
\testStart
A.$x \in (-\infty,-20) \cup (-18,19)$\\
B.$x \in (-\infty,-20) \cup (-18,19]$\\
C.$x \in (-\infty,-20) \cup [-18,19)$\\
D.$x \in (-\infty,-20] \cup (-18,19)$\\
E.$x \in (-\infty,-20] \cup (-18,19]$\\
F.$x \in (-\infty,-20] \cup [-18,19)$\\
G.$x \in (-\infty,-20) \cup [-18,19]$\\
H.$x \in (-\infty,-20] \cup [-18,19]$
\testStop
\kluczStart
A
\kluczStop





\end{document}
