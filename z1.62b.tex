\documentclass[12pt, a4paper]{article}
\usepackage[utf8]{inputenc}
\usepackage{polski}
\usepackage{amsthm}  %pakiet do tworzenia twierdzeń itp.
\usepackage{amsmath} %pakiet do niektórych symboli matematycznych
\usepackage{amssymb} %pakiet do symboli mat., np. \nsubseteq
\usepackage{amsfonts}
\usepackage{graphicx} %obsługa plików graficznych z rozszerzeniem png, jpg
\theoremstyle{definition} %styl dla definicji
\newtheorem{zad}{} 
\title{Multizestaw zadań}
\author{Patryk Wirkus}
%\date{\today}
\date{}
\newcommand{\kategoria}[1]{\section{#1}}
\newcommand{\zadStart}[1]{\begin{zad}#1\newline}
\newcommand{\zadStop}{\end{zad}}
\newcommand{\rozwStart}[2]{\noindent \textbf{Rozwiązanie (autor #1 , recenzent #2): }\newline}
\newcommand{\rozwStop}{\newline}                                           
\newcommand{\odpStart}{\noindent \textbf{Odpowiedź:}\newline}
\newcommand{\odpStop}{\newline}
\newcommand{\testStart}{\noindent \textbf{Test:}\newline}
\newcommand{\testStop}{\newline}
\newcommand{\kluczStart}{\noindent \textbf{Test poprawna odpowiedź:}\newline}
\newcommand{\kluczStop}{\newline}
\newcommand{\wstawGrafike}[2]{\begin{figure}[h] \includegraphics[scale=#2] {#1} \end{figure}}

\begin{document}
\maketitle

\kategoria{Wikieł/1.62b}


\zadStart{Zadanie z Wikieł Z 1.62 b) moja wersja nr 1}

Rozwiązać nierówności $(x+3)(2-x)(x+1)\ge0$.
\zadStop
\rozwStart{Patryk Wirkus}{Laura Mieczkowska}
Miejsca zerowe naszego wielomianu to: $-3, 2, -1$.\\
Wielomian jest stopnia nieparzystego, ponadto znak współczynnika przy\linebreak najwyższej potędze x jest ujemny.\\ W związku z tym wykres wielomianu zaczyna się od lewej strony powyżej osi OX. A więc $$x \in (-\infty,-3) \cup (-1,2).$$
\rozwStop
\odpStart
$x \in (-\infty,-3) \cup (-1,2)$
\odpStop
\testStart
A.$x \in (-\infty,-3) \cup (-1,2)$\\
B.$x \in (-\infty,-3) \cup (-1,2]$\\
C.$x \in (-\infty,-3) \cup [-1,2)$\\
D.$x \in (-\infty,-3] \cup (-1,2)$\\
E.$x \in (-\infty,-3] \cup (-1,2]$\\
F.$x \in (-\infty,-3] \cup [-1,2)$\\
G.$x \in (-\infty,-3) \cup [-1,2]$\\
H.$x \in (-\infty,-3] \cup [-1,2]$
\testStop
\kluczStart
A
\kluczStop



\zadStart{Zadanie z Wikieł Z 1.62 b) moja wersja nr 2}

Rozwiązać nierówności $(x+4)(2-x)(x+1)\ge0$.
\zadStop
\rozwStart{Patryk Wirkus}{Laura Mieczkowska}
Miejsca zerowe naszego wielomianu to: $-4, 2, -1$.\\
Wielomian jest stopnia nieparzystego, ponadto znak współczynnika przy\linebreak najwyższej potędze x jest ujemny.\\ W związku z tym wykres wielomianu zaczyna się od lewej strony powyżej osi OX. A więc $$x \in (-\infty,-4) \cup (-1,2).$$
\rozwStop
\odpStart
$x \in (-\infty,-4) \cup (-1,2)$
\odpStop
\testStart
A.$x \in (-\infty,-4) \cup (-1,2)$\\
B.$x \in (-\infty,-4) \cup (-1,2]$\\
C.$x \in (-\infty,-4) \cup [-1,2)$\\
D.$x \in (-\infty,-4] \cup (-1,2)$\\
E.$x \in (-\infty,-4] \cup (-1,2]$\\
F.$x \in (-\infty,-4] \cup [-1,2)$\\
G.$x \in (-\infty,-4) \cup [-1,2]$\\
H.$x \in (-\infty,-4] \cup [-1,2]$
\testStop
\kluczStart
A
\kluczStop



\zadStart{Zadanie z Wikieł Z 1.62 b) moja wersja nr 3}

Rozwiązać nierówności $(x+5)(2-x)(x+1)\ge0$.
\zadStop
\rozwStart{Patryk Wirkus}{Laura Mieczkowska}
Miejsca zerowe naszego wielomianu to: $-5, 2, -1$.\\
Wielomian jest stopnia nieparzystego, ponadto znak współczynnika przy\linebreak najwyższej potędze x jest ujemny.\\ W związku z tym wykres wielomianu zaczyna się od lewej strony powyżej osi OX. A więc $$x \in (-\infty,-5) \cup (-1,2).$$
\rozwStop
\odpStart
$x \in (-\infty,-5) \cup (-1,2)$
\odpStop
\testStart
A.$x \in (-\infty,-5) \cup (-1,2)$\\
B.$x \in (-\infty,-5) \cup (-1,2]$\\
C.$x \in (-\infty,-5) \cup [-1,2)$\\
D.$x \in (-\infty,-5] \cup (-1,2)$\\
E.$x \in (-\infty,-5] \cup (-1,2]$\\
F.$x \in (-\infty,-5] \cup [-1,2)$\\
G.$x \in (-\infty,-5) \cup [-1,2]$\\
H.$x \in (-\infty,-5] \cup [-1,2]$
\testStop
\kluczStart
A
\kluczStop



\zadStart{Zadanie z Wikieł Z 1.62 b) moja wersja nr 4}

Rozwiązać nierówności $(x+6)(2-x)(x+1)\ge0$.
\zadStop
\rozwStart{Patryk Wirkus}{Laura Mieczkowska}
Miejsca zerowe naszego wielomianu to: $-6, 2, -1$.\\
Wielomian jest stopnia nieparzystego, ponadto znak współczynnika przy\linebreak najwyższej potędze x jest ujemny.\\ W związku z tym wykres wielomianu zaczyna się od lewej strony powyżej osi OX. A więc $$x \in (-\infty,-6) \cup (-1,2).$$
\rozwStop
\odpStart
$x \in (-\infty,-6) \cup (-1,2)$
\odpStop
\testStart
A.$x \in (-\infty,-6) \cup (-1,2)$\\
B.$x \in (-\infty,-6) \cup (-1,2]$\\
C.$x \in (-\infty,-6) \cup [-1,2)$\\
D.$x \in (-\infty,-6] \cup (-1,2)$\\
E.$x \in (-\infty,-6] \cup (-1,2]$\\
F.$x \in (-\infty,-6] \cup [-1,2)$\\
G.$x \in (-\infty,-6) \cup [-1,2]$\\
H.$x \in (-\infty,-6] \cup [-1,2]$
\testStop
\kluczStart
A
\kluczStop



\zadStart{Zadanie z Wikieł Z 1.62 b) moja wersja nr 5}

Rozwiązać nierówności $(x+7)(2-x)(x+1)\ge0$.
\zadStop
\rozwStart{Patryk Wirkus}{Laura Mieczkowska}
Miejsca zerowe naszego wielomianu to: $-7, 2, -1$.\\
Wielomian jest stopnia nieparzystego, ponadto znak współczynnika przy\linebreak najwyższej potędze x jest ujemny.\\ W związku z tym wykres wielomianu zaczyna się od lewej strony powyżej osi OX. A więc $$x \in (-\infty,-7) \cup (-1,2).$$
\rozwStop
\odpStart
$x \in (-\infty,-7) \cup (-1,2)$
\odpStop
\testStart
A.$x \in (-\infty,-7) \cup (-1,2)$\\
B.$x \in (-\infty,-7) \cup (-1,2]$\\
C.$x \in (-\infty,-7) \cup [-1,2)$\\
D.$x \in (-\infty,-7] \cup (-1,2)$\\
E.$x \in (-\infty,-7] \cup (-1,2]$\\
F.$x \in (-\infty,-7] \cup [-1,2)$\\
G.$x \in (-\infty,-7) \cup [-1,2]$\\
H.$x \in (-\infty,-7] \cup [-1,2]$
\testStop
\kluczStart
A
\kluczStop



\zadStart{Zadanie z Wikieł Z 1.62 b) moja wersja nr 6}

Rozwiązać nierówności $(x+8)(2-x)(x+1)\ge0$.
\zadStop
\rozwStart{Patryk Wirkus}{Laura Mieczkowska}
Miejsca zerowe naszego wielomianu to: $-8, 2, -1$.\\
Wielomian jest stopnia nieparzystego, ponadto znak współczynnika przy\linebreak najwyższej potędze x jest ujemny.\\ W związku z tym wykres wielomianu zaczyna się od lewej strony powyżej osi OX. A więc $$x \in (-\infty,-8) \cup (-1,2).$$
\rozwStop
\odpStart
$x \in (-\infty,-8) \cup (-1,2)$
\odpStop
\testStart
A.$x \in (-\infty,-8) \cup (-1,2)$\\
B.$x \in (-\infty,-8) \cup (-1,2]$\\
C.$x \in (-\infty,-8) \cup [-1,2)$\\
D.$x \in (-\infty,-8] \cup (-1,2)$\\
E.$x \in (-\infty,-8] \cup (-1,2]$\\
F.$x \in (-\infty,-8] \cup [-1,2)$\\
G.$x \in (-\infty,-8) \cup [-1,2]$\\
H.$x \in (-\infty,-8] \cup [-1,2]$
\testStop
\kluczStart
A
\kluczStop



\zadStart{Zadanie z Wikieł Z 1.62 b) moja wersja nr 7}

Rozwiązać nierówności $(x+9)(2-x)(x+1)\ge0$.
\zadStop
\rozwStart{Patryk Wirkus}{Laura Mieczkowska}
Miejsca zerowe naszego wielomianu to: $-9, 2, -1$.\\
Wielomian jest stopnia nieparzystego, ponadto znak współczynnika przy\linebreak najwyższej potędze x jest ujemny.\\ W związku z tym wykres wielomianu zaczyna się od lewej strony powyżej osi OX. A więc $$x \in (-\infty,-9) \cup (-1,2).$$
\rozwStop
\odpStart
$x \in (-\infty,-9) \cup (-1,2)$
\odpStop
\testStart
A.$x \in (-\infty,-9) \cup (-1,2)$\\
B.$x \in (-\infty,-9) \cup (-1,2]$\\
C.$x \in (-\infty,-9) \cup [-1,2)$\\
D.$x \in (-\infty,-9] \cup (-1,2)$\\
E.$x \in (-\infty,-9] \cup (-1,2]$\\
F.$x \in (-\infty,-9] \cup [-1,2)$\\
G.$x \in (-\infty,-9) \cup [-1,2]$\\
H.$x \in (-\infty,-9] \cup [-1,2]$
\testStop
\kluczStart
A
\kluczStop



\zadStart{Zadanie z Wikieł Z 1.62 b) moja wersja nr 8}

Rozwiązać nierówności $(x+10)(2-x)(x+1)\ge0$.
\zadStop
\rozwStart{Patryk Wirkus}{Laura Mieczkowska}
Miejsca zerowe naszego wielomianu to: $-10, 2, -1$.\\
Wielomian jest stopnia nieparzystego, ponadto znak współczynnika przy\linebreak najwyższej potędze x jest ujemny.\\ W związku z tym wykres wielomianu zaczyna się od lewej strony powyżej osi OX. A więc $$x \in (-\infty,-10) \cup (-1,2).$$
\rozwStop
\odpStart
$x \in (-\infty,-10) \cup (-1,2)$
\odpStop
\testStart
A.$x \in (-\infty,-10) \cup (-1,2)$\\
B.$x \in (-\infty,-10) \cup (-1,2]$\\
C.$x \in (-\infty,-10) \cup [-1,2)$\\
D.$x \in (-\infty,-10] \cup (-1,2)$\\
E.$x \in (-\infty,-10] \cup (-1,2]$\\
F.$x \in (-\infty,-10] \cup [-1,2)$\\
G.$x \in (-\infty,-10) \cup [-1,2]$\\
H.$x \in (-\infty,-10] \cup [-1,2]$
\testStop
\kluczStart
A
\kluczStop



\zadStart{Zadanie z Wikieł Z 1.62 b) moja wersja nr 9}

Rozwiązać nierówności $(x+11)(2-x)(x+1)\ge0$.
\zadStop
\rozwStart{Patryk Wirkus}{Laura Mieczkowska}
Miejsca zerowe naszego wielomianu to: $-11, 2, -1$.\\
Wielomian jest stopnia nieparzystego, ponadto znak współczynnika przy\linebreak najwyższej potędze x jest ujemny.\\ W związku z tym wykres wielomianu zaczyna się od lewej strony powyżej osi OX. A więc $$x \in (-\infty,-11) \cup (-1,2).$$
\rozwStop
\odpStart
$x \in (-\infty,-11) \cup (-1,2)$
\odpStop
\testStart
A.$x \in (-\infty,-11) \cup (-1,2)$\\
B.$x \in (-\infty,-11) \cup (-1,2]$\\
C.$x \in (-\infty,-11) \cup [-1,2)$\\
D.$x \in (-\infty,-11] \cup (-1,2)$\\
E.$x \in (-\infty,-11] \cup (-1,2]$\\
F.$x \in (-\infty,-11] \cup [-1,2)$\\
G.$x \in (-\infty,-11) \cup [-1,2]$\\
H.$x \in (-\infty,-11] \cup [-1,2]$
\testStop
\kluczStart
A
\kluczStop



\zadStart{Zadanie z Wikieł Z 1.62 b) moja wersja nr 10}

Rozwiązać nierówności $(x+12)(2-x)(x+1)\ge0$.
\zadStop
\rozwStart{Patryk Wirkus}{Laura Mieczkowska}
Miejsca zerowe naszego wielomianu to: $-12, 2, -1$.\\
Wielomian jest stopnia nieparzystego, ponadto znak współczynnika przy\linebreak najwyższej potędze x jest ujemny.\\ W związku z tym wykres wielomianu zaczyna się od lewej strony powyżej osi OX. A więc $$x \in (-\infty,-12) \cup (-1,2).$$
\rozwStop
\odpStart
$x \in (-\infty,-12) \cup (-1,2)$
\odpStop
\testStart
A.$x \in (-\infty,-12) \cup (-1,2)$\\
B.$x \in (-\infty,-12) \cup (-1,2]$\\
C.$x \in (-\infty,-12) \cup [-1,2)$\\
D.$x \in (-\infty,-12] \cup (-1,2)$\\
E.$x \in (-\infty,-12] \cup (-1,2]$\\
F.$x \in (-\infty,-12] \cup [-1,2)$\\
G.$x \in (-\infty,-12) \cup [-1,2]$\\
H.$x \in (-\infty,-12] \cup [-1,2]$
\testStop
\kluczStart
A
\kluczStop



\zadStart{Zadanie z Wikieł Z 1.62 b) moja wersja nr 11}

Rozwiązać nierówności $(x+13)(2-x)(x+1)\ge0$.
\zadStop
\rozwStart{Patryk Wirkus}{Laura Mieczkowska}
Miejsca zerowe naszego wielomianu to: $-13, 2, -1$.\\
Wielomian jest stopnia nieparzystego, ponadto znak współczynnika przy\linebreak najwyższej potędze x jest ujemny.\\ W związku z tym wykres wielomianu zaczyna się od lewej strony powyżej osi OX. A więc $$x \in (-\infty,-13) \cup (-1,2).$$
\rozwStop
\odpStart
$x \in (-\infty,-13) \cup (-1,2)$
\odpStop
\testStart
A.$x \in (-\infty,-13) \cup (-1,2)$\\
B.$x \in (-\infty,-13) \cup (-1,2]$\\
C.$x \in (-\infty,-13) \cup [-1,2)$\\
D.$x \in (-\infty,-13] \cup (-1,2)$\\
E.$x \in (-\infty,-13] \cup (-1,2]$\\
F.$x \in (-\infty,-13] \cup [-1,2)$\\
G.$x \in (-\infty,-13) \cup [-1,2]$\\
H.$x \in (-\infty,-13] \cup [-1,2]$
\testStop
\kluczStart
A
\kluczStop



\zadStart{Zadanie z Wikieł Z 1.62 b) moja wersja nr 12}

Rozwiązać nierówności $(x+14)(2-x)(x+1)\ge0$.
\zadStop
\rozwStart{Patryk Wirkus}{Laura Mieczkowska}
Miejsca zerowe naszego wielomianu to: $-14, 2, -1$.\\
Wielomian jest stopnia nieparzystego, ponadto znak współczynnika przy\linebreak najwyższej potędze x jest ujemny.\\ W związku z tym wykres wielomianu zaczyna się od lewej strony powyżej osi OX. A więc $$x \in (-\infty,-14) \cup (-1,2).$$
\rozwStop
\odpStart
$x \in (-\infty,-14) \cup (-1,2)$
\odpStop
\testStart
A.$x \in (-\infty,-14) \cup (-1,2)$\\
B.$x \in (-\infty,-14) \cup (-1,2]$\\
C.$x \in (-\infty,-14) \cup [-1,2)$\\
D.$x \in (-\infty,-14] \cup (-1,2)$\\
E.$x \in (-\infty,-14] \cup (-1,2]$\\
F.$x \in (-\infty,-14] \cup [-1,2)$\\
G.$x \in (-\infty,-14) \cup [-1,2]$\\
H.$x \in (-\infty,-14] \cup [-1,2]$
\testStop
\kluczStart
A
\kluczStop



\zadStart{Zadanie z Wikieł Z 1.62 b) moja wersja nr 13}

Rozwiązać nierówności $(x+15)(2-x)(x+1)\ge0$.
\zadStop
\rozwStart{Patryk Wirkus}{Laura Mieczkowska}
Miejsca zerowe naszego wielomianu to: $-15, 2, -1$.\\
Wielomian jest stopnia nieparzystego, ponadto znak współczynnika przy\linebreak najwyższej potędze x jest ujemny.\\ W związku z tym wykres wielomianu zaczyna się od lewej strony powyżej osi OX. A więc $$x \in (-\infty,-15) \cup (-1,2).$$
\rozwStop
\odpStart
$x \in (-\infty,-15) \cup (-1,2)$
\odpStop
\testStart
A.$x \in (-\infty,-15) \cup (-1,2)$\\
B.$x \in (-\infty,-15) \cup (-1,2]$\\
C.$x \in (-\infty,-15) \cup [-1,2)$\\
D.$x \in (-\infty,-15] \cup (-1,2)$\\
E.$x \in (-\infty,-15] \cup (-1,2]$\\
F.$x \in (-\infty,-15] \cup [-1,2)$\\
G.$x \in (-\infty,-15) \cup [-1,2]$\\
H.$x \in (-\infty,-15] \cup [-1,2]$
\testStop
\kluczStart
A
\kluczStop



\zadStart{Zadanie z Wikieł Z 1.62 b) moja wersja nr 14}

Rozwiązać nierówności $(x+4)(3-x)(x+1)\ge0$.
\zadStop
\rozwStart{Patryk Wirkus}{Laura Mieczkowska}
Miejsca zerowe naszego wielomianu to: $-4, 3, -1$.\\
Wielomian jest stopnia nieparzystego, ponadto znak współczynnika przy\linebreak najwyższej potędze x jest ujemny.\\ W związku z tym wykres wielomianu zaczyna się od lewej strony powyżej osi OX. A więc $$x \in (-\infty,-4) \cup (-1,3).$$
\rozwStop
\odpStart
$x \in (-\infty,-4) \cup (-1,3)$
\odpStop
\testStart
A.$x \in (-\infty,-4) \cup (-1,3)$\\
B.$x \in (-\infty,-4) \cup (-1,3]$\\
C.$x \in (-\infty,-4) \cup [-1,3)$\\
D.$x \in (-\infty,-4] \cup (-1,3)$\\
E.$x \in (-\infty,-4] \cup (-1,3]$\\
F.$x \in (-\infty,-4] \cup [-1,3)$\\
G.$x \in (-\infty,-4) \cup [-1,3]$\\
H.$x \in (-\infty,-4] \cup [-1,3]$
\testStop
\kluczStart
A
\kluczStop



\zadStart{Zadanie z Wikieł Z 1.62 b) moja wersja nr 15}

Rozwiązać nierówności $(x+5)(3-x)(x+1)\ge0$.
\zadStop
\rozwStart{Patryk Wirkus}{Laura Mieczkowska}
Miejsca zerowe naszego wielomianu to: $-5, 3, -1$.\\
Wielomian jest stopnia nieparzystego, ponadto znak współczynnika przy\linebreak najwyższej potędze x jest ujemny.\\ W związku z tym wykres wielomianu zaczyna się od lewej strony powyżej osi OX. A więc $$x \in (-\infty,-5) \cup (-1,3).$$
\rozwStop
\odpStart
$x \in (-\infty,-5) \cup (-1,3)$
\odpStop
\testStart
A.$x \in (-\infty,-5) \cup (-1,3)$\\
B.$x \in (-\infty,-5) \cup (-1,3]$\\
C.$x \in (-\infty,-5) \cup [-1,3)$\\
D.$x \in (-\infty,-5] \cup (-1,3)$\\
E.$x \in (-\infty,-5] \cup (-1,3]$\\
F.$x \in (-\infty,-5] \cup [-1,3)$\\
G.$x \in (-\infty,-5) \cup [-1,3]$\\
H.$x \in (-\infty,-5] \cup [-1,3]$
\testStop
\kluczStart
A
\kluczStop



\zadStart{Zadanie z Wikieł Z 1.62 b) moja wersja nr 16}

Rozwiązać nierówności $(x+6)(3-x)(x+1)\ge0$.
\zadStop
\rozwStart{Patryk Wirkus}{Laura Mieczkowska}
Miejsca zerowe naszego wielomianu to: $-6, 3, -1$.\\
Wielomian jest stopnia nieparzystego, ponadto znak współczynnika przy\linebreak najwyższej potędze x jest ujemny.\\ W związku z tym wykres wielomianu zaczyna się od lewej strony powyżej osi OX. A więc $$x \in (-\infty,-6) \cup (-1,3).$$
\rozwStop
\odpStart
$x \in (-\infty,-6) \cup (-1,3)$
\odpStop
\testStart
A.$x \in (-\infty,-6) \cup (-1,3)$\\
B.$x \in (-\infty,-6) \cup (-1,3]$\\
C.$x \in (-\infty,-6) \cup [-1,3)$\\
D.$x \in (-\infty,-6] \cup (-1,3)$\\
E.$x \in (-\infty,-6] \cup (-1,3]$\\
F.$x \in (-\infty,-6] \cup [-1,3)$\\
G.$x \in (-\infty,-6) \cup [-1,3]$\\
H.$x \in (-\infty,-6] \cup [-1,3]$
\testStop
\kluczStart
A
\kluczStop



\zadStart{Zadanie z Wikieł Z 1.62 b) moja wersja nr 17}

Rozwiązać nierówności $(x+7)(3-x)(x+1)\ge0$.
\zadStop
\rozwStart{Patryk Wirkus}{Laura Mieczkowska}
Miejsca zerowe naszego wielomianu to: $-7, 3, -1$.\\
Wielomian jest stopnia nieparzystego, ponadto znak współczynnika przy\linebreak najwyższej potędze x jest ujemny.\\ W związku z tym wykres wielomianu zaczyna się od lewej strony powyżej osi OX. A więc $$x \in (-\infty,-7) \cup (-1,3).$$
\rozwStop
\odpStart
$x \in (-\infty,-7) \cup (-1,3)$
\odpStop
\testStart
A.$x \in (-\infty,-7) \cup (-1,3)$\\
B.$x \in (-\infty,-7) \cup (-1,3]$\\
C.$x \in (-\infty,-7) \cup [-1,3)$\\
D.$x \in (-\infty,-7] \cup (-1,3)$\\
E.$x \in (-\infty,-7] \cup (-1,3]$\\
F.$x \in (-\infty,-7] \cup [-1,3)$\\
G.$x \in (-\infty,-7) \cup [-1,3]$\\
H.$x \in (-\infty,-7] \cup [-1,3]$
\testStop
\kluczStart
A
\kluczStop



\zadStart{Zadanie z Wikieł Z 1.62 b) moja wersja nr 18}

Rozwiązać nierówności $(x+8)(3-x)(x+1)\ge0$.
\zadStop
\rozwStart{Patryk Wirkus}{Laura Mieczkowska}
Miejsca zerowe naszego wielomianu to: $-8, 3, -1$.\\
Wielomian jest stopnia nieparzystego, ponadto znak współczynnika przy\linebreak najwyższej potędze x jest ujemny.\\ W związku z tym wykres wielomianu zaczyna się od lewej strony powyżej osi OX. A więc $$x \in (-\infty,-8) \cup (-1,3).$$
\rozwStop
\odpStart
$x \in (-\infty,-8) \cup (-1,3)$
\odpStop
\testStart
A.$x \in (-\infty,-8) \cup (-1,3)$\\
B.$x \in (-\infty,-8) \cup (-1,3]$\\
C.$x \in (-\infty,-8) \cup [-1,3)$\\
D.$x \in (-\infty,-8] \cup (-1,3)$\\
E.$x \in (-\infty,-8] \cup (-1,3]$\\
F.$x \in (-\infty,-8] \cup [-1,3)$\\
G.$x \in (-\infty,-8) \cup [-1,3]$\\
H.$x \in (-\infty,-8] \cup [-1,3]$
\testStop
\kluczStart
A
\kluczStop



\zadStart{Zadanie z Wikieł Z 1.62 b) moja wersja nr 19}

Rozwiązać nierówności $(x+9)(3-x)(x+1)\ge0$.
\zadStop
\rozwStart{Patryk Wirkus}{Laura Mieczkowska}
Miejsca zerowe naszego wielomianu to: $-9, 3, -1$.\\
Wielomian jest stopnia nieparzystego, ponadto znak współczynnika przy\linebreak najwyższej potędze x jest ujemny.\\ W związku z tym wykres wielomianu zaczyna się od lewej strony powyżej osi OX. A więc $$x \in (-\infty,-9) \cup (-1,3).$$
\rozwStop
\odpStart
$x \in (-\infty,-9) \cup (-1,3)$
\odpStop
\testStart
A.$x \in (-\infty,-9) \cup (-1,3)$\\
B.$x \in (-\infty,-9) \cup (-1,3]$\\
C.$x \in (-\infty,-9) \cup [-1,3)$\\
D.$x \in (-\infty,-9] \cup (-1,3)$\\
E.$x \in (-\infty,-9] \cup (-1,3]$\\
F.$x \in (-\infty,-9] \cup [-1,3)$\\
G.$x \in (-\infty,-9) \cup [-1,3]$\\
H.$x \in (-\infty,-9] \cup [-1,3]$
\testStop
\kluczStart
A
\kluczStop



\zadStart{Zadanie z Wikieł Z 1.62 b) moja wersja nr 20}

Rozwiązać nierówności $(x+10)(3-x)(x+1)\ge0$.
\zadStop
\rozwStart{Patryk Wirkus}{Laura Mieczkowska}
Miejsca zerowe naszego wielomianu to: $-10, 3, -1$.\\
Wielomian jest stopnia nieparzystego, ponadto znak współczynnika przy\linebreak najwyższej potędze x jest ujemny.\\ W związku z tym wykres wielomianu zaczyna się od lewej strony powyżej osi OX. A więc $$x \in (-\infty,-10) \cup (-1,3).$$
\rozwStop
\odpStart
$x \in (-\infty,-10) \cup (-1,3)$
\odpStop
\testStart
A.$x \in (-\infty,-10) \cup (-1,3)$\\
B.$x \in (-\infty,-10) \cup (-1,3]$\\
C.$x \in (-\infty,-10) \cup [-1,3)$\\
D.$x \in (-\infty,-10] \cup (-1,3)$\\
E.$x \in (-\infty,-10] \cup (-1,3]$\\
F.$x \in (-\infty,-10] \cup [-1,3)$\\
G.$x \in (-\infty,-10) \cup [-1,3]$\\
H.$x \in (-\infty,-10] \cup [-1,3]$
\testStop
\kluczStart
A
\kluczStop



\zadStart{Zadanie z Wikieł Z 1.62 b) moja wersja nr 21}

Rozwiązać nierówności $(x+11)(3-x)(x+1)\ge0$.
\zadStop
\rozwStart{Patryk Wirkus}{Laura Mieczkowska}
Miejsca zerowe naszego wielomianu to: $-11, 3, -1$.\\
Wielomian jest stopnia nieparzystego, ponadto znak współczynnika przy\linebreak najwyższej potędze x jest ujemny.\\ W związku z tym wykres wielomianu zaczyna się od lewej strony powyżej osi OX. A więc $$x \in (-\infty,-11) \cup (-1,3).$$
\rozwStop
\odpStart
$x \in (-\infty,-11) \cup (-1,3)$
\odpStop
\testStart
A.$x \in (-\infty,-11) \cup (-1,3)$\\
B.$x \in (-\infty,-11) \cup (-1,3]$\\
C.$x \in (-\infty,-11) \cup [-1,3)$\\
D.$x \in (-\infty,-11] \cup (-1,3)$\\
E.$x \in (-\infty,-11] \cup (-1,3]$\\
F.$x \in (-\infty,-11] \cup [-1,3)$\\
G.$x \in (-\infty,-11) \cup [-1,3]$\\
H.$x \in (-\infty,-11] \cup [-1,3]$
\testStop
\kluczStart
A
\kluczStop



\zadStart{Zadanie z Wikieł Z 1.62 b) moja wersja nr 22}

Rozwiązać nierówności $(x+12)(3-x)(x+1)\ge0$.
\zadStop
\rozwStart{Patryk Wirkus}{Laura Mieczkowska}
Miejsca zerowe naszego wielomianu to: $-12, 3, -1$.\\
Wielomian jest stopnia nieparzystego, ponadto znak współczynnika przy\linebreak najwyższej potędze x jest ujemny.\\ W związku z tym wykres wielomianu zaczyna się od lewej strony powyżej osi OX. A więc $$x \in (-\infty,-12) \cup (-1,3).$$
\rozwStop
\odpStart
$x \in (-\infty,-12) \cup (-1,3)$
\odpStop
\testStart
A.$x \in (-\infty,-12) \cup (-1,3)$\\
B.$x \in (-\infty,-12) \cup (-1,3]$\\
C.$x \in (-\infty,-12) \cup [-1,3)$\\
D.$x \in (-\infty,-12] \cup (-1,3)$\\
E.$x \in (-\infty,-12] \cup (-1,3]$\\
F.$x \in (-\infty,-12] \cup [-1,3)$\\
G.$x \in (-\infty,-12) \cup [-1,3]$\\
H.$x \in (-\infty,-12] \cup [-1,3]$
\testStop
\kluczStart
A
\kluczStop



\zadStart{Zadanie z Wikieł Z 1.62 b) moja wersja nr 23}

Rozwiązać nierówności $(x+13)(3-x)(x+1)\ge0$.
\zadStop
\rozwStart{Patryk Wirkus}{Laura Mieczkowska}
Miejsca zerowe naszego wielomianu to: $-13, 3, -1$.\\
Wielomian jest stopnia nieparzystego, ponadto znak współczynnika przy\linebreak najwyższej potędze x jest ujemny.\\ W związku z tym wykres wielomianu zaczyna się od lewej strony powyżej osi OX. A więc $$x \in (-\infty,-13) \cup (-1,3).$$
\rozwStop
\odpStart
$x \in (-\infty,-13) \cup (-1,3)$
\odpStop
\testStart
A.$x \in (-\infty,-13) \cup (-1,3)$\\
B.$x \in (-\infty,-13) \cup (-1,3]$\\
C.$x \in (-\infty,-13) \cup [-1,3)$\\
D.$x \in (-\infty,-13] \cup (-1,3)$\\
E.$x \in (-\infty,-13] \cup (-1,3]$\\
F.$x \in (-\infty,-13] \cup [-1,3)$\\
G.$x \in (-\infty,-13) \cup [-1,3]$\\
H.$x \in (-\infty,-13] \cup [-1,3]$
\testStop
\kluczStart
A
\kluczStop



\zadStart{Zadanie z Wikieł Z 1.62 b) moja wersja nr 24}

Rozwiązać nierówności $(x+14)(3-x)(x+1)\ge0$.
\zadStop
\rozwStart{Patryk Wirkus}{Laura Mieczkowska}
Miejsca zerowe naszego wielomianu to: $-14, 3, -1$.\\
Wielomian jest stopnia nieparzystego, ponadto znak współczynnika przy\linebreak najwyższej potędze x jest ujemny.\\ W związku z tym wykres wielomianu zaczyna się od lewej strony powyżej osi OX. A więc $$x \in (-\infty,-14) \cup (-1,3).$$
\rozwStop
\odpStart
$x \in (-\infty,-14) \cup (-1,3)$
\odpStop
\testStart
A.$x \in (-\infty,-14) \cup (-1,3)$\\
B.$x \in (-\infty,-14) \cup (-1,3]$\\
C.$x \in (-\infty,-14) \cup [-1,3)$\\
D.$x \in (-\infty,-14] \cup (-1,3)$\\
E.$x \in (-\infty,-14] \cup (-1,3]$\\
F.$x \in (-\infty,-14] \cup [-1,3)$\\
G.$x \in (-\infty,-14) \cup [-1,3]$\\
H.$x \in (-\infty,-14] \cup [-1,3]$
\testStop
\kluczStart
A
\kluczStop



\zadStart{Zadanie z Wikieł Z 1.62 b) moja wersja nr 25}

Rozwiązać nierówności $(x+15)(3-x)(x+1)\ge0$.
\zadStop
\rozwStart{Patryk Wirkus}{Laura Mieczkowska}
Miejsca zerowe naszego wielomianu to: $-15, 3, -1$.\\
Wielomian jest stopnia nieparzystego, ponadto znak współczynnika przy\linebreak najwyższej potędze x jest ujemny.\\ W związku z tym wykres wielomianu zaczyna się od lewej strony powyżej osi OX. A więc $$x \in (-\infty,-15) \cup (-1,3).$$
\rozwStop
\odpStart
$x \in (-\infty,-15) \cup (-1,3)$
\odpStop
\testStart
A.$x \in (-\infty,-15) \cup (-1,3)$\\
B.$x \in (-\infty,-15) \cup (-1,3]$\\
C.$x \in (-\infty,-15) \cup [-1,3)$\\
D.$x \in (-\infty,-15] \cup (-1,3)$\\
E.$x \in (-\infty,-15] \cup (-1,3]$\\
F.$x \in (-\infty,-15] \cup [-1,3)$\\
G.$x \in (-\infty,-15) \cup [-1,3]$\\
H.$x \in (-\infty,-15] \cup [-1,3]$
\testStop
\kluczStart
A
\kluczStop



\zadStart{Zadanie z Wikieł Z 1.62 b) moja wersja nr 26}

Rozwiązać nierówności $(x+5)(4-x)(x+1)\ge0$.
\zadStop
\rozwStart{Patryk Wirkus}{Laura Mieczkowska}
Miejsca zerowe naszego wielomianu to: $-5, 4, -1$.\\
Wielomian jest stopnia nieparzystego, ponadto znak współczynnika przy\linebreak najwyższej potędze x jest ujemny.\\ W związku z tym wykres wielomianu zaczyna się od lewej strony powyżej osi OX. A więc $$x \in (-\infty,-5) \cup (-1,4).$$
\rozwStop
\odpStart
$x \in (-\infty,-5) \cup (-1,4)$
\odpStop
\testStart
A.$x \in (-\infty,-5) \cup (-1,4)$\\
B.$x \in (-\infty,-5) \cup (-1,4]$\\
C.$x \in (-\infty,-5) \cup [-1,4)$\\
D.$x \in (-\infty,-5] \cup (-1,4)$\\
E.$x \in (-\infty,-5] \cup (-1,4]$\\
F.$x \in (-\infty,-5] \cup [-1,4)$\\
G.$x \in (-\infty,-5) \cup [-1,4]$\\
H.$x \in (-\infty,-5] \cup [-1,4]$
\testStop
\kluczStart
A
\kluczStop



\zadStart{Zadanie z Wikieł Z 1.62 b) moja wersja nr 27}

Rozwiązać nierówności $(x+6)(4-x)(x+1)\ge0$.
\zadStop
\rozwStart{Patryk Wirkus}{Laura Mieczkowska}
Miejsca zerowe naszego wielomianu to: $-6, 4, -1$.\\
Wielomian jest stopnia nieparzystego, ponadto znak współczynnika przy\linebreak najwyższej potędze x jest ujemny.\\ W związku z tym wykres wielomianu zaczyna się od lewej strony powyżej osi OX. A więc $$x \in (-\infty,-6) \cup (-1,4).$$
\rozwStop
\odpStart
$x \in (-\infty,-6) \cup (-1,4)$
\odpStop
\testStart
A.$x \in (-\infty,-6) \cup (-1,4)$\\
B.$x \in (-\infty,-6) \cup (-1,4]$\\
C.$x \in (-\infty,-6) \cup [-1,4)$\\
D.$x \in (-\infty,-6] \cup (-1,4)$\\
E.$x \in (-\infty,-6] \cup (-1,4]$\\
F.$x \in (-\infty,-6] \cup [-1,4)$\\
G.$x \in (-\infty,-6) \cup [-1,4]$\\
H.$x \in (-\infty,-6] \cup [-1,4]$
\testStop
\kluczStart
A
\kluczStop



\zadStart{Zadanie z Wikieł Z 1.62 b) moja wersja nr 28}

Rozwiązać nierówności $(x+7)(4-x)(x+1)\ge0$.
\zadStop
\rozwStart{Patryk Wirkus}{Laura Mieczkowska}
Miejsca zerowe naszego wielomianu to: $-7, 4, -1$.\\
Wielomian jest stopnia nieparzystego, ponadto znak współczynnika przy\linebreak najwyższej potędze x jest ujemny.\\ W związku z tym wykres wielomianu zaczyna się od lewej strony powyżej osi OX. A więc $$x \in (-\infty,-7) \cup (-1,4).$$
\rozwStop
\odpStart
$x \in (-\infty,-7) \cup (-1,4)$
\odpStop
\testStart
A.$x \in (-\infty,-7) \cup (-1,4)$\\
B.$x \in (-\infty,-7) \cup (-1,4]$\\
C.$x \in (-\infty,-7) \cup [-1,4)$\\
D.$x \in (-\infty,-7] \cup (-1,4)$\\
E.$x \in (-\infty,-7] \cup (-1,4]$\\
F.$x \in (-\infty,-7] \cup [-1,4)$\\
G.$x \in (-\infty,-7) \cup [-1,4]$\\
H.$x \in (-\infty,-7] \cup [-1,4]$
\testStop
\kluczStart
A
\kluczStop



\zadStart{Zadanie z Wikieł Z 1.62 b) moja wersja nr 29}

Rozwiązać nierówności $(x+8)(4-x)(x+1)\ge0$.
\zadStop
\rozwStart{Patryk Wirkus}{Laura Mieczkowska}
Miejsca zerowe naszego wielomianu to: $-8, 4, -1$.\\
Wielomian jest stopnia nieparzystego, ponadto znak współczynnika przy\linebreak najwyższej potędze x jest ujemny.\\ W związku z tym wykres wielomianu zaczyna się od lewej strony powyżej osi OX. A więc $$x \in (-\infty,-8) \cup (-1,4).$$
\rozwStop
\odpStart
$x \in (-\infty,-8) \cup (-1,4)$
\odpStop
\testStart
A.$x \in (-\infty,-8) \cup (-1,4)$\\
B.$x \in (-\infty,-8) \cup (-1,4]$\\
C.$x \in (-\infty,-8) \cup [-1,4)$\\
D.$x \in (-\infty,-8] \cup (-1,4)$\\
E.$x \in (-\infty,-8] \cup (-1,4]$\\
F.$x \in (-\infty,-8] \cup [-1,4)$\\
G.$x \in (-\infty,-8) \cup [-1,4]$\\
H.$x \in (-\infty,-8] \cup [-1,4]$
\testStop
\kluczStart
A
\kluczStop



\zadStart{Zadanie z Wikieł Z 1.62 b) moja wersja nr 30}

Rozwiązać nierówności $(x+9)(4-x)(x+1)\ge0$.
\zadStop
\rozwStart{Patryk Wirkus}{Laura Mieczkowska}
Miejsca zerowe naszego wielomianu to: $-9, 4, -1$.\\
Wielomian jest stopnia nieparzystego, ponadto znak współczynnika przy\linebreak najwyższej potędze x jest ujemny.\\ W związku z tym wykres wielomianu zaczyna się od lewej strony powyżej osi OX. A więc $$x \in (-\infty,-9) \cup (-1,4).$$
\rozwStop
\odpStart
$x \in (-\infty,-9) \cup (-1,4)$
\odpStop
\testStart
A.$x \in (-\infty,-9) \cup (-1,4)$\\
B.$x \in (-\infty,-9) \cup (-1,4]$\\
C.$x \in (-\infty,-9) \cup [-1,4)$\\
D.$x \in (-\infty,-9] \cup (-1,4)$\\
E.$x \in (-\infty,-9] \cup (-1,4]$\\
F.$x \in (-\infty,-9] \cup [-1,4)$\\
G.$x \in (-\infty,-9) \cup [-1,4]$\\
H.$x \in (-\infty,-9] \cup [-1,4]$
\testStop
\kluczStart
A
\kluczStop



\zadStart{Zadanie z Wikieł Z 1.62 b) moja wersja nr 31}

Rozwiązać nierówności $(x+10)(4-x)(x+1)\ge0$.
\zadStop
\rozwStart{Patryk Wirkus}{Laura Mieczkowska}
Miejsca zerowe naszego wielomianu to: $-10, 4, -1$.\\
Wielomian jest stopnia nieparzystego, ponadto znak współczynnika przy\linebreak najwyższej potędze x jest ujemny.\\ W związku z tym wykres wielomianu zaczyna się od lewej strony powyżej osi OX. A więc $$x \in (-\infty,-10) \cup (-1,4).$$
\rozwStop
\odpStart
$x \in (-\infty,-10) \cup (-1,4)$
\odpStop
\testStart
A.$x \in (-\infty,-10) \cup (-1,4)$\\
B.$x \in (-\infty,-10) \cup (-1,4]$\\
C.$x \in (-\infty,-10) \cup [-1,4)$\\
D.$x \in (-\infty,-10] \cup (-1,4)$\\
E.$x \in (-\infty,-10] \cup (-1,4]$\\
F.$x \in (-\infty,-10] \cup [-1,4)$\\
G.$x \in (-\infty,-10) \cup [-1,4]$\\
H.$x \in (-\infty,-10] \cup [-1,4]$
\testStop
\kluczStart
A
\kluczStop



\zadStart{Zadanie z Wikieł Z 1.62 b) moja wersja nr 32}

Rozwiązać nierówności $(x+11)(4-x)(x+1)\ge0$.
\zadStop
\rozwStart{Patryk Wirkus}{Laura Mieczkowska}
Miejsca zerowe naszego wielomianu to: $-11, 4, -1$.\\
Wielomian jest stopnia nieparzystego, ponadto znak współczynnika przy\linebreak najwyższej potędze x jest ujemny.\\ W związku z tym wykres wielomianu zaczyna się od lewej strony powyżej osi OX. A więc $$x \in (-\infty,-11) \cup (-1,4).$$
\rozwStop
\odpStart
$x \in (-\infty,-11) \cup (-1,4)$
\odpStop
\testStart
A.$x \in (-\infty,-11) \cup (-1,4)$\\
B.$x \in (-\infty,-11) \cup (-1,4]$\\
C.$x \in (-\infty,-11) \cup [-1,4)$\\
D.$x \in (-\infty,-11] \cup (-1,4)$\\
E.$x \in (-\infty,-11] \cup (-1,4]$\\
F.$x \in (-\infty,-11] \cup [-1,4)$\\
G.$x \in (-\infty,-11) \cup [-1,4]$\\
H.$x \in (-\infty,-11] \cup [-1,4]$
\testStop
\kluczStart
A
\kluczStop



\zadStart{Zadanie z Wikieł Z 1.62 b) moja wersja nr 33}

Rozwiązać nierówności $(x+12)(4-x)(x+1)\ge0$.
\zadStop
\rozwStart{Patryk Wirkus}{Laura Mieczkowska}
Miejsca zerowe naszego wielomianu to: $-12, 4, -1$.\\
Wielomian jest stopnia nieparzystego, ponadto znak współczynnika przy\linebreak najwyższej potędze x jest ujemny.\\ W związku z tym wykres wielomianu zaczyna się od lewej strony powyżej osi OX. A więc $$x \in (-\infty,-12) \cup (-1,4).$$
\rozwStop
\odpStart
$x \in (-\infty,-12) \cup (-1,4)$
\odpStop
\testStart
A.$x \in (-\infty,-12) \cup (-1,4)$\\
B.$x \in (-\infty,-12) \cup (-1,4]$\\
C.$x \in (-\infty,-12) \cup [-1,4)$\\
D.$x \in (-\infty,-12] \cup (-1,4)$\\
E.$x \in (-\infty,-12] \cup (-1,4]$\\
F.$x \in (-\infty,-12] \cup [-1,4)$\\
G.$x \in (-\infty,-12) \cup [-1,4]$\\
H.$x \in (-\infty,-12] \cup [-1,4]$
\testStop
\kluczStart
A
\kluczStop



\zadStart{Zadanie z Wikieł Z 1.62 b) moja wersja nr 34}

Rozwiązać nierówności $(x+13)(4-x)(x+1)\ge0$.
\zadStop
\rozwStart{Patryk Wirkus}{Laura Mieczkowska}
Miejsca zerowe naszego wielomianu to: $-13, 4, -1$.\\
Wielomian jest stopnia nieparzystego, ponadto znak współczynnika przy\linebreak najwyższej potędze x jest ujemny.\\ W związku z tym wykres wielomianu zaczyna się od lewej strony powyżej osi OX. A więc $$x \in (-\infty,-13) \cup (-1,4).$$
\rozwStop
\odpStart
$x \in (-\infty,-13) \cup (-1,4)$
\odpStop
\testStart
A.$x \in (-\infty,-13) \cup (-1,4)$\\
B.$x \in (-\infty,-13) \cup (-1,4]$\\
C.$x \in (-\infty,-13) \cup [-1,4)$\\
D.$x \in (-\infty,-13] \cup (-1,4)$\\
E.$x \in (-\infty,-13] \cup (-1,4]$\\
F.$x \in (-\infty,-13] \cup [-1,4)$\\
G.$x \in (-\infty,-13) \cup [-1,4]$\\
H.$x \in (-\infty,-13] \cup [-1,4]$
\testStop
\kluczStart
A
\kluczStop



\zadStart{Zadanie z Wikieł Z 1.62 b) moja wersja nr 35}

Rozwiązać nierówności $(x+14)(4-x)(x+1)\ge0$.
\zadStop
\rozwStart{Patryk Wirkus}{Laura Mieczkowska}
Miejsca zerowe naszego wielomianu to: $-14, 4, -1$.\\
Wielomian jest stopnia nieparzystego, ponadto znak współczynnika przy\linebreak najwyższej potędze x jest ujemny.\\ W związku z tym wykres wielomianu zaczyna się od lewej strony powyżej osi OX. A więc $$x \in (-\infty,-14) \cup (-1,4).$$
\rozwStop
\odpStart
$x \in (-\infty,-14) \cup (-1,4)$
\odpStop
\testStart
A.$x \in (-\infty,-14) \cup (-1,4)$\\
B.$x \in (-\infty,-14) \cup (-1,4]$\\
C.$x \in (-\infty,-14) \cup [-1,4)$\\
D.$x \in (-\infty,-14] \cup (-1,4)$\\
E.$x \in (-\infty,-14] \cup (-1,4]$\\
F.$x \in (-\infty,-14] \cup [-1,4)$\\
G.$x \in (-\infty,-14) \cup [-1,4]$\\
H.$x \in (-\infty,-14] \cup [-1,4]$
\testStop
\kluczStart
A
\kluczStop



\zadStart{Zadanie z Wikieł Z 1.62 b) moja wersja nr 36}

Rozwiązać nierówności $(x+15)(4-x)(x+1)\ge0$.
\zadStop
\rozwStart{Patryk Wirkus}{Laura Mieczkowska}
Miejsca zerowe naszego wielomianu to: $-15, 4, -1$.\\
Wielomian jest stopnia nieparzystego, ponadto znak współczynnika przy\linebreak najwyższej potędze x jest ujemny.\\ W związku z tym wykres wielomianu zaczyna się od lewej strony powyżej osi OX. A więc $$x \in (-\infty,-15) \cup (-1,4).$$
\rozwStop
\odpStart
$x \in (-\infty,-15) \cup (-1,4)$
\odpStop
\testStart
A.$x \in (-\infty,-15) \cup (-1,4)$\\
B.$x \in (-\infty,-15) \cup (-1,4]$\\
C.$x \in (-\infty,-15) \cup [-1,4)$\\
D.$x \in (-\infty,-15] \cup (-1,4)$\\
E.$x \in (-\infty,-15] \cup (-1,4]$\\
F.$x \in (-\infty,-15] \cup [-1,4)$\\
G.$x \in (-\infty,-15) \cup [-1,4]$\\
H.$x \in (-\infty,-15] \cup [-1,4]$
\testStop
\kluczStart
A
\kluczStop



\zadStart{Zadanie z Wikieł Z 1.62 b) moja wersja nr 37}

Rozwiązać nierówności $(x+6)(5-x)(x+1)\ge0$.
\zadStop
\rozwStart{Patryk Wirkus}{Laura Mieczkowska}
Miejsca zerowe naszego wielomianu to: $-6, 5, -1$.\\
Wielomian jest stopnia nieparzystego, ponadto znak współczynnika przy\linebreak najwyższej potędze x jest ujemny.\\ W związku z tym wykres wielomianu zaczyna się od lewej strony powyżej osi OX. A więc $$x \in (-\infty,-6) \cup (-1,5).$$
\rozwStop
\odpStart
$x \in (-\infty,-6) \cup (-1,5)$
\odpStop
\testStart
A.$x \in (-\infty,-6) \cup (-1,5)$\\
B.$x \in (-\infty,-6) \cup (-1,5]$\\
C.$x \in (-\infty,-6) \cup [-1,5)$\\
D.$x \in (-\infty,-6] \cup (-1,5)$\\
E.$x \in (-\infty,-6] \cup (-1,5]$\\
F.$x \in (-\infty,-6] \cup [-1,5)$\\
G.$x \in (-\infty,-6) \cup [-1,5]$\\
H.$x \in (-\infty,-6] \cup [-1,5]$
\testStop
\kluczStart
A
\kluczStop



\zadStart{Zadanie z Wikieł Z 1.62 b) moja wersja nr 38}

Rozwiązać nierówności $(x+7)(5-x)(x+1)\ge0$.
\zadStop
\rozwStart{Patryk Wirkus}{Laura Mieczkowska}
Miejsca zerowe naszego wielomianu to: $-7, 5, -1$.\\
Wielomian jest stopnia nieparzystego, ponadto znak współczynnika przy\linebreak najwyższej potędze x jest ujemny.\\ W związku z tym wykres wielomianu zaczyna się od lewej strony powyżej osi OX. A więc $$x \in (-\infty,-7) \cup (-1,5).$$
\rozwStop
\odpStart
$x \in (-\infty,-7) \cup (-1,5)$
\odpStop
\testStart
A.$x \in (-\infty,-7) \cup (-1,5)$\\
B.$x \in (-\infty,-7) \cup (-1,5]$\\
C.$x \in (-\infty,-7) \cup [-1,5)$\\
D.$x \in (-\infty,-7] \cup (-1,5)$\\
E.$x \in (-\infty,-7] \cup (-1,5]$\\
F.$x \in (-\infty,-7] \cup [-1,5)$\\
G.$x \in (-\infty,-7) \cup [-1,5]$\\
H.$x \in (-\infty,-7] \cup [-1,5]$
\testStop
\kluczStart
A
\kluczStop



\zadStart{Zadanie z Wikieł Z 1.62 b) moja wersja nr 39}

Rozwiązać nierówności $(x+8)(5-x)(x+1)\ge0$.
\zadStop
\rozwStart{Patryk Wirkus}{Laura Mieczkowska}
Miejsca zerowe naszego wielomianu to: $-8, 5, -1$.\\
Wielomian jest stopnia nieparzystego, ponadto znak współczynnika przy\linebreak najwyższej potędze x jest ujemny.\\ W związku z tym wykres wielomianu zaczyna się od lewej strony powyżej osi OX. A więc $$x \in (-\infty,-8) \cup (-1,5).$$
\rozwStop
\odpStart
$x \in (-\infty,-8) \cup (-1,5)$
\odpStop
\testStart
A.$x \in (-\infty,-8) \cup (-1,5)$\\
B.$x \in (-\infty,-8) \cup (-1,5]$\\
C.$x \in (-\infty,-8) \cup [-1,5)$\\
D.$x \in (-\infty,-8] \cup (-1,5)$\\
E.$x \in (-\infty,-8] \cup (-1,5]$\\
F.$x \in (-\infty,-8] \cup [-1,5)$\\
G.$x \in (-\infty,-8) \cup [-1,5]$\\
H.$x \in (-\infty,-8] \cup [-1,5]$
\testStop
\kluczStart
A
\kluczStop



\zadStart{Zadanie z Wikieł Z 1.62 b) moja wersja nr 40}

Rozwiązać nierówności $(x+9)(5-x)(x+1)\ge0$.
\zadStop
\rozwStart{Patryk Wirkus}{Laura Mieczkowska}
Miejsca zerowe naszego wielomianu to: $-9, 5, -1$.\\
Wielomian jest stopnia nieparzystego, ponadto znak współczynnika przy\linebreak najwyższej potędze x jest ujemny.\\ W związku z tym wykres wielomianu zaczyna się od lewej strony powyżej osi OX. A więc $$x \in (-\infty,-9) \cup (-1,5).$$
\rozwStop
\odpStart
$x \in (-\infty,-9) \cup (-1,5)$
\odpStop
\testStart
A.$x \in (-\infty,-9) \cup (-1,5)$\\
B.$x \in (-\infty,-9) \cup (-1,5]$\\
C.$x \in (-\infty,-9) \cup [-1,5)$\\
D.$x \in (-\infty,-9] \cup (-1,5)$\\
E.$x \in (-\infty,-9] \cup (-1,5]$\\
F.$x \in (-\infty,-9] \cup [-1,5)$\\
G.$x \in (-\infty,-9) \cup [-1,5]$\\
H.$x \in (-\infty,-9] \cup [-1,5]$
\testStop
\kluczStart
A
\kluczStop



\zadStart{Zadanie z Wikieł Z 1.62 b) moja wersja nr 41}

Rozwiązać nierówności $(x+10)(5-x)(x+1)\ge0$.
\zadStop
\rozwStart{Patryk Wirkus}{Laura Mieczkowska}
Miejsca zerowe naszego wielomianu to: $-10, 5, -1$.\\
Wielomian jest stopnia nieparzystego, ponadto znak współczynnika przy\linebreak najwyższej potędze x jest ujemny.\\ W związku z tym wykres wielomianu zaczyna się od lewej strony powyżej osi OX. A więc $$x \in (-\infty,-10) \cup (-1,5).$$
\rozwStop
\odpStart
$x \in (-\infty,-10) \cup (-1,5)$
\odpStop
\testStart
A.$x \in (-\infty,-10) \cup (-1,5)$\\
B.$x \in (-\infty,-10) \cup (-1,5]$\\
C.$x \in (-\infty,-10) \cup [-1,5)$\\
D.$x \in (-\infty,-10] \cup (-1,5)$\\
E.$x \in (-\infty,-10] \cup (-1,5]$\\
F.$x \in (-\infty,-10] \cup [-1,5)$\\
G.$x \in (-\infty,-10) \cup [-1,5]$\\
H.$x \in (-\infty,-10] \cup [-1,5]$
\testStop
\kluczStart
A
\kluczStop



\zadStart{Zadanie z Wikieł Z 1.62 b) moja wersja nr 42}

Rozwiązać nierówności $(x+11)(5-x)(x+1)\ge0$.
\zadStop
\rozwStart{Patryk Wirkus}{Laura Mieczkowska}
Miejsca zerowe naszego wielomianu to: $-11, 5, -1$.\\
Wielomian jest stopnia nieparzystego, ponadto znak współczynnika przy\linebreak najwyższej potędze x jest ujemny.\\ W związku z tym wykres wielomianu zaczyna się od lewej strony powyżej osi OX. A więc $$x \in (-\infty,-11) \cup (-1,5).$$
\rozwStop
\odpStart
$x \in (-\infty,-11) \cup (-1,5)$
\odpStop
\testStart
A.$x \in (-\infty,-11) \cup (-1,5)$\\
B.$x \in (-\infty,-11) \cup (-1,5]$\\
C.$x \in (-\infty,-11) \cup [-1,5)$\\
D.$x \in (-\infty,-11] \cup (-1,5)$\\
E.$x \in (-\infty,-11] \cup (-1,5]$\\
F.$x \in (-\infty,-11] \cup [-1,5)$\\
G.$x \in (-\infty,-11) \cup [-1,5]$\\
H.$x \in (-\infty,-11] \cup [-1,5]$
\testStop
\kluczStart
A
\kluczStop



\zadStart{Zadanie z Wikieł Z 1.62 b) moja wersja nr 43}

Rozwiązać nierówności $(x+12)(5-x)(x+1)\ge0$.
\zadStop
\rozwStart{Patryk Wirkus}{Laura Mieczkowska}
Miejsca zerowe naszego wielomianu to: $-12, 5, -1$.\\
Wielomian jest stopnia nieparzystego, ponadto znak współczynnika przy\linebreak najwyższej potędze x jest ujemny.\\ W związku z tym wykres wielomianu zaczyna się od lewej strony powyżej osi OX. A więc $$x \in (-\infty,-12) \cup (-1,5).$$
\rozwStop
\odpStart
$x \in (-\infty,-12) \cup (-1,5)$
\odpStop
\testStart
A.$x \in (-\infty,-12) \cup (-1,5)$\\
B.$x \in (-\infty,-12) \cup (-1,5]$\\
C.$x \in (-\infty,-12) \cup [-1,5)$\\
D.$x \in (-\infty,-12] \cup (-1,5)$\\
E.$x \in (-\infty,-12] \cup (-1,5]$\\
F.$x \in (-\infty,-12] \cup [-1,5)$\\
G.$x \in (-\infty,-12) \cup [-1,5]$\\
H.$x \in (-\infty,-12] \cup [-1,5]$
\testStop
\kluczStart
A
\kluczStop



\zadStart{Zadanie z Wikieł Z 1.62 b) moja wersja nr 44}

Rozwiązać nierówności $(x+13)(5-x)(x+1)\ge0$.
\zadStop
\rozwStart{Patryk Wirkus}{Laura Mieczkowska}
Miejsca zerowe naszego wielomianu to: $-13, 5, -1$.\\
Wielomian jest stopnia nieparzystego, ponadto znak współczynnika przy\linebreak najwyższej potędze x jest ujemny.\\ W związku z tym wykres wielomianu zaczyna się od lewej strony powyżej osi OX. A więc $$x \in (-\infty,-13) \cup (-1,5).$$
\rozwStop
\odpStart
$x \in (-\infty,-13) \cup (-1,5)$
\odpStop
\testStart
A.$x \in (-\infty,-13) \cup (-1,5)$\\
B.$x \in (-\infty,-13) \cup (-1,5]$\\
C.$x \in (-\infty,-13) \cup [-1,5)$\\
D.$x \in (-\infty,-13] \cup (-1,5)$\\
E.$x \in (-\infty,-13] \cup (-1,5]$\\
F.$x \in (-\infty,-13] \cup [-1,5)$\\
G.$x \in (-\infty,-13) \cup [-1,5]$\\
H.$x \in (-\infty,-13] \cup [-1,5]$
\testStop
\kluczStart
A
\kluczStop



\zadStart{Zadanie z Wikieł Z 1.62 b) moja wersja nr 45}

Rozwiązać nierówności $(x+14)(5-x)(x+1)\ge0$.
\zadStop
\rozwStart{Patryk Wirkus}{Laura Mieczkowska}
Miejsca zerowe naszego wielomianu to: $-14, 5, -1$.\\
Wielomian jest stopnia nieparzystego, ponadto znak współczynnika przy\linebreak najwyższej potędze x jest ujemny.\\ W związku z tym wykres wielomianu zaczyna się od lewej strony powyżej osi OX. A więc $$x \in (-\infty,-14) \cup (-1,5).$$
\rozwStop
\odpStart
$x \in (-\infty,-14) \cup (-1,5)$
\odpStop
\testStart
A.$x \in (-\infty,-14) \cup (-1,5)$\\
B.$x \in (-\infty,-14) \cup (-1,5]$\\
C.$x \in (-\infty,-14) \cup [-1,5)$\\
D.$x \in (-\infty,-14] \cup (-1,5)$\\
E.$x \in (-\infty,-14] \cup (-1,5]$\\
F.$x \in (-\infty,-14] \cup [-1,5)$\\
G.$x \in (-\infty,-14) \cup [-1,5]$\\
H.$x \in (-\infty,-14] \cup [-1,5]$
\testStop
\kluczStart
A
\kluczStop



\zadStart{Zadanie z Wikieł Z 1.62 b) moja wersja nr 46}

Rozwiązać nierówności $(x+15)(5-x)(x+1)\ge0$.
\zadStop
\rozwStart{Patryk Wirkus}{Laura Mieczkowska}
Miejsca zerowe naszego wielomianu to: $-15, 5, -1$.\\
Wielomian jest stopnia nieparzystego, ponadto znak współczynnika przy\linebreak najwyższej potędze x jest ujemny.\\ W związku z tym wykres wielomianu zaczyna się od lewej strony powyżej osi OX. A więc $$x \in (-\infty,-15) \cup (-1,5).$$
\rozwStop
\odpStart
$x \in (-\infty,-15) \cup (-1,5)$
\odpStop
\testStart
A.$x \in (-\infty,-15) \cup (-1,5)$\\
B.$x \in (-\infty,-15) \cup (-1,5]$\\
C.$x \in (-\infty,-15) \cup [-1,5)$\\
D.$x \in (-\infty,-15] \cup (-1,5)$\\
E.$x \in (-\infty,-15] \cup (-1,5]$\\
F.$x \in (-\infty,-15] \cup [-1,5)$\\
G.$x \in (-\infty,-15) \cup [-1,5]$\\
H.$x \in (-\infty,-15] \cup [-1,5]$
\testStop
\kluczStart
A
\kluczStop



\zadStart{Zadanie z Wikieł Z 1.62 b) moja wersja nr 47}

Rozwiązać nierówności $(x+7)(6-x)(x+1)\ge0$.
\zadStop
\rozwStart{Patryk Wirkus}{Laura Mieczkowska}
Miejsca zerowe naszego wielomianu to: $-7, 6, -1$.\\
Wielomian jest stopnia nieparzystego, ponadto znak współczynnika przy\linebreak najwyższej potędze x jest ujemny.\\ W związku z tym wykres wielomianu zaczyna się od lewej strony powyżej osi OX. A więc $$x \in (-\infty,-7) \cup (-1,6).$$
\rozwStop
\odpStart
$x \in (-\infty,-7) \cup (-1,6)$
\odpStop
\testStart
A.$x \in (-\infty,-7) \cup (-1,6)$\\
B.$x \in (-\infty,-7) \cup (-1,6]$\\
C.$x \in (-\infty,-7) \cup [-1,6)$\\
D.$x \in (-\infty,-7] \cup (-1,6)$\\
E.$x \in (-\infty,-7] \cup (-1,6]$\\
F.$x \in (-\infty,-7] \cup [-1,6)$\\
G.$x \in (-\infty,-7) \cup [-1,6]$\\
H.$x \in (-\infty,-7] \cup [-1,6]$
\testStop
\kluczStart
A
\kluczStop



\zadStart{Zadanie z Wikieł Z 1.62 b) moja wersja nr 48}

Rozwiązać nierówności $(x+8)(6-x)(x+1)\ge0$.
\zadStop
\rozwStart{Patryk Wirkus}{Laura Mieczkowska}
Miejsca zerowe naszego wielomianu to: $-8, 6, -1$.\\
Wielomian jest stopnia nieparzystego, ponadto znak współczynnika przy\linebreak najwyższej potędze x jest ujemny.\\ W związku z tym wykres wielomianu zaczyna się od lewej strony powyżej osi OX. A więc $$x \in (-\infty,-8) \cup (-1,6).$$
\rozwStop
\odpStart
$x \in (-\infty,-8) \cup (-1,6)$
\odpStop
\testStart
A.$x \in (-\infty,-8) \cup (-1,6)$\\
B.$x \in (-\infty,-8) \cup (-1,6]$\\
C.$x \in (-\infty,-8) \cup [-1,6)$\\
D.$x \in (-\infty,-8] \cup (-1,6)$\\
E.$x \in (-\infty,-8] \cup (-1,6]$\\
F.$x \in (-\infty,-8] \cup [-1,6)$\\
G.$x \in (-\infty,-8) \cup [-1,6]$\\
H.$x \in (-\infty,-8] \cup [-1,6]$
\testStop
\kluczStart
A
\kluczStop



\zadStart{Zadanie z Wikieł Z 1.62 b) moja wersja nr 49}

Rozwiązać nierówności $(x+9)(6-x)(x+1)\ge0$.
\zadStop
\rozwStart{Patryk Wirkus}{Laura Mieczkowska}
Miejsca zerowe naszego wielomianu to: $-9, 6, -1$.\\
Wielomian jest stopnia nieparzystego, ponadto znak współczynnika przy\linebreak najwyższej potędze x jest ujemny.\\ W związku z tym wykres wielomianu zaczyna się od lewej strony powyżej osi OX. A więc $$x \in (-\infty,-9) \cup (-1,6).$$
\rozwStop
\odpStart
$x \in (-\infty,-9) \cup (-1,6)$
\odpStop
\testStart
A.$x \in (-\infty,-9) \cup (-1,6)$\\
B.$x \in (-\infty,-9) \cup (-1,6]$\\
C.$x \in (-\infty,-9) \cup [-1,6)$\\
D.$x \in (-\infty,-9] \cup (-1,6)$\\
E.$x \in (-\infty,-9] \cup (-1,6]$\\
F.$x \in (-\infty,-9] \cup [-1,6)$\\
G.$x \in (-\infty,-9) \cup [-1,6]$\\
H.$x \in (-\infty,-9] \cup [-1,6]$
\testStop
\kluczStart
A
\kluczStop



\zadStart{Zadanie z Wikieł Z 1.62 b) moja wersja nr 50}

Rozwiązać nierówności $(x+10)(6-x)(x+1)\ge0$.
\zadStop
\rozwStart{Patryk Wirkus}{Laura Mieczkowska}
Miejsca zerowe naszego wielomianu to: $-10, 6, -1$.\\
Wielomian jest stopnia nieparzystego, ponadto znak współczynnika przy\linebreak najwyższej potędze x jest ujemny.\\ W związku z tym wykres wielomianu zaczyna się od lewej strony powyżej osi OX. A więc $$x \in (-\infty,-10) \cup (-1,6).$$
\rozwStop
\odpStart
$x \in (-\infty,-10) \cup (-1,6)$
\odpStop
\testStart
A.$x \in (-\infty,-10) \cup (-1,6)$\\
B.$x \in (-\infty,-10) \cup (-1,6]$\\
C.$x \in (-\infty,-10) \cup [-1,6)$\\
D.$x \in (-\infty,-10] \cup (-1,6)$\\
E.$x \in (-\infty,-10] \cup (-1,6]$\\
F.$x \in (-\infty,-10] \cup [-1,6)$\\
G.$x \in (-\infty,-10) \cup [-1,6]$\\
H.$x \in (-\infty,-10] \cup [-1,6]$
\testStop
\kluczStart
A
\kluczStop



\zadStart{Zadanie z Wikieł Z 1.62 b) moja wersja nr 51}

Rozwiązać nierówności $(x+11)(6-x)(x+1)\ge0$.
\zadStop
\rozwStart{Patryk Wirkus}{Laura Mieczkowska}
Miejsca zerowe naszego wielomianu to: $-11, 6, -1$.\\
Wielomian jest stopnia nieparzystego, ponadto znak współczynnika przy\linebreak najwyższej potędze x jest ujemny.\\ W związku z tym wykres wielomianu zaczyna się od lewej strony powyżej osi OX. A więc $$x \in (-\infty,-11) \cup (-1,6).$$
\rozwStop
\odpStart
$x \in (-\infty,-11) \cup (-1,6)$
\odpStop
\testStart
A.$x \in (-\infty,-11) \cup (-1,6)$\\
B.$x \in (-\infty,-11) \cup (-1,6]$\\
C.$x \in (-\infty,-11) \cup [-1,6)$\\
D.$x \in (-\infty,-11] \cup (-1,6)$\\
E.$x \in (-\infty,-11] \cup (-1,6]$\\
F.$x \in (-\infty,-11] \cup [-1,6)$\\
G.$x \in (-\infty,-11) \cup [-1,6]$\\
H.$x \in (-\infty,-11] \cup [-1,6]$
\testStop
\kluczStart
A
\kluczStop



\zadStart{Zadanie z Wikieł Z 1.62 b) moja wersja nr 52}

Rozwiązać nierówności $(x+12)(6-x)(x+1)\ge0$.
\zadStop
\rozwStart{Patryk Wirkus}{Laura Mieczkowska}
Miejsca zerowe naszego wielomianu to: $-12, 6, -1$.\\
Wielomian jest stopnia nieparzystego, ponadto znak współczynnika przy\linebreak najwyższej potędze x jest ujemny.\\ W związku z tym wykres wielomianu zaczyna się od lewej strony powyżej osi OX. A więc $$x \in (-\infty,-12) \cup (-1,6).$$
\rozwStop
\odpStart
$x \in (-\infty,-12) \cup (-1,6)$
\odpStop
\testStart
A.$x \in (-\infty,-12) \cup (-1,6)$\\
B.$x \in (-\infty,-12) \cup (-1,6]$\\
C.$x \in (-\infty,-12) \cup [-1,6)$\\
D.$x \in (-\infty,-12] \cup (-1,6)$\\
E.$x \in (-\infty,-12] \cup (-1,6]$\\
F.$x \in (-\infty,-12] \cup [-1,6)$\\
G.$x \in (-\infty,-12) \cup [-1,6]$\\
H.$x \in (-\infty,-12] \cup [-1,6]$
\testStop
\kluczStart
A
\kluczStop



\zadStart{Zadanie z Wikieł Z 1.62 b) moja wersja nr 53}

Rozwiązać nierówności $(x+13)(6-x)(x+1)\ge0$.
\zadStop
\rozwStart{Patryk Wirkus}{Laura Mieczkowska}
Miejsca zerowe naszego wielomianu to: $-13, 6, -1$.\\
Wielomian jest stopnia nieparzystego, ponadto znak współczynnika przy\linebreak najwyższej potędze x jest ujemny.\\ W związku z tym wykres wielomianu zaczyna się od lewej strony powyżej osi OX. A więc $$x \in (-\infty,-13) \cup (-1,6).$$
\rozwStop
\odpStart
$x \in (-\infty,-13) \cup (-1,6)$
\odpStop
\testStart
A.$x \in (-\infty,-13) \cup (-1,6)$\\
B.$x \in (-\infty,-13) \cup (-1,6]$\\
C.$x \in (-\infty,-13) \cup [-1,6)$\\
D.$x \in (-\infty,-13] \cup (-1,6)$\\
E.$x \in (-\infty,-13] \cup (-1,6]$\\
F.$x \in (-\infty,-13] \cup [-1,6)$\\
G.$x \in (-\infty,-13) \cup [-1,6]$\\
H.$x \in (-\infty,-13] \cup [-1,6]$
\testStop
\kluczStart
A
\kluczStop



\zadStart{Zadanie z Wikieł Z 1.62 b) moja wersja nr 54}

Rozwiązać nierówności $(x+14)(6-x)(x+1)\ge0$.
\zadStop
\rozwStart{Patryk Wirkus}{Laura Mieczkowska}
Miejsca zerowe naszego wielomianu to: $-14, 6, -1$.\\
Wielomian jest stopnia nieparzystego, ponadto znak współczynnika przy\linebreak najwyższej potędze x jest ujemny.\\ W związku z tym wykres wielomianu zaczyna się od lewej strony powyżej osi OX. A więc $$x \in (-\infty,-14) \cup (-1,6).$$
\rozwStop
\odpStart
$x \in (-\infty,-14) \cup (-1,6)$
\odpStop
\testStart
A.$x \in (-\infty,-14) \cup (-1,6)$\\
B.$x \in (-\infty,-14) \cup (-1,6]$\\
C.$x \in (-\infty,-14) \cup [-1,6)$\\
D.$x \in (-\infty,-14] \cup (-1,6)$\\
E.$x \in (-\infty,-14] \cup (-1,6]$\\
F.$x \in (-\infty,-14] \cup [-1,6)$\\
G.$x \in (-\infty,-14) \cup [-1,6]$\\
H.$x \in (-\infty,-14] \cup [-1,6]$
\testStop
\kluczStart
A
\kluczStop



\zadStart{Zadanie z Wikieł Z 1.62 b) moja wersja nr 55}

Rozwiązać nierówności $(x+15)(6-x)(x+1)\ge0$.
\zadStop
\rozwStart{Patryk Wirkus}{Laura Mieczkowska}
Miejsca zerowe naszego wielomianu to: $-15, 6, -1$.\\
Wielomian jest stopnia nieparzystego, ponadto znak współczynnika przy\linebreak najwyższej potędze x jest ujemny.\\ W związku z tym wykres wielomianu zaczyna się od lewej strony powyżej osi OX. A więc $$x \in (-\infty,-15) \cup (-1,6).$$
\rozwStop
\odpStart
$x \in (-\infty,-15) \cup (-1,6)$
\odpStop
\testStart
A.$x \in (-\infty,-15) \cup (-1,6)$\\
B.$x \in (-\infty,-15) \cup (-1,6]$\\
C.$x \in (-\infty,-15) \cup [-1,6)$\\
D.$x \in (-\infty,-15] \cup (-1,6)$\\
E.$x \in (-\infty,-15] \cup (-1,6]$\\
F.$x \in (-\infty,-15] \cup [-1,6)$\\
G.$x \in (-\infty,-15) \cup [-1,6]$\\
H.$x \in (-\infty,-15] \cup [-1,6]$
\testStop
\kluczStart
A
\kluczStop



\zadStart{Zadanie z Wikieł Z 1.62 b) moja wersja nr 56}

Rozwiązać nierówności $(x+8)(7-x)(x+1)\ge0$.
\zadStop
\rozwStart{Patryk Wirkus}{Laura Mieczkowska}
Miejsca zerowe naszego wielomianu to: $-8, 7, -1$.\\
Wielomian jest stopnia nieparzystego, ponadto znak współczynnika przy\linebreak najwyższej potędze x jest ujemny.\\ W związku z tym wykres wielomianu zaczyna się od lewej strony powyżej osi OX. A więc $$x \in (-\infty,-8) \cup (-1,7).$$
\rozwStop
\odpStart
$x \in (-\infty,-8) \cup (-1,7)$
\odpStop
\testStart
A.$x \in (-\infty,-8) \cup (-1,7)$\\
B.$x \in (-\infty,-8) \cup (-1,7]$\\
C.$x \in (-\infty,-8) \cup [-1,7)$\\
D.$x \in (-\infty,-8] \cup (-1,7)$\\
E.$x \in (-\infty,-8] \cup (-1,7]$\\
F.$x \in (-\infty,-8] \cup [-1,7)$\\
G.$x \in (-\infty,-8) \cup [-1,7]$\\
H.$x \in (-\infty,-8] \cup [-1,7]$
\testStop
\kluczStart
A
\kluczStop



\zadStart{Zadanie z Wikieł Z 1.62 b) moja wersja nr 57}

Rozwiązać nierówności $(x+9)(7-x)(x+1)\ge0$.
\zadStop
\rozwStart{Patryk Wirkus}{Laura Mieczkowska}
Miejsca zerowe naszego wielomianu to: $-9, 7, -1$.\\
Wielomian jest stopnia nieparzystego, ponadto znak współczynnika przy\linebreak najwyższej potędze x jest ujemny.\\ W związku z tym wykres wielomianu zaczyna się od lewej strony powyżej osi OX. A więc $$x \in (-\infty,-9) \cup (-1,7).$$
\rozwStop
\odpStart
$x \in (-\infty,-9) \cup (-1,7)$
\odpStop
\testStart
A.$x \in (-\infty,-9) \cup (-1,7)$\\
B.$x \in (-\infty,-9) \cup (-1,7]$\\
C.$x \in (-\infty,-9) \cup [-1,7)$\\
D.$x \in (-\infty,-9] \cup (-1,7)$\\
E.$x \in (-\infty,-9] \cup (-1,7]$\\
F.$x \in (-\infty,-9] \cup [-1,7)$\\
G.$x \in (-\infty,-9) \cup [-1,7]$\\
H.$x \in (-\infty,-9] \cup [-1,7]$
\testStop
\kluczStart
A
\kluczStop



\zadStart{Zadanie z Wikieł Z 1.62 b) moja wersja nr 58}

Rozwiązać nierówności $(x+10)(7-x)(x+1)\ge0$.
\zadStop
\rozwStart{Patryk Wirkus}{Laura Mieczkowska}
Miejsca zerowe naszego wielomianu to: $-10, 7, -1$.\\
Wielomian jest stopnia nieparzystego, ponadto znak współczynnika przy\linebreak najwyższej potędze x jest ujemny.\\ W związku z tym wykres wielomianu zaczyna się od lewej strony powyżej osi OX. A więc $$x \in (-\infty,-10) \cup (-1,7).$$
\rozwStop
\odpStart
$x \in (-\infty,-10) \cup (-1,7)$
\odpStop
\testStart
A.$x \in (-\infty,-10) \cup (-1,7)$\\
B.$x \in (-\infty,-10) \cup (-1,7]$\\
C.$x \in (-\infty,-10) \cup [-1,7)$\\
D.$x \in (-\infty,-10] \cup (-1,7)$\\
E.$x \in (-\infty,-10] \cup (-1,7]$\\
F.$x \in (-\infty,-10] \cup [-1,7)$\\
G.$x \in (-\infty,-10) \cup [-1,7]$\\
H.$x \in (-\infty,-10] \cup [-1,7]$
\testStop
\kluczStart
A
\kluczStop



\zadStart{Zadanie z Wikieł Z 1.62 b) moja wersja nr 59}

Rozwiązać nierówności $(x+11)(7-x)(x+1)\ge0$.
\zadStop
\rozwStart{Patryk Wirkus}{Laura Mieczkowska}
Miejsca zerowe naszego wielomianu to: $-11, 7, -1$.\\
Wielomian jest stopnia nieparzystego, ponadto znak współczynnika przy\linebreak najwyższej potędze x jest ujemny.\\ W związku z tym wykres wielomianu zaczyna się od lewej strony powyżej osi OX. A więc $$x \in (-\infty,-11) \cup (-1,7).$$
\rozwStop
\odpStart
$x \in (-\infty,-11) \cup (-1,7)$
\odpStop
\testStart
A.$x \in (-\infty,-11) \cup (-1,7)$\\
B.$x \in (-\infty,-11) \cup (-1,7]$\\
C.$x \in (-\infty,-11) \cup [-1,7)$\\
D.$x \in (-\infty,-11] \cup (-1,7)$\\
E.$x \in (-\infty,-11] \cup (-1,7]$\\
F.$x \in (-\infty,-11] \cup [-1,7)$\\
G.$x \in (-\infty,-11) \cup [-1,7]$\\
H.$x \in (-\infty,-11] \cup [-1,7]$
\testStop
\kluczStart
A
\kluczStop



\zadStart{Zadanie z Wikieł Z 1.62 b) moja wersja nr 60}

Rozwiązać nierówności $(x+12)(7-x)(x+1)\ge0$.
\zadStop
\rozwStart{Patryk Wirkus}{Laura Mieczkowska}
Miejsca zerowe naszego wielomianu to: $-12, 7, -1$.\\
Wielomian jest stopnia nieparzystego, ponadto znak współczynnika przy\linebreak najwyższej potędze x jest ujemny.\\ W związku z tym wykres wielomianu zaczyna się od lewej strony powyżej osi OX. A więc $$x \in (-\infty,-12) \cup (-1,7).$$
\rozwStop
\odpStart
$x \in (-\infty,-12) \cup (-1,7)$
\odpStop
\testStart
A.$x \in (-\infty,-12) \cup (-1,7)$\\
B.$x \in (-\infty,-12) \cup (-1,7]$\\
C.$x \in (-\infty,-12) \cup [-1,7)$\\
D.$x \in (-\infty,-12] \cup (-1,7)$\\
E.$x \in (-\infty,-12] \cup (-1,7]$\\
F.$x \in (-\infty,-12] \cup [-1,7)$\\
G.$x \in (-\infty,-12) \cup [-1,7]$\\
H.$x \in (-\infty,-12] \cup [-1,7]$
\testStop
\kluczStart
A
\kluczStop



\zadStart{Zadanie z Wikieł Z 1.62 b) moja wersja nr 61}

Rozwiązać nierówności $(x+13)(7-x)(x+1)\ge0$.
\zadStop
\rozwStart{Patryk Wirkus}{Laura Mieczkowska}
Miejsca zerowe naszego wielomianu to: $-13, 7, -1$.\\
Wielomian jest stopnia nieparzystego, ponadto znak współczynnika przy\linebreak najwyższej potędze x jest ujemny.\\ W związku z tym wykres wielomianu zaczyna się od lewej strony powyżej osi OX. A więc $$x \in (-\infty,-13) \cup (-1,7).$$
\rozwStop
\odpStart
$x \in (-\infty,-13) \cup (-1,7)$
\odpStop
\testStart
A.$x \in (-\infty,-13) \cup (-1,7)$\\
B.$x \in (-\infty,-13) \cup (-1,7]$\\
C.$x \in (-\infty,-13) \cup [-1,7)$\\
D.$x \in (-\infty,-13] \cup (-1,7)$\\
E.$x \in (-\infty,-13] \cup (-1,7]$\\
F.$x \in (-\infty,-13] \cup [-1,7)$\\
G.$x \in (-\infty,-13) \cup [-1,7]$\\
H.$x \in (-\infty,-13] \cup [-1,7]$
\testStop
\kluczStart
A
\kluczStop



\zadStart{Zadanie z Wikieł Z 1.62 b) moja wersja nr 62}

Rozwiązać nierówności $(x+14)(7-x)(x+1)\ge0$.
\zadStop
\rozwStart{Patryk Wirkus}{Laura Mieczkowska}
Miejsca zerowe naszego wielomianu to: $-14, 7, -1$.\\
Wielomian jest stopnia nieparzystego, ponadto znak współczynnika przy\linebreak najwyższej potędze x jest ujemny.\\ W związku z tym wykres wielomianu zaczyna się od lewej strony powyżej osi OX. A więc $$x \in (-\infty,-14) \cup (-1,7).$$
\rozwStop
\odpStart
$x \in (-\infty,-14) \cup (-1,7)$
\odpStop
\testStart
A.$x \in (-\infty,-14) \cup (-1,7)$\\
B.$x \in (-\infty,-14) \cup (-1,7]$\\
C.$x \in (-\infty,-14) \cup [-1,7)$\\
D.$x \in (-\infty,-14] \cup (-1,7)$\\
E.$x \in (-\infty,-14] \cup (-1,7]$\\
F.$x \in (-\infty,-14] \cup [-1,7)$\\
G.$x \in (-\infty,-14) \cup [-1,7]$\\
H.$x \in (-\infty,-14] \cup [-1,7]$
\testStop
\kluczStart
A
\kluczStop



\zadStart{Zadanie z Wikieł Z 1.62 b) moja wersja nr 63}

Rozwiązać nierówności $(x+15)(7-x)(x+1)\ge0$.
\zadStop
\rozwStart{Patryk Wirkus}{Laura Mieczkowska}
Miejsca zerowe naszego wielomianu to: $-15, 7, -1$.\\
Wielomian jest stopnia nieparzystego, ponadto znak współczynnika przy\linebreak najwyższej potędze x jest ujemny.\\ W związku z tym wykres wielomianu zaczyna się od lewej strony powyżej osi OX. A więc $$x \in (-\infty,-15) \cup (-1,7).$$
\rozwStop
\odpStart
$x \in (-\infty,-15) \cup (-1,7)$
\odpStop
\testStart
A.$x \in (-\infty,-15) \cup (-1,7)$\\
B.$x \in (-\infty,-15) \cup (-1,7]$\\
C.$x \in (-\infty,-15) \cup [-1,7)$\\
D.$x \in (-\infty,-15] \cup (-1,7)$\\
E.$x \in (-\infty,-15] \cup (-1,7]$\\
F.$x \in (-\infty,-15] \cup [-1,7)$\\
G.$x \in (-\infty,-15) \cup [-1,7]$\\
H.$x \in (-\infty,-15] \cup [-1,7]$
\testStop
\kluczStart
A
\kluczStop



\zadStart{Zadanie z Wikieł Z 1.62 b) moja wersja nr 64}

Rozwiązać nierówności $(x+9)(8-x)(x+1)\ge0$.
\zadStop
\rozwStart{Patryk Wirkus}{Laura Mieczkowska}
Miejsca zerowe naszego wielomianu to: $-9, 8, -1$.\\
Wielomian jest stopnia nieparzystego, ponadto znak współczynnika przy\linebreak najwyższej potędze x jest ujemny.\\ W związku z tym wykres wielomianu zaczyna się od lewej strony powyżej osi OX. A więc $$x \in (-\infty,-9) \cup (-1,8).$$
\rozwStop
\odpStart
$x \in (-\infty,-9) \cup (-1,8)$
\odpStop
\testStart
A.$x \in (-\infty,-9) \cup (-1,8)$\\
B.$x \in (-\infty,-9) \cup (-1,8]$\\
C.$x \in (-\infty,-9) \cup [-1,8)$\\
D.$x \in (-\infty,-9] \cup (-1,8)$\\
E.$x \in (-\infty,-9] \cup (-1,8]$\\
F.$x \in (-\infty,-9] \cup [-1,8)$\\
G.$x \in (-\infty,-9) \cup [-1,8]$\\
H.$x \in (-\infty,-9] \cup [-1,8]$
\testStop
\kluczStart
A
\kluczStop



\zadStart{Zadanie z Wikieł Z 1.62 b) moja wersja nr 65}

Rozwiązać nierówności $(x+10)(8-x)(x+1)\ge0$.
\zadStop
\rozwStart{Patryk Wirkus}{Laura Mieczkowska}
Miejsca zerowe naszego wielomianu to: $-10, 8, -1$.\\
Wielomian jest stopnia nieparzystego, ponadto znak współczynnika przy\linebreak najwyższej potędze x jest ujemny.\\ W związku z tym wykres wielomianu zaczyna się od lewej strony powyżej osi OX. A więc $$x \in (-\infty,-10) \cup (-1,8).$$
\rozwStop
\odpStart
$x \in (-\infty,-10) \cup (-1,8)$
\odpStop
\testStart
A.$x \in (-\infty,-10) \cup (-1,8)$\\
B.$x \in (-\infty,-10) \cup (-1,8]$\\
C.$x \in (-\infty,-10) \cup [-1,8)$\\
D.$x \in (-\infty,-10] \cup (-1,8)$\\
E.$x \in (-\infty,-10] \cup (-1,8]$\\
F.$x \in (-\infty,-10] \cup [-1,8)$\\
G.$x \in (-\infty,-10) \cup [-1,8]$\\
H.$x \in (-\infty,-10] \cup [-1,8]$
\testStop
\kluczStart
A
\kluczStop



\zadStart{Zadanie z Wikieł Z 1.62 b) moja wersja nr 66}

Rozwiązać nierówności $(x+11)(8-x)(x+1)\ge0$.
\zadStop
\rozwStart{Patryk Wirkus}{Laura Mieczkowska}
Miejsca zerowe naszego wielomianu to: $-11, 8, -1$.\\
Wielomian jest stopnia nieparzystego, ponadto znak współczynnika przy\linebreak najwyższej potędze x jest ujemny.\\ W związku z tym wykres wielomianu zaczyna się od lewej strony powyżej osi OX. A więc $$x \in (-\infty,-11) \cup (-1,8).$$
\rozwStop
\odpStart
$x \in (-\infty,-11) \cup (-1,8)$
\odpStop
\testStart
A.$x \in (-\infty,-11) \cup (-1,8)$\\
B.$x \in (-\infty,-11) \cup (-1,8]$\\
C.$x \in (-\infty,-11) \cup [-1,8)$\\
D.$x \in (-\infty,-11] \cup (-1,8)$\\
E.$x \in (-\infty,-11] \cup (-1,8]$\\
F.$x \in (-\infty,-11] \cup [-1,8)$\\
G.$x \in (-\infty,-11) \cup [-1,8]$\\
H.$x \in (-\infty,-11] \cup [-1,8]$
\testStop
\kluczStart
A
\kluczStop



\zadStart{Zadanie z Wikieł Z 1.62 b) moja wersja nr 67}

Rozwiązać nierówności $(x+12)(8-x)(x+1)\ge0$.
\zadStop
\rozwStart{Patryk Wirkus}{Laura Mieczkowska}
Miejsca zerowe naszego wielomianu to: $-12, 8, -1$.\\
Wielomian jest stopnia nieparzystego, ponadto znak współczynnika przy\linebreak najwyższej potędze x jest ujemny.\\ W związku z tym wykres wielomianu zaczyna się od lewej strony powyżej osi OX. A więc $$x \in (-\infty,-12) \cup (-1,8).$$
\rozwStop
\odpStart
$x \in (-\infty,-12) \cup (-1,8)$
\odpStop
\testStart
A.$x \in (-\infty,-12) \cup (-1,8)$\\
B.$x \in (-\infty,-12) \cup (-1,8]$\\
C.$x \in (-\infty,-12) \cup [-1,8)$\\
D.$x \in (-\infty,-12] \cup (-1,8)$\\
E.$x \in (-\infty,-12] \cup (-1,8]$\\
F.$x \in (-\infty,-12] \cup [-1,8)$\\
G.$x \in (-\infty,-12) \cup [-1,8]$\\
H.$x \in (-\infty,-12] \cup [-1,8]$
\testStop
\kluczStart
A
\kluczStop



\zadStart{Zadanie z Wikieł Z 1.62 b) moja wersja nr 68}

Rozwiązać nierówności $(x+13)(8-x)(x+1)\ge0$.
\zadStop
\rozwStart{Patryk Wirkus}{Laura Mieczkowska}
Miejsca zerowe naszego wielomianu to: $-13, 8, -1$.\\
Wielomian jest stopnia nieparzystego, ponadto znak współczynnika przy\linebreak najwyższej potędze x jest ujemny.\\ W związku z tym wykres wielomianu zaczyna się od lewej strony powyżej osi OX. A więc $$x \in (-\infty,-13) \cup (-1,8).$$
\rozwStop
\odpStart
$x \in (-\infty,-13) \cup (-1,8)$
\odpStop
\testStart
A.$x \in (-\infty,-13) \cup (-1,8)$\\
B.$x \in (-\infty,-13) \cup (-1,8]$\\
C.$x \in (-\infty,-13) \cup [-1,8)$\\
D.$x \in (-\infty,-13] \cup (-1,8)$\\
E.$x \in (-\infty,-13] \cup (-1,8]$\\
F.$x \in (-\infty,-13] \cup [-1,8)$\\
G.$x \in (-\infty,-13) \cup [-1,8]$\\
H.$x \in (-\infty,-13] \cup [-1,8]$
\testStop
\kluczStart
A
\kluczStop



\zadStart{Zadanie z Wikieł Z 1.62 b) moja wersja nr 69}

Rozwiązać nierówności $(x+14)(8-x)(x+1)\ge0$.
\zadStop
\rozwStart{Patryk Wirkus}{Laura Mieczkowska}
Miejsca zerowe naszego wielomianu to: $-14, 8, -1$.\\
Wielomian jest stopnia nieparzystego, ponadto znak współczynnika przy\linebreak najwyższej potędze x jest ujemny.\\ W związku z tym wykres wielomianu zaczyna się od lewej strony powyżej osi OX. A więc $$x \in (-\infty,-14) \cup (-1,8).$$
\rozwStop
\odpStart
$x \in (-\infty,-14) \cup (-1,8)$
\odpStop
\testStart
A.$x \in (-\infty,-14) \cup (-1,8)$\\
B.$x \in (-\infty,-14) \cup (-1,8]$\\
C.$x \in (-\infty,-14) \cup [-1,8)$\\
D.$x \in (-\infty,-14] \cup (-1,8)$\\
E.$x \in (-\infty,-14] \cup (-1,8]$\\
F.$x \in (-\infty,-14] \cup [-1,8)$\\
G.$x \in (-\infty,-14) \cup [-1,8]$\\
H.$x \in (-\infty,-14] \cup [-1,8]$
\testStop
\kluczStart
A
\kluczStop



\zadStart{Zadanie z Wikieł Z 1.62 b) moja wersja nr 70}

Rozwiązać nierówności $(x+15)(8-x)(x+1)\ge0$.
\zadStop
\rozwStart{Patryk Wirkus}{Laura Mieczkowska}
Miejsca zerowe naszego wielomianu to: $-15, 8, -1$.\\
Wielomian jest stopnia nieparzystego, ponadto znak współczynnika przy\linebreak najwyższej potędze x jest ujemny.\\ W związku z tym wykres wielomianu zaczyna się od lewej strony powyżej osi OX. A więc $$x \in (-\infty,-15) \cup (-1,8).$$
\rozwStop
\odpStart
$x \in (-\infty,-15) \cup (-1,8)$
\odpStop
\testStart
A.$x \in (-\infty,-15) \cup (-1,8)$\\
B.$x \in (-\infty,-15) \cup (-1,8]$\\
C.$x \in (-\infty,-15) \cup [-1,8)$\\
D.$x \in (-\infty,-15] \cup (-1,8)$\\
E.$x \in (-\infty,-15] \cup (-1,8]$\\
F.$x \in (-\infty,-15] \cup [-1,8)$\\
G.$x \in (-\infty,-15) \cup [-1,8]$\\
H.$x \in (-\infty,-15] \cup [-1,8]$
\testStop
\kluczStart
A
\kluczStop



\zadStart{Zadanie z Wikieł Z 1.62 b) moja wersja nr 71}

Rozwiązać nierówności $(x+10)(9-x)(x+1)\ge0$.
\zadStop
\rozwStart{Patryk Wirkus}{Laura Mieczkowska}
Miejsca zerowe naszego wielomianu to: $-10, 9, -1$.\\
Wielomian jest stopnia nieparzystego, ponadto znak współczynnika przy\linebreak najwyższej potędze x jest ujemny.\\ W związku z tym wykres wielomianu zaczyna się od lewej strony powyżej osi OX. A więc $$x \in (-\infty,-10) \cup (-1,9).$$
\rozwStop
\odpStart
$x \in (-\infty,-10) \cup (-1,9)$
\odpStop
\testStart
A.$x \in (-\infty,-10) \cup (-1,9)$\\
B.$x \in (-\infty,-10) \cup (-1,9]$\\
C.$x \in (-\infty,-10) \cup [-1,9)$\\
D.$x \in (-\infty,-10] \cup (-1,9)$\\
E.$x \in (-\infty,-10] \cup (-1,9]$\\
F.$x \in (-\infty,-10] \cup [-1,9)$\\
G.$x \in (-\infty,-10) \cup [-1,9]$\\
H.$x \in (-\infty,-10] \cup [-1,9]$
\testStop
\kluczStart
A
\kluczStop



\zadStart{Zadanie z Wikieł Z 1.62 b) moja wersja nr 72}

Rozwiązać nierówności $(x+11)(9-x)(x+1)\ge0$.
\zadStop
\rozwStart{Patryk Wirkus}{Laura Mieczkowska}
Miejsca zerowe naszego wielomianu to: $-11, 9, -1$.\\
Wielomian jest stopnia nieparzystego, ponadto znak współczynnika przy\linebreak najwyższej potędze x jest ujemny.\\ W związku z tym wykres wielomianu zaczyna się od lewej strony powyżej osi OX. A więc $$x \in (-\infty,-11) \cup (-1,9).$$
\rozwStop
\odpStart
$x \in (-\infty,-11) \cup (-1,9)$
\odpStop
\testStart
A.$x \in (-\infty,-11) \cup (-1,9)$\\
B.$x \in (-\infty,-11) \cup (-1,9]$\\
C.$x \in (-\infty,-11) \cup [-1,9)$\\
D.$x \in (-\infty,-11] \cup (-1,9)$\\
E.$x \in (-\infty,-11] \cup (-1,9]$\\
F.$x \in (-\infty,-11] \cup [-1,9)$\\
G.$x \in (-\infty,-11) \cup [-1,9]$\\
H.$x \in (-\infty,-11] \cup [-1,9]$
\testStop
\kluczStart
A
\kluczStop



\zadStart{Zadanie z Wikieł Z 1.62 b) moja wersja nr 73}

Rozwiązać nierówności $(x+12)(9-x)(x+1)\ge0$.
\zadStop
\rozwStart{Patryk Wirkus}{Laura Mieczkowska}
Miejsca zerowe naszego wielomianu to: $-12, 9, -1$.\\
Wielomian jest stopnia nieparzystego, ponadto znak współczynnika przy\linebreak najwyższej potędze x jest ujemny.\\ W związku z tym wykres wielomianu zaczyna się od lewej strony powyżej osi OX. A więc $$x \in (-\infty,-12) \cup (-1,9).$$
\rozwStop
\odpStart
$x \in (-\infty,-12) \cup (-1,9)$
\odpStop
\testStart
A.$x \in (-\infty,-12) \cup (-1,9)$\\
B.$x \in (-\infty,-12) \cup (-1,9]$\\
C.$x \in (-\infty,-12) \cup [-1,9)$\\
D.$x \in (-\infty,-12] \cup (-1,9)$\\
E.$x \in (-\infty,-12] \cup (-1,9]$\\
F.$x \in (-\infty,-12] \cup [-1,9)$\\
G.$x \in (-\infty,-12) \cup [-1,9]$\\
H.$x \in (-\infty,-12] \cup [-1,9]$
\testStop
\kluczStart
A
\kluczStop



\zadStart{Zadanie z Wikieł Z 1.62 b) moja wersja nr 74}

Rozwiązać nierówności $(x+13)(9-x)(x+1)\ge0$.
\zadStop
\rozwStart{Patryk Wirkus}{Laura Mieczkowska}
Miejsca zerowe naszego wielomianu to: $-13, 9, -1$.\\
Wielomian jest stopnia nieparzystego, ponadto znak współczynnika przy\linebreak najwyższej potędze x jest ujemny.\\ W związku z tym wykres wielomianu zaczyna się od lewej strony powyżej osi OX. A więc $$x \in (-\infty,-13) \cup (-1,9).$$
\rozwStop
\odpStart
$x \in (-\infty,-13) \cup (-1,9)$
\odpStop
\testStart
A.$x \in (-\infty,-13) \cup (-1,9)$\\
B.$x \in (-\infty,-13) \cup (-1,9]$\\
C.$x \in (-\infty,-13) \cup [-1,9)$\\
D.$x \in (-\infty,-13] \cup (-1,9)$\\
E.$x \in (-\infty,-13] \cup (-1,9]$\\
F.$x \in (-\infty,-13] \cup [-1,9)$\\
G.$x \in (-\infty,-13) \cup [-1,9]$\\
H.$x \in (-\infty,-13] \cup [-1,9]$
\testStop
\kluczStart
A
\kluczStop



\zadStart{Zadanie z Wikieł Z 1.62 b) moja wersja nr 75}

Rozwiązać nierówności $(x+14)(9-x)(x+1)\ge0$.
\zadStop
\rozwStart{Patryk Wirkus}{Laura Mieczkowska}
Miejsca zerowe naszego wielomianu to: $-14, 9, -1$.\\
Wielomian jest stopnia nieparzystego, ponadto znak współczynnika przy\linebreak najwyższej potędze x jest ujemny.\\ W związku z tym wykres wielomianu zaczyna się od lewej strony powyżej osi OX. A więc $$x \in (-\infty,-14) \cup (-1,9).$$
\rozwStop
\odpStart
$x \in (-\infty,-14) \cup (-1,9)$
\odpStop
\testStart
A.$x \in (-\infty,-14) \cup (-1,9)$\\
B.$x \in (-\infty,-14) \cup (-1,9]$\\
C.$x \in (-\infty,-14) \cup [-1,9)$\\
D.$x \in (-\infty,-14] \cup (-1,9)$\\
E.$x \in (-\infty,-14] \cup (-1,9]$\\
F.$x \in (-\infty,-14] \cup [-1,9)$\\
G.$x \in (-\infty,-14) \cup [-1,9]$\\
H.$x \in (-\infty,-14] \cup [-1,9]$
\testStop
\kluczStart
A
\kluczStop



\zadStart{Zadanie z Wikieł Z 1.62 b) moja wersja nr 76}

Rozwiązać nierówności $(x+15)(9-x)(x+1)\ge0$.
\zadStop
\rozwStart{Patryk Wirkus}{Laura Mieczkowska}
Miejsca zerowe naszego wielomianu to: $-15, 9, -1$.\\
Wielomian jest stopnia nieparzystego, ponadto znak współczynnika przy\linebreak najwyższej potędze x jest ujemny.\\ W związku z tym wykres wielomianu zaczyna się od lewej strony powyżej osi OX. A więc $$x \in (-\infty,-15) \cup (-1,9).$$
\rozwStop
\odpStart
$x \in (-\infty,-15) \cup (-1,9)$
\odpStop
\testStart
A.$x \in (-\infty,-15) \cup (-1,9)$\\
B.$x \in (-\infty,-15) \cup (-1,9]$\\
C.$x \in (-\infty,-15) \cup [-1,9)$\\
D.$x \in (-\infty,-15] \cup (-1,9)$\\
E.$x \in (-\infty,-15] \cup (-1,9]$\\
F.$x \in (-\infty,-15] \cup [-1,9)$\\
G.$x \in (-\infty,-15) \cup [-1,9]$\\
H.$x \in (-\infty,-15] \cup [-1,9]$
\testStop
\kluczStart
A
\kluczStop



\zadStart{Zadanie z Wikieł Z 1.62 b) moja wersja nr 77}

Rozwiązać nierówności $(x+11)(10-x)(x+1)\ge0$.
\zadStop
\rozwStart{Patryk Wirkus}{Laura Mieczkowska}
Miejsca zerowe naszego wielomianu to: $-11, 10, -1$.\\
Wielomian jest stopnia nieparzystego, ponadto znak współczynnika przy\linebreak najwyższej potędze x jest ujemny.\\ W związku z tym wykres wielomianu zaczyna się od lewej strony powyżej osi OX. A więc $$x \in (-\infty,-11) \cup (-1,10).$$
\rozwStop
\odpStart
$x \in (-\infty,-11) \cup (-1,10)$
\odpStop
\testStart
A.$x \in (-\infty,-11) \cup (-1,10)$\\
B.$x \in (-\infty,-11) \cup (-1,10]$\\
C.$x \in (-\infty,-11) \cup [-1,10)$\\
D.$x \in (-\infty,-11] \cup (-1,10)$\\
E.$x \in (-\infty,-11] \cup (-1,10]$\\
F.$x \in (-\infty,-11] \cup [-1,10)$\\
G.$x \in (-\infty,-11) \cup [-1,10]$\\
H.$x \in (-\infty,-11] \cup [-1,10]$
\testStop
\kluczStart
A
\kluczStop



\zadStart{Zadanie z Wikieł Z 1.62 b) moja wersja nr 78}

Rozwiązać nierówności $(x+12)(10-x)(x+1)\ge0$.
\zadStop
\rozwStart{Patryk Wirkus}{Laura Mieczkowska}
Miejsca zerowe naszego wielomianu to: $-12, 10, -1$.\\
Wielomian jest stopnia nieparzystego, ponadto znak współczynnika przy\linebreak najwyższej potędze x jest ujemny.\\ W związku z tym wykres wielomianu zaczyna się od lewej strony powyżej osi OX. A więc $$x \in (-\infty,-12) \cup (-1,10).$$
\rozwStop
\odpStart
$x \in (-\infty,-12) \cup (-1,10)$
\odpStop
\testStart
A.$x \in (-\infty,-12) \cup (-1,10)$\\
B.$x \in (-\infty,-12) \cup (-1,10]$\\
C.$x \in (-\infty,-12) \cup [-1,10)$\\
D.$x \in (-\infty,-12] \cup (-1,10)$\\
E.$x \in (-\infty,-12] \cup (-1,10]$\\
F.$x \in (-\infty,-12] \cup [-1,10)$\\
G.$x \in (-\infty,-12) \cup [-1,10]$\\
H.$x \in (-\infty,-12] \cup [-1,10]$
\testStop
\kluczStart
A
\kluczStop



\zadStart{Zadanie z Wikieł Z 1.62 b) moja wersja nr 79}

Rozwiązać nierówności $(x+13)(10-x)(x+1)\ge0$.
\zadStop
\rozwStart{Patryk Wirkus}{Laura Mieczkowska}
Miejsca zerowe naszego wielomianu to: $-13, 10, -1$.\\
Wielomian jest stopnia nieparzystego, ponadto znak współczynnika przy\linebreak najwyższej potędze x jest ujemny.\\ W związku z tym wykres wielomianu zaczyna się od lewej strony powyżej osi OX. A więc $$x \in (-\infty,-13) \cup (-1,10).$$
\rozwStop
\odpStart
$x \in (-\infty,-13) \cup (-1,10)$
\odpStop
\testStart
A.$x \in (-\infty,-13) \cup (-1,10)$\\
B.$x \in (-\infty,-13) \cup (-1,10]$\\
C.$x \in (-\infty,-13) \cup [-1,10)$\\
D.$x \in (-\infty,-13] \cup (-1,10)$\\
E.$x \in (-\infty,-13] \cup (-1,10]$\\
F.$x \in (-\infty,-13] \cup [-1,10)$\\
G.$x \in (-\infty,-13) \cup [-1,10]$\\
H.$x \in (-\infty,-13] \cup [-1,10]$
\testStop
\kluczStart
A
\kluczStop



\zadStart{Zadanie z Wikieł Z 1.62 b) moja wersja nr 80}

Rozwiązać nierówności $(x+14)(10-x)(x+1)\ge0$.
\zadStop
\rozwStart{Patryk Wirkus}{Laura Mieczkowska}
Miejsca zerowe naszego wielomianu to: $-14, 10, -1$.\\
Wielomian jest stopnia nieparzystego, ponadto znak współczynnika przy\linebreak najwyższej potędze x jest ujemny.\\ W związku z tym wykres wielomianu zaczyna się od lewej strony powyżej osi OX. A więc $$x \in (-\infty,-14) \cup (-1,10).$$
\rozwStop
\odpStart
$x \in (-\infty,-14) \cup (-1,10)$
\odpStop
\testStart
A.$x \in (-\infty,-14) \cup (-1,10)$\\
B.$x \in (-\infty,-14) \cup (-1,10]$\\
C.$x \in (-\infty,-14) \cup [-1,10)$\\
D.$x \in (-\infty,-14] \cup (-1,10)$\\
E.$x \in (-\infty,-14] \cup (-1,10]$\\
F.$x \in (-\infty,-14] \cup [-1,10)$\\
G.$x \in (-\infty,-14) \cup [-1,10]$\\
H.$x \in (-\infty,-14] \cup [-1,10]$
\testStop
\kluczStart
A
\kluczStop



\zadStart{Zadanie z Wikieł Z 1.62 b) moja wersja nr 81}

Rozwiązać nierówności $(x+15)(10-x)(x+1)\ge0$.
\zadStop
\rozwStart{Patryk Wirkus}{Laura Mieczkowska}
Miejsca zerowe naszego wielomianu to: $-15, 10, -1$.\\
Wielomian jest stopnia nieparzystego, ponadto znak współczynnika przy\linebreak najwyższej potędze x jest ujemny.\\ W związku z tym wykres wielomianu zaczyna się od lewej strony powyżej osi OX. A więc $$x \in (-\infty,-15) \cup (-1,10).$$
\rozwStop
\odpStart
$x \in (-\infty,-15) \cup (-1,10)$
\odpStop
\testStart
A.$x \in (-\infty,-15) \cup (-1,10)$\\
B.$x \in (-\infty,-15) \cup (-1,10]$\\
C.$x \in (-\infty,-15) \cup [-1,10)$\\
D.$x \in (-\infty,-15] \cup (-1,10)$\\
E.$x \in (-\infty,-15] \cup (-1,10]$\\
F.$x \in (-\infty,-15] \cup [-1,10)$\\
G.$x \in (-\infty,-15) \cup [-1,10]$\\
H.$x \in (-\infty,-15] \cup [-1,10]$
\testStop
\kluczStart
A
\kluczStop



\zadStart{Zadanie z Wikieł Z 1.62 b) moja wersja nr 82}

Rozwiązać nierówności $(x+4)(3-x)(x+2)\ge0$.
\zadStop
\rozwStart{Patryk Wirkus}{Laura Mieczkowska}
Miejsca zerowe naszego wielomianu to: $-4, 3, -2$.\\
Wielomian jest stopnia nieparzystego, ponadto znak współczynnika przy\linebreak najwyższej potędze x jest ujemny.\\ W związku z tym wykres wielomianu zaczyna się od lewej strony powyżej osi OX. A więc $$x \in (-\infty,-4) \cup (-2,3).$$
\rozwStop
\odpStart
$x \in (-\infty,-4) \cup (-2,3)$
\odpStop
\testStart
A.$x \in (-\infty,-4) \cup (-2,3)$\\
B.$x \in (-\infty,-4) \cup (-2,3]$\\
C.$x \in (-\infty,-4) \cup [-2,3)$\\
D.$x \in (-\infty,-4] \cup (-2,3)$\\
E.$x \in (-\infty,-4] \cup (-2,3]$\\
F.$x \in (-\infty,-4] \cup [-2,3)$\\
G.$x \in (-\infty,-4) \cup [-2,3]$\\
H.$x \in (-\infty,-4] \cup [-2,3]$
\testStop
\kluczStart
A
\kluczStop



\zadStart{Zadanie z Wikieł Z 1.62 b) moja wersja nr 83}

Rozwiązać nierówności $(x+5)(3-x)(x+2)\ge0$.
\zadStop
\rozwStart{Patryk Wirkus}{Laura Mieczkowska}
Miejsca zerowe naszego wielomianu to: $-5, 3, -2$.\\
Wielomian jest stopnia nieparzystego, ponadto znak współczynnika przy\linebreak najwyższej potędze x jest ujemny.\\ W związku z tym wykres wielomianu zaczyna się od lewej strony powyżej osi OX. A więc $$x \in (-\infty,-5) \cup (-2,3).$$
\rozwStop
\odpStart
$x \in (-\infty,-5) \cup (-2,3)$
\odpStop
\testStart
A.$x \in (-\infty,-5) \cup (-2,3)$\\
B.$x \in (-\infty,-5) \cup (-2,3]$\\
C.$x \in (-\infty,-5) \cup [-2,3)$\\
D.$x \in (-\infty,-5] \cup (-2,3)$\\
E.$x \in (-\infty,-5] \cup (-2,3]$\\
F.$x \in (-\infty,-5] \cup [-2,3)$\\
G.$x \in (-\infty,-5) \cup [-2,3]$\\
H.$x \in (-\infty,-5] \cup [-2,3]$
\testStop
\kluczStart
A
\kluczStop



\zadStart{Zadanie z Wikieł Z 1.62 b) moja wersja nr 84}

Rozwiązać nierówności $(x+6)(3-x)(x+2)\ge0$.
\zadStop
\rozwStart{Patryk Wirkus}{Laura Mieczkowska}
Miejsca zerowe naszego wielomianu to: $-6, 3, -2$.\\
Wielomian jest stopnia nieparzystego, ponadto znak współczynnika przy\linebreak najwyższej potędze x jest ujemny.\\ W związku z tym wykres wielomianu zaczyna się od lewej strony powyżej osi OX. A więc $$x \in (-\infty,-6) \cup (-2,3).$$
\rozwStop
\odpStart
$x \in (-\infty,-6) \cup (-2,3)$
\odpStop
\testStart
A.$x \in (-\infty,-6) \cup (-2,3)$\\
B.$x \in (-\infty,-6) \cup (-2,3]$\\
C.$x \in (-\infty,-6) \cup [-2,3)$\\
D.$x \in (-\infty,-6] \cup (-2,3)$\\
E.$x \in (-\infty,-6] \cup (-2,3]$\\
F.$x \in (-\infty,-6] \cup [-2,3)$\\
G.$x \in (-\infty,-6) \cup [-2,3]$\\
H.$x \in (-\infty,-6] \cup [-2,3]$
\testStop
\kluczStart
A
\kluczStop



\zadStart{Zadanie z Wikieł Z 1.62 b) moja wersja nr 85}

Rozwiązać nierówności $(x+7)(3-x)(x+2)\ge0$.
\zadStop
\rozwStart{Patryk Wirkus}{Laura Mieczkowska}
Miejsca zerowe naszego wielomianu to: $-7, 3, -2$.\\
Wielomian jest stopnia nieparzystego, ponadto znak współczynnika przy\linebreak najwyższej potędze x jest ujemny.\\ W związku z tym wykres wielomianu zaczyna się od lewej strony powyżej osi OX. A więc $$x \in (-\infty,-7) \cup (-2,3).$$
\rozwStop
\odpStart
$x \in (-\infty,-7) \cup (-2,3)$
\odpStop
\testStart
A.$x \in (-\infty,-7) \cup (-2,3)$\\
B.$x \in (-\infty,-7) \cup (-2,3]$\\
C.$x \in (-\infty,-7) \cup [-2,3)$\\
D.$x \in (-\infty,-7] \cup (-2,3)$\\
E.$x \in (-\infty,-7] \cup (-2,3]$\\
F.$x \in (-\infty,-7] \cup [-2,3)$\\
G.$x \in (-\infty,-7) \cup [-2,3]$\\
H.$x \in (-\infty,-7] \cup [-2,3]$
\testStop
\kluczStart
A
\kluczStop



\zadStart{Zadanie z Wikieł Z 1.62 b) moja wersja nr 86}

Rozwiązać nierówności $(x+8)(3-x)(x+2)\ge0$.
\zadStop
\rozwStart{Patryk Wirkus}{Laura Mieczkowska}
Miejsca zerowe naszego wielomianu to: $-8, 3, -2$.\\
Wielomian jest stopnia nieparzystego, ponadto znak współczynnika przy\linebreak najwyższej potędze x jest ujemny.\\ W związku z tym wykres wielomianu zaczyna się od lewej strony powyżej osi OX. A więc $$x \in (-\infty,-8) \cup (-2,3).$$
\rozwStop
\odpStart
$x \in (-\infty,-8) \cup (-2,3)$
\odpStop
\testStart
A.$x \in (-\infty,-8) \cup (-2,3)$\\
B.$x \in (-\infty,-8) \cup (-2,3]$\\
C.$x \in (-\infty,-8) \cup [-2,3)$\\
D.$x \in (-\infty,-8] \cup (-2,3)$\\
E.$x \in (-\infty,-8] \cup (-2,3]$\\
F.$x \in (-\infty,-8] \cup [-2,3)$\\
G.$x \in (-\infty,-8) \cup [-2,3]$\\
H.$x \in (-\infty,-8] \cup [-2,3]$
\testStop
\kluczStart
A
\kluczStop



\zadStart{Zadanie z Wikieł Z 1.62 b) moja wersja nr 87}

Rozwiązać nierówności $(x+9)(3-x)(x+2)\ge0$.
\zadStop
\rozwStart{Patryk Wirkus}{Laura Mieczkowska}
Miejsca zerowe naszego wielomianu to: $-9, 3, -2$.\\
Wielomian jest stopnia nieparzystego, ponadto znak współczynnika przy\linebreak najwyższej potędze x jest ujemny.\\ W związku z tym wykres wielomianu zaczyna się od lewej strony powyżej osi OX. A więc $$x \in (-\infty,-9) \cup (-2,3).$$
\rozwStop
\odpStart
$x \in (-\infty,-9) \cup (-2,3)$
\odpStop
\testStart
A.$x \in (-\infty,-9) \cup (-2,3)$\\
B.$x \in (-\infty,-9) \cup (-2,3]$\\
C.$x \in (-\infty,-9) \cup [-2,3)$\\
D.$x \in (-\infty,-9] \cup (-2,3)$\\
E.$x \in (-\infty,-9] \cup (-2,3]$\\
F.$x \in (-\infty,-9] \cup [-2,3)$\\
G.$x \in (-\infty,-9) \cup [-2,3]$\\
H.$x \in (-\infty,-9] \cup [-2,3]$
\testStop
\kluczStart
A
\kluczStop



\zadStart{Zadanie z Wikieł Z 1.62 b) moja wersja nr 88}

Rozwiązać nierówności $(x+10)(3-x)(x+2)\ge0$.
\zadStop
\rozwStart{Patryk Wirkus}{Laura Mieczkowska}
Miejsca zerowe naszego wielomianu to: $-10, 3, -2$.\\
Wielomian jest stopnia nieparzystego, ponadto znak współczynnika przy\linebreak najwyższej potędze x jest ujemny.\\ W związku z tym wykres wielomianu zaczyna się od lewej strony powyżej osi OX. A więc $$x \in (-\infty,-10) \cup (-2,3).$$
\rozwStop
\odpStart
$x \in (-\infty,-10) \cup (-2,3)$
\odpStop
\testStart
A.$x \in (-\infty,-10) \cup (-2,3)$\\
B.$x \in (-\infty,-10) \cup (-2,3]$\\
C.$x \in (-\infty,-10) \cup [-2,3)$\\
D.$x \in (-\infty,-10] \cup (-2,3)$\\
E.$x \in (-\infty,-10] \cup (-2,3]$\\
F.$x \in (-\infty,-10] \cup [-2,3)$\\
G.$x \in (-\infty,-10) \cup [-2,3]$\\
H.$x \in (-\infty,-10] \cup [-2,3]$
\testStop
\kluczStart
A
\kluczStop



\zadStart{Zadanie z Wikieł Z 1.62 b) moja wersja nr 89}

Rozwiązać nierówności $(x+11)(3-x)(x+2)\ge0$.
\zadStop
\rozwStart{Patryk Wirkus}{Laura Mieczkowska}
Miejsca zerowe naszego wielomianu to: $-11, 3, -2$.\\
Wielomian jest stopnia nieparzystego, ponadto znak współczynnika przy\linebreak najwyższej potędze x jest ujemny.\\ W związku z tym wykres wielomianu zaczyna się od lewej strony powyżej osi OX. A więc $$x \in (-\infty,-11) \cup (-2,3).$$
\rozwStop
\odpStart
$x \in (-\infty,-11) \cup (-2,3)$
\odpStop
\testStart
A.$x \in (-\infty,-11) \cup (-2,3)$\\
B.$x \in (-\infty,-11) \cup (-2,3]$\\
C.$x \in (-\infty,-11) \cup [-2,3)$\\
D.$x \in (-\infty,-11] \cup (-2,3)$\\
E.$x \in (-\infty,-11] \cup (-2,3]$\\
F.$x \in (-\infty,-11] \cup [-2,3)$\\
G.$x \in (-\infty,-11) \cup [-2,3]$\\
H.$x \in (-\infty,-11] \cup [-2,3]$
\testStop
\kluczStart
A
\kluczStop



\zadStart{Zadanie z Wikieł Z 1.62 b) moja wersja nr 90}

Rozwiązać nierówności $(x+12)(3-x)(x+2)\ge0$.
\zadStop
\rozwStart{Patryk Wirkus}{Laura Mieczkowska}
Miejsca zerowe naszego wielomianu to: $-12, 3, -2$.\\
Wielomian jest stopnia nieparzystego, ponadto znak współczynnika przy\linebreak najwyższej potędze x jest ujemny.\\ W związku z tym wykres wielomianu zaczyna się od lewej strony powyżej osi OX. A więc $$x \in (-\infty,-12) \cup (-2,3).$$
\rozwStop
\odpStart
$x \in (-\infty,-12) \cup (-2,3)$
\odpStop
\testStart
A.$x \in (-\infty,-12) \cup (-2,3)$\\
B.$x \in (-\infty,-12) \cup (-2,3]$\\
C.$x \in (-\infty,-12) \cup [-2,3)$\\
D.$x \in (-\infty,-12] \cup (-2,3)$\\
E.$x \in (-\infty,-12] \cup (-2,3]$\\
F.$x \in (-\infty,-12] \cup [-2,3)$\\
G.$x \in (-\infty,-12) \cup [-2,3]$\\
H.$x \in (-\infty,-12] \cup [-2,3]$
\testStop
\kluczStart
A
\kluczStop



\zadStart{Zadanie z Wikieł Z 1.62 b) moja wersja nr 91}

Rozwiązać nierówności $(x+13)(3-x)(x+2)\ge0$.
\zadStop
\rozwStart{Patryk Wirkus}{Laura Mieczkowska}
Miejsca zerowe naszego wielomianu to: $-13, 3, -2$.\\
Wielomian jest stopnia nieparzystego, ponadto znak współczynnika przy\linebreak najwyższej potędze x jest ujemny.\\ W związku z tym wykres wielomianu zaczyna się od lewej strony powyżej osi OX. A więc $$x \in (-\infty,-13) \cup (-2,3).$$
\rozwStop
\odpStart
$x \in (-\infty,-13) \cup (-2,3)$
\odpStop
\testStart
A.$x \in (-\infty,-13) \cup (-2,3)$\\
B.$x \in (-\infty,-13) \cup (-2,3]$\\
C.$x \in (-\infty,-13) \cup [-2,3)$\\
D.$x \in (-\infty,-13] \cup (-2,3)$\\
E.$x \in (-\infty,-13] \cup (-2,3]$\\
F.$x \in (-\infty,-13] \cup [-2,3)$\\
G.$x \in (-\infty,-13) \cup [-2,3]$\\
H.$x \in (-\infty,-13] \cup [-2,3]$
\testStop
\kluczStart
A
\kluczStop



\zadStart{Zadanie z Wikieł Z 1.62 b) moja wersja nr 92}

Rozwiązać nierówności $(x+14)(3-x)(x+2)\ge0$.
\zadStop
\rozwStart{Patryk Wirkus}{Laura Mieczkowska}
Miejsca zerowe naszego wielomianu to: $-14, 3, -2$.\\
Wielomian jest stopnia nieparzystego, ponadto znak współczynnika przy\linebreak najwyższej potędze x jest ujemny.\\ W związku z tym wykres wielomianu zaczyna się od lewej strony powyżej osi OX. A więc $$x \in (-\infty,-14) \cup (-2,3).$$
\rozwStop
\odpStart
$x \in (-\infty,-14) \cup (-2,3)$
\odpStop
\testStart
A.$x \in (-\infty,-14) \cup (-2,3)$\\
B.$x \in (-\infty,-14) \cup (-2,3]$\\
C.$x \in (-\infty,-14) \cup [-2,3)$\\
D.$x \in (-\infty,-14] \cup (-2,3)$\\
E.$x \in (-\infty,-14] \cup (-2,3]$\\
F.$x \in (-\infty,-14] \cup [-2,3)$\\
G.$x \in (-\infty,-14) \cup [-2,3]$\\
H.$x \in (-\infty,-14] \cup [-2,3]$
\testStop
\kluczStart
A
\kluczStop



\zadStart{Zadanie z Wikieł Z 1.62 b) moja wersja nr 93}

Rozwiązać nierówności $(x+15)(3-x)(x+2)\ge0$.
\zadStop
\rozwStart{Patryk Wirkus}{Laura Mieczkowska}
Miejsca zerowe naszego wielomianu to: $-15, 3, -2$.\\
Wielomian jest stopnia nieparzystego, ponadto znak współczynnika przy\linebreak najwyższej potędze x jest ujemny.\\ W związku z tym wykres wielomianu zaczyna się od lewej strony powyżej osi OX. A więc $$x \in (-\infty,-15) \cup (-2,3).$$
\rozwStop
\odpStart
$x \in (-\infty,-15) \cup (-2,3)$
\odpStop
\testStart
A.$x \in (-\infty,-15) \cup (-2,3)$\\
B.$x \in (-\infty,-15) \cup (-2,3]$\\
C.$x \in (-\infty,-15) \cup [-2,3)$\\
D.$x \in (-\infty,-15] \cup (-2,3)$\\
E.$x \in (-\infty,-15] \cup (-2,3]$\\
F.$x \in (-\infty,-15] \cup [-2,3)$\\
G.$x \in (-\infty,-15) \cup [-2,3]$\\
H.$x \in (-\infty,-15] \cup [-2,3]$
\testStop
\kluczStart
A
\kluczStop



\zadStart{Zadanie z Wikieł Z 1.62 b) moja wersja nr 94}

Rozwiązać nierówności $(x+5)(4-x)(x+2)\ge0$.
\zadStop
\rozwStart{Patryk Wirkus}{Laura Mieczkowska}
Miejsca zerowe naszego wielomianu to: $-5, 4, -2$.\\
Wielomian jest stopnia nieparzystego, ponadto znak współczynnika przy\linebreak najwyższej potędze x jest ujemny.\\ W związku z tym wykres wielomianu zaczyna się od lewej strony powyżej osi OX. A więc $$x \in (-\infty,-5) \cup (-2,4).$$
\rozwStop
\odpStart
$x \in (-\infty,-5) \cup (-2,4)$
\odpStop
\testStart
A.$x \in (-\infty,-5) \cup (-2,4)$\\
B.$x \in (-\infty,-5) \cup (-2,4]$\\
C.$x \in (-\infty,-5) \cup [-2,4)$\\
D.$x \in (-\infty,-5] \cup (-2,4)$\\
E.$x \in (-\infty,-5] \cup (-2,4]$\\
F.$x \in (-\infty,-5] \cup [-2,4)$\\
G.$x \in (-\infty,-5) \cup [-2,4]$\\
H.$x \in (-\infty,-5] \cup [-2,4]$
\testStop
\kluczStart
A
\kluczStop



\zadStart{Zadanie z Wikieł Z 1.62 b) moja wersja nr 95}

Rozwiązać nierówności $(x+6)(4-x)(x+2)\ge0$.
\zadStop
\rozwStart{Patryk Wirkus}{Laura Mieczkowska}
Miejsca zerowe naszego wielomianu to: $-6, 4, -2$.\\
Wielomian jest stopnia nieparzystego, ponadto znak współczynnika przy\linebreak najwyższej potędze x jest ujemny.\\ W związku z tym wykres wielomianu zaczyna się od lewej strony powyżej osi OX. A więc $$x \in (-\infty,-6) \cup (-2,4).$$
\rozwStop
\odpStart
$x \in (-\infty,-6) \cup (-2,4)$
\odpStop
\testStart
A.$x \in (-\infty,-6) \cup (-2,4)$\\
B.$x \in (-\infty,-6) \cup (-2,4]$\\
C.$x \in (-\infty,-6) \cup [-2,4)$\\
D.$x \in (-\infty,-6] \cup (-2,4)$\\
E.$x \in (-\infty,-6] \cup (-2,4]$\\
F.$x \in (-\infty,-6] \cup [-2,4)$\\
G.$x \in (-\infty,-6) \cup [-2,4]$\\
H.$x \in (-\infty,-6] \cup [-2,4]$
\testStop
\kluczStart
A
\kluczStop



\zadStart{Zadanie z Wikieł Z 1.62 b) moja wersja nr 96}

Rozwiązać nierówności $(x+7)(4-x)(x+2)\ge0$.
\zadStop
\rozwStart{Patryk Wirkus}{Laura Mieczkowska}
Miejsca zerowe naszego wielomianu to: $-7, 4, -2$.\\
Wielomian jest stopnia nieparzystego, ponadto znak współczynnika przy\linebreak najwyższej potędze x jest ujemny.\\ W związku z tym wykres wielomianu zaczyna się od lewej strony powyżej osi OX. A więc $$x \in (-\infty,-7) \cup (-2,4).$$
\rozwStop
\odpStart
$x \in (-\infty,-7) \cup (-2,4)$
\odpStop
\testStart
A.$x \in (-\infty,-7) \cup (-2,4)$\\
B.$x \in (-\infty,-7) \cup (-2,4]$\\
C.$x \in (-\infty,-7) \cup [-2,4)$\\
D.$x \in (-\infty,-7] \cup (-2,4)$\\
E.$x \in (-\infty,-7] \cup (-2,4]$\\
F.$x \in (-\infty,-7] \cup [-2,4)$\\
G.$x \in (-\infty,-7) \cup [-2,4]$\\
H.$x \in (-\infty,-7] \cup [-2,4]$
\testStop
\kluczStart
A
\kluczStop



\zadStart{Zadanie z Wikieł Z 1.62 b) moja wersja nr 97}

Rozwiązać nierówności $(x+8)(4-x)(x+2)\ge0$.
\zadStop
\rozwStart{Patryk Wirkus}{Laura Mieczkowska}
Miejsca zerowe naszego wielomianu to: $-8, 4, -2$.\\
Wielomian jest stopnia nieparzystego, ponadto znak współczynnika przy\linebreak najwyższej potędze x jest ujemny.\\ W związku z tym wykres wielomianu zaczyna się od lewej strony powyżej osi OX. A więc $$x \in (-\infty,-8) \cup (-2,4).$$
\rozwStop
\odpStart
$x \in (-\infty,-8) \cup (-2,4)$
\odpStop
\testStart
A.$x \in (-\infty,-8) \cup (-2,4)$\\
B.$x \in (-\infty,-8) \cup (-2,4]$\\
C.$x \in (-\infty,-8) \cup [-2,4)$\\
D.$x \in (-\infty,-8] \cup (-2,4)$\\
E.$x \in (-\infty,-8] \cup (-2,4]$\\
F.$x \in (-\infty,-8] \cup [-2,4)$\\
G.$x \in (-\infty,-8) \cup [-2,4]$\\
H.$x \in (-\infty,-8] \cup [-2,4]$
\testStop
\kluczStart
A
\kluczStop



\zadStart{Zadanie z Wikieł Z 1.62 b) moja wersja nr 98}

Rozwiązać nierówności $(x+9)(4-x)(x+2)\ge0$.
\zadStop
\rozwStart{Patryk Wirkus}{Laura Mieczkowska}
Miejsca zerowe naszego wielomianu to: $-9, 4, -2$.\\
Wielomian jest stopnia nieparzystego, ponadto znak współczynnika przy\linebreak najwyższej potędze x jest ujemny.\\ W związku z tym wykres wielomianu zaczyna się od lewej strony powyżej osi OX. A więc $$x \in (-\infty,-9) \cup (-2,4).$$
\rozwStop
\odpStart
$x \in (-\infty,-9) \cup (-2,4)$
\odpStop
\testStart
A.$x \in (-\infty,-9) \cup (-2,4)$\\
B.$x \in (-\infty,-9) \cup (-2,4]$\\
C.$x \in (-\infty,-9) \cup [-2,4)$\\
D.$x \in (-\infty,-9] \cup (-2,4)$\\
E.$x \in (-\infty,-9] \cup (-2,4]$\\
F.$x \in (-\infty,-9] \cup [-2,4)$\\
G.$x \in (-\infty,-9) \cup [-2,4]$\\
H.$x \in (-\infty,-9] \cup [-2,4]$
\testStop
\kluczStart
A
\kluczStop



\zadStart{Zadanie z Wikieł Z 1.62 b) moja wersja nr 99}

Rozwiązać nierówności $(x+10)(4-x)(x+2)\ge0$.
\zadStop
\rozwStart{Patryk Wirkus}{Laura Mieczkowska}
Miejsca zerowe naszego wielomianu to: $-10, 4, -2$.\\
Wielomian jest stopnia nieparzystego, ponadto znak współczynnika przy\linebreak najwyższej potędze x jest ujemny.\\ W związku z tym wykres wielomianu zaczyna się od lewej strony powyżej osi OX. A więc $$x \in (-\infty,-10) \cup (-2,4).$$
\rozwStop
\odpStart
$x \in (-\infty,-10) \cup (-2,4)$
\odpStop
\testStart
A.$x \in (-\infty,-10) \cup (-2,4)$\\
B.$x \in (-\infty,-10) \cup (-2,4]$\\
C.$x \in (-\infty,-10) \cup [-2,4)$\\
D.$x \in (-\infty,-10] \cup (-2,4)$\\
E.$x \in (-\infty,-10] \cup (-2,4]$\\
F.$x \in (-\infty,-10] \cup [-2,4)$\\
G.$x \in (-\infty,-10) \cup [-2,4]$\\
H.$x \in (-\infty,-10] \cup [-2,4]$
\testStop
\kluczStart
A
\kluczStop



\zadStart{Zadanie z Wikieł Z 1.62 b) moja wersja nr 100}

Rozwiązać nierówności $(x+11)(4-x)(x+2)\ge0$.
\zadStop
\rozwStart{Patryk Wirkus}{Laura Mieczkowska}
Miejsca zerowe naszego wielomianu to: $-11, 4, -2$.\\
Wielomian jest stopnia nieparzystego, ponadto znak współczynnika przy\linebreak najwyższej potędze x jest ujemny.\\ W związku z tym wykres wielomianu zaczyna się od lewej strony powyżej osi OX. A więc $$x \in (-\infty,-11) \cup (-2,4).$$
\rozwStop
\odpStart
$x \in (-\infty,-11) \cup (-2,4)$
\odpStop
\testStart
A.$x \in (-\infty,-11) \cup (-2,4)$\\
B.$x \in (-\infty,-11) \cup (-2,4]$\\
C.$x \in (-\infty,-11) \cup [-2,4)$\\
D.$x \in (-\infty,-11] \cup (-2,4)$\\
E.$x \in (-\infty,-11] \cup (-2,4]$\\
F.$x \in (-\infty,-11] \cup [-2,4)$\\
G.$x \in (-\infty,-11) \cup [-2,4]$\\
H.$x \in (-\infty,-11] \cup [-2,4]$
\testStop
\kluczStart
A
\kluczStop



\zadStart{Zadanie z Wikieł Z 1.62 b) moja wersja nr 101}

Rozwiązać nierówności $(x+12)(4-x)(x+2)\ge0$.
\zadStop
\rozwStart{Patryk Wirkus}{Laura Mieczkowska}
Miejsca zerowe naszego wielomianu to: $-12, 4, -2$.\\
Wielomian jest stopnia nieparzystego, ponadto znak współczynnika przy\linebreak najwyższej potędze x jest ujemny.\\ W związku z tym wykres wielomianu zaczyna się od lewej strony powyżej osi OX. A więc $$x \in (-\infty,-12) \cup (-2,4).$$
\rozwStop
\odpStart
$x \in (-\infty,-12) \cup (-2,4)$
\odpStop
\testStart
A.$x \in (-\infty,-12) \cup (-2,4)$\\
B.$x \in (-\infty,-12) \cup (-2,4]$\\
C.$x \in (-\infty,-12) \cup [-2,4)$\\
D.$x \in (-\infty,-12] \cup (-2,4)$\\
E.$x \in (-\infty,-12] \cup (-2,4]$\\
F.$x \in (-\infty,-12] \cup [-2,4)$\\
G.$x \in (-\infty,-12) \cup [-2,4]$\\
H.$x \in (-\infty,-12] \cup [-2,4]$
\testStop
\kluczStart
A
\kluczStop



\zadStart{Zadanie z Wikieł Z 1.62 b) moja wersja nr 102}

Rozwiązać nierówności $(x+13)(4-x)(x+2)\ge0$.
\zadStop
\rozwStart{Patryk Wirkus}{Laura Mieczkowska}
Miejsca zerowe naszego wielomianu to: $-13, 4, -2$.\\
Wielomian jest stopnia nieparzystego, ponadto znak współczynnika przy\linebreak najwyższej potędze x jest ujemny.\\ W związku z tym wykres wielomianu zaczyna się od lewej strony powyżej osi OX. A więc $$x \in (-\infty,-13) \cup (-2,4).$$
\rozwStop
\odpStart
$x \in (-\infty,-13) \cup (-2,4)$
\odpStop
\testStart
A.$x \in (-\infty,-13) \cup (-2,4)$\\
B.$x \in (-\infty,-13) \cup (-2,4]$\\
C.$x \in (-\infty,-13) \cup [-2,4)$\\
D.$x \in (-\infty,-13] \cup (-2,4)$\\
E.$x \in (-\infty,-13] \cup (-2,4]$\\
F.$x \in (-\infty,-13] \cup [-2,4)$\\
G.$x \in (-\infty,-13) \cup [-2,4]$\\
H.$x \in (-\infty,-13] \cup [-2,4]$
\testStop
\kluczStart
A
\kluczStop



\zadStart{Zadanie z Wikieł Z 1.62 b) moja wersja nr 103}

Rozwiązać nierówności $(x+14)(4-x)(x+2)\ge0$.
\zadStop
\rozwStart{Patryk Wirkus}{Laura Mieczkowska}
Miejsca zerowe naszego wielomianu to: $-14, 4, -2$.\\
Wielomian jest stopnia nieparzystego, ponadto znak współczynnika przy\linebreak najwyższej potędze x jest ujemny.\\ W związku z tym wykres wielomianu zaczyna się od lewej strony powyżej osi OX. A więc $$x \in (-\infty,-14) \cup (-2,4).$$
\rozwStop
\odpStart
$x \in (-\infty,-14) \cup (-2,4)$
\odpStop
\testStart
A.$x \in (-\infty,-14) \cup (-2,4)$\\
B.$x \in (-\infty,-14) \cup (-2,4]$\\
C.$x \in (-\infty,-14) \cup [-2,4)$\\
D.$x \in (-\infty,-14] \cup (-2,4)$\\
E.$x \in (-\infty,-14] \cup (-2,4]$\\
F.$x \in (-\infty,-14] \cup [-2,4)$\\
G.$x \in (-\infty,-14) \cup [-2,4]$\\
H.$x \in (-\infty,-14] \cup [-2,4]$
\testStop
\kluczStart
A
\kluczStop



\zadStart{Zadanie z Wikieł Z 1.62 b) moja wersja nr 104}

Rozwiązać nierówności $(x+15)(4-x)(x+2)\ge0$.
\zadStop
\rozwStart{Patryk Wirkus}{Laura Mieczkowska}
Miejsca zerowe naszego wielomianu to: $-15, 4, -2$.\\
Wielomian jest stopnia nieparzystego, ponadto znak współczynnika przy\linebreak najwyższej potędze x jest ujemny.\\ W związku z tym wykres wielomianu zaczyna się od lewej strony powyżej osi OX. A więc $$x \in (-\infty,-15) \cup (-2,4).$$
\rozwStop
\odpStart
$x \in (-\infty,-15) \cup (-2,4)$
\odpStop
\testStart
A.$x \in (-\infty,-15) \cup (-2,4)$\\
B.$x \in (-\infty,-15) \cup (-2,4]$\\
C.$x \in (-\infty,-15) \cup [-2,4)$\\
D.$x \in (-\infty,-15] \cup (-2,4)$\\
E.$x \in (-\infty,-15] \cup (-2,4]$\\
F.$x \in (-\infty,-15] \cup [-2,4)$\\
G.$x \in (-\infty,-15) \cup [-2,4]$\\
H.$x \in (-\infty,-15] \cup [-2,4]$
\testStop
\kluczStart
A
\kluczStop



\zadStart{Zadanie z Wikieł Z 1.62 b) moja wersja nr 105}

Rozwiązać nierówności $(x+6)(5-x)(x+2)\ge0$.
\zadStop
\rozwStart{Patryk Wirkus}{Laura Mieczkowska}
Miejsca zerowe naszego wielomianu to: $-6, 5, -2$.\\
Wielomian jest stopnia nieparzystego, ponadto znak współczynnika przy\linebreak najwyższej potędze x jest ujemny.\\ W związku z tym wykres wielomianu zaczyna się od lewej strony powyżej osi OX. A więc $$x \in (-\infty,-6) \cup (-2,5).$$
\rozwStop
\odpStart
$x \in (-\infty,-6) \cup (-2,5)$
\odpStop
\testStart
A.$x \in (-\infty,-6) \cup (-2,5)$\\
B.$x \in (-\infty,-6) \cup (-2,5]$\\
C.$x \in (-\infty,-6) \cup [-2,5)$\\
D.$x \in (-\infty,-6] \cup (-2,5)$\\
E.$x \in (-\infty,-6] \cup (-2,5]$\\
F.$x \in (-\infty,-6] \cup [-2,5)$\\
G.$x \in (-\infty,-6) \cup [-2,5]$\\
H.$x \in (-\infty,-6] \cup [-2,5]$
\testStop
\kluczStart
A
\kluczStop



\zadStart{Zadanie z Wikieł Z 1.62 b) moja wersja nr 106}

Rozwiązać nierówności $(x+7)(5-x)(x+2)\ge0$.
\zadStop
\rozwStart{Patryk Wirkus}{Laura Mieczkowska}
Miejsca zerowe naszego wielomianu to: $-7, 5, -2$.\\
Wielomian jest stopnia nieparzystego, ponadto znak współczynnika przy\linebreak najwyższej potędze x jest ujemny.\\ W związku z tym wykres wielomianu zaczyna się od lewej strony powyżej osi OX. A więc $$x \in (-\infty,-7) \cup (-2,5).$$
\rozwStop
\odpStart
$x \in (-\infty,-7) \cup (-2,5)$
\odpStop
\testStart
A.$x \in (-\infty,-7) \cup (-2,5)$\\
B.$x \in (-\infty,-7) \cup (-2,5]$\\
C.$x \in (-\infty,-7) \cup [-2,5)$\\
D.$x \in (-\infty,-7] \cup (-2,5)$\\
E.$x \in (-\infty,-7] \cup (-2,5]$\\
F.$x \in (-\infty,-7] \cup [-2,5)$\\
G.$x \in (-\infty,-7) \cup [-2,5]$\\
H.$x \in (-\infty,-7] \cup [-2,5]$
\testStop
\kluczStart
A
\kluczStop



\zadStart{Zadanie z Wikieł Z 1.62 b) moja wersja nr 107}

Rozwiązać nierówności $(x+8)(5-x)(x+2)\ge0$.
\zadStop
\rozwStart{Patryk Wirkus}{Laura Mieczkowska}
Miejsca zerowe naszego wielomianu to: $-8, 5, -2$.\\
Wielomian jest stopnia nieparzystego, ponadto znak współczynnika przy\linebreak najwyższej potędze x jest ujemny.\\ W związku z tym wykres wielomianu zaczyna się od lewej strony powyżej osi OX. A więc $$x \in (-\infty,-8) \cup (-2,5).$$
\rozwStop
\odpStart
$x \in (-\infty,-8) \cup (-2,5)$
\odpStop
\testStart
A.$x \in (-\infty,-8) \cup (-2,5)$\\
B.$x \in (-\infty,-8) \cup (-2,5]$\\
C.$x \in (-\infty,-8) \cup [-2,5)$\\
D.$x \in (-\infty,-8] \cup (-2,5)$\\
E.$x \in (-\infty,-8] \cup (-2,5]$\\
F.$x \in (-\infty,-8] \cup [-2,5)$\\
G.$x \in (-\infty,-8) \cup [-2,5]$\\
H.$x \in (-\infty,-8] \cup [-2,5]$
\testStop
\kluczStart
A
\kluczStop



\zadStart{Zadanie z Wikieł Z 1.62 b) moja wersja nr 108}

Rozwiązać nierówności $(x+9)(5-x)(x+2)\ge0$.
\zadStop
\rozwStart{Patryk Wirkus}{Laura Mieczkowska}
Miejsca zerowe naszego wielomianu to: $-9, 5, -2$.\\
Wielomian jest stopnia nieparzystego, ponadto znak współczynnika przy\linebreak najwyższej potędze x jest ujemny.\\ W związku z tym wykres wielomianu zaczyna się od lewej strony powyżej osi OX. A więc $$x \in (-\infty,-9) \cup (-2,5).$$
\rozwStop
\odpStart
$x \in (-\infty,-9) \cup (-2,5)$
\odpStop
\testStart
A.$x \in (-\infty,-9) \cup (-2,5)$\\
B.$x \in (-\infty,-9) \cup (-2,5]$\\
C.$x \in (-\infty,-9) \cup [-2,5)$\\
D.$x \in (-\infty,-9] \cup (-2,5)$\\
E.$x \in (-\infty,-9] \cup (-2,5]$\\
F.$x \in (-\infty,-9] \cup [-2,5)$\\
G.$x \in (-\infty,-9) \cup [-2,5]$\\
H.$x \in (-\infty,-9] \cup [-2,5]$
\testStop
\kluczStart
A
\kluczStop



\zadStart{Zadanie z Wikieł Z 1.62 b) moja wersja nr 109}

Rozwiązać nierówności $(x+10)(5-x)(x+2)\ge0$.
\zadStop
\rozwStart{Patryk Wirkus}{Laura Mieczkowska}
Miejsca zerowe naszego wielomianu to: $-10, 5, -2$.\\
Wielomian jest stopnia nieparzystego, ponadto znak współczynnika przy\linebreak najwyższej potędze x jest ujemny.\\ W związku z tym wykres wielomianu zaczyna się od lewej strony powyżej osi OX. A więc $$x \in (-\infty,-10) \cup (-2,5).$$
\rozwStop
\odpStart
$x \in (-\infty,-10) \cup (-2,5)$
\odpStop
\testStart
A.$x \in (-\infty,-10) \cup (-2,5)$\\
B.$x \in (-\infty,-10) \cup (-2,5]$\\
C.$x \in (-\infty,-10) \cup [-2,5)$\\
D.$x \in (-\infty,-10] \cup (-2,5)$\\
E.$x \in (-\infty,-10] \cup (-2,5]$\\
F.$x \in (-\infty,-10] \cup [-2,5)$\\
G.$x \in (-\infty,-10) \cup [-2,5]$\\
H.$x \in (-\infty,-10] \cup [-2,5]$
\testStop
\kluczStart
A
\kluczStop



\zadStart{Zadanie z Wikieł Z 1.62 b) moja wersja nr 110}

Rozwiązać nierówności $(x+11)(5-x)(x+2)\ge0$.
\zadStop
\rozwStart{Patryk Wirkus}{Laura Mieczkowska}
Miejsca zerowe naszego wielomianu to: $-11, 5, -2$.\\
Wielomian jest stopnia nieparzystego, ponadto znak współczynnika przy\linebreak najwyższej potędze x jest ujemny.\\ W związku z tym wykres wielomianu zaczyna się od lewej strony powyżej osi OX. A więc $$x \in (-\infty,-11) \cup (-2,5).$$
\rozwStop
\odpStart
$x \in (-\infty,-11) \cup (-2,5)$
\odpStop
\testStart
A.$x \in (-\infty,-11) \cup (-2,5)$\\
B.$x \in (-\infty,-11) \cup (-2,5]$\\
C.$x \in (-\infty,-11) \cup [-2,5)$\\
D.$x \in (-\infty,-11] \cup (-2,5)$\\
E.$x \in (-\infty,-11] \cup (-2,5]$\\
F.$x \in (-\infty,-11] \cup [-2,5)$\\
G.$x \in (-\infty,-11) \cup [-2,5]$\\
H.$x \in (-\infty,-11] \cup [-2,5]$
\testStop
\kluczStart
A
\kluczStop



\zadStart{Zadanie z Wikieł Z 1.62 b) moja wersja nr 111}

Rozwiązać nierówności $(x+12)(5-x)(x+2)\ge0$.
\zadStop
\rozwStart{Patryk Wirkus}{Laura Mieczkowska}
Miejsca zerowe naszego wielomianu to: $-12, 5, -2$.\\
Wielomian jest stopnia nieparzystego, ponadto znak współczynnika przy\linebreak najwyższej potędze x jest ujemny.\\ W związku z tym wykres wielomianu zaczyna się od lewej strony powyżej osi OX. A więc $$x \in (-\infty,-12) \cup (-2,5).$$
\rozwStop
\odpStart
$x \in (-\infty,-12) \cup (-2,5)$
\odpStop
\testStart
A.$x \in (-\infty,-12) \cup (-2,5)$\\
B.$x \in (-\infty,-12) \cup (-2,5]$\\
C.$x \in (-\infty,-12) \cup [-2,5)$\\
D.$x \in (-\infty,-12] \cup (-2,5)$\\
E.$x \in (-\infty,-12] \cup (-2,5]$\\
F.$x \in (-\infty,-12] \cup [-2,5)$\\
G.$x \in (-\infty,-12) \cup [-2,5]$\\
H.$x \in (-\infty,-12] \cup [-2,5]$
\testStop
\kluczStart
A
\kluczStop



\zadStart{Zadanie z Wikieł Z 1.62 b) moja wersja nr 112}

Rozwiązać nierówności $(x+13)(5-x)(x+2)\ge0$.
\zadStop
\rozwStart{Patryk Wirkus}{Laura Mieczkowska}
Miejsca zerowe naszego wielomianu to: $-13, 5, -2$.\\
Wielomian jest stopnia nieparzystego, ponadto znak współczynnika przy\linebreak najwyższej potędze x jest ujemny.\\ W związku z tym wykres wielomianu zaczyna się od lewej strony powyżej osi OX. A więc $$x \in (-\infty,-13) \cup (-2,5).$$
\rozwStop
\odpStart
$x \in (-\infty,-13) \cup (-2,5)$
\odpStop
\testStart
A.$x \in (-\infty,-13) \cup (-2,5)$\\
B.$x \in (-\infty,-13) \cup (-2,5]$\\
C.$x \in (-\infty,-13) \cup [-2,5)$\\
D.$x \in (-\infty,-13] \cup (-2,5)$\\
E.$x \in (-\infty,-13] \cup (-2,5]$\\
F.$x \in (-\infty,-13] \cup [-2,5)$\\
G.$x \in (-\infty,-13) \cup [-2,5]$\\
H.$x \in (-\infty,-13] \cup [-2,5]$
\testStop
\kluczStart
A
\kluczStop



\zadStart{Zadanie z Wikieł Z 1.62 b) moja wersja nr 113}

Rozwiązać nierówności $(x+14)(5-x)(x+2)\ge0$.
\zadStop
\rozwStart{Patryk Wirkus}{Laura Mieczkowska}
Miejsca zerowe naszego wielomianu to: $-14, 5, -2$.\\
Wielomian jest stopnia nieparzystego, ponadto znak współczynnika przy\linebreak najwyższej potędze x jest ujemny.\\ W związku z tym wykres wielomianu zaczyna się od lewej strony powyżej osi OX. A więc $$x \in (-\infty,-14) \cup (-2,5).$$
\rozwStop
\odpStart
$x \in (-\infty,-14) \cup (-2,5)$
\odpStop
\testStart
A.$x \in (-\infty,-14) \cup (-2,5)$\\
B.$x \in (-\infty,-14) \cup (-2,5]$\\
C.$x \in (-\infty,-14) \cup [-2,5)$\\
D.$x \in (-\infty,-14] \cup (-2,5)$\\
E.$x \in (-\infty,-14] \cup (-2,5]$\\
F.$x \in (-\infty,-14] \cup [-2,5)$\\
G.$x \in (-\infty,-14) \cup [-2,5]$\\
H.$x \in (-\infty,-14] \cup [-2,5]$
\testStop
\kluczStart
A
\kluczStop



\zadStart{Zadanie z Wikieł Z 1.62 b) moja wersja nr 114}

Rozwiązać nierówności $(x+15)(5-x)(x+2)\ge0$.
\zadStop
\rozwStart{Patryk Wirkus}{Laura Mieczkowska}
Miejsca zerowe naszego wielomianu to: $-15, 5, -2$.\\
Wielomian jest stopnia nieparzystego, ponadto znak współczynnika przy\linebreak najwyższej potędze x jest ujemny.\\ W związku z tym wykres wielomianu zaczyna się od lewej strony powyżej osi OX. A więc $$x \in (-\infty,-15) \cup (-2,5).$$
\rozwStop
\odpStart
$x \in (-\infty,-15) \cup (-2,5)$
\odpStop
\testStart
A.$x \in (-\infty,-15) \cup (-2,5)$\\
B.$x \in (-\infty,-15) \cup (-2,5]$\\
C.$x \in (-\infty,-15) \cup [-2,5)$\\
D.$x \in (-\infty,-15] \cup (-2,5)$\\
E.$x \in (-\infty,-15] \cup (-2,5]$\\
F.$x \in (-\infty,-15] \cup [-2,5)$\\
G.$x \in (-\infty,-15) \cup [-2,5]$\\
H.$x \in (-\infty,-15] \cup [-2,5]$
\testStop
\kluczStart
A
\kluczStop



\zadStart{Zadanie z Wikieł Z 1.62 b) moja wersja nr 115}

Rozwiązać nierówności $(x+7)(6-x)(x+2)\ge0$.
\zadStop
\rozwStart{Patryk Wirkus}{Laura Mieczkowska}
Miejsca zerowe naszego wielomianu to: $-7, 6, -2$.\\
Wielomian jest stopnia nieparzystego, ponadto znak współczynnika przy\linebreak najwyższej potędze x jest ujemny.\\ W związku z tym wykres wielomianu zaczyna się od lewej strony powyżej osi OX. A więc $$x \in (-\infty,-7) \cup (-2,6).$$
\rozwStop
\odpStart
$x \in (-\infty,-7) \cup (-2,6)$
\odpStop
\testStart
A.$x \in (-\infty,-7) \cup (-2,6)$\\
B.$x \in (-\infty,-7) \cup (-2,6]$\\
C.$x \in (-\infty,-7) \cup [-2,6)$\\
D.$x \in (-\infty,-7] \cup (-2,6)$\\
E.$x \in (-\infty,-7] \cup (-2,6]$\\
F.$x \in (-\infty,-7] \cup [-2,6)$\\
G.$x \in (-\infty,-7) \cup [-2,6]$\\
H.$x \in (-\infty,-7] \cup [-2,6]$
\testStop
\kluczStart
A
\kluczStop



\zadStart{Zadanie z Wikieł Z 1.62 b) moja wersja nr 116}

Rozwiązać nierówności $(x+8)(6-x)(x+2)\ge0$.
\zadStop
\rozwStart{Patryk Wirkus}{Laura Mieczkowska}
Miejsca zerowe naszego wielomianu to: $-8, 6, -2$.\\
Wielomian jest stopnia nieparzystego, ponadto znak współczynnika przy\linebreak najwyższej potędze x jest ujemny.\\ W związku z tym wykres wielomianu zaczyna się od lewej strony powyżej osi OX. A więc $$x \in (-\infty,-8) \cup (-2,6).$$
\rozwStop
\odpStart
$x \in (-\infty,-8) \cup (-2,6)$
\odpStop
\testStart
A.$x \in (-\infty,-8) \cup (-2,6)$\\
B.$x \in (-\infty,-8) \cup (-2,6]$\\
C.$x \in (-\infty,-8) \cup [-2,6)$\\
D.$x \in (-\infty,-8] \cup (-2,6)$\\
E.$x \in (-\infty,-8] \cup (-2,6]$\\
F.$x \in (-\infty,-8] \cup [-2,6)$\\
G.$x \in (-\infty,-8) \cup [-2,6]$\\
H.$x \in (-\infty,-8] \cup [-2,6]$
\testStop
\kluczStart
A
\kluczStop



\zadStart{Zadanie z Wikieł Z 1.62 b) moja wersja nr 117}

Rozwiązać nierówności $(x+9)(6-x)(x+2)\ge0$.
\zadStop
\rozwStart{Patryk Wirkus}{Laura Mieczkowska}
Miejsca zerowe naszego wielomianu to: $-9, 6, -2$.\\
Wielomian jest stopnia nieparzystego, ponadto znak współczynnika przy\linebreak najwyższej potędze x jest ujemny.\\ W związku z tym wykres wielomianu zaczyna się od lewej strony powyżej osi OX. A więc $$x \in (-\infty,-9) \cup (-2,6).$$
\rozwStop
\odpStart
$x \in (-\infty,-9) \cup (-2,6)$
\odpStop
\testStart
A.$x \in (-\infty,-9) \cup (-2,6)$\\
B.$x \in (-\infty,-9) \cup (-2,6]$\\
C.$x \in (-\infty,-9) \cup [-2,6)$\\
D.$x \in (-\infty,-9] \cup (-2,6)$\\
E.$x \in (-\infty,-9] \cup (-2,6]$\\
F.$x \in (-\infty,-9] \cup [-2,6)$\\
G.$x \in (-\infty,-9) \cup [-2,6]$\\
H.$x \in (-\infty,-9] \cup [-2,6]$
\testStop
\kluczStart
A
\kluczStop



\zadStart{Zadanie z Wikieł Z 1.62 b) moja wersja nr 118}

Rozwiązać nierówności $(x+10)(6-x)(x+2)\ge0$.
\zadStop
\rozwStart{Patryk Wirkus}{Laura Mieczkowska}
Miejsca zerowe naszego wielomianu to: $-10, 6, -2$.\\
Wielomian jest stopnia nieparzystego, ponadto znak współczynnika przy\linebreak najwyższej potędze x jest ujemny.\\ W związku z tym wykres wielomianu zaczyna się od lewej strony powyżej osi OX. A więc $$x \in (-\infty,-10) \cup (-2,6).$$
\rozwStop
\odpStart
$x \in (-\infty,-10) \cup (-2,6)$
\odpStop
\testStart
A.$x \in (-\infty,-10) \cup (-2,6)$\\
B.$x \in (-\infty,-10) \cup (-2,6]$\\
C.$x \in (-\infty,-10) \cup [-2,6)$\\
D.$x \in (-\infty,-10] \cup (-2,6)$\\
E.$x \in (-\infty,-10] \cup (-2,6]$\\
F.$x \in (-\infty,-10] \cup [-2,6)$\\
G.$x \in (-\infty,-10) \cup [-2,6]$\\
H.$x \in (-\infty,-10] \cup [-2,6]$
\testStop
\kluczStart
A
\kluczStop



\zadStart{Zadanie z Wikieł Z 1.62 b) moja wersja nr 119}

Rozwiązać nierówności $(x+11)(6-x)(x+2)\ge0$.
\zadStop
\rozwStart{Patryk Wirkus}{Laura Mieczkowska}
Miejsca zerowe naszego wielomianu to: $-11, 6, -2$.\\
Wielomian jest stopnia nieparzystego, ponadto znak współczynnika przy\linebreak najwyższej potędze x jest ujemny.\\ W związku z tym wykres wielomianu zaczyna się od lewej strony powyżej osi OX. A więc $$x \in (-\infty,-11) \cup (-2,6).$$
\rozwStop
\odpStart
$x \in (-\infty,-11) \cup (-2,6)$
\odpStop
\testStart
A.$x \in (-\infty,-11) \cup (-2,6)$\\
B.$x \in (-\infty,-11) \cup (-2,6]$\\
C.$x \in (-\infty,-11) \cup [-2,6)$\\
D.$x \in (-\infty,-11] \cup (-2,6)$\\
E.$x \in (-\infty,-11] \cup (-2,6]$\\
F.$x \in (-\infty,-11] \cup [-2,6)$\\
G.$x \in (-\infty,-11) \cup [-2,6]$\\
H.$x \in (-\infty,-11] \cup [-2,6]$
\testStop
\kluczStart
A
\kluczStop



\zadStart{Zadanie z Wikieł Z 1.62 b) moja wersja nr 120}

Rozwiązać nierówności $(x+12)(6-x)(x+2)\ge0$.
\zadStop
\rozwStart{Patryk Wirkus}{Laura Mieczkowska}
Miejsca zerowe naszego wielomianu to: $-12, 6, -2$.\\
Wielomian jest stopnia nieparzystego, ponadto znak współczynnika przy\linebreak najwyższej potędze x jest ujemny.\\ W związku z tym wykres wielomianu zaczyna się od lewej strony powyżej osi OX. A więc $$x \in (-\infty,-12) \cup (-2,6).$$
\rozwStop
\odpStart
$x \in (-\infty,-12) \cup (-2,6)$
\odpStop
\testStart
A.$x \in (-\infty,-12) \cup (-2,6)$\\
B.$x \in (-\infty,-12) \cup (-2,6]$\\
C.$x \in (-\infty,-12) \cup [-2,6)$\\
D.$x \in (-\infty,-12] \cup (-2,6)$\\
E.$x \in (-\infty,-12] \cup (-2,6]$\\
F.$x \in (-\infty,-12] \cup [-2,6)$\\
G.$x \in (-\infty,-12) \cup [-2,6]$\\
H.$x \in (-\infty,-12] \cup [-2,6]$
\testStop
\kluczStart
A
\kluczStop



\zadStart{Zadanie z Wikieł Z 1.62 b) moja wersja nr 121}

Rozwiązać nierówności $(x+13)(6-x)(x+2)\ge0$.
\zadStop
\rozwStart{Patryk Wirkus}{Laura Mieczkowska}
Miejsca zerowe naszego wielomianu to: $-13, 6, -2$.\\
Wielomian jest stopnia nieparzystego, ponadto znak współczynnika przy\linebreak najwyższej potędze x jest ujemny.\\ W związku z tym wykres wielomianu zaczyna się od lewej strony powyżej osi OX. A więc $$x \in (-\infty,-13) \cup (-2,6).$$
\rozwStop
\odpStart
$x \in (-\infty,-13) \cup (-2,6)$
\odpStop
\testStart
A.$x \in (-\infty,-13) \cup (-2,6)$\\
B.$x \in (-\infty,-13) \cup (-2,6]$\\
C.$x \in (-\infty,-13) \cup [-2,6)$\\
D.$x \in (-\infty,-13] \cup (-2,6)$\\
E.$x \in (-\infty,-13] \cup (-2,6]$\\
F.$x \in (-\infty,-13] \cup [-2,6)$\\
G.$x \in (-\infty,-13) \cup [-2,6]$\\
H.$x \in (-\infty,-13] \cup [-2,6]$
\testStop
\kluczStart
A
\kluczStop



\zadStart{Zadanie z Wikieł Z 1.62 b) moja wersja nr 122}

Rozwiązać nierówności $(x+14)(6-x)(x+2)\ge0$.
\zadStop
\rozwStart{Patryk Wirkus}{Laura Mieczkowska}
Miejsca zerowe naszego wielomianu to: $-14, 6, -2$.\\
Wielomian jest stopnia nieparzystego, ponadto znak współczynnika przy\linebreak najwyższej potędze x jest ujemny.\\ W związku z tym wykres wielomianu zaczyna się od lewej strony powyżej osi OX. A więc $$x \in (-\infty,-14) \cup (-2,6).$$
\rozwStop
\odpStart
$x \in (-\infty,-14) \cup (-2,6)$
\odpStop
\testStart
A.$x \in (-\infty,-14) \cup (-2,6)$\\
B.$x \in (-\infty,-14) \cup (-2,6]$\\
C.$x \in (-\infty,-14) \cup [-2,6)$\\
D.$x \in (-\infty,-14] \cup (-2,6)$\\
E.$x \in (-\infty,-14] \cup (-2,6]$\\
F.$x \in (-\infty,-14] \cup [-2,6)$\\
G.$x \in (-\infty,-14) \cup [-2,6]$\\
H.$x \in (-\infty,-14] \cup [-2,6]$
\testStop
\kluczStart
A
\kluczStop



\zadStart{Zadanie z Wikieł Z 1.62 b) moja wersja nr 123}

Rozwiązać nierówności $(x+15)(6-x)(x+2)\ge0$.
\zadStop
\rozwStart{Patryk Wirkus}{Laura Mieczkowska}
Miejsca zerowe naszego wielomianu to: $-15, 6, -2$.\\
Wielomian jest stopnia nieparzystego, ponadto znak współczynnika przy\linebreak najwyższej potędze x jest ujemny.\\ W związku z tym wykres wielomianu zaczyna się od lewej strony powyżej osi OX. A więc $$x \in (-\infty,-15) \cup (-2,6).$$
\rozwStop
\odpStart
$x \in (-\infty,-15) \cup (-2,6)$
\odpStop
\testStart
A.$x \in (-\infty,-15) \cup (-2,6)$\\
B.$x \in (-\infty,-15) \cup (-2,6]$\\
C.$x \in (-\infty,-15) \cup [-2,6)$\\
D.$x \in (-\infty,-15] \cup (-2,6)$\\
E.$x \in (-\infty,-15] \cup (-2,6]$\\
F.$x \in (-\infty,-15] \cup [-2,6)$\\
G.$x \in (-\infty,-15) \cup [-2,6]$\\
H.$x \in (-\infty,-15] \cup [-2,6]$
\testStop
\kluczStart
A
\kluczStop



\zadStart{Zadanie z Wikieł Z 1.62 b) moja wersja nr 124}

Rozwiązać nierówności $(x+8)(7-x)(x+2)\ge0$.
\zadStop
\rozwStart{Patryk Wirkus}{Laura Mieczkowska}
Miejsca zerowe naszego wielomianu to: $-8, 7, -2$.\\
Wielomian jest stopnia nieparzystego, ponadto znak współczynnika przy\linebreak najwyższej potędze x jest ujemny.\\ W związku z tym wykres wielomianu zaczyna się od lewej strony powyżej osi OX. A więc $$x \in (-\infty,-8) \cup (-2,7).$$
\rozwStop
\odpStart
$x \in (-\infty,-8) \cup (-2,7)$
\odpStop
\testStart
A.$x \in (-\infty,-8) \cup (-2,7)$\\
B.$x \in (-\infty,-8) \cup (-2,7]$\\
C.$x \in (-\infty,-8) \cup [-2,7)$\\
D.$x \in (-\infty,-8] \cup (-2,7)$\\
E.$x \in (-\infty,-8] \cup (-2,7]$\\
F.$x \in (-\infty,-8] \cup [-2,7)$\\
G.$x \in (-\infty,-8) \cup [-2,7]$\\
H.$x \in (-\infty,-8] \cup [-2,7]$
\testStop
\kluczStart
A
\kluczStop



\zadStart{Zadanie z Wikieł Z 1.62 b) moja wersja nr 125}

Rozwiązać nierówności $(x+9)(7-x)(x+2)\ge0$.
\zadStop
\rozwStart{Patryk Wirkus}{Laura Mieczkowska}
Miejsca zerowe naszego wielomianu to: $-9, 7, -2$.\\
Wielomian jest stopnia nieparzystego, ponadto znak współczynnika przy\linebreak najwyższej potędze x jest ujemny.\\ W związku z tym wykres wielomianu zaczyna się od lewej strony powyżej osi OX. A więc $$x \in (-\infty,-9) \cup (-2,7).$$
\rozwStop
\odpStart
$x \in (-\infty,-9) \cup (-2,7)$
\odpStop
\testStart
A.$x \in (-\infty,-9) \cup (-2,7)$\\
B.$x \in (-\infty,-9) \cup (-2,7]$\\
C.$x \in (-\infty,-9) \cup [-2,7)$\\
D.$x \in (-\infty,-9] \cup (-2,7)$\\
E.$x \in (-\infty,-9] \cup (-2,7]$\\
F.$x \in (-\infty,-9] \cup [-2,7)$\\
G.$x \in (-\infty,-9) \cup [-2,7]$\\
H.$x \in (-\infty,-9] \cup [-2,7]$
\testStop
\kluczStart
A
\kluczStop



\zadStart{Zadanie z Wikieł Z 1.62 b) moja wersja nr 126}

Rozwiązać nierówności $(x+10)(7-x)(x+2)\ge0$.
\zadStop
\rozwStart{Patryk Wirkus}{Laura Mieczkowska}
Miejsca zerowe naszego wielomianu to: $-10, 7, -2$.\\
Wielomian jest stopnia nieparzystego, ponadto znak współczynnika przy\linebreak najwyższej potędze x jest ujemny.\\ W związku z tym wykres wielomianu zaczyna się od lewej strony powyżej osi OX. A więc $$x \in (-\infty,-10) \cup (-2,7).$$
\rozwStop
\odpStart
$x \in (-\infty,-10) \cup (-2,7)$
\odpStop
\testStart
A.$x \in (-\infty,-10) \cup (-2,7)$\\
B.$x \in (-\infty,-10) \cup (-2,7]$\\
C.$x \in (-\infty,-10) \cup [-2,7)$\\
D.$x \in (-\infty,-10] \cup (-2,7)$\\
E.$x \in (-\infty,-10] \cup (-2,7]$\\
F.$x \in (-\infty,-10] \cup [-2,7)$\\
G.$x \in (-\infty,-10) \cup [-2,7]$\\
H.$x \in (-\infty,-10] \cup [-2,7]$
\testStop
\kluczStart
A
\kluczStop



\zadStart{Zadanie z Wikieł Z 1.62 b) moja wersja nr 127}

Rozwiązać nierówności $(x+11)(7-x)(x+2)\ge0$.
\zadStop
\rozwStart{Patryk Wirkus}{Laura Mieczkowska}
Miejsca zerowe naszego wielomianu to: $-11, 7, -2$.\\
Wielomian jest stopnia nieparzystego, ponadto znak współczynnika przy\linebreak najwyższej potędze x jest ujemny.\\ W związku z tym wykres wielomianu zaczyna się od lewej strony powyżej osi OX. A więc $$x \in (-\infty,-11) \cup (-2,7).$$
\rozwStop
\odpStart
$x \in (-\infty,-11) \cup (-2,7)$
\odpStop
\testStart
A.$x \in (-\infty,-11) \cup (-2,7)$\\
B.$x \in (-\infty,-11) \cup (-2,7]$\\
C.$x \in (-\infty,-11) \cup [-2,7)$\\
D.$x \in (-\infty,-11] \cup (-2,7)$\\
E.$x \in (-\infty,-11] \cup (-2,7]$\\
F.$x \in (-\infty,-11] \cup [-2,7)$\\
G.$x \in (-\infty,-11) \cup [-2,7]$\\
H.$x \in (-\infty,-11] \cup [-2,7]$
\testStop
\kluczStart
A
\kluczStop



\zadStart{Zadanie z Wikieł Z 1.62 b) moja wersja nr 128}

Rozwiązać nierówności $(x+12)(7-x)(x+2)\ge0$.
\zadStop
\rozwStart{Patryk Wirkus}{Laura Mieczkowska}
Miejsca zerowe naszego wielomianu to: $-12, 7, -2$.\\
Wielomian jest stopnia nieparzystego, ponadto znak współczynnika przy\linebreak najwyższej potędze x jest ujemny.\\ W związku z tym wykres wielomianu zaczyna się od lewej strony powyżej osi OX. A więc $$x \in (-\infty,-12) \cup (-2,7).$$
\rozwStop
\odpStart
$x \in (-\infty,-12) \cup (-2,7)$
\odpStop
\testStart
A.$x \in (-\infty,-12) \cup (-2,7)$\\
B.$x \in (-\infty,-12) \cup (-2,7]$\\
C.$x \in (-\infty,-12) \cup [-2,7)$\\
D.$x \in (-\infty,-12] \cup (-2,7)$\\
E.$x \in (-\infty,-12] \cup (-2,7]$\\
F.$x \in (-\infty,-12] \cup [-2,7)$\\
G.$x \in (-\infty,-12) \cup [-2,7]$\\
H.$x \in (-\infty,-12] \cup [-2,7]$
\testStop
\kluczStart
A
\kluczStop



\zadStart{Zadanie z Wikieł Z 1.62 b) moja wersja nr 129}

Rozwiązać nierówności $(x+13)(7-x)(x+2)\ge0$.
\zadStop
\rozwStart{Patryk Wirkus}{Laura Mieczkowska}
Miejsca zerowe naszego wielomianu to: $-13, 7, -2$.\\
Wielomian jest stopnia nieparzystego, ponadto znak współczynnika przy\linebreak najwyższej potędze x jest ujemny.\\ W związku z tym wykres wielomianu zaczyna się od lewej strony powyżej osi OX. A więc $$x \in (-\infty,-13) \cup (-2,7).$$
\rozwStop
\odpStart
$x \in (-\infty,-13) \cup (-2,7)$
\odpStop
\testStart
A.$x \in (-\infty,-13) \cup (-2,7)$\\
B.$x \in (-\infty,-13) \cup (-2,7]$\\
C.$x \in (-\infty,-13) \cup [-2,7)$\\
D.$x \in (-\infty,-13] \cup (-2,7)$\\
E.$x \in (-\infty,-13] \cup (-2,7]$\\
F.$x \in (-\infty,-13] \cup [-2,7)$\\
G.$x \in (-\infty,-13) \cup [-2,7]$\\
H.$x \in (-\infty,-13] \cup [-2,7]$
\testStop
\kluczStart
A
\kluczStop



\zadStart{Zadanie z Wikieł Z 1.62 b) moja wersja nr 130}

Rozwiązać nierówności $(x+14)(7-x)(x+2)\ge0$.
\zadStop
\rozwStart{Patryk Wirkus}{Laura Mieczkowska}
Miejsca zerowe naszego wielomianu to: $-14, 7, -2$.\\
Wielomian jest stopnia nieparzystego, ponadto znak współczynnika przy\linebreak najwyższej potędze x jest ujemny.\\ W związku z tym wykres wielomianu zaczyna się od lewej strony powyżej osi OX. A więc $$x \in (-\infty,-14) \cup (-2,7).$$
\rozwStop
\odpStart
$x \in (-\infty,-14) \cup (-2,7)$
\odpStop
\testStart
A.$x \in (-\infty,-14) \cup (-2,7)$\\
B.$x \in (-\infty,-14) \cup (-2,7]$\\
C.$x \in (-\infty,-14) \cup [-2,7)$\\
D.$x \in (-\infty,-14] \cup (-2,7)$\\
E.$x \in (-\infty,-14] \cup (-2,7]$\\
F.$x \in (-\infty,-14] \cup [-2,7)$\\
G.$x \in (-\infty,-14) \cup [-2,7]$\\
H.$x \in (-\infty,-14] \cup [-2,7]$
\testStop
\kluczStart
A
\kluczStop



\zadStart{Zadanie z Wikieł Z 1.62 b) moja wersja nr 131}

Rozwiązać nierówności $(x+15)(7-x)(x+2)\ge0$.
\zadStop
\rozwStart{Patryk Wirkus}{Laura Mieczkowska}
Miejsca zerowe naszego wielomianu to: $-15, 7, -2$.\\
Wielomian jest stopnia nieparzystego, ponadto znak współczynnika przy\linebreak najwyższej potędze x jest ujemny.\\ W związku z tym wykres wielomianu zaczyna się od lewej strony powyżej osi OX. A więc $$x \in (-\infty,-15) \cup (-2,7).$$
\rozwStop
\odpStart
$x \in (-\infty,-15) \cup (-2,7)$
\odpStop
\testStart
A.$x \in (-\infty,-15) \cup (-2,7)$\\
B.$x \in (-\infty,-15) \cup (-2,7]$\\
C.$x \in (-\infty,-15) \cup [-2,7)$\\
D.$x \in (-\infty,-15] \cup (-2,7)$\\
E.$x \in (-\infty,-15] \cup (-2,7]$\\
F.$x \in (-\infty,-15] \cup [-2,7)$\\
G.$x \in (-\infty,-15) \cup [-2,7]$\\
H.$x \in (-\infty,-15] \cup [-2,7]$
\testStop
\kluczStart
A
\kluczStop



\zadStart{Zadanie z Wikieł Z 1.62 b) moja wersja nr 132}

Rozwiązać nierówności $(x+9)(8-x)(x+2)\ge0$.
\zadStop
\rozwStart{Patryk Wirkus}{Laura Mieczkowska}
Miejsca zerowe naszego wielomianu to: $-9, 8, -2$.\\
Wielomian jest stopnia nieparzystego, ponadto znak współczynnika przy\linebreak najwyższej potędze x jest ujemny.\\ W związku z tym wykres wielomianu zaczyna się od lewej strony powyżej osi OX. A więc $$x \in (-\infty,-9) \cup (-2,8).$$
\rozwStop
\odpStart
$x \in (-\infty,-9) \cup (-2,8)$
\odpStop
\testStart
A.$x \in (-\infty,-9) \cup (-2,8)$\\
B.$x \in (-\infty,-9) \cup (-2,8]$\\
C.$x \in (-\infty,-9) \cup [-2,8)$\\
D.$x \in (-\infty,-9] \cup (-2,8)$\\
E.$x \in (-\infty,-9] \cup (-2,8]$\\
F.$x \in (-\infty,-9] \cup [-2,8)$\\
G.$x \in (-\infty,-9) \cup [-2,8]$\\
H.$x \in (-\infty,-9] \cup [-2,8]$
\testStop
\kluczStart
A
\kluczStop



\zadStart{Zadanie z Wikieł Z 1.62 b) moja wersja nr 133}

Rozwiązać nierówności $(x+10)(8-x)(x+2)\ge0$.
\zadStop
\rozwStart{Patryk Wirkus}{Laura Mieczkowska}
Miejsca zerowe naszego wielomianu to: $-10, 8, -2$.\\
Wielomian jest stopnia nieparzystego, ponadto znak współczynnika przy\linebreak najwyższej potędze x jest ujemny.\\ W związku z tym wykres wielomianu zaczyna się od lewej strony powyżej osi OX. A więc $$x \in (-\infty,-10) \cup (-2,8).$$
\rozwStop
\odpStart
$x \in (-\infty,-10) \cup (-2,8)$
\odpStop
\testStart
A.$x \in (-\infty,-10) \cup (-2,8)$\\
B.$x \in (-\infty,-10) \cup (-2,8]$\\
C.$x \in (-\infty,-10) \cup [-2,8)$\\
D.$x \in (-\infty,-10] \cup (-2,8)$\\
E.$x \in (-\infty,-10] \cup (-2,8]$\\
F.$x \in (-\infty,-10] \cup [-2,8)$\\
G.$x \in (-\infty,-10) \cup [-2,8]$\\
H.$x \in (-\infty,-10] \cup [-2,8]$
\testStop
\kluczStart
A
\kluczStop



\zadStart{Zadanie z Wikieł Z 1.62 b) moja wersja nr 134}

Rozwiązać nierówności $(x+11)(8-x)(x+2)\ge0$.
\zadStop
\rozwStart{Patryk Wirkus}{Laura Mieczkowska}
Miejsca zerowe naszego wielomianu to: $-11, 8, -2$.\\
Wielomian jest stopnia nieparzystego, ponadto znak współczynnika przy\linebreak najwyższej potędze x jest ujemny.\\ W związku z tym wykres wielomianu zaczyna się od lewej strony powyżej osi OX. A więc $$x \in (-\infty,-11) \cup (-2,8).$$
\rozwStop
\odpStart
$x \in (-\infty,-11) \cup (-2,8)$
\odpStop
\testStart
A.$x \in (-\infty,-11) \cup (-2,8)$\\
B.$x \in (-\infty,-11) \cup (-2,8]$\\
C.$x \in (-\infty,-11) \cup [-2,8)$\\
D.$x \in (-\infty,-11] \cup (-2,8)$\\
E.$x \in (-\infty,-11] \cup (-2,8]$\\
F.$x \in (-\infty,-11] \cup [-2,8)$\\
G.$x \in (-\infty,-11) \cup [-2,8]$\\
H.$x \in (-\infty,-11] \cup [-2,8]$
\testStop
\kluczStart
A
\kluczStop



\zadStart{Zadanie z Wikieł Z 1.62 b) moja wersja nr 135}

Rozwiązać nierówności $(x+12)(8-x)(x+2)\ge0$.
\zadStop
\rozwStart{Patryk Wirkus}{Laura Mieczkowska}
Miejsca zerowe naszego wielomianu to: $-12, 8, -2$.\\
Wielomian jest stopnia nieparzystego, ponadto znak współczynnika przy\linebreak najwyższej potędze x jest ujemny.\\ W związku z tym wykres wielomianu zaczyna się od lewej strony powyżej osi OX. A więc $$x \in (-\infty,-12) \cup (-2,8).$$
\rozwStop
\odpStart
$x \in (-\infty,-12) \cup (-2,8)$
\odpStop
\testStart
A.$x \in (-\infty,-12) \cup (-2,8)$\\
B.$x \in (-\infty,-12) \cup (-2,8]$\\
C.$x \in (-\infty,-12) \cup [-2,8)$\\
D.$x \in (-\infty,-12] \cup (-2,8)$\\
E.$x \in (-\infty,-12] \cup (-2,8]$\\
F.$x \in (-\infty,-12] \cup [-2,8)$\\
G.$x \in (-\infty,-12) \cup [-2,8]$\\
H.$x \in (-\infty,-12] \cup [-2,8]$
\testStop
\kluczStart
A
\kluczStop



\zadStart{Zadanie z Wikieł Z 1.62 b) moja wersja nr 136}

Rozwiązać nierówności $(x+13)(8-x)(x+2)\ge0$.
\zadStop
\rozwStart{Patryk Wirkus}{Laura Mieczkowska}
Miejsca zerowe naszego wielomianu to: $-13, 8, -2$.\\
Wielomian jest stopnia nieparzystego, ponadto znak współczynnika przy\linebreak najwyższej potędze x jest ujemny.\\ W związku z tym wykres wielomianu zaczyna się od lewej strony powyżej osi OX. A więc $$x \in (-\infty,-13) \cup (-2,8).$$
\rozwStop
\odpStart
$x \in (-\infty,-13) \cup (-2,8)$
\odpStop
\testStart
A.$x \in (-\infty,-13) \cup (-2,8)$\\
B.$x \in (-\infty,-13) \cup (-2,8]$\\
C.$x \in (-\infty,-13) \cup [-2,8)$\\
D.$x \in (-\infty,-13] \cup (-2,8)$\\
E.$x \in (-\infty,-13] \cup (-2,8]$\\
F.$x \in (-\infty,-13] \cup [-2,8)$\\
G.$x \in (-\infty,-13) \cup [-2,8]$\\
H.$x \in (-\infty,-13] \cup [-2,8]$
\testStop
\kluczStart
A
\kluczStop



\zadStart{Zadanie z Wikieł Z 1.62 b) moja wersja nr 137}

Rozwiązać nierówności $(x+14)(8-x)(x+2)\ge0$.
\zadStop
\rozwStart{Patryk Wirkus}{Laura Mieczkowska}
Miejsca zerowe naszego wielomianu to: $-14, 8, -2$.\\
Wielomian jest stopnia nieparzystego, ponadto znak współczynnika przy\linebreak najwyższej potędze x jest ujemny.\\ W związku z tym wykres wielomianu zaczyna się od lewej strony powyżej osi OX. A więc $$x \in (-\infty,-14) \cup (-2,8).$$
\rozwStop
\odpStart
$x \in (-\infty,-14) \cup (-2,8)$
\odpStop
\testStart
A.$x \in (-\infty,-14) \cup (-2,8)$\\
B.$x \in (-\infty,-14) \cup (-2,8]$\\
C.$x \in (-\infty,-14) \cup [-2,8)$\\
D.$x \in (-\infty,-14] \cup (-2,8)$\\
E.$x \in (-\infty,-14] \cup (-2,8]$\\
F.$x \in (-\infty,-14] \cup [-2,8)$\\
G.$x \in (-\infty,-14) \cup [-2,8]$\\
H.$x \in (-\infty,-14] \cup [-2,8]$
\testStop
\kluczStart
A
\kluczStop



\zadStart{Zadanie z Wikieł Z 1.62 b) moja wersja nr 138}

Rozwiązać nierówności $(x+15)(8-x)(x+2)\ge0$.
\zadStop
\rozwStart{Patryk Wirkus}{Laura Mieczkowska}
Miejsca zerowe naszego wielomianu to: $-15, 8, -2$.\\
Wielomian jest stopnia nieparzystego, ponadto znak współczynnika przy\linebreak najwyższej potędze x jest ujemny.\\ W związku z tym wykres wielomianu zaczyna się od lewej strony powyżej osi OX. A więc $$x \in (-\infty,-15) \cup (-2,8).$$
\rozwStop
\odpStart
$x \in (-\infty,-15) \cup (-2,8)$
\odpStop
\testStart
A.$x \in (-\infty,-15) \cup (-2,8)$\\
B.$x \in (-\infty,-15) \cup (-2,8]$\\
C.$x \in (-\infty,-15) \cup [-2,8)$\\
D.$x \in (-\infty,-15] \cup (-2,8)$\\
E.$x \in (-\infty,-15] \cup (-2,8]$\\
F.$x \in (-\infty,-15] \cup [-2,8)$\\
G.$x \in (-\infty,-15) \cup [-2,8]$\\
H.$x \in (-\infty,-15] \cup [-2,8]$
\testStop
\kluczStart
A
\kluczStop



\zadStart{Zadanie z Wikieł Z 1.62 b) moja wersja nr 139}

Rozwiązać nierówności $(x+10)(9-x)(x+2)\ge0$.
\zadStop
\rozwStart{Patryk Wirkus}{Laura Mieczkowska}
Miejsca zerowe naszego wielomianu to: $-10, 9, -2$.\\
Wielomian jest stopnia nieparzystego, ponadto znak współczynnika przy\linebreak najwyższej potędze x jest ujemny.\\ W związku z tym wykres wielomianu zaczyna się od lewej strony powyżej osi OX. A więc $$x \in (-\infty,-10) \cup (-2,9).$$
\rozwStop
\odpStart
$x \in (-\infty,-10) \cup (-2,9)$
\odpStop
\testStart
A.$x \in (-\infty,-10) \cup (-2,9)$\\
B.$x \in (-\infty,-10) \cup (-2,9]$\\
C.$x \in (-\infty,-10) \cup [-2,9)$\\
D.$x \in (-\infty,-10] \cup (-2,9)$\\
E.$x \in (-\infty,-10] \cup (-2,9]$\\
F.$x \in (-\infty,-10] \cup [-2,9)$\\
G.$x \in (-\infty,-10) \cup [-2,9]$\\
H.$x \in (-\infty,-10] \cup [-2,9]$
\testStop
\kluczStart
A
\kluczStop



\zadStart{Zadanie z Wikieł Z 1.62 b) moja wersja nr 140}

Rozwiązać nierówności $(x+11)(9-x)(x+2)\ge0$.
\zadStop
\rozwStart{Patryk Wirkus}{Laura Mieczkowska}
Miejsca zerowe naszego wielomianu to: $-11, 9, -2$.\\
Wielomian jest stopnia nieparzystego, ponadto znak współczynnika przy\linebreak najwyższej potędze x jest ujemny.\\ W związku z tym wykres wielomianu zaczyna się od lewej strony powyżej osi OX. A więc $$x \in (-\infty,-11) \cup (-2,9).$$
\rozwStop
\odpStart
$x \in (-\infty,-11) \cup (-2,9)$
\odpStop
\testStart
A.$x \in (-\infty,-11) \cup (-2,9)$\\
B.$x \in (-\infty,-11) \cup (-2,9]$\\
C.$x \in (-\infty,-11) \cup [-2,9)$\\
D.$x \in (-\infty,-11] \cup (-2,9)$\\
E.$x \in (-\infty,-11] \cup (-2,9]$\\
F.$x \in (-\infty,-11] \cup [-2,9)$\\
G.$x \in (-\infty,-11) \cup [-2,9]$\\
H.$x \in (-\infty,-11] \cup [-2,9]$
\testStop
\kluczStart
A
\kluczStop



\zadStart{Zadanie z Wikieł Z 1.62 b) moja wersja nr 141}

Rozwiązać nierówności $(x+12)(9-x)(x+2)\ge0$.
\zadStop
\rozwStart{Patryk Wirkus}{Laura Mieczkowska}
Miejsca zerowe naszego wielomianu to: $-12, 9, -2$.\\
Wielomian jest stopnia nieparzystego, ponadto znak współczynnika przy\linebreak najwyższej potędze x jest ujemny.\\ W związku z tym wykres wielomianu zaczyna się od lewej strony powyżej osi OX. A więc $$x \in (-\infty,-12) \cup (-2,9).$$
\rozwStop
\odpStart
$x \in (-\infty,-12) \cup (-2,9)$
\odpStop
\testStart
A.$x \in (-\infty,-12) \cup (-2,9)$\\
B.$x \in (-\infty,-12) \cup (-2,9]$\\
C.$x \in (-\infty,-12) \cup [-2,9)$\\
D.$x \in (-\infty,-12] \cup (-2,9)$\\
E.$x \in (-\infty,-12] \cup (-2,9]$\\
F.$x \in (-\infty,-12] \cup [-2,9)$\\
G.$x \in (-\infty,-12) \cup [-2,9]$\\
H.$x \in (-\infty,-12] \cup [-2,9]$
\testStop
\kluczStart
A
\kluczStop



\zadStart{Zadanie z Wikieł Z 1.62 b) moja wersja nr 142}

Rozwiązać nierówności $(x+13)(9-x)(x+2)\ge0$.
\zadStop
\rozwStart{Patryk Wirkus}{Laura Mieczkowska}
Miejsca zerowe naszego wielomianu to: $-13, 9, -2$.\\
Wielomian jest stopnia nieparzystego, ponadto znak współczynnika przy\linebreak najwyższej potędze x jest ujemny.\\ W związku z tym wykres wielomianu zaczyna się od lewej strony powyżej osi OX. A więc $$x \in (-\infty,-13) \cup (-2,9).$$
\rozwStop
\odpStart
$x \in (-\infty,-13) \cup (-2,9)$
\odpStop
\testStart
A.$x \in (-\infty,-13) \cup (-2,9)$\\
B.$x \in (-\infty,-13) \cup (-2,9]$\\
C.$x \in (-\infty,-13) \cup [-2,9)$\\
D.$x \in (-\infty,-13] \cup (-2,9)$\\
E.$x \in (-\infty,-13] \cup (-2,9]$\\
F.$x \in (-\infty,-13] \cup [-2,9)$\\
G.$x \in (-\infty,-13) \cup [-2,9]$\\
H.$x \in (-\infty,-13] \cup [-2,9]$
\testStop
\kluczStart
A
\kluczStop



\zadStart{Zadanie z Wikieł Z 1.62 b) moja wersja nr 143}

Rozwiązać nierówności $(x+14)(9-x)(x+2)\ge0$.
\zadStop
\rozwStart{Patryk Wirkus}{Laura Mieczkowska}
Miejsca zerowe naszego wielomianu to: $-14, 9, -2$.\\
Wielomian jest stopnia nieparzystego, ponadto znak współczynnika przy\linebreak najwyższej potędze x jest ujemny.\\ W związku z tym wykres wielomianu zaczyna się od lewej strony powyżej osi OX. A więc $$x \in (-\infty,-14) \cup (-2,9).$$
\rozwStop
\odpStart
$x \in (-\infty,-14) \cup (-2,9)$
\odpStop
\testStart
A.$x \in (-\infty,-14) \cup (-2,9)$\\
B.$x \in (-\infty,-14) \cup (-2,9]$\\
C.$x \in (-\infty,-14) \cup [-2,9)$\\
D.$x \in (-\infty,-14] \cup (-2,9)$\\
E.$x \in (-\infty,-14] \cup (-2,9]$\\
F.$x \in (-\infty,-14] \cup [-2,9)$\\
G.$x \in (-\infty,-14) \cup [-2,9]$\\
H.$x \in (-\infty,-14] \cup [-2,9]$
\testStop
\kluczStart
A
\kluczStop



\zadStart{Zadanie z Wikieł Z 1.62 b) moja wersja nr 144}

Rozwiązać nierówności $(x+15)(9-x)(x+2)\ge0$.
\zadStop
\rozwStart{Patryk Wirkus}{Laura Mieczkowska}
Miejsca zerowe naszego wielomianu to: $-15, 9, -2$.\\
Wielomian jest stopnia nieparzystego, ponadto znak współczynnika przy\linebreak najwyższej potędze x jest ujemny.\\ W związku z tym wykres wielomianu zaczyna się od lewej strony powyżej osi OX. A więc $$x \in (-\infty,-15) \cup (-2,9).$$
\rozwStop
\odpStart
$x \in (-\infty,-15) \cup (-2,9)$
\odpStop
\testStart
A.$x \in (-\infty,-15) \cup (-2,9)$\\
B.$x \in (-\infty,-15) \cup (-2,9]$\\
C.$x \in (-\infty,-15) \cup [-2,9)$\\
D.$x \in (-\infty,-15] \cup (-2,9)$\\
E.$x \in (-\infty,-15] \cup (-2,9]$\\
F.$x \in (-\infty,-15] \cup [-2,9)$\\
G.$x \in (-\infty,-15) \cup [-2,9]$\\
H.$x \in (-\infty,-15] \cup [-2,9]$
\testStop
\kluczStart
A
\kluczStop



\zadStart{Zadanie z Wikieł Z 1.62 b) moja wersja nr 145}

Rozwiązać nierówności $(x+11)(10-x)(x+2)\ge0$.
\zadStop
\rozwStart{Patryk Wirkus}{Laura Mieczkowska}
Miejsca zerowe naszego wielomianu to: $-11, 10, -2$.\\
Wielomian jest stopnia nieparzystego, ponadto znak współczynnika przy\linebreak najwyższej potędze x jest ujemny.\\ W związku z tym wykres wielomianu zaczyna się od lewej strony powyżej osi OX. A więc $$x \in (-\infty,-11) \cup (-2,10).$$
\rozwStop
\odpStart
$x \in (-\infty,-11) \cup (-2,10)$
\odpStop
\testStart
A.$x \in (-\infty,-11) \cup (-2,10)$\\
B.$x \in (-\infty,-11) \cup (-2,10]$\\
C.$x \in (-\infty,-11) \cup [-2,10)$\\
D.$x \in (-\infty,-11] \cup (-2,10)$\\
E.$x \in (-\infty,-11] \cup (-2,10]$\\
F.$x \in (-\infty,-11] \cup [-2,10)$\\
G.$x \in (-\infty,-11) \cup [-2,10]$\\
H.$x \in (-\infty,-11] \cup [-2,10]$
\testStop
\kluczStart
A
\kluczStop



\zadStart{Zadanie z Wikieł Z 1.62 b) moja wersja nr 146}

Rozwiązać nierówności $(x+12)(10-x)(x+2)\ge0$.
\zadStop
\rozwStart{Patryk Wirkus}{Laura Mieczkowska}
Miejsca zerowe naszego wielomianu to: $-12, 10, -2$.\\
Wielomian jest stopnia nieparzystego, ponadto znak współczynnika przy\linebreak najwyższej potędze x jest ujemny.\\ W związku z tym wykres wielomianu zaczyna się od lewej strony powyżej osi OX. A więc $$x \in (-\infty,-12) \cup (-2,10).$$
\rozwStop
\odpStart
$x \in (-\infty,-12) \cup (-2,10)$
\odpStop
\testStart
A.$x \in (-\infty,-12) \cup (-2,10)$\\
B.$x \in (-\infty,-12) \cup (-2,10]$\\
C.$x \in (-\infty,-12) \cup [-2,10)$\\
D.$x \in (-\infty,-12] \cup (-2,10)$\\
E.$x \in (-\infty,-12] \cup (-2,10]$\\
F.$x \in (-\infty,-12] \cup [-2,10)$\\
G.$x \in (-\infty,-12) \cup [-2,10]$\\
H.$x \in (-\infty,-12] \cup [-2,10]$
\testStop
\kluczStart
A
\kluczStop



\zadStart{Zadanie z Wikieł Z 1.62 b) moja wersja nr 147}

Rozwiązać nierówności $(x+13)(10-x)(x+2)\ge0$.
\zadStop
\rozwStart{Patryk Wirkus}{Laura Mieczkowska}
Miejsca zerowe naszego wielomianu to: $-13, 10, -2$.\\
Wielomian jest stopnia nieparzystego, ponadto znak współczynnika przy\linebreak najwyższej potędze x jest ujemny.\\ W związku z tym wykres wielomianu zaczyna się od lewej strony powyżej osi OX. A więc $$x \in (-\infty,-13) \cup (-2,10).$$
\rozwStop
\odpStart
$x \in (-\infty,-13) \cup (-2,10)$
\odpStop
\testStart
A.$x \in (-\infty,-13) \cup (-2,10)$\\
B.$x \in (-\infty,-13) \cup (-2,10]$\\
C.$x \in (-\infty,-13) \cup [-2,10)$\\
D.$x \in (-\infty,-13] \cup (-2,10)$\\
E.$x \in (-\infty,-13] \cup (-2,10]$\\
F.$x \in (-\infty,-13] \cup [-2,10)$\\
G.$x \in (-\infty,-13) \cup [-2,10]$\\
H.$x \in (-\infty,-13] \cup [-2,10]$
\testStop
\kluczStart
A
\kluczStop



\zadStart{Zadanie z Wikieł Z 1.62 b) moja wersja nr 148}

Rozwiązać nierówności $(x+14)(10-x)(x+2)\ge0$.
\zadStop
\rozwStart{Patryk Wirkus}{Laura Mieczkowska}
Miejsca zerowe naszego wielomianu to: $-14, 10, -2$.\\
Wielomian jest stopnia nieparzystego, ponadto znak współczynnika przy\linebreak najwyższej potędze x jest ujemny.\\ W związku z tym wykres wielomianu zaczyna się od lewej strony powyżej osi OX. A więc $$x \in (-\infty,-14) \cup (-2,10).$$
\rozwStop
\odpStart
$x \in (-\infty,-14) \cup (-2,10)$
\odpStop
\testStart
A.$x \in (-\infty,-14) \cup (-2,10)$\\
B.$x \in (-\infty,-14) \cup (-2,10]$\\
C.$x \in (-\infty,-14) \cup [-2,10)$\\
D.$x \in (-\infty,-14] \cup (-2,10)$\\
E.$x \in (-\infty,-14] \cup (-2,10]$\\
F.$x \in (-\infty,-14] \cup [-2,10)$\\
G.$x \in (-\infty,-14) \cup [-2,10]$\\
H.$x \in (-\infty,-14] \cup [-2,10]$
\testStop
\kluczStart
A
\kluczStop



\zadStart{Zadanie z Wikieł Z 1.62 b) moja wersja nr 149}

Rozwiązać nierówności $(x+15)(10-x)(x+2)\ge0$.
\zadStop
\rozwStart{Patryk Wirkus}{Laura Mieczkowska}
Miejsca zerowe naszego wielomianu to: $-15, 10, -2$.\\
Wielomian jest stopnia nieparzystego, ponadto znak współczynnika przy\linebreak najwyższej potędze x jest ujemny.\\ W związku z tym wykres wielomianu zaczyna się od lewej strony powyżej osi OX. A więc $$x \in (-\infty,-15) \cup (-2,10).$$
\rozwStop
\odpStart
$x \in (-\infty,-15) \cup (-2,10)$
\odpStop
\testStart
A.$x \in (-\infty,-15) \cup (-2,10)$\\
B.$x \in (-\infty,-15) \cup (-2,10]$\\
C.$x \in (-\infty,-15) \cup [-2,10)$\\
D.$x \in (-\infty,-15] \cup (-2,10)$\\
E.$x \in (-\infty,-15] \cup (-2,10]$\\
F.$x \in (-\infty,-15] \cup [-2,10)$\\
G.$x \in (-\infty,-15) \cup [-2,10]$\\
H.$x \in (-\infty,-15] \cup [-2,10]$
\testStop
\kluczStart
A
\kluczStop



\zadStart{Zadanie z Wikieł Z 1.62 b) moja wersja nr 150}

Rozwiązać nierówności $(x+5)(4-x)(x+3)\ge0$.
\zadStop
\rozwStart{Patryk Wirkus}{Laura Mieczkowska}
Miejsca zerowe naszego wielomianu to: $-5, 4, -3$.\\
Wielomian jest stopnia nieparzystego, ponadto znak współczynnika przy\linebreak najwyższej potędze x jest ujemny.\\ W związku z tym wykres wielomianu zaczyna się od lewej strony powyżej osi OX. A więc $$x \in (-\infty,-5) \cup (-3,4).$$
\rozwStop
\odpStart
$x \in (-\infty,-5) \cup (-3,4)$
\odpStop
\testStart
A.$x \in (-\infty,-5) \cup (-3,4)$\\
B.$x \in (-\infty,-5) \cup (-3,4]$\\
C.$x \in (-\infty,-5) \cup [-3,4)$\\
D.$x \in (-\infty,-5] \cup (-3,4)$\\
E.$x \in (-\infty,-5] \cup (-3,4]$\\
F.$x \in (-\infty,-5] \cup [-3,4)$\\
G.$x \in (-\infty,-5) \cup [-3,4]$\\
H.$x \in (-\infty,-5] \cup [-3,4]$
\testStop
\kluczStart
A
\kluczStop



\zadStart{Zadanie z Wikieł Z 1.62 b) moja wersja nr 151}

Rozwiązać nierówności $(x+6)(4-x)(x+3)\ge0$.
\zadStop
\rozwStart{Patryk Wirkus}{Laura Mieczkowska}
Miejsca zerowe naszego wielomianu to: $-6, 4, -3$.\\
Wielomian jest stopnia nieparzystego, ponadto znak współczynnika przy\linebreak najwyższej potędze x jest ujemny.\\ W związku z tym wykres wielomianu zaczyna się od lewej strony powyżej osi OX. A więc $$x \in (-\infty,-6) \cup (-3,4).$$
\rozwStop
\odpStart
$x \in (-\infty,-6) \cup (-3,4)$
\odpStop
\testStart
A.$x \in (-\infty,-6) \cup (-3,4)$\\
B.$x \in (-\infty,-6) \cup (-3,4]$\\
C.$x \in (-\infty,-6) \cup [-3,4)$\\
D.$x \in (-\infty,-6] \cup (-3,4)$\\
E.$x \in (-\infty,-6] \cup (-3,4]$\\
F.$x \in (-\infty,-6] \cup [-3,4)$\\
G.$x \in (-\infty,-6) \cup [-3,4]$\\
H.$x \in (-\infty,-6] \cup [-3,4]$
\testStop
\kluczStart
A
\kluczStop



\zadStart{Zadanie z Wikieł Z 1.62 b) moja wersja nr 152}

Rozwiązać nierówności $(x+7)(4-x)(x+3)\ge0$.
\zadStop
\rozwStart{Patryk Wirkus}{Laura Mieczkowska}
Miejsca zerowe naszego wielomianu to: $-7, 4, -3$.\\
Wielomian jest stopnia nieparzystego, ponadto znak współczynnika przy\linebreak najwyższej potędze x jest ujemny.\\ W związku z tym wykres wielomianu zaczyna się od lewej strony powyżej osi OX. A więc $$x \in (-\infty,-7) \cup (-3,4).$$
\rozwStop
\odpStart
$x \in (-\infty,-7) \cup (-3,4)$
\odpStop
\testStart
A.$x \in (-\infty,-7) \cup (-3,4)$\\
B.$x \in (-\infty,-7) \cup (-3,4]$\\
C.$x \in (-\infty,-7) \cup [-3,4)$\\
D.$x \in (-\infty,-7] \cup (-3,4)$\\
E.$x \in (-\infty,-7] \cup (-3,4]$\\
F.$x \in (-\infty,-7] \cup [-3,4)$\\
G.$x \in (-\infty,-7) \cup [-3,4]$\\
H.$x \in (-\infty,-7] \cup [-3,4]$
\testStop
\kluczStart
A
\kluczStop



\zadStart{Zadanie z Wikieł Z 1.62 b) moja wersja nr 153}

Rozwiązać nierówności $(x+8)(4-x)(x+3)\ge0$.
\zadStop
\rozwStart{Patryk Wirkus}{Laura Mieczkowska}
Miejsca zerowe naszego wielomianu to: $-8, 4, -3$.\\
Wielomian jest stopnia nieparzystego, ponadto znak współczynnika przy\linebreak najwyższej potędze x jest ujemny.\\ W związku z tym wykres wielomianu zaczyna się od lewej strony powyżej osi OX. A więc $$x \in (-\infty,-8) \cup (-3,4).$$
\rozwStop
\odpStart
$x \in (-\infty,-8) \cup (-3,4)$
\odpStop
\testStart
A.$x \in (-\infty,-8) \cup (-3,4)$\\
B.$x \in (-\infty,-8) \cup (-3,4]$\\
C.$x \in (-\infty,-8) \cup [-3,4)$\\
D.$x \in (-\infty,-8] \cup (-3,4)$\\
E.$x \in (-\infty,-8] \cup (-3,4]$\\
F.$x \in (-\infty,-8] \cup [-3,4)$\\
G.$x \in (-\infty,-8) \cup [-3,4]$\\
H.$x \in (-\infty,-8] \cup [-3,4]$
\testStop
\kluczStart
A
\kluczStop



\zadStart{Zadanie z Wikieł Z 1.62 b) moja wersja nr 154}

Rozwiązać nierówności $(x+9)(4-x)(x+3)\ge0$.
\zadStop
\rozwStart{Patryk Wirkus}{Laura Mieczkowska}
Miejsca zerowe naszego wielomianu to: $-9, 4, -3$.\\
Wielomian jest stopnia nieparzystego, ponadto znak współczynnika przy\linebreak najwyższej potędze x jest ujemny.\\ W związku z tym wykres wielomianu zaczyna się od lewej strony powyżej osi OX. A więc $$x \in (-\infty,-9) \cup (-3,4).$$
\rozwStop
\odpStart
$x \in (-\infty,-9) \cup (-3,4)$
\odpStop
\testStart
A.$x \in (-\infty,-9) \cup (-3,4)$\\
B.$x \in (-\infty,-9) \cup (-3,4]$\\
C.$x \in (-\infty,-9) \cup [-3,4)$\\
D.$x \in (-\infty,-9] \cup (-3,4)$\\
E.$x \in (-\infty,-9] \cup (-3,4]$\\
F.$x \in (-\infty,-9] \cup [-3,4)$\\
G.$x \in (-\infty,-9) \cup [-3,4]$\\
H.$x \in (-\infty,-9] \cup [-3,4]$
\testStop
\kluczStart
A
\kluczStop



\zadStart{Zadanie z Wikieł Z 1.62 b) moja wersja nr 155}

Rozwiązać nierówności $(x+10)(4-x)(x+3)\ge0$.
\zadStop
\rozwStart{Patryk Wirkus}{Laura Mieczkowska}
Miejsca zerowe naszego wielomianu to: $-10, 4, -3$.\\
Wielomian jest stopnia nieparzystego, ponadto znak współczynnika przy\linebreak najwyższej potędze x jest ujemny.\\ W związku z tym wykres wielomianu zaczyna się od lewej strony powyżej osi OX. A więc $$x \in (-\infty,-10) \cup (-3,4).$$
\rozwStop
\odpStart
$x \in (-\infty,-10) \cup (-3,4)$
\odpStop
\testStart
A.$x \in (-\infty,-10) \cup (-3,4)$\\
B.$x \in (-\infty,-10) \cup (-3,4]$\\
C.$x \in (-\infty,-10) \cup [-3,4)$\\
D.$x \in (-\infty,-10] \cup (-3,4)$\\
E.$x \in (-\infty,-10] \cup (-3,4]$\\
F.$x \in (-\infty,-10] \cup [-3,4)$\\
G.$x \in (-\infty,-10) \cup [-3,4]$\\
H.$x \in (-\infty,-10] \cup [-3,4]$
\testStop
\kluczStart
A
\kluczStop



\zadStart{Zadanie z Wikieł Z 1.62 b) moja wersja nr 156}

Rozwiązać nierówności $(x+11)(4-x)(x+3)\ge0$.
\zadStop
\rozwStart{Patryk Wirkus}{Laura Mieczkowska}
Miejsca zerowe naszego wielomianu to: $-11, 4, -3$.\\
Wielomian jest stopnia nieparzystego, ponadto znak współczynnika przy\linebreak najwyższej potędze x jest ujemny.\\ W związku z tym wykres wielomianu zaczyna się od lewej strony powyżej osi OX. A więc $$x \in (-\infty,-11) \cup (-3,4).$$
\rozwStop
\odpStart
$x \in (-\infty,-11) \cup (-3,4)$
\odpStop
\testStart
A.$x \in (-\infty,-11) \cup (-3,4)$\\
B.$x \in (-\infty,-11) \cup (-3,4]$\\
C.$x \in (-\infty,-11) \cup [-3,4)$\\
D.$x \in (-\infty,-11] \cup (-3,4)$\\
E.$x \in (-\infty,-11] \cup (-3,4]$\\
F.$x \in (-\infty,-11] \cup [-3,4)$\\
G.$x \in (-\infty,-11) \cup [-3,4]$\\
H.$x \in (-\infty,-11] \cup [-3,4]$
\testStop
\kluczStart
A
\kluczStop



\zadStart{Zadanie z Wikieł Z 1.62 b) moja wersja nr 157}

Rozwiązać nierówności $(x+12)(4-x)(x+3)\ge0$.
\zadStop
\rozwStart{Patryk Wirkus}{Laura Mieczkowska}
Miejsca zerowe naszego wielomianu to: $-12, 4, -3$.\\
Wielomian jest stopnia nieparzystego, ponadto znak współczynnika przy\linebreak najwyższej potędze x jest ujemny.\\ W związku z tym wykres wielomianu zaczyna się od lewej strony powyżej osi OX. A więc $$x \in (-\infty,-12) \cup (-3,4).$$
\rozwStop
\odpStart
$x \in (-\infty,-12) \cup (-3,4)$
\odpStop
\testStart
A.$x \in (-\infty,-12) \cup (-3,4)$\\
B.$x \in (-\infty,-12) \cup (-3,4]$\\
C.$x \in (-\infty,-12) \cup [-3,4)$\\
D.$x \in (-\infty,-12] \cup (-3,4)$\\
E.$x \in (-\infty,-12] \cup (-3,4]$\\
F.$x \in (-\infty,-12] \cup [-3,4)$\\
G.$x \in (-\infty,-12) \cup [-3,4]$\\
H.$x \in (-\infty,-12] \cup [-3,4]$
\testStop
\kluczStart
A
\kluczStop



\zadStart{Zadanie z Wikieł Z 1.62 b) moja wersja nr 158}

Rozwiązać nierówności $(x+13)(4-x)(x+3)\ge0$.
\zadStop
\rozwStart{Patryk Wirkus}{Laura Mieczkowska}
Miejsca zerowe naszego wielomianu to: $-13, 4, -3$.\\
Wielomian jest stopnia nieparzystego, ponadto znak współczynnika przy\linebreak najwyższej potędze x jest ujemny.\\ W związku z tym wykres wielomianu zaczyna się od lewej strony powyżej osi OX. A więc $$x \in (-\infty,-13) \cup (-3,4).$$
\rozwStop
\odpStart
$x \in (-\infty,-13) \cup (-3,4)$
\odpStop
\testStart
A.$x \in (-\infty,-13) \cup (-3,4)$\\
B.$x \in (-\infty,-13) \cup (-3,4]$\\
C.$x \in (-\infty,-13) \cup [-3,4)$\\
D.$x \in (-\infty,-13] \cup (-3,4)$\\
E.$x \in (-\infty,-13] \cup (-3,4]$\\
F.$x \in (-\infty,-13] \cup [-3,4)$\\
G.$x \in (-\infty,-13) \cup [-3,4]$\\
H.$x \in (-\infty,-13] \cup [-3,4]$
\testStop
\kluczStart
A
\kluczStop



\zadStart{Zadanie z Wikieł Z 1.62 b) moja wersja nr 159}

Rozwiązać nierówności $(x+14)(4-x)(x+3)\ge0$.
\zadStop
\rozwStart{Patryk Wirkus}{Laura Mieczkowska}
Miejsca zerowe naszego wielomianu to: $-14, 4, -3$.\\
Wielomian jest stopnia nieparzystego, ponadto znak współczynnika przy\linebreak najwyższej potędze x jest ujemny.\\ W związku z tym wykres wielomianu zaczyna się od lewej strony powyżej osi OX. A więc $$x \in (-\infty,-14) \cup (-3,4).$$
\rozwStop
\odpStart
$x \in (-\infty,-14) \cup (-3,4)$
\odpStop
\testStart
A.$x \in (-\infty,-14) \cup (-3,4)$\\
B.$x \in (-\infty,-14) \cup (-3,4]$\\
C.$x \in (-\infty,-14) \cup [-3,4)$\\
D.$x \in (-\infty,-14] \cup (-3,4)$\\
E.$x \in (-\infty,-14] \cup (-3,4]$\\
F.$x \in (-\infty,-14] \cup [-3,4)$\\
G.$x \in (-\infty,-14) \cup [-3,4]$\\
H.$x \in (-\infty,-14] \cup [-3,4]$
\testStop
\kluczStart
A
\kluczStop



\zadStart{Zadanie z Wikieł Z 1.62 b) moja wersja nr 160}

Rozwiązać nierówności $(x+15)(4-x)(x+3)\ge0$.
\zadStop
\rozwStart{Patryk Wirkus}{Laura Mieczkowska}
Miejsca zerowe naszego wielomianu to: $-15, 4, -3$.\\
Wielomian jest stopnia nieparzystego, ponadto znak współczynnika przy\linebreak najwyższej potędze x jest ujemny.\\ W związku z tym wykres wielomianu zaczyna się od lewej strony powyżej osi OX. A więc $$x \in (-\infty,-15) \cup (-3,4).$$
\rozwStop
\odpStart
$x \in (-\infty,-15) \cup (-3,4)$
\odpStop
\testStart
A.$x \in (-\infty,-15) \cup (-3,4)$\\
B.$x \in (-\infty,-15) \cup (-3,4]$\\
C.$x \in (-\infty,-15) \cup [-3,4)$\\
D.$x \in (-\infty,-15] \cup (-3,4)$\\
E.$x \in (-\infty,-15] \cup (-3,4]$\\
F.$x \in (-\infty,-15] \cup [-3,4)$\\
G.$x \in (-\infty,-15) \cup [-3,4]$\\
H.$x \in (-\infty,-15] \cup [-3,4]$
\testStop
\kluczStart
A
\kluczStop



\zadStart{Zadanie z Wikieł Z 1.62 b) moja wersja nr 161}

Rozwiązać nierówności $(x+6)(5-x)(x+3)\ge0$.
\zadStop
\rozwStart{Patryk Wirkus}{Laura Mieczkowska}
Miejsca zerowe naszego wielomianu to: $-6, 5, -3$.\\
Wielomian jest stopnia nieparzystego, ponadto znak współczynnika przy\linebreak najwyższej potędze x jest ujemny.\\ W związku z tym wykres wielomianu zaczyna się od lewej strony powyżej osi OX. A więc $$x \in (-\infty,-6) \cup (-3,5).$$
\rozwStop
\odpStart
$x \in (-\infty,-6) \cup (-3,5)$
\odpStop
\testStart
A.$x \in (-\infty,-6) \cup (-3,5)$\\
B.$x \in (-\infty,-6) \cup (-3,5]$\\
C.$x \in (-\infty,-6) \cup [-3,5)$\\
D.$x \in (-\infty,-6] \cup (-3,5)$\\
E.$x \in (-\infty,-6] \cup (-3,5]$\\
F.$x \in (-\infty,-6] \cup [-3,5)$\\
G.$x \in (-\infty,-6) \cup [-3,5]$\\
H.$x \in (-\infty,-6] \cup [-3,5]$
\testStop
\kluczStart
A
\kluczStop



\zadStart{Zadanie z Wikieł Z 1.62 b) moja wersja nr 162}

Rozwiązać nierówności $(x+7)(5-x)(x+3)\ge0$.
\zadStop
\rozwStart{Patryk Wirkus}{Laura Mieczkowska}
Miejsca zerowe naszego wielomianu to: $-7, 5, -3$.\\
Wielomian jest stopnia nieparzystego, ponadto znak współczynnika przy\linebreak najwyższej potędze x jest ujemny.\\ W związku z tym wykres wielomianu zaczyna się od lewej strony powyżej osi OX. A więc $$x \in (-\infty,-7) \cup (-3,5).$$
\rozwStop
\odpStart
$x \in (-\infty,-7) \cup (-3,5)$
\odpStop
\testStart
A.$x \in (-\infty,-7) \cup (-3,5)$\\
B.$x \in (-\infty,-7) \cup (-3,5]$\\
C.$x \in (-\infty,-7) \cup [-3,5)$\\
D.$x \in (-\infty,-7] \cup (-3,5)$\\
E.$x \in (-\infty,-7] \cup (-3,5]$\\
F.$x \in (-\infty,-7] \cup [-3,5)$\\
G.$x \in (-\infty,-7) \cup [-3,5]$\\
H.$x \in (-\infty,-7] \cup [-3,5]$
\testStop
\kluczStart
A
\kluczStop



\zadStart{Zadanie z Wikieł Z 1.62 b) moja wersja nr 163}

Rozwiązać nierówności $(x+8)(5-x)(x+3)\ge0$.
\zadStop
\rozwStart{Patryk Wirkus}{Laura Mieczkowska}
Miejsca zerowe naszego wielomianu to: $-8, 5, -3$.\\
Wielomian jest stopnia nieparzystego, ponadto znak współczynnika przy\linebreak najwyższej potędze x jest ujemny.\\ W związku z tym wykres wielomianu zaczyna się od lewej strony powyżej osi OX. A więc $$x \in (-\infty,-8) \cup (-3,5).$$
\rozwStop
\odpStart
$x \in (-\infty,-8) \cup (-3,5)$
\odpStop
\testStart
A.$x \in (-\infty,-8) \cup (-3,5)$\\
B.$x \in (-\infty,-8) \cup (-3,5]$\\
C.$x \in (-\infty,-8) \cup [-3,5)$\\
D.$x \in (-\infty,-8] \cup (-3,5)$\\
E.$x \in (-\infty,-8] \cup (-3,5]$\\
F.$x \in (-\infty,-8] \cup [-3,5)$\\
G.$x \in (-\infty,-8) \cup [-3,5]$\\
H.$x \in (-\infty,-8] \cup [-3,5]$
\testStop
\kluczStart
A
\kluczStop



\zadStart{Zadanie z Wikieł Z 1.62 b) moja wersja nr 164}

Rozwiązać nierówności $(x+9)(5-x)(x+3)\ge0$.
\zadStop
\rozwStart{Patryk Wirkus}{Laura Mieczkowska}
Miejsca zerowe naszego wielomianu to: $-9, 5, -3$.\\
Wielomian jest stopnia nieparzystego, ponadto znak współczynnika przy\linebreak najwyższej potędze x jest ujemny.\\ W związku z tym wykres wielomianu zaczyna się od lewej strony powyżej osi OX. A więc $$x \in (-\infty,-9) \cup (-3,5).$$
\rozwStop
\odpStart
$x \in (-\infty,-9) \cup (-3,5)$
\odpStop
\testStart
A.$x \in (-\infty,-9) \cup (-3,5)$\\
B.$x \in (-\infty,-9) \cup (-3,5]$\\
C.$x \in (-\infty,-9) \cup [-3,5)$\\
D.$x \in (-\infty,-9] \cup (-3,5)$\\
E.$x \in (-\infty,-9] \cup (-3,5]$\\
F.$x \in (-\infty,-9] \cup [-3,5)$\\
G.$x \in (-\infty,-9) \cup [-3,5]$\\
H.$x \in (-\infty,-9] \cup [-3,5]$
\testStop
\kluczStart
A
\kluczStop



\zadStart{Zadanie z Wikieł Z 1.62 b) moja wersja nr 165}

Rozwiązać nierówności $(x+10)(5-x)(x+3)\ge0$.
\zadStop
\rozwStart{Patryk Wirkus}{Laura Mieczkowska}
Miejsca zerowe naszego wielomianu to: $-10, 5, -3$.\\
Wielomian jest stopnia nieparzystego, ponadto znak współczynnika przy\linebreak najwyższej potędze x jest ujemny.\\ W związku z tym wykres wielomianu zaczyna się od lewej strony powyżej osi OX. A więc $$x \in (-\infty,-10) \cup (-3,5).$$
\rozwStop
\odpStart
$x \in (-\infty,-10) \cup (-3,5)$
\odpStop
\testStart
A.$x \in (-\infty,-10) \cup (-3,5)$\\
B.$x \in (-\infty,-10) \cup (-3,5]$\\
C.$x \in (-\infty,-10) \cup [-3,5)$\\
D.$x \in (-\infty,-10] \cup (-3,5)$\\
E.$x \in (-\infty,-10] \cup (-3,5]$\\
F.$x \in (-\infty,-10] \cup [-3,5)$\\
G.$x \in (-\infty,-10) \cup [-3,5]$\\
H.$x \in (-\infty,-10] \cup [-3,5]$
\testStop
\kluczStart
A
\kluczStop



\zadStart{Zadanie z Wikieł Z 1.62 b) moja wersja nr 166}

Rozwiązać nierówności $(x+11)(5-x)(x+3)\ge0$.
\zadStop
\rozwStart{Patryk Wirkus}{Laura Mieczkowska}
Miejsca zerowe naszego wielomianu to: $-11, 5, -3$.\\
Wielomian jest stopnia nieparzystego, ponadto znak współczynnika przy\linebreak najwyższej potędze x jest ujemny.\\ W związku z tym wykres wielomianu zaczyna się od lewej strony powyżej osi OX. A więc $$x \in (-\infty,-11) \cup (-3,5).$$
\rozwStop
\odpStart
$x \in (-\infty,-11) \cup (-3,5)$
\odpStop
\testStart
A.$x \in (-\infty,-11) \cup (-3,5)$\\
B.$x \in (-\infty,-11) \cup (-3,5]$\\
C.$x \in (-\infty,-11) \cup [-3,5)$\\
D.$x \in (-\infty,-11] \cup (-3,5)$\\
E.$x \in (-\infty,-11] \cup (-3,5]$\\
F.$x \in (-\infty,-11] \cup [-3,5)$\\
G.$x \in (-\infty,-11) \cup [-3,5]$\\
H.$x \in (-\infty,-11] \cup [-3,5]$
\testStop
\kluczStart
A
\kluczStop



\zadStart{Zadanie z Wikieł Z 1.62 b) moja wersja nr 167}

Rozwiązać nierówności $(x+12)(5-x)(x+3)\ge0$.
\zadStop
\rozwStart{Patryk Wirkus}{Laura Mieczkowska}
Miejsca zerowe naszego wielomianu to: $-12, 5, -3$.\\
Wielomian jest stopnia nieparzystego, ponadto znak współczynnika przy\linebreak najwyższej potędze x jest ujemny.\\ W związku z tym wykres wielomianu zaczyna się od lewej strony powyżej osi OX. A więc $$x \in (-\infty,-12) \cup (-3,5).$$
\rozwStop
\odpStart
$x \in (-\infty,-12) \cup (-3,5)$
\odpStop
\testStart
A.$x \in (-\infty,-12) \cup (-3,5)$\\
B.$x \in (-\infty,-12) \cup (-3,5]$\\
C.$x \in (-\infty,-12) \cup [-3,5)$\\
D.$x \in (-\infty,-12] \cup (-3,5)$\\
E.$x \in (-\infty,-12] \cup (-3,5]$\\
F.$x \in (-\infty,-12] \cup [-3,5)$\\
G.$x \in (-\infty,-12) \cup [-3,5]$\\
H.$x \in (-\infty,-12] \cup [-3,5]$
\testStop
\kluczStart
A
\kluczStop



\zadStart{Zadanie z Wikieł Z 1.62 b) moja wersja nr 168}

Rozwiązać nierówności $(x+13)(5-x)(x+3)\ge0$.
\zadStop
\rozwStart{Patryk Wirkus}{Laura Mieczkowska}
Miejsca zerowe naszego wielomianu to: $-13, 5, -3$.\\
Wielomian jest stopnia nieparzystego, ponadto znak współczynnika przy\linebreak najwyższej potędze x jest ujemny.\\ W związku z tym wykres wielomianu zaczyna się od lewej strony powyżej osi OX. A więc $$x \in (-\infty,-13) \cup (-3,5).$$
\rozwStop
\odpStart
$x \in (-\infty,-13) \cup (-3,5)$
\odpStop
\testStart
A.$x \in (-\infty,-13) \cup (-3,5)$\\
B.$x \in (-\infty,-13) \cup (-3,5]$\\
C.$x \in (-\infty,-13) \cup [-3,5)$\\
D.$x \in (-\infty,-13] \cup (-3,5)$\\
E.$x \in (-\infty,-13] \cup (-3,5]$\\
F.$x \in (-\infty,-13] \cup [-3,5)$\\
G.$x \in (-\infty,-13) \cup [-3,5]$\\
H.$x \in (-\infty,-13] \cup [-3,5]$
\testStop
\kluczStart
A
\kluczStop



\zadStart{Zadanie z Wikieł Z 1.62 b) moja wersja nr 169}

Rozwiązać nierówności $(x+14)(5-x)(x+3)\ge0$.
\zadStop
\rozwStart{Patryk Wirkus}{Laura Mieczkowska}
Miejsca zerowe naszego wielomianu to: $-14, 5, -3$.\\
Wielomian jest stopnia nieparzystego, ponadto znak współczynnika przy\linebreak najwyższej potędze x jest ujemny.\\ W związku z tym wykres wielomianu zaczyna się od lewej strony powyżej osi OX. A więc $$x \in (-\infty,-14) \cup (-3,5).$$
\rozwStop
\odpStart
$x \in (-\infty,-14) \cup (-3,5)$
\odpStop
\testStart
A.$x \in (-\infty,-14) \cup (-3,5)$\\
B.$x \in (-\infty,-14) \cup (-3,5]$\\
C.$x \in (-\infty,-14) \cup [-3,5)$\\
D.$x \in (-\infty,-14] \cup (-3,5)$\\
E.$x \in (-\infty,-14] \cup (-3,5]$\\
F.$x \in (-\infty,-14] \cup [-3,5)$\\
G.$x \in (-\infty,-14) \cup [-3,5]$\\
H.$x \in (-\infty,-14] \cup [-3,5]$
\testStop
\kluczStart
A
\kluczStop



\zadStart{Zadanie z Wikieł Z 1.62 b) moja wersja nr 170}

Rozwiązać nierówności $(x+15)(5-x)(x+3)\ge0$.
\zadStop
\rozwStart{Patryk Wirkus}{Laura Mieczkowska}
Miejsca zerowe naszego wielomianu to: $-15, 5, -3$.\\
Wielomian jest stopnia nieparzystego, ponadto znak współczynnika przy\linebreak najwyższej potędze x jest ujemny.\\ W związku z tym wykres wielomianu zaczyna się od lewej strony powyżej osi OX. A więc $$x \in (-\infty,-15) \cup (-3,5).$$
\rozwStop
\odpStart
$x \in (-\infty,-15) \cup (-3,5)$
\odpStop
\testStart
A.$x \in (-\infty,-15) \cup (-3,5)$\\
B.$x \in (-\infty,-15) \cup (-3,5]$\\
C.$x \in (-\infty,-15) \cup [-3,5)$\\
D.$x \in (-\infty,-15] \cup (-3,5)$\\
E.$x \in (-\infty,-15] \cup (-3,5]$\\
F.$x \in (-\infty,-15] \cup [-3,5)$\\
G.$x \in (-\infty,-15) \cup [-3,5]$\\
H.$x \in (-\infty,-15] \cup [-3,5]$
\testStop
\kluczStart
A
\kluczStop



\zadStart{Zadanie z Wikieł Z 1.62 b) moja wersja nr 171}

Rozwiązać nierówności $(x+7)(6-x)(x+3)\ge0$.
\zadStop
\rozwStart{Patryk Wirkus}{Laura Mieczkowska}
Miejsca zerowe naszego wielomianu to: $-7, 6, -3$.\\
Wielomian jest stopnia nieparzystego, ponadto znak współczynnika przy\linebreak najwyższej potędze x jest ujemny.\\ W związku z tym wykres wielomianu zaczyna się od lewej strony powyżej osi OX. A więc $$x \in (-\infty,-7) \cup (-3,6).$$
\rozwStop
\odpStart
$x \in (-\infty,-7) \cup (-3,6)$
\odpStop
\testStart
A.$x \in (-\infty,-7) \cup (-3,6)$\\
B.$x \in (-\infty,-7) \cup (-3,6]$\\
C.$x \in (-\infty,-7) \cup [-3,6)$\\
D.$x \in (-\infty,-7] \cup (-3,6)$\\
E.$x \in (-\infty,-7] \cup (-3,6]$\\
F.$x \in (-\infty,-7] \cup [-3,6)$\\
G.$x \in (-\infty,-7) \cup [-3,6]$\\
H.$x \in (-\infty,-7] \cup [-3,6]$
\testStop
\kluczStart
A
\kluczStop



\zadStart{Zadanie z Wikieł Z 1.62 b) moja wersja nr 172}

Rozwiązać nierówności $(x+8)(6-x)(x+3)\ge0$.
\zadStop
\rozwStart{Patryk Wirkus}{Laura Mieczkowska}
Miejsca zerowe naszego wielomianu to: $-8, 6, -3$.\\
Wielomian jest stopnia nieparzystego, ponadto znak współczynnika przy\linebreak najwyższej potędze x jest ujemny.\\ W związku z tym wykres wielomianu zaczyna się od lewej strony powyżej osi OX. A więc $$x \in (-\infty,-8) \cup (-3,6).$$
\rozwStop
\odpStart
$x \in (-\infty,-8) \cup (-3,6)$
\odpStop
\testStart
A.$x \in (-\infty,-8) \cup (-3,6)$\\
B.$x \in (-\infty,-8) \cup (-3,6]$\\
C.$x \in (-\infty,-8) \cup [-3,6)$\\
D.$x \in (-\infty,-8] \cup (-3,6)$\\
E.$x \in (-\infty,-8] \cup (-3,6]$\\
F.$x \in (-\infty,-8] \cup [-3,6)$\\
G.$x \in (-\infty,-8) \cup [-3,6]$\\
H.$x \in (-\infty,-8] \cup [-3,6]$
\testStop
\kluczStart
A
\kluczStop



\zadStart{Zadanie z Wikieł Z 1.62 b) moja wersja nr 173}

Rozwiązać nierówności $(x+9)(6-x)(x+3)\ge0$.
\zadStop
\rozwStart{Patryk Wirkus}{Laura Mieczkowska}
Miejsca zerowe naszego wielomianu to: $-9, 6, -3$.\\
Wielomian jest stopnia nieparzystego, ponadto znak współczynnika przy\linebreak najwyższej potędze x jest ujemny.\\ W związku z tym wykres wielomianu zaczyna się od lewej strony powyżej osi OX. A więc $$x \in (-\infty,-9) \cup (-3,6).$$
\rozwStop
\odpStart
$x \in (-\infty,-9) \cup (-3,6)$
\odpStop
\testStart
A.$x \in (-\infty,-9) \cup (-3,6)$\\
B.$x \in (-\infty,-9) \cup (-3,6]$\\
C.$x \in (-\infty,-9) \cup [-3,6)$\\
D.$x \in (-\infty,-9] \cup (-3,6)$\\
E.$x \in (-\infty,-9] \cup (-3,6]$\\
F.$x \in (-\infty,-9] \cup [-3,6)$\\
G.$x \in (-\infty,-9) \cup [-3,6]$\\
H.$x \in (-\infty,-9] \cup [-3,6]$
\testStop
\kluczStart
A
\kluczStop



\zadStart{Zadanie z Wikieł Z 1.62 b) moja wersja nr 174}

Rozwiązać nierówności $(x+10)(6-x)(x+3)\ge0$.
\zadStop
\rozwStart{Patryk Wirkus}{Laura Mieczkowska}
Miejsca zerowe naszego wielomianu to: $-10, 6, -3$.\\
Wielomian jest stopnia nieparzystego, ponadto znak współczynnika przy\linebreak najwyższej potędze x jest ujemny.\\ W związku z tym wykres wielomianu zaczyna się od lewej strony powyżej osi OX. A więc $$x \in (-\infty,-10) \cup (-3,6).$$
\rozwStop
\odpStart
$x \in (-\infty,-10) \cup (-3,6)$
\odpStop
\testStart
A.$x \in (-\infty,-10) \cup (-3,6)$\\
B.$x \in (-\infty,-10) \cup (-3,6]$\\
C.$x \in (-\infty,-10) \cup [-3,6)$\\
D.$x \in (-\infty,-10] \cup (-3,6)$\\
E.$x \in (-\infty,-10] \cup (-3,6]$\\
F.$x \in (-\infty,-10] \cup [-3,6)$\\
G.$x \in (-\infty,-10) \cup [-3,6]$\\
H.$x \in (-\infty,-10] \cup [-3,6]$
\testStop
\kluczStart
A
\kluczStop



\zadStart{Zadanie z Wikieł Z 1.62 b) moja wersja nr 175}

Rozwiązać nierówności $(x+11)(6-x)(x+3)\ge0$.
\zadStop
\rozwStart{Patryk Wirkus}{Laura Mieczkowska}
Miejsca zerowe naszego wielomianu to: $-11, 6, -3$.\\
Wielomian jest stopnia nieparzystego, ponadto znak współczynnika przy\linebreak najwyższej potędze x jest ujemny.\\ W związku z tym wykres wielomianu zaczyna się od lewej strony powyżej osi OX. A więc $$x \in (-\infty,-11) \cup (-3,6).$$
\rozwStop
\odpStart
$x \in (-\infty,-11) \cup (-3,6)$
\odpStop
\testStart
A.$x \in (-\infty,-11) \cup (-3,6)$\\
B.$x \in (-\infty,-11) \cup (-3,6]$\\
C.$x \in (-\infty,-11) \cup [-3,6)$\\
D.$x \in (-\infty,-11] \cup (-3,6)$\\
E.$x \in (-\infty,-11] \cup (-3,6]$\\
F.$x \in (-\infty,-11] \cup [-3,6)$\\
G.$x \in (-\infty,-11) \cup [-3,6]$\\
H.$x \in (-\infty,-11] \cup [-3,6]$
\testStop
\kluczStart
A
\kluczStop



\zadStart{Zadanie z Wikieł Z 1.62 b) moja wersja nr 176}

Rozwiązać nierówności $(x+12)(6-x)(x+3)\ge0$.
\zadStop
\rozwStart{Patryk Wirkus}{Laura Mieczkowska}
Miejsca zerowe naszego wielomianu to: $-12, 6, -3$.\\
Wielomian jest stopnia nieparzystego, ponadto znak współczynnika przy\linebreak najwyższej potędze x jest ujemny.\\ W związku z tym wykres wielomianu zaczyna się od lewej strony powyżej osi OX. A więc $$x \in (-\infty,-12) \cup (-3,6).$$
\rozwStop
\odpStart
$x \in (-\infty,-12) \cup (-3,6)$
\odpStop
\testStart
A.$x \in (-\infty,-12) \cup (-3,6)$\\
B.$x \in (-\infty,-12) \cup (-3,6]$\\
C.$x \in (-\infty,-12) \cup [-3,6)$\\
D.$x \in (-\infty,-12] \cup (-3,6)$\\
E.$x \in (-\infty,-12] \cup (-3,6]$\\
F.$x \in (-\infty,-12] \cup [-3,6)$\\
G.$x \in (-\infty,-12) \cup [-3,6]$\\
H.$x \in (-\infty,-12] \cup [-3,6]$
\testStop
\kluczStart
A
\kluczStop



\zadStart{Zadanie z Wikieł Z 1.62 b) moja wersja nr 177}

Rozwiązać nierówności $(x+13)(6-x)(x+3)\ge0$.
\zadStop
\rozwStart{Patryk Wirkus}{Laura Mieczkowska}
Miejsca zerowe naszego wielomianu to: $-13, 6, -3$.\\
Wielomian jest stopnia nieparzystego, ponadto znak współczynnika przy\linebreak najwyższej potędze x jest ujemny.\\ W związku z tym wykres wielomianu zaczyna się od lewej strony powyżej osi OX. A więc $$x \in (-\infty,-13) \cup (-3,6).$$
\rozwStop
\odpStart
$x \in (-\infty,-13) \cup (-3,6)$
\odpStop
\testStart
A.$x \in (-\infty,-13) \cup (-3,6)$\\
B.$x \in (-\infty,-13) \cup (-3,6]$\\
C.$x \in (-\infty,-13) \cup [-3,6)$\\
D.$x \in (-\infty,-13] \cup (-3,6)$\\
E.$x \in (-\infty,-13] \cup (-3,6]$\\
F.$x \in (-\infty,-13] \cup [-3,6)$\\
G.$x \in (-\infty,-13) \cup [-3,6]$\\
H.$x \in (-\infty,-13] \cup [-3,6]$
\testStop
\kluczStart
A
\kluczStop



\zadStart{Zadanie z Wikieł Z 1.62 b) moja wersja nr 178}

Rozwiązać nierówności $(x+14)(6-x)(x+3)\ge0$.
\zadStop
\rozwStart{Patryk Wirkus}{Laura Mieczkowska}
Miejsca zerowe naszego wielomianu to: $-14, 6, -3$.\\
Wielomian jest stopnia nieparzystego, ponadto znak współczynnika przy\linebreak najwyższej potędze x jest ujemny.\\ W związku z tym wykres wielomianu zaczyna się od lewej strony powyżej osi OX. A więc $$x \in (-\infty,-14) \cup (-3,6).$$
\rozwStop
\odpStart
$x \in (-\infty,-14) \cup (-3,6)$
\odpStop
\testStart
A.$x \in (-\infty,-14) \cup (-3,6)$\\
B.$x \in (-\infty,-14) \cup (-3,6]$\\
C.$x \in (-\infty,-14) \cup [-3,6)$\\
D.$x \in (-\infty,-14] \cup (-3,6)$\\
E.$x \in (-\infty,-14] \cup (-3,6]$\\
F.$x \in (-\infty,-14] \cup [-3,6)$\\
G.$x \in (-\infty,-14) \cup [-3,6]$\\
H.$x \in (-\infty,-14] \cup [-3,6]$
\testStop
\kluczStart
A
\kluczStop



\zadStart{Zadanie z Wikieł Z 1.62 b) moja wersja nr 179}

Rozwiązać nierówności $(x+15)(6-x)(x+3)\ge0$.
\zadStop
\rozwStart{Patryk Wirkus}{Laura Mieczkowska}
Miejsca zerowe naszego wielomianu to: $-15, 6, -3$.\\
Wielomian jest stopnia nieparzystego, ponadto znak współczynnika przy\linebreak najwyższej potędze x jest ujemny.\\ W związku z tym wykres wielomianu zaczyna się od lewej strony powyżej osi OX. A więc $$x \in (-\infty,-15) \cup (-3,6).$$
\rozwStop
\odpStart
$x \in (-\infty,-15) \cup (-3,6)$
\odpStop
\testStart
A.$x \in (-\infty,-15) \cup (-3,6)$\\
B.$x \in (-\infty,-15) \cup (-3,6]$\\
C.$x \in (-\infty,-15) \cup [-3,6)$\\
D.$x \in (-\infty,-15] \cup (-3,6)$\\
E.$x \in (-\infty,-15] \cup (-3,6]$\\
F.$x \in (-\infty,-15] \cup [-3,6)$\\
G.$x \in (-\infty,-15) \cup [-3,6]$\\
H.$x \in (-\infty,-15] \cup [-3,6]$
\testStop
\kluczStart
A
\kluczStop



\zadStart{Zadanie z Wikieł Z 1.62 b) moja wersja nr 180}

Rozwiązać nierówności $(x+8)(7-x)(x+3)\ge0$.
\zadStop
\rozwStart{Patryk Wirkus}{Laura Mieczkowska}
Miejsca zerowe naszego wielomianu to: $-8, 7, -3$.\\
Wielomian jest stopnia nieparzystego, ponadto znak współczynnika przy\linebreak najwyższej potędze x jest ujemny.\\ W związku z tym wykres wielomianu zaczyna się od lewej strony powyżej osi OX. A więc $$x \in (-\infty,-8) \cup (-3,7).$$
\rozwStop
\odpStart
$x \in (-\infty,-8) \cup (-3,7)$
\odpStop
\testStart
A.$x \in (-\infty,-8) \cup (-3,7)$\\
B.$x \in (-\infty,-8) \cup (-3,7]$\\
C.$x \in (-\infty,-8) \cup [-3,7)$\\
D.$x \in (-\infty,-8] \cup (-3,7)$\\
E.$x \in (-\infty,-8] \cup (-3,7]$\\
F.$x \in (-\infty,-8] \cup [-3,7)$\\
G.$x \in (-\infty,-8) \cup [-3,7]$\\
H.$x \in (-\infty,-8] \cup [-3,7]$
\testStop
\kluczStart
A
\kluczStop



\zadStart{Zadanie z Wikieł Z 1.62 b) moja wersja nr 181}

Rozwiązać nierówności $(x+9)(7-x)(x+3)\ge0$.
\zadStop
\rozwStart{Patryk Wirkus}{Laura Mieczkowska}
Miejsca zerowe naszego wielomianu to: $-9, 7, -3$.\\
Wielomian jest stopnia nieparzystego, ponadto znak współczynnika przy\linebreak najwyższej potędze x jest ujemny.\\ W związku z tym wykres wielomianu zaczyna się od lewej strony powyżej osi OX. A więc $$x \in (-\infty,-9) \cup (-3,7).$$
\rozwStop
\odpStart
$x \in (-\infty,-9) \cup (-3,7)$
\odpStop
\testStart
A.$x \in (-\infty,-9) \cup (-3,7)$\\
B.$x \in (-\infty,-9) \cup (-3,7]$\\
C.$x \in (-\infty,-9) \cup [-3,7)$\\
D.$x \in (-\infty,-9] \cup (-3,7)$\\
E.$x \in (-\infty,-9] \cup (-3,7]$\\
F.$x \in (-\infty,-9] \cup [-3,7)$\\
G.$x \in (-\infty,-9) \cup [-3,7]$\\
H.$x \in (-\infty,-9] \cup [-3,7]$
\testStop
\kluczStart
A
\kluczStop



\zadStart{Zadanie z Wikieł Z 1.62 b) moja wersja nr 182}

Rozwiązać nierówności $(x+10)(7-x)(x+3)\ge0$.
\zadStop
\rozwStart{Patryk Wirkus}{Laura Mieczkowska}
Miejsca zerowe naszego wielomianu to: $-10, 7, -3$.\\
Wielomian jest stopnia nieparzystego, ponadto znak współczynnika przy\linebreak najwyższej potędze x jest ujemny.\\ W związku z tym wykres wielomianu zaczyna się od lewej strony powyżej osi OX. A więc $$x \in (-\infty,-10) \cup (-3,7).$$
\rozwStop
\odpStart
$x \in (-\infty,-10) \cup (-3,7)$
\odpStop
\testStart
A.$x \in (-\infty,-10) \cup (-3,7)$\\
B.$x \in (-\infty,-10) \cup (-3,7]$\\
C.$x \in (-\infty,-10) \cup [-3,7)$\\
D.$x \in (-\infty,-10] \cup (-3,7)$\\
E.$x \in (-\infty,-10] \cup (-3,7]$\\
F.$x \in (-\infty,-10] \cup [-3,7)$\\
G.$x \in (-\infty,-10) \cup [-3,7]$\\
H.$x \in (-\infty,-10] \cup [-3,7]$
\testStop
\kluczStart
A
\kluczStop



\zadStart{Zadanie z Wikieł Z 1.62 b) moja wersja nr 183}

Rozwiązać nierówności $(x+11)(7-x)(x+3)\ge0$.
\zadStop
\rozwStart{Patryk Wirkus}{Laura Mieczkowska}
Miejsca zerowe naszego wielomianu to: $-11, 7, -3$.\\
Wielomian jest stopnia nieparzystego, ponadto znak współczynnika przy\linebreak najwyższej potędze x jest ujemny.\\ W związku z tym wykres wielomianu zaczyna się od lewej strony powyżej osi OX. A więc $$x \in (-\infty,-11) \cup (-3,7).$$
\rozwStop
\odpStart
$x \in (-\infty,-11) \cup (-3,7)$
\odpStop
\testStart
A.$x \in (-\infty,-11) \cup (-3,7)$\\
B.$x \in (-\infty,-11) \cup (-3,7]$\\
C.$x \in (-\infty,-11) \cup [-3,7)$\\
D.$x \in (-\infty,-11] \cup (-3,7)$\\
E.$x \in (-\infty,-11] \cup (-3,7]$\\
F.$x \in (-\infty,-11] \cup [-3,7)$\\
G.$x \in (-\infty,-11) \cup [-3,7]$\\
H.$x \in (-\infty,-11] \cup [-3,7]$
\testStop
\kluczStart
A
\kluczStop



\zadStart{Zadanie z Wikieł Z 1.62 b) moja wersja nr 184}

Rozwiązać nierówności $(x+12)(7-x)(x+3)\ge0$.
\zadStop
\rozwStart{Patryk Wirkus}{Laura Mieczkowska}
Miejsca zerowe naszego wielomianu to: $-12, 7, -3$.\\
Wielomian jest stopnia nieparzystego, ponadto znak współczynnika przy\linebreak najwyższej potędze x jest ujemny.\\ W związku z tym wykres wielomianu zaczyna się od lewej strony powyżej osi OX. A więc $$x \in (-\infty,-12) \cup (-3,7).$$
\rozwStop
\odpStart
$x \in (-\infty,-12) \cup (-3,7)$
\odpStop
\testStart
A.$x \in (-\infty,-12) \cup (-3,7)$\\
B.$x \in (-\infty,-12) \cup (-3,7]$\\
C.$x \in (-\infty,-12) \cup [-3,7)$\\
D.$x \in (-\infty,-12] \cup (-3,7)$\\
E.$x \in (-\infty,-12] \cup (-3,7]$\\
F.$x \in (-\infty,-12] \cup [-3,7)$\\
G.$x \in (-\infty,-12) \cup [-3,7]$\\
H.$x \in (-\infty,-12] \cup [-3,7]$
\testStop
\kluczStart
A
\kluczStop



\zadStart{Zadanie z Wikieł Z 1.62 b) moja wersja nr 185}

Rozwiązać nierówności $(x+13)(7-x)(x+3)\ge0$.
\zadStop
\rozwStart{Patryk Wirkus}{Laura Mieczkowska}
Miejsca zerowe naszego wielomianu to: $-13, 7, -3$.\\
Wielomian jest stopnia nieparzystego, ponadto znak współczynnika przy\linebreak najwyższej potędze x jest ujemny.\\ W związku z tym wykres wielomianu zaczyna się od lewej strony powyżej osi OX. A więc $$x \in (-\infty,-13) \cup (-3,7).$$
\rozwStop
\odpStart
$x \in (-\infty,-13) \cup (-3,7)$
\odpStop
\testStart
A.$x \in (-\infty,-13) \cup (-3,7)$\\
B.$x \in (-\infty,-13) \cup (-3,7]$\\
C.$x \in (-\infty,-13) \cup [-3,7)$\\
D.$x \in (-\infty,-13] \cup (-3,7)$\\
E.$x \in (-\infty,-13] \cup (-3,7]$\\
F.$x \in (-\infty,-13] \cup [-3,7)$\\
G.$x \in (-\infty,-13) \cup [-3,7]$\\
H.$x \in (-\infty,-13] \cup [-3,7]$
\testStop
\kluczStart
A
\kluczStop



\zadStart{Zadanie z Wikieł Z 1.62 b) moja wersja nr 186}

Rozwiązać nierówności $(x+14)(7-x)(x+3)\ge0$.
\zadStop
\rozwStart{Patryk Wirkus}{Laura Mieczkowska}
Miejsca zerowe naszego wielomianu to: $-14, 7, -3$.\\
Wielomian jest stopnia nieparzystego, ponadto znak współczynnika przy\linebreak najwyższej potędze x jest ujemny.\\ W związku z tym wykres wielomianu zaczyna się od lewej strony powyżej osi OX. A więc $$x \in (-\infty,-14) \cup (-3,7).$$
\rozwStop
\odpStart
$x \in (-\infty,-14) \cup (-3,7)$
\odpStop
\testStart
A.$x \in (-\infty,-14) \cup (-3,7)$\\
B.$x \in (-\infty,-14) \cup (-3,7]$\\
C.$x \in (-\infty,-14) \cup [-3,7)$\\
D.$x \in (-\infty,-14] \cup (-3,7)$\\
E.$x \in (-\infty,-14] \cup (-3,7]$\\
F.$x \in (-\infty,-14] \cup [-3,7)$\\
G.$x \in (-\infty,-14) \cup [-3,7]$\\
H.$x \in (-\infty,-14] \cup [-3,7]$
\testStop
\kluczStart
A
\kluczStop



\zadStart{Zadanie z Wikieł Z 1.62 b) moja wersja nr 187}

Rozwiązać nierówności $(x+15)(7-x)(x+3)\ge0$.
\zadStop
\rozwStart{Patryk Wirkus}{Laura Mieczkowska}
Miejsca zerowe naszego wielomianu to: $-15, 7, -3$.\\
Wielomian jest stopnia nieparzystego, ponadto znak współczynnika przy\linebreak najwyższej potędze x jest ujemny.\\ W związku z tym wykres wielomianu zaczyna się od lewej strony powyżej osi OX. A więc $$x \in (-\infty,-15) \cup (-3,7).$$
\rozwStop
\odpStart
$x \in (-\infty,-15) \cup (-3,7)$
\odpStop
\testStart
A.$x \in (-\infty,-15) \cup (-3,7)$\\
B.$x \in (-\infty,-15) \cup (-3,7]$\\
C.$x \in (-\infty,-15) \cup [-3,7)$\\
D.$x \in (-\infty,-15] \cup (-3,7)$\\
E.$x \in (-\infty,-15] \cup (-3,7]$\\
F.$x \in (-\infty,-15] \cup [-3,7)$\\
G.$x \in (-\infty,-15) \cup [-3,7]$\\
H.$x \in (-\infty,-15] \cup [-3,7]$
\testStop
\kluczStart
A
\kluczStop



\zadStart{Zadanie z Wikieł Z 1.62 b) moja wersja nr 188}

Rozwiązać nierówności $(x+9)(8-x)(x+3)\ge0$.
\zadStop
\rozwStart{Patryk Wirkus}{Laura Mieczkowska}
Miejsca zerowe naszego wielomianu to: $-9, 8, -3$.\\
Wielomian jest stopnia nieparzystego, ponadto znak współczynnika przy\linebreak najwyższej potędze x jest ujemny.\\ W związku z tym wykres wielomianu zaczyna się od lewej strony powyżej osi OX. A więc $$x \in (-\infty,-9) \cup (-3,8).$$
\rozwStop
\odpStart
$x \in (-\infty,-9) \cup (-3,8)$
\odpStop
\testStart
A.$x \in (-\infty,-9) \cup (-3,8)$\\
B.$x \in (-\infty,-9) \cup (-3,8]$\\
C.$x \in (-\infty,-9) \cup [-3,8)$\\
D.$x \in (-\infty,-9] \cup (-3,8)$\\
E.$x \in (-\infty,-9] \cup (-3,8]$\\
F.$x \in (-\infty,-9] \cup [-3,8)$\\
G.$x \in (-\infty,-9) \cup [-3,8]$\\
H.$x \in (-\infty,-9] \cup [-3,8]$
\testStop
\kluczStart
A
\kluczStop



\zadStart{Zadanie z Wikieł Z 1.62 b) moja wersja nr 189}

Rozwiązać nierówności $(x+10)(8-x)(x+3)\ge0$.
\zadStop
\rozwStart{Patryk Wirkus}{Laura Mieczkowska}
Miejsca zerowe naszego wielomianu to: $-10, 8, -3$.\\
Wielomian jest stopnia nieparzystego, ponadto znak współczynnika przy\linebreak najwyższej potędze x jest ujemny.\\ W związku z tym wykres wielomianu zaczyna się od lewej strony powyżej osi OX. A więc $$x \in (-\infty,-10) \cup (-3,8).$$
\rozwStop
\odpStart
$x \in (-\infty,-10) \cup (-3,8)$
\odpStop
\testStart
A.$x \in (-\infty,-10) \cup (-3,8)$\\
B.$x \in (-\infty,-10) \cup (-3,8]$\\
C.$x \in (-\infty,-10) \cup [-3,8)$\\
D.$x \in (-\infty,-10] \cup (-3,8)$\\
E.$x \in (-\infty,-10] \cup (-3,8]$\\
F.$x \in (-\infty,-10] \cup [-3,8)$\\
G.$x \in (-\infty,-10) \cup [-3,8]$\\
H.$x \in (-\infty,-10] \cup [-3,8]$
\testStop
\kluczStart
A
\kluczStop



\zadStart{Zadanie z Wikieł Z 1.62 b) moja wersja nr 190}

Rozwiązać nierówności $(x+11)(8-x)(x+3)\ge0$.
\zadStop
\rozwStart{Patryk Wirkus}{Laura Mieczkowska}
Miejsca zerowe naszego wielomianu to: $-11, 8, -3$.\\
Wielomian jest stopnia nieparzystego, ponadto znak współczynnika przy\linebreak najwyższej potędze x jest ujemny.\\ W związku z tym wykres wielomianu zaczyna się od lewej strony powyżej osi OX. A więc $$x \in (-\infty,-11) \cup (-3,8).$$
\rozwStop
\odpStart
$x \in (-\infty,-11) \cup (-3,8)$
\odpStop
\testStart
A.$x \in (-\infty,-11) \cup (-3,8)$\\
B.$x \in (-\infty,-11) \cup (-3,8]$\\
C.$x \in (-\infty,-11) \cup [-3,8)$\\
D.$x \in (-\infty,-11] \cup (-3,8)$\\
E.$x \in (-\infty,-11] \cup (-3,8]$\\
F.$x \in (-\infty,-11] \cup [-3,8)$\\
G.$x \in (-\infty,-11) \cup [-3,8]$\\
H.$x \in (-\infty,-11] \cup [-3,8]$
\testStop
\kluczStart
A
\kluczStop



\zadStart{Zadanie z Wikieł Z 1.62 b) moja wersja nr 191}

Rozwiązać nierówności $(x+12)(8-x)(x+3)\ge0$.
\zadStop
\rozwStart{Patryk Wirkus}{Laura Mieczkowska}
Miejsca zerowe naszego wielomianu to: $-12, 8, -3$.\\
Wielomian jest stopnia nieparzystego, ponadto znak współczynnika przy\linebreak najwyższej potędze x jest ujemny.\\ W związku z tym wykres wielomianu zaczyna się od lewej strony powyżej osi OX. A więc $$x \in (-\infty,-12) \cup (-3,8).$$
\rozwStop
\odpStart
$x \in (-\infty,-12) \cup (-3,8)$
\odpStop
\testStart
A.$x \in (-\infty,-12) \cup (-3,8)$\\
B.$x \in (-\infty,-12) \cup (-3,8]$\\
C.$x \in (-\infty,-12) \cup [-3,8)$\\
D.$x \in (-\infty,-12] \cup (-3,8)$\\
E.$x \in (-\infty,-12] \cup (-3,8]$\\
F.$x \in (-\infty,-12] \cup [-3,8)$\\
G.$x \in (-\infty,-12) \cup [-3,8]$\\
H.$x \in (-\infty,-12] \cup [-3,8]$
\testStop
\kluczStart
A
\kluczStop



\zadStart{Zadanie z Wikieł Z 1.62 b) moja wersja nr 192}

Rozwiązać nierówności $(x+13)(8-x)(x+3)\ge0$.
\zadStop
\rozwStart{Patryk Wirkus}{Laura Mieczkowska}
Miejsca zerowe naszego wielomianu to: $-13, 8, -3$.\\
Wielomian jest stopnia nieparzystego, ponadto znak współczynnika przy\linebreak najwyższej potędze x jest ujemny.\\ W związku z tym wykres wielomianu zaczyna się od lewej strony powyżej osi OX. A więc $$x \in (-\infty,-13) \cup (-3,8).$$
\rozwStop
\odpStart
$x \in (-\infty,-13) \cup (-3,8)$
\odpStop
\testStart
A.$x \in (-\infty,-13) \cup (-3,8)$\\
B.$x \in (-\infty,-13) \cup (-3,8]$\\
C.$x \in (-\infty,-13) \cup [-3,8)$\\
D.$x \in (-\infty,-13] \cup (-3,8)$\\
E.$x \in (-\infty,-13] \cup (-3,8]$\\
F.$x \in (-\infty,-13] \cup [-3,8)$\\
G.$x \in (-\infty,-13) \cup [-3,8]$\\
H.$x \in (-\infty,-13] \cup [-3,8]$
\testStop
\kluczStart
A
\kluczStop



\zadStart{Zadanie z Wikieł Z 1.62 b) moja wersja nr 193}

Rozwiązać nierówności $(x+14)(8-x)(x+3)\ge0$.
\zadStop
\rozwStart{Patryk Wirkus}{Laura Mieczkowska}
Miejsca zerowe naszego wielomianu to: $-14, 8, -3$.\\
Wielomian jest stopnia nieparzystego, ponadto znak współczynnika przy\linebreak najwyższej potędze x jest ujemny.\\ W związku z tym wykres wielomianu zaczyna się od lewej strony powyżej osi OX. A więc $$x \in (-\infty,-14) \cup (-3,8).$$
\rozwStop
\odpStart
$x \in (-\infty,-14) \cup (-3,8)$
\odpStop
\testStart
A.$x \in (-\infty,-14) \cup (-3,8)$\\
B.$x \in (-\infty,-14) \cup (-3,8]$\\
C.$x \in (-\infty,-14) \cup [-3,8)$\\
D.$x \in (-\infty,-14] \cup (-3,8)$\\
E.$x \in (-\infty,-14] \cup (-3,8]$\\
F.$x \in (-\infty,-14] \cup [-3,8)$\\
G.$x \in (-\infty,-14) \cup [-3,8]$\\
H.$x \in (-\infty,-14] \cup [-3,8]$
\testStop
\kluczStart
A
\kluczStop



\zadStart{Zadanie z Wikieł Z 1.62 b) moja wersja nr 194}

Rozwiązać nierówności $(x+15)(8-x)(x+3)\ge0$.
\zadStop
\rozwStart{Patryk Wirkus}{Laura Mieczkowska}
Miejsca zerowe naszego wielomianu to: $-15, 8, -3$.\\
Wielomian jest stopnia nieparzystego, ponadto znak współczynnika przy\linebreak najwyższej potędze x jest ujemny.\\ W związku z tym wykres wielomianu zaczyna się od lewej strony powyżej osi OX. A więc $$x \in (-\infty,-15) \cup (-3,8).$$
\rozwStop
\odpStart
$x \in (-\infty,-15) \cup (-3,8)$
\odpStop
\testStart
A.$x \in (-\infty,-15) \cup (-3,8)$\\
B.$x \in (-\infty,-15) \cup (-3,8]$\\
C.$x \in (-\infty,-15) \cup [-3,8)$\\
D.$x \in (-\infty,-15] \cup (-3,8)$\\
E.$x \in (-\infty,-15] \cup (-3,8]$\\
F.$x \in (-\infty,-15] \cup [-3,8)$\\
G.$x \in (-\infty,-15) \cup [-3,8]$\\
H.$x \in (-\infty,-15] \cup [-3,8]$
\testStop
\kluczStart
A
\kluczStop



\zadStart{Zadanie z Wikieł Z 1.62 b) moja wersja nr 195}

Rozwiązać nierówności $(x+10)(9-x)(x+3)\ge0$.
\zadStop
\rozwStart{Patryk Wirkus}{Laura Mieczkowska}
Miejsca zerowe naszego wielomianu to: $-10, 9, -3$.\\
Wielomian jest stopnia nieparzystego, ponadto znak współczynnika przy\linebreak najwyższej potędze x jest ujemny.\\ W związku z tym wykres wielomianu zaczyna się od lewej strony powyżej osi OX. A więc $$x \in (-\infty,-10) \cup (-3,9).$$
\rozwStop
\odpStart
$x \in (-\infty,-10) \cup (-3,9)$
\odpStop
\testStart
A.$x \in (-\infty,-10) \cup (-3,9)$\\
B.$x \in (-\infty,-10) \cup (-3,9]$\\
C.$x \in (-\infty,-10) \cup [-3,9)$\\
D.$x \in (-\infty,-10] \cup (-3,9)$\\
E.$x \in (-\infty,-10] \cup (-3,9]$\\
F.$x \in (-\infty,-10] \cup [-3,9)$\\
G.$x \in (-\infty,-10) \cup [-3,9]$\\
H.$x \in (-\infty,-10] \cup [-3,9]$
\testStop
\kluczStart
A
\kluczStop



\zadStart{Zadanie z Wikieł Z 1.62 b) moja wersja nr 196}

Rozwiązać nierówności $(x+11)(9-x)(x+3)\ge0$.
\zadStop
\rozwStart{Patryk Wirkus}{Laura Mieczkowska}
Miejsca zerowe naszego wielomianu to: $-11, 9, -3$.\\
Wielomian jest stopnia nieparzystego, ponadto znak współczynnika przy\linebreak najwyższej potędze x jest ujemny.\\ W związku z tym wykres wielomianu zaczyna się od lewej strony powyżej osi OX. A więc $$x \in (-\infty,-11) \cup (-3,9).$$
\rozwStop
\odpStart
$x \in (-\infty,-11) \cup (-3,9)$
\odpStop
\testStart
A.$x \in (-\infty,-11) \cup (-3,9)$\\
B.$x \in (-\infty,-11) \cup (-3,9]$\\
C.$x \in (-\infty,-11) \cup [-3,9)$\\
D.$x \in (-\infty,-11] \cup (-3,9)$\\
E.$x \in (-\infty,-11] \cup (-3,9]$\\
F.$x \in (-\infty,-11] \cup [-3,9)$\\
G.$x \in (-\infty,-11) \cup [-3,9]$\\
H.$x \in (-\infty,-11] \cup [-3,9]$
\testStop
\kluczStart
A
\kluczStop



\zadStart{Zadanie z Wikieł Z 1.62 b) moja wersja nr 197}

Rozwiązać nierówności $(x+12)(9-x)(x+3)\ge0$.
\zadStop
\rozwStart{Patryk Wirkus}{Laura Mieczkowska}
Miejsca zerowe naszego wielomianu to: $-12, 9, -3$.\\
Wielomian jest stopnia nieparzystego, ponadto znak współczynnika przy\linebreak najwyższej potędze x jest ujemny.\\ W związku z tym wykres wielomianu zaczyna się od lewej strony powyżej osi OX. A więc $$x \in (-\infty,-12) \cup (-3,9).$$
\rozwStop
\odpStart
$x \in (-\infty,-12) \cup (-3,9)$
\odpStop
\testStart
A.$x \in (-\infty,-12) \cup (-3,9)$\\
B.$x \in (-\infty,-12) \cup (-3,9]$\\
C.$x \in (-\infty,-12) \cup [-3,9)$\\
D.$x \in (-\infty,-12] \cup (-3,9)$\\
E.$x \in (-\infty,-12] \cup (-3,9]$\\
F.$x \in (-\infty,-12] \cup [-3,9)$\\
G.$x \in (-\infty,-12) \cup [-3,9]$\\
H.$x \in (-\infty,-12] \cup [-3,9]$
\testStop
\kluczStart
A
\kluczStop



\zadStart{Zadanie z Wikieł Z 1.62 b) moja wersja nr 198}

Rozwiązać nierówności $(x+13)(9-x)(x+3)\ge0$.
\zadStop
\rozwStart{Patryk Wirkus}{Laura Mieczkowska}
Miejsca zerowe naszego wielomianu to: $-13, 9, -3$.\\
Wielomian jest stopnia nieparzystego, ponadto znak współczynnika przy\linebreak najwyższej potędze x jest ujemny.\\ W związku z tym wykres wielomianu zaczyna się od lewej strony powyżej osi OX. A więc $$x \in (-\infty,-13) \cup (-3,9).$$
\rozwStop
\odpStart
$x \in (-\infty,-13) \cup (-3,9)$
\odpStop
\testStart
A.$x \in (-\infty,-13) \cup (-3,9)$\\
B.$x \in (-\infty,-13) \cup (-3,9]$\\
C.$x \in (-\infty,-13) \cup [-3,9)$\\
D.$x \in (-\infty,-13] \cup (-3,9)$\\
E.$x \in (-\infty,-13] \cup (-3,9]$\\
F.$x \in (-\infty,-13] \cup [-3,9)$\\
G.$x \in (-\infty,-13) \cup [-3,9]$\\
H.$x \in (-\infty,-13] \cup [-3,9]$
\testStop
\kluczStart
A
\kluczStop



\zadStart{Zadanie z Wikieł Z 1.62 b) moja wersja nr 199}

Rozwiązać nierówności $(x+14)(9-x)(x+3)\ge0$.
\zadStop
\rozwStart{Patryk Wirkus}{Laura Mieczkowska}
Miejsca zerowe naszego wielomianu to: $-14, 9, -3$.\\
Wielomian jest stopnia nieparzystego, ponadto znak współczynnika przy\linebreak najwyższej potędze x jest ujemny.\\ W związku z tym wykres wielomianu zaczyna się od lewej strony powyżej osi OX. A więc $$x \in (-\infty,-14) \cup (-3,9).$$
\rozwStop
\odpStart
$x \in (-\infty,-14) \cup (-3,9)$
\odpStop
\testStart
A.$x \in (-\infty,-14) \cup (-3,9)$\\
B.$x \in (-\infty,-14) \cup (-3,9]$\\
C.$x \in (-\infty,-14) \cup [-3,9)$\\
D.$x \in (-\infty,-14] \cup (-3,9)$\\
E.$x \in (-\infty,-14] \cup (-3,9]$\\
F.$x \in (-\infty,-14] \cup [-3,9)$\\
G.$x \in (-\infty,-14) \cup [-3,9]$\\
H.$x \in (-\infty,-14] \cup [-3,9]$
\testStop
\kluczStart
A
\kluczStop



\zadStart{Zadanie z Wikieł Z 1.62 b) moja wersja nr 200}

Rozwiązać nierówności $(x+15)(9-x)(x+3)\ge0$.
\zadStop
\rozwStart{Patryk Wirkus}{Laura Mieczkowska}
Miejsca zerowe naszego wielomianu to: $-15, 9, -3$.\\
Wielomian jest stopnia nieparzystego, ponadto znak współczynnika przy\linebreak najwyższej potędze x jest ujemny.\\ W związku z tym wykres wielomianu zaczyna się od lewej strony powyżej osi OX. A więc $$x \in (-\infty,-15) \cup (-3,9).$$
\rozwStop
\odpStart
$x \in (-\infty,-15) \cup (-3,9)$
\odpStop
\testStart
A.$x \in (-\infty,-15) \cup (-3,9)$\\
B.$x \in (-\infty,-15) \cup (-3,9]$\\
C.$x \in (-\infty,-15) \cup [-3,9)$\\
D.$x \in (-\infty,-15] \cup (-3,9)$\\
E.$x \in (-\infty,-15] \cup (-3,9]$\\
F.$x \in (-\infty,-15] \cup [-3,9)$\\
G.$x \in (-\infty,-15) \cup [-3,9]$\\
H.$x \in (-\infty,-15] \cup [-3,9]$
\testStop
\kluczStart
A
\kluczStop



\zadStart{Zadanie z Wikieł Z 1.62 b) moja wersja nr 201}

Rozwiązać nierówności $(x+11)(10-x)(x+3)\ge0$.
\zadStop
\rozwStart{Patryk Wirkus}{Laura Mieczkowska}
Miejsca zerowe naszego wielomianu to: $-11, 10, -3$.\\
Wielomian jest stopnia nieparzystego, ponadto znak współczynnika przy\linebreak najwyższej potędze x jest ujemny.\\ W związku z tym wykres wielomianu zaczyna się od lewej strony powyżej osi OX. A więc $$x \in (-\infty,-11) \cup (-3,10).$$
\rozwStop
\odpStart
$x \in (-\infty,-11) \cup (-3,10)$
\odpStop
\testStart
A.$x \in (-\infty,-11) \cup (-3,10)$\\
B.$x \in (-\infty,-11) \cup (-3,10]$\\
C.$x \in (-\infty,-11) \cup [-3,10)$\\
D.$x \in (-\infty,-11] \cup (-3,10)$\\
E.$x \in (-\infty,-11] \cup (-3,10]$\\
F.$x \in (-\infty,-11] \cup [-3,10)$\\
G.$x \in (-\infty,-11) \cup [-3,10]$\\
H.$x \in (-\infty,-11] \cup [-3,10]$
\testStop
\kluczStart
A
\kluczStop



\zadStart{Zadanie z Wikieł Z 1.62 b) moja wersja nr 202}

Rozwiązać nierówności $(x+12)(10-x)(x+3)\ge0$.
\zadStop
\rozwStart{Patryk Wirkus}{Laura Mieczkowska}
Miejsca zerowe naszego wielomianu to: $-12, 10, -3$.\\
Wielomian jest stopnia nieparzystego, ponadto znak współczynnika przy\linebreak najwyższej potędze x jest ujemny.\\ W związku z tym wykres wielomianu zaczyna się od lewej strony powyżej osi OX. A więc $$x \in (-\infty,-12) \cup (-3,10).$$
\rozwStop
\odpStart
$x \in (-\infty,-12) \cup (-3,10)$
\odpStop
\testStart
A.$x \in (-\infty,-12) \cup (-3,10)$\\
B.$x \in (-\infty,-12) \cup (-3,10]$\\
C.$x \in (-\infty,-12) \cup [-3,10)$\\
D.$x \in (-\infty,-12] \cup (-3,10)$\\
E.$x \in (-\infty,-12] \cup (-3,10]$\\
F.$x \in (-\infty,-12] \cup [-3,10)$\\
G.$x \in (-\infty,-12) \cup [-3,10]$\\
H.$x \in (-\infty,-12] \cup [-3,10]$
\testStop
\kluczStart
A
\kluczStop



\zadStart{Zadanie z Wikieł Z 1.62 b) moja wersja nr 203}

Rozwiązać nierówności $(x+13)(10-x)(x+3)\ge0$.
\zadStop
\rozwStart{Patryk Wirkus}{Laura Mieczkowska}
Miejsca zerowe naszego wielomianu to: $-13, 10, -3$.\\
Wielomian jest stopnia nieparzystego, ponadto znak współczynnika przy\linebreak najwyższej potędze x jest ujemny.\\ W związku z tym wykres wielomianu zaczyna się od lewej strony powyżej osi OX. A więc $$x \in (-\infty,-13) \cup (-3,10).$$
\rozwStop
\odpStart
$x \in (-\infty,-13) \cup (-3,10)$
\odpStop
\testStart
A.$x \in (-\infty,-13) \cup (-3,10)$\\
B.$x \in (-\infty,-13) \cup (-3,10]$\\
C.$x \in (-\infty,-13) \cup [-3,10)$\\
D.$x \in (-\infty,-13] \cup (-3,10)$\\
E.$x \in (-\infty,-13] \cup (-3,10]$\\
F.$x \in (-\infty,-13] \cup [-3,10)$\\
G.$x \in (-\infty,-13) \cup [-3,10]$\\
H.$x \in (-\infty,-13] \cup [-3,10]$
\testStop
\kluczStart
A
\kluczStop



\zadStart{Zadanie z Wikieł Z 1.62 b) moja wersja nr 204}

Rozwiązać nierówności $(x+14)(10-x)(x+3)\ge0$.
\zadStop
\rozwStart{Patryk Wirkus}{Laura Mieczkowska}
Miejsca zerowe naszego wielomianu to: $-14, 10, -3$.\\
Wielomian jest stopnia nieparzystego, ponadto znak współczynnika przy\linebreak najwyższej potędze x jest ujemny.\\ W związku z tym wykres wielomianu zaczyna się od lewej strony powyżej osi OX. A więc $$x \in (-\infty,-14) \cup (-3,10).$$
\rozwStop
\odpStart
$x \in (-\infty,-14) \cup (-3,10)$
\odpStop
\testStart
A.$x \in (-\infty,-14) \cup (-3,10)$\\
B.$x \in (-\infty,-14) \cup (-3,10]$\\
C.$x \in (-\infty,-14) \cup [-3,10)$\\
D.$x \in (-\infty,-14] \cup (-3,10)$\\
E.$x \in (-\infty,-14] \cup (-3,10]$\\
F.$x \in (-\infty,-14] \cup [-3,10)$\\
G.$x \in (-\infty,-14) \cup [-3,10]$\\
H.$x \in (-\infty,-14] \cup [-3,10]$
\testStop
\kluczStart
A
\kluczStop



\zadStart{Zadanie z Wikieł Z 1.62 b) moja wersja nr 205}

Rozwiązać nierówności $(x+15)(10-x)(x+3)\ge0$.
\zadStop
\rozwStart{Patryk Wirkus}{Laura Mieczkowska}
Miejsca zerowe naszego wielomianu to: $-15, 10, -3$.\\
Wielomian jest stopnia nieparzystego, ponadto znak współczynnika przy\linebreak najwyższej potędze x jest ujemny.\\ W związku z tym wykres wielomianu zaczyna się od lewej strony powyżej osi OX. A więc $$x \in (-\infty,-15) \cup (-3,10).$$
\rozwStop
\odpStart
$x \in (-\infty,-15) \cup (-3,10)$
\odpStop
\testStart
A.$x \in (-\infty,-15) \cup (-3,10)$\\
B.$x \in (-\infty,-15) \cup (-3,10]$\\
C.$x \in (-\infty,-15) \cup [-3,10)$\\
D.$x \in (-\infty,-15] \cup (-3,10)$\\
E.$x \in (-\infty,-15] \cup (-3,10]$\\
F.$x \in (-\infty,-15] \cup [-3,10)$\\
G.$x \in (-\infty,-15) \cup [-3,10]$\\
H.$x \in (-\infty,-15] \cup [-3,10]$
\testStop
\kluczStart
A
\kluczStop



\zadStart{Zadanie z Wikieł Z 1.62 b) moja wersja nr 206}

Rozwiązać nierówności $(x+6)(5-x)(x+4)\ge0$.
\zadStop
\rozwStart{Patryk Wirkus}{Laura Mieczkowska}
Miejsca zerowe naszego wielomianu to: $-6, 5, -4$.\\
Wielomian jest stopnia nieparzystego, ponadto znak współczynnika przy\linebreak najwyższej potędze x jest ujemny.\\ W związku z tym wykres wielomianu zaczyna się od lewej strony powyżej osi OX. A więc $$x \in (-\infty,-6) \cup (-4,5).$$
\rozwStop
\odpStart
$x \in (-\infty,-6) \cup (-4,5)$
\odpStop
\testStart
A.$x \in (-\infty,-6) \cup (-4,5)$\\
B.$x \in (-\infty,-6) \cup (-4,5]$\\
C.$x \in (-\infty,-6) \cup [-4,5)$\\
D.$x \in (-\infty,-6] \cup (-4,5)$\\
E.$x \in (-\infty,-6] \cup (-4,5]$\\
F.$x \in (-\infty,-6] \cup [-4,5)$\\
G.$x \in (-\infty,-6) \cup [-4,5]$\\
H.$x \in (-\infty,-6] \cup [-4,5]$
\testStop
\kluczStart
A
\kluczStop



\zadStart{Zadanie z Wikieł Z 1.62 b) moja wersja nr 207}

Rozwiązać nierówności $(x+7)(5-x)(x+4)\ge0$.
\zadStop
\rozwStart{Patryk Wirkus}{Laura Mieczkowska}
Miejsca zerowe naszego wielomianu to: $-7, 5, -4$.\\
Wielomian jest stopnia nieparzystego, ponadto znak współczynnika przy\linebreak najwyższej potędze x jest ujemny.\\ W związku z tym wykres wielomianu zaczyna się od lewej strony powyżej osi OX. A więc $$x \in (-\infty,-7) \cup (-4,5).$$
\rozwStop
\odpStart
$x \in (-\infty,-7) \cup (-4,5)$
\odpStop
\testStart
A.$x \in (-\infty,-7) \cup (-4,5)$\\
B.$x \in (-\infty,-7) \cup (-4,5]$\\
C.$x \in (-\infty,-7) \cup [-4,5)$\\
D.$x \in (-\infty,-7] \cup (-4,5)$\\
E.$x \in (-\infty,-7] \cup (-4,5]$\\
F.$x \in (-\infty,-7] \cup [-4,5)$\\
G.$x \in (-\infty,-7) \cup [-4,5]$\\
H.$x \in (-\infty,-7] \cup [-4,5]$
\testStop
\kluczStart
A
\kluczStop



\zadStart{Zadanie z Wikieł Z 1.62 b) moja wersja nr 208}

Rozwiązać nierówności $(x+8)(5-x)(x+4)\ge0$.
\zadStop
\rozwStart{Patryk Wirkus}{Laura Mieczkowska}
Miejsca zerowe naszego wielomianu to: $-8, 5, -4$.\\
Wielomian jest stopnia nieparzystego, ponadto znak współczynnika przy\linebreak najwyższej potędze x jest ujemny.\\ W związku z tym wykres wielomianu zaczyna się od lewej strony powyżej osi OX. A więc $$x \in (-\infty,-8) \cup (-4,5).$$
\rozwStop
\odpStart
$x \in (-\infty,-8) \cup (-4,5)$
\odpStop
\testStart
A.$x \in (-\infty,-8) \cup (-4,5)$\\
B.$x \in (-\infty,-8) \cup (-4,5]$\\
C.$x \in (-\infty,-8) \cup [-4,5)$\\
D.$x \in (-\infty,-8] \cup (-4,5)$\\
E.$x \in (-\infty,-8] \cup (-4,5]$\\
F.$x \in (-\infty,-8] \cup [-4,5)$\\
G.$x \in (-\infty,-8) \cup [-4,5]$\\
H.$x \in (-\infty,-8] \cup [-4,5]$
\testStop
\kluczStart
A
\kluczStop



\zadStart{Zadanie z Wikieł Z 1.62 b) moja wersja nr 209}

Rozwiązać nierówności $(x+9)(5-x)(x+4)\ge0$.
\zadStop
\rozwStart{Patryk Wirkus}{Laura Mieczkowska}
Miejsca zerowe naszego wielomianu to: $-9, 5, -4$.\\
Wielomian jest stopnia nieparzystego, ponadto znak współczynnika przy\linebreak najwyższej potędze x jest ujemny.\\ W związku z tym wykres wielomianu zaczyna się od lewej strony powyżej osi OX. A więc $$x \in (-\infty,-9) \cup (-4,5).$$
\rozwStop
\odpStart
$x \in (-\infty,-9) \cup (-4,5)$
\odpStop
\testStart
A.$x \in (-\infty,-9) \cup (-4,5)$\\
B.$x \in (-\infty,-9) \cup (-4,5]$\\
C.$x \in (-\infty,-9) \cup [-4,5)$\\
D.$x \in (-\infty,-9] \cup (-4,5)$\\
E.$x \in (-\infty,-9] \cup (-4,5]$\\
F.$x \in (-\infty,-9] \cup [-4,5)$\\
G.$x \in (-\infty,-9) \cup [-4,5]$\\
H.$x \in (-\infty,-9] \cup [-4,5]$
\testStop
\kluczStart
A
\kluczStop



\zadStart{Zadanie z Wikieł Z 1.62 b) moja wersja nr 210}

Rozwiązać nierówności $(x+10)(5-x)(x+4)\ge0$.
\zadStop
\rozwStart{Patryk Wirkus}{Laura Mieczkowska}
Miejsca zerowe naszego wielomianu to: $-10, 5, -4$.\\
Wielomian jest stopnia nieparzystego, ponadto znak współczynnika przy\linebreak najwyższej potędze x jest ujemny.\\ W związku z tym wykres wielomianu zaczyna się od lewej strony powyżej osi OX. A więc $$x \in (-\infty,-10) \cup (-4,5).$$
\rozwStop
\odpStart
$x \in (-\infty,-10) \cup (-4,5)$
\odpStop
\testStart
A.$x \in (-\infty,-10) \cup (-4,5)$\\
B.$x \in (-\infty,-10) \cup (-4,5]$\\
C.$x \in (-\infty,-10) \cup [-4,5)$\\
D.$x \in (-\infty,-10] \cup (-4,5)$\\
E.$x \in (-\infty,-10] \cup (-4,5]$\\
F.$x \in (-\infty,-10] \cup [-4,5)$\\
G.$x \in (-\infty,-10) \cup [-4,5]$\\
H.$x \in (-\infty,-10] \cup [-4,5]$
\testStop
\kluczStart
A
\kluczStop



\zadStart{Zadanie z Wikieł Z 1.62 b) moja wersja nr 211}

Rozwiązać nierówności $(x+11)(5-x)(x+4)\ge0$.
\zadStop
\rozwStart{Patryk Wirkus}{Laura Mieczkowska}
Miejsca zerowe naszego wielomianu to: $-11, 5, -4$.\\
Wielomian jest stopnia nieparzystego, ponadto znak współczynnika przy\linebreak najwyższej potędze x jest ujemny.\\ W związku z tym wykres wielomianu zaczyna się od lewej strony powyżej osi OX. A więc $$x \in (-\infty,-11) \cup (-4,5).$$
\rozwStop
\odpStart
$x \in (-\infty,-11) \cup (-4,5)$
\odpStop
\testStart
A.$x \in (-\infty,-11) \cup (-4,5)$\\
B.$x \in (-\infty,-11) \cup (-4,5]$\\
C.$x \in (-\infty,-11) \cup [-4,5)$\\
D.$x \in (-\infty,-11] \cup (-4,5)$\\
E.$x \in (-\infty,-11] \cup (-4,5]$\\
F.$x \in (-\infty,-11] \cup [-4,5)$\\
G.$x \in (-\infty,-11) \cup [-4,5]$\\
H.$x \in (-\infty,-11] \cup [-4,5]$
\testStop
\kluczStart
A
\kluczStop



\zadStart{Zadanie z Wikieł Z 1.62 b) moja wersja nr 212}

Rozwiązać nierówności $(x+12)(5-x)(x+4)\ge0$.
\zadStop
\rozwStart{Patryk Wirkus}{Laura Mieczkowska}
Miejsca zerowe naszego wielomianu to: $-12, 5, -4$.\\
Wielomian jest stopnia nieparzystego, ponadto znak współczynnika przy\linebreak najwyższej potędze x jest ujemny.\\ W związku z tym wykres wielomianu zaczyna się od lewej strony powyżej osi OX. A więc $$x \in (-\infty,-12) \cup (-4,5).$$
\rozwStop
\odpStart
$x \in (-\infty,-12) \cup (-4,5)$
\odpStop
\testStart
A.$x \in (-\infty,-12) \cup (-4,5)$\\
B.$x \in (-\infty,-12) \cup (-4,5]$\\
C.$x \in (-\infty,-12) \cup [-4,5)$\\
D.$x \in (-\infty,-12] \cup (-4,5)$\\
E.$x \in (-\infty,-12] \cup (-4,5]$\\
F.$x \in (-\infty,-12] \cup [-4,5)$\\
G.$x \in (-\infty,-12) \cup [-4,5]$\\
H.$x \in (-\infty,-12] \cup [-4,5]$
\testStop
\kluczStart
A
\kluczStop



\zadStart{Zadanie z Wikieł Z 1.62 b) moja wersja nr 213}

Rozwiązać nierówności $(x+13)(5-x)(x+4)\ge0$.
\zadStop
\rozwStart{Patryk Wirkus}{Laura Mieczkowska}
Miejsca zerowe naszego wielomianu to: $-13, 5, -4$.\\
Wielomian jest stopnia nieparzystego, ponadto znak współczynnika przy\linebreak najwyższej potędze x jest ujemny.\\ W związku z tym wykres wielomianu zaczyna się od lewej strony powyżej osi OX. A więc $$x \in (-\infty,-13) \cup (-4,5).$$
\rozwStop
\odpStart
$x \in (-\infty,-13) \cup (-4,5)$
\odpStop
\testStart
A.$x \in (-\infty,-13) \cup (-4,5)$\\
B.$x \in (-\infty,-13) \cup (-4,5]$\\
C.$x \in (-\infty,-13) \cup [-4,5)$\\
D.$x \in (-\infty,-13] \cup (-4,5)$\\
E.$x \in (-\infty,-13] \cup (-4,5]$\\
F.$x \in (-\infty,-13] \cup [-4,5)$\\
G.$x \in (-\infty,-13) \cup [-4,5]$\\
H.$x \in (-\infty,-13] \cup [-4,5]$
\testStop
\kluczStart
A
\kluczStop



\zadStart{Zadanie z Wikieł Z 1.62 b) moja wersja nr 214}

Rozwiązać nierówności $(x+14)(5-x)(x+4)\ge0$.
\zadStop
\rozwStart{Patryk Wirkus}{Laura Mieczkowska}
Miejsca zerowe naszego wielomianu to: $-14, 5, -4$.\\
Wielomian jest stopnia nieparzystego, ponadto znak współczynnika przy\linebreak najwyższej potędze x jest ujemny.\\ W związku z tym wykres wielomianu zaczyna się od lewej strony powyżej osi OX. A więc $$x \in (-\infty,-14) \cup (-4,5).$$
\rozwStop
\odpStart
$x \in (-\infty,-14) \cup (-4,5)$
\odpStop
\testStart
A.$x \in (-\infty,-14) \cup (-4,5)$\\
B.$x \in (-\infty,-14) \cup (-4,5]$\\
C.$x \in (-\infty,-14) \cup [-4,5)$\\
D.$x \in (-\infty,-14] \cup (-4,5)$\\
E.$x \in (-\infty,-14] \cup (-4,5]$\\
F.$x \in (-\infty,-14] \cup [-4,5)$\\
G.$x \in (-\infty,-14) \cup [-4,5]$\\
H.$x \in (-\infty,-14] \cup [-4,5]$
\testStop
\kluczStart
A
\kluczStop



\zadStart{Zadanie z Wikieł Z 1.62 b) moja wersja nr 215}

Rozwiązać nierówności $(x+15)(5-x)(x+4)\ge0$.
\zadStop
\rozwStart{Patryk Wirkus}{Laura Mieczkowska}
Miejsca zerowe naszego wielomianu to: $-15, 5, -4$.\\
Wielomian jest stopnia nieparzystego, ponadto znak współczynnika przy\linebreak najwyższej potędze x jest ujemny.\\ W związku z tym wykres wielomianu zaczyna się od lewej strony powyżej osi OX. A więc $$x \in (-\infty,-15) \cup (-4,5).$$
\rozwStop
\odpStart
$x \in (-\infty,-15) \cup (-4,5)$
\odpStop
\testStart
A.$x \in (-\infty,-15) \cup (-4,5)$\\
B.$x \in (-\infty,-15) \cup (-4,5]$\\
C.$x \in (-\infty,-15) \cup [-4,5)$\\
D.$x \in (-\infty,-15] \cup (-4,5)$\\
E.$x \in (-\infty,-15] \cup (-4,5]$\\
F.$x \in (-\infty,-15] \cup [-4,5)$\\
G.$x \in (-\infty,-15) \cup [-4,5]$\\
H.$x \in (-\infty,-15] \cup [-4,5]$
\testStop
\kluczStart
A
\kluczStop



\zadStart{Zadanie z Wikieł Z 1.62 b) moja wersja nr 216}

Rozwiązać nierówności $(x+7)(6-x)(x+4)\ge0$.
\zadStop
\rozwStart{Patryk Wirkus}{Laura Mieczkowska}
Miejsca zerowe naszego wielomianu to: $-7, 6, -4$.\\
Wielomian jest stopnia nieparzystego, ponadto znak współczynnika przy\linebreak najwyższej potędze x jest ujemny.\\ W związku z tym wykres wielomianu zaczyna się od lewej strony powyżej osi OX. A więc $$x \in (-\infty,-7) \cup (-4,6).$$
\rozwStop
\odpStart
$x \in (-\infty,-7) \cup (-4,6)$
\odpStop
\testStart
A.$x \in (-\infty,-7) \cup (-4,6)$\\
B.$x \in (-\infty,-7) \cup (-4,6]$\\
C.$x \in (-\infty,-7) \cup [-4,6)$\\
D.$x \in (-\infty,-7] \cup (-4,6)$\\
E.$x \in (-\infty,-7] \cup (-4,6]$\\
F.$x \in (-\infty,-7] \cup [-4,6)$\\
G.$x \in (-\infty,-7) \cup [-4,6]$\\
H.$x \in (-\infty,-7] \cup [-4,6]$
\testStop
\kluczStart
A
\kluczStop



\zadStart{Zadanie z Wikieł Z 1.62 b) moja wersja nr 217}

Rozwiązać nierówności $(x+8)(6-x)(x+4)\ge0$.
\zadStop
\rozwStart{Patryk Wirkus}{Laura Mieczkowska}
Miejsca zerowe naszego wielomianu to: $-8, 6, -4$.\\
Wielomian jest stopnia nieparzystego, ponadto znak współczynnika przy\linebreak najwyższej potędze x jest ujemny.\\ W związku z tym wykres wielomianu zaczyna się od lewej strony powyżej osi OX. A więc $$x \in (-\infty,-8) \cup (-4,6).$$
\rozwStop
\odpStart
$x \in (-\infty,-8) \cup (-4,6)$
\odpStop
\testStart
A.$x \in (-\infty,-8) \cup (-4,6)$\\
B.$x \in (-\infty,-8) \cup (-4,6]$\\
C.$x \in (-\infty,-8) \cup [-4,6)$\\
D.$x \in (-\infty,-8] \cup (-4,6)$\\
E.$x \in (-\infty,-8] \cup (-4,6]$\\
F.$x \in (-\infty,-8] \cup [-4,6)$\\
G.$x \in (-\infty,-8) \cup [-4,6]$\\
H.$x \in (-\infty,-8] \cup [-4,6]$
\testStop
\kluczStart
A
\kluczStop



\zadStart{Zadanie z Wikieł Z 1.62 b) moja wersja nr 218}

Rozwiązać nierówności $(x+9)(6-x)(x+4)\ge0$.
\zadStop
\rozwStart{Patryk Wirkus}{Laura Mieczkowska}
Miejsca zerowe naszego wielomianu to: $-9, 6, -4$.\\
Wielomian jest stopnia nieparzystego, ponadto znak współczynnika przy\linebreak najwyższej potędze x jest ujemny.\\ W związku z tym wykres wielomianu zaczyna się od lewej strony powyżej osi OX. A więc $$x \in (-\infty,-9) \cup (-4,6).$$
\rozwStop
\odpStart
$x \in (-\infty,-9) \cup (-4,6)$
\odpStop
\testStart
A.$x \in (-\infty,-9) \cup (-4,6)$\\
B.$x \in (-\infty,-9) \cup (-4,6]$\\
C.$x \in (-\infty,-9) \cup [-4,6)$\\
D.$x \in (-\infty,-9] \cup (-4,6)$\\
E.$x \in (-\infty,-9] \cup (-4,6]$\\
F.$x \in (-\infty,-9] \cup [-4,6)$\\
G.$x \in (-\infty,-9) \cup [-4,6]$\\
H.$x \in (-\infty,-9] \cup [-4,6]$
\testStop
\kluczStart
A
\kluczStop



\zadStart{Zadanie z Wikieł Z 1.62 b) moja wersja nr 219}

Rozwiązać nierówności $(x+10)(6-x)(x+4)\ge0$.
\zadStop
\rozwStart{Patryk Wirkus}{Laura Mieczkowska}
Miejsca zerowe naszego wielomianu to: $-10, 6, -4$.\\
Wielomian jest stopnia nieparzystego, ponadto znak współczynnika przy\linebreak najwyższej potędze x jest ujemny.\\ W związku z tym wykres wielomianu zaczyna się od lewej strony powyżej osi OX. A więc $$x \in (-\infty,-10) \cup (-4,6).$$
\rozwStop
\odpStart
$x \in (-\infty,-10) \cup (-4,6)$
\odpStop
\testStart
A.$x \in (-\infty,-10) \cup (-4,6)$\\
B.$x \in (-\infty,-10) \cup (-4,6]$\\
C.$x \in (-\infty,-10) \cup [-4,6)$\\
D.$x \in (-\infty,-10] \cup (-4,6)$\\
E.$x \in (-\infty,-10] \cup (-4,6]$\\
F.$x \in (-\infty,-10] \cup [-4,6)$\\
G.$x \in (-\infty,-10) \cup [-4,6]$\\
H.$x \in (-\infty,-10] \cup [-4,6]$
\testStop
\kluczStart
A
\kluczStop



\zadStart{Zadanie z Wikieł Z 1.62 b) moja wersja nr 220}

Rozwiązać nierówności $(x+11)(6-x)(x+4)\ge0$.
\zadStop
\rozwStart{Patryk Wirkus}{Laura Mieczkowska}
Miejsca zerowe naszego wielomianu to: $-11, 6, -4$.\\
Wielomian jest stopnia nieparzystego, ponadto znak współczynnika przy\linebreak najwyższej potędze x jest ujemny.\\ W związku z tym wykres wielomianu zaczyna się od lewej strony powyżej osi OX. A więc $$x \in (-\infty,-11) \cup (-4,6).$$
\rozwStop
\odpStart
$x \in (-\infty,-11) \cup (-4,6)$
\odpStop
\testStart
A.$x \in (-\infty,-11) \cup (-4,6)$\\
B.$x \in (-\infty,-11) \cup (-4,6]$\\
C.$x \in (-\infty,-11) \cup [-4,6)$\\
D.$x \in (-\infty,-11] \cup (-4,6)$\\
E.$x \in (-\infty,-11] \cup (-4,6]$\\
F.$x \in (-\infty,-11] \cup [-4,6)$\\
G.$x \in (-\infty,-11) \cup [-4,6]$\\
H.$x \in (-\infty,-11] \cup [-4,6]$
\testStop
\kluczStart
A
\kluczStop



\zadStart{Zadanie z Wikieł Z 1.62 b) moja wersja nr 221}

Rozwiązać nierówności $(x+12)(6-x)(x+4)\ge0$.
\zadStop
\rozwStart{Patryk Wirkus}{Laura Mieczkowska}
Miejsca zerowe naszego wielomianu to: $-12, 6, -4$.\\
Wielomian jest stopnia nieparzystego, ponadto znak współczynnika przy\linebreak najwyższej potędze x jest ujemny.\\ W związku z tym wykres wielomianu zaczyna się od lewej strony powyżej osi OX. A więc $$x \in (-\infty,-12) \cup (-4,6).$$
\rozwStop
\odpStart
$x \in (-\infty,-12) \cup (-4,6)$
\odpStop
\testStart
A.$x \in (-\infty,-12) \cup (-4,6)$\\
B.$x \in (-\infty,-12) \cup (-4,6]$\\
C.$x \in (-\infty,-12) \cup [-4,6)$\\
D.$x \in (-\infty,-12] \cup (-4,6)$\\
E.$x \in (-\infty,-12] \cup (-4,6]$\\
F.$x \in (-\infty,-12] \cup [-4,6)$\\
G.$x \in (-\infty,-12) \cup [-4,6]$\\
H.$x \in (-\infty,-12] \cup [-4,6]$
\testStop
\kluczStart
A
\kluczStop



\zadStart{Zadanie z Wikieł Z 1.62 b) moja wersja nr 222}

Rozwiązać nierówności $(x+13)(6-x)(x+4)\ge0$.
\zadStop
\rozwStart{Patryk Wirkus}{Laura Mieczkowska}
Miejsca zerowe naszego wielomianu to: $-13, 6, -4$.\\
Wielomian jest stopnia nieparzystego, ponadto znak współczynnika przy\linebreak najwyższej potędze x jest ujemny.\\ W związku z tym wykres wielomianu zaczyna się od lewej strony powyżej osi OX. A więc $$x \in (-\infty,-13) \cup (-4,6).$$
\rozwStop
\odpStart
$x \in (-\infty,-13) \cup (-4,6)$
\odpStop
\testStart
A.$x \in (-\infty,-13) \cup (-4,6)$\\
B.$x \in (-\infty,-13) \cup (-4,6]$\\
C.$x \in (-\infty,-13) \cup [-4,6)$\\
D.$x \in (-\infty,-13] \cup (-4,6)$\\
E.$x \in (-\infty,-13] \cup (-4,6]$\\
F.$x \in (-\infty,-13] \cup [-4,6)$\\
G.$x \in (-\infty,-13) \cup [-4,6]$\\
H.$x \in (-\infty,-13] \cup [-4,6]$
\testStop
\kluczStart
A
\kluczStop



\zadStart{Zadanie z Wikieł Z 1.62 b) moja wersja nr 223}

Rozwiązać nierówności $(x+14)(6-x)(x+4)\ge0$.
\zadStop
\rozwStart{Patryk Wirkus}{Laura Mieczkowska}
Miejsca zerowe naszego wielomianu to: $-14, 6, -4$.\\
Wielomian jest stopnia nieparzystego, ponadto znak współczynnika przy\linebreak najwyższej potędze x jest ujemny.\\ W związku z tym wykres wielomianu zaczyna się od lewej strony powyżej osi OX. A więc $$x \in (-\infty,-14) \cup (-4,6).$$
\rozwStop
\odpStart
$x \in (-\infty,-14) \cup (-4,6)$
\odpStop
\testStart
A.$x \in (-\infty,-14) \cup (-4,6)$\\
B.$x \in (-\infty,-14) \cup (-4,6]$\\
C.$x \in (-\infty,-14) \cup [-4,6)$\\
D.$x \in (-\infty,-14] \cup (-4,6)$\\
E.$x \in (-\infty,-14] \cup (-4,6]$\\
F.$x \in (-\infty,-14] \cup [-4,6)$\\
G.$x \in (-\infty,-14) \cup [-4,6]$\\
H.$x \in (-\infty,-14] \cup [-4,6]$
\testStop
\kluczStart
A
\kluczStop



\zadStart{Zadanie z Wikieł Z 1.62 b) moja wersja nr 224}

Rozwiązać nierówności $(x+15)(6-x)(x+4)\ge0$.
\zadStop
\rozwStart{Patryk Wirkus}{Laura Mieczkowska}
Miejsca zerowe naszego wielomianu to: $-15, 6, -4$.\\
Wielomian jest stopnia nieparzystego, ponadto znak współczynnika przy\linebreak najwyższej potędze x jest ujemny.\\ W związku z tym wykres wielomianu zaczyna się od lewej strony powyżej osi OX. A więc $$x \in (-\infty,-15) \cup (-4,6).$$
\rozwStop
\odpStart
$x \in (-\infty,-15) \cup (-4,6)$
\odpStop
\testStart
A.$x \in (-\infty,-15) \cup (-4,6)$\\
B.$x \in (-\infty,-15) \cup (-4,6]$\\
C.$x \in (-\infty,-15) \cup [-4,6)$\\
D.$x \in (-\infty,-15] \cup (-4,6)$\\
E.$x \in (-\infty,-15] \cup (-4,6]$\\
F.$x \in (-\infty,-15] \cup [-4,6)$\\
G.$x \in (-\infty,-15) \cup [-4,6]$\\
H.$x \in (-\infty,-15] \cup [-4,6]$
\testStop
\kluczStart
A
\kluczStop



\zadStart{Zadanie z Wikieł Z 1.62 b) moja wersja nr 225}

Rozwiązać nierówności $(x+8)(7-x)(x+4)\ge0$.
\zadStop
\rozwStart{Patryk Wirkus}{Laura Mieczkowska}
Miejsca zerowe naszego wielomianu to: $-8, 7, -4$.\\
Wielomian jest stopnia nieparzystego, ponadto znak współczynnika przy\linebreak najwyższej potędze x jest ujemny.\\ W związku z tym wykres wielomianu zaczyna się od lewej strony powyżej osi OX. A więc $$x \in (-\infty,-8) \cup (-4,7).$$
\rozwStop
\odpStart
$x \in (-\infty,-8) \cup (-4,7)$
\odpStop
\testStart
A.$x \in (-\infty,-8) \cup (-4,7)$\\
B.$x \in (-\infty,-8) \cup (-4,7]$\\
C.$x \in (-\infty,-8) \cup [-4,7)$\\
D.$x \in (-\infty,-8] \cup (-4,7)$\\
E.$x \in (-\infty,-8] \cup (-4,7]$\\
F.$x \in (-\infty,-8] \cup [-4,7)$\\
G.$x \in (-\infty,-8) \cup [-4,7]$\\
H.$x \in (-\infty,-8] \cup [-4,7]$
\testStop
\kluczStart
A
\kluczStop



\zadStart{Zadanie z Wikieł Z 1.62 b) moja wersja nr 226}

Rozwiązać nierówności $(x+9)(7-x)(x+4)\ge0$.
\zadStop
\rozwStart{Patryk Wirkus}{Laura Mieczkowska}
Miejsca zerowe naszego wielomianu to: $-9, 7, -4$.\\
Wielomian jest stopnia nieparzystego, ponadto znak współczynnika przy\linebreak najwyższej potędze x jest ujemny.\\ W związku z tym wykres wielomianu zaczyna się od lewej strony powyżej osi OX. A więc $$x \in (-\infty,-9) \cup (-4,7).$$
\rozwStop
\odpStart
$x \in (-\infty,-9) \cup (-4,7)$
\odpStop
\testStart
A.$x \in (-\infty,-9) \cup (-4,7)$\\
B.$x \in (-\infty,-9) \cup (-4,7]$\\
C.$x \in (-\infty,-9) \cup [-4,7)$\\
D.$x \in (-\infty,-9] \cup (-4,7)$\\
E.$x \in (-\infty,-9] \cup (-4,7]$\\
F.$x \in (-\infty,-9] \cup [-4,7)$\\
G.$x \in (-\infty,-9) \cup [-4,7]$\\
H.$x \in (-\infty,-9] \cup [-4,7]$
\testStop
\kluczStart
A
\kluczStop



\zadStart{Zadanie z Wikieł Z 1.62 b) moja wersja nr 227}

Rozwiązać nierówności $(x+10)(7-x)(x+4)\ge0$.
\zadStop
\rozwStart{Patryk Wirkus}{Laura Mieczkowska}
Miejsca zerowe naszego wielomianu to: $-10, 7, -4$.\\
Wielomian jest stopnia nieparzystego, ponadto znak współczynnika przy\linebreak najwyższej potędze x jest ujemny.\\ W związku z tym wykres wielomianu zaczyna się od lewej strony powyżej osi OX. A więc $$x \in (-\infty,-10) \cup (-4,7).$$
\rozwStop
\odpStart
$x \in (-\infty,-10) \cup (-4,7)$
\odpStop
\testStart
A.$x \in (-\infty,-10) \cup (-4,7)$\\
B.$x \in (-\infty,-10) \cup (-4,7]$\\
C.$x \in (-\infty,-10) \cup [-4,7)$\\
D.$x \in (-\infty,-10] \cup (-4,7)$\\
E.$x \in (-\infty,-10] \cup (-4,7]$\\
F.$x \in (-\infty,-10] \cup [-4,7)$\\
G.$x \in (-\infty,-10) \cup [-4,7]$\\
H.$x \in (-\infty,-10] \cup [-4,7]$
\testStop
\kluczStart
A
\kluczStop



\zadStart{Zadanie z Wikieł Z 1.62 b) moja wersja nr 228}

Rozwiązać nierówności $(x+11)(7-x)(x+4)\ge0$.
\zadStop
\rozwStart{Patryk Wirkus}{Laura Mieczkowska}
Miejsca zerowe naszego wielomianu to: $-11, 7, -4$.\\
Wielomian jest stopnia nieparzystego, ponadto znak współczynnika przy\linebreak najwyższej potędze x jest ujemny.\\ W związku z tym wykres wielomianu zaczyna się od lewej strony powyżej osi OX. A więc $$x \in (-\infty,-11) \cup (-4,7).$$
\rozwStop
\odpStart
$x \in (-\infty,-11) \cup (-4,7)$
\odpStop
\testStart
A.$x \in (-\infty,-11) \cup (-4,7)$\\
B.$x \in (-\infty,-11) \cup (-4,7]$\\
C.$x \in (-\infty,-11) \cup [-4,7)$\\
D.$x \in (-\infty,-11] \cup (-4,7)$\\
E.$x \in (-\infty,-11] \cup (-4,7]$\\
F.$x \in (-\infty,-11] \cup [-4,7)$\\
G.$x \in (-\infty,-11) \cup [-4,7]$\\
H.$x \in (-\infty,-11] \cup [-4,7]$
\testStop
\kluczStart
A
\kluczStop



\zadStart{Zadanie z Wikieł Z 1.62 b) moja wersja nr 229}

Rozwiązać nierówności $(x+12)(7-x)(x+4)\ge0$.
\zadStop
\rozwStart{Patryk Wirkus}{Laura Mieczkowska}
Miejsca zerowe naszego wielomianu to: $-12, 7, -4$.\\
Wielomian jest stopnia nieparzystego, ponadto znak współczynnika przy\linebreak najwyższej potędze x jest ujemny.\\ W związku z tym wykres wielomianu zaczyna się od lewej strony powyżej osi OX. A więc $$x \in (-\infty,-12) \cup (-4,7).$$
\rozwStop
\odpStart
$x \in (-\infty,-12) \cup (-4,7)$
\odpStop
\testStart
A.$x \in (-\infty,-12) \cup (-4,7)$\\
B.$x \in (-\infty,-12) \cup (-4,7]$\\
C.$x \in (-\infty,-12) \cup [-4,7)$\\
D.$x \in (-\infty,-12] \cup (-4,7)$\\
E.$x \in (-\infty,-12] \cup (-4,7]$\\
F.$x \in (-\infty,-12] \cup [-4,7)$\\
G.$x \in (-\infty,-12) \cup [-4,7]$\\
H.$x \in (-\infty,-12] \cup [-4,7]$
\testStop
\kluczStart
A
\kluczStop



\zadStart{Zadanie z Wikieł Z 1.62 b) moja wersja nr 230}

Rozwiązać nierówności $(x+13)(7-x)(x+4)\ge0$.
\zadStop
\rozwStart{Patryk Wirkus}{Laura Mieczkowska}
Miejsca zerowe naszego wielomianu to: $-13, 7, -4$.\\
Wielomian jest stopnia nieparzystego, ponadto znak współczynnika przy\linebreak najwyższej potędze x jest ujemny.\\ W związku z tym wykres wielomianu zaczyna się od lewej strony powyżej osi OX. A więc $$x \in (-\infty,-13) \cup (-4,7).$$
\rozwStop
\odpStart
$x \in (-\infty,-13) \cup (-4,7)$
\odpStop
\testStart
A.$x \in (-\infty,-13) \cup (-4,7)$\\
B.$x \in (-\infty,-13) \cup (-4,7]$\\
C.$x \in (-\infty,-13) \cup [-4,7)$\\
D.$x \in (-\infty,-13] \cup (-4,7)$\\
E.$x \in (-\infty,-13] \cup (-4,7]$\\
F.$x \in (-\infty,-13] \cup [-4,7)$\\
G.$x \in (-\infty,-13) \cup [-4,7]$\\
H.$x \in (-\infty,-13] \cup [-4,7]$
\testStop
\kluczStart
A
\kluczStop



\zadStart{Zadanie z Wikieł Z 1.62 b) moja wersja nr 231}

Rozwiązać nierówności $(x+14)(7-x)(x+4)\ge0$.
\zadStop
\rozwStart{Patryk Wirkus}{Laura Mieczkowska}
Miejsca zerowe naszego wielomianu to: $-14, 7, -4$.\\
Wielomian jest stopnia nieparzystego, ponadto znak współczynnika przy\linebreak najwyższej potędze x jest ujemny.\\ W związku z tym wykres wielomianu zaczyna się od lewej strony powyżej osi OX. A więc $$x \in (-\infty,-14) \cup (-4,7).$$
\rozwStop
\odpStart
$x \in (-\infty,-14) \cup (-4,7)$
\odpStop
\testStart
A.$x \in (-\infty,-14) \cup (-4,7)$\\
B.$x \in (-\infty,-14) \cup (-4,7]$\\
C.$x \in (-\infty,-14) \cup [-4,7)$\\
D.$x \in (-\infty,-14] \cup (-4,7)$\\
E.$x \in (-\infty,-14] \cup (-4,7]$\\
F.$x \in (-\infty,-14] \cup [-4,7)$\\
G.$x \in (-\infty,-14) \cup [-4,7]$\\
H.$x \in (-\infty,-14] \cup [-4,7]$
\testStop
\kluczStart
A
\kluczStop



\zadStart{Zadanie z Wikieł Z 1.62 b) moja wersja nr 232}

Rozwiązać nierówności $(x+15)(7-x)(x+4)\ge0$.
\zadStop
\rozwStart{Patryk Wirkus}{Laura Mieczkowska}
Miejsca zerowe naszego wielomianu to: $-15, 7, -4$.\\
Wielomian jest stopnia nieparzystego, ponadto znak współczynnika przy\linebreak najwyższej potędze x jest ujemny.\\ W związku z tym wykres wielomianu zaczyna się od lewej strony powyżej osi OX. A więc $$x \in (-\infty,-15) \cup (-4,7).$$
\rozwStop
\odpStart
$x \in (-\infty,-15) \cup (-4,7)$
\odpStop
\testStart
A.$x \in (-\infty,-15) \cup (-4,7)$\\
B.$x \in (-\infty,-15) \cup (-4,7]$\\
C.$x \in (-\infty,-15) \cup [-4,7)$\\
D.$x \in (-\infty,-15] \cup (-4,7)$\\
E.$x \in (-\infty,-15] \cup (-4,7]$\\
F.$x \in (-\infty,-15] \cup [-4,7)$\\
G.$x \in (-\infty,-15) \cup [-4,7]$\\
H.$x \in (-\infty,-15] \cup [-4,7]$
\testStop
\kluczStart
A
\kluczStop



\zadStart{Zadanie z Wikieł Z 1.62 b) moja wersja nr 233}

Rozwiązać nierówności $(x+9)(8-x)(x+4)\ge0$.
\zadStop
\rozwStart{Patryk Wirkus}{Laura Mieczkowska}
Miejsca zerowe naszego wielomianu to: $-9, 8, -4$.\\
Wielomian jest stopnia nieparzystego, ponadto znak współczynnika przy\linebreak najwyższej potędze x jest ujemny.\\ W związku z tym wykres wielomianu zaczyna się od lewej strony powyżej osi OX. A więc $$x \in (-\infty,-9) \cup (-4,8).$$
\rozwStop
\odpStart
$x \in (-\infty,-9) \cup (-4,8)$
\odpStop
\testStart
A.$x \in (-\infty,-9) \cup (-4,8)$\\
B.$x \in (-\infty,-9) \cup (-4,8]$\\
C.$x \in (-\infty,-9) \cup [-4,8)$\\
D.$x \in (-\infty,-9] \cup (-4,8)$\\
E.$x \in (-\infty,-9] \cup (-4,8]$\\
F.$x \in (-\infty,-9] \cup [-4,8)$\\
G.$x \in (-\infty,-9) \cup [-4,8]$\\
H.$x \in (-\infty,-9] \cup [-4,8]$
\testStop
\kluczStart
A
\kluczStop



\zadStart{Zadanie z Wikieł Z 1.62 b) moja wersja nr 234}

Rozwiązać nierówności $(x+10)(8-x)(x+4)\ge0$.
\zadStop
\rozwStart{Patryk Wirkus}{Laura Mieczkowska}
Miejsca zerowe naszego wielomianu to: $-10, 8, -4$.\\
Wielomian jest stopnia nieparzystego, ponadto znak współczynnika przy\linebreak najwyższej potędze x jest ujemny.\\ W związku z tym wykres wielomianu zaczyna się od lewej strony powyżej osi OX. A więc $$x \in (-\infty,-10) \cup (-4,8).$$
\rozwStop
\odpStart
$x \in (-\infty,-10) \cup (-4,8)$
\odpStop
\testStart
A.$x \in (-\infty,-10) \cup (-4,8)$\\
B.$x \in (-\infty,-10) \cup (-4,8]$\\
C.$x \in (-\infty,-10) \cup [-4,8)$\\
D.$x \in (-\infty,-10] \cup (-4,8)$\\
E.$x \in (-\infty,-10] \cup (-4,8]$\\
F.$x \in (-\infty,-10] \cup [-4,8)$\\
G.$x \in (-\infty,-10) \cup [-4,8]$\\
H.$x \in (-\infty,-10] \cup [-4,8]$
\testStop
\kluczStart
A
\kluczStop



\zadStart{Zadanie z Wikieł Z 1.62 b) moja wersja nr 235}

Rozwiązać nierówności $(x+11)(8-x)(x+4)\ge0$.
\zadStop
\rozwStart{Patryk Wirkus}{Laura Mieczkowska}
Miejsca zerowe naszego wielomianu to: $-11, 8, -4$.\\
Wielomian jest stopnia nieparzystego, ponadto znak współczynnika przy\linebreak najwyższej potędze x jest ujemny.\\ W związku z tym wykres wielomianu zaczyna się od lewej strony powyżej osi OX. A więc $$x \in (-\infty,-11) \cup (-4,8).$$
\rozwStop
\odpStart
$x \in (-\infty,-11) \cup (-4,8)$
\odpStop
\testStart
A.$x \in (-\infty,-11) \cup (-4,8)$\\
B.$x \in (-\infty,-11) \cup (-4,8]$\\
C.$x \in (-\infty,-11) \cup [-4,8)$\\
D.$x \in (-\infty,-11] \cup (-4,8)$\\
E.$x \in (-\infty,-11] \cup (-4,8]$\\
F.$x \in (-\infty,-11] \cup [-4,8)$\\
G.$x \in (-\infty,-11) \cup [-4,8]$\\
H.$x \in (-\infty,-11] \cup [-4,8]$
\testStop
\kluczStart
A
\kluczStop



\zadStart{Zadanie z Wikieł Z 1.62 b) moja wersja nr 236}

Rozwiązać nierówności $(x+12)(8-x)(x+4)\ge0$.
\zadStop
\rozwStart{Patryk Wirkus}{Laura Mieczkowska}
Miejsca zerowe naszego wielomianu to: $-12, 8, -4$.\\
Wielomian jest stopnia nieparzystego, ponadto znak współczynnika przy\linebreak najwyższej potędze x jest ujemny.\\ W związku z tym wykres wielomianu zaczyna się od lewej strony powyżej osi OX. A więc $$x \in (-\infty,-12) \cup (-4,8).$$
\rozwStop
\odpStart
$x \in (-\infty,-12) \cup (-4,8)$
\odpStop
\testStart
A.$x \in (-\infty,-12) \cup (-4,8)$\\
B.$x \in (-\infty,-12) \cup (-4,8]$\\
C.$x \in (-\infty,-12) \cup [-4,8)$\\
D.$x \in (-\infty,-12] \cup (-4,8)$\\
E.$x \in (-\infty,-12] \cup (-4,8]$\\
F.$x \in (-\infty,-12] \cup [-4,8)$\\
G.$x \in (-\infty,-12) \cup [-4,8]$\\
H.$x \in (-\infty,-12] \cup [-4,8]$
\testStop
\kluczStart
A
\kluczStop



\zadStart{Zadanie z Wikieł Z 1.62 b) moja wersja nr 237}

Rozwiązać nierówności $(x+13)(8-x)(x+4)\ge0$.
\zadStop
\rozwStart{Patryk Wirkus}{Laura Mieczkowska}
Miejsca zerowe naszego wielomianu to: $-13, 8, -4$.\\
Wielomian jest stopnia nieparzystego, ponadto znak współczynnika przy\linebreak najwyższej potędze x jest ujemny.\\ W związku z tym wykres wielomianu zaczyna się od lewej strony powyżej osi OX. A więc $$x \in (-\infty,-13) \cup (-4,8).$$
\rozwStop
\odpStart
$x \in (-\infty,-13) \cup (-4,8)$
\odpStop
\testStart
A.$x \in (-\infty,-13) \cup (-4,8)$\\
B.$x \in (-\infty,-13) \cup (-4,8]$\\
C.$x \in (-\infty,-13) \cup [-4,8)$\\
D.$x \in (-\infty,-13] \cup (-4,8)$\\
E.$x \in (-\infty,-13] \cup (-4,8]$\\
F.$x \in (-\infty,-13] \cup [-4,8)$\\
G.$x \in (-\infty,-13) \cup [-4,8]$\\
H.$x \in (-\infty,-13] \cup [-4,8]$
\testStop
\kluczStart
A
\kluczStop



\zadStart{Zadanie z Wikieł Z 1.62 b) moja wersja nr 238}

Rozwiązać nierówności $(x+14)(8-x)(x+4)\ge0$.
\zadStop
\rozwStart{Patryk Wirkus}{Laura Mieczkowska}
Miejsca zerowe naszego wielomianu to: $-14, 8, -4$.\\
Wielomian jest stopnia nieparzystego, ponadto znak współczynnika przy\linebreak najwyższej potędze x jest ujemny.\\ W związku z tym wykres wielomianu zaczyna się od lewej strony powyżej osi OX. A więc $$x \in (-\infty,-14) \cup (-4,8).$$
\rozwStop
\odpStart
$x \in (-\infty,-14) \cup (-4,8)$
\odpStop
\testStart
A.$x \in (-\infty,-14) \cup (-4,8)$\\
B.$x \in (-\infty,-14) \cup (-4,8]$\\
C.$x \in (-\infty,-14) \cup [-4,8)$\\
D.$x \in (-\infty,-14] \cup (-4,8)$\\
E.$x \in (-\infty,-14] \cup (-4,8]$\\
F.$x \in (-\infty,-14] \cup [-4,8)$\\
G.$x \in (-\infty,-14) \cup [-4,8]$\\
H.$x \in (-\infty,-14] \cup [-4,8]$
\testStop
\kluczStart
A
\kluczStop



\zadStart{Zadanie z Wikieł Z 1.62 b) moja wersja nr 239}

Rozwiązać nierówności $(x+15)(8-x)(x+4)\ge0$.
\zadStop
\rozwStart{Patryk Wirkus}{Laura Mieczkowska}
Miejsca zerowe naszego wielomianu to: $-15, 8, -4$.\\
Wielomian jest stopnia nieparzystego, ponadto znak współczynnika przy\linebreak najwyższej potędze x jest ujemny.\\ W związku z tym wykres wielomianu zaczyna się od lewej strony powyżej osi OX. A więc $$x \in (-\infty,-15) \cup (-4,8).$$
\rozwStop
\odpStart
$x \in (-\infty,-15) \cup (-4,8)$
\odpStop
\testStart
A.$x \in (-\infty,-15) \cup (-4,8)$\\
B.$x \in (-\infty,-15) \cup (-4,8]$\\
C.$x \in (-\infty,-15) \cup [-4,8)$\\
D.$x \in (-\infty,-15] \cup (-4,8)$\\
E.$x \in (-\infty,-15] \cup (-4,8]$\\
F.$x \in (-\infty,-15] \cup [-4,8)$\\
G.$x \in (-\infty,-15) \cup [-4,8]$\\
H.$x \in (-\infty,-15] \cup [-4,8]$
\testStop
\kluczStart
A
\kluczStop



\zadStart{Zadanie z Wikieł Z 1.62 b) moja wersja nr 240}

Rozwiązać nierówności $(x+10)(9-x)(x+4)\ge0$.
\zadStop
\rozwStart{Patryk Wirkus}{Laura Mieczkowska}
Miejsca zerowe naszego wielomianu to: $-10, 9, -4$.\\
Wielomian jest stopnia nieparzystego, ponadto znak współczynnika przy\linebreak najwyższej potędze x jest ujemny.\\ W związku z tym wykres wielomianu zaczyna się od lewej strony powyżej osi OX. A więc $$x \in (-\infty,-10) \cup (-4,9).$$
\rozwStop
\odpStart
$x \in (-\infty,-10) \cup (-4,9)$
\odpStop
\testStart
A.$x \in (-\infty,-10) \cup (-4,9)$\\
B.$x \in (-\infty,-10) \cup (-4,9]$\\
C.$x \in (-\infty,-10) \cup [-4,9)$\\
D.$x \in (-\infty,-10] \cup (-4,9)$\\
E.$x \in (-\infty,-10] \cup (-4,9]$\\
F.$x \in (-\infty,-10] \cup [-4,9)$\\
G.$x \in (-\infty,-10) \cup [-4,9]$\\
H.$x \in (-\infty,-10] \cup [-4,9]$
\testStop
\kluczStart
A
\kluczStop



\zadStart{Zadanie z Wikieł Z 1.62 b) moja wersja nr 241}

Rozwiązać nierówności $(x+11)(9-x)(x+4)\ge0$.
\zadStop
\rozwStart{Patryk Wirkus}{Laura Mieczkowska}
Miejsca zerowe naszego wielomianu to: $-11, 9, -4$.\\
Wielomian jest stopnia nieparzystego, ponadto znak współczynnika przy\linebreak najwyższej potędze x jest ujemny.\\ W związku z tym wykres wielomianu zaczyna się od lewej strony powyżej osi OX. A więc $$x \in (-\infty,-11) \cup (-4,9).$$
\rozwStop
\odpStart
$x \in (-\infty,-11) \cup (-4,9)$
\odpStop
\testStart
A.$x \in (-\infty,-11) \cup (-4,9)$\\
B.$x \in (-\infty,-11) \cup (-4,9]$\\
C.$x \in (-\infty,-11) \cup [-4,9)$\\
D.$x \in (-\infty,-11] \cup (-4,9)$\\
E.$x \in (-\infty,-11] \cup (-4,9]$\\
F.$x \in (-\infty,-11] \cup [-4,9)$\\
G.$x \in (-\infty,-11) \cup [-4,9]$\\
H.$x \in (-\infty,-11] \cup [-4,9]$
\testStop
\kluczStart
A
\kluczStop



\zadStart{Zadanie z Wikieł Z 1.62 b) moja wersja nr 242}

Rozwiązać nierówności $(x+12)(9-x)(x+4)\ge0$.
\zadStop
\rozwStart{Patryk Wirkus}{Laura Mieczkowska}
Miejsca zerowe naszego wielomianu to: $-12, 9, -4$.\\
Wielomian jest stopnia nieparzystego, ponadto znak współczynnika przy\linebreak najwyższej potędze x jest ujemny.\\ W związku z tym wykres wielomianu zaczyna się od lewej strony powyżej osi OX. A więc $$x \in (-\infty,-12) \cup (-4,9).$$
\rozwStop
\odpStart
$x \in (-\infty,-12) \cup (-4,9)$
\odpStop
\testStart
A.$x \in (-\infty,-12) \cup (-4,9)$\\
B.$x \in (-\infty,-12) \cup (-4,9]$\\
C.$x \in (-\infty,-12) \cup [-4,9)$\\
D.$x \in (-\infty,-12] \cup (-4,9)$\\
E.$x \in (-\infty,-12] \cup (-4,9]$\\
F.$x \in (-\infty,-12] \cup [-4,9)$\\
G.$x \in (-\infty,-12) \cup [-4,9]$\\
H.$x \in (-\infty,-12] \cup [-4,9]$
\testStop
\kluczStart
A
\kluczStop



\zadStart{Zadanie z Wikieł Z 1.62 b) moja wersja nr 243}

Rozwiązać nierówności $(x+13)(9-x)(x+4)\ge0$.
\zadStop
\rozwStart{Patryk Wirkus}{Laura Mieczkowska}
Miejsca zerowe naszego wielomianu to: $-13, 9, -4$.\\
Wielomian jest stopnia nieparzystego, ponadto znak współczynnika przy\linebreak najwyższej potędze x jest ujemny.\\ W związku z tym wykres wielomianu zaczyna się od lewej strony powyżej osi OX. A więc $$x \in (-\infty,-13) \cup (-4,9).$$
\rozwStop
\odpStart
$x \in (-\infty,-13) \cup (-4,9)$
\odpStop
\testStart
A.$x \in (-\infty,-13) \cup (-4,9)$\\
B.$x \in (-\infty,-13) \cup (-4,9]$\\
C.$x \in (-\infty,-13) \cup [-4,9)$\\
D.$x \in (-\infty,-13] \cup (-4,9)$\\
E.$x \in (-\infty,-13] \cup (-4,9]$\\
F.$x \in (-\infty,-13] \cup [-4,9)$\\
G.$x \in (-\infty,-13) \cup [-4,9]$\\
H.$x \in (-\infty,-13] \cup [-4,9]$
\testStop
\kluczStart
A
\kluczStop



\zadStart{Zadanie z Wikieł Z 1.62 b) moja wersja nr 244}

Rozwiązać nierówności $(x+14)(9-x)(x+4)\ge0$.
\zadStop
\rozwStart{Patryk Wirkus}{Laura Mieczkowska}
Miejsca zerowe naszego wielomianu to: $-14, 9, -4$.\\
Wielomian jest stopnia nieparzystego, ponadto znak współczynnika przy\linebreak najwyższej potędze x jest ujemny.\\ W związku z tym wykres wielomianu zaczyna się od lewej strony powyżej osi OX. A więc $$x \in (-\infty,-14) \cup (-4,9).$$
\rozwStop
\odpStart
$x \in (-\infty,-14) \cup (-4,9)$
\odpStop
\testStart
A.$x \in (-\infty,-14) \cup (-4,9)$\\
B.$x \in (-\infty,-14) \cup (-4,9]$\\
C.$x \in (-\infty,-14) \cup [-4,9)$\\
D.$x \in (-\infty,-14] \cup (-4,9)$\\
E.$x \in (-\infty,-14] \cup (-4,9]$\\
F.$x \in (-\infty,-14] \cup [-4,9)$\\
G.$x \in (-\infty,-14) \cup [-4,9]$\\
H.$x \in (-\infty,-14] \cup [-4,9]$
\testStop
\kluczStart
A
\kluczStop



\zadStart{Zadanie z Wikieł Z 1.62 b) moja wersja nr 245}

Rozwiązać nierówności $(x+15)(9-x)(x+4)\ge0$.
\zadStop
\rozwStart{Patryk Wirkus}{Laura Mieczkowska}
Miejsca zerowe naszego wielomianu to: $-15, 9, -4$.\\
Wielomian jest stopnia nieparzystego, ponadto znak współczynnika przy\linebreak najwyższej potędze x jest ujemny.\\ W związku z tym wykres wielomianu zaczyna się od lewej strony powyżej osi OX. A więc $$x \in (-\infty,-15) \cup (-4,9).$$
\rozwStop
\odpStart
$x \in (-\infty,-15) \cup (-4,9)$
\odpStop
\testStart
A.$x \in (-\infty,-15) \cup (-4,9)$\\
B.$x \in (-\infty,-15) \cup (-4,9]$\\
C.$x \in (-\infty,-15) \cup [-4,9)$\\
D.$x \in (-\infty,-15] \cup (-4,9)$\\
E.$x \in (-\infty,-15] \cup (-4,9]$\\
F.$x \in (-\infty,-15] \cup [-4,9)$\\
G.$x \in (-\infty,-15) \cup [-4,9]$\\
H.$x \in (-\infty,-15] \cup [-4,9]$
\testStop
\kluczStart
A
\kluczStop



\zadStart{Zadanie z Wikieł Z 1.62 b) moja wersja nr 246}

Rozwiązać nierówności $(x+11)(10-x)(x+4)\ge0$.
\zadStop
\rozwStart{Patryk Wirkus}{Laura Mieczkowska}
Miejsca zerowe naszego wielomianu to: $-11, 10, -4$.\\
Wielomian jest stopnia nieparzystego, ponadto znak współczynnika przy\linebreak najwyższej potędze x jest ujemny.\\ W związku z tym wykres wielomianu zaczyna się od lewej strony powyżej osi OX. A więc $$x \in (-\infty,-11) \cup (-4,10).$$
\rozwStop
\odpStart
$x \in (-\infty,-11) \cup (-4,10)$
\odpStop
\testStart
A.$x \in (-\infty,-11) \cup (-4,10)$\\
B.$x \in (-\infty,-11) \cup (-4,10]$\\
C.$x \in (-\infty,-11) \cup [-4,10)$\\
D.$x \in (-\infty,-11] \cup (-4,10)$\\
E.$x \in (-\infty,-11] \cup (-4,10]$\\
F.$x \in (-\infty,-11] \cup [-4,10)$\\
G.$x \in (-\infty,-11) \cup [-4,10]$\\
H.$x \in (-\infty,-11] \cup [-4,10]$
\testStop
\kluczStart
A
\kluczStop



\zadStart{Zadanie z Wikieł Z 1.62 b) moja wersja nr 247}

Rozwiązać nierówności $(x+12)(10-x)(x+4)\ge0$.
\zadStop
\rozwStart{Patryk Wirkus}{Laura Mieczkowska}
Miejsca zerowe naszego wielomianu to: $-12, 10, -4$.\\
Wielomian jest stopnia nieparzystego, ponadto znak współczynnika przy\linebreak najwyższej potędze x jest ujemny.\\ W związku z tym wykres wielomianu zaczyna się od lewej strony powyżej osi OX. A więc $$x \in (-\infty,-12) \cup (-4,10).$$
\rozwStop
\odpStart
$x \in (-\infty,-12) \cup (-4,10)$
\odpStop
\testStart
A.$x \in (-\infty,-12) \cup (-4,10)$\\
B.$x \in (-\infty,-12) \cup (-4,10]$\\
C.$x \in (-\infty,-12) \cup [-4,10)$\\
D.$x \in (-\infty,-12] \cup (-4,10)$\\
E.$x \in (-\infty,-12] \cup (-4,10]$\\
F.$x \in (-\infty,-12] \cup [-4,10)$\\
G.$x \in (-\infty,-12) \cup [-4,10]$\\
H.$x \in (-\infty,-12] \cup [-4,10]$
\testStop
\kluczStart
A
\kluczStop



\zadStart{Zadanie z Wikieł Z 1.62 b) moja wersja nr 248}

Rozwiązać nierówności $(x+13)(10-x)(x+4)\ge0$.
\zadStop
\rozwStart{Patryk Wirkus}{Laura Mieczkowska}
Miejsca zerowe naszego wielomianu to: $-13, 10, -4$.\\
Wielomian jest stopnia nieparzystego, ponadto znak współczynnika przy\linebreak najwyższej potędze x jest ujemny.\\ W związku z tym wykres wielomianu zaczyna się od lewej strony powyżej osi OX. A więc $$x \in (-\infty,-13) \cup (-4,10).$$
\rozwStop
\odpStart
$x \in (-\infty,-13) \cup (-4,10)$
\odpStop
\testStart
A.$x \in (-\infty,-13) \cup (-4,10)$\\
B.$x \in (-\infty,-13) \cup (-4,10]$\\
C.$x \in (-\infty,-13) \cup [-4,10)$\\
D.$x \in (-\infty,-13] \cup (-4,10)$\\
E.$x \in (-\infty,-13] \cup (-4,10]$\\
F.$x \in (-\infty,-13] \cup [-4,10)$\\
G.$x \in (-\infty,-13) \cup [-4,10]$\\
H.$x \in (-\infty,-13] \cup [-4,10]$
\testStop
\kluczStart
A
\kluczStop



\zadStart{Zadanie z Wikieł Z 1.62 b) moja wersja nr 249}

Rozwiązać nierówności $(x+14)(10-x)(x+4)\ge0$.
\zadStop
\rozwStart{Patryk Wirkus}{Laura Mieczkowska}
Miejsca zerowe naszego wielomianu to: $-14, 10, -4$.\\
Wielomian jest stopnia nieparzystego, ponadto znak współczynnika przy\linebreak najwyższej potędze x jest ujemny.\\ W związku z tym wykres wielomianu zaczyna się od lewej strony powyżej osi OX. A więc $$x \in (-\infty,-14) \cup (-4,10).$$
\rozwStop
\odpStart
$x \in (-\infty,-14) \cup (-4,10)$
\odpStop
\testStart
A.$x \in (-\infty,-14) \cup (-4,10)$\\
B.$x \in (-\infty,-14) \cup (-4,10]$\\
C.$x \in (-\infty,-14) \cup [-4,10)$\\
D.$x \in (-\infty,-14] \cup (-4,10)$\\
E.$x \in (-\infty,-14] \cup (-4,10]$\\
F.$x \in (-\infty,-14] \cup [-4,10)$\\
G.$x \in (-\infty,-14) \cup [-4,10]$\\
H.$x \in (-\infty,-14] \cup [-4,10]$
\testStop
\kluczStart
A
\kluczStop



\zadStart{Zadanie z Wikieł Z 1.62 b) moja wersja nr 250}

Rozwiązać nierówności $(x+15)(10-x)(x+4)\ge0$.
\zadStop
\rozwStart{Patryk Wirkus}{Laura Mieczkowska}
Miejsca zerowe naszego wielomianu to: $-15, 10, -4$.\\
Wielomian jest stopnia nieparzystego, ponadto znak współczynnika przy\linebreak najwyższej potędze x jest ujemny.\\ W związku z tym wykres wielomianu zaczyna się od lewej strony powyżej osi OX. A więc $$x \in (-\infty,-15) \cup (-4,10).$$
\rozwStop
\odpStart
$x \in (-\infty,-15) \cup (-4,10)$
\odpStop
\testStart
A.$x \in (-\infty,-15) \cup (-4,10)$\\
B.$x \in (-\infty,-15) \cup (-4,10]$\\
C.$x \in (-\infty,-15) \cup [-4,10)$\\
D.$x \in (-\infty,-15] \cup (-4,10)$\\
E.$x \in (-\infty,-15] \cup (-4,10]$\\
F.$x \in (-\infty,-15] \cup [-4,10)$\\
G.$x \in (-\infty,-15) \cup [-4,10]$\\
H.$x \in (-\infty,-15] \cup [-4,10]$
\testStop
\kluczStart
A
\kluczStop



\zadStart{Zadanie z Wikieł Z 1.62 b) moja wersja nr 251}

Rozwiązać nierówności $(x+7)(6-x)(x+5)\ge0$.
\zadStop
\rozwStart{Patryk Wirkus}{Laura Mieczkowska}
Miejsca zerowe naszego wielomianu to: $-7, 6, -5$.\\
Wielomian jest stopnia nieparzystego, ponadto znak współczynnika przy\linebreak najwyższej potędze x jest ujemny.\\ W związku z tym wykres wielomianu zaczyna się od lewej strony powyżej osi OX. A więc $$x \in (-\infty,-7) \cup (-5,6).$$
\rozwStop
\odpStart
$x \in (-\infty,-7) \cup (-5,6)$
\odpStop
\testStart
A.$x \in (-\infty,-7) \cup (-5,6)$\\
B.$x \in (-\infty,-7) \cup (-5,6]$\\
C.$x \in (-\infty,-7) \cup [-5,6)$\\
D.$x \in (-\infty,-7] \cup (-5,6)$\\
E.$x \in (-\infty,-7] \cup (-5,6]$\\
F.$x \in (-\infty,-7] \cup [-5,6)$\\
G.$x \in (-\infty,-7) \cup [-5,6]$\\
H.$x \in (-\infty,-7] \cup [-5,6]$
\testStop
\kluczStart
A
\kluczStop



\zadStart{Zadanie z Wikieł Z 1.62 b) moja wersja nr 252}

Rozwiązać nierówności $(x+8)(6-x)(x+5)\ge0$.
\zadStop
\rozwStart{Patryk Wirkus}{Laura Mieczkowska}
Miejsca zerowe naszego wielomianu to: $-8, 6, -5$.\\
Wielomian jest stopnia nieparzystego, ponadto znak współczynnika przy\linebreak najwyższej potędze x jest ujemny.\\ W związku z tym wykres wielomianu zaczyna się od lewej strony powyżej osi OX. A więc $$x \in (-\infty,-8) \cup (-5,6).$$
\rozwStop
\odpStart
$x \in (-\infty,-8) \cup (-5,6)$
\odpStop
\testStart
A.$x \in (-\infty,-8) \cup (-5,6)$\\
B.$x \in (-\infty,-8) \cup (-5,6]$\\
C.$x \in (-\infty,-8) \cup [-5,6)$\\
D.$x \in (-\infty,-8] \cup (-5,6)$\\
E.$x \in (-\infty,-8] \cup (-5,6]$\\
F.$x \in (-\infty,-8] \cup [-5,6)$\\
G.$x \in (-\infty,-8) \cup [-5,6]$\\
H.$x \in (-\infty,-8] \cup [-5,6]$
\testStop
\kluczStart
A
\kluczStop



\zadStart{Zadanie z Wikieł Z 1.62 b) moja wersja nr 253}

Rozwiązać nierówności $(x+9)(6-x)(x+5)\ge0$.
\zadStop
\rozwStart{Patryk Wirkus}{Laura Mieczkowska}
Miejsca zerowe naszego wielomianu to: $-9, 6, -5$.\\
Wielomian jest stopnia nieparzystego, ponadto znak współczynnika przy\linebreak najwyższej potędze x jest ujemny.\\ W związku z tym wykres wielomianu zaczyna się od lewej strony powyżej osi OX. A więc $$x \in (-\infty,-9) \cup (-5,6).$$
\rozwStop
\odpStart
$x \in (-\infty,-9) \cup (-5,6)$
\odpStop
\testStart
A.$x \in (-\infty,-9) \cup (-5,6)$\\
B.$x \in (-\infty,-9) \cup (-5,6]$\\
C.$x \in (-\infty,-9) \cup [-5,6)$\\
D.$x \in (-\infty,-9] \cup (-5,6)$\\
E.$x \in (-\infty,-9] \cup (-5,6]$\\
F.$x \in (-\infty,-9] \cup [-5,6)$\\
G.$x \in (-\infty,-9) \cup [-5,6]$\\
H.$x \in (-\infty,-9] \cup [-5,6]$
\testStop
\kluczStart
A
\kluczStop



\zadStart{Zadanie z Wikieł Z 1.62 b) moja wersja nr 254}

Rozwiązać nierówności $(x+10)(6-x)(x+5)\ge0$.
\zadStop
\rozwStart{Patryk Wirkus}{Laura Mieczkowska}
Miejsca zerowe naszego wielomianu to: $-10, 6, -5$.\\
Wielomian jest stopnia nieparzystego, ponadto znak współczynnika przy\linebreak najwyższej potędze x jest ujemny.\\ W związku z tym wykres wielomianu zaczyna się od lewej strony powyżej osi OX. A więc $$x \in (-\infty,-10) \cup (-5,6).$$
\rozwStop
\odpStart
$x \in (-\infty,-10) \cup (-5,6)$
\odpStop
\testStart
A.$x \in (-\infty,-10) \cup (-5,6)$\\
B.$x \in (-\infty,-10) \cup (-5,6]$\\
C.$x \in (-\infty,-10) \cup [-5,6)$\\
D.$x \in (-\infty,-10] \cup (-5,6)$\\
E.$x \in (-\infty,-10] \cup (-5,6]$\\
F.$x \in (-\infty,-10] \cup [-5,6)$\\
G.$x \in (-\infty,-10) \cup [-5,6]$\\
H.$x \in (-\infty,-10] \cup [-5,6]$
\testStop
\kluczStart
A
\kluczStop



\zadStart{Zadanie z Wikieł Z 1.62 b) moja wersja nr 255}

Rozwiązać nierówności $(x+11)(6-x)(x+5)\ge0$.
\zadStop
\rozwStart{Patryk Wirkus}{Laura Mieczkowska}
Miejsca zerowe naszego wielomianu to: $-11, 6, -5$.\\
Wielomian jest stopnia nieparzystego, ponadto znak współczynnika przy\linebreak najwyższej potędze x jest ujemny.\\ W związku z tym wykres wielomianu zaczyna się od lewej strony powyżej osi OX. A więc $$x \in (-\infty,-11) \cup (-5,6).$$
\rozwStop
\odpStart
$x \in (-\infty,-11) \cup (-5,6)$
\odpStop
\testStart
A.$x \in (-\infty,-11) \cup (-5,6)$\\
B.$x \in (-\infty,-11) \cup (-5,6]$\\
C.$x \in (-\infty,-11) \cup [-5,6)$\\
D.$x \in (-\infty,-11] \cup (-5,6)$\\
E.$x \in (-\infty,-11] \cup (-5,6]$\\
F.$x \in (-\infty,-11] \cup [-5,6)$\\
G.$x \in (-\infty,-11) \cup [-5,6]$\\
H.$x \in (-\infty,-11] \cup [-5,6]$
\testStop
\kluczStart
A
\kluczStop



\zadStart{Zadanie z Wikieł Z 1.62 b) moja wersja nr 256}

Rozwiązać nierówności $(x+12)(6-x)(x+5)\ge0$.
\zadStop
\rozwStart{Patryk Wirkus}{Laura Mieczkowska}
Miejsca zerowe naszego wielomianu to: $-12, 6, -5$.\\
Wielomian jest stopnia nieparzystego, ponadto znak współczynnika przy\linebreak najwyższej potędze x jest ujemny.\\ W związku z tym wykres wielomianu zaczyna się od lewej strony powyżej osi OX. A więc $$x \in (-\infty,-12) \cup (-5,6).$$
\rozwStop
\odpStart
$x \in (-\infty,-12) \cup (-5,6)$
\odpStop
\testStart
A.$x \in (-\infty,-12) \cup (-5,6)$\\
B.$x \in (-\infty,-12) \cup (-5,6]$\\
C.$x \in (-\infty,-12) \cup [-5,6)$\\
D.$x \in (-\infty,-12] \cup (-5,6)$\\
E.$x \in (-\infty,-12] \cup (-5,6]$\\
F.$x \in (-\infty,-12] \cup [-5,6)$\\
G.$x \in (-\infty,-12) \cup [-5,6]$\\
H.$x \in (-\infty,-12] \cup [-5,6]$
\testStop
\kluczStart
A
\kluczStop



\zadStart{Zadanie z Wikieł Z 1.62 b) moja wersja nr 257}

Rozwiązać nierówności $(x+13)(6-x)(x+5)\ge0$.
\zadStop
\rozwStart{Patryk Wirkus}{Laura Mieczkowska}
Miejsca zerowe naszego wielomianu to: $-13, 6, -5$.\\
Wielomian jest stopnia nieparzystego, ponadto znak współczynnika przy\linebreak najwyższej potędze x jest ujemny.\\ W związku z tym wykres wielomianu zaczyna się od lewej strony powyżej osi OX. A więc $$x \in (-\infty,-13) \cup (-5,6).$$
\rozwStop
\odpStart
$x \in (-\infty,-13) \cup (-5,6)$
\odpStop
\testStart
A.$x \in (-\infty,-13) \cup (-5,6)$\\
B.$x \in (-\infty,-13) \cup (-5,6]$\\
C.$x \in (-\infty,-13) \cup [-5,6)$\\
D.$x \in (-\infty,-13] \cup (-5,6)$\\
E.$x \in (-\infty,-13] \cup (-5,6]$\\
F.$x \in (-\infty,-13] \cup [-5,6)$\\
G.$x \in (-\infty,-13) \cup [-5,6]$\\
H.$x \in (-\infty,-13] \cup [-5,6]$
\testStop
\kluczStart
A
\kluczStop



\zadStart{Zadanie z Wikieł Z 1.62 b) moja wersja nr 258}

Rozwiązać nierówności $(x+14)(6-x)(x+5)\ge0$.
\zadStop
\rozwStart{Patryk Wirkus}{Laura Mieczkowska}
Miejsca zerowe naszego wielomianu to: $-14, 6, -5$.\\
Wielomian jest stopnia nieparzystego, ponadto znak współczynnika przy\linebreak najwyższej potędze x jest ujemny.\\ W związku z tym wykres wielomianu zaczyna się od lewej strony powyżej osi OX. A więc $$x \in (-\infty,-14) \cup (-5,6).$$
\rozwStop
\odpStart
$x \in (-\infty,-14) \cup (-5,6)$
\odpStop
\testStart
A.$x \in (-\infty,-14) \cup (-5,6)$\\
B.$x \in (-\infty,-14) \cup (-5,6]$\\
C.$x \in (-\infty,-14) \cup [-5,6)$\\
D.$x \in (-\infty,-14] \cup (-5,6)$\\
E.$x \in (-\infty,-14] \cup (-5,6]$\\
F.$x \in (-\infty,-14] \cup [-5,6)$\\
G.$x \in (-\infty,-14) \cup [-5,6]$\\
H.$x \in (-\infty,-14] \cup [-5,6]$
\testStop
\kluczStart
A
\kluczStop



\zadStart{Zadanie z Wikieł Z 1.62 b) moja wersja nr 259}

Rozwiązać nierówności $(x+15)(6-x)(x+5)\ge0$.
\zadStop
\rozwStart{Patryk Wirkus}{Laura Mieczkowska}
Miejsca zerowe naszego wielomianu to: $-15, 6, -5$.\\
Wielomian jest stopnia nieparzystego, ponadto znak współczynnika przy\linebreak najwyższej potędze x jest ujemny.\\ W związku z tym wykres wielomianu zaczyna się od lewej strony powyżej osi OX. A więc $$x \in (-\infty,-15) \cup (-5,6).$$
\rozwStop
\odpStart
$x \in (-\infty,-15) \cup (-5,6)$
\odpStop
\testStart
A.$x \in (-\infty,-15) \cup (-5,6)$\\
B.$x \in (-\infty,-15) \cup (-5,6]$\\
C.$x \in (-\infty,-15) \cup [-5,6)$\\
D.$x \in (-\infty,-15] \cup (-5,6)$\\
E.$x \in (-\infty,-15] \cup (-5,6]$\\
F.$x \in (-\infty,-15] \cup [-5,6)$\\
G.$x \in (-\infty,-15) \cup [-5,6]$\\
H.$x \in (-\infty,-15] \cup [-5,6]$
\testStop
\kluczStart
A
\kluczStop



\zadStart{Zadanie z Wikieł Z 1.62 b) moja wersja nr 260}

Rozwiązać nierówności $(x+8)(7-x)(x+5)\ge0$.
\zadStop
\rozwStart{Patryk Wirkus}{Laura Mieczkowska}
Miejsca zerowe naszego wielomianu to: $-8, 7, -5$.\\
Wielomian jest stopnia nieparzystego, ponadto znak współczynnika przy\linebreak najwyższej potędze x jest ujemny.\\ W związku z tym wykres wielomianu zaczyna się od lewej strony powyżej osi OX. A więc $$x \in (-\infty,-8) \cup (-5,7).$$
\rozwStop
\odpStart
$x \in (-\infty,-8) \cup (-5,7)$
\odpStop
\testStart
A.$x \in (-\infty,-8) \cup (-5,7)$\\
B.$x \in (-\infty,-8) \cup (-5,7]$\\
C.$x \in (-\infty,-8) \cup [-5,7)$\\
D.$x \in (-\infty,-8] \cup (-5,7)$\\
E.$x \in (-\infty,-8] \cup (-5,7]$\\
F.$x \in (-\infty,-8] \cup [-5,7)$\\
G.$x \in (-\infty,-8) \cup [-5,7]$\\
H.$x \in (-\infty,-8] \cup [-5,7]$
\testStop
\kluczStart
A
\kluczStop



\zadStart{Zadanie z Wikieł Z 1.62 b) moja wersja nr 261}

Rozwiązać nierówności $(x+9)(7-x)(x+5)\ge0$.
\zadStop
\rozwStart{Patryk Wirkus}{Laura Mieczkowska}
Miejsca zerowe naszego wielomianu to: $-9, 7, -5$.\\
Wielomian jest stopnia nieparzystego, ponadto znak współczynnika przy\linebreak najwyższej potędze x jest ujemny.\\ W związku z tym wykres wielomianu zaczyna się od lewej strony powyżej osi OX. A więc $$x \in (-\infty,-9) \cup (-5,7).$$
\rozwStop
\odpStart
$x \in (-\infty,-9) \cup (-5,7)$
\odpStop
\testStart
A.$x \in (-\infty,-9) \cup (-5,7)$\\
B.$x \in (-\infty,-9) \cup (-5,7]$\\
C.$x \in (-\infty,-9) \cup [-5,7)$\\
D.$x \in (-\infty,-9] \cup (-5,7)$\\
E.$x \in (-\infty,-9] \cup (-5,7]$\\
F.$x \in (-\infty,-9] \cup [-5,7)$\\
G.$x \in (-\infty,-9) \cup [-5,7]$\\
H.$x \in (-\infty,-9] \cup [-5,7]$
\testStop
\kluczStart
A
\kluczStop



\zadStart{Zadanie z Wikieł Z 1.62 b) moja wersja nr 262}

Rozwiązać nierówności $(x+10)(7-x)(x+5)\ge0$.
\zadStop
\rozwStart{Patryk Wirkus}{Laura Mieczkowska}
Miejsca zerowe naszego wielomianu to: $-10, 7, -5$.\\
Wielomian jest stopnia nieparzystego, ponadto znak współczynnika przy\linebreak najwyższej potędze x jest ujemny.\\ W związku z tym wykres wielomianu zaczyna się od lewej strony powyżej osi OX. A więc $$x \in (-\infty,-10) \cup (-5,7).$$
\rozwStop
\odpStart
$x \in (-\infty,-10) \cup (-5,7)$
\odpStop
\testStart
A.$x \in (-\infty,-10) \cup (-5,7)$\\
B.$x \in (-\infty,-10) \cup (-5,7]$\\
C.$x \in (-\infty,-10) \cup [-5,7)$\\
D.$x \in (-\infty,-10] \cup (-5,7)$\\
E.$x \in (-\infty,-10] \cup (-5,7]$\\
F.$x \in (-\infty,-10] \cup [-5,7)$\\
G.$x \in (-\infty,-10) \cup [-5,7]$\\
H.$x \in (-\infty,-10] \cup [-5,7]$
\testStop
\kluczStart
A
\kluczStop



\zadStart{Zadanie z Wikieł Z 1.62 b) moja wersja nr 263}

Rozwiązać nierówności $(x+11)(7-x)(x+5)\ge0$.
\zadStop
\rozwStart{Patryk Wirkus}{Laura Mieczkowska}
Miejsca zerowe naszego wielomianu to: $-11, 7, -5$.\\
Wielomian jest stopnia nieparzystego, ponadto znak współczynnika przy\linebreak najwyższej potędze x jest ujemny.\\ W związku z tym wykres wielomianu zaczyna się od lewej strony powyżej osi OX. A więc $$x \in (-\infty,-11) \cup (-5,7).$$
\rozwStop
\odpStart
$x \in (-\infty,-11) \cup (-5,7)$
\odpStop
\testStart
A.$x \in (-\infty,-11) \cup (-5,7)$\\
B.$x \in (-\infty,-11) \cup (-5,7]$\\
C.$x \in (-\infty,-11) \cup [-5,7)$\\
D.$x \in (-\infty,-11] \cup (-5,7)$\\
E.$x \in (-\infty,-11] \cup (-5,7]$\\
F.$x \in (-\infty,-11] \cup [-5,7)$\\
G.$x \in (-\infty,-11) \cup [-5,7]$\\
H.$x \in (-\infty,-11] \cup [-5,7]$
\testStop
\kluczStart
A
\kluczStop



\zadStart{Zadanie z Wikieł Z 1.62 b) moja wersja nr 264}

Rozwiązać nierówności $(x+12)(7-x)(x+5)\ge0$.
\zadStop
\rozwStart{Patryk Wirkus}{Laura Mieczkowska}
Miejsca zerowe naszego wielomianu to: $-12, 7, -5$.\\
Wielomian jest stopnia nieparzystego, ponadto znak współczynnika przy\linebreak najwyższej potędze x jest ujemny.\\ W związku z tym wykres wielomianu zaczyna się od lewej strony powyżej osi OX. A więc $$x \in (-\infty,-12) \cup (-5,7).$$
\rozwStop
\odpStart
$x \in (-\infty,-12) \cup (-5,7)$
\odpStop
\testStart
A.$x \in (-\infty,-12) \cup (-5,7)$\\
B.$x \in (-\infty,-12) \cup (-5,7]$\\
C.$x \in (-\infty,-12) \cup [-5,7)$\\
D.$x \in (-\infty,-12] \cup (-5,7)$\\
E.$x \in (-\infty,-12] \cup (-5,7]$\\
F.$x \in (-\infty,-12] \cup [-5,7)$\\
G.$x \in (-\infty,-12) \cup [-5,7]$\\
H.$x \in (-\infty,-12] \cup [-5,7]$
\testStop
\kluczStart
A
\kluczStop



\zadStart{Zadanie z Wikieł Z 1.62 b) moja wersja nr 265}

Rozwiązać nierówności $(x+13)(7-x)(x+5)\ge0$.
\zadStop
\rozwStart{Patryk Wirkus}{Laura Mieczkowska}
Miejsca zerowe naszego wielomianu to: $-13, 7, -5$.\\
Wielomian jest stopnia nieparzystego, ponadto znak współczynnika przy\linebreak najwyższej potędze x jest ujemny.\\ W związku z tym wykres wielomianu zaczyna się od lewej strony powyżej osi OX. A więc $$x \in (-\infty,-13) \cup (-5,7).$$
\rozwStop
\odpStart
$x \in (-\infty,-13) \cup (-5,7)$
\odpStop
\testStart
A.$x \in (-\infty,-13) \cup (-5,7)$\\
B.$x \in (-\infty,-13) \cup (-5,7]$\\
C.$x \in (-\infty,-13) \cup [-5,7)$\\
D.$x \in (-\infty,-13] \cup (-5,7)$\\
E.$x \in (-\infty,-13] \cup (-5,7]$\\
F.$x \in (-\infty,-13] \cup [-5,7)$\\
G.$x \in (-\infty,-13) \cup [-5,7]$\\
H.$x \in (-\infty,-13] \cup [-5,7]$
\testStop
\kluczStart
A
\kluczStop



\zadStart{Zadanie z Wikieł Z 1.62 b) moja wersja nr 266}

Rozwiązać nierówności $(x+14)(7-x)(x+5)\ge0$.
\zadStop
\rozwStart{Patryk Wirkus}{Laura Mieczkowska}
Miejsca zerowe naszego wielomianu to: $-14, 7, -5$.\\
Wielomian jest stopnia nieparzystego, ponadto znak współczynnika przy\linebreak najwyższej potędze x jest ujemny.\\ W związku z tym wykres wielomianu zaczyna się od lewej strony powyżej osi OX. A więc $$x \in (-\infty,-14) \cup (-5,7).$$
\rozwStop
\odpStart
$x \in (-\infty,-14) \cup (-5,7)$
\odpStop
\testStart
A.$x \in (-\infty,-14) \cup (-5,7)$\\
B.$x \in (-\infty,-14) \cup (-5,7]$\\
C.$x \in (-\infty,-14) \cup [-5,7)$\\
D.$x \in (-\infty,-14] \cup (-5,7)$\\
E.$x \in (-\infty,-14] \cup (-5,7]$\\
F.$x \in (-\infty,-14] \cup [-5,7)$\\
G.$x \in (-\infty,-14) \cup [-5,7]$\\
H.$x \in (-\infty,-14] \cup [-5,7]$
\testStop
\kluczStart
A
\kluczStop



\zadStart{Zadanie z Wikieł Z 1.62 b) moja wersja nr 267}

Rozwiązać nierówności $(x+15)(7-x)(x+5)\ge0$.
\zadStop
\rozwStart{Patryk Wirkus}{Laura Mieczkowska}
Miejsca zerowe naszego wielomianu to: $-15, 7, -5$.\\
Wielomian jest stopnia nieparzystego, ponadto znak współczynnika przy\linebreak najwyższej potędze x jest ujemny.\\ W związku z tym wykres wielomianu zaczyna się od lewej strony powyżej osi OX. A więc $$x \in (-\infty,-15) \cup (-5,7).$$
\rozwStop
\odpStart
$x \in (-\infty,-15) \cup (-5,7)$
\odpStop
\testStart
A.$x \in (-\infty,-15) \cup (-5,7)$\\
B.$x \in (-\infty,-15) \cup (-5,7]$\\
C.$x \in (-\infty,-15) \cup [-5,7)$\\
D.$x \in (-\infty,-15] \cup (-5,7)$\\
E.$x \in (-\infty,-15] \cup (-5,7]$\\
F.$x \in (-\infty,-15] \cup [-5,7)$\\
G.$x \in (-\infty,-15) \cup [-5,7]$\\
H.$x \in (-\infty,-15] \cup [-5,7]$
\testStop
\kluczStart
A
\kluczStop



\zadStart{Zadanie z Wikieł Z 1.62 b) moja wersja nr 268}

Rozwiązać nierówności $(x+9)(8-x)(x+5)\ge0$.
\zadStop
\rozwStart{Patryk Wirkus}{Laura Mieczkowska}
Miejsca zerowe naszego wielomianu to: $-9, 8, -5$.\\
Wielomian jest stopnia nieparzystego, ponadto znak współczynnika przy\linebreak najwyższej potędze x jest ujemny.\\ W związku z tym wykres wielomianu zaczyna się od lewej strony powyżej osi OX. A więc $$x \in (-\infty,-9) \cup (-5,8).$$
\rozwStop
\odpStart
$x \in (-\infty,-9) \cup (-5,8)$
\odpStop
\testStart
A.$x \in (-\infty,-9) \cup (-5,8)$\\
B.$x \in (-\infty,-9) \cup (-5,8]$\\
C.$x \in (-\infty,-9) \cup [-5,8)$\\
D.$x \in (-\infty,-9] \cup (-5,8)$\\
E.$x \in (-\infty,-9] \cup (-5,8]$\\
F.$x \in (-\infty,-9] \cup [-5,8)$\\
G.$x \in (-\infty,-9) \cup [-5,8]$\\
H.$x \in (-\infty,-9] \cup [-5,8]$
\testStop
\kluczStart
A
\kluczStop



\zadStart{Zadanie z Wikieł Z 1.62 b) moja wersja nr 269}

Rozwiązać nierówności $(x+10)(8-x)(x+5)\ge0$.
\zadStop
\rozwStart{Patryk Wirkus}{Laura Mieczkowska}
Miejsca zerowe naszego wielomianu to: $-10, 8, -5$.\\
Wielomian jest stopnia nieparzystego, ponadto znak współczynnika przy\linebreak najwyższej potędze x jest ujemny.\\ W związku z tym wykres wielomianu zaczyna się od lewej strony powyżej osi OX. A więc $$x \in (-\infty,-10) \cup (-5,8).$$
\rozwStop
\odpStart
$x \in (-\infty,-10) \cup (-5,8)$
\odpStop
\testStart
A.$x \in (-\infty,-10) \cup (-5,8)$\\
B.$x \in (-\infty,-10) \cup (-5,8]$\\
C.$x \in (-\infty,-10) \cup [-5,8)$\\
D.$x \in (-\infty,-10] \cup (-5,8)$\\
E.$x \in (-\infty,-10] \cup (-5,8]$\\
F.$x \in (-\infty,-10] \cup [-5,8)$\\
G.$x \in (-\infty,-10) \cup [-5,8]$\\
H.$x \in (-\infty,-10] \cup [-5,8]$
\testStop
\kluczStart
A
\kluczStop



\zadStart{Zadanie z Wikieł Z 1.62 b) moja wersja nr 270}

Rozwiązać nierówności $(x+11)(8-x)(x+5)\ge0$.
\zadStop
\rozwStart{Patryk Wirkus}{Laura Mieczkowska}
Miejsca zerowe naszego wielomianu to: $-11, 8, -5$.\\
Wielomian jest stopnia nieparzystego, ponadto znak współczynnika przy\linebreak najwyższej potędze x jest ujemny.\\ W związku z tym wykres wielomianu zaczyna się od lewej strony powyżej osi OX. A więc $$x \in (-\infty,-11) \cup (-5,8).$$
\rozwStop
\odpStart
$x \in (-\infty,-11) \cup (-5,8)$
\odpStop
\testStart
A.$x \in (-\infty,-11) \cup (-5,8)$\\
B.$x \in (-\infty,-11) \cup (-5,8]$\\
C.$x \in (-\infty,-11) \cup [-5,8)$\\
D.$x \in (-\infty,-11] \cup (-5,8)$\\
E.$x \in (-\infty,-11] \cup (-5,8]$\\
F.$x \in (-\infty,-11] \cup [-5,8)$\\
G.$x \in (-\infty,-11) \cup [-5,8]$\\
H.$x \in (-\infty,-11] \cup [-5,8]$
\testStop
\kluczStart
A
\kluczStop



\zadStart{Zadanie z Wikieł Z 1.62 b) moja wersja nr 271}

Rozwiązać nierówności $(x+12)(8-x)(x+5)\ge0$.
\zadStop
\rozwStart{Patryk Wirkus}{Laura Mieczkowska}
Miejsca zerowe naszego wielomianu to: $-12, 8, -5$.\\
Wielomian jest stopnia nieparzystego, ponadto znak współczynnika przy\linebreak najwyższej potędze x jest ujemny.\\ W związku z tym wykres wielomianu zaczyna się od lewej strony powyżej osi OX. A więc $$x \in (-\infty,-12) \cup (-5,8).$$
\rozwStop
\odpStart
$x \in (-\infty,-12) \cup (-5,8)$
\odpStop
\testStart
A.$x \in (-\infty,-12) \cup (-5,8)$\\
B.$x \in (-\infty,-12) \cup (-5,8]$\\
C.$x \in (-\infty,-12) \cup [-5,8)$\\
D.$x \in (-\infty,-12] \cup (-5,8)$\\
E.$x \in (-\infty,-12] \cup (-5,8]$\\
F.$x \in (-\infty,-12] \cup [-5,8)$\\
G.$x \in (-\infty,-12) \cup [-5,8]$\\
H.$x \in (-\infty,-12] \cup [-5,8]$
\testStop
\kluczStart
A
\kluczStop



\zadStart{Zadanie z Wikieł Z 1.62 b) moja wersja nr 272}

Rozwiązać nierówności $(x+13)(8-x)(x+5)\ge0$.
\zadStop
\rozwStart{Patryk Wirkus}{Laura Mieczkowska}
Miejsca zerowe naszego wielomianu to: $-13, 8, -5$.\\
Wielomian jest stopnia nieparzystego, ponadto znak współczynnika przy\linebreak najwyższej potędze x jest ujemny.\\ W związku z tym wykres wielomianu zaczyna się od lewej strony powyżej osi OX. A więc $$x \in (-\infty,-13) \cup (-5,8).$$
\rozwStop
\odpStart
$x \in (-\infty,-13) \cup (-5,8)$
\odpStop
\testStart
A.$x \in (-\infty,-13) \cup (-5,8)$\\
B.$x \in (-\infty,-13) \cup (-5,8]$\\
C.$x \in (-\infty,-13) \cup [-5,8)$\\
D.$x \in (-\infty,-13] \cup (-5,8)$\\
E.$x \in (-\infty,-13] \cup (-5,8]$\\
F.$x \in (-\infty,-13] \cup [-5,8)$\\
G.$x \in (-\infty,-13) \cup [-5,8]$\\
H.$x \in (-\infty,-13] \cup [-5,8]$
\testStop
\kluczStart
A
\kluczStop



\zadStart{Zadanie z Wikieł Z 1.62 b) moja wersja nr 273}

Rozwiązać nierówności $(x+14)(8-x)(x+5)\ge0$.
\zadStop
\rozwStart{Patryk Wirkus}{Laura Mieczkowska}
Miejsca zerowe naszego wielomianu to: $-14, 8, -5$.\\
Wielomian jest stopnia nieparzystego, ponadto znak współczynnika przy\linebreak najwyższej potędze x jest ujemny.\\ W związku z tym wykres wielomianu zaczyna się od lewej strony powyżej osi OX. A więc $$x \in (-\infty,-14) \cup (-5,8).$$
\rozwStop
\odpStart
$x \in (-\infty,-14) \cup (-5,8)$
\odpStop
\testStart
A.$x \in (-\infty,-14) \cup (-5,8)$\\
B.$x \in (-\infty,-14) \cup (-5,8]$\\
C.$x \in (-\infty,-14) \cup [-5,8)$\\
D.$x \in (-\infty,-14] \cup (-5,8)$\\
E.$x \in (-\infty,-14] \cup (-5,8]$\\
F.$x \in (-\infty,-14] \cup [-5,8)$\\
G.$x \in (-\infty,-14) \cup [-5,8]$\\
H.$x \in (-\infty,-14] \cup [-5,8]$
\testStop
\kluczStart
A
\kluczStop



\zadStart{Zadanie z Wikieł Z 1.62 b) moja wersja nr 274}

Rozwiązać nierówności $(x+15)(8-x)(x+5)\ge0$.
\zadStop
\rozwStart{Patryk Wirkus}{Laura Mieczkowska}
Miejsca zerowe naszego wielomianu to: $-15, 8, -5$.\\
Wielomian jest stopnia nieparzystego, ponadto znak współczynnika przy\linebreak najwyższej potędze x jest ujemny.\\ W związku z tym wykres wielomianu zaczyna się od lewej strony powyżej osi OX. A więc $$x \in (-\infty,-15) \cup (-5,8).$$
\rozwStop
\odpStart
$x \in (-\infty,-15) \cup (-5,8)$
\odpStop
\testStart
A.$x \in (-\infty,-15) \cup (-5,8)$\\
B.$x \in (-\infty,-15) \cup (-5,8]$\\
C.$x \in (-\infty,-15) \cup [-5,8)$\\
D.$x \in (-\infty,-15] \cup (-5,8)$\\
E.$x \in (-\infty,-15] \cup (-5,8]$\\
F.$x \in (-\infty,-15] \cup [-5,8)$\\
G.$x \in (-\infty,-15) \cup [-5,8]$\\
H.$x \in (-\infty,-15] \cup [-5,8]$
\testStop
\kluczStart
A
\kluczStop



\zadStart{Zadanie z Wikieł Z 1.62 b) moja wersja nr 275}

Rozwiązać nierówności $(x+10)(9-x)(x+5)\ge0$.
\zadStop
\rozwStart{Patryk Wirkus}{Laura Mieczkowska}
Miejsca zerowe naszego wielomianu to: $-10, 9, -5$.\\
Wielomian jest stopnia nieparzystego, ponadto znak współczynnika przy\linebreak najwyższej potędze x jest ujemny.\\ W związku z tym wykres wielomianu zaczyna się od lewej strony powyżej osi OX. A więc $$x \in (-\infty,-10) \cup (-5,9).$$
\rozwStop
\odpStart
$x \in (-\infty,-10) \cup (-5,9)$
\odpStop
\testStart
A.$x \in (-\infty,-10) \cup (-5,9)$\\
B.$x \in (-\infty,-10) \cup (-5,9]$\\
C.$x \in (-\infty,-10) \cup [-5,9)$\\
D.$x \in (-\infty,-10] \cup (-5,9)$\\
E.$x \in (-\infty,-10] \cup (-5,9]$\\
F.$x \in (-\infty,-10] \cup [-5,9)$\\
G.$x \in (-\infty,-10) \cup [-5,9]$\\
H.$x \in (-\infty,-10] \cup [-5,9]$
\testStop
\kluczStart
A
\kluczStop



\zadStart{Zadanie z Wikieł Z 1.62 b) moja wersja nr 276}

Rozwiązać nierówności $(x+11)(9-x)(x+5)\ge0$.
\zadStop
\rozwStart{Patryk Wirkus}{Laura Mieczkowska}
Miejsca zerowe naszego wielomianu to: $-11, 9, -5$.\\
Wielomian jest stopnia nieparzystego, ponadto znak współczynnika przy\linebreak najwyższej potędze x jest ujemny.\\ W związku z tym wykres wielomianu zaczyna się od lewej strony powyżej osi OX. A więc $$x \in (-\infty,-11) \cup (-5,9).$$
\rozwStop
\odpStart
$x \in (-\infty,-11) \cup (-5,9)$
\odpStop
\testStart
A.$x \in (-\infty,-11) \cup (-5,9)$\\
B.$x \in (-\infty,-11) \cup (-5,9]$\\
C.$x \in (-\infty,-11) \cup [-5,9)$\\
D.$x \in (-\infty,-11] \cup (-5,9)$\\
E.$x \in (-\infty,-11] \cup (-5,9]$\\
F.$x \in (-\infty,-11] \cup [-5,9)$\\
G.$x \in (-\infty,-11) \cup [-5,9]$\\
H.$x \in (-\infty,-11] \cup [-5,9]$
\testStop
\kluczStart
A
\kluczStop



\zadStart{Zadanie z Wikieł Z 1.62 b) moja wersja nr 277}

Rozwiązać nierówności $(x+12)(9-x)(x+5)\ge0$.
\zadStop
\rozwStart{Patryk Wirkus}{Laura Mieczkowska}
Miejsca zerowe naszego wielomianu to: $-12, 9, -5$.\\
Wielomian jest stopnia nieparzystego, ponadto znak współczynnika przy\linebreak najwyższej potędze x jest ujemny.\\ W związku z tym wykres wielomianu zaczyna się od lewej strony powyżej osi OX. A więc $$x \in (-\infty,-12) \cup (-5,9).$$
\rozwStop
\odpStart
$x \in (-\infty,-12) \cup (-5,9)$
\odpStop
\testStart
A.$x \in (-\infty,-12) \cup (-5,9)$\\
B.$x \in (-\infty,-12) \cup (-5,9]$\\
C.$x \in (-\infty,-12) \cup [-5,9)$\\
D.$x \in (-\infty,-12] \cup (-5,9)$\\
E.$x \in (-\infty,-12] \cup (-5,9]$\\
F.$x \in (-\infty,-12] \cup [-5,9)$\\
G.$x \in (-\infty,-12) \cup [-5,9]$\\
H.$x \in (-\infty,-12] \cup [-5,9]$
\testStop
\kluczStart
A
\kluczStop



\zadStart{Zadanie z Wikieł Z 1.62 b) moja wersja nr 278}

Rozwiązać nierówności $(x+13)(9-x)(x+5)\ge0$.
\zadStop
\rozwStart{Patryk Wirkus}{Laura Mieczkowska}
Miejsca zerowe naszego wielomianu to: $-13, 9, -5$.\\
Wielomian jest stopnia nieparzystego, ponadto znak współczynnika przy\linebreak najwyższej potędze x jest ujemny.\\ W związku z tym wykres wielomianu zaczyna się od lewej strony powyżej osi OX. A więc $$x \in (-\infty,-13) \cup (-5,9).$$
\rozwStop
\odpStart
$x \in (-\infty,-13) \cup (-5,9)$
\odpStop
\testStart
A.$x \in (-\infty,-13) \cup (-5,9)$\\
B.$x \in (-\infty,-13) \cup (-5,9]$\\
C.$x \in (-\infty,-13) \cup [-5,9)$\\
D.$x \in (-\infty,-13] \cup (-5,9)$\\
E.$x \in (-\infty,-13] \cup (-5,9]$\\
F.$x \in (-\infty,-13] \cup [-5,9)$\\
G.$x \in (-\infty,-13) \cup [-5,9]$\\
H.$x \in (-\infty,-13] \cup [-5,9]$
\testStop
\kluczStart
A
\kluczStop



\zadStart{Zadanie z Wikieł Z 1.62 b) moja wersja nr 279}

Rozwiązać nierówności $(x+14)(9-x)(x+5)\ge0$.
\zadStop
\rozwStart{Patryk Wirkus}{Laura Mieczkowska}
Miejsca zerowe naszego wielomianu to: $-14, 9, -5$.\\
Wielomian jest stopnia nieparzystego, ponadto znak współczynnika przy\linebreak najwyższej potędze x jest ujemny.\\ W związku z tym wykres wielomianu zaczyna się od lewej strony powyżej osi OX. A więc $$x \in (-\infty,-14) \cup (-5,9).$$
\rozwStop
\odpStart
$x \in (-\infty,-14) \cup (-5,9)$
\odpStop
\testStart
A.$x \in (-\infty,-14) \cup (-5,9)$\\
B.$x \in (-\infty,-14) \cup (-5,9]$\\
C.$x \in (-\infty,-14) \cup [-5,9)$\\
D.$x \in (-\infty,-14] \cup (-5,9)$\\
E.$x \in (-\infty,-14] \cup (-5,9]$\\
F.$x \in (-\infty,-14] \cup [-5,9)$\\
G.$x \in (-\infty,-14) \cup [-5,9]$\\
H.$x \in (-\infty,-14] \cup [-5,9]$
\testStop
\kluczStart
A
\kluczStop



\zadStart{Zadanie z Wikieł Z 1.62 b) moja wersja nr 280}

Rozwiązać nierówności $(x+15)(9-x)(x+5)\ge0$.
\zadStop
\rozwStart{Patryk Wirkus}{Laura Mieczkowska}
Miejsca zerowe naszego wielomianu to: $-15, 9, -5$.\\
Wielomian jest stopnia nieparzystego, ponadto znak współczynnika przy\linebreak najwyższej potędze x jest ujemny.\\ W związku z tym wykres wielomianu zaczyna się od lewej strony powyżej osi OX. A więc $$x \in (-\infty,-15) \cup (-5,9).$$
\rozwStop
\odpStart
$x \in (-\infty,-15) \cup (-5,9)$
\odpStop
\testStart
A.$x \in (-\infty,-15) \cup (-5,9)$\\
B.$x \in (-\infty,-15) \cup (-5,9]$\\
C.$x \in (-\infty,-15) \cup [-5,9)$\\
D.$x \in (-\infty,-15] \cup (-5,9)$\\
E.$x \in (-\infty,-15] \cup (-5,9]$\\
F.$x \in (-\infty,-15] \cup [-5,9)$\\
G.$x \in (-\infty,-15) \cup [-5,9]$\\
H.$x \in (-\infty,-15] \cup [-5,9]$
\testStop
\kluczStart
A
\kluczStop



\zadStart{Zadanie z Wikieł Z 1.62 b) moja wersja nr 281}

Rozwiązać nierówności $(x+11)(10-x)(x+5)\ge0$.
\zadStop
\rozwStart{Patryk Wirkus}{Laura Mieczkowska}
Miejsca zerowe naszego wielomianu to: $-11, 10, -5$.\\
Wielomian jest stopnia nieparzystego, ponadto znak współczynnika przy\linebreak najwyższej potędze x jest ujemny.\\ W związku z tym wykres wielomianu zaczyna się od lewej strony powyżej osi OX. A więc $$x \in (-\infty,-11) \cup (-5,10).$$
\rozwStop
\odpStart
$x \in (-\infty,-11) \cup (-5,10)$
\odpStop
\testStart
A.$x \in (-\infty,-11) \cup (-5,10)$\\
B.$x \in (-\infty,-11) \cup (-5,10]$\\
C.$x \in (-\infty,-11) \cup [-5,10)$\\
D.$x \in (-\infty,-11] \cup (-5,10)$\\
E.$x \in (-\infty,-11] \cup (-5,10]$\\
F.$x \in (-\infty,-11] \cup [-5,10)$\\
G.$x \in (-\infty,-11) \cup [-5,10]$\\
H.$x \in (-\infty,-11] \cup [-5,10]$
\testStop
\kluczStart
A
\kluczStop



\zadStart{Zadanie z Wikieł Z 1.62 b) moja wersja nr 282}

Rozwiązać nierówności $(x+12)(10-x)(x+5)\ge0$.
\zadStop
\rozwStart{Patryk Wirkus}{Laura Mieczkowska}
Miejsca zerowe naszego wielomianu to: $-12, 10, -5$.\\
Wielomian jest stopnia nieparzystego, ponadto znak współczynnika przy\linebreak najwyższej potędze x jest ujemny.\\ W związku z tym wykres wielomianu zaczyna się od lewej strony powyżej osi OX. A więc $$x \in (-\infty,-12) \cup (-5,10).$$
\rozwStop
\odpStart
$x \in (-\infty,-12) \cup (-5,10)$
\odpStop
\testStart
A.$x \in (-\infty,-12) \cup (-5,10)$\\
B.$x \in (-\infty,-12) \cup (-5,10]$\\
C.$x \in (-\infty,-12) \cup [-5,10)$\\
D.$x \in (-\infty,-12] \cup (-5,10)$\\
E.$x \in (-\infty,-12] \cup (-5,10]$\\
F.$x \in (-\infty,-12] \cup [-5,10)$\\
G.$x \in (-\infty,-12) \cup [-5,10]$\\
H.$x \in (-\infty,-12] \cup [-5,10]$
\testStop
\kluczStart
A
\kluczStop



\zadStart{Zadanie z Wikieł Z 1.62 b) moja wersja nr 283}

Rozwiązać nierówności $(x+13)(10-x)(x+5)\ge0$.
\zadStop
\rozwStart{Patryk Wirkus}{Laura Mieczkowska}
Miejsca zerowe naszego wielomianu to: $-13, 10, -5$.\\
Wielomian jest stopnia nieparzystego, ponadto znak współczynnika przy\linebreak najwyższej potędze x jest ujemny.\\ W związku z tym wykres wielomianu zaczyna się od lewej strony powyżej osi OX. A więc $$x \in (-\infty,-13) \cup (-5,10).$$
\rozwStop
\odpStart
$x \in (-\infty,-13) \cup (-5,10)$
\odpStop
\testStart
A.$x \in (-\infty,-13) \cup (-5,10)$\\
B.$x \in (-\infty,-13) \cup (-5,10]$\\
C.$x \in (-\infty,-13) \cup [-5,10)$\\
D.$x \in (-\infty,-13] \cup (-5,10)$\\
E.$x \in (-\infty,-13] \cup (-5,10]$\\
F.$x \in (-\infty,-13] \cup [-5,10)$\\
G.$x \in (-\infty,-13) \cup [-5,10]$\\
H.$x \in (-\infty,-13] \cup [-5,10]$
\testStop
\kluczStart
A
\kluczStop



\zadStart{Zadanie z Wikieł Z 1.62 b) moja wersja nr 284}

Rozwiązać nierówności $(x+14)(10-x)(x+5)\ge0$.
\zadStop
\rozwStart{Patryk Wirkus}{Laura Mieczkowska}
Miejsca zerowe naszego wielomianu to: $-14, 10, -5$.\\
Wielomian jest stopnia nieparzystego, ponadto znak współczynnika przy\linebreak najwyższej potędze x jest ujemny.\\ W związku z tym wykres wielomianu zaczyna się od lewej strony powyżej osi OX. A więc $$x \in (-\infty,-14) \cup (-5,10).$$
\rozwStop
\odpStart
$x \in (-\infty,-14) \cup (-5,10)$
\odpStop
\testStart
A.$x \in (-\infty,-14) \cup (-5,10)$\\
B.$x \in (-\infty,-14) \cup (-5,10]$\\
C.$x \in (-\infty,-14) \cup [-5,10)$\\
D.$x \in (-\infty,-14] \cup (-5,10)$\\
E.$x \in (-\infty,-14] \cup (-5,10]$\\
F.$x \in (-\infty,-14] \cup [-5,10)$\\
G.$x \in (-\infty,-14) \cup [-5,10]$\\
H.$x \in (-\infty,-14] \cup [-5,10]$
\testStop
\kluczStart
A
\kluczStop



\zadStart{Zadanie z Wikieł Z 1.62 b) moja wersja nr 285}

Rozwiązać nierówności $(x+15)(10-x)(x+5)\ge0$.
\zadStop
\rozwStart{Patryk Wirkus}{Laura Mieczkowska}
Miejsca zerowe naszego wielomianu to: $-15, 10, -5$.\\
Wielomian jest stopnia nieparzystego, ponadto znak współczynnika przy\linebreak najwyższej potędze x jest ujemny.\\ W związku z tym wykres wielomianu zaczyna się od lewej strony powyżej osi OX. A więc $$x \in (-\infty,-15) \cup (-5,10).$$
\rozwStop
\odpStart
$x \in (-\infty,-15) \cup (-5,10)$
\odpStop
\testStart
A.$x \in (-\infty,-15) \cup (-5,10)$\\
B.$x \in (-\infty,-15) \cup (-5,10]$\\
C.$x \in (-\infty,-15) \cup [-5,10)$\\
D.$x \in (-\infty,-15] \cup (-5,10)$\\
E.$x \in (-\infty,-15] \cup (-5,10]$\\
F.$x \in (-\infty,-15] \cup [-5,10)$\\
G.$x \in (-\infty,-15) \cup [-5,10]$\\
H.$x \in (-\infty,-15] \cup [-5,10]$
\testStop
\kluczStart
A
\kluczStop





\end{document}
