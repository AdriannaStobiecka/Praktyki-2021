\documentclass[12pt, a4paper]{article}
\usepackage[utf8]{inputenc}
\usepackage{polski}

\usepackage{amsthm}  %pakiet do tworzenia twierdzeń itp.
\usepackage{amsmath} %pakiet do niektórych symboli matematycznych
\usepackage{amssymb} %pakiet do symboli mat., np. \nsubseteq
\usepackage{amsfonts}
\usepackage{graphicx} %obsługa plików graficznych z rozszerzeniem png, jpg
\theoremstyle{definition} %styl dla definicji
\newtheorem{zad}{} 
\title{Multizestaw zadań}
\author{Robert Fidytek}
%\date{\today}
\date{}
\newcounter{liczniksekcji}
\newcommand{\kategoria}[1]{\section{#1}} %olreślamy nazwę kateforii zadań
\newcommand{\zadStart}[1]{\begin{zad}#1\newline} %oznaczenie początku zadania
\newcommand{\zadStop}{\end{zad}}   %oznaczenie końca zadania
%Makra opcjonarne (nie muszą występować):
\newcommand{\rozwStart}[2]{\noindent \textbf{Rozwiązanie (autor #1 , recenzent #2): }\newline} %oznaczenie początku rozwiązania, opcjonarnie można wprowadzić informację o autorze rozwiązania zadania i recenzencie poprawności wykonania rozwiązania zadania
\newcommand{\rozwStop}{\newline}                                            %oznaczenie końca rozwiązania
\newcommand{\odpStart}{\noindent \textbf{Odpowiedź:}\newline}    %oznaczenie początku odpowiedzi końcowej (wypisanie wyniku)
\newcommand{\odpStop}{\newline}                                             %oznaczenie końca odpowiedzi końcowej (wypisanie wyniku)
\newcommand{\testStart}{\noindent \textbf{Test:}\newline} %ewentualne możliwe opcje odpowiedzi testowej: A. ? B. ? C. ? D. ? itd.
\newcommand{\testStop}{\newline} %koniec wprowadzania odpowiedzi testowych
\newcommand{\kluczStart}{\noindent \textbf{Test poprawna odpowiedź:}\newline} %klucz, poprawna odpowiedź pytania testowego (jedna literka): A lub B lub C lub D itd.
\newcommand{\kluczStop}{\newline} %koniec poprawnej odpowiedzi pytania testowego 
\newcommand{\wstawGrafike}[2]{\begin{figure}[h] \centering \includegraphics[scale=#2] {#1} \end{figure}} %gdyby była potrzeba wstawienia obrazka, parametry: nazwa pliku, skala (jak nie wiesz co wpisać, to wpisz 1)

\begin{document}
\maketitle

\kategoria{Wikieł/Z5.43c}

\zadStart{Zadanie z Wikieł Z 5.43 c) moja wersja nr [nrWersji]}
%[a]:[2,3,4,5,6,7,8,9,10,11]
%[b]:[2,3,4,5,6,7,8,9,10,11]
%[c]=[b]*[a]
%math.gcd([b],3)==1
Obliczyć i przedstawić w najprostszej postaci pochodną funkcji f.
$$f(x) = [a]\ctg^3\Big(\frac{[b]x}{3}\Big)$$
\zadStop

\rozwStart{Natalia Danieluk}{}
Dziedzina: $\quad \mathcal{D}_f=\mathbb{R}\backslash\{k\pi : k \in \mathbb{Z} \}$
$$f'(x) = [a]\Bigg(\ctg^3\Big(\frac{[b]x}{3}\Big)\Bigg)' \cdot \Bigg(\ctg\Big(\frac{[b]x}{3}\Big)\Bigg)' \cdot \Big(\frac{[b]x}{3}\Big)' 
\mathrel{\stackrel{\makebox[0pt]{\mbox{\normalfont\scriptsize\textbf{(*)}}}}{=}}$$
$$\quad= [a] \cdot 3 \ctg^2\Big(\frac{[b]x}{3}\Big) \cdot \Bigg(\boldsymbol{-\frac{1}{\sin^2\big(\frac{[b]x}{3}\big)}}\Bigg) \cdot \frac{[b]}{3} =$$
$$= -\frac{[c]\ctg^2\big(\frac{[b]x}{3}\big)}{\sin^2\big(\frac{[b]x}{3}\big)}$$
{\normalfont\scriptsize\textbf{(*)}\\
$(\ctg x)' = \big(\frac{\cos x}{\sin x} \big)' = \frac{(\cos x)'\sin x - (\sin x)'\cos x}{\sin^2 x} = \frac{-\sin^2 x - \cos^2 x}{\sin^2 x} = -\frac{1}{\sin^2 x}$}
\rozwStop

\odpStart
$f'(x) = -\frac{[c]\ctg^2\big(\frac{[b]x}{3}\big)}{\sin^2\big(\frac{[b]x}{3}\big)}$
\odpStop

\testStart
A. $f'(x) = [a]\ctg^3\Big(\frac{[b]x}{3}\Big)$
B. $f'(x) = [a] \cdot 3 \ctg^2\Big(\frac{[b]x}{3}\Big) \cdot \Bigg(-\frac{1}{\sin^2\big(\frac{[b]x}{3}\big)}\Bigg) \cdot \frac{[b]}{3}$\\
C. $f'(x) = -\frac{[a]\ctg^2\big(\frac{[b]x}{3}\big)}{\sin^2\big(\frac{[b]x}{3}\big)}$
D. $f'(x) = -\frac{[c]\ctg^2\big(\frac{[b]x}{3}\big)}{\sin^2\big(\frac{[b]x}{3}\big)}$
\testStop

\kluczStart
D
\kluczStop

\end{document}
