\documentclass[12pt, a4paper]{article}
\usepackage[utf8]{inputenc}
\usepackage{polski}

\usepackage{amsthm}  %pakiet do tworzenia twierdzeń itp.
\usepackage{amsmath} %pakiet do niektórych symboli matematycznych
\usepackage{amssymb} %pakiet do symboli mat., np. \nsubseteq
\usepackage{amsfonts}
\usepackage{graphicx} %obsługa plików graficznych z rozszerzeniem png, jpg
\theoremstyle{definition} %styl dla definicji
\newtheorem{zad}{} 
\title{Multizestaw zadań}
\author{Robert Fidytek}
%\date{\today}
\date{}
\newcounter{liczniksekcji}
\newcommand{\kategoria}[1]{\section{#1}} %olreślamy nazwę kateforii zadań
\newcommand{\zadStart}[1]{\begin{zad}#1\newline} %oznaczenie początku zadania
\newcommand{\zadStop}{\end{zad}}   %oznaczenie końca zadania
%Makra opcjonarne (nie muszą występować):
\newcommand{\rozwStart}[2]{\noindent \textbf{Rozwiązanie (autor #1 , recenzent #2): }\newline} %oznaczenie początku rozwiązania, opcjonarnie można wprowadzić informację o autorze rozwiązania zadania i recenzencie poprawności wykonania rozwiązania zadania
\newcommand{\rozwStop}{\newline}                                            %oznaczenie końca rozwiązania
\newcommand{\odpStart}{\noindent \textbf{Odpowiedź:}\newline}    %oznaczenie początku odpowiedzi końcowej (wypisanie wyniku)
\newcommand{\odpStop}{\newline}                                             %oznaczenie końca odpowiedzi końcowej (wypisanie wyniku)
\newcommand{\testStart}{\noindent \textbf{Test:}\newline} %ewentualne możliwe opcje odpowiedzi testowej: A. ? B. ? C. ? D. ? itd.
\newcommand{\testStop}{\newline} %koniec wprowadzania odpowiedzi testowych
\newcommand{\kluczStart}{\noindent \textbf{Test poprawna odpowiedź:}\newline} %klucz, poprawna odpowiedź pytania testowego (jedna literka): A lub B lub C lub D itd.
\newcommand{\kluczStop}{\newline} %koniec poprawnej odpowiedzi pytania testowego 
\newcommand{\wstawGrafike}[2]{\begin{figure}[h] \includegraphics[scale=#2] {#1} \end{figure}} %gdyby była potrzeba wstawienia obrazka, parametry: nazwa pliku, skala (jak nie wiesz co wpisać, to wpisz 1)

\begin{document}
\maketitle


\kategoria{Wikieł/Z1.31b}
\zadStart{Zadanie z Wikieł Z 1.31 b) moja wersja nr [nrWersji]}
%[c]:[2,3,4,5,6,7,8,9,10,11,12,15,17]
%[b]:[2,3,4,5,6,7,8,9,10,11,12,15,17]
%[a]=random.randint(2,100)
%[m]=random.randint(2,100)
%math.gcd([a],[m])==1
Wyznaczyć funkcję odwrotną $f^{-1}$ dla funkcji $f$ określonej poniżej. \\
$f(x)=[m]x^2-[a]$, $x\geq 0$
\zadStop
\rozwStart{Katarzyna Filipowicz}{}
$$
y=[m]x^2-[a] 
$$ $$
y+[a]=[m]x^2
$$ $$
\frac{y+[a]}{[m]}=x^2
$$ $$
x=\sqrt{\frac{y+[a]}{[m]}}
$$ $$
f(x)^{-1}=\sqrt{\frac{x+[a]}{[m]}}
$$ 
Dziedzina:
$$
0\leq \frac{x+[a]}{[m]} \quad \Rightarrow   \quad  x \geq -[a] \quad \Rightarrow \quad x\in [-[a],\infty)
$$
\rozwStop
\odpStart
$f(x)^{-1}=\sqrt{\frac{x+[a]}{[m]}}$, $x\in [-[a],\infty)$
\odpStop
\testStart
A.$f(x)^{-1}=\sqrt{x+[a]}$, $x\in [-[a],\infty)$
B.$0$
C.$f(x)^{-1}=\sqrt{x+[a]}$, $x \in R$
D.$f(x)^{-1}=x$
E.$f(x)^{-1}=\sqrt{x-[a]}$, $x \in  [[a],\infty)$
F.$f(x)^{-1}=y+[a]$
G.$f(x)^{-1}=-[a]x$
H.$f(x)^{-1}=[b]x$
I.$f(x)^{-1}=\infty$
\testStop
\kluczStart
A
\kluczStop



\end{document}