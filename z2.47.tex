\documentclass[12pt, a4paper]{article}
\usepackage[utf8]{inputenc}
\usepackage{polski}

\usepackage{amsthm}  %pakiet do tworzenia twierdzeń itp.
\usepackage{amsmath} %pakiet do niektórych symboli matematycznych
\usepackage{amssymb} %pakiet do symboli mat., np. \nsubseteq
\usepackage{amsfonts}
\usepackage{graphicx} %obsługa plików graficznych z rozszerzeniem png, jpg
\theoremstyle{definition} %styl dla definicji
\newtheorem{zad}{} 
\title{Multizestaw zadań}
\author{Robert Fidytek}
%\date{\today}
\date{}
\newcounter{liczniksekcji}
\newcommand{\kategoria}[1]{\section{#1}} %olreślamy nazwę kateforii zadań
\newcommand{\zadStart}[1]{\begin{zad}#1\newline} %oznaczenie początku zadania
\newcommand{\zadStop}{\end{zad}}   %oznaczenie końca zadania
%Makra opcjonarne (nie muszą występować):
\newcommand{\rozwStart}[2]{\noindent \textbf{Rozwiązanie (autor #1 , recenzent #2): }\newline} %oznaczenie początku rozwiązania, opcjonarnie można wprowadzić informację o autorze rozwiązania zadania i recenzencie poprawności wykonania rozwiązania zadania
\newcommand{\rozwStop}{\newline}                                            %oznaczenie końca rozwiązania
\newcommand{\odpStart}{\noindent \textbf{Odpowiedź:}\newline}    %oznaczenie początku odpowiedzi końcowej (wypisanie wyniku)
\newcommand{\odpStop}{\newline}                                             %oznaczenie końca odpowiedzi końcowej (wypisanie wyniku)
\newcommand{\testStart}{\noindent \textbf{Test:}\newline} %ewentualne możliwe opcje odpowiedzi testowej: A. ? B. ? C. ? D. ? itd.
\newcommand{\testStop}{\newline} %koniec wprowadzania odpowiedzi testowych
\newcommand{\kluczStart}{\noindent \textbf{Test poprawna odpowiedź:}\newline} %klucz, poprawna odpowiedź pytania testowego (jedna literka): A lub B lub C lub D itd.
\newcommand{\kluczStop}{\newline} %koniec poprawnej odpowiedzi pytania testowego 
\newcommand{\wstawGrafike}[2]{\begin{figure}[h] \includegraphics[scale=#2] {#1} \end{figure}} %gdyby była potrzeba wstawienia obrazka, parametry: nazwa pliku, skala (jak nie wiesz co wpisać, to wpisz 1)

\begin{document}
\maketitle


\kategoria{Wikieł/Z2.47}
\zadStart{Zadanie z Wikieł Z 2.47 ) moja wersja nr [nrWersji]}
%[p1]:[2,3,4,5,6,10,11]
%[p2]:[2,3,4,7,8,9,10]
%[p3]:[2,3,4,5,6,7,8,9]
%[p4]:[2,3,4,5,6,7,11]
%[a]=random.randint(1,10)
%[b]=random.randint(1,10)
%[c]=random.randint(1,10)
%[d]=random.randint(1,10)
%[w]=[p1]*[p4]-[p3]*[p2]
%[wxm]=[p4]
%[wx]=[p4]*[a]-[p2]*[b]
%[wym]=-[p3]
%[wy]=[b]*[p1]-[p3]*[a]
%[abswx]=abs([wx])
%[absw]=abs([w])
%[abswym]=abs([wym])
%[mx]=round(([c]-([wx]/([w]+0.00001)))/([wxm]/([w]+0.00001)),2)
%[my]=round(([d]-([wy]/([w]+0.00001)))/([wym]/([w]+0.00001)),2)
%[pierwmx]=round(math.sqrt(abs([mx])),2)
%[pierwmy]=round(math.sqrt(abs([my])),2)
%([p1]*[p4]-[p3]*[p2])<0 and ([p4]*[a]-[p2]*[b])<0 and ([b]*[p1]-[p3]*[a])>0 and (([p4]*[a]-[p2]*[b])/([p1]*[p4]-[p3]*[p2]))>[c] and [my]>[mx] and ([p1]*[p4]-[p3]*[p2])!=-1
Podać, dla jakich wartości parametru $m$ rozwiązaniem układu równań:
$$
 \left\{ \begin{array}{ll}
[p1]\cdot x+[p2]\cdot y=m^{2}+[a] & \\
{[p3]}\cdot x+[p4]\cdot y=[b]  & 
\end{array} \right.
$$
jest para liczb spełniających warunek $x\geq [c]$ i $y\leq[d]$.
\zadStop
\rozwStart{Wojciech Przybylski}{}
$$
W =
\left| \begin{array}{ccc}
[p1] & [p2]  \\
{[p3]} & [p4]  \\
\end{array} \right| =[p1]\cdot[p4]-[p3]\cdot[p2]=[w]
$$
$$
W_{x} =
\left| \begin{array}{ccc}
m^{2}+[a] & [p2]  \\
{[b]} & [p4]  \\
\end{array} \right| =(m^{2}+[a])\cdot[p4]-[b]\cdot[p2]=[wxm]m^{2}-[abswx]
$$
$$
W_{y} =
\left| \begin{array}{ccc}
[p1] & m^{2}+[a] \\
{[p3]} & [b]  \\
\end{array} \right| =[p1]\cdot[b]-[p3]\cdot(m^{2}+[a])=[wym]m^{2}+[wy]
$$
W $\neq 0$, więc układ ma dokładnie jedno rozwiązanie
$$x=\frac{W_{x}}{W}=\frac{[wxm]m-[abswx]}{[w]},$$
$$y=\frac{W_{y}}{W}=\frac{[wym]m+[wy]}{[w]}$$
Szukamy pary liczb  spełniających warunek $x\geq [c]$ i $y\leq[d]$.\\
I $x\geq [c]$
$$\frac{[wxm]}{[w]}m^{2}+\frac{[abswx]}{[absw]}\geq[c]$$
$$m^{2}\leq[mx]$$
$$-[pierwmx]\leq m\leq[pierwmx]$$
II $y\leq[d]$
$$\frac{[abswym]}{[absw]}m^2-\frac{[wy]}{[absw]}\leq[d]$$
$$m^{2}\leq[my]$$
$$-[pierwmy]\leq m\leq[pierwmy]$$
$$\mbox{Ostatecznie }m\in(-[pierwmx],[pierwmx])$$
\rozwStop
\odpStart
$m\in(-[pierwmx],[pierwmx])$
\odpStop
\testStart
A. $m\in(-[pierwmx],[pierwmx])$\\
B. $m\in([pierwmx],[pierwmy])$\\
C. $m\in(-[pierwmx],[pierwmy])$\\
D. $m\in([pierwmy],[pierwmy])$\\
E. nie ma takiego m 
\testStop
\kluczStart
A
\kluczStop



\end{document}