\documentclass[12pt, a4paper]{article}
\usepackage[utf8]{inputenc}
\usepackage{polski}

\usepackage{amsthm}  %pakiet do tworzenia twierdzeń itp.
\usepackage{amsmath} %pakiet do niektórych symboli matematycznych
\usepackage{amssymb} %pakiet do symboli mat., np. \nsubseteq
\usepackage{amsfonts}
\usepackage{graphicx} %obsługa plików graficznych z rozszerzeniem png, jpg
\theoremstyle{definition} %styl dla definicji
\newtheorem{zad}{} 
\title{Multizestaw zadań}
\author{Robert Fidytek}
%\date{\today}
\date{}\documentclass[12pt, a4paper]{article}
\usepackage[utf8]{inputenc}
\usepackage{polski}

\usepackage{amsthm}  %pakiet do tworzenia twierdzeń itp.
\usepackage{amsmath} %pakiet do niektórych symboli matematycznych
\usepackage{amssymb} %pakiet do symboli mat., np. \nsubseteq
\usepackage{amsfonts}
\usepackage{graphicx} %obsługa plików graficznych z rozszerzeniem png, jpg
\theoremstyle{definition} %styl dla definicji
\newtheorem{zad}{} 
\title{Multizestaw zadań}
\author{Robert Fidytek}
%\date{\today}
\date{}
\newcounter{liczniksekcji}
\newcommand{\kategoria}[1]{\section{#1}} %olreślamy nazwę kateforii zadań
\newcommand{\zadStart}[1]{\begin{zad}#1\newline} %oznaczenie początku zadania
\newcommand{\zadStop}{\end{zad}}   %oznaczenie końca zadania
%Makra opcjonarne (nie muszą występować):
\newcommand{\rozwStart}[2]{\noindent \textbf{Rozwiązanie (autor #1 , recenzent #2): }\newline} %oznaczenie początku rozwiązania, opcjonarnie można wprowadzić informację o autorze rozwiązania zadania i recenzencie poprawności wykonania rozwiązania zadania
\newcommand{\rozwStop}{\newline}                                            %oznaczenie końca rozwiązania
\newcommand{\odpStart}{\noindent \textbf{Odpowiedź:}\newline}    %oznaczenie początku odpowiedzi końcowej (wypisanie wyniku)
\newcommand{\odpStop}{\newline}                                             %oznaczenie końca odpowiedzi końcowej (wypisanie wyniku)
\newcommand{\testStart}{\noindent \textbf{Test:}\newline} %ewentualne możliwe opcje odpowiedzi testowej: A. ? B. ? C. ? D. ? itd.
\newcommand{\testStop}{\newline} %koniec wprowadzania odpowiedzi testowych
\newcommand{\kluczStart}{\noindent \textbf{Test poprawna odpowiedź:}\newline} %klucz, poprawna odpowiedź pytania testowego (jedna literka): A lub B lub C lub D itd.
\newcommand{\kluczStop}{\newline} %koniec poprawnej odpowiedzi pytania testowego 
\newcommand{\wstawGrafike}[2]{\begin{figure}[h] \includegraphics[scale=#2] {#1} \end{figure}} %gdyby była potrzeba wstawienia obrazka, parametry: nazwa pliku, skala (jak nie wiesz co wpisać, to wpisz 1)

\begin{document}
\maketitle


\kategoria{Wikieł/Z1.64b}
\zadStart{Zadanie z Wikieł Z 1.64b moja wersja nr [nrWersji]}
%[p1]:[2,3,4,5,6,7,8,9]
%[p2]:[2,3,4,5,6,7,8,9]
%[del]=1+4*[p2]*[p1]
%[pdel]=round(math.sqrt(abs([del])),2)
%[del2]=1-4*[p2]*[p1]
%[2p2]=2*[p2]
%[x1]=round((1-[pdel])/[2p2],2)
%[x2]=round((1+[pdel])/[2p2],2)

Rozwiązać podwójną nierówność $-[p1]\leq x-[p2]x^{2} < [p1].$
\zadStop

\rozwStart{Maja Szabłowska}{}
$$-[p1]\leq x-[p2]x^{2} < [p1]$$
$$-[p1]-x+[p2]x^{2} \leq 0 \quad \land \quad x-[p2]x^{2}-[p1]<0$$
\begin{enumerate}
    \item $$-[p1]-x+[p2]x^{2} \leq 0$$
    $$\Delta=(-1)^{2}-4\cdot[p2]\cdot(-[p1])=[del] \Rightarrow \sqrt{\Delta}=[pdel]$$
    $$x_{1}=\frac{1-[pdel]}{[2p2]}=[x1], \quad x_{1}=\frac{1+[pdel]}{[2p2]}=[x2]$$
    $$x\in[[x1],[x2]]$$
    
    \item $$x-[p2]x^{2}-[p1]<0$$
    $$\Delta=1^{2}-4\cdot(-[p2])\cdot(-[p1])=[del2] < 0$$
    $$x\in\mathbb{R}$$
\end{enumerate}
Zatem ostatecznie $x\in[[x1],[x2]].$
\rozwStop


\odpStart
$x\in[[x1],[x2]]$
\odpStop
\testStart
A.$x\in[[x1],[x2]]$
B.$x\in([p1],\infty)$
C.$x\in(-\infty,[x2]]$
D.$x\in(-\infty,[x2]]\cup ([p1],\infty)$
E.$x\in[[del], \infty)$
F.$x\in\emptyset$


\testStop
\kluczStart
A
\kluczStop



\end{document}
