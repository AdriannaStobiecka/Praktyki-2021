\documentclass[12pt, a4paper]{article}
\usepackage[utf8]{inputenc}
\usepackage{polski}

\usepackage{amsthm}  %pakiet do tworzenia twierdzeń itp.
\usepackage{amsmath} %pakiet do niektórych symboli matematycznych
\usepackage{amssymb} %pakiet do symboli mat., np. \nsubseteq
\usepackage{amsfonts}
\usepackage{graphicx} %obsługa plików graficznych z rozszerzeniem png, jpg
\theoremstyle{definition} %styl dla definicji
\newtheorem{zad}{} 
\title{Multizestaw zadań}
\author{Robert Fidytek}
%\date{\today}
\date{}
\newcounter{liczniksekcji}
\newcommand{\kategoria}[1]{\section{#1}} %olreślamy nazwę kateforii zadań
\newcommand{\zadStart}[1]{\begin{zad}#1\newline} %oznaczenie początku zadania
\newcommand{\zadStop}{\end{zad}}   %oznaczenie końca zadania
%Makra opcjonarne (nie muszą występować):
\newcommand{\rozwStart}[2]{\noindent \textbf{Rozwiązanie (autor #1 , recenzent #2): }\newline} %oznaczenie początku rozwiązania, opcjonarnie można wprowadzić informację o autorze rozwiązania zadania i recenzencie poprawności wykonania rozwiązania zadania
\newcommand{\rozwStop}{\newline}                                            %oznaczenie końca rozwiązania
\newcommand{\odpStart}{\noindent \textbf{Odpowiedź:}\newline}    %oznaczenie początku odpowiedzi końcowej (wypisanie wyniku)
\newcommand{\odpStop}{\newline}                                             %oznaczenie końca odpowiedzi końcowej (wypisanie wyniku)
\newcommand{\testStart}{\noindent \textbf{Test:}\newline} %ewentualne możliwe opcje odpowiedzi testowej: A. ? B. ? C. ? D. ? itd.
\newcommand{\testStop}{\newline} %koniec wprowadzania odpowiedzi testowych
\newcommand{\kluczStart}{\noindent \textbf{Test poprawna odpowiedź:}\newline} %klucz, poprawna odpowiedź pytania testowego (jedna literka): A lub B lub C lub D itd.
\newcommand{\kluczStop}{\newline} %koniec poprawnej odpowiedzi pytania testowego 
\newcommand{\wstawGrafike}[2]{\begin{figure}[h] \includegraphics[scale=#2] {#1} \end{figure}} %gdyby była potrzeba wstawienia obrazka, parametry: nazwa pliku, skala (jak nie wiesz co wpisać, to wpisz 1)

\begin{document}
\maketitle


\kategoria{Wikieł/Z2.38}
\zadStart{Zadanie z Wikieł Z 2.38  moja wersja nr [nrWersji]}
%[p1]=random.randint(-10,-2)
%[p2]:[2,3,4,5,6,7,8,9,10]
%[p3]:[2,3,4,5,6,7,8,9,10]
%[p4]:[2,3,4,5,6,7,8,9,10]
%[p5]=random.randint(2,10)
%[p24]=[p2]*[p4]
%[p52]=[p5]*[p2]
%[p23]=[p2]*[p3]
%[p15]=[p1]*[p5]
%[m]=round([p24]/[p1],2)
%[m1]=round([p23]/[p15],2)
%[wx]=[p3]-[p52]
%math.gcd([p24],[p1])==1 and math.gcd([p23],[p15])==1 and [m]<[m1]


Podać wartości parametru $m$, dla których rozwiązaniem układu
$$\left\{\begin{array}{rcl}
[p1]x+[p2]y&=&[p3]\\
[p4]x+my&=&[p5]m
\end{array} \right.$$
jest para liczb dodatnich.
\zadStop
\rozwStart{Maja Szabłowska}{}
$$W=\left| \begin{array}{lccr} [p1] & [p2]  \\ [p4] & m \end{array}\right| = [p1]\cdot m - [p2]\cdot [p4]=[p1]m-[p24] \neq 0$$

$$[p1]m-[p24] \neq 0 \Rightarrow [p1]m\neq [p24] \Rightarrow m\neq [m]$$

$$W_{x}=\left| \begin{array}{lccr} [p3] & [p2] \\ [p5]m & m \end{array}\right| = [p3]\cdot m - [p5]m\cdot [p2]=[p3]m-[p52]m=[wx]m$$

$$W_{y}=\left| \begin{array}{lccr} [p1] & [p3]  \\ [p4] & [p5]m \end{array}\right| = [p1]\cdot[p5]m - [p2]\cdot[p3]=[p15]m-[p23]$$

$$x=\frac{W_{x}}{W}=\frac{[wx]m}{[p1]m-[p24]}>0$$
$$([wx]m)([p1]m-[p24])>0$$ 
$$m_{1}=0 \land m_{2}=\frac{[p24]}{[p1]}=[m] $$
$$m\in(-\infty,[m])\cup(0,\infty)$$

$$y=\frac{W_{y}}{W}=\frac{[p15]m-[p23]}{[p1]m-[p24]}>0$$
$$([p15]m-[p23])([p1]m-[p24])>0$$ 
$$m_{1}=\frac{[p23]}{[p15]}=[m1] \land m_{2}=\frac{[p24]}{[p1]}=[m] $$
$$m\in(-\infty,[m])\cup([m1],\infty)$$
Zatem $m\in(-\infty,[m])\cup(0,\infty).$
\rozwStop
\odpStart
$m\in(-\infty,[m])\cup(0,\infty)$
\odpStop
\testStart
A.$m\in(-\infty,[m])\cup(0,\infty)$
B.$m\in(-\infty,[m])\cup([m1],\infty)$
C.$m\in(-\infty,[m])\cup([p23],\infty)$
D.$m\in(-\infty,[m1])\cup(0,\infty)$
E.$m\in(-\infty,[m])\cup([p15],\infty)$
F.$m\in(-\infty,[p23])\cup(0,\infty)$
G.$m\in(-\infty,[m])$
H.$m\in(0,\infty)$
I.$m=2$
\testStop
\kluczStart
A
\kluczStop



\end{document}