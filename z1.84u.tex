\documentclass[12pt, a4paper]{article}
\usepackage[utf8]{inputenc}
\usepackage{polski}

\usepackage{amsthm}  %pakiet do tworzenia twierdzeń itp.
\usepackage{amsmath} %pakiet do niektórych symboli matematycznych
\usepackage{amssymb} %pakiet do symboli mat., np. \nsubseteq
\usepackage{amsfonts}
\usepackage{graphicx} %obsługa plików graficznych z rozszerzeniem png, jpg
\theoremstyle{definition} %styl dla definicji
\newtheorem{zad}{} 
\title{Multizestaw zadań}
\author{Robert Fidytek}
%\date{\today}
\date{}
\newcounter{liczniksekcji}
\newcommand{\kategoria}[1]{\section{#1}} %olreślamy nazwę kateforii zadań
\newcommand{\zadStart}[1]{\begin{zad}#1\newline} %oznaczenie początku zadania
\newcommand{\zadStop}{\end{zad}}   %oznaczenie końca zadania
%Makra opcjonarne (nie muszą występować):
\newcommand{\rozwStart}[2]{\noindent \textbf{Rozwiązanie (autor #1 , recenzent #2): }\newline} %oznaczenie początku rozwiązania, opcjonarnie można wprowadzić informację o autorze rozwiązania zadania i recenzencie poprawności wykonania rozwiązania zadania
\newcommand{\rozwStop}{\newline}                                            %oznaczenie końca rozwiązania
\newcommand{\odpStart}{\noindent \textbf{Odpowiedź:}\newline}    %oznaczenie początku odpowiedzi końcowej (wypisanie wyniku)
\newcommand{\odpStop}{\newline}                                             %oznaczenie końca odpowiedzi końcowej (wypisanie wyniku)
\newcommand{\testStart}{\noindent \textbf{Test:}\newline} %ewentualne możliwe opcje odpowiedzi testowej: A. ? B. ? C. ? D. ? itd.
\newcommand{\testStop}{\newline} %koniec wprowadzania odpowiedzi testowych
\newcommand{\kluczStart}{\noindent \textbf{Test poprawna odpowiedź:}\newline} %klucz, poprawna odpowiedź pytania testowego (jedna literka): A lub B lub C lub D itd.
\newcommand{\kluczStop}{\newline} %koniec poprawnej odpowiedzi pytania testowego 
\newcommand{\wstawGrafike}[2]{\begin{figure}[h] \includegraphics[scale=#2] {#1} \end{figure}} %gdyby była potrzeba wstawienia obrazka, parametry: nazwa pliku, skala (jak nie wiesz co wpisać, to wpisz 1)

\begin{document}
\maketitle


\kategoria{Wikieł/Z1.84u}
\zadStart{Zadanie z Wikieł Z 1.84 u) moja wersja nr [nrWersji]}
%[a]:[2,3,4,5,6,7,8,9]
%[e]:[2,3,4,5,6,7,8,9]
%[z]:[2,3]
%[g]:[2,3,4]
%[c]=[g]*[a]+1
%[f]=pow([a],[z])
%[b]=pow([e],[z])
%[d]=[g]*[e]-1
%[d2]=[g]*[e]
%[c2]=[g]*[a]
%[a]!=[e] and math.gcd([f],[b])==1
Rozwiązać równanie $[c] \cdot {[a]}^{[z]x}- {[b]}^x= [d] \cdot {[e]}^{[z]x}+ {[f]}^x$.
\zadStop
\rozwStart{Barbara Bączek}{}
$$[c] \cdot {[a]}^{[z]x}- {[b]}^x= [d] \cdot {[e]}^{[z]x}+ {[f]}^x$$
$$[c] \cdot {[f]}^x- {[b]}^x= [d] \cdot {[b]}^x+ {[f]}^x$$
$${[f]}^x([c]-1)= {[b]}^x([d]+1)$$
$$[c2] \cdot {[f]}^x= [d2] \cdot {[b]}^x$$
$$[a] \cdot  {[f]}^x= [e] \cdot {[b]}^x $$
$$\frac{[a]}{[e]}= \frac{{[b]}^x}{{[f]}^x}$$
$${\Bigg{(}\frac{[e]}{[a]} \Bigg{)}}^{-1}={\Bigg{(} \frac{[b]}{[f]} \Bigg{)}}^x$$
$${\Bigg{(}\frac{[b]}{[f]} \Bigg{)}}^{-\frac{1}{[z]}}={\Bigg{(} \frac{[b]}{[f]} \Bigg{)}}^x$$
$$x=-\frac{1}{[z]}$$
\rozwStop
\odpStart
$-\frac{1}{[z]}$
\odpStop
\testStart
A.$-[e]$
B.$-[a]$
C.$\frac{1}{[z]}$
D.$0$
E.$[a]$
G.$-\frac{1}{[z]}$
H.$[e]$
\testStop
\kluczStart
G
\kluczStop



\end{document}