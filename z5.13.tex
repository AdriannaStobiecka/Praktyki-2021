\documentclass[12pt, a4paper]{article}
\usepackage[utf8]{inputenc}
\usepackage{polski}

\usepackage{amsthm}  %pakiet do tworzenia twierdzeń itp.
\usepackage{amsmath} %pakiet do niektórych symboli matematycznych
\usepackage{amssymb} %pakiet do symboli mat., np. \nsubseteq
\usepackage{amsfonts}
\usepackage{graphicx} %obsługa plików graficznych z rozszerzeniem png, jpg
\theoremstyle{definition} %styl dla definicji
\newtheorem{zad}{} 
\title{Multizestaw zadań}
\author{Robert Fidytek}
%\date{\today}
\date{}
\newcounter{liczniksekcji}
\newcommand{\kategoria}[1]{\section{#1}} %olreślamy nazwę kateforii zadań
\newcommand{\zadStart}[1]{\begin{zad}#1\newline} %oznaczenie początku zadania
\newcommand{\zadStop}{\end{zad}}   %oznaczenie końca zadania
%Makra opcjonarne (nie muszą występować):
\newcommand{\rozwStart}[2]{\noindent \textbf{Rozwiązanie (autor #1 , recenzent #2): }\newline} %oznaczenie początku rozwiązania, opcjonarnie można wprowadzić informację o autorze rozwiązania zadania i recenzencie poprawności wykonania rozwiązania zadania
\newcommand{\rozwStop}{\newline}                                            %oznaczenie końca rozwiązania
\newcommand{\odpStart}{\noindent \textbf{Odpowiedź:}\newline}    %oznaczenie początku odpowiedzi końcowej (wypisanie wyniku)
\newcommand{\odpStop}{\newline}                                             %oznaczenie końca odpowiedzi końcowej (wypisanie wyniku)
\newcommand{\testStart}{\noindent \textbf{Test:}\newline} %ewentualne możliwe opcje odpowiedzi testowej: A. ? B. ? C. ? D. ? itd.
\newcommand{\testStop}{\newline} %koniec wprowadzania odpowiedzi testowych
\newcommand{\kluczStart}{\noindent \textbf{Test poprawna odpowiedź:}\newline} %klucz, poprawna odpowiedź pytania testowego (jedna literka): A lub B lub C lub D itd.
\newcommand{\kluczStop}{\newline} %koniec poprawnej odpowiedzi pytania testowego 
\newcommand{\wstawGrafike}[2]{\begin{figure}[h] \includegraphics[scale=#2] {#1} \end{figure}} %gdyby była potrzeba wstawienia obrazka, parametry: nazwa pliku, skala (jak nie wiesz co wpisać, to wpisz 1)

\begin{document}
\maketitle


\kategoria{Wikieł/Z5.13}
\zadStart{Zadanie z Wikieł Z 5.13 moja wersja nr [nrWersji]}
%[a]:[2,3,4,5,6,7,8,9]
%[b]:[2,3,4,5,6,7,8,9]
%[c]=[a]*[b]
%[a]!=0
Wyznacz wzór na drugą pochodną funkcji \\ $g(x)=[a]\log(f([b]x))$.
\zadStop
\rozwStart{Joanna Świerzbin}{}
$$g(x)=[a]\log(f([b]x))$$\\
$$g'(x)= \left([a]\log(f([b]x)) \right)' = \frac{[a]}{f([b]x)\ln 10} (f([b]x))' =$$
$$= \frac{[a]\cdot[b]}{f([b]x)\ln 10} f'([b]x) $$\\
$$g''(x)= \left( \frac{[a]\cdot[b]}{f([b]x)\ln 10} f'([b]x) \right)'= \frac{[a]\cdot[b]}{\ln 10} \left(\frac{f'([b]x)}{f([b]x)} \right)' =$$
$$= \frac{[a]\cdot[b]}{\ln 10} \left(\frac{(f'([b]x))'f([b]x)-f'([b]x)(f([b]x))'}{(f([b]x))^2} \right) =$$
$$= \frac{[a]\cdot[b]}{\ln 10} \left(\frac{[b] f''([b]x) f([b]x) -[b] f'([b]x)f'([b]x)}{(f([b]x))^2} \right) =$$
$$= \frac{[c]}{\ln 10} \left(\frac{[b] f''([b]x) f([b]x) -[b] (f'([b]x))^2}{(f([b]x))^2} \right) $$
\rozwStop
\odpStart
$g''(x)= \frac{[c]}{\ln 10} \left(\frac{[b] f''([b]x) f([b]x) -[b] (f'([b]x))^2}{(f([b]x))^2} \right) $
\odpStop
\testStart
A. $g''(x)= \frac{[c]}{\ln 10} \left(\frac{[b] f''([b]x) f([b]x) -[b] (f'([b]x))^2}{(f([b]x))^2} \right) $\\
B. $g''(x)= \frac{[b]}{\ln 10} \left(\frac{[b] f''([b]x) f([b]x) -[b] (f'([b]x))^2}{(f([b]x))^2} \right) $ \\
C. $g''(x)= \frac{[a]}{\ln 10} \left(\frac{[b] f''([b]x) f([b]x) -[b] (f'([b]x))^2}{(f([b]x))^2} \right) $ \\
D. $g''(x)= \frac{[c]}{\ln 10} \left(\frac{f''([b]x) f([b]x) -(f'([b]x))^2}{(f([b]x))^2} \right) $\\
E. $g''(x)= \frac{[c]}{\ln 10} \left(\frac{[b] f''([b]x) f([b]x) -[b] f'([b]x)}{(f([b]x))^2} \right) $\\
F. $g''(x)= \frac{1}{\ln 10} \left(\frac{[b] f''([b]x) f([b]x) -[b] (f'([b]x))^2}{(f([b]x))^2} \right) $
\testStop
\kluczStart
A
\kluczStop



\end{document}