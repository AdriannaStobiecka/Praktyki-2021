\documentclass[12pt, a4paper]{article}
\usepackage[utf8]{inputenc}
\usepackage{polski}
\usepackage{amsthm}  %pakiet do tworzenia twierdzeń itp.
\usepackage{amsmath} %pakiet do niektórych symboli matematycznych
\usepackage{amssymb} %pakiet do symboli mat., np. \nsubseteq
\usepackage{amsfonts}
\usepackage{graphicx} %obsługa plików graficznych z rozszerzeniem png, jpg
\theoremstyle{definition} %styl dla definicji
\newtheorem{zad}{} 
\title{Multizestaw zadań}
\author{Patryk Wirkus}
%\date{\today}
\date{}
\newcommand{\kategoria}[1]{\section{#1}}
\newcommand{\zadStart}[1]{\begin{zad}#1\newline}
\newcommand{\zadStop}{\end{zad}}
\newcommand{\rozwStart}[2]{\noindent \textbf{Rozwiązanie (autor #1 , recenzent #2): }\newline}
\newcommand{\rozwStop}{\newline}                                           
\newcommand{\odpStart}{\noindent \textbf{Odpowiedź:}\newline}
\newcommand{\odpStop}{\newline}
\newcommand{\testStart}{\noindent \textbf{Test:}\newline}
\newcommand{\testStop}{\newline}
\newcommand{\kluczStart}{\noindent \textbf{Test poprawna odpowiedź:}\newline}
\newcommand{\kluczStop}{\newline}
\newcommand{\wstawGrafike}[2]{\begin{figure}[h] \includegraphics[scale=#2] {#1} \end{figure}}

\begin{document}
\maketitle

\kategoria{Wikieł/1.4g}


\zadStart{Zadanie z Wikieł Z 1.4 g) moja wersja nr 1}

Oblicz wartość wyrażenia $[8,25-0,5^{-0,5}(2^{-0,5}+4^{-0,25})]^{\frac{1}{3}}$.
\zadStop
\rozwStart{Patryk Wirkus}{}
$$[8,25-0,5^{-0,5}(2^{-0,5}+4^{-0,25})]^{\frac{1}{3}} = [8,25-2^{0,5}(0,5^{0,5}+0,25^{0,25})]^{\frac{1}{3}} =$$
$$=[8,25 - 1 - \sqrt{2} \cdot 0,25^{0,25}]^{\frac{1}{3}} = [8,25 - 1 - (4\cdot 0,25)^{0,25}]^{\frac{1}{3}} = 6,25^{\frac{1}{3}} = (\frac{25}{4})^{\frac{1}{3}}$$
\rozwStop
\odpStart
$(\frac{25}{4})^{\frac{1}{3}}$
\odpStop
\testStart
A.$(\frac{25}{4})^{\frac{1}{3}}$\\ B.$-(\frac{25}{4})^{\frac{1}{3}}$\\ C.$\frac{25}{4}$\\ D.$-\frac{25}{4}$\\ E.$\frac{1}{3}$
\testStop
\kluczStart
A
\kluczStop



\zadStart{Zadanie z Wikieł Z 1.4 g) moja wersja nr 2}

Oblicz wartość wyrażenia $[8,25-0,5^{-0,5}(2^{-0,5}+4^{-0,25})]^{\frac{1}{5}}$.
\zadStop
\rozwStart{Patryk Wirkus}{}
$$[8,25-0,5^{-0,5}(2^{-0,5}+4^{-0,25})]^{\frac{1}{5}} = [8,25-2^{0,5}(0,5^{0,5}+0,25^{0,25})]^{\frac{1}{5}} =$$
$$=[8,25 - 1 - \sqrt{2} \cdot 0,25^{0,25}]^{\frac{1}{5}} = [8,25 - 1 - (4\cdot 0,25)^{0,25}]^{\frac{1}{5}} = 6,25^{\frac{1}{5}} = (\frac{25}{4})^{\frac{1}{5}}$$
\rozwStop
\odpStart
$(\frac{25}{4})^{\frac{1}{5}}$
\odpStop
\testStart
A.$(\frac{25}{4})^{\frac{1}{5}}$\\ B.$-(\frac{25}{4})^{\frac{1}{5}}$\\ C.$\frac{25}{4}$\\ D.$-\frac{25}{4}$\\ E.$\frac{1}{5}$
\testStop
\kluczStart
A
\kluczStop



\zadStart{Zadanie z Wikieł Z 1.4 g) moja wersja nr 3}

Oblicz wartość wyrażenia $[8,25-0,5^{-0,5}(2^{-0,5}+4^{-0,25})]^{\frac{1}{7}}$.
\zadStop
\rozwStart{Patryk Wirkus}{}
$$[8,25-0,5^{-0,5}(2^{-0,5}+4^{-0,25})]^{\frac{1}{7}} = [8,25-2^{0,5}(0,5^{0,5}+0,25^{0,25})]^{\frac{1}{7}} =$$
$$=[8,25 - 1 - \sqrt{2} \cdot 0,25^{0,25}]^{\frac{1}{7}} = [8,25 - 1 - (4\cdot 0,25)^{0,25}]^{\frac{1}{7}} = 6,25^{\frac{1}{7}} = (\frac{25}{4})^{\frac{1}{7}}$$
\rozwStop
\odpStart
$(\frac{25}{4})^{\frac{1}{7}}$
\odpStop
\testStart
A.$(\frac{25}{4})^{\frac{1}{7}}$\\ B.$-(\frac{25}{4})^{\frac{1}{7}}$\\ C.$\frac{25}{4}$\\ D.$-\frac{25}{4}$\\ E.$\frac{1}{7}$
\testStop
\kluczStart
A
\kluczStop



\zadStart{Zadanie z Wikieł Z 1.4 g) moja wersja nr 4}

Oblicz wartość wyrażenia $[8,25-0,5^{-0,5}(2^{-0,5}+4^{-0,25})]^{\frac{1}{9}}$.
\zadStop
\rozwStart{Patryk Wirkus}{}
$$[8,25-0,5^{-0,5}(2^{-0,5}+4^{-0,25})]^{\frac{1}{9}} = [8,25-2^{0,5}(0,5^{0,5}+0,25^{0,25})]^{\frac{1}{9}} =$$
$$=[8,25 - 1 - \sqrt{2} \cdot 0,25^{0,25}]^{\frac{1}{9}} = [8,25 - 1 - (4\cdot 0,25)^{0,25}]^{\frac{1}{9}} = 6,25^{\frac{1}{9}} = (\frac{25}{4})^{\frac{1}{9}}$$
\rozwStop
\odpStart
$(\frac{25}{4})^{\frac{1}{9}}$
\odpStop
\testStart
A.$(\frac{25}{4})^{\frac{1}{9}}$\\ B.$-(\frac{25}{4})^{\frac{1}{9}}$\\ C.$\frac{25}{4}$\\ D.$-\frac{25}{4}$\\ E.$\frac{1}{9}$
\testStop
\kluczStart
A
\kluczStop



\zadStart{Zadanie z Wikieł Z 1.4 g) moja wersja nr 5}

Oblicz wartość wyrażenia $[8,25-0,5^{-0,5}(2^{-0,5}+4^{-0,25})]^{\frac{1}{11}}$.
\zadStop
\rozwStart{Patryk Wirkus}{}
$$[8,25-0,5^{-0,5}(2^{-0,5}+4^{-0,25})]^{\frac{1}{11}} = [8,25-2^{0,5}(0,5^{0,5}+0,25^{0,25})]^{\frac{1}{11}} =$$
$$=[8,25 - 1 - \sqrt{2} \cdot 0,25^{0,25}]^{\frac{1}{11}} = [8,25 - 1 - (4\cdot 0,25)^{0,25}]^{\frac{1}{11}} = 6,25^{\frac{1}{11}} = (\frac{25}{4})^{\frac{1}{11}}$$
\rozwStop
\odpStart
$(\frac{25}{4})^{\frac{1}{11}}$
\odpStop
\testStart
A.$(\frac{25}{4})^{\frac{1}{11}}$\\ B.$-(\frac{25}{4})^{\frac{1}{11}}$\\ C.$\frac{25}{4}$\\ D.$-\frac{25}{4}$\\ E.$\frac{1}{11}$
\testStop
\kluczStart
A
\kluczStop



\zadStart{Zadanie z Wikieł Z 1.4 g) moja wersja nr 6}

Oblicz wartość wyrażenia $[8,25-0,5^{-0,5}(2^{-0,5}+4^{-0,25})]^{\frac{1}{13}}$.
\zadStop
\rozwStart{Patryk Wirkus}{}
$$[8,25-0,5^{-0,5}(2^{-0,5}+4^{-0,25})]^{\frac{1}{13}} = [8,25-2^{0,5}(0,5^{0,5}+0,25^{0,25})]^{\frac{1}{13}} =$$
$$=[8,25 - 1 - \sqrt{2} \cdot 0,25^{0,25}]^{\frac{1}{13}} = [8,25 - 1 - (4\cdot 0,25)^{0,25}]^{\frac{1}{13}} = 6,25^{\frac{1}{13}} = (\frac{25}{4})^{\frac{1}{13}}$$
\rozwStop
\odpStart
$(\frac{25}{4})^{\frac{1}{13}}$
\odpStop
\testStart
A.$(\frac{25}{4})^{\frac{1}{13}}$\\ B.$-(\frac{25}{4})^{\frac{1}{13}}$\\ C.$\frac{25}{4}$\\ D.$-\frac{25}{4}$\\ E.$\frac{1}{13}$
\testStop
\kluczStart
A
\kluczStop



\zadStart{Zadanie z Wikieł Z 1.4 g) moja wersja nr 7}

Oblicz wartość wyrażenia $[8,25-0,5^{-0,5}(2^{-0,5}+4^{-0,25})]^{\frac{1}{15}}$.
\zadStop
\rozwStart{Patryk Wirkus}{}
$$[8,25-0,5^{-0,5}(2^{-0,5}+4^{-0,25})]^{\frac{1}{15}} = [8,25-2^{0,5}(0,5^{0,5}+0,25^{0,25})]^{\frac{1}{15}} =$$
$$=[8,25 - 1 - \sqrt{2} \cdot 0,25^{0,25}]^{\frac{1}{15}} = [8,25 - 1 - (4\cdot 0,25)^{0,25}]^{\frac{1}{15}} = 6,25^{\frac{1}{15}} = (\frac{25}{4})^{\frac{1}{15}}$$
\rozwStop
\odpStart
$(\frac{25}{4})^{\frac{1}{15}}$
\odpStop
\testStart
A.$(\frac{25}{4})^{\frac{1}{15}}$\\ B.$-(\frac{25}{4})^{\frac{1}{15}}$\\ C.$\frac{25}{4}$\\ D.$-\frac{25}{4}$\\ E.$\frac{1}{15}$
\testStop
\kluczStart
A
\kluczStop



\zadStart{Zadanie z Wikieł Z 1.4 g) moja wersja nr 8}

Oblicz wartość wyrażenia $[8,25-0,5^{-0,5}(2^{-0,5}+4^{-0,25})]^{\frac{1}{17}}$.
\zadStop
\rozwStart{Patryk Wirkus}{}
$$[8,25-0,5^{-0,5}(2^{-0,5}+4^{-0,25})]^{\frac{1}{17}} = [8,25-2^{0,5}(0,5^{0,5}+0,25^{0,25})]^{\frac{1}{17}} =$$
$$=[8,25 - 1 - \sqrt{2} \cdot 0,25^{0,25}]^{\frac{1}{17}} = [8,25 - 1 - (4\cdot 0,25)^{0,25}]^{\frac{1}{17}} = 6,25^{\frac{1}{17}} = (\frac{25}{4})^{\frac{1}{17}}$$
\rozwStop
\odpStart
$(\frac{25}{4})^{\frac{1}{17}}$
\odpStop
\testStart
A.$(\frac{25}{4})^{\frac{1}{17}}$\\ B.$-(\frac{25}{4})^{\frac{1}{17}}$\\ C.$\frac{25}{4}$\\ D.$-\frac{25}{4}$\\ E.$\frac{1}{17}$
\testStop
\kluczStart
A
\kluczStop



\zadStart{Zadanie z Wikieł Z 1.4 g) moja wersja nr 9}

Oblicz wartość wyrażenia $[8,25-0,5^{-0,5}(2^{-0,5}+4^{-0,25})]^{\frac{1}{19}}$.
\zadStop
\rozwStart{Patryk Wirkus}{}
$$[8,25-0,5^{-0,5}(2^{-0,5}+4^{-0,25})]^{\frac{1}{19}} = [8,25-2^{0,5}(0,5^{0,5}+0,25^{0,25})]^{\frac{1}{19}} =$$
$$=[8,25 - 1 - \sqrt{2} \cdot 0,25^{0,25}]^{\frac{1}{19}} = [8,25 - 1 - (4\cdot 0,25)^{0,25}]^{\frac{1}{19}} = 6,25^{\frac{1}{19}} = (\frac{25}{4})^{\frac{1}{19}}$$
\rozwStop
\odpStart
$(\frac{25}{4})^{\frac{1}{19}}$
\odpStop
\testStart
A.$(\frac{25}{4})^{\frac{1}{19}}$\\ B.$-(\frac{25}{4})^{\frac{1}{19}}$\\ C.$\frac{25}{4}$\\ D.$-\frac{25}{4}$\\ E.$\frac{1}{19}$
\testStop
\kluczStart
A
\kluczStop



\zadStart{Zadanie z Wikieł Z 1.4 g) moja wersja nr 10}

Oblicz wartość wyrażenia $[8,25-0,5^{-0,5}(2^{-0,5}+4^{-0,25})]^{\frac{1}{21}}$.
\zadStop
\rozwStart{Patryk Wirkus}{}
$$[8,25-0,5^{-0,5}(2^{-0,5}+4^{-0,25})]^{\frac{1}{21}} = [8,25-2^{0,5}(0,5^{0,5}+0,25^{0,25})]^{\frac{1}{21}} =$$
$$=[8,25 - 1 - \sqrt{2} \cdot 0,25^{0,25}]^{\frac{1}{21}} = [8,25 - 1 - (4\cdot 0,25)^{0,25}]^{\frac{1}{21}} = 6,25^{\frac{1}{21}} = (\frac{25}{4})^{\frac{1}{21}}$$
\rozwStop
\odpStart
$(\frac{25}{4})^{\frac{1}{21}}$
\odpStop
\testStart
A.$(\frac{25}{4})^{\frac{1}{21}}$\\ B.$-(\frac{25}{4})^{\frac{1}{21}}$\\ C.$\frac{25}{4}$\\ D.$-\frac{25}{4}$\\ E.$\frac{1}{21}$
\testStop
\kluczStart
A
\kluczStop





\end{document}
