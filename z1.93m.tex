\documentclass[12pt, a4paper]{article}
\usepackage[utf8]{inputenc}
\usepackage{polski}

\usepackage{amsthm}  %pakiet do tworzenia twierdzeń itp.
\usepackage{amsmath} %pakiet do niektórych symboli matematycznych
\usepackage{amssymb} %pakiet do symboli mat., np. \nsubseteq
\usepackage{amsfonts}
\usepackage{graphicx} %obsługa plików graficznych z rozszerzeniem png, jpg
\theoremstyle{definition} %styl dla definicji
\newtheorem{zad}{} 
\title{Multizestaw zadań}
\author{Robert Fidytek}
%\date{\today}
\date{}
\newcounter{liczniksekcji}
\newcommand{\kategoria}[1]{\section{#1}} %olreślamy nazwę kateforii zadań
\newcommand{\zadStart}[1]{\begin{zad}#1\newline} %oznaczenie początku zadania
\newcommand{\zadStop}{\end{zad}}   %oznaczenie końca zadania
%Makra opcjonarne (nie muszą występować):
\newcommand{\rozwStart}[2]{\noindent \textbf{Rozwiązanie (autor #1 , recenzent #2): }\newline} %oznaczenie początku rozwiązania, opcjonarnie można wprowadzić informację o autorze rozwiązania zadania i recenzencie poprawności wykonania rozwiązania zadania
\newcommand{\rozwStop}{\newline}                                            %oznaczenie końca rozwiązania
\newcommand{\odpStart}{\noindent \textbf{Odpowiedź:}\newline}    %oznaczenie początku odpowiedzi końcowej (wypisanie wyniku)
\newcommand{\odpStop}{\newline}                                             %oznaczenie końca odpowiedzi końcowej (wypisanie wyniku)
\newcommand{\testStart}{\noindent \textbf{Test:}\newline} %ewentualne możliwe opcje odpowiedzi testowej: A. ? B. ? C. ? D. ? itd.
\newcommand{\testStop}{\newline} %koniec wprowadzania odpowiedzi testowych
\newcommand{\kluczStart}{\noindent \textbf{Test poprawna odpowiedź:}\newline} %klucz, poprawna odpowiedź pytania testowego (jedna literka): A lub B lub C lub D itd.
\newcommand{\kluczStop}{\newline} %koniec poprawnej odpowiedzi pytania testowego 
\newcommand{\wstawGrafike}[2]{\begin{figure}[h] \includegraphics[scale=#2] {#1} \end{figure}} %gdyby była potrzeba wstawienia obrazka, parametry: nazwa pliku, skala (jak nie wiesz co wpisać, to wpisz 1)

\begin{document}
\maketitle


\kategoria{Wikieł/Z1.93m}
\zadStart{Zadanie z Wikieł Z1.93 m) moja wersja nr [nrWersji]}
%[z]:[10,100,1000,10000,100000]
%[l]=int(math.log([z], 10))
%[c]=-[l]
%[ac]=-4*1*[c]
%[ac2]=0+[ac]
%[p]=int(math.sqrt( [ac] ))
%[t]=int([p]/2)
%[w1]=[z]**[t]
%[w2]=[z]**(-[t])
Rozwiąż równanie.\\
Podane równanie $ x^{\log x} = [z] $\\
\zadStop
\rozwStart{Martyna Czarnobaj}{}
\begin{center}
	$ x^{\log x} = [z] $\\
\end{center}
	Logarytmujemy obie strony i zakładamy, że $ x>0 $.\\
\begin{center}
	$ \log x^{\log x} = \log [z] $\\
	$ \log x * \log x = [l] $\\
	$ (\log x)^{2}=[l] $\\
\end{center}
Przyjmujemy, że $ \log x = t $ i otrzymujemy:\\
\begin{center}
$ t^{2} - [l] = 0 $\\	 
\end{center}
Aby wyliczyć $ t $ wykorzystamy wzór równania kwadratowego:\\
\begin{center}
	$ t = \frac{-b \pm \sqrt{b^{2} - 4ac}}{2a} $
\end{center}
Z naszego wzoru na $ t $ otzrymujemy, że $ a = 1, b = 0 i c = -[l] $\\
\begin{center}
	$ t = \frac{-0 \pm \sqrt{0^{2}-4*1*([c])}}{2*1} $\\
	$ t = \frac{0 \pm \sqrt{0-4*1*([c])}}{2*1} $\\
	$ t = \frac{0 \pm \sqrt{0+[ac]}}{2} $\\
	$ t = \frac{0 \pm \sqrt{[ac2]}}{2} $\\
	$ t = \frac{0 \pm [p]}{2} $\\
	$ t1 = \frac{0 + [p]}{2} $ i $ t2 = \frac{0 - [p]}{2} $\\
	$ t1 = \frac{[p]}{2} $ i $ t2 = -\frac{[p]}{2} $\\
	$ t1 = [t] $ i $ t2 = -[t] $\\
\end{center}
Wyniki podstawiamy.
\begin{center}
	$ \log t1  = [t] \iff [z]^{[t]}=[w1] $ i $ \log t2 = -[t] \iff [z]^{-[t]}=[w2]$\\
\end{center}
Koniec rozwiązania.\\
\rozwStop
\odpStart
$ \log t1  = [t] \iff [z]^{[t]}=[w1] $ i $ \log t2 = -[t] 
\iff [z]^{-[t]}=[w2]$\\
\odpStop
\testStart
A.$ [w1] $ i $ [w2]$\\
B.$ [t] $ i $ [w2]$\\
C.$ [w1] $ i $ [t]$\\
\testStop
\kluczStart
A
\kluczStop



\end{document}