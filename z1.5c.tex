\documentclass[12pt, a4paper]{article}
\usepackage[utf8]{inputenc}
\usepackage{polski}

\usepackage{amsthm}  %pakiet do tworzenia twierdzeń itp.
\usepackage{amsmath} %pakiet do niektórych symboli matematycznych
\usepackage{amssymb} %pakiet do symboli mat., np. \nsubseteq
\usepackage{amsfonts}
\usepackage{graphicx} %obsługa plików graficznych z rozszerzeniem png, jpg
\theoremstyle{definition} %styl dla definicji
\newtheorem{zad}{} 
\title{Multizestaw zadań}
\author{Robert Fidytek}
%\date{\today}
\date{}
\newcounter{liczniksekcji}
\newcommand{\kategoria}[1]{\section{#1}} %olreślamy nazwę kateforii zadań
\newcommand{\zadStart}[1]{\begin{zad}#1\newline} %oznaczenie początku zadania
\newcommand{\zadStop}{\end{zad}}   %oznaczenie końca zadania
%Makra opcjonarne (nie muszą występować):
\newcommand{\rozwStart}[2]{\noindent \textbf{Rozwiązanie (autor #1 , recenzent #2): }\newline} %oznaczenie początku rozwiązania, opcjonarnie można wprowadzić informację o autorze rozwiązania zadania i recenzencie poprawności wykonania rozwiązania zadania
\newcommand{\rozwStop}{\newline}                                            %oznaczenie końca rozwiązania
\newcommand{\odpStart}{\noindent \textbf{Odpowiedź:}\newline}    %oznaczenie początku odpowiedzi końcowej (wypisanie wyniku)
\newcommand{\odpStop}{\newline}                                             %oznaczenie końca odpowiedzi końcowej (wypisanie wyniku)
\newcommand{\testStart}{\noindent \textbf{Test:}\newline} %ewentualne możliwe opcje odpowiedzi testowej: A. ? B. ? C. ? D. ? itd.
\newcommand{\testStop}{\newline} %koniec wprowadzania odpowiedzi testowych
\newcommand{\kluczStart}{\noindent \textbf{Test poprawna odpowiedź:}\newline} %klucz, poprawna odpowiedź pytania testowego (jedna literka): A lub B lub C lub D itd.
\newcommand{\kluczStop}{\newline} %koniec poprawnej odpowiedzi pytania testowego 
\newcommand{\wstawGrafike}[2]{\begin{figure}[h] \includegraphics[scale=#2] {#1} \end{figure}} %gdyby była potrzeba wstawienia obrazka, parametry: nazwa pliku, skala (jak nie wiesz co wpisać, to wpisz 1)

\begin{document}
\maketitle


\kategoria{Wikieł/Z1.5c}
\zadStart{Zadanie z Wikieł Z 1.5 c) moja wersja nr [nrWersji]}
%[b]:[2, 3, 4, 5, 6, 7, 8, 9,10]
%[c]:[2, 3, 4, 5, 6, 7, 8, 9,10]
%[e]=random.randint(2,100)
%[d]=random.randint(2,12)
%[f]=random.randint(2,102)
%[m1]=1+[b]-[c]
%[m]=2*[b]
%[bc]=[b]*[c]
%[p]=-2+[m1]
%[m11]=[m1]*[m1]
%[4b]=4*[b]
%[wm]=[m11]-[4b]
%[2b]=2*[b]
%[w]=[m1]-[2b]
%[pierw]=pow([bc],1/2)
%[b]!=[c] and [m1]!=0 and [p]!=0 and [b]!=4 and [b]!=9 and [c]!=4 and [c]!=9 and [pierw].is_integer()==False and [wm]!=0 and [wm]!=1
Usunąć niewymierność z mianownika $\frac{1}{1+\sqrt{[b]}+\sqrt{[c]}}$.
\zadStop
\rozwStart{Jakub Ulrych}{Pascal Nawrocki}
$$\frac{1}{1+\sqrt{[b]}+\sqrt{[c]}}$$
$$\frac{1}{1+\sqrt{[b]}+\sqrt{[c]}}*\frac{1+\sqrt{[b]}-\sqrt{[c]}}{1+\sqrt{[b]}-\sqrt{[c]}}$$
$$\frac{1+\sqrt{[b]}-(\sqrt{[c]})}{[m1]+2\sqrt{[b]}}$$
$$\frac{1+\sqrt{[b]}-(\sqrt{[c]})}{[m1]+2\sqrt{[b]}}*\frac{[m1]-2\sqrt{[b]}}{[m1]-2\sqrt{[b]}}$$
$$\frac{[m1]-2\sqrt{[b]}+[m1]\sqrt{[b]}-[2b]-([m1]\sqrt{[c]})+2\sqrt{[bc]}}{[m1]*[m1]-4[b]}$$
$$\frac{[w]+[p]\sqrt{[b]}-([m1]\sqrt{[c]})+2\sqrt{[bc]}}{[m11]-[4b]}$$
$$\frac{[w]+[p]\sqrt{[b]}-([m1]\sqrt{[c]})+2\sqrt{[bc]}}{[wm]}$$
\rozwStop
\odpStart
$$\frac{[w]+[p]\sqrt{[b]}-([m1]\sqrt{[c]})+2\sqrt{[bc]}}{[wm]}$$
\odpStop
\testStart
A.$\frac{[w]+[p]\sqrt{[b]}-([m1]\sqrt{[c]})+2\sqrt{[bc]}}{[wm]}$
B.$\frac{[p]+[w]\sqrt{[b]}-([f]\sqrt{[c]})+2\sqrt{[e]}}{[wm]}$
C.$\frac{[e]+[f]\sqrt{[b]}-([f]\sqrt{[c]})+2\sqrt{[e]}}{[f]}$
D.$\frac{[w]+[f]\sqrt{[b]}-([m1]\sqrt{[c]})+2\sqrt{[bc]}}{[f]}$
\testStop
\kluczStart
A
\kluczStop



\end{document}