\documentclass[12pt, a4paper]{article}
\usepackage[utf8]{inputenc}
\usepackage{polski}

\usepackage{amsthm}  %pakiet do tworzenia twierdzeń itp.
\usepackage{amsmath} %pakiet do niektórych symboli matematycznych
\usepackage{amssymb} %pakiet do symboli mat., np. \nsubseteq
\usepackage{amsfonts}
\usepackage{graphicx} %obsługa plików graficznych z rozszerzeniem png, jpg
\theoremstyle{definition} %styl dla definicji
\newtheorem{zad}{} 
\title{Multizestaw zadań}
\author{Robert Fidytek}
%\date{\today}
\date{}
\newcounter{liczniksekcji}
\newcommand{\kategoria}[1]{\section{#1}} %olreślamy nazwę kateforii zadań
\newcommand{\zadStart}[1]{\begin{zad}#1\newline} %oznaczenie początku zadania
\newcommand{\zadStop}{\end{zad}}   %oznaczenie końca zadania
%Makra opcjonarne (nie muszą występować):
\newcommand{\rozwStart}[2]{\noindent \textbf{Rozwiązanie (autor #1 , recenzent #2): }\newline} %oznaczenie początku rozwiązania, opcjonarnie można wprowadzić informację o autorze rozwiązania zadania i recenzencie poprawności wykonania rozwiązania zadania
\newcommand{\rozwStop}{\newline}                                            %oznaczenie końca rozwiązania
\newcommand{\odpStart}{\noindent \textbf{Odpowiedź:}\newline}    %oznaczenie początku odpowiedzi końcowej (wypisanie wyniku)
\newcommand{\odpStop}{\newline}                                             %oznaczenie końca odpowiedzi końcowej (wypisanie wyniku)
\newcommand{\testStart}{\noindent \textbf{Test:}\newline} %ewentualne możliwe opcje odpowiedzi testowej: A. ? B. ? C. ? D. ? itd.
\newcommand{\testStop}{\newline} %koniec wprowadzania odpowiedzi testowych
\newcommand{\kluczStart}{\noindent \textbf{Test poprawna odpowiedź:}\newline} %klucz, poprawna odpowiedź pytania testowego (jedna literka): A lub B lub C lub D itd.
\newcommand{\kluczStop}{\newline} %koniec poprawnej odpowiedzi pytania testowego 
\newcommand{\wstawGrafike}[2]{\begin{figure}[h] \includegraphics[scale=#2] {#1} \end{figure}} %gdyby była potrzeba wstawienia obrazka, parametry: nazwa pliku, skala (jak nie wiesz co wpisać, to wpisz 1)

\begin{document}
\maketitle


\kategoria{Wikieł/Z5.11a}
\zadStart{Zadanie z Wikieł Z 5.11 a) moja wersja nr [nrWersji]}
%[a]:[11,12,13]
%[b]:[11,12,13,14,15,16,17,18]
%[c]:[8,10]
%[d]:[7,8,9]
%[e]:[2,3,4,5,6,7,8,9]
%[f]=2
%[da]=[a]-[d]
%[db]=(-1)*[b]+[e]
%[dc]=[c]-[f]
%[delta]= ([db]*[db])-(4*[da]*[dc])
%[delta2]=abs([delta])
%[delta3]=int(math.sqrt([delta2]))
%[l1]=(-[db]-[delta3])
%[l2]=(-[db]+[delta3])
%[m]=2*[da]
%[x1]=[l1]/[m]
%[x2]=[l2]/[m]
%[ix1]=int([l1]/[m])
%[ix2]=int([l2]/[m])
%[a1]=2*[a]
%[a2]=[a1]*[ix1]
%[a3]=[a2]-[b]
%[a4]=[a1]*[ix2]
%[a5]=[a4]-[b]
%[d1]=2*[d]
%[d2]=[d1]*[ix1]
%[d3]=[d2]-[e]
%[d4]=[d1]*[ix2]
%[d5]=[d4]-[e]
%[fx1]=[ix1]*[ix1]
%[fx2]=[ix2]*[ix2]
%[bx1]=[b]*[ix1]
%[bx2]=[b]*[ix2]
%[ax1]=[a]*[fx1]
%[ax2]=[a]*[fx2]
%[x1c]=[c]-[bx1]
%[x2c]=[c]-[bx2]
%[ax1c]=[ax1]+[x1c]
%[ax2c]=[ax2]+[x2c]
%[ex1]=[e]*[ix1]
%[ex2]=[e]*[ix2]
%[dx1]=[d]*[fx1]
%[dx2]=[d]*[fx2]
%[x1f]=[f]-[ex1]
%[x2f]=[f]-[ex2]
%[dx1f]=[dx1]+[x1f]
%[dx2f]=[dx2]+[x2f]
%[a3x1]=[a3]*[ix1]
%[a5x2]=[a5]*[ix2]
%[d3x1]=[d3]*[ix1]
%[d5x2]=[d5]*[ix2]
%[a3x1c]=-[a3x1]+[ax1c]
%[a5x2c]=-[a5x2]+[ax2c]
%[d3x1f]=-[d3x1]+[dx1f]
%[d5x2f]=-[d5x2]+[dx2f]
%[a3d3]=[a3]-[d3]
%[a3d3m]=[a3]*[d3]
%[1a3d3m]=1+[a3d3m]
%[a5d5]=[a5]-[d5]
%[a5d5m]=[a5]*[d5]
%[1a5d5m]=1+[a5d5m]
%[a3d33]=-1*[a3d3]
%[dzie1]=math.gcd([a3d33],[1a3d3m])
%[dzie2]=math.gcd([a5d5],[1a5d5m])
%[o1]=int([a3d33]/[dzie1])
%[r1]=int([1a3d3m]/[dzie1])
%[o2]=int([a5d5]/[dzie2])
%[r2]=int([1a5d5m]/[dzie2])
%([delta] ==4 or [delta] ==9 or [delta] ==16) and [x1].is_integer()==True and [x2].is_integer()==True and [x2f]<0 and [x2c]<0 and [x1c]<0 and [x1f]<0 and [a3d3]<0 and [a5d5]>0 and [a3x1c]<0 and [d3x1f]<0 and [a5x2c]<0 and [d5x2f]<0
Wyznaczyć równania stycznych do krzywych $y=f(x)$ i $y=g(x)$ w punktach ich przecięcia, a następnie obliczyć tangens kąta ostrego między tymi krzywymi, jeżeli:\\
$f(x)=[a]x^2-[b]x+[c], \ \ \ \ \ g(x)=[d]x^2-[e]x+[f]$.
\zadStop
\rozwStart{Joanna Świerzbin}{}
$$f(x)=[a]x^2-[b]x+[c] \ \ \ \ \ g(x)=[d]x^2-[e]x+[f]$$
$$[a]x^2-[b]x+[c]=[d]x^2-[e]x+[f]$$
$$([a]-[d])x^2+(-[b]+[e])x+[c]-[f]=0$$
$$[da]x^2 [db] x+[dc]=0$$
$$\Delta = ([db])^2-4\cdot[da]\cdot[dc]$$
$$\Delta = [delta] $$ 
$$\sqrt{\Delta}=[delta3]$$
$$x_1=[ix1] \ \ \ \ \ \ \ x_2=[ix2]$$
\\
Równanie stycznej do wykresu funkcji $y=f(x)$ w punkcie $P(x_0,f(x_0))$:
$$
y-f(x_0)=f'(x_0)(x-x_0)
$$
Wzór na tangens kąta między krzywymi $f$ i $g$:
$$\tg(\alpha)=\Big|\frac{f'(x_0)-g'(x_0)}{1+f'(x_0)g'(x_0)}\Big| $$ \\
\begin{enumerate}
\item $x_1=[ix1]$:
\begin{itemize}
\item $f(x)$:
$$f'(x)=2\cdot[a]x-[b]=[a1]x-[b]$$
$$f'([ix1])=[a1]\cdot[ix1]-[b]=[a2]-[b]=[a3]$$
$$f([ix1])=[a]([ix1])^2-[b]\cdot[ix1]+[c]=$$
$$=[a]\cdot[fx1]-[bx1]+[c]=[ax1] [x1c]=[ax1c]$$
$$y_1=[a3](x-[ix1])+[ax1c]=[a3]x-[a3]\cdot[ix1]+[ax1c]=$$
$$=[a3]x-[a3x1]+[ax1c]=[a3]x [a3x1c]$$
\item $g(x)$:
$$g'(x)=2\cdot[d]x-[e]=[d1]x-[e]$$
$$g'([ix1])=[d1]\cdot[ix1]-[e]=[d2]-[e]=[d3]$$
$$g([ix1])=[d]([ix1])^2-[e]\cdot[ix1]+[f]=$$
$$=[d]\cdot[fx1]-[ex1]+[f]=[dx1] [x1f]=[dx1f]$$
$$y_2=[d3](x-[ix1])+[dx1f]=[d3]x-[d3]\cdot[ix1]+[dx1f]=$$
$$=[d3]x-[d3x1]+[dx1f]=[d3]x [d3x1f]$$
\end{itemize}
$$y_1'=[a3]$$
$$y_2'=[d3]$$
$$\tg(\alpha)=\Big|\frac{f'(x_0)-g'(x_0)}{1+f'(x_0)g'(x_0)}\Big| = \Big|\frac{[a3]-[d3]}{1+[a3]\cdot [d3]}\Big| = \Big|\frac{[a3d3]}{1+[a3d3m]}\Big|=
 \Big|\frac{[a3d3]}{[1a3d3m]}\Big| =\frac{[o1]}{[r1]} $$
\item $x_2=[ix2]$:
\begin{itemize}
\item $f(x)$:
$$f'(x)=[a1]x-[b]$$
$$f'([ix2])=[a1]\cdot[ix2]-[b]=[a4]-[b]=[a5]$$
$$f([ix2])=[a]([ix2])^2-[b]\cdot[ix2]+[c]=$$
$$=[a]\cdot[fx2]-[bx2]+[c]=[ax2] [x2c]=[ax2c]$$
$$y_1=[a5](x-[ix2])+[ax2c]=[a5]x-[a5]\cdot[ix2]+[ax2c]=$$
$$=[a5]x-[a5x2]+[ax2c]=[a5]x [a5x2c]$$
\item $g(x)$:
$$g'(x)=[d1]x-[e]$$
$$g'([ix2])=[d1]\cdot[ix2]-[e]=[d4]-[e]=[d5]$$
$$g([ix2])=[d]([ix2])^2-[e]\cdot[ix2]+[f]=$$
$$=[d]\cdot[fx2]-[ex2]+[f]=[dx2] [x2f]=[dx2f]$$
$$y_2=[d5](x-[ix2])+[dx2f]=[d5]x-[d5]\cdot[ix2]+[dx2f]=$$
$$=[d5]x-[d5x2]+[dx2f]=[d5]x [d5x2f]$$
\end{itemize}
$$y_1'=[a5]$$
$$y_2'=[d5]$$
$$\tg(\alpha)=\Big|\frac{f'(x_0)-g'(x_0)}{1+f'(x_0)g'(x_0)}\Big| = \Big|\frac{[a5]-[d5]}{1+[a5]\cdot [d5]}\Big| = \Big|\frac{[a5d5]}{1+[a5d5m]}\Big|=
 \Big|\frac{[a5d5]}{[1a5d5m]}\Big| =\frac{[o2]}{[r2]}$$
\end{enumerate}
Dla $x=[ix1]$ mamy $y_1=[a3]x [a3x1c]$ i $y_2=[d3]x [d3x1f]$, a $\tg(\alpha)= \frac{[o1]}{[r1]} $,\\
dla $x=[ix2]$ mamy $y_1=[a5]x [a5x2c]$ i $y_2=[d5]x [d5x2f]$, a $\tg(\alpha)= \frac{[o2]}{[r2]} $.\\
\rozwStop
\odpStart
Dla $x=[ix1]$ mamy $y_1=[a3]x [a3x1c]$ i $y_2=[d3]x [d3x1f]$, a $\tg(\alpha)= \frac{[o1]}{[r1]} $,\\
dla $x=[ix2]$ mamy $y_1=[a5]x [a5x2c]$ i $y_2=[d5]x [d5x2f]$, a $\tg(\alpha)= \frac{[o2]}{[r2]} $.\\
\odpStop
\testStart
A. Dla $x=[ix1]$ mamy $y_1=[a3]x [a3x1c]$ i $y_2=[d3]x [d3x1f]$, a $\tg(\alpha)= \frac{[o1]}{[r1]} $,\\
dla $x=[ix2]$ mamy $y_1=[a5]x [a5x2c]$ i $y_2=[d5]x [d5x2f]$, a $\tg(\alpha)= \frac{[o2]}{[r2]} $.\\
B. Dla $x=[ix2]$ mamy $y_1=[a3]x [a3x1c]$ i $y_2=[d3]x [d3x1f]$, a $\tg(\alpha)= \frac{[o1]}{[r1]} $,\\
dla $x=[ix1]$ mamy $y_1=[a5]x [a5x2c]$ i $y_2=[d5]x [d5x2f]$, a $\tg(\alpha)= \frac{[o2]}{[r2]} $.\\
C. Dla $x=[ix1]$ mamy $y_1=x [a3x1c]$ i $y_2=[d3]x [d3x1f]$, a $\tg(\alpha)= \frac{[o1]}{[r1]} $,\\
dla $x=[ix2]$ mamy $y_1=[a5]x [a5x2c]$ i $y_2=[d5]x [d5x2f]$, a $\tg(\alpha)= \frac{[o2]}{[r2]} $.\\
D. Dla $x=[ix1]$ mamy $y_1=[a3]x [a3x1c]$ i $y_2=[d3]x [d3x1f]$, a $\tg(\alpha)= \frac{[o1]}{[r1]} $,\\
dla $x=[ix2]$ mamy $y_1=x [a5x2c]$ i $y_2=[d5]x [d5x2f]$, a $\tg(\alpha)= \frac{[o2]}{[r2]} $.\\
E. Dla $x=[ix1]$ mamy $y_1=[a3]x [a3x1c]$ i $y_2=x [d3x1f]$, a $\tg(\alpha)= \frac{[o1]}{[r1]} $,\\
dla $x=[ix2]$ mamy $y_1=[a5]x [a5x2c]$ i $y_2=[d5]x [d5x2f]$, a $\tg(\alpha)= \frac{[o2]}{[r2]} $.\\
F. Dla $x=[ix1]$ mamy $y_1=[a3]x [a3x1c]$ i $y_2=[d3]x [d3x1f]$, a $\tg(\alpha)= \frac{[o1]}{[r1]} $,\\
dla $x=[ix2]$ mamy $y_1=[a5]x [a5x2c]$ i $y_2=x [d5x2f]$, a $\tg(\alpha)= \frac{[o2]}{[r2]} $.\\
\testStop
\kluczStart
A
\kluczStop



\end{document}