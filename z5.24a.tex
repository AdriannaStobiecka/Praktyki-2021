\documentclass[12pt, a4paper]{article}
\usepackage[utf8]{inputenc}
\usepackage{polski}

\usepackage{amsthm}  %pakiet do tworzenia twierdzeń itp.
\usepackage{amsmath} %pakiet do niektórych symboli matematycznych
\usepackage{amssymb} %pakiet do symboli mat., np. \nsubseteq
\usepackage{amsfonts}
\usepackage{graphicx} %obsługa plików graficznych z rozszerzeniem png, jpg
\theoremstyle{definition} %styl dla definicji
\newtheorem{zad}{} 
\title{Multizestaw zadań}
\author{Radosław Grzyb}
%\date{\today}
\date{}
\newcounter{liczniksekcji}
\newcommand{\kategoria}[1]{\section{#1}} %olreślamy nazwę kateforii zadań
\newcommand{\zadStart}[1]{\begin{zad}#1\newline} %oznaczenie początku zadania
\newcommand{\zadStop}{\end{zad}}   %oznaczenie końca zadania
%Makra opcjonarne (nie muszą występować):
\newcommand{\rozwStart}[2]{\noindent \textbf{Rozwiązanie (autor #1 , recenzent #2): }\newline} %oznaczenie początku rozwiązania, opcjonarnie można wprowadzić informację o autorze rozwiązania zadania i recenzencie poprawności wykonania rozwiązania zadania
\newcommand{\rozwStop}{\newline}                                            %oznaczenie końca rozwiązania
\newcommand{\odpStart}{\noindent \textbf{Odpowiedź:}\newline}    %oznaczenie początku odpowiedzi końcowej (wypisanie wyniku)
\newcommand{\odpStop}{\newline}                                             %oznaczenie końca odpowiedzi końcowej (wypisanie wyniku)
\newcommand{\testStart}{\noindent \textbf{Test:}\newline} %ewentualne możliwe opcje odpowiedzi testowej: A. ? B. ? C. ? D. ? itd.
\newcommand{\testStop}{\newline} %koniec wprowadzania odpowiedzi testowych
\newcommand{\kluczStart}{\noindent \textbf{Test poprawna odpowiedź:}\newline} %klucz, poprawna odpowiedź pytania testowego (jedna literka): A lub B lub C lub D itd.
\newcommand{\kluczStop}{\newline} %koniec poprawnej odpowiedzi pytania testowego 
\newcommand{\wstawGrafike}[2]{\begin{figure}[h] \includegraphics[scale=#2] {#1} \end{figure}} %gdyby była potrzeba wstawienia obrazka, parametry: nazwa pliku, skala (jak nie wiesz co wpisać, to wpisz 1)

\begin{document}
\maketitle


\kategoria{Wikieł/Z5.24a}
\zadStart{Zadanie z Wikieł Z 5.24 a) moja wersja nr [nrWersji]}
%[b]:[2,3,4,5,6,7,8,9,10,11,12]
%[b]:[2,3,4,5,6,7,8,9,10,11,12]
%[c]=random.randint(2,10)
%[d]=3*[a]
%[e]=2*[b]
%[f]=random.randint(1,10)
%[g]=[e]^2
%[h]=4*[d]*[c]
%[i]=[g]-[h]
%[i]<0
Wykazać, że dana funkcja nie ma ekstremum:
$$f(x)=[a]x^3+[b]x^2+[c]x+[f]$$
\zadStop
\rozwStart{Klaudia Klejdysz}{}
Funkcja $f(x)$ jest różniczkowalna w przedziale $(-\infty,\infty)$, więc może mieć ekstremum tylko w tych punktach, w których pochodna jest równa zeru. Ponieważ:
$$f'(x)=[d]x^2+[e]x+[c]\text{.}$$
Z warunku koniecznego istnienia ekstremum mamy do rozwiązania równanie $f'(x)=0$.
$$[d]x^2+[e]x+[c]=0$$
Wyznaczamy miejsca zerowe:
$$\Delta=[e]^2-4*[d]*[c]=[g]-[h]=[i]<0$$
Delta jest mniejsza od zera, więc nasze równanie nie ma pierwiastków rzeczywistych, a co za tym idzie funkcja $f(x)$ nie ma ekstremum.
\rozwStop
\odpStart
Funkcja $f(x)$ nie ma ekstremum
\odpStop



\end{document}

