\documentclass[12pt, a4paper]{article}
\usepackage[utf8]{inputenc}
\usepackage{polski}

\usepackage{amsthm}  %pakiet do tworzenia twierdzeń itp.
\usepackage{amsmath} %pakiet do niektórych symboli matematycznych
\usepackage{amssymb} %pakiet do symboli mat., np. \nsubseteq
\usepackage{amsfonts}
\usepackage{graphicx} %obsługa plików graficznych z rozszerzeniem png, jpg
\theoremstyle{definition} %styl dla definicji
\newtheorem{zad}{} 
\title{Multizestaw zadań}
\author{Robert Fidytek}
%\date{\today}
\date{}
\newcounter{liczniksekcji}
\newcommand{\kategoria}[1]{\section{#1}} %olreślamy nazwę kateforii zadań
\newcommand{\zadStart}[1]{\begin{zad}#1\newline} %oznaczenie początku zadania
\newcommand{\zadStop}{\end{zad}}   %oznaczenie końca zadania
%Makra opcjonarne (nie muszą występować):
\newcommand{\rozwStart}[2]{\noindent \textbf{Rozwiązanie (autor #1 , recenzent #2): }\newline} %oznaczenie początku rozwiązania, opcjonarnie można wprowadzić informację o autorze rozwiązania zadania i recenzencie poprawności wykonania rozwiązania zadania
\newcommand{\rozwStop}{\newline}                                            %oznaczenie końca rozwiązania
\newcommand{\odpStart}{\noindent \textbf{Odpowiedź:}\newline}    %oznaczenie początku odpowiedzi końcowej (wypisanie wyniku)
\newcommand{\odpStop}{\newline}                                             %oznaczenie końca odpowiedzi końcowej (wypisanie wyniku)
\newcommand{\testStart}{\noindent \textbf{Test:}\newline} %ewentualne możliwe opcje odpowiedzi testowej: A. ? B. ? C. ? D. ? itd.
\newcommand{\testStop}{\newline} %koniec wprowadzania odpowiedzi testowych
\newcommand{\kluczStart}{\noindent \textbf{Test poprawna odpowiedź:}\newline} %klucz, poprawna odpowiedź pytania testowego (jedna literka): A lub B lub C lub D itd.
\newcommand{\kluczStop}{\newline} %koniec poprawnej odpowiedzi pytania testowego 
\newcommand{\wstawGrafike}[2]{\begin{figure}[h] \includegraphics[scale=#2] {#1} \end{figure}} %gdyby była potrzeba wstawienia obrazka, parametry: nazwa pliku, skala (jak nie wiesz co wpisać, to wpisz 1)

\begin{document}
\maketitle


\kategoria{Wikieł/Z2.63}
\zadStart{Zadanie z Wikieł Z 2.63 moja wersja nr [nrWersji]}
%[a1]:[2]
%[b1]:[1]
%[b2]:[3,4,5,6,7,8,9,10,11,12,13,14,15,16,17,18,19,20,21,22,23,24,25,26,27,28,29,30,31,32,33,34,35,36,37,38,39,40,41,42,43,44,45,46,47,48,49,50]
%[a2]=[b2]*[b2]
%[aa1]=[a1]*[a1]
%[aa2]=[a2]*[a2]
%[bb1]=[b1]*[b1]
%[bb2]=[b2]*[b2]
%[2bb1]=[bb1]*[bb2]
%[a]=[2bb1]-[aa1]
%[x]=[bb2]-1
%[cx]=[x]*[aa1]
%[d]=[cx]-[a]
%[aae2]=[aa2]*[a]
%[y]=[aae2]/[a]
%[cy]=int([y])
%[yx]=[cy]*[x]
%[b]=[aae2]/[d]
%[cb]=int([b])
%[f]=[yx]/[d]
%[cf]=int([f])
%[a]>0 and [d]>0 and [y].is_integer()==True and [b].is_integer()==True and [f].is_integer()==True 
Napisać równanie hiperboli, której osiami symetrii są osie układu, mając dane współrzędne dwóch punktów A(-[a1],[a2]), B([b1],[b2]) należących do tej hiperboli.
\zadStop
\rozwStart{Aleksandra Pasińska}{}
$$\frac{x^2}{a^2}-\frac{y^2}{b^2}=1$$
$$\left\{ \begin{array}{ll}
\frac{[aa1]}{a^2}-\frac{[aa2]}{b^2}=1\\
\frac{[bb1]}{a^2}-\frac{[bb2]}{b^2}=1/\cdot [bb2]
\end{array} \right.$$
$$\left\{ \begin{array}{ll}
\frac{[aa1]}{a^2}-\frac{[aa2]}{b^2}=1\\
\frac{[2bb1]}{a^2}-\frac{[aa2]}{b^2}=[bb2]
\end{array} \right.$$
$$\frac{[aa1]}{a^2}-\frac{[aa2]}{b^2}-\frac{[2bb1]}{a^2}+\frac{[aa2]}{b^2}=-[x]$$
$$-\frac{[a]}{a^2}=-[x],a^2=\frac{[a]}{[x]}$$
$$\frac{[cx]}{[a]}-\frac{[aa2]}{b^2}=1$$
$$-\frac{[aa2]}{b^2}=-\frac{[d]}{[a]}$$
$$b^2=\frac{[aae2]}{[d]}$$
$$\frac{[x]}{[a]}x^2-\frac{[d]}{[aae2]}y^2=1$$
$$[yx]x^2-[d]y^2=[aae2]$$
$$[cf]x^2-y^2=[cb]$$
\rozwStop
\odpStart
$[cf]x^2-y^2=[cb]$\\
\odpStop
\testStart
A.$[cf]x^2-y^2=[cb]$
B.$ 3x^2+[d]y^2=[aae2]$
C.$-3x^2+y^2=[aae2]$
D.$x^2+[d]y^2=[aae2]$
E.$-x^2+[d]y^2=[aae2]$
F.$-3x^2+[d]y=[aae2]$
G.$-3x+[d]y^2=[aae2]$
H.$-3x^2+[d]y^3=[aae2]$
I.$-3x^3+[d]y^2=[aae2]$
\testStop
\kluczStart
A
\kluczStop



\end{document}