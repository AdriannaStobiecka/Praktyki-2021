\documentclass[12pt, a4paper]{article}
\usepackage[utf8]{inputenc}
\usepackage{polski}
\usepackage{amsthm}  %pakiet do tworzenia twierdzeń itp.
\usepackage{amsmath} %pakiet do niektórych symboli matematycznych
\usepackage{amssymb} %pakiet do symboli mat., np. \nsubseteq
\usepackage{amsfonts}
\usepackage{graphicx} %obsługa plików graficznych z rozszerzeniem png, jpg
\theoremstyle{definition} %styl dla definicji
\newtheorem{zad}{} 
\title{Multizestaw zadań}
\author{Patryk Wirkus}
%\date{\today}
\date{}
\newcommand{\kategoria}[1]{\section{#1}}
\newcommand{\zadStart}[1]{\begin{zad}#1\newline}
\newcommand{\zadStop}{\end{zad}}
\newcommand{\rozwStart}[2]{\noindent \textbf{Rozwiązanie (autor #1 , recenzent #2): }\newline}
\newcommand{\rozwStop}{\newline}                                           
\newcommand{\odpStart}{\noindent \textbf{Odpowiedź:}\newline}
\newcommand{\odpStop}{\newline}
\newcommand{\testStart}{\noindent \textbf{Test:}\newline}
\newcommand{\testStop}{\newline}
\newcommand{\kluczStart}{\noindent \textbf{Test poprawna odpowiedź:}\newline}
\newcommand{\kluczStop}{\newline}
\newcommand{\wstawGrafike}[2]{\begin{figure}[h] \includegraphics[scale=#2] {#1} \end{figure}}

\begin{document}
\maketitle

\kategoria{Wikieł/Z1.16d}


\zadStart{Zadanie z Wikieł Z 1.16 d) moja wersja nr 1}

Obliczyć symbol Newtona ${63 \choose 63}$.
\zadStop
\rozwStart{Patryk Wirkus}{}
$${63 \choose 63} = \frac{63!}{63! \cdot (63-63)!} = \frac{63!}{0! \cdot 63!} = \frac{63!}{1 \cdot 63!} = 1$$
\rozwStop
\odpStart
$1$
\odpStop
\testStart
A.$1$ B.$-1$ C.$0$ D.$63$ E.$-63$
\testStop
\kluczStart
A
\kluczStop



\zadStart{Zadanie z Wikieł Z 1.16 d) moja wersja nr 2}

Obliczyć symbol Newtona ${67 \choose 67}$.
\zadStop
\rozwStart{Patryk Wirkus}{}
$${67 \choose 67} = \frac{67!}{67! \cdot (67-67)!} = \frac{67!}{0! \cdot 67!} = \frac{67!}{1 \cdot 67!} = 1$$
\rozwStop
\odpStart
$1$
\odpStop
\testStart
A.$1$ B.$-1$ C.$0$ D.$67$ E.$-67$
\testStop
\kluczStart
A
\kluczStop



\zadStart{Zadanie z Wikieł Z 1.16 d) moja wersja nr 3}

Obliczyć symbol Newtona ${73 \choose 73}$.
\zadStop
\rozwStart{Patryk Wirkus}{}
$${73 \choose 73} = \frac{73!}{73! \cdot (73-73)!} = \frac{73!}{0! \cdot 73!} = \frac{73!}{1 \cdot 73!} = 1$$
\rozwStop
\odpStart
$1$
\odpStop
\testStart
A.$1$ B.$-1$ C.$0$ D.$73$ E.$-73$
\testStop
\kluczStart
A
\kluczStop



\zadStart{Zadanie z Wikieł Z 1.16 d) moja wersja nr 4}

Obliczyć symbol Newtona ${79 \choose 79}$.
\zadStop
\rozwStart{Patryk Wirkus}{}
$${79 \choose 79} = \frac{79!}{79! \cdot (79-79)!} = \frac{79!}{0! \cdot 79!} = \frac{79!}{1 \cdot 79!} = 1$$
\rozwStop
\odpStart
$1$
\odpStop
\testStart
A.$1$ B.$-1$ C.$0$ D.$79$ E.$-79$
\testStop
\kluczStart
A
\kluczStop



\zadStart{Zadanie z Wikieł Z 1.16 d) moja wersja nr 5}

Obliczyć symbol Newtona ${51 \choose 51}$.
\zadStop
\rozwStart{Patryk Wirkus}{}
$${51 \choose 51} = \frac{51!}{51! \cdot (51-51)!} = \frac{51!}{0! \cdot 51!} = \frac{51!}{1 \cdot 51!} = 1$$
\rozwStop
\odpStart
$1$
\odpStop
\testStart
A.$1$ B.$-1$ C.$0$ D.$51$ E.$-51$
\testStop
\kluczStart
A
\kluczStop



\zadStart{Zadanie z Wikieł Z 1.16 d) moja wersja nr 6}

Obliczyć symbol Newtona ${57 \choose 57}$.
\zadStop
\rozwStart{Patryk Wirkus}{}
$${57 \choose 57} = \frac{57!}{57! \cdot (57-57)!} = \frac{57!}{0! \cdot 57!} = \frac{57!}{1 \cdot 57!} = 1$$
\rozwStop
\odpStart
$1$
\odpStop
\testStart
A.$1$ B.$-1$ C.$0$ D.$57$ E.$-57$
\testStop
\kluczStart
A
\kluczStop



\zadStart{Zadanie z Wikieł Z 1.16 d) moja wersja nr 7}

Obliczyć symbol Newtona ${61 \choose 61}$.
\zadStop
\rozwStart{Patryk Wirkus}{}
$${61 \choose 61} = \frac{61!}{61! \cdot (61-61)!} = \frac{61!}{0! \cdot 61!} = \frac{61!}{1 \cdot 61!} = 1$$
\rozwStop
\odpStart
$1$
\odpStop
\testStart
A.$1$ B.$-1$ C.$0$ D.$61$ E.$-61$
\testStop
\kluczStart
A
\kluczStop



\zadStart{Zadanie z Wikieł Z 1.16 d) moja wersja nr 8}

Obliczyć symbol Newtona ${69 \choose 69}$.
\zadStop
\rozwStart{Patryk Wirkus}{}
$${69 \choose 69} = \frac{69!}{69! \cdot (69-69)!} = \frac{69!}{0! \cdot 69!} = \frac{69!}{1 \cdot 69!} = 1$$
\rozwStop
\odpStart
$1$
\odpStop
\testStart
A.$1$ B.$-1$ C.$0$ D.$69$ E.$-69$
\testStop
\kluczStart
A
\kluczStop



\zadStart{Zadanie z Wikieł Z 1.16 d) moja wersja nr 9}

Obliczyć symbol Newtona ${71 \choose 71}$.
\zadStop
\rozwStart{Patryk Wirkus}{}
$${71 \choose 71} = \frac{71!}{71! \cdot (71-71)!} = \frac{71!}{0! \cdot 71!} = \frac{71!}{1 \cdot 71!} = 1$$
\rozwStop
\odpStart
$1$
\odpStop
\testStart
A.$1$ B.$-1$ C.$0$ D.$71$ E.$-71$
\testStop
\kluczStart
A
\kluczStop



\zadStart{Zadanie z Wikieł Z 1.16 d) moja wersja nr 10}

Obliczyć symbol Newtona ${77 \choose 77}$.
\zadStop
\rozwStart{Patryk Wirkus}{}
$${77 \choose 77} = \frac{77!}{77! \cdot (77-77)!} = \frac{77!}{0! \cdot 77!} = \frac{77!}{1 \cdot 77!} = 1$$
\rozwStop
\odpStart
$1$
\odpStop
\testStart
A.$1$ B.$-1$ C.$0$ D.$77$ E.$-77$
\testStop
\kluczStart
A
\kluczStop





\end{document}
