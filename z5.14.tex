\documentclass[12pt, a4paper]{article}
\usepackage[utf8]{inputenc}
\usepackage{polski}

\usepackage{amsthm}  %pakiet do tworzenia twierdzeń itp.
\usepackage{amsmath} %pakiet do niektórych symboli matematycznych
\usepackage{amssymb} %pakiet do symboli mat., np. \nsubseteq
\usepackage{amsfonts}
\usepackage{graphicx} %obsługa plików graficznych z rozszerzeniem png, jpg
\theoremstyle{definition} %styl dla definicji
\newtheorem{zad}{} 
\title{Multizestaw zadań}
\author{Robert Fidytek}
%\date{\today}
\date{}
\newcounter{liczniksekcji}
\newcommand{\kategoria}[1]{\section{#1}} %olreślamy nazwę kateforii zadań
\newcommand{\zadStart}[1]{\begin{zad}#1\newline} %oznaczenie początku zadania
\newcommand{\zadStop}{\end{zad}}   %oznaczenie końca zadania
%Makra opcjonarne (nie muszą występować):
\newcommand{\rozwStart}[2]{\noindent \textbf{Rozwiązanie (autor #1 , recenzent #2): }\newline} %oznaczenie początku rozwiązania, opcjonarnie można wprowadzić informację o autorze rozwiązania zadania i recenzencie poprawności wykonania rozwiązania zadania
\newcommand{\rozwStop}{\newline}                                            %oznaczenie końca rozwiązania
\newcommand{\odpStart}{\noindent \textbf{Odpowiedź:}\newline}    %oznaczenie początku odpowiedzi końcowej (wypisanie wyniku)
\newcommand{\odpStop}{\newline}                                             %oznaczenie końca odpowiedzi końcowej (wypisanie wyniku)
\newcommand{\testStart}{\noindent \textbf{Test:}\newline} %ewentualne możliwe opcje odpowiedzi testowej: A. ? B. ? C. ? D. ? itd.
\newcommand{\testStop}{\newline} %koniec wprowadzania odpowiedzi testowych
\newcommand{\kluczStart}{\noindent \textbf{Test poprawna odpowiedź:}\newline} %klucz, poprawna odpowiedź pytania testowego (jedna literka): A lub B lub C lub D itd.
\newcommand{\kluczStop}{\newline} %koniec poprawnej odpowiedzi pytania testowego 
\newcommand{\wstawGrafike}[2]{\begin{figure}[h] \includegraphics[scale=#2] {#1} \end{figure}} %gdyby była potrzeba wstawienia obrazka, parametry: nazwa pliku, skala (jak nie wiesz co wpisać, to wpisz 1)

\begin{document}
\maketitle


\kategoria{Wikieł/Z5.14}
\zadStart{Zadanie z Wikieł Z 5.14 moja wersja nr [nrWersji]}
%[a]:[2,4,6,8,10,12]
%[b]:[2,3,4,5,6,7,8,9]
%[c]=[a]-1
%[d]=[a]+[c]
%[e]=[b]*[c]
%[f]=[a]*[a]
%[g]=[e]*[d]
%[h]=[f]*[a]
%[i]=[a]+[d]
%math.gcd([b],[a])==1 and math.gcd([e],[f])==1 and math.gcd([g],[h])==1
Dla funkcji $g(x)=\sqrt[{[a]}]{[b]x}$ wyznaczyć $ f'''(x)$ i ustalić jej dziedzinę.
\zadStop
\rozwStart{Joanna Świerzbin}{}
$$g(x)=\sqrt[{[a]}]{[b]x} = ([b]x)^{\frac{1}{[a]}}$$
$$g'(x)= \left( ([b]x)^{\frac{1}{[a]}} \right)' = \frac{[b]}{[a]}x^{\frac{1}{[a]}-1} = \frac{[b]}{[a]}x^{-\frac{[c]}{[a]}} $$
$$g''(x)= \left( \frac{[b]}{[a]}x^{-\frac{[c]}{[a]}} \right)'= -\frac{[b]}{[a]}\cdot \frac{[c]}{[a]}x^{-\frac{[c]}{[a]}-1}= -\frac{[e]}{[f]} x^{-\frac{[d]}{[a]}}$$
$$g'''(x)= \left( -\frac{[e]}{[f]} x^{-\frac{[d]}{[a]}} \right)'= \frac{[e]}{[f]}\cdot \frac{[d]}{[a]} x^{-\frac{[d]}{[a]}-1} = \frac{[g]}{[h]} x^{-\frac{[i]}{[a]}} $$
Dziedzina:
$$ x^{-\frac{[i]}{[a]}} \neq 0 \land x>0$$
$$ x \in ( 0, \infty ) $$
\rozwStop
\odpStart
$g'''(x)=  \frac{[g]}{[h]} x^{-\frac{[i]}{[a]}}\ \ \  ; \ \ \ D_{g'''} : x \in ( 0, \infty ) $
\odpStop
\testStart
A. $g'''(x)=  \frac{[g]}{[h]} x^{-\frac{[i]}{[a]}}\ \ \  ; \ \ \ D_{g'''} : x \in ( 0, \infty ) $\\
B. $g'''(x)=  \frac{1}{[h]} x^{-\frac{[i]}{[a]}}\ \ \  ; \ \ \ D_{g'''} : x \in ( 0, \infty ) $ \\
C. $g'''(x)=  x^{-\frac{[i]}{[a]}}\ \ \  ; \ \ \ D_{g'''} : x \in ( 0, \infty ) $ \\
D. $g'''(x)=  \frac{[g]}{[h]} x^{\frac{[i]}{[a]}}\ \ \  ; \ \ \ D_{g'''} : x \in ( 0, \infty ) $\\
E. $g'''(x)=  \frac{[g]}{[h]} x^{-\frac{[i]}{[a]}}\ \ \  ; \ \ \ D_{g'''} : x \in \mathbb{R} $\\
F. $g'''(x)=  \frac{[g]}{[h]} x^{-\frac{[i]}{[a]}}\ \ \  ; \ \ \ D_{g'''} : x \in ( 1, \infty ) $
\testStop
\kluczStart
A
\kluczStop



\end{document}