\documentclass[12pt, a4paper]{article}
\usepackage[utf8]{inputenc}
\usepackage{polski}

\usepackage{amsthm}  %pakiet do tworzenia twierdzeń itp.
\usepackage{amsmath} %pakiet do niektórych symboli matematycznych
\usepackage{amssymb} %pakiet do symboli mat., np. \nsubseteq
\usepackage{amsfonts}
\usepackage{graphicx} %obsługa plików graficznych z rozszerzeniem png, jpg
\theoremstyle{definition} %styl dla definicji
\newtheorem{zad}{} 
\title{Multizestaw zadań}
\author{Robert Fidytek}
%\date{\today}
\date{}
\newcounter{liczniksekcji}
\newcommand{\kategoria}[1]{\section{#1}} %olreślamy nazwę kateforii zadań
\newcommand{\zadStart}[1]{\begin{zad}#1\newline} %oznaczenie początku zadania
\newcommand{\zadStop}{\end{zad}}   %oznaczenie końca zadania
%Makra opcjonarne (nie muszą występować):
\newcommand{\rozwStart}[2]{\noindent \textbf{Rozwiązanie (autor #1 , recenzent #2): }\newline} %oznaczenie początku rozwiązania, opcjonarnie można wprowadzić informację o autorze rozwiązania zadania i recenzencie poprawności wykonania rozwiązania zadania
\newcommand{\rozwStop}{\newline}                                            %oznaczenie końca rozwiązania
\newcommand{\odpStart}{\noindent \textbf{Odpowiedź:}\newline}    %oznaczenie początku odpowiedzi końcowej (wypisanie wyniku)
\newcommand{\odpStop}{\newline}                                             %oznaczenie końca odpowiedzi końcowej (wypisanie wyniku)
\newcommand{\testStart}{\noindent \textbf{Test:}\newline} %ewentualne możliwe opcje odpowiedzi testowej: A. ? B. ? C. ? D. ? itd.
\newcommand{\testStop}{\newline} %koniec wprowadzania odpowiedzi testowych
\newcommand{\kluczStart}{\noindent \textbf{Test poprawna odpowiedź:}\newline} %klucz, poprawna odpowiedź pytania testowego (jedna literka): A lub B lub C lub D itd.
\newcommand{\kluczStop}{\newline} %koniec poprawnej odpowiedzi pytania testowego 
\newcommand{\wstawGrafike}[2]{\begin{figure}[h] \includegraphics[scale=#2] {#1} \end{figure}} %gdyby była potrzeba wstawienia obrazka, parametry: nazwa pliku, skala (jak nie wiesz co wpisać, to wpisz 1)

\begin{document}
\maketitle


\kategoria{Wikieł/Z2.23}
\zadStart{Zadanie z Wikieł Z 2.23 moja wersja nr [nrWersji]}
%[a1]:[2,3,4,5,6,7,8,9,10,11,12,13,14,15,16,17,18,19,20]
%[b1]:[2,3,4,5,6,7,8,9,10,11,12,13,14,15,16,17,18,19,20]
%[ab2]=[b1]-[a1]
%[ab2]>0 and [ab2]!=1 
Podać równanie parametryczne prostej przechodzącej przez punkty A([a1],[a1]), B([b1],[b1]).
\zadStop
\rozwStart{Aleksandra Pasińska}{}
$$\frac{x-[a1]}{[ab2]}=\frac{y-[a1]}{[ab2]}=t$$
$$x-[a1]=y-[a1]=[ab2]t$$
$$\left\{ \begin{array}{ll}
x=[ab2]t+[a1]\\
y=[ab2]t+[a1], t\in \mathbb{R}
\end{array} \right.$$
\rozwStop
\odpStart
$\left\{ \begin{array}{ll}
x=[ab2]t+[a1]\\
y=[ab2]t+[a1], t\in \mathbb{R}
\end{array} \right.$\\
\odpStop
\testStart
A.$\left\{ \begin{array}{ll}
x=[ab2]t+[a1]\\
y=[ab2]t+[a1], t\in \mathbb{R}
\end{array} \right.$
B.$[ab2]x-9=0$
C.$[ab2]x-[ab2]y=7$
D.$[ab2]x-[ab2]y+4=9$
E.$[ab2]y+1=0$
F.$[ab2]x+2=0$
G.$[ab2]x-[ab2]y=0$
H.$[ab2]x-[ab2]y+5=4$
I.$[ab2]x-[ab2]y+3=2$
\testStop
\kluczStart
A
\kluczStop



\end{document}