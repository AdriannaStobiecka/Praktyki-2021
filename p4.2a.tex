\documentclass[12pt, a4paper]{article}
\usepackage[utf8]{inputenc}
\usepackage{polski}

\usepackage{amsthm}  %pakiet do tworzenia twierdzeń itp.
\usepackage{amsmath} %pakiet do niektórych symboli matematycznych
\usepackage{amssymb} %pakiet do symboli mat., np. \nsubseteq
\usepackage{amsfonts}
\usepackage{graphicx} %obsługa plików graficznych z rozszerzeniem png, jpg
\theoremstyle{definition} %styl dla definicji
\newtheorem{zad}{} 
\title{Multizestaw zadań}
\author{Robert Fidytek}
%\date{\today}
\date{}
\newcounter{liczniksekcji}
\newcommand{\kategoria}[1]{\section{#1}} %olreślamy nazwę kateforii zadań
\newcommand{\zadStart}[1]{\begin{zad}#1\newline} %oznaczenie początku zadania
\newcommand{\zadStop}{\end{zad}}   %oznaczenie końca zadania
%Makra opcjonarne (nie muszą występować):
\newcommand{\rozwStart}[2]{\noindent \textbf{Rozwiązanie (autor #1 , recenzent #2): }\newline} %oznaczenie początku rozwiązania, opcjonarnie można wprowadzić informację o autorze rozwiązania zadania i recenzencie poprawności wykonania rozwiązania zadania
\newcommand{\rozwStop}{\newline}                                            %oznaczenie końca rozwiązania
\newcommand{\odpStart}{\noindent \textbf{Odpowiedź:}\newline}    %oznaczenie początku odpowiedzi końcowej (wypisanie wyniku)
\newcommand{\odpStop}{\newline}                                             %oznaczenie końca odpowiedzi końcowej (wypisanie wyniku)
\newcommand{\testStart}{\noindent \textbf{Test:}\newline} %ewentualne możliwe opcje odpowiedzi testowej: A. ? B. ? C. ? D. ? itd.
\newcommand{\testStop}{\newline} %koniec wprowadzania odpowiedzi testowych
\newcommand{\kluczStart}{\noindent \textbf{Test poprawna odpowiedź:}\newline} %klucz, poprawna odpowiedź pytania testowego (jedna literka): A lub B lub C lub D itd.
\newcommand{\kluczStop}{\newline} %koniec poprawnej odpowiedzi pytania testowego 
\newcommand{\wstawGrafike}[2]{\begin{figure}[h] \includegraphics[scale=#2] {#1} \end{figure}} %gdyby była potrzeba wstawienia obrazka, parametry: nazwa pliku, skala (jak nie wiesz co wpisać, to wpisz 1)

\begin{document}
\maketitle


\kategoria{Wikieł/P4.2a}
\zadStart{Zadanie z Wikieł P 4.2 a) moja wersja nr [nrWersji]}
%[a]:[2,3,4,5,6]
%[b]:[2,3,4,5,6]
%[c]:[2,3,4,5,6]
%[d]:[2,3,4,5,6]
%[e]:[2,3,4,5,6]
%[f]:[2,3,4,5,6]
%[a]=random.randint(2,15)
%[e]=random.randint(2,15)
%[c]=random.randint(1,15)
%[d]=random.randint(2,15)
%[b]=random.randint(2,15)
%[f]=random.randint(1,15)
%[cpier]=math.sqrt([c])
%[cpier2]=int([cpier])
%[ab]=int([a]/[b])
%[ciper]=[cpier2]*2
%math.gcd([a],[b])==[b] and math.gcd([c],[ab])==[ab] and [a]>=[b] and [cpier].is_integer()==True and [ab]>1 and math.gcd([b],[ciper])==1
Obliczyć granice funkcji $\displaystyle{\lim_{x \to [c]}} \frac{[a]-[b]\sqrt{x}}{x-[c]}$
\zadStop
\rozwStart{Pascal Nawrocki}{Martyna Czarnobaj}
Korzystając w mianowniku ze wzoru na różnicę kwadratów $a^2-b^2=(a-b)(a+b)$, otrzymujemy:
$$\displaystyle{\lim_{x \to [c]}} \frac{[a]-[b]\sqrt{x}}{x-[c]}=\big[\frac{0}{0}\big]=\displaystyle{\lim_{x \to [c]}} \frac{-[b](\sqrt{x}-[ab])}{(\sqrt{x}-[cpier2])(\sqrt{x}+[cpier2])}=\displaystyle{\lim_{x \to [c]}} \frac{-[b]}{\sqrt{x}+[cpier2]}=-\frac{[b]}{[ciper]}$$
\rozwStop
\odpStart
$-\frac{[b]}{[ciper]}$
\odpStop
\testStart
A.$-\frac{[b]}{[ciper]}$
B.$\infty$
C.$0$
D.$-\infty$
\testStop
\kluczStart
A
\kluczStop
\end{document}