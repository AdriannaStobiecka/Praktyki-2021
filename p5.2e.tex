\documentclass[12pt, a4paper]{article}
\usepackage[utf8]{inputenc}
\usepackage{polski}

\usepackage{amsthm}  %pakiet do tworzenia twierdzeń itp.
\usepackage{amsmath} %pakiet do niektórych symboli matematycznych
\usepackage{amssymb} %pakiet do symboli mat., np. \nsubseteq
\usepackage{amsfonts}
\usepackage{graphicx} %obsługa plików graficznych z rozszerzeniem png, jpg
\theoremstyle{definition} %styl dla definicji
\newtheorem{zad}{} 
\title{Multizestaw zadań}
\author{Jacek Jabłoński}
%\date{\today}
\date{}
\newcounter{liczniksekcji}
\newcommand{\kategoria}[1]{\section{#1}} %olreślamy nazwę kateforii zadań
\newcommand{\zadStart}[1]{\begin{zad}#1\newline} %oznaczenie początku zadania
\newcommand{\zadStop}{\end{zad}}   %oznaczenie końca zadania
%Makra opcjonarne (nie muszą występować):
\newcommand{\rozwStart}[2]{\noindent \textbf{Rozwiązanie (autor #1 , recenzent #2): }\newline} %oznaczenie początku rozwiązania, opcjonarnie można wprowadzić informację o autorze rozwiązania zadania i recenzencie poprawności wykonania rozwiązania zadania
\newcommand{\rozwStop}{\newline}                                            %oznaczenie końca rozwiązania
\newcommand{\odpStart}{\noindent \textbf{Odpowiedź:}\newline}    %oznaczenie początku odpowiedzi końcowej (wypisanie wyniku)
\newcommand{\odpStop}{\newline}                                             %oznaczenie końca odpowiedzi końcowej (wypisanie wyniku)
\newcommand{\testStart}{\noindent \textbf{Test:}\newline} %ewentualne możliwe opcje odpowiedzi testowej: A. ? B. ? C. ? D. ? itd.
\newcommand{\testStop}{\newline} %koniec wprowadzania odpowiedzi testowych
\newcommand{\kluczStart}{\noindent \textbf{Test poprawna odpowiedź:}\newline} %klucz, poprawna odpowiedź pytania testowego (jedna literka): A lub B lub C lub D itd.
\newcommand{\kluczStop}{\newline} %koniec poprawnej odpowiedzi pytania testowego 
\newcommand{\wstawGrafike}[2]{\begin{figure}[h] \includegraphics[scale=#2] {#1} \end{figure}} %gdyby była potrzeba wstawienia obrazka, parametry: nazwa pliku, skala (jak nie wiesz co wpisać, to wpisz 1)

\begin{document}
\maketitle


\kategoria{Wikieł/P5.2e}
\zadStart{Zadanie z Wikieł P 5.2e) moja wersja nr [nrWersji]}
%[p1]:[2,4,8,16,32,64,128,256,512,1024]
%[p2]:[2,4,8,16,32,64,128,256,512,1024]
%[a]=random.randint(2,12)
%[b]=random.randint(2,12)
%[c]=random.randint(2,12)
%[d]=random.randint(2,10)
%[f]=random.randint(2,10)
%[g]=random.randint(2,10)
Korzystając z podanych wzorów i twierdzeń, wyznacz pochodną funkcji:
e) $$f(x)=[a]e^{x}cos([c]x)$$
\zadStop
\rozwStart{Jacek Jabłoński}{}
$$f(x)' = ([a]e^{x} \cdot cos([c]x))' = ([a]e^{x})' cos([c]x) + [a]e^{x} (cos([c]x))' = $$
$$= [a]e^{x} cos([c]x) + [a]e^{x} (-[c]sin([c]x)) = [a]e^{x} (cos([c]x) - [c]sin([c]x)) = $$
$$= -[a]e^{x} ([c]sin([c]x) - cos([c]x))$$
\rozwStop
\odpStart
$$-[a]e^{x} ([c]sin([c]x) - cos([c]x))$$
\odpStop
\testStart
A. $$-[a]e^{x} ([c]sin([c]x) - cos([c]x))$$
B. $$-[a]e^{x} ([c]sin([c]x) + cos([c]x))$$
C. $$ [a]e^{x} (2cos([c]x))$$
D. $$ [a]e^{x} (2sin([c]x))$$
E. $$ [g]e^{[f]x} (sin([c]x) + [c]cos([c]x))$$
F. $$ [g]e^{[f]x} (sin([c]x) - [c]cos([c]x))$$
G. $$ [f]e^{[g]x} (2cos([c]x))$$
H. $$ [f]e^{[g]x} (2sin([c]x))$$
I. $$ [b]e^{x}$$
\testStop
\kluczStart
A
\kluczStop



\end{document}