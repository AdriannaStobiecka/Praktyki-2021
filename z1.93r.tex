\documentclass[12pt, a4paper]{article}
\usepackage[utf8]{inputenc}
\usepackage{polski}

\usepackage{amsthm}  %pakiet do tworzenia twierdzeń itp.
\usepackage{amsmath} %pakiet do niektórych symboli matematycznych
\usepackage{amssymb} %pakiet do symboli mat., np. \nsubseteq
\usepackage{amsfonts}
\usepackage{graphicx} %obsługa plików graficznych z rozszerzeniem png, jpg
\theoremstyle{definition} %styl dla definicji
\newtheorem{zad}{} 
\title{Multizestaw zadań}
\author{Robert Fidytek}
%\date{\today}
\date{}
\newcounter{liczniksekcji}
\newcommand{\kategoria}[1]{\section{#1}} %olreślamy nazwę kateforii zadań
\newcommand{\zadStart}[1]{\begin{zad}#1\newline} %oznaczenie początku zadania
\newcommand{\zadStop}{\end{zad}}   %oznaczenie końca zadania
%Makra opcjonarne (nie muszą występować):
\newcommand{\rozwStart}[2]{\noindent \textbf{Rozwiązanie (autor #1 , recenzent #2): }\newline} %oznaczenie początku rozwiązania, opcjonarnie można wprowadzić informację o autorze rozwiązania zadania i recenzencie poprawności wykonania rozwiązania zadania
\newcommand{\rozwStop}{\newline}                                            %oznaczenie końca rozwiązania
\newcommand{\odpStart}{\noindent \textbf{Odpowiedź:}\newline}    %oznaczenie początku odpowiedzi końcowej (wypisanie wyniku)
\newcommand{\odpStop}{\newline}                                             %oznaczenie końca odpowiedzi końcowej (wypisanie wyniku)
\newcommand{\testStart}{\noindent \textbf{Test:}\newline} %ewentualne możliwe opcje odpowiedzi testowej: A. ? B. ? C. ? D. ? itd.
\newcommand{\testStop}{\newline} %koniec wprowadzania odpowiedzi testowych
\newcommand{\kluczStart}{\noindent \textbf{Test poprawna odpowiedź:}\newline} %klucz, poprawna odpowiedź pytania testowego (jedna literka): A lub B lub C lub D itd.
\newcommand{\kluczStop}{\newline} %koniec poprawnej odpowiedzi pytania testowego 
\newcommand{\wstawGrafike}[2]{\begin{figure}[h] \includegraphics[scale=#2] {#1} \end{figure}} %gdyby była potrzeba wstawienia obrazka, parametry: nazwa pliku, skala (jak nie wiesz co wpisać, to wpisz 1)

\begin{document}
\maketitle


\kategoria{Wikieł/Z1.93r}
\zadStart{Zadanie z Wikieł Z 1.93 r) moja wersja nr [nrWersji]}
%[a]:[2,3,4,5,6,7,8,9,10,11,12,13,14,15,16,17,18,19,20]
%[b]:[2,3,4,5,6,7,8,9,10,11,12]
%[c]=[a]-[b]
%[d]=[b]**2-4*[c]
%[pr2]=(pow([d],(1/2)))
%[pr1]=[pr2].real
%[pr]=int([pr1])
%[zz1]=((-[b]-[pr])/2)
%[zz2]=((-[b]+[pr])/2)
%[z1]=int([zz1])
%[z2]=int([zz2])
%[abs1]=abs([z1])
%[abs2]=abs([z2])
%[w1]=(pow(10,[abs1]))
%[w2]=(pow(10,[abs2]))
%[a]>[b] and [d]>0 and [pr2].is_integer()==True and [z1]<0 and [z2]<0 and math.gcd((-[b]-[pr]),2)==2 and math.gcd((-[b]+[pr]),2)==2
Rozwiązać równanie $\log^2{x}+[b]\log{10x} =[a]$
\zadStop
\rozwStart{Małgorzata Ugowska}{}
$$\log^2{x}+[b]\log{10x} =[a] \quad \Longleftrightarrow \quad \log^2{x}+[b](\log{x}+\log{10}) =[a] $$
$$ \quad \Longleftrightarrow \quad \log^2{x}+[b](\log{x}+1) =[a] \quad \Longleftrightarrow \quad \log^2{x}+[b]\log{x}-[c]=0$$
Podstawmy: $y=\log{x}$\\
Mamy wtedy:
$$y^2 +[b]y-[c]=0$$
$$ \bigtriangleup = [b]^2-4 \cdot [c] = [d] \Longrightarrow y_1=\frac{-[b]-\sqrt{\bigtriangleup}}{2} = [z1], \quad y_2=\frac{-[b]+\sqrt{\bigtriangleup}}{2} = [z2]$$
dla $y=[z1]$ mamy:
$$\log{x} = [z1] \quad \Longleftrightarrow \quad 10^{[z1]}=x \quad \Longleftrightarrow \quad x=\frac{1}{[w1]}$$
dla $y=[z2]$ mamy:
$$\log{x} = [z2] \quad \Longleftrightarrow \quad 10^{[z2]}=x \quad \Longleftrightarrow \quad x=\frac{1}{[w2]}$$
\rozwStop
\odpStart
$x \in \{\frac{1}{[w1]}, \frac{1}{[w2]}\}$
\odpStop
\testStart
A. $x \in \{\frac{1}{[w1]}, \frac{1}{[w2]}\}$\\
B. $x \in \{-4, 2\}$\\
C. $x \in \{[abs1], [abs2]\}$\\
D. $x \in \{[z1], [z2]\}$\\
E. $x \in \{[w1], [w2]\}$
\testStop
\kluczStart
A
\kluczStop



\end{document}