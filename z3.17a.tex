\documentclass[12pt, a4paper]{article}
\usepackage[utf8]{inputenc}
\usepackage{polski}

\usepackage{amsthm}  %pakiet do tworzenia twierdzeń itp.
\usepackage{amsmath} %pakiet do niektórych symboli matematycznych
\usepackage{amssymb} %pakiet do symboli mat., np. \nsubseteq
\usepackage{amsfonts}
\usepackage{graphicx} %obsługa plików graficznych z rozszerzeniem png, jpg
\theoremstyle{definition} %styl dla definicji
\newtheorem{zad}{} 
\title{Multizestaw zadań}
\author{Robert Fidytek}
%\date{\today}
\date{}
\newcounter{liczniksekcji}
\newcommand{\kategoria}[1]{\section{#1}} %olreślamy nazwę kateforii zadań
\newcommand{\zadStart}[1]{\begin{zad}#1\newline} %oznaczenie początku zadania
\newcommand{\zadStop}{\end{zad}}   %oznaczenie końca zadania
%Makra opcjonarne (nie muszą występować):
\newcommand{\rozwStart}[2]{\noindent \textbf{Rozwiązanie (autor #1 , recenzent #2): }\newline} %oznaczenie początku rozwiązania, opcjonarnie można wprowadzić informację o autorze rozwiązania zadania i recenzencie poprawności wykonania rozwiązania zadania
\newcommand{\rozwStop}{\newline}                                            %oznaczenie końca rozwiązania
\newcommand{\odpStart}{\noindent \textbf{Odpowiedź:}\newline}    %oznaczenie początku odpowiedzi końcowej (wypisanie wyniku)
\newcommand{\odpStop}{\newline}                                             %oznaczenie końca odpowiedzi końcowej (wypisanie wyniku)
\newcommand{\testStart}{\noindent \textbf{Test:}\newline} %ewentualne możliwe opcje odpowiedzi testowej: A. ? B. ? C. ? D. ? itd.
\newcommand{\testStop}{\newline} %koniec wprowadzania odpowiedzi testowych
\newcommand{\kluczStart}{\noindent \textbf{Test poprawna odpowiedź:}\newline} %klucz, poprawna odpowiedź pytania testowego (jedna literka): A lub B lub C lub D itd.
\newcommand{\kluczStop}{\newline} %koniec poprawnej odpowiedzi pytania testowego 
\newcommand{\wstawGrafike}[2]{\begin{figure}[h] \includegraphics[scale=#2] {#1} \end{figure}} %gdyby była potrzeba wstawienia obrazka, parametry: nazwa pliku, skala (jak nie wiesz co wpisać, to wpisz 1)

\begin{document}
\maketitle


\kategoria{Wikieł/Z3.17a}
\zadStart{Zadanie z Wikieł Z 3.17 a) moja wersja nr [nrWersji]}
%[a]:[3,4,5,6,7,8,9,10,11,12,13,14,15,16,17,18,19,20]
%[a1]=[a]-1
%[a2]=[a]**2
%[reszta]=[a2]%[a1]
%[calosci]=[a2]//[a1]
%[a2]/[a1]>1
Obliczyć poniższą sumę
$$[a]+1+\frac{1}{[a]}+\frac{1}{[a2]}+\cdots.$$
\zadStop
\rozwStart{Adrianna Stobiecka}{}
Zauważamy, że liczby $[a]$, $1$, $\frac{1}{[a]}$, $\frac{1}{[a2]}$, $\dots$ są kolejnymi wyrazami ciągu geometrycznego o pierwszym wyrazie $a_1=[a]$ oraz ilorazie $q=\frac{1}{[a]}$. Widzimy, że 
$$|q|=\bigg|\frac{1}{[a]}\bigg|=\frac{1}{[a]}<1.$$
Zatem powyższą sumę obliczymy jako sumę nieskończonego ciągu geometrycznego.
$$S=\frac{a_1}{1-q}=\frac{[a]}{1-\frac{1}{[a]}}=\frac{[a]}{\frac{[a1]}{[a]}}=[a]\cdot\frac{[a]}{[a1]}=\frac{[a2]}{[a1]}=[calosci]\frac{[reszta]}{[a1]}$$
Otrzymyjemy zatem, że suma jest równa $[calosci]\frac{[reszta]}{[a1]}$.
\rozwStop
\odpStart
$[calosci]\frac{[reszta]}{[a1]}$
\odpStop
\testStart
A.$[calosci]\frac{[reszta]}{[a1]}$
B.$[a1]$
C.$[a]$
D.$[calosci]\frac{[reszta]}{[a]}$
E.$\frac{1}{[a]}$
F.$[a2]$
G.$\frac{1}{[a1]}$
H.$[calosci]\frac{[reszta]}{[a2]}$
I.$\frac{1}{[a2]}$
\testStop
\kluczStart
A
\kluczStop



\end{document}
