\documentclass[12pt, a4paper]{article}
\usepackage[utf8]{inputenc}
\usepackage{polski}

\usepackage{amsthm}  %pakiet do tworzenia twierdzeń itp.
\usepackage{amsmath} %pakiet do niektórych symboli matematycznych
\usepackage{amssymb} %pakiet do symboli mat., np. \nsubseteq
\usepackage{amsfonts}
\usepackage{graphicx} %obsługa plików graficznych z rozszerzeniem png, jpg
\theoremstyle{definition} %styl dla definicji
\newtheorem{zad}{} 
\title{Multizestaw zadań}
\author{Robert Fidytek}
%\date{\today}
\date{}
\newcounter{liczniksekcji}
\newcommand{\kategoria}[1]{\section{#1}} %olreślamy nazwę kateforii zadań
\newcommand{\zadStart}[1]{\begin{zad}#1\newline} %oznaczenie początku zadania
\newcommand{\zadStop}{\end{zad}}   %oznaczenie końca zadania
%Makra opcjonarne (nie muszą występować):
\newcommand{\rozwStart}[2]{\noindent \textbf{Rozwiązanie (autor #1 , recenzent #2): }\newline} %oznaczenie początku rozwiązania, opcjonarnie można wprowadzić informację o autorze rozwiązania zadania i recenzencie poprawności wykonania rozwiązania zadania
\newcommand{\rozwStop}{\newline}                                            %oznaczenie końca rozwiązania
\newcommand{\odpStart}{\noindent \textbf{Odpowiedź:}\newline}    %oznaczenie początku odpowiedzi końcowej (wypisanie wyniku)
\newcommand{\odpStop}{\newline}                                             %oznaczenie końca odpowiedzi końcowej (wypisanie wyniku)
\newcommand{\testStart}{\noindent \textbf{Test:}\newline} %ewentualne możliwe opcje odpowiedzi testowej: A. ? B. ? C. ? D. ? itd.
\newcommand{\testStop}{\newline} %koniec wprowadzania odpowiedzi testowych
\newcommand{\kluczStart}{\noindent \textbf{Test poprawna odpowiedź:}\newline} %klucz, poprawna odpowiedź pytania testowego (jedna literka): A lub B lub C lub D itd.
\newcommand{\kluczStop}{\newline} %koniec poprawnej odpowiedzi pytania testowego 
\newcommand{\wstawGrafike}[2]{\begin{figure}[h] \includegraphics[scale=#2] {#1} \end{figure}} %gdyby była potrzeba wstawienia obrazka, parametry: nazwa pliku, skala (jak nie wiesz co wpisać, to wpisz 1)

\begin{document}
\maketitle
\kategoria{Wikieł/C1.1h}
\zadStart{Zadanie z Wikieł C 1.1h moja wersja nr [nrWersji]}
%[a]:[2,4,6,8,10,12,14]
%[b]=[a]**4
%[c]=[a]**2
%[d]=int([c]/2)
%[ac]=[a]*[c]
Oblicz całkę $$\int \frac{x^{4}-[b]}{x+[a]} dx.$$
\zadStop
\rozwStart{Justyna Chojecka}{}
$$\int \frac{x^{4}-[b]}{x+[a]} dx=\int \frac{(x^{2}-[c])(x^{2}+[c])}{x+[a]}dx=\int \frac{(x-[a])(x+[a])(x^{2}+[c])}{x+[a]}dx$$$$=\int (x-[a])(x^{2}+[c])dx=\int x^{3}-[a]x^{2}+[c]x-[ac]dx$$$$=\int x^{3}dx-\int [a]x^{2}dx+\int[c]xdx-\int [ac]dx=\frac{x^{4}}{4}-\frac{[a]x^{3}}{3}+[d]x^{2}-[ac]x+C$$
\rozwStop
\odpStart
$\frac{x^{4}}{4}-\frac{[a]x^{3}}{3}+[d]x^{2}-[ac]x+C$
\odpStop
\testStart
A.$\frac{x^{4}}{4}-\frac{[a]x^{3}}{3}+[d]x^{2}-[ac]x+C$\\
B.$\frac{x^{4}}{4}-\frac{[a]x^{3}}{3}+[d]x^{2}+[ac]x+C$\\
C.$\frac{x^{4}}{4}+\frac{[a]x^{3}}{3}+[d]x^{2}-[ac]x+C$\\
D.$\frac{x^{4}}{4}-\frac{[a]x^{3}}{3}-[d]x^{2}-[ac]x+C$\\
E.$-\frac{x^{4}}{4}-\frac{[a]x^{3}}{3}+[d]x^{2}-[ac]x+C$\\
F.$\frac{x^{4}}{4}+\frac{[a]x^{3}}{3}+[d]x^{2}+[ac]x+C$\\
G.$\frac{5x^{4}}{4}-\frac{[a]x^{3}}{3}+[d]x^{2}-[ac]x+C$\\
H.$\frac{3x^{4}}{4}-\frac{[a]x^{3}}{3}+[d]x^{2}-[ac]x+C$\\
I.$\frac{5x^{5}}{4}-\frac{[a]x^{3}}{3}+[d]x^{2}-[ac]x+C$\\
\testStop
\kluczStart
A
\kluczStop



\end{document}