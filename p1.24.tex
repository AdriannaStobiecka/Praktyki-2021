\documentclass[12pt, a4paper]{article}
\usepackage[utf8]{inputenc}
\usepackage{polski}

\usepackage{amsthm}  %pakiet do tworzenia twierdzeń itp.
\usepackage{amsmath} %pakiet do niektórych symboli matematycznych
\usepackage{amssymb} %pakiet do symboli mat., np. \nsubseteq
\usepackage{amsfonts}
\usepackage{graphicx} %obsługa plików graficznych z rozszerzeniem png, jpg
\theoremstyle{definition} %styl dla definicji
\newtheorem{zad}{} 
\title{Multizestaw zadań}
\author{Robert Fidytek}
%\date{\today}
\date{}
\newcounter{liczniksekcji}
\newcommand{\kategoria}[1]{\section{#1}} %olreślamy nazwę kateforii zadań
\newcommand{\zadStart}[1]{\begin{zad}#1\newline} %oznaczenie początku zadania
\newcommand{\zadStop}{\end{zad}}   %oznaczenie końca zadania
%Makra opcjonarne (nie muszą występować):
\newcommand{\rozwStart}[2]{\noindent \textbf{Rozwiązanie (autor #1 , recenzent #2): }\newline} %oznaczenie początku rozwiązania, opcjonarnie można wprowadzić informację o autorze rozwiązania zadania i recenzencie poprawności wykonania rozwiązania zadania
\newcommand{\rozwStop}{\newline}                                            %oznaczenie końca rozwiązania
\newcommand{\odpStart}{\noindent \textbf{Odpowiedź:}\newline}    %oznaczenie początku odpowiedzi końcowej (wypisanie wyniku)
\newcommand{\odpStop}{\newline}                                             %oznaczenie końca odpowiedzi końcowej (wypisanie wyniku)
\newcommand{\testStart}{\noindent \textbf{Test:}\newline} %ewentualne możliwe opcje odpowiedzi testowej: A. ? B. ? C. ? D. ? itd.
\newcommand{\testStop}{\newline} %koniec wprowadzania odpowiedzi testowych
\newcommand{\kluczStart}{\noindent \textbf{Test poprawna odpowiedź:}\newline} %klucz, poprawna odpowiedź pytania testowego (jedna literka): A lub B lub C lub D itd.
\newcommand{\kluczStop}{\newline} %koniec poprawnej odpowiedzi pytania testowego 
\newcommand{\wstawGrafike}[2]{\begin{figure}[h] \includegraphics[scale=#2] {#1} \end{figure}} %gdyby była potrzeba wstawienia obrazka, parametry: nazwa pliku, skala (jak nie wiesz co wpisać, to wpisz 1)

\begin{document}
\maketitle


\kategoria{Wikieł/P1.20c}
\zadStart{Zadanie z Wikieł P 1.24  moja wersja nr [nrWersji]}
%[p1]:[4,5,6,7,8,9,10,11,12]
%[p2]:[2,3,4,5,6,7,8,9,10,11,12]
%math.gcd([p1],[p2])==1
Wyznaczyć funkcję odwrotną do funkcji $f(x)=-[p1]x+[p2]$.
\zadStop
\rozwStart{Maja Szabłowska}{}
Funkcja $f(x)=-[p1]x+[p2]$ jest funkcją liniową, a więc $D_{f}=\mathbb{R}$, $W_{f}=\mathbb{R}$ i $f$ jest funkcją wzajemnie jednoznaczną. Zatem $f^{-1}$ istnieje. Wzór funkcji odwrotnej:
$$y=-[p1]x+[p2] \Leftrightarrow  -[p1]x=y-[p2] \Leftrightarrow 
x=\frac{1}{[p1]}y+\frac{[p2]}{[p1]}$$
\rozwStop
\odpStart
$f^{-1}=\frac{1}{[p1]}y+\frac{[p2]}{[p1]}$
\odpStop
\testStart
A.$f^{-1}=\frac{1}{[p1]}y+\frac{[p2]}{[p1]}$
B.$f^{-1}=\frac{1}{[p2]}y+\frac{[p2]}{[p1]}$
C.$f^{-1}=\frac{1}{[p2]}y+[p2]$
D.$f^{-1}=\frac{1}{[p1]}y+[p1]$
E.$f^{-1}=\frac{1}{[p1]}y+[p2]$
F.$f^{-1}=\frac{1}{[p2]}y+[p1]$
G.nie da się wyznaczyć takiej funkcji
H.$f^{-1}=-[p1]x+[p2]$
I.$f^{-1}=-[p2]x+[p1]$
\testStop
\kluczStart
A
\kluczStop



\end{document}