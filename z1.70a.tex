\documentclass[12pt, a4paper]{article}
\usepackage[utf8]{inputenc}
\usepackage{polski}

\usepackage{amsthm}  %pakiet do tworzenia twierdzeń itp.
\usepackage{amsmath} %pakiet do niektórych symboli matematycznych
\usepackage{amssymb} %pakiet do symboli mat., np. \nsubseteq
\usepackage{amsfonts}
\usepackage{graphicx} %obsługa plików graficznych z rozszerzeniem png, jpg
\theoremstyle{definition} %styl dla definicji
\newtheorem{zad}{} 
\title{Multizestaw zadań}
\author{Robert Fidytek}
%\date{\today}
\date{}
\newcounter{liczniksekcji}
\newcommand{\kategoria}[1]{\section{#1}} %olreślamy nazwę kateforii zadań
\newcommand{\zadStart}[1]{\begin{zad}#1\newline} %oznaczenie początku zadania
\newcommand{\zadStop}{\end{zad}}   %oznaczenie końca zadania
%Makra opcjonarne (nie muszą występować):
\newcommand{\rozwStart}[2]{\noindent \textbf{Rozwiązanie (autor #1 , recenzent #2): }\newline} %oznaczenie początku rozwiązania, opcjonarnie można wprowadzić informację o autorze rozwiązania zadania i recenzencie poprawności wykonania rozwiązania zadania
\newcommand{\rozwStop}{\newline}                                            %oznaczenie końca rozwiązania
\newcommand{\odpStart}{\noindent \textbf{Odpowiedź:}\newline}    %oznaczenie początku odpowiedzi końcowej (wypisanie wyniku)
\newcommand{\odpStop}{\newline}                                             %oznaczenie końca odpowiedzi końcowej (wypisanie wyniku)
\newcommand{\testStart}{\noindent \textbf{Test:}\newline} %ewentualne możliwe opcje odpowiedzi testowej: A. ? B. ? C. ? D. ? itd.
\newcommand{\testStop}{\newline} %koniec wprowadzania odpowiedzi testowych
\newcommand{\kluczStart}{\noindent \textbf{Test poprawna odpowiedź:}\newline} %klucz, poprawna odpowiedź pytania testowego (jedna literka): A lub B lub C lub D itd.
\newcommand{\kluczStop}{\newline} %koniec poprawnej odpowiedzi pytania testowego 
\newcommand{\wstawGrafike}[2]{\begin{figure}[h] \includegraphics[scale=#2] {#1} \end{figure}} %gdyby była potrzeba wstawienia obrazka, parametry: nazwa pliku, skala (jak nie wiesz co wpisać, to wpisz 1)

\begin{document}
\maketitle


\kategoria{Wikieł/Z1.68b}
\zadStart{Zadanie z Wikieł Z 1.68 b) moja wersja nr [nrWersji]}
%[a]:[2,3,4,5,6,7]
%[b]:[2,3,4,5,6,7]
%[c]:[2,3,4,5,6,7]
%[a]=random.randint(2,7)
%[b]=random.randint(2,7)
%[c]=random.randint(2,7)
%[a]>[b] and [c]<[b]
Rozwiązać równanie: $\frac{(x+[a])(x-[b])}{(x+[c])^2}\leq0$
\zadStop
\rozwStart{Pascal Nawrocki}{}
Zaczniemy od wyznaczenia dziedziny:
$$(x+[c])^2\neq0\Leftrightarrow x\neq-[c] \Leftrightarrow Df:x\in\mathbb{R}\symbol{92}\{-[c]\}$$
Jako, że mianownik nie ma wpływu na znak ułamka oraz została dziedzina wyznaczona przez nas, to możemy przystąpić do rozwiązania tej nierówności w postaci:
$$(x+[a])(x-[b])\leq0$$
Jak możemy zobaczyć, mamy do czynienia po lewej stronie z funkcją kwadratową w postaci iloczynowej o współczynniku $a=1$ oraz miejscach zerowych $x_1=-[a]$ i $x_2=[b]$. 
Co oznacza, że nasza parabolka jest "uśmiechnięta" czyli skierowana w górę. Teraz możemy na podstawie otrzymanych informacji stwierdzić w jakich przedziałach funkcja ta ma wartości $\leq0$. Okazuje się, że są to przedziały, dla których $x\in(-[a],[b])$. Po sprawdzeniu założenia, ostateczną odpowiedzią będzie:  $x\in(-[a],[b])\symbol{92}\{[c]\}$.
\odpStart
$x\in(-[a],[b])\symbol{92}\{[c]\}$
\odpStop
\testStart
A.$x\in\{-[a],[b]\}\symbol{92}\{[c]\}$
B.$x\in([b],\infty)$
C.$x\in\emptyset$
D.$x\in(-\infty,-[a])$
\testStop
\kluczStart
A
\kluczStop
\end{document}