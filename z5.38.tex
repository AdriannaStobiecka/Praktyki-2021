\documentclass[12pt, a4paper]{article}
\usepackage[utf8]{inputenc}
\usepackage{polski}

\usepackage{amsthm}  %pakiet do tworzenia twierdzeń itp.
\usepackage{amsmath} %pakiet do niektórych symboli matematycznych
\usepackage{amssymb} %pakiet do symboli mat., np. \nsubseteq
\usepackage{amsfonts}
\usepackage{graphicx} %obsługa plików graficznych z rozszerzeniem png, jpg
\theoremstyle{definition} %styl dla definicji
\newtheorem{zad}{} 
\title{Multizestaw zadań}
\author{Robert Fidytek}
%\date{\today}
\date{}
\newcounter{liczniksekcji}
\newcommand{\kategoria}[1]{\section{#1}} %olreślamy nazwę kateforii zadań
\newcommand{\zadStart}[1]{\begin{zad}#1\newline} %oznaczenie początku zadania
\newcommand{\zadStop}{\end{zad}}   %oznaczenie końca zadania
%Makra opcjonarne (nie muszą występować):
\newcommand{\rozwStart}[2]{\noindent \textbf{Rozwiązanie (autor #1 , recenzent #2): }\newline} %oznaczenie początku rozwiązania, opcjonarnie można wprowadzić informację o autorze rozwiązania zadania i recenzencie poprawności wykonania rozwiązania zadania
\newcommand{\rozwStop}{\newline}                                            %oznaczenie końca rozwiązania
\newcommand{\odpStart}{\noindent \textbf{Odpowiedź:}\newline}    %oznaczenie początku odpowiedzi końcowej (wypisanie wyniku)
\newcommand{\odpStop}{\newline}                                             %oznaczenie końca odpowiedzi końcowej (wypisanie wyniku)
\newcommand{\testStart}{\noindent \textbf{Test:}\newline} %ewentualne możliwe opcje odpowiedzi testowej: A. ? B. ? C. ? D. ? itd.
\newcommand{\testStop}{\newline} %koniec wprowadzania odpowiedzi testowych
\newcommand{\kluczStart}{\noindent \textbf{Test poprawna odpowiedź:}\newline} %klucz, poprawna odpowiedź pytania testowego (jedna literka): A lub B lub C lub D itd.
\newcommand{\kluczStop}{\newline} %koniec poprawnej odpowiedzi pytania testowego 
\newcommand{\wstawGrafike}[2]{\begin{figure}[h] \centering \includegraphics[scale=#2] {#1} \end{figure}} %gdyby była potrzeba wstawienia obrazka, parametry: nazwa pliku, skala (jak nie wiesz co wpisać, to wpisz 1)

\begin{document}
\maketitle

\kategoria{Wikieł/Z5.38}

\zadStart{Zadanie z Wikieł Z 5.38 moja wersja nr [nrWersji]}
%[a]:[1,2,3,4,5,6,7,8,9,10]
%[b]=[a]*3
%[c]=[b]*4
Zbadać wypukłość i znaleźć punkty przegięcia wykresu funkcji:
$$y =\frac{[a]}{x^3}$$
\zadStop

\rozwStart{Natalia Danieluk}{}
Postępujemy następująco:
\begin{enumerate}
\item Określamy dziedzinę funkcji: $\quad \mathcal{D}_f=\mathbb{R}\backslash\{0\}$. \\
\item Obliczamy pochodne: 
$$\quad f'(x) = - \frac{[b]}{x^4},$$
$$\quad f''(x) = \frac{[c]}{x^5}$$
i określamy ich dziedziny: $\quad \mathcal{D}_{f'}=\mathcal{D}_{f''}=\mathbb{R}\backslash\{0\}$. \\
\item Badamy znak $f''$. \\
Funkcja nie posiada miejsc zerowych. Należy jeszcze sprawdzić, czy w punkcie nieciągłości, tj. $x=0$ występuje asymptota pionowa. Aby to zrobić liczymy jego granice z prawej i lewej strony:
$$\lim_{x \rightarrow 0^+} f(x) = \lim_{x \rightarrow 0^+} \frac{[a]}{x^3} = \Bigg[ \frac{[a]}{0^{+}}\Bigg] = \infty$$ 
$$\lim_{x \rightarrow 0^-} f(x) = \lim_{x \rightarrow 0^-} \frac{[a]}{x^3} = \Bigg[ \frac{[a]}{0^{-}}\Bigg] = -\infty$$
Otrzymujemy asymptotę obustronną. Wówczas wykres pochodnej wygląda mniej więcej jak poniżej (wartości jedynie zbliżają się do osi OX). \\
\wstawGrafike{wykres_z5_38.png}{0.75}
	\begin{enumerate}
	\item $f''(x) > 0 \Leftrightarrow x \in (0,\infty)$ i w tym przedziale wykres funkcji $f$ jest wypukły (wypukły w dół) $ \smile $ \\
	\item $f''(x) < 0 \Leftrightarrow x \in (-\infty,0)$ i w tym przedziale wykres funkcji $f$ jest wklęsły (wypukły w górę) $ \frown $ \\
	\end{enumerate}
\end{enumerate}
Druga pochodna zmienia znak w sąsiedztwie punktu $x=0$, nie należy on jednak do dziedziny. Wykres funkcji nie ma punktu przegięcia.
\rozwStop

\odpStart
Funkcja jest wypukła w $(0,\infty)$ i wklęsła w $(-\infty,0)$.
\odpStop

\testStart
A. Funkcja jest wypukła w całej dziedzinie.
B. Funkcja jest wklęsła w całej dziedzinie.
C. Funkcja nie jest ani wypukła, ani wklęsła.
D. Funkcja jest wypukła w $(-\infty,0)$ i wklęsła w $(0,\infty)$.
E. Funkcja jest wypukła w $(0,\infty)$ i wklęsła w $(-\infty,0)$.
F. Funkcja jest wypukła w $([a],\infty)$ i wklęsła w $(-\infty,[a])$.
\testStop

\kluczStart
E
\kluczStop

\end{document}