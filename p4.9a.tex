\documentclass[12pt, a4paper]{article}
\usepackage[utf8]{inputenc}
\usepackage{polski}

\usepackage{amsthm}  %pakiet do tworzenia twierdzeń itp.
\usepackage{amsmath} %pakiet do niektórych symboli matematycznych
\usepackage{amssymb} %pakiet do symboli mat., np. \nsubseteq
\usepackage{amsfonts}
\usepackage{graphicx} %obsługa plików graficznych z rozszerzeniem png, jpg
\theoremstyle{definition} %styl dla definicji
\newtheorem{zad}{} 
\title{Multizestaw zadań}
\author{Radosław Grzyb}
%\date{\today}
\date{}
\newcounter{liczniksekcji}
\newcommand{\kategoria}[1]{\section{#1}} %olreślamy nazwę kateforii zadań
\newcommand{\zadStart}[1]{\begin{zad}#1\newline} %oznaczenie początku zadania
\newcommand{\zadStop}{\end{zad}}   %oznaczenie końca zadania
%Makra opcjonarne (nie muszą występować):
\newcommand{\rozwStart}[2]{\noindent \textbf{Rozwiązanie (autor #1 , recenzent #2): }\newline} %oznaczenie początku rozwiązania, opcjonarnie można wprowadzić informację o autorze rozwiązania zadania i recenzencie poprawności wykonania rozwiązania zadania
\newcommand{\rozwStop}{\newline}                                            %oznaczenie końca rozwiązania
\newcommand{\odpStart}{\noindent \textbf{Odpowiedź:}\newline}    %oznaczenie początku odpowiedzi końcowej (wypisanie wyniku)
\newcommand{\odpStop}{\newline}                                             %oznaczenie końca odpowiedzi końcowej (wypisanie wyniku)
\newcommand{\testStart}{\noindent \textbf{Test:}\newline} %ewentualne możliwe opcje odpowiedzi testowej: A. ? B. ? C. ? D. ? itd.
\newcommand{\testStop}{\newline} %koniec wprowadzania odpowiedzi testowych
\newcommand{\kluczStart}{\noindent \textbf{Test poprawna odpowiedź:}\newline} %klucz, poprawna odpowiedź pytania testowego (jedna literka): A lub B lub C lub D itd.
\newcommand{\kluczStop}{\newline} %koniec poprawnej odpowiedzi pytania testowego 
\newcommand{\wstawGrafike}[2]{\begin{figure}[h] \includegraphics[scale=#2] {#1} \end{figure}} %gdyby była potrzeba wstawienia obrazka, parametry: nazwa pliku, skala (jak nie wiesz co wpisać, to wpisz 1)

\begin{document}
\maketitle


\kategoria{Wikieł/P4.9a}
\zadStart{Zadanie z Wikieł P 4.9 a) moja wersja nr [nrWersji]}
%[p1]:[2,3,4,5,6,7,8,9,10,11,12,13,14,15]
%[a]=random.randint(1,5)
%[b]:[1,2,3,4,5,6,7,8,9]
%[c]:[1,2,3,4,5,6,7,8,9]
%[d]=[a]*[p1]
%[f]=int(([d]*[b])/[p1])
%[g]=int(([d]*[c])/[p1])
%[h]=int([f]-[g])
Obliczyć granicę.
$$\lim_{x\to\infty}\left(\frac{[p1]x+[b]}{[p1]x+[c]}\right)^{[d]x}$$
\zadStop
\rozwStart{Klaudia Klejdysz}{}
$$\lim_{x\to\infty}\left(\frac{[p1]x+[b]}{[p1]x+[c]}\right)^{[d]x}=\lim_{x\to\infty}\left(\frac{[p1]x(1+\frac{[b]}{[p1]x})}{[p1]x(1+\frac{[c]}{[p1]x})}\right)^{[d]x}=\lim_{x\to\infty}\left(\frac{1+\frac{[b]}{[p1]x}}{1+\frac{[c]}{[p1]x}}\right)^{[d]x}=$$
$$=\lim_{x\to\infty}\frac{((1+\frac{[b]}{[p1]x})^{\frac{[p1]x}{[b]}})^{[f]}}{((1+\frac{[c]}{[p1]x})^{\frac{[p1]x}{[c]}})^{[g]}}=\frac{e^{[f]}}{e^{[g]}}=e^{[h]}$$
\rozwStop
\odpStart
$e^{[h]}$
\odpStop
\testStart
A.$e^{[h]}$
B.$[h]$
C.$e^{[f]}$
D.$[h]x$
E.$e^{[g]}$
F.$[p1]x$
\testStop
\kluczStart
A
\kluczStop



\end{document}