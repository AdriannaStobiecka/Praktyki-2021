\documentclass[12pt, a4paper]{article}
\usepackage[utf8]{inputenc}
\usepackage{polski}

\usepackage{amsthm}  %pakiet do tworzenia twierdzeń itp.
\usepackage{amsmath} %pakiet do niektórych symboli matematycznych
\usepackage{amssymb} %pakiet do symboli mat., np. \nsubseteq
\usepackage{amsfonts}
\usepackage{graphicx} %obsługa plików graficznych z rozszerzeniem png, jpg
\theoremstyle{definition} %styl dla definicji
\newtheorem{zad}{} 
\title{Multizestaw zadań}
\author{Robert Fidytek}
%\date{\today}
\date{}
\newcounter{liczniksekcji}
\newcommand{\kategoria}[1]{\section{#1}} %olreślamy nazwę kateforii zadań
\newcommand{\zadStart}[1]{\begin{zad}#1\newline} %oznaczenie początku zadania
\newcommand{\zadStop}{\end{zad}}   %oznaczenie końca zadania
%Makra opcjonarne (nie muszą występować):
\newcommand{\rozwStart}[2]{\noindent \textbf{Rozwiązanie (autor #1 , recenzent #2): }\newline} %oznaczenie początku rozwiązania, opcjonarnie można wprowadzić informację o autorze rozwiązania zadania i recenzencie poprawności wykonania rozwiązania zadania
\newcommand{\rozwStop}{\newline}                                            %oznaczenie końca rozwiązania
\newcommand{\odpStart}{\noindent \textbf{Odpowiedź:}\newline}    %oznaczenie początku odpowiedzi końcowej (wypisanie wyniku)
\newcommand{\odpStop}{\newline}                                             %oznaczenie końca odpowiedzi końcowej (wypisanie wyniku)
\newcommand{\testStart}{\noindent \textbf{Test:}\newline} %ewentualne możliwe opcje odpowiedzi testowej: A. ? B. ? C. ? D. ? itd.
\newcommand{\testStop}{\newline} %koniec wprowadzania odpowiedzi testowych
\newcommand{\kluczStart}{\noindent \textbf{Test poprawna odpowiedź:}\newline} %klucz, poprawna odpowiedź pytania testowego (jedna literka): A lub B lub C lub D itd.
\newcommand{\kluczStop}{\newline} %koniec poprawnej odpowiedzi pytania testowego 
\newcommand{\wstawGrafike}[2]{\begin{figure}[h] \includegraphics[scale=#2] {#1} \end{figure}} %gdyby była potrzeba wstawienia obrazka, parametry: nazwa pliku, skala (jak nie wiesz co wpisać, to wpisz 1)

\begin{document}
\maketitle


\kategoria{Wikieł/Z1.68b}
\zadStart{Zadanie z Wikieł Z 1.68 b) moja wersja nr [nrWersji]}
%[a]:[2,3,4,5,6,7]
%[b]:[2,3,4,5,6,7]
%[c]:[2,3,4,5,6,7]
%[a]=random.randint(2,12)
%[b]=random.randint(2,12)
%[c]=random.randint(1,12)
%[-b]=(-1)*[b]
%[bkw]=[b]*[b]
%[e]=[a]+[b]
%[f]=([a]*[b]+[c])
%[g]=4*[f]
%[delta]=[e]*[e]+[g]
%[pierw]=(pow([delta],1/2))
%[pierw1]=[pierw].real
%[pierw2]=int([pierw1])
%[x1]=int(((-1)*[e]+[pierw2])/2)
%[x2]=int(((-1)*[e]-[pierw2])/2)
%[delta]>0 and [pierw].is_integer()==True and [x1]!=[b] and [x1]!=[-b] and [x2]!=[b] and [x2]!=[-b]
Rozwiązać równanie: $\frac{[a]}{x+[b]}+\frac{x}{x-[b]}=\frac{[c]}{x^2-[bkw]}$
\zadStop
\rozwStart{Pascal Nawrocki}{}
Zaczynamy od wyznaczania dziedziny:
\begin{enumerate}
\item $x+[b]\neq0 \Leftrightarrow x\neq-[b]$
\item $x-[b] \neq0 \Leftrightarrow x\neq[b]$
\item $x^2-[bkw]\neq0 \Leftrightarrow x\neq[b] \vee x\neq-[b]$
\end{enumerate}
Stąd otrzymujemy dziedzinę $x\in\mathbb{R}\symbol{92}\{-[b],[b]\}$
Teraz możemy zabrać się za rozwiązywanie równania:
$$\frac{[a]}{x+[b]}+\frac{x}{x-[b]}=\frac{[c]}{x^2-[bkw]}$$
$$\frac{[a](x-[b])+x(x+[b])}{x^2-[bkw]}-\frac{[c]}{x^2-[bkw]}=0$$
$$\frac{[a](x-[b])+x(x+[b])-[c]}{x^2-[bkw]}=0$$
Zauważmy, że dane równanie jest spełnione w naszej dziedzinie wtedy i tylko wtedy, gdy nasz licznik = 0. Stąd:
$$x^2+[e]x-[f]=0$$
Liczymy deltę:
$$\Delta=[e]^2+4\cdot[f]=[delta]$$
$$\sqrt{\Delta}=\sqrt{[delta]}=[pierw2]$$
$$x_1=\frac{-[e]+[pierw2]}{2}=[x1]\in\text{Dziedziny} \vee x_2=\frac{-[e]-[pierw2]}{2}=[x2]\in\text{Dziedziny}$$
\odpStart
$x=[x1] \vee x=[x2]$
\odpStop
\testStart
A.$x=[x1] \vee x=[x2]$
B.$x=[x1]$
C.$x\in\emptyset$
D.$x=[x2]$
\testStop
\kluczStart
A
\kluczStop
\end{document}