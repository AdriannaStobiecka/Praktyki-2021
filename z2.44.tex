\documentclass[12pt, a4paper]{article}
\usepackage[utf8]{inputenc}
\usepackage{polski}

\usepackage{amsthm}  %pakiet do tworzenia twierdzeń itp.
\usepackage{amsmath} %pakiet do niektórych symboli matematycznych
\usepackage{amssymb} %pakiet do symboli mat., np. \nsubseteq
\usepackage{amsfonts}
\usepackage{graphicx} %obsługa plików graficznych z rozszerzeniem png, jpg
\theoremstyle{definition} %styl dla definicji
\newtheorem{zad}{} 
\title{Multizestaw zadań}
\author{Robert Fidytek}
%\date{\today}
\date{}\documentclass[12pt, a4paper]{article}
\usepackage[utf8]{inputenc}
\usepackage{polski}

\usepackage{amsthm}  %pakiet do tworzenia twierdzeń itp.
\usepackage{amsmath} %pakiet do niektórych symboli matematycznych
\usepackage{amssymb} %pakiet do symboli mat., np. \nsubseteq
\usepackage{amsfonts}
\usepackage{graphicx} %obsługa plików graficznych z rozszerzeniem png, jpg
\theoremstyle{definition} %styl dla definicji
\newtheorem{zad}{} 
\title{Multizestaw zadań}
\author{Robert Fidytek}
%\date{\today}
\date{}
\newcounter{liczniksekcji}
\newcommand{\kategoria}[1]{\section{#1}} %olreślamy nazwę kateforii zadań
\newcommand{\zadStart}[1]{\begin{zad}#1\newline} %oznaczenie początku zadania
\newcommand{\zadStop}{\end{zad}}   %oznaczenie końca zadania
%Makra opcjonarne (nie muszą występować):
\newcommand{\rozwStart}[2]{\noindent \textbf{Rozwiązanie (autor #1 , recenzent #2): }\newline} %oznaczenie początku rozwiązania, opcjonarnie można wprowadzić informację o autorze rozwiązania zadania i recenzencie poprawności wykonania rozwiązania zadania
\newcommand{\rozwStop}{\newline}                                            %oznaczenie końca rozwiązania
\newcommand{\odpStart}{\noindent \textbf{Odpowiedź:}\newline}    %oznaczenie początku odpowiedzi końcowej (wypisanie wyniku)
\newcommand{\odpStop}{\newline}                                             %oznaczenie końca odpowiedzi końcowej (wypisanie wyniku)
\newcommand{\testStart}{\noindent \textbf{Test:}\newline} %ewentualne możliwe opcje odpowiedzi testowej: A. ? B. ? C. ? D. ? itd.
\newcommand{\testStop}{\newline} %koniec wprowadzania odpowiedzi testowych
\newcommand{\kluczStart}{\noindent \textbf{Test poprawna odpowiedź:}\newline} %klucz, poprawna odpowiedź pytania testowego (jedna literka): A lub B lub C lub D itd.
\newcommand{\kluczStop}{\newline} %koniec poprawnej odpowiedzi pytania testowego 
\newcommand{\wstawGrafike}[2]{\begin{figure}[h] \includegraphics[scale=#2] {#1} \end{figure}} %gdyby była potrzeba wstawienia obrazka, parametry: nazwa pliku, skala (jak nie wiesz co wpisać, to wpisz 1)

\begin{document}
\maketitle


\kategoria{Wikieł/Z2.44}
\zadStart{Zadanie z Wikieł Z 2.44  moja wersja nr [nrWersji]}
%[p1]:[2,3,4,5,6,7,8,9,10]
%[p2]:[2,3,4,5,6,7,8,9,10]
%[p3]=random.randint(2,10)
%[p4]:[2,3,4,5,6,7,8,9,10]
%[p5]=random.randint(2,10)
%[p6]=random.randint(2,10)
%[p7]=random.randint(2,10)
%[p1p5]=[p1]*[p5]
%[p2p4]=[p2]*[p4]
%[p2p7]=[p2]*[p7]
%[p1p6]=[p1]*[p6]
%[p1p7]=[p1]*[p7]
%[p3p6]=[p3]*[p6]
%[p3p4]=[p3]*[p4]
%[p3p5]=[p3]*[p5]
%[p2p6]=[p2]*[p6]
%[w]=-[p1p5]-[p2p4]
%[wx]=-[p3p5]+[p2p6]
%[wy]=-[p1p6]-[p3p4]
%math.gcd([p2p7],[wx])==1 and math.gcd([p1p7],[wy])==1 and [wx]<0 and [wy]<0 and [w]<0 and [p2p7]/[wx]<[p1p7]/[wy]

Podać, dla jakich wartości parametru $m$ rozwiązaniem układu równań
$$\left\{\begin{array}{ccc}
[p1]x+[p2]y&=&[p3]m\\
[p4]x-[p5]y&=&-[p6]m+[p7]
\end{array} \right.$$
jest para liczb: dodatnich, ujemnych,o różnych znakach. 
\zadStop

\rozwStart{Maja Szabłowska}{}
Powyższemu układowi równań odpowiadają wyznaczniki:
$$W=\left| \begin{array}{lccr} [p1] & [p2] \\ [p4] & -[p5] \end{array}\right| = [p1]\cdot(-[p5]) - [p2]\cdot[p4]=-[p1p5]-[p2p4]=[w]$$

$$W_{x}=\left| \begin{array}{lccr} [p3]m & [p2] \\ -[p6]m+[p7] & -[p5] \end{array}\right| = [p3]m\cdot(-[p5]) - [p2]\cdot(-[p6]m+[p7])=-[p3p5]m+[p2p6]m-[p2p7]=[wx]m-[p2p7]$$

$$W_{y}=\left| \begin{array}{lccr} [p1] & [p3]m \\ [p4] & -[p6]m+[p7] \end{array}\right| = [p1]\cdot(-[p6]m+[p7]) - [p3]m\cdot[p4]=-[p1p6]m+[p1p7]-[p3p4]m=[p1p7][wy]m$$

$$W\neq 0 \iff [w] \neq 0, m\in\mathbb{R} $$

$$x=\frac{W_{x}}{W}=\frac{[wx]m-[p2p7]}{[w]}$$

$$y=\frac{W_{y}}{W}=\frac{[p1p7][wy]m}{[w]}$$

\begin{enumerate}
    \item Para liczb dodatnich.
    $$x=\frac{[wx]m-[p2p7]}{[w]}>0\quad \land \quad  y=\frac{[p1p7][wy]m}{[w]}>0$$
    
    $$[wx]m-[p2p7]<0 \quad \land \quad [p1p7][wy]m<0$$
    
    $$[wx]m<[p2p7] \quad \land \quad [wy]m<[p1p7]$$
    
    $$m>\frac{[p2p7]}{[wx]} \quad \land \quad m>\frac{[p1p7]}{[wy]}$$
    
    $$m\in\left(\frac{[p1p7]}{[wy]} ,\infty\right)$$
    
    \item Para liczb ujemnych.
    $$x=\frac{[wx]m-[p2p7]}{[w]}<0\quad \land \quad  y=\frac{[p1p7][wy]m}{[w]}<0$$
    
    $$[wx]m-[p2p7]>0 \quad \land \quad [p1p7][wy]m>0$$
    
     $$[wx]m>[p2p7] \quad \land \quad [wy]m>[p1p7]$$
     
      $$m<\frac{[p2p7]}{[wx]} \quad \land \quad m<\frac{[p1p7]}{[wy]}$$
      
       $$m\in\left(-\infty,\frac{[p2p7]}{[wx]} \right) $$
      
      \item Różne znaki.
      $$x\cdot y <0 \iff \frac{[wx]m-[p2p7]}{[w]} \cdot \frac{[p1p7][wy]m}{[w]}<0$$
      
      $$\frac{([wx]m-[p2p7])([p1p7][wy]m)}{([w])^{2}}<0 $$
      $$([wx]m-[p2p7])([p1p7][wy]m)<0$$
      $$m_{1}=\frac{[p2p7]}{[wx]}, m_{2}=\frac{[p1p7]}{[wy]} $$
      $$m\in\left(-\infty,\frac{[p2p7]}{[wx]} \right) \cup \left(\frac{[p1p7]}{[wy]}  ,\infty\right)$$
    
\end{enumerate}

\rozwStop
\odpStart
Dodatnie dla $m\in\left(\frac{[p1p7]}{[wy]} ,\infty\right)$, ujemne dla $m\in\left(-\infty,\frac{[p2p7]}{[wx]} \right) $, różne dla $m\in\left(-\infty,\frac{[p2p7]}{[wx]} \right) \cup \left(\frac{[p1p7]}{[wy]}  ,\infty\right)$.
\odpStop
\testStart
A.Dodatnie dla $m\in\left(\frac{[p1p7]}{[wy]} ,\infty\right)$, ujemne dla $m\in\left(-\infty,\frac{[p2p7]}{[wx]} \right) $, różne dla $m\in\left(-\infty,\frac{[p2p7]}{[wx]} \right) \cup \left(\frac{[p1p7]}{[wy]}  ,\infty\right)$.
B.Dodatnie dla $m\in\left(\frac{[p2p7]}{[wy]} ,\infty\right)$, ujemne dla $m\in\left(-\infty,\frac{[p1p7]}{[wx]} \right) $, różne dla $m\in\left(-\infty,\frac{[p2p7]}{[wx]} \right) \cup \left(\frac{[p1p7]}{[wy]}  ,\infty\right)$.
C.Dodatnie dla $m\in\mathbb{R}$, ujemne dla $m\in\left(-\infty,\frac{[p2p7]}{[wx]} \right) $, różne dla $m\in\left(-\infty,\frac{[p2p7]}{[wx]} \right) \cup \left(\frac{[p1p7]}{[wy]}  ,\infty\right)$.
D.Dodatnie dla $m\in\left(\frac{[p1p7]}{[wy]} ,\infty\right)$, ujemne dla $m\in\left(-\infty,\frac{[p2p7]}{[wx]} \right) $, różne dla $m\in\emptyset$.
E.Dodatnie dla $m\in\left([p1p7] ,\infty\right)$, ujemne dla $m\in\left(-\infty,\frac{[p2p7]}{[wx]} \right) $, różne dla $m\in\left(-\infty,\frac{[p2p7]}{[wx]} \right) \cup \left(\frac{[p1p7]}{[wy]}  ,\infty\right)$.
F.Dodatnie dla $m\in\left(\frac{[p1p7]}{[wy]} ,\infty\right)$, ujemne dla $m\in\emptyset$, różne dla $m\in\left(-\infty,\frac{[p2p7]}{[wy]} \right) \cup \left(\frac{[p1p7]}{[wy]}  ,\infty\right)$.
G.Dodatnie dla $m\in\emptyset$, ujemne dla $m\in\mathbb{R} $, różne dla $m\in\emptyset$.

\testStop
\kluczStart
A
\kluczStop



\end{document}
