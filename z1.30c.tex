\documentclass[12pt, a4paper]{article}
\usepackage[utf8]{inputenc}
\usepackage{polski}

\usepackage{amsthm}  %pakiet do tworzenia twierdzeń itp.
\usepackage{amsmath} %pakiet do niektórych symboli matematycznych
\usepackage{amssymb} %pakiet do symboli mat., np. \nsubseteq
\usepackage{amsfonts}
\usepackage{graphicx} %obsługa plików graficznych z rozszerzeniem png, jpg
\theoremstyle{definition} %styl dla definicji
\newtheorem{zad}{} 
\title{Multizestaw zadań}
\author{Robert Fidytek}
%\date{\today}
\date{}
\newcounter{liczniksekcji}
\newcommand{\kategoria}[1]{\section{#1}} %olreślamy nazwę kateforii zadań
\newcommand{\zadStart}[1]{\begin{zad}#1\newline} %oznaczenie początku zadania
\newcommand{\zadStop}{\end{zad}}   %oznaczenie końca zadania
%Makra opcjonarne (nie muszą występować):
\newcommand{\rozwStart}[2]{\noindent \textbf{Rozwiązanie (autor #1 , recenzent #2): }\newline} %oznaczenie początku rozwiązania, opcjonarnie można wprowadzić informację o autorze rozwiązania zadania i recenzencie poprawności wykonania rozwiązania zadania
\newcommand{\rozwStop}{\newline}                                            %oznaczenie końca rozwiązania
\newcommand{\odpStart}{\noindent \textbf{Odpowiedź:}\newline}    %oznaczenie początku odpowiedzi końcowej (wypisanie wyniku)
\newcommand{\odpStop}{\newline}                                             %oznaczenie końca odpowiedzi końcowej (wypisanie wyniku)
\newcommand{\testStart}{\noindent \textbf{Test:}\newline} %ewentualne możliwe opcje odpowiedzi testowej: A. ? B. ? C. ? D. ? itd.
\newcommand{\testStop}{\newline} %koniec wprowadzania odpowiedzi testowych
\newcommand{\kluczStart}{\noindent \textbf{Test poprawna odpowiedź:}\newline} %klucz, poprawna odpowiedź pytania testowego (jedna literka): A lub B lub C lub D itd.
\newcommand{\kluczStop}{\newline} %koniec poprawnej odpowiedzi pytania testowego 
\newcommand{\wstawGrafike}[2]{\begin{figure}[h] \includegraphics[scale=#2] {#1} \end{figure}} %gdyby była potrzeba wstawienia obrazka, parametry: nazwa pliku, skala (jak nie wiesz co wpisać, to wpisz 1)

\begin{document}
\maketitle


\kategoria{Wikieł/Z1.30c}
\zadStart{Zadanie z Wikieł Z 1.30 c)  moja wersja nr [nrWersji]}
%[p1]:[3,5,6,7,8,2,10,11,12]
%[p2]:[2,4,5,6,7,8,9,10,11,12]
%[p1p2m]=[p1]*[p2]


Podać wzór określający złożenie funkcji $f(g(x))$ oraz $g(f(x))$ dla funkcji $f(x)=\sqrt{[p1]x},\quad g(x)=[p2]x^{2}.$
\zadStop
\rozwStart{Maja Szabłowska}{}
$$(f\circ g)(x)=f(g(x))=f([p2]x^{2})=\sqrt{[p1p2m]x^{2}}=\sqrt{[p1p2m]}|x|$$
$$(g\circ f)(x)=g(f(x))=g(\sqrt{[p1]x})=[p2](\sqrt{[p1]x})^{2}=[p1p2m]x$$
\rozwStop
\odpStart
$(f\circ g)(x)=\sqrt{[p1p2m]}|x|, \quad (g\circ f)(x)=[p1p2m]x$
\odpStop
\testStart

A.$(f\circ g)(x)=\sqrt{[p1p2m]}|x|, \quad (g\circ f)(x)=[p1p2m]x$

B.$(f\circ g)(x)=\sqrt{[p1p2m]}|x|, \quad (g\circ f)(x)=[p1]x$

C.$(f\circ g)(x)=\sqrt{[p1p2m]}|x|, \quad (g\circ f)(x)=[p2]x$

D.$(f\circ g)(x)=\sqrt{[p1]}|x|, \quad (g\circ f)(x)=[p1p2m]x$

E.$(f\circ g)(x)=\sqrt{[p2]}|x|, \quad (g\circ f)(x)=[p1p2m]x$

F.$(f\circ g)(x)=[p1p2m]|x|, \quad (g\circ f)(x)=[p1p2m]x$

G.$(f\circ g)(x)=\sqrt{[p1p2m]}|x|, \quad (g\circ f)(x)=[p1p2m]x^{2}$

H.$(f\circ g)(x)=\sqrt{[p1]}|x|, \quad (g\circ f)(x)=[p1]x$

I.$(f\circ g)(x)=\sqrt{[p2m]}|x|, \quad (g\circ f)(x)=[p2]x$

\testStop
\kluczStart
A
\kluczStop



\end{document}