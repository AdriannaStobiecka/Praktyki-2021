\documentclass[12pt, a4paper]{article}
\usepackage[utf8]{inputenc}
\usepackage{polski}

\usepackage{amsthm}  %pakiet do tworzenia twierdzeń itp.
\usepackage{amsmath} %pakiet do niektórych symboli matematycznych
\usepackage{amssymb} %pakiet do symboli mat., np. \nsubseteq
\usepackage{amsfonts}
\usepackage{graphicx} %obsługa plików graficznych z rozszerzeniem png, jpg
\theoremstyle{definition} %styl dla definicji
\newtheorem{zad}{} 
\title{Multizestaw zadań}
\author{Robert Fidytek}
%\date{\today}
\date{}
\newcounter{liczniksekcji}
\newcommand{\kategoria}[1]{\section{#1}} %olreślamy nazwę kateforii zadań
\newcommand{\zadStart}[1]{\begin{zad}#1\newline} %oznaczenie początku zadania
\newcommand{\zadStop}{\end{zad}}   %oznaczenie końca zadania
%Makra opcjonarne (nie muszą występować):
\newcommand{\rozwStart}[2]{\noindent \textbf{Rozwiązanie (autor #1 , recenzent #2): }\newline} %oznaczenie początku rozwiązania, opcjonarnie można wprowadzić informację o autorze rozwiązania zadania i recenzencie poprawności wykonania rozwiązania zadania
\newcommand{\rozwStop}{\newline}                                            %oznaczenie końca rozwiązania
\newcommand{\odpStart}{\noindent \textbf{Odpowiedź:}\newline}    %oznaczenie początku odpowiedzi końcowej (wypisanie wyniku)
\newcommand{\odpStop}{\newline}                                             %oznaczenie końca odpowiedzi końcowej (wypisanie wyniku)
\newcommand{\testStart}{\noindent \textbf{Test:}\newline} %ewentualne możliwe opcje odpowiedzi testowej: A. ? B. ? C. ? D. ? itd.
\newcommand{\testStop}{\newline} %koniec wprowadzania odpowiedzi testowych
\newcommand{\kluczStart}{\noindent \textbf{Test poprawna odpowiedź:}\newline} %klucz, poprawna odpowiedź pytania testowego (jedna literka): A lub B lub C lub D itd.
\newcommand{\kluczStop}{\newline} %koniec poprawnej odpowiedzi pytania testowego 
\newcommand{\wstawGrafike}[2]{\begin{figure}[h] \includegraphics[scale=#2] {#1} \end{figure}} %gdyby była potrzeba wstawienia obrazka, parametry: nazwa pliku, skala (jak nie wiesz co wpisać, to wpisz 1)

\begin{document}
\maketitle


\kategoria{Wikieł/Z3.13j}
\zadStart{Zadanie z Wikieł Z 3.13 j) moja wersja nr [nrWersji]}
%[a]:[3,4,5,6,7,8,9,10]
%[b]:[2,3,4,5,6,7,8,9]
%[calosci]=[a]//2
%math.gcd([a],2)==1
Obliczyć granicę ciągu 
$$a_n=(n-\sqrt{n^2-[a]n+[b]}).$$
\zadStop
\rozwStart{Adrianna Stobiecka}{}
$$\lim_{n\to\infty}(n-\sqrt{n^2-[a]n+[b]})=\lim_{n\to\infty}\frac{(n-\sqrt{n^2-[a]n+[b]})(n+\sqrt{n^2-[a]n+[b]})}{n+\sqrt{n^2-[a]n+[b]}}$$
$$=\lim_{n\to\infty}\frac{n^2-n^2+[a]n-[b]}{n+\sqrt{n^2-[a]n+[b]}}=\lim_{n\to\infty}\frac{[a]n-[b]}{n+\sqrt{n^2-[a]n+[b]}}$$
$$=\lim_{n\to\infty}\frac{n([a]-\frac{[b]}{n})}{n+n\sqrt{1-\frac{[a]}{n}+\frac{[b]}{n^2}}}=\lim_{n\to\infty}\frac{[a]-\frac{[b]}{n}}{1+\sqrt{1-\frac{[a]}{n}+\frac{[b]}{n^2}}}$$
$$=\frac{[a]}{1+1}=\frac{[a]}{2}=[calosci]\frac{1}{2}$$
\rozwStop
\odpStart
$[calosci]\frac{1}{2}$
\odpStop
\testStart
A.$\infty$
B.$0$
C.$\frac{1}{2}$
D.$-[calosci]$
E.$[calosci]$
F.$-\infty$
G.$[calosci]\frac{1}{2}$
H.$-\frac{1}{2}$
I.$-[calosci]\frac{1}{2}$
\testStop
\kluczStart
G
\kluczStop



\end{document}
