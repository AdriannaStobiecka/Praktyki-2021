\documentclass[12pt, a4paper]{article}
\usepackage[utf8]{inputenc}
\usepackage{polski}

\usepackage{amsthm}  %pakiet do tworzenia twierdzeń itp.
\usepackage{amsmath} %pakiet do niektórych symboli matematycznych
\usepackage{amssymb} %pakiet do symboli mat., np. \nsubseteq
\usepackage{amsfonts}
\usepackage{graphicx} %obsługa plików graficznych z rozszerzeniem png, jpg
\theoremstyle{definition} %styl dla definicji
\newtheorem{zad}{} 
\title{Multizestaw zadań}
\author{Robert Fidytek}
%\date{\today}
\date{}
\newcounter{liczniksekcji}
\newcommand{\kategoria}[1]{\section{#1}} %olreślamy nazwę kateforii zadań
\newcommand{\zadStart}[1]{\begin{zad}#1\newline} %oznaczenie początku zadania
\newcommand{\zadStop}{\end{zad}}   %oznaczenie końca zadania
%Makra opcjonarne (nie muszą występować):
\newcommand{\rozwStart}[2]{\noindent \textbf{Rozwiązanie (autor #1 , recenzent #2): }\newline} %oznaczenie początku rozwiązania, opcjonarnie można wprowadzić informację o autorze rozwiązania zadania i recenzencie poprawności wykonania rozwiązania zadania
\newcommand{\rozwStop}{\newline}                                            %oznaczenie końca rozwiązania
\newcommand{\odpStart}{\noindent \textbf{Odpowiedź:}\newline}    %oznaczenie początku odpowiedzi końcowej (wypisanie wyniku)
\newcommand{\odpStop}{\newline}                                             %oznaczenie końca odpowiedzi końcowej (wypisanie wyniku)
\newcommand{\testStart}{\noindent \textbf{Test:}\newline} %ewentualne możliwe opcje odpowiedzi testowej: A. ? B. ? C. ? D. ? itd.
\newcommand{\testStop}{\newline} %koniec wprowadzania odpowiedzi testowych
\newcommand{\kluczStart}{\noindent \textbf{Test poprawna odpowiedź:}\newline} %klucz, poprawna odpowiedź pytania testowego (jedna literka): A lub B lub C lub D itd.
\newcommand{\kluczStop}{\newline} %koniec poprawnej odpowiedzi pytania testowego 
\newcommand{\wstawGrafike}[2]{\begin{figure}[h] \includegraphics[scale=#2] {#1} \end{figure}} %gdyby była potrzeba wstawienia obrazka, parametry: nazwa pliku, skala (jak nie wiesz co wpisać, to wpisz 1)

\begin{document}
\maketitle


\kategoria{Wikieł/Z5.6g}
\zadStart{Zadanie z Wikieł Z 5.6g) moja wersja nr [nrWersji]}
%[a]:[2,3,4,5,6,7,8,9]
%[b]:[2,3,4,5,6,7,8,9]
%[a]!=[b]
Obliczyć pochodną funkcji $f$ oraz określić dziedzinę funkcji $f$ i funkcji pochodnej $f'$.\\
$f(x)=\log_{[b]}{(\log_{[b]}([a]x))}$
\zadStop
\rozwStart{Joanna Świerzbin}{}
Dziedzina $D_f: $
$$[a]x>0 \land \log_{[b]}([a]x)>0 $$
$$x>0 \land [a]x>1$$
$$x\in \left(\frac{1}{[a]},\infty \right) $$
$$ f'(x)=\left(\log_{[b]}{(\log_{[b]}([a]x))}\right)'=\frac{1}{\log_{[b]}([a]x)\ln([b])}\left( \log_{[b]} ([a]x) \right)'= $$
$$ = \frac{1}{\log_{[b]}([a]x)\ln([b])}\frac{1}{([a]x)ln([b])}\left([a]x\right)' =$$ 
$$ = \frac{[a]}{([a]x)\ln^2([b])\log_{[b]}([a]x)} =$$ 
$$ = \frac{1}{x\ln^2([b])\log_{[b]}([a]x)}$$ 
Dziedzina $D_{f'}:$
$$ [a]x> 0 \land {x\ln^2([b])\log_{[b]}([a]x)} \neq 0 $$
$$ x> 0 \land x \neq 0 \land \log_{[b]}([a]x) \neq 0 $$
$$ x> 0 \land [a]x \neq 1 $$
$$ x> 0 \land x \neq \frac{1}{[a]} $$
$$x \in \left(0,\frac{1}{[a]}\right) \cup \left( \frac{1}{[a]}, \infty \right) $$
\rozwStop
\odpStart
$f'(x)= \frac{1}{x\ln^2([b])\log_{[b]}([a]x)} , D_f: x\in \left(\frac{1}{[a]},\infty \right), D_{f'}: x \in \left(0,\frac{1}{[a]}\right) \cup \left( \frac{1}{[a]}, \infty \right)$
\odpStop
\testStart
A. $f'(x)= \frac{1}{x\ln^2([b])\log_{[b]}([a]x)} , D_f: x\in \left(\frac{1}{[a]},\infty \right), D_{f'}: x \in \left(0,\frac{1}{[a]}\right) \cup \left( \frac{1}{[a]}, \infty \right)$\\
B. $f'(x)= \frac{1}{x\ln([b])\log_{[b]}([a]x)} , D_f: x\in \left(\frac{1}{[a]},\infty \right), D_{f'}: x \in \left(0,\frac{1}{[a]}\right) \cup \left( \frac{1}{[a]}, \infty \right)$\\
C. $f'(x)= \frac{1}{x\ln^2([b])} , D_f: x\in \left(\frac{1}{[a]},\infty \right), D_{f'}: x \in \left(0,\frac{1}{[a]}\right) \cup \left( \frac{1}{[a]}, \infty \right)$\\
D. $f'(x)= \frac{1}{x\ln^2([b])\log_{[b]}([a]x)} , D_f: x\in \left(\frac{1}{[a]},\infty \right), D_{f'}: x \in \left(0,\frac{1}{[a]}\right) \cup \left( \frac{1}{[a]}, \infty \right)$\\
E. $f'(x)= \frac{1}{x\ln^2([b])\log_{[b]}([a]x)} , D_f: x\in \left(\frac{1}{[a]},\infty \right), D_{f'}: x \in \left(0,-\frac{1}{[a]}\right) \cup \left( \frac{1}{[a]}, \infty \right)$\\
F. $f'(x)= \frac{1}{\ln^2([b])\log_{[b]}([a]x)} , D_f: x\in \left(-\frac{1}{[a]}, \frac{1}{[a]} \right), D_{f'}: x \in \left(0,\frac{1}{[a]}\right) \cup \left( \frac{1}{[a]}, \infty \right)$
\testStop
\kluczStart
A
\kluczStop



\end{document}