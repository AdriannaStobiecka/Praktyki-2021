\documentclass[12pt, a4paper]{article}
\usepackage[utf8]{inputenc}
\usepackage{polski}

\usepackage{amsthm}  %pakiet do tworzenia twierdzeń itp.
\usepackage{amsmath} %pakiet do niektórych symboli matematycznych
\usepackage{amssymb} %pakiet do symboli mat., np. \nsubseteq
\usepackage{amsfonts}
\usepackage{graphicx} %obsługa plików graficznych z rozszerzeniem png, jpg
\theoremstyle{definition} %styl dla definicji
\newtheorem{zad}{} 
\title{Multizestaw zadań}
\author{Robert Fidytek}
%\date{\today}
\date{}\documentclass[12pt, a4paper]{article}
\usepackage[utf8]{inputenc}
\usepackage{polski}

\usepackage{amsthm}  %pakiet do tworzenia twierdzeń itp.
\usepackage{amsmath} %pakiet do niektórych symboli matematycznych
\usepackage{amssymb} %pakiet do symboli mat., np. \nsubseteq
\usepackage{amsfonts}
\usepackage{graphicx} %obsługa plików graficznych z rozszerzeniem png, jpg
\theoremstyle{definition} %styl dla definicji
\newtheorem{zad}{} 
\title{Multizestaw zadań}
\author{Robert Fidytek}
%\date{\today}
\date{}
\newcounter{liczniksekcji}
\newcommand{\kategoria}[1]{\section{#1}} %olreślamy nazwę kateforii zadań
\newcommand{\zadStart}[1]{\begin{zad}#1\newline} %oznaczenie początku zadania
\newcommand{\zadStop}{\end{zad}}   %oznaczenie końca zadania
%Makra opcjonarne (nie muszą występować):
\newcommand{\rozwStart}[2]{\noindent \textbf{Rozwiązanie (autor #1 , recenzent #2): }\newline} %oznaczenie początku rozwiązania, opcjonarnie można wprowadzić informację o autorze rozwiązania zadania i recenzencie poprawności wykonania rozwiązania zadania
\newcommand{\rozwStop}{\newline}                                            %oznaczenie końca rozwiązania
\newcommand{\odpStart}{\noindent \textbf{Odpowiedź:}\newline}    %oznaczenie początku odpowiedzi końcowej (wypisanie wyniku)
\newcommand{\odpStop}{\newline}                                             %oznaczenie końca odpowiedzi końcowej (wypisanie wyniku)
\newcommand{\testStart}{\noindent \textbf{Test:}\newline} %ewentualne możliwe opcje odpowiedzi testowej: A. ? B. ? C. ? D. ? itd.
\newcommand{\testStop}{\newline} %koniec wprowadzania odpowiedzi testowych
\newcommand{\kluczStart}{\noindent \textbf{Test poprawna odpowiedź:}\newline} %klucz, poprawna odpowiedź pytania testowego (jedna literka): A lub B lub C lub D itd.
\newcommand{\kluczStop}{\newline} %koniec poprawnej odpowiedzi pytania testowego 
\newcommand{\wstawGrafike}[2]{\begin{figure}[h] \includegraphics[scale=#2] {#1} \end{figure}} %gdyby była potrzeba wstawienia obrazka, parametry: nazwa pliku, skala (jak nie wiesz co wpisać, to wpisz 1)

\begin{document}
\maketitle


\kategoria{Wikieł/Z1.129t}
\zadStart{Zadanie z Wikieł Z 1.129 t) moja wersja nr [nrWersji]}
%[p1]:[2,3,4,5,6,7,8,9,10]
%[p2]=random.randint(2,10)
%[p3]:[2,3,4,5,6,7,8,9,10]
%[p3k]=[p3]*[p3]
%[2p3p1]=2*[p3]*[p1]
%[p1k]=[p1]*[p1]
%[xk]=-1-[p1k]
%[x]=[p2]+[2p3p1]
%[del]=[x]*[x]+4*[x]*[p3k]
%[pdel]=round(math.sqrt([del]),2)
%[2xk]=2*[xk]
%[x1]=round((-[x]-[pdel])/[2xk],2)
%[x2]=round((-[x]+[pdel])/[2xk],2)
%[del]>0 and [x2]<[x1]

Wyznaczyć dziedzinę naturalną funkcji.
$$f(x)=\log \left([p1]x+\sqrt{[p2]x-x^{2}}-[p3]\right)$$
\zadStop

\rozwStart{Maja Szabłowska}{}
$$[p2]x-x^{2}\geq 0$$
$$x([p2]-x)\geq 0$$
$$x\in[0,[p2]]$$

$$[p1]x+\sqrt{[p2]x-x^{2}}-[p3]>0$$
$$\sqrt{[p2]x-x^{2}}>[p3]-[p1]x$$
$$[p2]x-x^{2}>([p3]-[p1]x)^{2}$$
$$[p2]x-x^{2}>[p3k]-[2p3p1]x+[p1k]x^{2} $$
$$[xk]x^{2}+[x]x-[p3k]>0$$
$$\Delta=[x]^{2}-4\cdot([xk])\cdot(-[p3k])=[del] \Rightarrow \sqrt{\Delta}=[pdel]$$
$$x_{1}=\frac{-[x]-[pdel]}{[2xk]}=[x1], \quad x_{2}=\frac{-[x]+[pdel]}{[2xk]}=[x2]$$
$$x\in([x2], [x1])$$

Zatem ostatecznie 
$$x\in[0,[x1])$$
\rozwStop
\odpStart
$x\in[0,[x1])$
\odpStop
\testStart
A.$x\in[0,[x1])$
B.$x\in[e^{[p2]},\infty)$
C.$x\in(-\infty, 0)$
D.$x\in(-\infty, -[p2]] \cup [\ln[p1],\infty)$
E.$x\in[[p1],\infty)$
F.$x\in([p2],\infty)$
G.$x\in\emptyset$

\testStop
\kluczStart
A
\kluczStop



\end{document}
