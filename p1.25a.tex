\documentclass[12pt, a4paper]{article}
\usepackage[utf8]{inputenc}
\usepackage{polski}

\usepackage{amsthm}  %pakiet do tworzenia twierdzeń itp.
\usepackage{amsmath} %pakiet do niektórych symboli matematycznych
\usepackage{amssymb} %pakiet do symboli mat., np. \nsubseteq
\usepackage{amsfonts}
\usepackage{graphicx} %obsługa plików graficznych z rozszerzeniem png, jpg
\theoremstyle{definition} %styl dla definicji
\newtheorem{zad}{} 
\title{Multizestaw zadań}
\author{Robert Fidytek}
%\date{\today}
\date{}
\newcounter{liczniksekcji}
\newcommand{\kategoria}[1]{\section{#1}} %olreślamy nazwę kateforii zadań
\newcommand{\zadStart}[1]{\begin{zad}#1\newline} %oznaczenie początku zadania
\newcommand{\zadStop}{\end{zad}}   %oznaczenie końca zadania
%Makra opcjonarne (nie muszą występować):
\newcommand{\rozwStart}[2]{\noindent \textbf{Rozwiązanie (autor #1 , recenzent #2): }\newline} %oznaczenie początku rozwiązania, opcjonarnie można wprowadzić informację o autorze rozwiązania zadania i recenzencie poprawności wykonania rozwiązania zadania
\newcommand{\rozwStop}{\newline}                                            %oznaczenie końca rozwiązania
\newcommand{\odpStart}{\noindent \textbf{Odpowiedź:}\newline}    %oznaczenie początku odpowiedzi końcowej (wypisanie wyniku)
\newcommand{\odpStop}{\newline}                                             %oznaczenie końca odpowiedzi końcowej (wypisanie wyniku)
\newcommand{\testStart}{\noindent \textbf{Test:}\newline} %ewentualne możliwe opcje odpowiedzi testowej: A. ? B. ? C. ? D. ? itd.
\newcommand{\testStop}{\newline} %koniec wprowadzania odpowiedzi testowych
\newcommand{\kluczStart}{\noindent \textbf{Test poprawna odpowiedź:}\newline} %klucz, poprawna odpowiedź pytania testowego (jedna literka): A lub B lub C lub D itd.
\newcommand{\kluczStop}{\newline} %koniec poprawnej odpowiedzi pytania testowego 
\newcommand{\wstawGrafike}[2]{\begin{figure}[h] \includegraphics[scale=#2] {#1} \end{figure}} %gdyby była potrzeba wstawienia obrazka, parametry: nazwa pliku, skala (jak nie wiesz co wpisać, to wpisz 1)

\begin{document}
\maketitle


\kategoria{Wikieł/P1.25a}
\zadStart{Zadanie z Wikieł P 1.25 a)  moja wersja nr [nrWersji]}
%[p1]:[2,3,4,5,6,7,8,9,10,11,12,13,14,15,16,17,18]
%[p2]=random.randint(1,10)
%[p3]=[p2]+[p1]
%[p4]=random.randint(1,10)
%[p5]=random.randint(1,10)
%[p6]=random.randint(1,10)
%[p7]=random.randint(1,10)
%[p4p5p6p7]=-[p4]+[p5]+[p6]+[p7]


Dany jest wielomian $W(x)=([p1]x^{5}-[p2]x^{2}+[p3])^{3}\cdot([p4]x^{7}+[p5]x^{4}-[p6]x+[p7])^{2007}.$ Sprawdzić, czy liczba $x_{0}=-1$ jest pierwiastkiem tego wielomianu.
\zadStop
\rozwStart{Maja Szabłowska}{}
Obliczamy
$$W(-1)=(-[p1]-[p2]+[p3])^{3}\cdot(-[p4]+[p5]+[p6]+[p7])^{2007}=$$
$$=0\cdot[p4p5p6p7]^{2007}=0 $$
Ponieważ $W(-1)=0,$ to liczba $x_{0}=-1$ jets pierwiastkiem wielomianu $W(x).$
\rozwStop
\odpStart
$W(-1)=0$
\odpStop
\testStart
A.$W(-1)=0$
B.$W(-1)=-1$
C.$W(-1)=2007$
D.$W(-1)=1$
E.$W(-1)=27$
F.$W(-1)=130$
G.$W(-1)=[p4p5p6p7]$
H.$W(-1)=[p4p5p6p7]^{2007}$
I.$W(-1)=[p4p5p6p7]^{2010}$
\testStop
\kluczStart
A
\kluczStop



\end{document}