\documentclass[12pt, a4paper]{article}
\usepackage[utf8]{inputenc}
\usepackage{polski}

\usepackage{amsthm}  %pakiet do tworzenia twierdzeń itp.
\usepackage{amsmath} %pakiet do niektórych symboli matematycznych
\usepackage{amssymb} %pakiet do symboli mat., np. \nsubseteq
\usepackage{amsfonts}
\usepackage{graphicx} %obsługa plików graficznych z rozszerzeniem png, jpg
\theoremstyle{definition} %styl dla definicji
\newtheorem{zad}{} 
\title{Multizestaw zadań}
\author{Robert Fidytek}
%\date{\today}
\date{}
\newcounter{liczniksekcji}
\newcommand{\kategoria}[1]{\section{#1}} %olreślamy nazwę kateforii zadań
\newcommand{\zadStart}[1]{\begin{zad}#1\newline} %oznaczenie początku zadania
\newcommand{\zadStop}{\end{zad}}   %oznaczenie końca zadania
%Makra opcjonarne (nie muszą występować):
\newcommand{\rozwStart}[2]{\noindent \textbf{Rozwiązanie (autor #1 , recenzent #2): }\newline} %oznaczenie początku rozwiązania, opcjonarnie można wprowadzić informację o autorze rozwiązania zadania i recenzencie poprawności wykonania rozwiązania zadania
\newcommand{\rozwStop}{\newline}                                            %oznaczenie końca rozwiązania
\newcommand{\odpStart}{\noindent \textbf{Odpowiedź:}\newline}    %oznaczenie początku odpowiedzi końcowej (wypisanie wyniku)
\newcommand{\odpStop}{\newline}                                             %oznaczenie końca odpowiedzi końcowej (wypisanie wyniku)
\newcommand{\testStart}{\noindent \textbf{Test:}\newline} %ewentualne możliwe opcje odpowiedzi testowej: A. ? B. ? C. ? D. ? itd.
\newcommand{\testStop}{\newline} %koniec wprowadzania odpowiedzi testowych
\newcommand{\kluczStart}{\noindent \textbf{Test poprawna odpowiedź:}\newline} %klucz, poprawna odpowiedź pytania testowego (jedna literka): A lub B lub C lub D itd.
\newcommand{\kluczStop}{\newline} %koniec poprawnej odpowiedzi pytania testowego 
\newcommand{\wstawGrafike}[2]{\begin{figure}[h] \includegraphics[scale=#2] {#1} \end{figure}} %gdyby była potrzeba wstawienia obrazka, parametry: nazwa pliku, skala (jak nie wiesz co wpisać, to wpisz 1)

\begin{document}
\maketitle


\kategoria{Wikieł/Z5.67d}
\zadStart{Zadanie z Wikieł Z 5.67 d) moja wersja nr [nrWersji]}
%[a]:[2,3,4,5,6,7,8,9,10,11,12,13,14]
%[a1]=[a]+1
Wyznaczyć dziedzinę funkcji oraz przedziały, w których dana funkcja jest malejąca lub rosnąca:\\
d) $f(x)=arcsin(\sqrt{[a]\cdot x})+\sqrt{x(1-[a]\cdot x)}$
\zadStop
\rozwStart{Wojciech Przybylski}{}
1. Wyznaczamy dziedzinę funkcji.
$$x\geq0 \hspace{3mm} \wedge \hspace{3mm} x(1-[a]\cdot x)\geq0 \Rightarrow x\in(0,\frac{1}{[a]})$$
$$D_{f}=(0,\frac{1}{[a]})$$
2. Wyznaczamy przedziały, w których funkcja maleje lub rośnie. 
$$f'(x)=\frac{1}{2\sqrt{(\frac{1}{[a]}-x)x}}+\frac{\frac{1}{[a]}-x}{2\sqrt{(\frac{1}{[a]}-x)x}}=\frac{\frac{[a1]}{[a]}-x}{2\sqrt{(\frac{1}{[a]}-x)x}}$$
$$f'(x)>0 \mbox{ - funckja rośnie, } f'(x)<0 \mbox{ - funckja maleje }$$
$$f'(x)>0\Rightarrow \frac{\frac{[a1]}{[a]}-x}{2\sqrt{(\frac{1}{[a]}-x)x}}>0 \Rightarrow \frac{[a1]}{[a]}-x>0 \Rightarrow \frac{[a1]}{[a]}>x,\hspace{2mm} x\in(0,\frac{1}{[a]})$$
$$f'(x)<0\Rightarrow \frac{\frac{[a1]}{[a]}-x}{2\sqrt{(\frac{1}{[a]}-x)x}}<0  \Rightarrow \frac{[a1]}{[a]}-x<0 \Rightarrow \frac{[a1]}{[a]}<x, \hspace{3mm} x\in\emptyset$$
\rozwStop
\odpStart
$D_{f}=(0,\frac{1}{[a]})$, $f(x)$ rośnie w $x\in(0,\frac{1}{[a]})$.
\odpStop
\testStart
A. $D_{f}=(0,\frac{1}{[a]})$, $f(x)$ rośnie w $x\in(0,\frac{1}{[a]})$.\\
B. $D_{f}=\mathbb{R}$, $f(x)$ maleje w $x\in(0,\frac{1}{[a]})$.\\
C. $D_{f}=(0,\frac{1}{[a]})$, $f(x)$ maleje w $x\in(0,\infty)$.\\
D. $D_{f}=\mathbb{R}$, $f(x)$ rośnie w $x\in(0,\infty)$.\\
E. $D_{f}=(0,\frac{1}{[a]})$, $f(x)$ rośnie w $x\in(1,\infty)$.\\
F. $D_{f}=\mathbb{R}$, funkcja jest stała.
\testStop
\kluczStart
A
\kluczStop



\end{document}