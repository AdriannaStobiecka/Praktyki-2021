\documentclass[12pt, a4paper]{article}
\usepackage[utf8]{inputenc}
\usepackage{polski}

\usepackage{amsthm}  %pakiet do tworzenia twierdzeń itp.
\usepackage{amsmath} %pakiet do niektórych symboli matematycznych
\usepackage{amssymb} %pakiet do symboli mat., np. \nsubseteq
\usepackage{amsfonts}
\usepackage{graphicx} %obsługa plików graficznych z rozszerzeniem png, jpg
\theoremstyle{definition} %styl dla definicji
\newtheorem{zad}{} 
\title{Multizestaw zadań}
\author{Robert Fidytek}
%\date{\today}
\date{}
\newcounter{liczniksekcji}
\newcommand{\kategoria}[1]{\section{#1}} %olreślamy nazwę kateforii zadań
\newcommand{\zadStart}[1]{\begin{zad}#1\newline} %oznaczenie początku zadania
\newcommand{\zadStop}{\end{zad}}   %oznaczenie końca zadania
%Makra opcjonarne (nie muszą występować):
\newcommand{\rozwStart}[2]{\noindent \textbf{Rozwiązanie (autor #1 , recenzent #2): }\newline} %oznaczenie początku rozwiązania, opcjonarnie można wprowadzić informację o autorze rozwiązania zadania i recenzencie poprawności wykonania rozwiązania zadania
\newcommand{\rozwStop}{\newline}                                            %oznaczenie końca rozwiązania
\newcommand{\odpStart}{\noindent \textbf{Odpowiedź:}\newline}    %oznaczenie początku odpowiedzi końcowej (wypisanie wyniku)
\newcommand{\odpStop}{\newline}                                             %oznaczenie końca odpowiedzi końcowej (wypisanie wyniku)
\newcommand{\testStart}{\noindent \textbf{Test:}\newline} %ewentualne możliwe opcje odpowiedzi testowej: A. ? B. ? C. ? D. ? itd.
\newcommand{\testStop}{\newline} %koniec wprowadzania odpowiedzi testowych
\newcommand{\kluczStart}{\noindent \textbf{Test poprawna odpowiedź:}\newline} %klucz, poprawna odpowiedź pytania testowego (jedna literka): A lub B lub C lub D itd.
\newcommand{\kluczStop}{\newline} %koniec poprawnej odpowiedzi pytania testowego 
\newcommand{\wstawGrafike}[2]{\begin{figure}[h] \includegraphics[scale=#2] {#1} \end{figure}} %gdyby była potrzeba wstawienia obrazka, parametry: nazwa pliku, skala (jak nie wiesz co wpisać, to wpisz 1)

\begin{document}
\maketitle


\kategoria{Wikieł/Z1.9c}
\zadStart{Zadanie z Wikieł Z 1.9 c) moja wersja nr [nrWersji]}
%[a]:[2,3,4]
%[b]:[2,3,4]
%[c]:[2,3,4]
%[a]=random.randint(2,3)
%[b]=random.randint(2,4)
%[c]=random.randint(2,3)
%[cb]=[c]*[b]*[c]
%[2a]=2*[a]
%[8a]=8*[a]
%[x]=8+[2a]^3
Uprościć wyrażenie: $(x+\sqrt{[a]})^4+[b]x(x+[c])(x-[c])-(x-\sqrt{[a]})^4$
\zadStop
\rozwStart{Pascal Nawrocki}{}
Stosujemy wzory skróconego mnożenia (polecam naukę korzystania z trójkąta Pascala, aby nie musieć uczyć się wzorów na pamięć) $(a+b)^4=a^4+4a^3b+6a^2b^2+4ab^3+b^4$, $(a-b)^4=a^4-4a^3b+6a^2b^2-4ab^3+b^4$ oraz $(a+b)(a-b)=a^2-b^2$ Zatem: \newline
$(x+\sqrt{[a]})^4+[b]x(x+[c])(x-[c])-(x-\sqrt{[a]})^4=$
$$=x^4+4x^3\cdot\sqrt{[a]}+6x^2\cdot(\sqrt{[a]})^2+4x(\sqrt{[a]})^3+(\sqrt{[a]})^4+[b]x^3-[cb]x-\big(x^4-4x^3\cdot\sqrt{[a]}+6x^2\cdot(\sqrt{[a]})^2-x(\sqrt{[a]})^3+(\sqrt{[a]})^4\big)=$$
$$=x^4+4x^3\cdot\sqrt{[a]}+6x^2\cdot(\sqrt{[a]})^2+4x(\sqrt{[a]})^3+(\sqrt{[a]})^4-x^4+[b]x^3-[cb]x+4x^3\cdot\sqrt{[a]}-6x^2\cdot(\sqrt{[a]})^2+x(\sqrt{[a]})^3-(\sqrt{[a]})^4=$$
Teraz musimy tylko uporządkować. To znaczy, że $x^4$ z $x^4$, $x^3$ z $x^3$ etc. dobrym pomysłem kiedy jest tego dużo, jest podkreślanie/zakreślanie kolorem liczb które będziemy łączyć.
$$=\big([b]+8\sqrt{[a]}\big)x^3+\big([8a]\sqrt{[a]}-[cb]\big)x$$
\odpStop
$=8\sqrt{[a]}x^3+[8a]\sqrt{[a]}x$
\testStart
A.$=8\sqrt{[a]}x^3+[8a]\sqrt{[a]}x$
B.$=8\sqrt{[a]}x^3-[8a]\sqrt{[a]}x$
C.$x^3+[8a]\sqrt{[a]}x$
D.$=x^3-[8a]\sqrt{[a]}x$
\testStop
\kluczStart
A
\kluczStop
\end{document}