\documentclass[12pt, a4paper]{article}
\usepackage[utf8]{inputenc}
\usepackage{polski}

\usepackage{amsthm}  %pakiet do tworzenia twierdzeń itp.
\usepackage{amsmath} %pakiet do niektórych symboli matematycznych
\usepackage{amssymb} %pakiet do symboli mat., np. \nsubseteq
\usepackage{amsfonts}
\usepackage{graphicx} %obsługa plików graficznych z rozszerzeniem png, jpg
\theoremstyle{definition} %styl dla definicji
\newtheorem{zad}{} 
\title{Multizestaw zadań}
\author{Robert Fidytek}
%\date{\today}
\date{}
\newcounter{liczniksekcji}
\newcommand{\kategoria}[1]{\section{#1}} %olreślamy nazwę kateforii zadań
\newcommand{\zadStart}[1]{\begin{zad}#1\newline} %oznaczenie początku zadania
\newcommand{\zadStop}{\end{zad}}   %oznaczenie końca zadania
%Makra opcjonarne (nie muszą występować):
\newcommand{\rozwStart}[2]{\noindent \textbf{Rozwiązanie (autor #1 , recenzent #2): }\newline} %oznaczenie początku rozwiązania, opcjonarnie można wprowadzić informację o autorze rozwiązania zadania i recenzencie poprawności wykonania rozwiązania zadania
\newcommand{\rozwStop}{\newline}                                            %oznaczenie końca rozwiązania
\newcommand{\odpStart}{\noindent \textbf{Odpowiedź:}\newline}    %oznaczenie początku odpowiedzi końcowej (wypisanie wyniku)
\newcommand{\odpStop}{\newline}                                             %oznaczenie końca odpowiedzi końcowej (wypisanie wyniku)
\newcommand{\testStart}{\noindent \textbf{Test:}\newline} %ewentualne możliwe opcje odpowiedzi testowej: A. ? B. ? C. ? D. ? itd.
\newcommand{\testStop}{\newline} %koniec wprowadzania odpowiedzi testowych
\newcommand{\kluczStart}{\noindent \textbf{Test poprawna odpowiedź:}\newline} %klucz, poprawna odpowiedź pytania testowego (jedna literka): A lub B lub C lub D itd.
\newcommand{\kluczStop}{\newline} %koniec poprawnej odpowiedzi pytania testowego 
\newcommand{\wstawGrafike}[2]{\begin{figure}[h] \includegraphics[scale=#2] {#1} \end{figure}} %gdyby była potrzeba wstawienia obrazka, parametry: nazwa pliku, skala (jak nie wiesz co wpisać, to wpisz 1)

\begin{document}
\maketitle


\kategoria{Wikieł/Z5.55h}
\zadStart{Zadanie z Wikieł Z 5.55h) moja wersja nr [nrWersji]}
%[a]=random.randint(0,10)
%[b]:[-7,-6,-5,-4,-3,-2,-1]
%[c]:[2,3,4,5,6,7]
%[d]:[2,3,4,5,6,7]
%[bc]=[b]*[c]
%[bcp]=2*[bc] 
%[bcd]=-[bc]
%[bcpd]=-[bcp]
%[cc]=[c]*[c]
%[bcd]!=[c] and [bcpd]!=[c] and [d]!=[c] and [bcd]!=[d] and [bcpd]!=[d] and [cc]!=[bcd] and [cc]!=[bcpd] and [cc]!=[d]
Na podstawie podanych wartości $g'([a])=[b],$ $g'([d])=[c],$ $g([a])=[d]$ obliczyć wartość następującej pochodnej $(g\circ g)'([a])$.
\zadStop
\rozwStart{Justyna Chojecka}{}
Zauważmy, że 
$$(g\circ g)'(x)=g'(g(x))\cdot g'(x).$$
Obliczamy wartość pochodnej $(g\circ g)'(x)$ dla $x=[a]$.
$$(g\circ g)'([a])=g'(g([a]))\cdot g'([a])=g'([d])\cdot ([b])=[c]\cdot ([b])=[bc]$$
\rozwStop
\odpStart
$[bc]$
\odpStop
\testStart
A.$[bc]$
B.$[bcd]$
C.$-[c]$
D.$[bcpd]$
E.$[d]$
F.$[bcp]$
G.$[cc]$
H.$[c]$
I.$-[d]$
\testStop
\kluczStart
A
\kluczStop



\end{document}