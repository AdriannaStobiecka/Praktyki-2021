\documentclass[12pt, a4paper]{article}
\usepackage[utf8]{inputenc}
\usepackage{polski}

\usepackage{amsthm}  %pakiet do tworzenia twierdzeń itp.
\usepackage{amsmath} %pakiet do niektórych symboli matematycznych
\usepackage{amssymb} %pakiet do symboli mat., np. \nsubseteq
\usepackage{amsfonts}
\usepackage{graphicx} %obsługa plików graficznych z rozszerzeniem png, jpg
\theoremstyle{definition} %styl dla definicji
\newtheorem{zad}{} 
\title{Multizestaw zadań}
\author{Robert Fidytek}
%\date{\today}
\date{}
\newcounter{liczniksekcji}
\newcommand{\kategoria}[1]{\section{#1}} %olreślamy nazwę kateforii zadań
\newcommand{\zadStart}[1]{\begin{zad}#1\newline} %oznaczenie początku zadania
\newcommand{\zadStop}{\end{zad}}   %oznaczenie końca zadania
%Makra opcjonarne (nie muszą występować):
\newcommand{\rozwStart}[2]{\noindent \textbf{Rozwiązanie (autor #1 , recenzent #2): }\newline} %oznaczenie początku rozwiązania, opcjonarnie można wprowadzić informację o autorze rozwiązania zadania i recenzencie poprawności wykonania rozwiązania zadania
\newcommand{\rozwStop}{\newline}                                            %oznaczenie końca rozwiązania
\newcommand{\odpStart}{\noindent \textbf{Odpowiedź:}\newline}    %oznaczenie początku odpowiedzi końcowej (wypisanie wyniku)
\newcommand{\odpStop}{\newline}                                             %oznaczenie końca odpowiedzi końcowej (wypisanie wyniku)
\newcommand{\testStart}{\noindent \textbf{Test:}\newline} %ewentualne możliwe opcje odpowiedzi testowej: A. ? B. ? C. ? D. ? itd.
\newcommand{\testStop}{\newline} %koniec wprowadzania odpowiedzi testowych
\newcommand{\kluczStart}{\noindent \textbf{Test poprawna odpowiedź:}\newline} %klucz, poprawna odpowiedź pytania testowego (jedna literka): A lub B lub C lub D itd.
\newcommand{\kluczStop}{\newline} %koniec poprawnej odpowiedzi pytania testowego 
\newcommand{\wstawGrafike}[2]{\begin{figure}[h] \includegraphics[scale=#2] {#1} \end{figure}} %gdyby była potrzeba wstawienia obrazka, parametry: nazwa pliku, skala (jak nie wiesz co wpisać, to wpisz 1)

\begin{document}
\maketitle


\kategoria{Wikieł/Z5.55k}
\zadStart{Zadanie z Wikieł Z 5.55k) moja wersja nr [nrWersji]}
%[a]:[1,2,3,4,5,6,7,8,9]
%[b]:[2,3,4,5,6,7,8,9,10,11,12,13]
%[c]:[4,9,16,25,36,49,64,81,100]
%[cpierw]=int(math.sqrt([c]))
%[m]=2*[cpierw]
%math.gcd([b],[m])==1 and [b]!=[cpierw] and [b]!=[a] and [b]!=[m] and [cpierw]!=[a] 
Na podstawie podanych wartości $f'([a])=[b],$ $f([a])=[c]$ obliczyć wartość następującej pochodnej $\left[\sqrt{f(x)}\right]' \big |_{x=[a]}$.
\zadStop
\rozwStart{Justyna Chojecka}{}
Zauważmy, że $\sqrt{f(x)}$ jest złożeniem funkcji $g(x)=\sqrt{x}$ oraz $f(x)$. Stąd 
$$\sqrt{f(x)}=g(f(x))=(g\circ f)(x).$$
Następnie obliczamy pochodną powyższego złożenia
$$\left(\sqrt{f(x)}\right)'=(g\circ f)'(x)=g'(f(x))\cdot f'(x)=\frac{1}{2\sqrt{f(x)}}\cdot f'(x).$$
Obliczamy wartość pochodnej $\left(\sqrt{f(x)}\right)'$ dla $x=[a]$.
$$\left(\sqrt{f([a])}\right)'=\frac{1}{2\sqrt{f([a])}}\cdot f'([a])=\frac{1}{2\cdot \sqrt{[c]}}\cdot [b]=\frac{[b]}{[m]}$$
\rozwStop
\odpStart
$\frac{[b]}{[m]}$
\odpStop
\testStart
A.$\frac{[b]}{[m]}$
B.$-\frac{[b]}{[cpierw]}$
C.$-\frac{[m]}{[b]}$
D.$\frac{[cpierw]}{[b]}$
E.$\frac{[b]}{[cpierw]}$
F.$-\frac{[b]}{[m]}$
G.$\frac{[m]}{[b]}$
H.$\frac{[a]}{[m]}$
I.$-\frac{[cpierw]}{[b]}$
\testStop
\kluczStart
A
\kluczStop



\end{document}