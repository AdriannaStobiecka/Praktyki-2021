\documentclass[12pt, a4paper]{article}
\usepackage[utf8]{inputenc}
\usepackage{polski}

\usepackage{amsthm}  %pakiet do tworzenia twierdzeń itp.
\usepackage{amsmath} %pakiet do niektórych symboli matematycznych
\usepackage{amssymb} %pakiet do symboli mat., np. \nsubseteq
\usepackage{amsfonts}
\usepackage{graphicx} %obsługa plików graficznych z rozszerzeniem png, jpg
\theoremstyle{definition} %styl dla definicji
\newtheorem{zad}{} 
\title{Multizestaw zadań}
\author{Mirella Narewska}
%\date{\today}
\date{}
\newcounter{liczniksekcji}
\newcommand{\kategoria}[1]{\section{#1}} %olreślamy nazwę kateforii zadań
\newcommand{\zadStart}[1]{\begin{zad}#1\newline} %oznaczenie początku zadania
\newcommand{\zadStop}{\end{zad}}   %oznaczenie końca zadania
%Makra opcjonarne (nie muszą występować):
\newcommand{\rozwStart}[2]{\noindent \textbf{Rozwiązanie (autor #1 , recenzent #2): }\newline} %oznaczenie początku rozwiązania, opcjonarnie można wprowadzić informację o autorze rozwiązania zadania i recenzencie poprawności wykonania rozwiązania zadania
\newcommand{\rozwStop}{\newline}                                            %oznaczenie końca rozwiązania
\newcommand{\odpStart}{\noindent \textbf{Odpowiedź:}\newline}    %oznaczenie początku odpowiedzi końcowej (wypisanie wyniku)
\newcommand{\odpStop}{\newline}                                             %oznaczenie końca odpowiedzi końcowej (wypisanie wyniku)
\newcommand{\testStart}{\noindent \textbf{Test:}\newline} %ewentualne możliwe opcje odpowiedzi testowej: A. ? B. ? C. ? D. ? itd.
\newcommand{\testStop}{\newline} %koniec wprowadzania odpowiedzi testowych
\newcommand{\kluczStart}{\noindent \textbf{Test poprawna odpowiedź:}\newline} %klucz, poprawna odpowiedź pytania testowego (jedna literka): A lub B lub C lub D itd.
\newcommand{\kluczStop}{\newline} %koniec poprawnej odpowiedzi pytania testowego 
\newcommand{\wstawGrafike}[2]{\begin{figure}[h] \includegraphics[scale=#2] {#1} \end{figure}} %gdyby była potrzeba wstawienia obrazka, parametry: nazwa pliku, skala (jak nie wiesz co wpisać, to wpisz 1)

\begin{document}
\maketitle


\kategoria{Wikieł/Z1.44}
\zadStart{Zadanie z Wikieł Z 1.44 moja wersja nr [nrWersji]}
%[a]:[2,3,4,5,6,7,8,9,10,11]
%[b]:[2,3,4,5,6,7,8,9,10]
%[c]:[1,2,3,4,5,6,7,8]
%[d]=[b]**2
%[e]=4*[c]
%[f]=4*[a]
%[g]=[e]-[f]
%[i]=4*[a]*[c]
%[h]=[d]+[i]
%[pierw]=[g]**2+16*[h]
%[delta1]=pow([pierw],1/2)
%[delta]=[delta1].real
%[delta2]=int([delta])
%[delta].is_integer()==True and [e]>[f]
%[m1]=(-[g]+[delta])/-8
%[m2]=(-[g]-[delta])/-8
%[p]=[b]*[a]
%[s1]=[p]/[b]
%[s]=int([s1])
Wyznaczyć wartości parametru m, dla których równanie $(m-[a])x^2-[b]x+m-[c]=0$ ma dwa pierwiastki dodatnie.
\zadStop
\rozwStart{Mirella Narewska}{}
Aby równanie $(m-[a])x^2-[b]x+m-[c]=0$ miało dwa pierwiastki dodatnie, muszą zostać spełnione warunki:
$$1) m-[a]>0$$
$$2)\Delta>0$$
Wzory viete'a
$$3) x1+x2>0 \Leftrightarrow \frac{[b]}{m-[a]}>0$$
$$4) x1*x2>0 \Leftrightarrow \frac{m-[c]}{m-[a]}>0$$
\\
(1)
\\
$$m-[a]>0$$
$$m>[a]$$
$$m \in ([a],\infty)$$
\\
(2)
\\
$$\Delta=[b]^2-4\cdot(m-[a])\cdot(m-[c])=-4m^2+[g]m+[h]$$
$$-4m^2+[g]m+[h]>0$$
$$\Delta_m=[g]^2-4\cdot(-4)\cdot[h]=[pierw] \Rightarrow \sqrt{\Delta}=[delta2]$$
$$m1=[m1] \vee m2=[m2]$$
$$m \in ([m1],[m2])$$
\\
(3)
$$\frac{[b]}{m-[a]}>0$$
$$[b]\cdot(m-[a])>0$$
$$[b]m-[p]>0$$
$$m>[s]$$
\\
(4)
$$\frac{m-[c]}{m-[a]}>0$$
$$(m-[c])\cdot(m-[a])>0$$
$$m1=[c] \vee m2=[a]$$
$$m \in (-\infty,[a]) \cup ([c],\infty)$$
\odpStart
$m \in (-\infty)$
$$\in ([a],\infty) \cap ([m1],[m2]) \cap ([s],\infty) \cap (-\infty,[a]) \cup ([c],\infty) \Rightarrow m \in ([c],[m2])$$
\odpStop
\testStart
A.$m \in ([c],[m2])$
B. $m \in \emptyset$ \\
C. $m \in ([m1],[m2])$ \\
D. $m \in (-[a];[b])$ \\
\testStop
\kluczStart
A
\kluczStop
\end{document}