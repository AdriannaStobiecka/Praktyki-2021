\documentclass[12pt, a4paper]{article}
\usepackage[utf8]{inputenc}
\usepackage{polski}

\usepackage{amsthm}  %pakiet do tworzenia twierdzeń itp.
\usepackage{amsmath} %pakiet do niektórych symboli matematycznych
\usepackage{amssymb} %pakiet do symboli mat., np. \nsubseteq
\usepackage{amsfonts}
\usepackage{graphicx} %obsługa plików graficznych z rozszerzeniem png, jpg
\theoremstyle{definition} %styl dla definicji
\newtheorem{zad}{} 
\title{Multizestaw zadań}
\author{Robert Fidytek}
%\date{\today}
\date{}
\newcounter{liczniksekcji}
\newcommand{\kategoria}[1]{\section{#1}} %olreślamy nazwę kateforii zadań
\newcommand{\zadStart}[1]{\begin{zad}#1\newline} %oznaczenie początku zadania
\newcommand{\zadStop}{\end{zad}}   %oznaczenie końca zadania
%Makra opcjonarne (nie muszą występować):
\newcommand{\rozwStart}[2]{\noindent \textbf{Rozwiązanie (autor #1 , recenzent #2): }\newline} %oznaczenie początku rozwiązania, opcjonarnie można wprowadzić informację o autorze rozwiązania zadania i recenzencie poprawności wykonania rozwiązania zadania
\newcommand{\rozwStop}{\newline}                                            %oznaczenie końca rozwiązania
\newcommand{\odpStart}{\noindent \textbf{Odpowiedź:}\newline}    %oznaczenie początku odpowiedzi końcowej (wypisanie wyniku)
\newcommand{\odpStop}{\newline}                                             %oznaczenie końca odpowiedzi końcowej (wypisanie wyniku)
\newcommand{\testStart}{\noindent \textbf{Test:}\newline} %ewentualne możliwe opcje odpowiedzi testowej: A. ? B. ? C. ? D. ? itd.
\newcommand{\testStop}{\newline} %koniec wprowadzania odpowiedzi testowych
\newcommand{\kluczStart}{\noindent \textbf{Test poprawna odpowiedź:}\newline} %klucz, poprawna odpowiedź pytania testowego (jedna literka): A lub B lub C lub D itd.
\newcommand{\kluczStop}{\newline} %koniec poprawnej odpowiedzi pytania testowego 
\newcommand{\wstawGrafike}[2]{\begin{figure}[h] \includegraphics[scale=#2] {#1} \end{figure}} %gdyby była potrzeba wstawienia obrazka, parametry: nazwa pliku, skala (jak nie wiesz co wpisać, to wpisz 1)

\begin{document}
\maketitle

\kategoria{Wikieł/Z5.26l}

\zadStart{Zadanie z Wikieł Z 5.26 l) moja wersja nr [nrWersji]}
%[a]:[1,2,3,4,5,6,7,8,9,10]
%[b]=3*[a]
%[f]=[b]/4
%[g]=- round([f],2)
%[h]=([g])**3*abs([g]+[a])
%[i]=round([h],2)
%[j]=(-4)**3*abs(-4+[a])
%[k]=1*abs(1+[a])
%[l]=max(0,[j],[k])
%[m]=min(0,[j],[k])
%[b]!=4 and [g]>-4 and [g]<1
Wyznaczyć wartość największą oraz wartość najmniejszą funkcji w przedziale. 
$$f(x) = x^3\mid x + [a] \mid, \langle-4,1\rangle$$
\zadStop

\rozwStart{Natalia Danieluk}{}
Funkcja $f$ jest ciągła w przedziale $\langle-4,1\rangle$. 
$$ f(x) = x^3\mid x + [a] \mid =  
\begin{cases} 
x^3(x + [a]), & \text{dla} \ x + [a] \ge 0 \\
- x^3(x + [a]), & \text{dla}\ x + [a] < 0
\end{cases} 
=  
\begin{cases} 
x^3(x + [a]), & \text{dla} \ x \ge - [a] \\
- x^3(x + [a]), & \text{dla}\ x < - [a]
\end{cases} 
$$
Wartość największą $M$ i najmniejszą $m$ znajdziemy wśród ekstremów tej funkcji w przedziale $\langle-4,1\rangle$ oraz jej wartości na końcach przedziału, tj. $f(-4)$ i $f(1)$. \\
A zatem obliczamy pochodną i wyznaczamy jej miejsca zerowe:
$$ f'(x) = 
\begin{cases} 
4x^3 + [b]x^2, & \text{dla} \ x \ge - [a] \\
- (4x^3 + [b]x^2), & \text{dla}\ x < - [a]
\end{cases} 
$$
$$ f'(x) = 0 \Leftrightarrow 4x^3 + [b]x^2 = 0 \Leftrightarrow x^2(4x + [b]) = 0 \Leftrightarrow x = 0 \quad \lor \quad x = -\frac{[b]}{4} \approx [g] $$ 
Oba punkty należą do przedziału. Sprawdzamy wartości funkcji w tych punktach oraz na końcach naszego przedziału: \\
$$ f(0) = 0,\quad f([g]) \approx [i],\quad f(-4) = [j],\quad f(1) = [k] $$
\rozwStop
\rozwStop

\odpStart
Wartość największa $M$ funkcji $f$ w przedziale $\langle-4,1\rangle$ to $[l]$, natomiast wartość najmniejsza $m$ to $[m]$.
\odpStop

\testStart
A. $M=[m], m=[l]$
B. $M=[g], m=0$
C. $M=[l], m=[m]$
D. $M=2, m=-2$
\testStop

\kluczStart
C
\kluczStop

\end{document}