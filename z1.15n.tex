\documentclass[12pt, a4paper]{article}
\usepackage[utf8]{inputenc}
\usepackage{polski}

\usepackage{amsthm}  %pakiet do tworzenia twierdzeń itp.
\usepackage{amsmath} %pakiet do niektórych symboli matematycznych
\usepackage{amssymb} %pakiet do symboli mat., np. \nsubseteq
\usepackage{amsfonts}
\usepackage{graphicx} %obsługa plików graficznych z rozszerzeniem png, jpg
\theoremstyle{definition} %styl dla definicji
\newtheorem{zad}{} 
\title{Multizestaw zadań}
\author{Robert Fidytek}
%\date{\today}
\date{}
\newcounter{liczniksekcji}
\newcommand{\kategoria}[1]{\section{#1}} %olreślamy nazwę kateforii zadań
\newcommand{\zadStart}[1]{\begin{zad}#1\newline} %oznaczenie początku zadania
\newcommand{\zadStop}{\end{zad}}   %oznaczenie końca zadania
%Makra opcjonarne (nie muszą występować):
\newcommand{\rozwStart}[2]{\noindent \textbf{Rozwiązanie (autor #1 , recenzent #2): }\newline} %oznaczenie początku rozwiązania, opcjonarnie można wprowadzić informację o autorze rozwiązania zadania i recenzencie poprawności wykonania rozwiązania zadania
\newcommand{\rozwStop}{\newline}                                            %oznaczenie końca rozwiązania
\newcommand{\odpStart}{\noindent \textbf{Odpowiedź:}\newline}    %oznaczenie początku odpowiedzi końcowej (wypisanie wyniku)
\newcommand{\odpStop}{\newline}                                             %oznaczenie końca odpowiedzi końcowej (wypisanie wyniku)
\newcommand{\testStart}{\noindent \textbf{Test:}\newline} %ewentualne możliwe opcje odpowiedzi testowej: A. ? B. ? C. ? D. ? itd.
\newcommand{\testStop}{\newline} %koniec wprowadzania odpowiedzi testowych
\newcommand{\kluczStart}{\noindent \textbf{Test poprawna odpowiedź:}\newline} %klucz, poprawna odpowiedź pytania testowego (jedna literka): A lub B lub C lub D itd.
\newcommand{\kluczStop}{\newline} %koniec poprawnej odpowiedzi pytania testowego 
\newcommand{\wstawGrafike}[2]{\begin{figure}[h] \includegraphics[scale=#2] {#1} \end{figure}} %gdyby była potrzeba wstawienia obrazka, parametry: nazwa pliku, skala (jak nie wiesz co wpisać, to wpisz 1)

\begin{document}
\maketitle


\kategoria{Wikieł/Z1.15 n}
\zadStart{Zadanie z Wikieł Z 1.15 n) moja wersja nr [nrWersji]}
%[a]:[2,3,4,5,6,7,8,9]
%[b]:[1,2,3,4,5,6,7,8,9]
%[c]=[a]+[a]
%math.gcd([b],[a])==1 and math.gcd([c],[b])==1 and [c]!=0


Rozwiązać nierówności  $|[a]x+[b]| < [a]|x|$.
\zadStop
\rozwStart{Joanna Świerzbin}{}
$$|[a]x+[b]| < [a]|x|$$
$$[a]x+[b] = 0 \Rightarrow [a]x = -[b] \Rightarrow x = \frac{-[b]}{[a]} $$
$$[a]x = 0 \Rightarrow x=0 $$
\begin{enumerate}
\item $$x \in \left( -\infty , \frac{-[b]}{[a]} \right) \land -\left([a]x+[b]\right)<-[a]x$$
$$x \in \left( -\infty , \frac{-[b]}{[a]} \right) \land -[a]x-[b]<-[a]x$$
$$x \in \left( -\infty , \frac{-[b]}{[a]} \right) \land 0<[b]$$
$$x \in \left( -\infty , \frac{-[b]}{[a]} \right) \land x \in \mathbb{R}$$
$$x \in \left( -\infty , \frac{-[b]}{[a]} \right)$$

\item $$x \in \left[ \frac{-[b]}{[a]},0 \right) \land ([a]x+[b])<-[a]x$$
$$x \in \left[ \frac{-[b]}{[a]},0 \right) \land [c]x<-[b]$$
$$x \in \left[ \frac{-[b]}{[a]},0 \right) \land x<\frac{-[b]}{[c]}$$
$$x \in \left[ \frac{-[b]}{[a]},0 \right) \land x\in \left(-\infty,\frac{-[b]}{[c]}\right)$$
$$x \in \left[ \frac{-[b]}{[a]} ,\frac{-[b]}{[c]}\right)$$

\item $$x \in \left[0 , \infty \right) \land ([a]x+[b])<[a]x$$
 $$x \in \left[0 , \infty \right) \land 0<-[b]$$
 $$x \in \left[0 , \infty \right) \land x \in \emptyset$$
 $$ x \in \emptyset$$
\end{enumerate}
$$x \in \left( -\infty , \frac{-[b]}{[a]} \right) \vee x \in \left[ \frac{-[b]}{[a]} ,\frac{-[b]}{[c]}\right) \vee  x \in \emptyset $$
$$x \in \left( -\infty , \frac{-[b]}{[c]}\right) $$
\rozwStop
\odpStart
$x \in \left( -\infty , \frac{-[b]}{[c]}\right) $
\odpStop
\testStart
A. $x \in \left( -\infty , \frac{-[b]}{[c]}\right) $
B. $x \in \left( -\infty , \frac{[b]}{[c]}\right) $
C. $x \in \mathbb{R} $
D. $x \in \left( \frac{-[b]}{[a]} , \frac{-[b]}{[c]}\right) $
E. $ x \in \emptyset $
\testStop
\kluczStart
A
\kluczStop



\end{document}