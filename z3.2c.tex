\documentclass[12pt, a4paper]{article}
\usepackage[utf8]{inputenc}
\usepackage{polski}

\usepackage{amsthm}  %pakiet do tworzenia twierdzeń itp.
\usepackage{amsmath} %pakiet do niektórych symboli matematycznych
\usepackage{amssymb} %pakiet do symboli mat., np. \nsubseteq
\usepackage{amsfonts}
\usepackage{graphicx} %obsługa plików graficznych z rozszerzeniem png, jpg
\theoremstyle{definition} %styl dla definicji
\newtheorem{zad}{} 
\title{Multizestaw zadań}
\author{Robert Fidytek}
%\date{\today}
\date{}
\newcounter{liczniksekcji}
\newcommand{\kategoria}[1]{\section{#1}} %olreślamy nazwę kateforii zadań
\newcommand{\zadStart}[1]{\begin{zad}#1\newline} %oznaczenie początku zadania
\newcommand{\zadStop}{\end{zad}}   %oznaczenie końca zadania
%Makra opcjonarne (nie muszą występować):
\newcommand{\rozwStart}[2]{\noindent \textbf{Rozwiązanie (autor #1 , recenzent #2): }\newline} %oznaczenie początku rozwiązania, opcjonarnie można wprowadzić informację o autorze rozwiązania zadania i recenzencie poprawności wykonania rozwiązania zadania
\newcommand{\rozwStop}{\newline}                                            %oznaczenie końca rozwiązania
\newcommand{\odpStart}{\noindent \textbf{Odpowiedź:}\newline}    %oznaczenie początku odpowiedzi końcowej (wypisanie wyniku)
\newcommand{\odpStop}{\newline}                                             %oznaczenie końca odpowiedzi końcowej (wypisanie wyniku)
\newcommand{\testStart}{\noindent \textbf{Test:}\newline} %ewentualne możliwe opcje odpowiedzi testowej: A. ? B. ? C. ? D. ? itd.
\newcommand{\testStop}{\newline} %koniec wprowadzania odpowiedzi testowych
\newcommand{\kluczStart}{\noindent \textbf{Test poprawna odpowiedź:}\newline} %klucz, poprawna odpowiedź pytania testowego (jedna literka): A lub B lub C lub D itd.
\newcommand{\kluczStop}{\newline} %koniec poprawnej odpowiedzi pytania testowego 
\newcommand{\wstawGrafike}[2]{\begin{figure}[h] \includegraphics[scale=#2] {#1} \end{figure}} %gdyby była potrzeba wstawienia obrazka, parametry: nazwa pliku, skala (jak nie wiesz co wpisać, to wpisz 1)

\begin{document}
\maketitle


\kategoria{Wikieł/Z3.2c}
\zadStart{Zadanie z Wikieł Z 3.2c moja wersja nr [nrWersji]}
%[p1]:[2,3,4,5,6,7,8,9,10,11,12]
%[p2]:[2,3,4,5,6,7,8,9,10,11,12]
Zbadać ograniczoność ciągu $a_{n}=\cos(n^{3}-n^{2}+n)\frac{[p1]n+[p2]}{n}$.
\zadStop

\rozwStart{Maja Szabłowska}{}
Zauważamy, że funkcja $\cos$ jest ograniczona z góry przez 1 oraz z dołu przez -1. 
Zauważamy również, że $\frac{[p1]n+[p2]}{n}=[p1]+\frac{[p2]}{n}.$ Zatem ciąg jest ogarniczony przez -[p1] oraz [p1].
\rozwStop


\odpStart
$a_{n}$ ograniczony z góry przez $[p1]$ i z dołu przez -[p1].
\odpStop
\testStart
A.$a_{n}$ ograniczony z góry przez $[p1]$ i z dołu przez -[p1].
B.$a_{n}$ ograniczony z góry przez 1 i z dołu przez 0.
C.$a_{n}$ ograniczony z góry przez $[p2]$.
D.$a_{n}$ ograniczony z dołu przez 0.
E.$a_{n}$ ograniczony z góry przez 3 i z dołu przez 0.
F.$a_{n}$ ograniczony z góry przez 2 i z dołu przez 0.
\testStop
\kluczStart
A
\kluczStop



\end{document}
