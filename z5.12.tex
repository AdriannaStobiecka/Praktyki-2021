\documentclass[12pt, a4paper]{article}
\usepackage[utf8]{inputenc}
\usepackage{polski}

\usepackage{amsthm}  %pakiet do tworzenia twierdzeń itp.
\usepackage{amsmath} %pakiet do niektórych symboli matematycznych
\usepackage{amssymb} %pakiet do symboli mat., np. \nsubseteq
\usepackage{amsfonts}
\usepackage{graphicx} %obsługa plików graficznych z rozszerzeniem png, jpg
\theoremstyle{definition} %styl dla definicji
\newtheorem{zad}{} 
\title{Multizestaw zadań}
\author{Robert Fidytek}
%\date{\today}
\date{}
\newcounter{liczniksekcji}
\newcommand{\kategoria}[1]{\section{#1}} %olreślamy nazwę kateforii zadań
\newcommand{\zadStart}[1]{\begin{zad}#1\newline} %oznaczenie początku zadania
\newcommand{\zadStop}{\end{zad}}   %oznaczenie końca zadania
%Makra opcjonarne (nie muszą występować):
\newcommand{\rozwStart}[2]{\noindent \textbf{Rozwiązanie (autor #1 , recenzent #2): }\newline} %oznaczenie początku rozwiązania, opcjonarnie można wprowadzić informację o autorze rozwiązania zadania i recenzencie poprawności wykonania rozwiązania zadania
\newcommand{\rozwStop}{\newline}                                            %oznaczenie końca rozwiązania
\newcommand{\odpStart}{\noindent \textbf{Odpowiedź:}\newline}    %oznaczenie początku odpowiedzi końcowej (wypisanie wyniku)
\newcommand{\odpStop}{\newline}                                             %oznaczenie końca odpowiedzi końcowej (wypisanie wyniku)
\newcommand{\testStart}{\noindent \textbf{Test:}\newline} %ewentualne możliwe opcje odpowiedzi testowej: A. ? B. ? C. ? D. ? itd.
\newcommand{\testStop}{\newline} %koniec wprowadzania odpowiedzi testowych
\newcommand{\kluczStart}{\noindent \textbf{Test poprawna odpowiedź:}\newline} %klucz, poprawna odpowiedź pytania testowego (jedna literka): A lub B lub C lub D itd.
\newcommand{\kluczStop}{\newline} %koniec poprawnej odpowiedzi pytania testowego 
\newcommand{\wstawGrafike}[2]{\begin{figure}[h] \includegraphics[scale=#2] {#1} \end{figure}} %gdyby była potrzeba wstawienia obrazka, parametry: nazwa pliku, skala (jak nie wiesz co wpisać, to wpisz 1)

\begin{document}
\maketitle


\kategoria{Wikieł/Z5.12}
\zadStart{Zadanie z Wikieł Z 5.12 moja wersja nr [nrWersji]}
%[x]:[2,3,4,5,6,7,8,9,10,11,12,13]
%[y]:[2,3,4,5,6,7,8,9,10,11,12,13]
%[a]=random.randint(2,10)
%[b]=random.randint(2,10)
%[c]=[a]*[b]
%[e]=2*[b]*[b]
%[f]=[c]*2
%[b2]=[b]*[b]
Obliczyć drugą pochodną funkcji:\\
$f(x)=\frac{[a]-[b]cos(x)}{[b]sin(x)}$
\zadStop
\rozwStart{Katarzyna Filipowicz}{}
$$
f'(x)=\frac{[b]sin(x)\cdot[b] sin(x)-[b]cos(x)([a]-[b]cos(x))}{[b]^2sin^2(x)}=
$$ $$
=\frac{[b2]sin^2(x)-[a]\cdot [b] cos(x)+[b2]cos^2(x)}{[b]^2sin^2(x)}=
$$ $$
=\frac{[b2]-[c]cos(x)}{[b]^2sin^2(x)}
$$ $$
f''(x)=\frac{[c]\cdot sin(x)\cdot[b]^2 sin^2(x)-2\cdot[b]^2\cdot sin(x)cos(x)([b2]-[c]cos(x))}{[b]^4sin^4(x)}=
$$ $$
=\frac{[c]sin^2(x)-[e]cos(x)+2\cdot [c]cos^2(x)}{[b]^2sin^3(x)}=
$$ $$
=\frac{[c]sin^2(x)-[e]cos(x)+[f]cos^2(x)}{[b2]sin^3(x)}
$$
\rozwStop
\odpStart
$f''(x)=\frac{[c]sin^2(x)-[e]cos(x)+[f]cos^2(x)}{[b2]sin^3(x)}$
\odpStop
\testStart
A. $f''(x)=\frac{[c]sin^2(x)-[e]cos(x)+[f]cos^2(x)}{[b2]sin^3(x)}$\\
B. $f''(x)=\frac{[c]sin^2(x)-[e]cos(x)+[f]cos^2(x)}{[b]sin^3(x)}$\\
C. $f''(x)=\frac{[c]sin^2(x)-[e]cos(x)+[f]cos^2(x)}{[b]sin^2(x)}$\\
D. $f''(x)=\frac{[c]sin^2(x)+[e]cos(x)+[f]cos^2(x)}{[b2]sin^3(x)}$\\
E. $f''(x)=\frac{[a]sin^2(x)-[e]cos(x)+[f]cos(x)}{[b2]sin^3(x)}$\\
F. $f''(x)=\frac{[c]sin^2(x)+[e]cos(x)+[b]cos(x)}{[b2]sin^3(x)}$
\testStop
\kluczStart
A
\kluczStop



\end{document}