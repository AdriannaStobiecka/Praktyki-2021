\documentclass[12pt, a4paper]{article}
\usepackage[utf8]{inputenc}
\usepackage{polski}

\usepackage{amsthm}  %pakiet do tworzenia twierdzeń itp.
\usepackage{amsmath} %pakiet do niektórych symboli matematycznych
\usepackage{amssymb} %pakiet do symboli mat., np. \nsubseteq
\usepackage{amsfonts}
\usepackage{graphicx} %obsługa plików graficznych z rozszerzeniem png, jpg
\theoremstyle{definition} %styl dla definicji
\newtheorem{zad}{} 
\title{Multizestaw zadań}
\author{Robert Fidytek}
%\date{\today}
\date{}\documentclass[12pt, a4paper]{article}
\usepackage[utf8]{inputenc}
\usepackage{polski}

\usepackage{amsthm}  %pakiet do tworzenia twierdzeń itp.
\usepackage{amsmath} %pakiet do niektórych symboli matematycznych
\usepackage{amssymb} %pakiet do symboli mat., np. \nsubseteq
\usepackage{amsfonts}
\usepackage{graphicx} %obsługa plików graficznych z rozszerzeniem png, jpg
\theoremstyle{definition} %styl dla definicji
\newtheorem{zad}{} 
\title{Multizestaw zadań}
\author{Robert Fidytek}
%\date{\today}
\date{}
\newcounter{liczniksekcji}
\newcommand{\kategoria}[1]{\section{#1}} %olreślamy nazwę kateforii zadań
\newcommand{\zadStart}[1]{\begin{zad}#1\newline} %oznaczenie początku zadania
\newcommand{\zadStop}{\end{zad}}   %oznaczenie końca zadania
%Makra opcjonarne (nie muszą występować):
\newcommand{\rozwStart}[2]{\noindent \textbf{Rozwiązanie (autor #1 , recenzent #2): }\newline} %oznaczenie początku rozwiązania, opcjonarnie można wprowadzić informację o autorze rozwiązania zadania i recenzencie poprawności wykonania rozwiązania zadania
\newcommand{\rozwStop}{\newline}                                            %oznaczenie końca rozwiązania
\newcommand{\odpStart}{\noindent \textbf{Odpowiedź:}\newline}    %oznaczenie początku odpowiedzi końcowej (wypisanie wyniku)
\newcommand{\odpStop}{\newline}                                             %oznaczenie końca odpowiedzi końcowej (wypisanie wyniku)
\newcommand{\testStart}{\noindent \textbf{Test:}\newline} %ewentualne możliwe opcje odpowiedzi testowej: A. ? B. ? C. ? D. ? itd.
\newcommand{\testStop}{\newline} %koniec wprowadzania odpowiedzi testowych
\newcommand{\kluczStart}{\noindent \textbf{Test poprawna odpowiedź:}\newline} %klucz, poprawna odpowiedź pytania testowego (jedna literka): A lub B lub C lub D itd.
\newcommand{\kluczStop}{\newline} %koniec poprawnej odpowiedzi pytania testowego 
\newcommand{\wstawGrafike}[2]{\begin{figure}[h] \includegraphics[scale=#2] {#1} \end{figure}} %gdyby była potrzeba wstawienia obrazka, parametry: nazwa pliku, skala (jak nie wiesz co wpisać, to wpisz 1)

\begin{document}
\maketitle


\kategoria{Wikieł/Z1.129r}
\zadStart{Zadanie z Wikieł Z 1.129 r) moja wersja nr [nrWersji]}
%[p1]:[2,3,4,5,6,7,8,9,10]
%[p2]:[2,3,4,5,6,7,8,9,10]
%[p3]:[2,3,4,5,6,7,8,9,10]
%[del]=[p2]*[p2]-4*[p1]*[p3]
%[2p1]=2*[p1]
%[2p2]=2*[p2]
%[del]<0 and math.gcd([p1],[p2])==1



Wyznaczyć dziedzinę naturalną funkcji.
$$f(x)=\arcsin\frac{[p1]x^{2}-[p2]x+[p3]}{[p1]x^{2}+[p2]x+[p3]}$$
\zadStop

\rozwStart{Maja Szabłowska}{}
$$-1\leq\frac{[p1]x^{2}-[p2]x+[p3]}{[p1]x^{2}+[p2]x+[p3]}\leq 1$$
Mnożymy przez mianownik bez zmiany znaków, gdyż jest on zawsze dodatni($\Delta=[del]<0$ i $[p1]>0$).
$$-([p1]x^{2}+[p2]x+[p3])\leq[p1]x^{2}-[p2]x+[p3]\leq[p1]x^{2}+[p2]x+[p3]$$
Rozdzielamy na dwie nierówności.
\begin{enumerate}
    \item $$-([p1]x^{2}+[p2]x+[p3])\leq[p1]x^{2}-[p2]x+[p3]$$
$$-[2p1]x^{2}-[2p2]x\leq 0 \iff -x([2p1]x+[2p2])\leq 0$$
$$ x \in \left(-\infty, -\frac{[p2]}{[p1]}\right]\cup[0,\infty)$$
\item $$[p1]x^{2}-[p2]x+[p3]\leq[p1]x^{2}+[p2]x+[p3]$$
$$-[2p2]x\leq0 \iff x\geq0 \iff x\in[0,\infty)$$

\end{enumerate}
Ostatecznie 
$$x\in[0,\infty)$$

\rozwStop
\odpStart
$x\in[0,\infty)$
\odpStop
\testStart
A.$x\in[0,\infty)$
B.$x\in[e^{[p2]},\infty)$
C.$x\in(-\infty, 0)$
D.$x\in(-\infty, -[p2]] \cup [\ln[p1],\infty)$
E.$x\in[[p1],\infty)$
F.$x\in([p2],\infty)$
G.$x\in\emptyset$

\testStop
\kluczStart
A
\kluczStop



\end{document}
