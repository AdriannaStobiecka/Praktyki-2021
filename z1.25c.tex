\documentclass[12pt, a4paper]{article}
\usepackage[utf8]{inputenc}
\usepackage{polski}

\usepackage{amsthm}  %pakiet do tworzenia twierdzeń itp.
\usepackage{amsmath} %pakiet do niektórych symboli matematycznych
\usepackage{amssymb} %pakiet do symboli mat., np. \nsubseteq
\usepackage{amsfonts}
\usepackage{graphicx} %obsługa plików graficznych z rozszerzeniem png, jpg
\theoremstyle{definition} %styl dla definicji
\newtheorem{zad}{} 
\title{Multizestaw zadań}
\author{Robert Fidytek}
%\date{\today}
\date{}
\newcounter{liczniksekcji}
\newcommand{\kategoria}[1]{\section{#1}} %olreślamy nazwę kateforii zadań
\newcommand{\zadStart}[1]{\begin{zad}#1\newline} %oznaczenie początku zadania
\newcommand{\zadStop}{\end{zad}}   %oznaczenie końca zadania
%Makra opcjonarne (nie muszą występować):
\newcommand{\rozwStart}[2]{\noindent \textbf{Rozwiązanie (autor #1 , recenzent #2): }\newline} %oznaczenie początku rozwiązania, opcjonarnie można wprowadzić informację o autorze rozwiązania zadania i recenzencie poprawności wykonania rozwiązania zadania
\newcommand{\rozwStop}{\newline}                                            %oznaczenie końca rozwiązania
\newcommand{\odpStart}{\noindent \textbf{Odpowiedź:}\newline}    %oznaczenie początku odpowiedzi końcowej (wypisanie wyniku)
\newcommand{\odpStop}{\newline}                                             %oznaczenie końca odpowiedzi końcowej (wypisanie wyniku)
\newcommand{\testStart}{\noindent \textbf{Test:}\newline} %ewentualne możliwe opcje odpowiedzi testowej: A. ? B. ? C. ? D. ? itd.
\newcommand{\testStop}{\newline} %koniec wprowadzania odpowiedzi testowych
\newcommand{\kluczStart}{\noindent \textbf{Test poprawna odpowiedź:}\newline} %klucz, poprawna odpowiedź pytania testowego (jedna literka): A lub B lub C lub D itd.
\newcommand{\kluczStop}{\newline} %koniec poprawnej odpowiedzi pytania testowego 
\newcommand{\wstawGrafike}[2]{\begin{figure}[h] \includegraphics[scale=#2] {#1} \end{figure}} %gdyby była potrzeba wstawienia obrazka, parametry: nazwa pliku, skala (jak nie wiesz co wpisać, to wpisz 1)

\begin{document}
\maketitle


\kategoria{Wikieł/Z1.25c}
\zadStart{Zadanie z Wikieł Z 1.25 c) moja wersja nr [nrWersji]}
%[a]:[2,3,4,5,6,7,8,9]
%[b]:[1,2,3,4,5,6,7,8,9]
%[c]:[1,2,3,4,5,6,7,8,9]
%[cbmm]=-[c]-[b]
%[cbm]=[c]-[b]
%math.gcd([cbmm],[a])==1 and math.gcd([cbm],[a])==1 and [cbmm]!=0 and [cbm]!=0


Wyznacz dziedzinę funkcji określonej podanym wzorem  $f(x)=\sqrt{|[a]x+[b]|-[c]}$.
\zadStop
\rozwStart{Joanna Świerzbin}{}
$$f(x)=\sqrt{|[a]x+[b]|-[c]}$$
$$|[a]x+[b]|-[c] \geq 0$$
$$|[a]x+[b]| \geq [c]$$
$$[a]x+[b] \leq -[c] \vee [a]x+[b] \geq [c]$$
$$[a]x \leq [cbmm] \vee [a]x \geq [cbm]$$
$$x \leq \frac{[cbmm]}{[a]} \vee x \geq \frac{[cbm]}{[a]}$$
$$x \in \left( -\infty , \frac{[cbmm]}{[a]} \right] \cup \left[\frac{[cbm]}{[a]}, \infty \right)$$
\rozwStop
\odpStart
$x \in \left( -\infty , \frac{[cbmm]}{[a]} \right] \cup \left[\frac{[cbm]}{[a]}, \infty \right)$
\odpStop
\testStart
A.$x \in \left( -\infty , \frac{[cbmm]}{[a]} \right] \cup \left[\frac{[cbm]}{[a]}, \infty \right)$\\
B.$x \in \left( -\infty , \frac{[cbmm]}{[a]} \right]$ \\
C. $x \in  \left[\frac{[cbm]}{[a]}, \infty \right)$\\
D. $x \in \left(\frac{[cbmm]}{[a]} , \frac{[cbm]}{[a]} \right)$\\
E. $x \in \mathbb{R}$\\
F. $x \in \emptyset $\\
\testStop
\kluczStart
A
\kluczStop



\end{document}