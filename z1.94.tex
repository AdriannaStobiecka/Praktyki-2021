\documentclass[12pt, a4paper]{article}
\usepackage[utf8]{inputenc}
\usepackage{polski}

\usepackage{amsthm}  %pakiet do tworzenia twierdzeń itp.
\usepackage{amsmath} %pakiet do niektórych symboli matematycznych
\usepackage{amssymb} %pakiet do symboli mat., np. \nsubseteq
\usepackage{amsfonts}
\usepackage{graphicx} %obsługa plików graficznych z rozszerzeniem png, jpg
\theoremstyle{definition} %styl dla definicji
\newtheorem{zad}{} 
\title{Multizestaw zadań}
\author{Robert Fidytek}
%\date{\today}
\date{}
\newcounter{liczniksekcji}
\newcommand{\kategoria}[1]{\section{#1}} %olreślamy nazwę kateforii zadań
\newcommand{\zadStart}[1]{\begin{zad}#1\newline} %oznaczenie początku zadania
\newcommand{\zadStop}{\end{zad}}   %oznaczenie końca zadania
%Makra opcjonarne (nie muszą występować):
\newcommand{\rozwStart}[2]{\noindent \textbf{Rozwiązanie (autor #1 , recenzent #2): }\newline} %oznaczenie początku rozwiązania, opcjonarnie można wprowadzić informację o autorze rozwiązania zadania i recenzencie poprawności wykonania rozwiązania zadania
\newcommand{\rozwStop}{\newline}                                            %oznaczenie końca rozwiązania
\newcommand{\odpStart}{\noindent \textbf{Odpowiedź:}\newline}    %oznaczenie początku odpowiedzi końcowej (wypisanie wyniku)
\newcommand{\odpStop}{\newline}                                             %oznaczenie końca odpowiedzi końcowej (wypisanie wyniku)
\newcommand{\testStart}{\noindent \textbf{Test:}\newline} %ewentualne możliwe opcje odpowiedzi testowej: A. ? B. ? C. ? D. ? itd.
\newcommand{\testStop}{\newline} %koniec wprowadzania odpowiedzi testowych
\newcommand{\kluczStart}{\noindent \textbf{Test poprawna odpowiedź:}\newline} %klucz, poprawna odpowiedź pytania testowego (jedna literka): A lub B lub C lub D itd.
\newcommand{\kluczStop}{\newline} %koniec poprawnej odpowiedzi pytania testowego 
\newcommand{\wstawGrafike}[2]{\begin{figure}[h] \includegraphics[scale=#2] {#1} \end{figure}} %gdyby była potrzeba wstawienia obrazka, parametry: nazwa pliku, skala (jak nie wiesz co wpisać, to wpisz 1)

\begin{document}
\maketitle


\kategoria{Wikieł/Z1.94}
\zadStart{Zadanie z Wikieł Z 1.94 moja wersja nr [nrWersji]}
%[c]:[2,3,4,5,6,7,8,9]
%[d]:[2,3,4,5,6,7,8,9]
%[cd]=[c]*[d]
%[d1]=[d]-1
%[calosci]=[cd]//[d1]
%[reszta]=[cd]%[d1]
%[cd]/[d1]>[c] and math.gcd([cd],[d1])==1 and [d1]!=1 and [d1]<[cd]
Znaleźć $a$, wiedząc, że dla dowolnego $b\in\mathbb{R}_+\setminus\{1\}$ zachodzi:
$$\log_b{(a-[c])}=\log_b{a}-\log_b{[d]}$$
\zadStop
\rozwStart{Adrianna Stobiecka}{}
Zakładamy, że $a-[c]>0$ oraz $a>0$, zatem otrzymujemy założenie $a\in([c],\infty)$.
$$\log_b{(a-[c])}=\log_b{a}-\log_b{[d]}\Leftrightarrow\log_b{(a-[c])}=\log_b{\frac{a}{[d]}}\Leftrightarrow a-[c]=\frac{a}{[d]}$$
$$\Leftrightarrow [d]a-[cd]=a\Leftrightarrow[d]a-a=[cd]\Leftrightarrow[d1]a=[cd]\Leftrightarrow a=[calosci]\frac{[reszta]}{[d1]}$$
Zauważamy, że $a=[calosci]\frac{[reszta]}{[d1]}\in([c],\infty)$.
\rozwStop
\odpStart
$[calosci]\frac{[reszta]}{[d1]}$
\odpStop
\testStart
A.$\frac{[reszta]}{[d1]}$
B.$-[calosci]$
C.$-[calosci]\frac{[reszta]}{[d1]}$
D.$[calosci]$
E.$0$
F.$[calosci]\frac{[reszta]}{[d1]}$
G.$\frac{[d1]}{[cd]}$
H.$-\frac{[d1]}{[cd]}$
I.$-\frac{[reszta]}{[d1]}$
\testStop
\kluczStart
F
\kluczStop



\end{document}
