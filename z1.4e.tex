\documentclass[12pt, a4paper]{article}
\usepackage[utf8]{inputenc}
\usepackage{polski}

\usepackage{amsthm}  %pakiet do tworzenia twierdzeń itp.
\usepackage{amsmath} %pakiet do niektórych symboli matematycznych
\usepackage{amssymb} %pakiet do symboli mat., np. \nsubseteq
\usepackage{amsfonts}
\usepackage{graphicx} %obsługa plików graficznych z rozszerzeniem png, jpg
\theoremstyle{definition} %styl dla definicji
\newtheorem{zad}{} 
\title{Multizestaw zadań}
\author{Laura Mieczkowska}
%\date{\today}
\date{}
\newcounter{liczniksekcji}
\newcommand{\kategoria}[1]{\section{#1}} %olreślamy nazwę kateforii zadań
\newcommand{\zadStart}[1]{\begin{zad}#1\newline} %oznaczenie początku zadania
\newcommand{\zadStop}{\end{zad}}   %oznaczenie końca zadania
%Makra opcjonarne (nie muszą występować):
\newcommand{\rozwStart}[2]{\noindent \textbf{Rozwiązanie (autor #1 , recenzent #2): }\newline} %oznaczenie początku rozwiązania, opcjonarnie można wprowadzić informację o autorze rozwiązania zadania i recenzencie poprawności wykonania rozwiązania zadania
\newcommand{\rozwStop}{\newline}                                            %oznaczenie końca rozwiązania
\newcommand{\odpStart}{\noindent \textbf{Odpowiedź:}\newline}    %oznaczenie początku odpowiedzi końcowej (wypisanie wyniku)
\newcommand{\odpStop}{\newline}                                             %oznaczenie końca odpowiedzi końcowej (wypisanie wyniku)
\newcommand{\testStart}{\noindent \textbf{Test:}\newline} %ewentualne możliwe opcje odpowiedzi testowej: A. ? B. ? C. ? D. ? itd.
\newcommand{\testStop}{\newline} %koniec wprowadzania odpowiedzi testowych
\newcommand{\kluczStart}{\noindent \textbf{Test poprawna odpowiedź:}\newline} %klucz, poprawna odpowiedź pytania testowego (jedna literka): A lub B lub C lub D itd.
\newcommand{\kluczStop}{\newline} %koniec poprawnej odpowiedzi pytania testowego 
\newcommand{\wstawGrafike}[2]{\begin{figure}[h] \includegraphics[scale=#2] {#1} \end{figure}} %gdyby była potrzeba wstawienia obrazka, parametry: nazwa pliku, skala (jak nie wiesz co wpisać, to wpisz 1)

\begin{document}
\maketitle


\kategoria{Wikieł/Z1.4e}
\zadStart{Zadanie z Wikieł Z 1.4 e) moja wersja nr [nrWersji]}
%[a]:[2,3,4,5,6,7,8,9,10]
%[b]:[2,3,4,5,6,7]
%[e]:[2,3,4,5,6,7,8,9,10]
%[f]:[2,3,4,5,6,7,8,9,10]
%[c]=[b]**3
%[licz]=[e]*[e]
%[mian]=[a]*[f]
%[pierw1]=math.sqrt([licz])
%[pierw]=int([pierw1])
%[pierw2]=math.sqrt([mian])
%[pierw3]=int([pierw2])
%[ulamek]=[pierw]/[pierw3]
%[w1]=[a]*[pierw3]-[pierw]
%[w]=[w1]/[pierw3]
%[pierw1].is_integer()==True and [pierw2].is_integer()==True and [a]!=[e] and [f]!=[e] and [licz]!=[mian] and [a]>[ulamek] and [ulamek].is_integer()==False and [w].is_integer()==False and math.gcd([w1],[pierw3])==1
Obliczyć wartość wyrażenia $[a]\sqrt{[c]}:[b]^{1,5}-\big(\frac{[a]}{[e]}\big)^{-\frac{1}{2}}\cdot\big(\frac{[f]}{[e]}\big)^{-\frac{1}{2}}$.
\zadStop
\rozwStart{Laura Mieczkowska}{}
$$[a]\sqrt{[c]}:[b]^{1,5}-\bigg(\frac{[a]}{[e]}\bigg)^{-\frac{1}{2}}\cdot\bigg(\frac{[f]}{[e]}\bigg)^{-\frac{1}{2}}=
[a]\cdot\big([b]^3\big)^{\frac{1}{2}}\cdot[b]^{-\frac{3}{2}}-\bigg(\frac{[e]}{[a]}\bigg)^{\frac{1}{2}}\cdot\bigg(\frac{[e]}{[f]}\bigg)^{\frac{1}{2}}=$$
$$=[a]\cdot[b]^{0}-\bigg(\frac{[licz]}{[mian]}\bigg)^{\frac{1}{2}}=
[a]-\sqrt{\frac{[licz]}{[mian]}}=[a]-\frac{[pierw]}{[pierw3]}=\frac{[w1]}{[pierw3]}$$
\odpStart
$\frac{[w1]}{[pierw3]}$
\odpStop
\testStart
A. $-\frac{[w1]}{[pierw3]}$ \\
B. $1$ \\
C. $\frac{[a]}{[e]}$ \\
D. $\frac{[w1]}{[pierw3]}$ 
\testStop
\kluczStart
D
\kluczStop



\end{document}