\documentclass[12pt, a4paper]{article}
\usepackage[utf8]{inputenc}
\usepackage{polski}

\usepackage{amsthm}  %pakiet do tworzenia twierdzeń itp.
\usepackage{amsmath} %pakiet do niektórych symboli matematycznych
\usepackage{amssymb} %pakiet do symboli mat., np. \nsubseteq
\usepackage{amsfonts}
\usepackage{graphicx} %obsługa plików graficznych z rozszerzeniem png, jpg
\theoremstyle{definition} %styl dla definicji
\newtheorem{zad}{} 
\title{Multizestaw zadań}
\author{Robert Fidytek}
%\date{\today}
\date{}
\newcounter{liczniksekcji}
\newcommand{\kategoria}[1]{\section{#1}} %olreślamy nazwę kateforii zadań
\newcommand{\zadStart}[1]{\begin{zad}#1\newline} %oznaczenie początku zadania
\newcommand{\zadStop}{\end{zad}}   %oznaczenie końca zadania
%Makra opcjonarne (nie muszą występować):
\newcommand{\rozwStart}[2]{\noindent \textbf{Rozwiązanie (autor #1 , recenzent #2): }\newline} %oznaczenie początku rozwiązania, opcjonarnie można wprowadzić informację o autorze rozwiązania zadania i recenzencie poprawności wykonania rozwiązania zadania
\newcommand{\rozwStop}{\newline}                                            %oznaczenie końca rozwiązania
\newcommand{\odpStart}{\noindent \textbf{Odpowiedź:}\newline}    %oznaczenie początku odpowiedzi końcowej (wypisanie wyniku)
\newcommand{\odpStop}{\newline}                                             %oznaczenie końca odpowiedzi końcowej (wypisanie wyniku)
\newcommand{\testStart}{\noindent \textbf{Test:}\newline} %ewentualne możliwe opcje odpowiedzi testowej: A. ? B. ? C. ? D. ? itd.
\newcommand{\testStop}{\newline} %koniec wprowadzania odpowiedzi testowych
\newcommand{\kluczStart}{\noindent \textbf{Test poprawna odpowiedź:}\newline} %klucz, poprawna odpowiedź pytania testowego (jedna literka): A lub B lub C lub D itd.
\newcommand{\kluczStop}{\newline} %koniec poprawnej odpowiedzi pytania testowego 
\newcommand{\wstawGrafike}[2]{\begin{figure}[h] \includegraphics[scale=#2] {#1} \end{figure}} %gdyby była potrzeba wstawienia obrazka, parametry: nazwa pliku, skala (jak nie wiesz co wpisać, to wpisz 1)

\begin{document}
\maketitle


\kategoria{Wikieł/Z1.134d}
\zadStart{Zadanie z Wikieł Z 1.134d) moja wersja nr [nrWersji]}
%[b]:[2,3,5,6,7,8,9,10]
%[a]:[2,3,4,5,6,7,8,9,10]
%[d]=random.randrange(2,10,2)
Wyznaczyć dziedzinę naturalną oraz zbadać parzystość (nieparzystość) funkcji:\\
$a) f(x)=[a]x\sqrt{x^{2}}+\frac{[b]x}{|x|}+x^{[d]}\cdot sgn(x)$
\zadStop
\rozwStart{Wojciech Przybylski}{}
I Wyznaczanie dziedziny naturalnej:\\
$$1.\hspace{2mm}\sqrt{x^{2}}\geq0 \Rightarrow x^{2}\geq0 \hspace{3mm} x\in\mathbb{R}$$
$$2.\hspace{2mm}|x|>0 \hspace{3mm} x\in\mathbb{R}\backslash\{0\}$$
$$\mathcal{D}_{f}\in\mathbb{R}\backslash\{0\}$$
II Badanie parzystości (nieparzystości) funkcji:
$$f(-x)=[a](-x)\sqrt{(-x)^{2}}+\frac{[b](-x)}{|-x|}+(-x)^{[d]}\cdot sgn(-x)=$$
$$=-[a]x\sqrt{x^{2}}-\frac{[b]x}{|x|}-x^{[d]}\cdot sgn(x)$$
$$=-([a]x\sqrt{x^{2}}+\frac{[b]x}{|x|}+x^{[d]}\cdot sgn(x))=-f(x)$$
\rozwStop
\odpStart
$\mathcal{D}_{f}\in\mathbb{R}\backslash\{0\} \mbox{ oraz } f(x)\mbox{ jest nieparzysta.}$ 
\odpStop
\testStart
A. $\mathcal{D}_{f}\in\mathbb{R}\backslash\{0\} \mbox{ oraz } f(x)\mbox{ jest nieparzysta.}$  \\
B. $\mathcal{D}_{f}\in (-[b],[b]) \mbox{ oraz } f(x)\mbox{ jest nieparzysta.}$ \\
C. $\mathcal{D}_{f}\in \mathbb{R} \mbox{ oraz } f(x)\mbox{ jest parzysta.}$ \\
D. $\mathcal{D}_{f}\in \mathbb{R}\backslash\{0\} \mbox{ oraz } f(x)\mbox{ jest parzysta.}$ \\
E. $\mathcal{D}_{f}\in \mathbb{R}{\backslash}\{-[b],[b]\} \mbox{ oraz } f(x)\mbox{ jest nieparzysta.}$ \\
F. $\mathcal{D}_{f}\in \mathbb{N} \mbox{ oraz } f(x)\mbox{ jest parzysta.}$ \\
\testStop
\kluczStart
A
\kluczStop



\end{document}