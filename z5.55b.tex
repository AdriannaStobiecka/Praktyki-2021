\documentclass[12pt, a4paper]{article}
\usepackage[utf8]{inputenc}
\usepackage{polski}

\usepackage{amsthm}  %pakiet do tworzenia twierdzeń itp.
\usepackage{amsmath} %pakiet do niektórych symboli matematycznych
\usepackage{amssymb} %pakiet do symboli mat., np. \nsubseteq
\usepackage{amsfonts}
\usepackage{graphicx} %obsługa plików graficznych z rozszerzeniem png, jpg
\theoremstyle{definition} %styl dla definicji
\newtheorem{zad}{} 
\title{Multizestaw zadań}
\author{Robert Fidytek}
%\date{\today}
\date{}
\newcounter{liczniksekcji}
\newcommand{\kategoria}[1]{\section{#1}} %olreślamy nazwę kateforii zadań
\newcommand{\zadStart}[1]{\begin{zad}#1\newline} %oznaczenie początku zadania
\newcommand{\zadStop}{\end{zad}}   %oznaczenie końca zadania
%Makra opcjonarne (nie muszą występować):
\newcommand{\rozwStart}[2]{\noindent \textbf{Rozwiązanie (autor #1 , recenzent #2): }\newline} %oznaczenie początku rozwiązania, opcjonarnie można wprowadzić informację o autorze rozwiązania zadania i recenzencie poprawności wykonania rozwiązania zadania
\newcommand{\rozwStop}{\newline}                                            %oznaczenie końca rozwiązania
\newcommand{\odpStart}{\noindent \textbf{Odpowiedź:}\newline}    %oznaczenie początku odpowiedzi końcowej (wypisanie wyniku)
\newcommand{\odpStop}{\newline}                                             %oznaczenie końca odpowiedzi końcowej (wypisanie wyniku)
\newcommand{\testStart}{\noindent \textbf{Test:}\newline} %ewentualne możliwe opcje odpowiedzi testowej: A. ? B. ? C. ? D. ? itd.
\newcommand{\testStop}{\newline} %koniec wprowadzania odpowiedzi testowych
\newcommand{\kluczStart}{\noindent \textbf{Test poprawna odpowiedź:}\newline} %klucz, poprawna odpowiedź pytania testowego (jedna literka): A lub B lub C lub D itd.
\newcommand{\kluczStop}{\newline} %koniec poprawnej odpowiedzi pytania testowego 
\newcommand{\wstawGrafike}[2]{\begin{figure}[h] \includegraphics[scale=#2] {#1} \end{figure}} %gdyby była potrzeba wstawienia obrazka, parametry: nazwa pliku, skala (jak nie wiesz co wpisać, to wpisz 1)

\begin{document}
\maketitle


\kategoria{Wikieł/Z5.55b}
\zadStart{Zadanie z Wikieł Z 5.55b) moja wersja nr [nrWersji]}
%[a]=random.randint(-10,-1)
%[b]:[-6,-5,-4,-3,-2,-1]
%[c]:[0,1,2,3,4,5]
%[d]:[2,3,4,5,6]
%[e]:[0,1,2,3,4,5]
%[be]=[b]*[e]
%[cd]=[c]*[d]
%[w]=[be]+[cd]
%[c]!=[e] and [w]!=[be] and [w]!=[e] and [w]!=[cd] and [w]!=[b] and [w]!=[d] and [be]!=[e] and [be]!=[cd] and [be]!=[d] and [be]!=[b] and [e]!=[cd] and [e]!=[d] and [cd]!=[d]
Na podstawie podanych wartości $f'([a])=[b],$ $f([a])=[c],$ $g'([a])=[d],$ $g([a])=[e]$ obliczyć wartość następującej pochodnej $\frac{d}{dx}\left[f(x)g(x)\right]|_{x=[a]}$.
\zadStop
\rozwStart{Justyna Chojecka}{}
Zauważmy, że 
$$\left(f(x)g(x)\right)'=f'(x)\cdot g(x)+f(x)\cdot g'(x).$$
Obliczamy wartość pochodnej $\left(f(x)g(x)\right)'$ dla $x=[a]$.
$$\left(f([a])g([a])\right)'=f'([a])\cdot g([a])+f([a])\cdot g'([a])$$$$=[b]\cdot [e]+[c]\cdot [d]=[be]+[cd]=[w]$$
\rozwStop
\odpStart
$[w]$
\odpStop
\testStart
A.$[w]$
B.$[be]$
C.$[e]$
D.$[cd]$
E.$-[e]$
F.$-[cd]$
G.$[b]$
H.$-[d]$
I.$[d]$
\testStop
\kluczStart
A
\kluczStop



\end{document}