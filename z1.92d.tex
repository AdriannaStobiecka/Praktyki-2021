\documentclass[12pt, a4paper]{article}
\usepackage[utf8]{inputenc}
\usepackage{polski}
\usepackage{amsthm}  %pakiet do tworzenia twierdzeń itp.
\usepackage{amsmath} %pakiet do niektórych symboli matematycznych
\usepackage{amssymb} %pakiet do symboli mat., np. \nsubseteq
\usepackage{amsfonts}
\usepackage{graphicx} %obsługa plików graficznych z rozszerzeniem png, jpg
\theoremstyle{definition} %styl dla definicji
\newtheorem{zad}{} 
\title{Multizestaw zadań}
\author{Robert Fidytek}
%\date{\today}
\date{}
\newcounter{liczniksekcji}
\newcommand{\kategoria}[1]{\section{#1}} %olreślamy nazwę kateforii zadań
\newcommand{\zadStart}[1]{\begin{zad}#1\newline} %oznaczenie początku zadania
\newcommand{\zadStop}{\end{zad}}   %oznaczenie końca zadania
%Makra opcjonarne (nie muszą występować):
\newcommand{\rozwStart}[2]{\noindent \textbf{Rozwiązanie (autor #1 , recenzent #2): }\newline} %oznaczenie początku rozwiązania, opcjonarnie można wprowadzić informację o autorze rozwiązania zadania i recenzencie poprawności wykonania rozwiązania zadania
\newcommand{\rozwStop}{\newline}                                            %oznaczenie końca rozwiązania
\newcommand{\odpStart}{\noindent \textbf{Odpowiedź:}\newline}    %oznaczenie początku odpowiedzi końcowej (wypisanie wyniku)
\newcommand{\odpStop}{\newline}                                             %oznaczenie końca odpowiedzi końcowej (wypisanie wyniku)
\newcommand{\testStart}{\noindent \textbf{Test:}\newline} %ewentualne możliwe opcje odpowiedzi testowej: A. ? B. ? C. ? D. ? itd.
\newcommand{\testStop}{\newline} %koniec wprowadzania odpowiedzi testowych
\newcommand{\kluczStart}{\noindent \textbf{Test poprawna odpowiedź:}\newline} %klucz, poprawna odpowiedź pytania testowego (jedna literka): A lub B lub C lub D itd.
\newcommand{\kluczStop}{\newline} %koniec poprawnej odpowiedzi pytania testowego 
\newcommand{\wstawGrafike}[2]{\begin{figure}[h] \includegraphics[scale=#2] {#1} \end{figure}} %gdyby była potrzeba wstawienia obrazka, parametry: nazwa pliku, skala (jak nie wiesz co wpisać, to wpisz 1)

\begin{document}
\maketitle

\kategoria{Wikieł/Z1.92d}
\zadStart{Zadanie z Wikieł Z1.92 d) moja wersja nr [nrWersji]}
%[a]:[2,4,8,10]
%[m]=int(6*[a])
%[o]=[m]-[a]
%[dz]=math.gcd([a],6)
%[g]=int([a]/[dz])
%[d]=int(6/[dz])
Rozwiąż równania.\\
Podane równanie: $ 6^{x + 1} - [a]^{x - 1} = 6^{x} $\\
\zadStop
\rozwStart{Martyna Czarnobaj}{}
\begin{center}
	$ 6^{x + 1} - [a]^{x - 1} = 6^{x} $\\
	$ 6^{x} \cdot 6 - \frac{[a]^{x}}{[a]} = 6^{x} | \cdot [a] $\\
	$ 6^{x} \cdot [m] - [a]^{x} = 6^{x} \cdot [a] $\\
	$ 6^{x} \cdot [m] - 6^{x} \cdot [a] = [a]^{x} $\\
	$ 6^{x}([m]-[a]) = [a]^{x} |:6^{x} $\\
	$ [o]=(\frac{[a]}{6})^{x} $\\
	$ [o]=\frac{[g]}{[d]}$\\
	$ \log_{a} b = c \Leftrightarrow a^{c} = b $\\
	$ a = \frac{[g]}{[d]}, c = x, b = [o] $\\
	Wynik: $ \log_{\frac{[g]}{[d]}} [o] = x $\\
\end{center}
Koniec rozwiązania.\\
\rozwStop
\odpStart
 $ \log_{\frac{[g]}{[d]}} [o] = x $\\
\odpStop
\testStart
A.$ \log_{\frac{[g]}{[d]}} [o] = x $\\\
B.$ \log_{\frac{[g]}{[m]}} [o] = x $\\
C.$ \log_{\frac{[g]}{[d]}} [m] = x $\\\\
\testStop
\kluczStart
A
\kluczStop



\end{document}