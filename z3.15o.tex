\documentclass[12pt, a4paper]{article}
\usepackage[utf8]{inputenc}
\usepackage{polski}

\usepackage{amsthm}  %pakiet do tworzenia twierdzeń itp.
\usepackage{amsmath} %pakiet do niektórych symboli matematycznych
\usepackage{amssymb} %pakiet do symboli mat., np. \nsubseteq
\usepackage{amsfonts}
\usepackage{graphicx} %obsługa plików graficznych z rozszerzeniem png, jpg
\theoremstyle{definition} %styl dla definicji
\newtheorem{zad}{} 
\title{Multizestaw zadań}
\author{Robert Fidytek}
%\date{\today}
\date{}
\newcounter{liczniksekcji}
\newcommand{\kategoria}[1]{\section{#1}} %olreślamy nazwę kateforii zadań
\newcommand{\zadStart}[1]{\begin{zad}#1\newline} %oznaczenie początku zadania
\newcommand{\zadStop}{\end{zad}}   %oznaczenie końca zadania
%Makra opcjonarne (nie muszą występować):
\newcommand{\rozwStart}[2]{\noindent \textbf{Rozwiązanie (autor #1 , recenzent #2): }\newline} %oznaczenie początku rozwiązania, opcjonarnie można wprowadzić informację o autorze rozwiązania zadania i recenzencie poprawności wykonania rozwiązania zadania
\newcommand{\rozwStop}{\newline}                                            %oznaczenie końca rozwiązania
\newcommand{\odpStart}{\noindent \textbf{Odpowiedź:}\newline}    %oznaczenie początku odpowiedzi końcowej (wypisanie wyniku)
\newcommand{\odpStop}{\newline}                                             %oznaczenie końca odpowiedzi końcowej (wypisanie wyniku)
\newcommand{\testStart}{\noindent \textbf{Test:}\newline} %ewentualne możliwe opcje odpowiedzi testowej: A. ? B. ? C. ? D. ? itd.
\newcommand{\testStop}{\newline} %koniec wprowadzania odpowiedzi testowych
\newcommand{\kluczStart}{\noindent \textbf{Test poprawna odpowiedź:}\newline} %klucz, poprawna odpowiedź pytania testowego (jedna literka): A lub B lub C lub D itd.
\newcommand{\kluczStop}{\newline} %koniec poprawnej odpowiedzi pytania testowego 
\newcommand{\wstawGrafike}[2]{\begin{figure}[h] \includegraphics[scale=#2] {#1} \end{figure}} %gdyby była potrzeba wstawienia obrazka, parametry: nazwa pliku, skala (jak nie wiesz co wpisać, to wpisz 1)

\begin{document}
\maketitle


\kategoria{Wikieł/Z3.15o}
\zadStart{Zadanie z Wikieł Z 3.15 o) moja wersja nr [nrWersji]}
%[a]:[2,3,4,5,6,7]
%[b]:[2,3,4,5,6,7]
%[c]:[2,3,4,5,6,7]
%[d]:[2,3,4,5,6,7]
%[bd]=[b]*[d]
%[ad]=[a]*[d]
%[w]=[bd]-[ad]
%[ab]=[a]/[b]
%math.gcd([a],[b])==1 and [c]!=[w] and [w]!=1 and [w]!=0 and [a]!=[c] and [b]!=[c] and [a]!=[d] and [b]!=[d]
Obliczyć granicę ciągu 
$$a_n=\bigg(\frac{n+[a]}{n+[b]}\bigg)^{[c]-[d]n}.$$
\zadStop
\rozwStart{Adrianna Stobiecka}{}
$$\lim_{n\to\infty}\bigg(\frac{n+[a]}{n+[b]}\bigg)^{[c]-[d]n}=\lim_{n\to\infty}\bigg(\frac{n(1+\frac{[a]}{n})}{n(1+\frac{[b]}{n})}\bigg)^{[c]-[d]n}=\lim_{n\to\infty}\bigg(\frac{1+\frac{[b]}{n}}{1+\frac{[a]}{n}}\bigg)^{[d]n}\bigg(\frac{1+\frac{[a]}{n}}{1+\frac{[b]}{n}}\bigg)^{[c]}$$
$$=\lim_{n\to\infty}\frac{(1+\frac{[b]}{n})^{[d]n}}{(1+\frac{[a]}{n})^{[d]n}}\bigg(\frac{1+\frac{[a]}{n}}{1+\frac{[b]}{n}}\bigg)^{[c]}=\lim_{n\to\infty}\frac{[(1+\frac{[b]}{n})^{\frac{n}{[b]}}]^{[b]\cdot[d]}}{[(1+\frac{[a]}{n})^{\frac{n}{[a]}}]^{[a]\cdot[d]}}\bigg(\frac{1+\frac{[a]}{n}}{1+\frac{[b]}{n}}\bigg)^{[c]}$$
$$=\frac{e^{[b]\cdot[d]}}{e^{[a]\cdot[d]}}\cdot1^{[c]}=\frac{e^{[bd]}}{e^{[ad]}}\cdot1=e^{[bd]-[ad]}=e^{[w]}$$
\rozwStop
\odpStart
 $e^{[w]}$
\odpStop
\testStart
A.$1$
B.$[w]$
C.$-\infty$
D.$0$
E.$e^{[w]}$
F.$\infty$
G.$\frac{[a]}{[b]}$
H.$e^{[c]}$
I.$\frac{[b]}{[a]}$
\testStop
\kluczStart
E
\kluczStop



\end{document}
