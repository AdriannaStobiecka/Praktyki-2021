\documentclass[12pt, a4paper]{article}
\usepackage[utf8]{inputenc}
\usepackage{polski}

\usepackage{amsthm}  %pakiet do tworzenia twierdzeń itp.
\usepackage{amsmath} %pakiet do niektórych symboli matematycznych
\usepackage{amssymb} %pakiet do symboli mat., np. \nsubseteq
\usepackage{amsfonts}
\usepackage{graphicx} %obsługa plików graficznych z rozszerzeniem png, jpg
\theoremstyle{definition} %styl dla definicji
\newtheorem{zad}{} 
\title{Multizestaw zadań}
\author{Robert Fidytek}
%\date{\today}
\date{}
\newcounter{liczniksekcji}
\newcommand{\kategoria}[1]{\section{#1}} %olreślamy nazwę kateforii zadań
\newcommand{\zadStart}[1]{\begin{zad}#1\newline} %oznaczenie początku zadania
\newcommand{\zadStop}{\end{zad}}   %oznaczenie końca zadania
%Makra opcjonarne (nie muszą występować):
\newcommand{\rozwStart}[2]{\noindent \textbf{Rozwiązanie (autor #1 , recenzent #2): }\newline} %oznaczenie początku rozwiązania, opcjonarnie można wprowadzić informację o autorze rozwiązania zadania i recenzencie poprawności wykonania rozwiązania zadania
\newcommand{\rozwStop}{\newline}                                            %oznaczenie końca rozwiązania
\newcommand{\odpStart}{\noindent \textbf{Odpowiedź:}\newline}    %oznaczenie początku odpowiedzi końcowej (wypisanie wyniku)
\newcommand{\odpStop}{\newline}                                             %oznaczenie końca odpowiedzi końcowej (wypisanie wyniku)
\newcommand{\testStart}{\noindent \textbf{Test:}\newline} %ewentualne możliwe opcje odpowiedzi testowej: A. ? B. ? C. ? D. ? itd.
\newcommand{\testStop}{\newline} %koniec wprowadzania odpowiedzi testowych
\newcommand{\kluczStart}{\noindent \textbf{Test poprawna odpowiedź:}\newline} %klucz, poprawna odpowiedź pytania testowego (jedna literka): A lub B lub C lub D itd.
\newcommand{\kluczStop}{\newline} %koniec poprawnej odpowiedzi pytania testowego 
\newcommand{\wstawGrafike}[2]{\begin{figure}[h] \includegraphics[scale=#2] {#1} \end{figure}} %gdyby była potrzeba wstawienia obrazka, parametry: nazwa pliku, skala (jak nie wiesz co wpisać, to wpisz 1)

\begin{document}
\maketitle


\kategoria{Wikieł/Z5.3b}
\zadStart{Zadanie z Wikieł Z 5.3 b) moja wersja nr [nrWersji]}
%[a]:[2,3,4,5,6,7,8,9]
%[b]:[2,3,4,5,6,7,8,9]
%[d]=2*[b]
%[a]!=0 
Zbadać różniczkowalność funkcji f w punkcje $x_0=0$.
$$
f(x) = \left\{ \begin{array}{ll}
[a]x^2+[b]x & \textrm{gdy $x<0$}\\
{[d]}x & \textrm{gdy $ x\geq 0$}
\end{array} \right.
$$
\zadStop
\rozwStart{Joanna Świerzbin}{}
$$f'_{-}(x)=\lim_{\Delta x \rightarrow 0^{-}} \frac{f(x+\Delta x)-f(x)}{\Delta x} = $$ 
$$= \lim_{\Delta x \rightarrow 0^{-}} \frac{[a](x+\Delta x)^2+[b](x+\Delta x) -([a]x^2+[b]x)}{\Delta x}= $$ $$ =
 \lim_{\Delta x \rightarrow 0^{-}}  \frac{[a]x^2+2 \cdot [a] x \Delta x+ [a] (\Delta x)^2+[b]x+[b]\Delta x -[a]x^2-[b]x}{\Delta x} =$$
 $$ = \lim_{\Delta x \rightarrow 0^{-}}  \frac{2 \cdot [a] x \Delta x+ [a] (\Delta x)^2+[b]\Delta x }{\Delta x} =$$
 $$ = \lim_{\Delta x \rightarrow 0^{-}}  {2 \cdot [a] x + [a] \Delta x+[b]} = 2\cdot[a]x+[b]$$
$$f'_{+}(x)=\lim_{\Delta x \rightarrow 0^{-}} \frac{f(x+\Delta x)-f(x)}{\Delta x} = $$ 
$$= \lim_{\Delta x \rightarrow 0^{+}} \frac{[d](x+\Delta x)-[d]x}{\Delta x} =
 \lim_{\Delta x \rightarrow 0^{+}}  \frac{[d]x+[d]\Delta x-[d]x}{\Delta x} =$$
 $$ = \lim_{\Delta x \rightarrow 0^{+}}  \frac{[d]\Delta x}{\Delta x} = [d] $$
$$f'_{-}(0) = 2\cdot[a]\cdot 0+[b] = [b]$$
$$f'_{+}(0) = [d]$$
$$ f'_{-}(0) \neq f'_{+}(0) $$
\rozwStop
\odpStart
Nie jest różniczkowalna
\odpStop
\testStart
A. Nie jest różniczkowalna
B. Jest różniczkowalna
\testStop
\kluczStart
A
\kluczStop



\end{document}