\documentclass[12pt, a4paper]{article}
\usepackage[utf8]{inputenc}
\usepackage{polski}

\usepackage{amsthm}  %pakiet do tworzenia twierdzeń itp.
\usepackage{amsmath} %pakiet do niektórych symboli matematycznych
\usepackage{amssymb} %pakiet do symboli mat., np. \nsubseteq
\usepackage{amsfonts}
\usepackage{graphicx} %obsługa plików graficznych z rozszerzeniem png, jpg
\theoremstyle{definition} %styl dla definicji
\newtheorem{zad}{} 
\title{Multizestaw zadań}
\author{Robert Fidytek}
%\date{\today}
\date{}
\newcounter{liczniksekcji}
\newcommand{\kategoria}[1]{\section{#1}} %olreślamy nazwę kateforii zadań
\newcommand{\zadStart}[1]{\begin{zad}#1\newline} %oznaczenie początku zadania
\newcommand{\zadStop}{\end{zad}}   %oznaczenie końca zadania
%Makra opcjonarne (nie muszą występować):
\newcommand{\rozwStart}[2]{\noindent \textbf{Rozwiązanie (autor #1 , recenzent #2): }\newline} %oznaczenie początku rozwiązania, opcjonarnie można wprowadzić informację o autorze rozwiązania zadania i recenzencie poprawności wykonania rozwiązania zadania
\newcommand{\rozwStop}{\newline}                                            %oznaczenie końca rozwiązania
\newcommand{\odpStart}{\noindent \textbf{Odpowiedź:}\newline}    %oznaczenie początku odpowiedzi końcowej (wypisanie wyniku)
\newcommand{\odpStop}{\newline}                                             %oznaczenie końca odpowiedzi końcowej (wypisanie wyniku)
\newcommand{\testStart}{\noindent \textbf{Test:}\newline} %ewentualne możliwe opcje odpowiedzi testowej: A. ? B. ? C. ? D. ? itd.
\newcommand{\testStop}{\newline} %koniec wprowadzania odpowiedzi testowych
\newcommand{\kluczStart}{\noindent \textbf{Test poprawna odpowiedź:}\newline} %klucz, poprawna odpowiedź pytania testowego (jedna literka): A lub B lub C lub D itd.
\newcommand{\kluczStop}{\newline} %koniec poprawnej odpowiedzi pytania testowego 
\newcommand{\wstawGrafike}[2]{\begin{figure}[h] \includegraphics[scale=#2] {#1} \end{figure}} %gdyby była potrzeba wstawienia obrazka, parametry: nazwa pliku, skala (jak nie wiesz co wpisać, to wpisz 1)

\begin{document}
\maketitle


\kategoria{Wikieł/Z5.55l}
\zadStart{Zadanie z Wikieł Z 5.55l) moja wersja nr [nrWersji]}
%[a]:[1,4,9,16,25,36,49,64,81,100,121]
%[b]=int(math.sqrt([a]))
%[c]:[1,2,3,4,5,6,7,8,9,10]
%[m]=2*[b]
%math.gcd([c],[m])==1
Na podstawie podanych wartości $f'([b])=[c]$ obliczyć wartość następującej pochodnej $\frac{d}{dx}\left[f(\sqrt{x})\right] \big |_{x=[a]}$.
\zadStop
\rozwStart{Justyna Chojecka}{}
Zauważmy, że 
$$(f(\sqrt{x}))'=f'(\sqrt{x})\cdot (\sqrt{x})'=f'(\sqrt{x})\cdot \frac{1}{2\sqrt{x}}.$$
Obliczamy wartość pochodnej $(f(\sqrt{x}))'$ dla $x=[a]$.
$$\left(f\left(\sqrt{[a]}\right)\right)'=f'\left(\sqrt{[a]}\right)\cdot \frac{1}{2\sqrt{[a]}}=f'([b])\cdot \frac{1}{2\cdot [b]}=\frac{[c]}{[m]}$$
\rozwStop
\odpStart
$\frac{[c]}{[m]}$
\odpStop
\testStart
A.$\frac{[c]}{[m]}$
B.$\frac{10}{11}$
C.$4$
D.$-\frac{[c]}{[m]}$
E.$2$
F.$-2$
G.$-\frac{10}{11}$
H.$0$
I.$-4$
\testStop
\kluczStart
A
\kluczStop



\end{document}