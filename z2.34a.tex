\documentclass[12pt, a4paper]{article}
\usepackage[utf8]{inputenc}
\usepackage{polski}

\usepackage{amsthm}  %pakiet do tworzenia twierdzeń itp.
\usepackage{amsmath} %pakiet do niektórych symboli matematycznych
\usepackage{amssymb} %pakiet do symboli mat., np. \nsubseteq
\usepackage{amsfonts}
\usepackage{graphicx} %obsługa plików graficznych z rozszerzeniem png, jpg
\theoremstyle{definition} %styl dla definicji
\newtheorem{zad}{} 
\title{Multizestaw zadań}
\author{Robert Fidytek}
%\date{\today}
\date{}
\newcounter{liczniksekcji}
\newcommand{\kategoria}[1]{\section{#1}} %olreślamy nazwę kateforii zadań
\newcommand{\zadStart}[1]{\begin{zad}#1\newline} %oznaczenie początku zadania
\newcommand{\zadStop}{\end{zad}}   %oznaczenie końca zadania
%Makra opcjonarne (nie muszą występować):
\newcommand{\rozwStart}[2]{\noindent \textbf{Rozwiązanie (autor #1 , recenzent #2): }\newline} %oznaczenie początku rozwiązania, opcjonarnie można wprowadzić informację o autorze rozwiązania zadania i recenzencie poprawności wykonania rozwiązania zadania
\newcommand{\rozwStop}{\newline}                                            %oznaczenie końca rozwiązania
\newcommand{\odpStart}{\noindent \textbf{Odpowiedź:}\newline}    %oznaczenie początku odpowiedzi końcowej (wypisanie wyniku)
\newcommand{\odpStop}{\newline}                                             %oznaczenie końca odpowiedzi końcowej (wypisanie wyniku)
\newcommand{\testStart}{\noindent \textbf{Test:}\newline} %ewentualne możliwe opcje odpowiedzi testowej: A. ? B. ? C. ? D. ? itd.
\newcommand{\testStop}{\newline} %koniec wprowadzania odpowiedzi testowych
\newcommand{\kluczStart}{\noindent \textbf{Test poprawna odpowiedź:}\newline} %klucz, poprawna odpowiedź pytania testowego (jedna literka): A lub B lub C lub D itd.
\newcommand{\kluczStop}{\newline} %koniec poprawnej odpowiedzi pytania testowego 
\newcommand{\wstawGrafike}[2]{\begin{figure}[h] \includegraphics[scale=#2] {#1} \end{figure}} %gdyby była potrzeba wstawienia obrazka, parametry: nazwa pliku, skala (jak nie wiesz co wpisać, to wpisz 1)

\begin{document}
\maketitle


\kategoria{Wikieł/Z2.34a}
\zadStart{Zadanie z Wikieł Z 2.34a  moja wersja nr [nrWersji]}
%[p1]:[2,3,4,5,6,7,8,9,10]
%[p2]=random.randint(2,10)
%[p3]:[2,3,4,5,6,7,8,9,10]
%[p4]=random.randint(2,10)
%[p5]=random.randint(2,10)
%[p6]:[2,3,4,5,6,7,8,9,10]
%[p7]=random.randint(2,10)
%[p8]=random.randint(2,10)
%[p9]=random.randint(2,10)
%[p10]=random.randint(2,10)
%[p1p6]=[p1]*[p6]
%[p1p7]=[p1]*[p7]
%[p2p6]=[p2]*[p6]
%[p2p7]=[p2]*[p7]
%[p3p8]=[p3]*[p8]
%[p3p9]=[p3]*[p9]
%[p4p8]=[p4]*[p8]
%[p4p9]=[p4]*[p9]
%[m2]=[p1p6]-[p4p8]
%[m]=-[p1p7]+[p2p6]+[p3p8]-[p4p9]
%[r]=-[p2p7]+[p3p9]
%[del]=[m]*[m]-4*[m2]*[r]
%[pdel]=round(math.sqrt(abs([del])),2)
%[2m2]=2*[m2]
%[mp]=-[m]
%[m1]=round((-[m]-[pdel])/([2m2]+0.0000001),2)
%[m22]=round((-[m]+[pdel])/([2m2]+0.0000001),2)
%[del]>0 and [m2]!=0 and [m]<0 and [r]<0

Podać wartość parametru $m$, dla których proste 
$$([p1]m+[p2])x+([p3]-[p4]m)y+[p5]=0, \quad ([p6]m-[p7])x+([p8]m+[p9])y-[p10]=0 $$są prostopadłe.
\zadStop

\rozwStart{Maja Szabłowska}{}
$$([p1]m+[p2])x+([p3]-[p4]m)y+[p5]=0 \iff ([p3]-[p4]m)y=-([p1]m+[p2])x-[p5]$$
$$y=-\frac{[p1]m+[p2]}{[p3]-[p4]m}x-\frac{[p5]}{[p3]-[p4]m}$$

$$([p6]m-[p7])x+([p8]m+[p9])y-[p10]=0 \iff ([p8]m+[p9])y=-([p6]m-[p7])x+[p10]$$
$$y=-\frac{[p6]m-[p7]}{[p8]m+[p9]}x+\frac{[p10]}{[p8]m+[p9]}$$

$$-\frac{[p1]m+[p2]}{[p3]-[p4]m}\cdot\left(-\frac{[p6]m-[p7]}{[p8]m+[p9]}\right)=-1$$ 
$$\frac{([p1]m+[p2])([p6]m-[p7])}{([p3]-[p4]m)([p8]m+[p9])}=-1$$
$$([p1]m+[p2])([p6]m-[p7])=-([p3]-[p4]m)([p8]m+[p9])$$
$$[p1p6]m^{2}-[p1p7]m+[p2p6]m-[p2p7]=-([p3p8]m+[p3p9]-[p4p8]m^{2}-[p4p9]m)$$
$$[m2]m^{2}+([m])m+([r])=0$$
$$\Delta=([m])^{2}-4\cdot([m2])\cdot([r])=[del] \Rightarrow \sqrt{\Delta}=[pdel]$$
$$m_{1}=\frac{[mp]-[pdel]}{[2m2]}=[m1], \quad m_{2}=\frac{[mp]+[pdel]}{[2m2]}=[m22]$$
\rozwStop


\odpStart
$m_{1}=[m1], \quad m_{2}=[m22]$
\odpStop
\testStart
A.$m_{1}=[m1], \quad m_{2}=[m22]$
B.$m_{1}=[m], \quad m_{2}=[m22]$
C.$m_{1}=[m1], \quad m_{2}=[m2]$
D.$m_{1}=[del], \quad m_{2}=[m22]$
E.$m_{1}=[del], \quad m_{2}=[pdel]$
F.$m_{1}=[p1], \quad m_{2}=[p2]$
G.$m_{1}=[p3], \quad m_{2}=[p3p8]$
H.$m_{1}=0, \quad m_{2}=[m22]$

\testStop
\kluczStart
A
\kluczStop



\end{document}
