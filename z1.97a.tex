\documentclass[12pt, a4paper]{article}
\usepackage[utf8]{inputenc}
\usepackage{polski}

\usepackage{amsthm}  %pakiet do tworzenia twierdzeń itp.
\usepackage{amsmath} %pakiet do niektórych symboli matematycznych
\usepackage{amssymb} %pakiet do symboli mat., np. \nsubseteq
\usepackage{amsfonts}
\usepackage{graphicx} %obsługa plików graficznych z rozszerzeniem png, jpg
\theoremstyle{definition} %styl dla definicji
\newtheorem{zad}{} 
\title{Multizestaw zadań}
\author{Robert Fidytek}
%\date{\today}
\date{}
\newcounter{liczniksekcji}
\newcommand{\kategoria}[1]{\section{#1}} %olreślamy nazwę kateforii zadań
\newcommand{\zadStart}[1]{\begin{zad}#1\newline} %oznaczenie początku zadania
\newcommand{\zadStop}{\end{zad}}   %oznaczenie końca zadania
%Makra opcjonarne (nie muszą występować):
\newcommand{\rozwStart}[2]{\noindent \textbf{Rozwiązanie (autor #1 , recenzent #2): }\newline} %oznaczenie początku rozwiązania, opcjonarnie można wprowadzić informację o autorze rozwiązania zadania i recenzencie poprawności wykonania rozwiązania zadania
\newcommand{\rozwStop}{\newline}                                            %oznaczenie końca rozwiązania
\newcommand{\odpStart}{\noindent \textbf{Odpowiedź:}\newline}    %oznaczenie początku odpowiedzi końcowej (wypisanie wyniku)
\newcommand{\odpStop}{\newline}                                             %oznaczenie końca odpowiedzi końcowej (wypisanie wyniku)
\newcommand{\testStart}{\noindent \textbf{Test:}\newline} %ewentualne możliwe opcje odpowiedzi testowej: A. ? B. ? C. ? D. ? itd.
\newcommand{\testStop}{\newline} %koniec wprowadzania odpowiedzi testowych
\newcommand{\kluczStart}{\noindent \textbf{Test poprawna odpowiedź:}\newline} %klucz, poprawna odpowiedź pytania testowego (jedna literka): A lub B lub C lub D itd.
\newcommand{\kluczStop}{\newline} %koniec poprawnej odpowiedzi pytania testowego 
\newcommand{\wstawGrafike}[2]{\begin{figure}[h] \includegraphics[scale=#2] {#1} \end{figure}} %gdyby była potrzeba wstawienia obrazka, parametry: nazwa pliku, skala (jak nie wiesz co wpisać, to wpisz 1)

\begin{document}
\maketitle


\kategoria{Wikieł/Z1.97a}
\zadStart{Zadanie z Wikieł Z 1.97 a) moja wersja nr [nrWersji]}
%[a]:[1,2,3,4,5,6,7,8,9,10]
%[b]:[1,2,3,4,5,6,7,8,9,10]
%[c]:[1,2,3,4,5,6,7,8,9,11,12,13,14,15,16]
%[10c]=10*[c]
%[ca]=[10c]*[a]
%[cc]=[10c]+1
%[h]=[b]+[ca]
%[g]=math.gcd([h],[cc])
%[l]=int([h]/[g])
%[m]=int([cc]/[g])
%[b]>[a] and ([h]/[cc]).is_integer()==False and ([h]/[cc])>[a] and ([h]/[cc])<[b]
Rozwiązać nierówno\'sć $\log{(x-[a])}-\log{([b]-x)} \le - \log{[c]}-1$
\zadStop
\rozwStart{Małgorzata Ugowska}{}
Wyznaczamy dziedzinę:
$$x-[a]>0 \quad \Longrightarrow \quad x>[a]$$
$$[b] -x>0 \quad \Longrightarrow \quad x<[b]$$
$$D = ([a], [b])$$
Następnie rozwiązujemy nierówno\'sć
$$\log{(x-[a])}-\log{([b]-x)} \le - \log{[c]}-1$$
$$ \Longleftrightarrow \quad -\log{(x-[a])}+\log{([b]-x)} \ge  \log{[c]}+1 $$
$$\Longleftrightarrow \quad \log{\frac{[b]-x}{x-[a]}} \ge  \log{[c]}+\log{10} \quad \Longleftrightarrow \quad \log{\frac{[b]-x}{x-[a]}} \ge \log{[10c]}$$
$$ \Longleftrightarrow \quad \frac{[b]-x}{x-[a]} \ge [10c] \quad \Longleftrightarrow \quad [b]-x \ge [10c] (x-[a])$$
$$ \Longleftrightarrow \quad [b]-x \ge [10c]x-[ca] \quad \Longleftrightarrow \quad [cc]x \le [h] \quad \Longleftrightarrow \quad x \le \frac{[l]}{[m]}$$
\rozwStop
\odpStart
$x \in ([a], \frac{[l]}{[m]}]$
\odpStop
\testStart
A. $x \in ([a], \frac{[l]}{[m]}]$\\
B. $x \in [\frac{[l]}{[m]},[b])$\\
C. $x \in (-\infty, -4) \cup (3,\infty)$\\
D. $x \in (\frac{[l]}{[m]},[b])$\\
E. $x \in \emptyset$
\testStop
\kluczStart
A
\kluczStop



\end{document}