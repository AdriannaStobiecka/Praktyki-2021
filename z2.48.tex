\documentclass[12pt, a4paper]{article}
\usepackage[utf8]{inputenc}
\usepackage{polski}

\usepackage{amsthm}  %pakiet do tworzenia twierdzeń itp.
\usepackage{amsmath} %pakiet do niektórych symboli matematycznych
\usepackage{amssymb} %pakiet do symboli mat., np. \nsubseteq
\usepackage{amsfonts}
\usepackage{graphicx} %obsługa plików graficznych z rozszerzeniem png, jpg
\theoremstyle{definition} %styl dla definicji
\newtheorem{zad}{} 
\title{Multizestaw zadań}
\author{Robert Fidytek}
%\date{\today}
\date{}\documentclass[12pt, a4paper]{article}
\usepackage[utf8]{inputenc}
\usepackage{polski}

\usepackage{amsthm}  %pakiet do tworzenia twierdzeń itp.
\usepackage{amsmath} %pakiet do niektórych symboli matematycznych
\usepackage{amssymb} %pakiet do symboli mat., np. \nsubseteq
\usepackage{amsfonts}
\usepackage{graphicx} %obsługa plików graficznych z rozszerzeniem png, jpg
\theoremstyle{definition} %styl dla definicji
\newtheorem{zad}{} 
\title{Multizestaw zadań}
\author{Robert Fidytek}
%\date{\today}
\date{}
\newcounter{liczniksekcji}
\newcommand{\kategoria}[1]{\section{#1}} %olreślamy nazwę kateforii zadań
\newcommand{\zadStart}[1]{\begin{zad}#1\newline} %oznaczenie początku zadania
\newcommand{\zadStop}{\end{zad}}   %oznaczenie końca zadania
%Makra opcjonarne (nie muszą występować):
\newcommand{\rozwStart}[2]{\noindent \textbf{Rozwiązanie (autor #1 , recenzent #2): }\newline} %oznaczenie początku rozwiązania, opcjonarnie można wprowadzić informację o autorze rozwiązania zadania i recenzencie poprawności wykonania rozwiązania zadania
\newcommand{\rozwStop}{\newline}                                            %oznaczenie końca rozwiązania
\newcommand{\odpStart}{\noindent \textbf{Odpowiedź:}\newline}    %oznaczenie początku odpowiedzi końcowej (wypisanie wyniku)
\newcommand{\odpStop}{\newline}                                             %oznaczenie końca odpowiedzi końcowej (wypisanie wyniku)
\newcommand{\testStart}{\noindent \textbf{Test:}\newline} %ewentualne możliwe opcje odpowiedzi testowej: A. ? B. ? C. ? D. ? itd.
\newcommand{\testStop}{\newline} %koniec wprowadzania odpowiedzi testowych
\newcommand{\kluczStart}{\noindent \textbf{Test poprawna odpowiedź:}\newline} %klucz, poprawna odpowiedź pytania testowego (jedna literka): A lub B lub C lub D itd.
\newcommand{\kluczStop}{\newline} %koniec poprawnej odpowiedzi pytania testowego 
\newcommand{\wstawGrafike}[2]{\begin{figure}[h] \includegraphics[scale=#2] {#1} \end{figure}} %gdyby była potrzeba wstawienia obrazka, parametry: nazwa pliku, skala (jak nie wiesz co wpisać, to wpisz 1)

\begin{document}
\maketitle


\kategoria{Wikieł/Z2.48}
\zadStart{Zadanie z Wikieł Z 2.48 moja wersja nr [nrWersji]}
%[p1]:[2,3,4,5,6,7,8]
%[p2]:[1,2,4,9,16,25,36,49,64]
%[p1k]=[p1]*[p1]
%[2p1]=2*[p1]
%[1p1k]=1+[p1k]
%[2p1k]=[2p1]*[2p1]
%[a]=4*[1p1k]
%[b]=-4*[1p1k]*(-[p2])
%[m2]=[2p1k]-[a]
%[dm2]=-[m2]
%[d]=math.gcd([b],[dm2])
%[l]=int([b]/[d])
%[m]=int([dm2]/[d])
%[w]=round([l]/[m],2)
%[m2]<0 


Podać dla jakieś wartości parametru $m$ prosta $y=[p1]x+m$ jest styczna do okręgu danego równaniem $x^{2}+y^{2}=[p2].$

\zadStop

\rozwStart{Maja Szabłowska}{}
$$ \left\{ \begin{array}{ll}
y=[p1]x+m \\
x^{2}+y^{2}=[p2]
\end{array} \right.$$
$$x^{2}+([p1]x+m)^{2}=[p2]$$
$$x^{2}+[p1k]x^{2}+[2p1]xm+m^{2}-[p2]=0$$
$$[1p1k]x^{2}+[2p1]xm+m^{2}-[p2]=0$$
$$\Delta=([2p1]m)^{2}-4\cdot[1p1k]\cdot(m^{2}-[p2])=[2p1k]m^{2}-[a]m^{2}+[b]=[m2]m^{2}+[b]$$
$$[m2]m^{2}+[b]=0 \iff [m2]m^{2}=-[b] \iff m^{2}=\frac{[b]}{[dm2]}=\frac{[l]}{[m]}=[w]$$
$$m_{1}=\sqrt{[w]}, \quad m_{2}=-\sqrt{[w]}$$
\rozwStop


\odpStart
$m_{1}=\sqrt{[w]}, \quad m_{2}=-\sqrt{[w]}$
\odpStop
\testStart
A.$m_{1}=\sqrt{[w]}, \quad m_{2}=-\sqrt{[w]}$
B.$m_{1}=1, \quad m_{2}=-1$
C.$m_{1}=-\sqrt{[p1k]}, \quad m_{2}=0$
D.$m_{1}=[p1], \quad m_{2}=-\sqrt{\frac{[l]}{[m]}}$
E.$m_{1}=\sqrt{\frac{[a]}{[m]}}, \quad m_{2}=-[p2]$
F.$m_{1}=[1p1k], \quad m_{2}=-\sqrt{\frac{[l]}{[b]}}$
G.$m_{1}=\sqrt{[a]}, \quad m_{2}=-\sqrt{[a]}$

\testStop
\kluczStart
A
\kluczStop



\end{document}
