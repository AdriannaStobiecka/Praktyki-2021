\documentclass[12pt, a4paper]{article}
\usepackage[utf8]{inputenc}
\usepackage{polski}

\usepackage{amsthm}  %pakiet do tworzenia twierdzeń itp.
\usepackage{amsmath} %pakiet do niektórych symboli matematycznych
\usepackage{amssymb} %pakiet do symboli mat., np. \nsubseteq
\usepackage{amsfonts}
\usepackage{graphicx} %obsługa plików graficznych z rozszerzeniem png, jpg
\theoremstyle{definition} %styl dla definicji
\newtheorem{zad}{} 
\title{Multizestaw zadań}
\author{Robert Fidytek}
%\date{\today}
\date{}\documentclass[12pt, a4paper]{article}
\usepackage[utf8]{inputenc}
\usepackage{polski}

\usepackage{amsthm}  %pakiet do tworzenia twierdzeń itp.
\usepackage{amsmath} %pakiet do niektórych symboli matematycznych
\usepackage{amssymb} %pakiet do symboli mat., np. \nsubseteq
\usepackage{amsfonts}
\usepackage{graphicx} %obsługa plików graficznych z rozszerzeniem png, jpg
\theoremstyle{definition} %styl dla definicji
\newtheorem{zad}{} 
\title{Multizestaw zadań}
\author{Robert Fidytek}
%\date{\today}
\date{}
\newcounter{liczniksekcji}
\newcommand{\kategoria}[1]{\section{#1}} %olreślamy nazwę kateforii zadań
\newcommand{\zadStart}[1]{\begin{zad}#1\newline} %oznaczenie początku zadania
\newcommand{\zadStop}{\end{zad}}   %oznaczenie końca zadania
%Makra opcjonarne (nie muszą występować):
\newcommand{\rozwStart}[2]{\noindent \textbf{Rozwiązanie (autor #1 , recenzent #2): }\newline} %oznaczenie początku rozwiązania, opcjonarnie można wprowadzić informację o autorze rozwiązania zadania i recenzencie poprawności wykonania rozwiązania zadania
\newcommand{\rozwStop}{\newline}                                            %oznaczenie końca rozwiązania
\newcommand{\odpStart}{\noindent \textbf{Odpowiedź:}\newline}    %oznaczenie początku odpowiedzi końcowej (wypisanie wyniku)
\newcommand{\odpStop}{\newline}                                             %oznaczenie końca odpowiedzi końcowej (wypisanie wyniku)
\newcommand{\testStart}{\noindent \textbf{Test:}\newline} %ewentualne możliwe opcje odpowiedzi testowej: A. ? B. ? C. ? D. ? itd.
\newcommand{\testStop}{\newline} %koniec wprowadzania odpowiedzi testowych
\newcommand{\kluczStart}{\noindent \textbf{Test poprawna odpowiedź:}\newline} %klucz, poprawna odpowiedź pytania testowego (jedna literka): A lub B lub C lub D itd.
\newcommand{\kluczStop}{\newline} %koniec poprawnej odpowiedzi pytania testowego 
\newcommand{\wstawGrafike}[2]{\begin{figure}[h] \includegraphics[scale=#2] {#1} \end{figure}} %gdyby była potrzeba wstawienia obrazka, parametry: nazwa pliku, skala (jak nie wiesz co wpisać, to wpisz 1)

\begin{document}
\maketitle


\kategoria{Wikieł/Z1.11}
\zadStart{Zadanie z Wikieł Z 1.11 moja wersja nr [nrWersji]}
%[p1]:[2,3,4,5,6,7,8]
%[a]=[p1]*[p1]
%[x]:[2,3,5,7,11,17,237]
%[n]=random.randint(2,10)



Uprościć wyrażenie $$\frac{[p1](xy)^{\frac{1}{n}}-y^{\frac{1}{n}}}{[a](xy)^{\frac{2}{n}}-y^{\frac{2}{n}}}\cdot \frac{y^{\frac{1}{n}}}{\left([p1]x^{\frac{1}{n}}+1\right)^{-2}},$$
a następnie obliczyć wartość tego wyrażenia dla $x=[p2], n=[n].$
\zadStop

\rozwStart{Maja Szabłowska}{}
$$\frac{[p1](xy)^{\frac{1}{n}}-y^{\frac{1}{n}}}{[a](xy)^{\frac{2}{n}}-y^{\frac{2}{n}}}\cdot \frac{y^{\frac{1}{n}}}{\left([p1]x^{\frac{1}{n}}+1\right)^{-2}}=\frac{[p1]x^{\frac{1}{n}}y^{\frac{2}{n}}-y^{\frac{2}{n}}}{[a](xy)^{\frac{2}{n}}-y^{\frac{2}{n}}}\cdot\left([p1]x^{\frac{1}{n}}+1\right)^{2}=$$

$$=\frac{y^{\frac{2}{n}}([p1]x^{\frac{1}{n}}-1)}{y^{\frac{2}{n}}([a]x^{\frac{2}{n}}-1)}\cdot \left([p1]x^{\frac{1}{n}}+1\right)^{2}=$$

$$=\frac{[p1]x^{\frac{1}{n}}-1}{[a]x^{\frac{2}{n}}-1}\cdot \left([p1]x^{\frac{1}{n}}+1\right) \cdot \left([p1]x^{\frac{1}{n}}+1\right)= \frac{[a]x^{\frac{2}{n}}-1}{[a]x^{\frac{2}{n}}-1}\cdot\left([p1]x^{\frac{1}{n}}+1\right)=[p1]x^{\frac{1}{n}}+1$$

Dla $x=[x]$ oraz $n=[n]$ mamy:
$$[p1]\cdot[x]^{\frac{1}{[n]}}+1=[p1]\sqrt[[n]]{[x]}+1$$


\rozwStop


\odpStart
$[p1]\sqrt[[n]]{[x]}+1$
\odpStop
\testStart
A.$[p1]\sqrt[[n]]{[x]}+1$
B.$[p1][x]$
C.$\sqrt{[p1]}+1$
D.$\sqrt{[x]}$
E.$[x]$
F.$[n]$


\testStop
\kluczStart
A
\kluczStop



\end{document}
