\documentclass[12pt, a4paper]{article}
\usepackage[utf8]{inputenc}
\usepackage{polski}

\usepackage{amsthm}  %pakiet do tworzenia twierdzeń itp.
\usepackage{amsmath} %pakiet do niektórych symboli matematycznych
\usepackage{amssymb} %pakiet do symboli mat., np. \nsubseteq
\usepackage{amsfonts}
\usepackage{graphicx} %obsługa plików graficznych z rozszerzeniem png, jpg
\theoremstyle{definition} %styl dla definicji
\newtheorem{zad}{} 
\title{Multizestaw zadań}
\author{Robert Fidytek}
%\date{\today}
\date{}
\newcounter{liczniksekcji}
\newcommand{\kategoria}[1]{\section{#1}} %olreślamy nazwę kateforii zadań
\newcommand{\zadStart}[1]{\begin{zad}#1\newline} %oznaczenie początku zadania
\newcommand{\zadStop}{\end{zad}}   %oznaczenie końca zadania
%Makra opcjonarne (nie muszą występować):
\newcommand{\rozwStart}[2]{\noindent \textbf{Rozwiązanie (autor #1 , recenzent #2): }\newline} %oznaczenie początku rozwiązania, opcjonarnie można wprowadzić informację o autorze rozwiązania zadania i recenzencie poprawności wykonania rozwiązania zadania
\newcommand{\rozwStop}{\newline}                                            %oznaczenie końca rozwiązania
\newcommand{\odpStart}{\noindent \textbf{Odpowiedź:}\newline}    %oznaczenie początku odpowiedzi końcowej (wypisanie wyniku)
\newcommand{\odpStop}{\newline}                                             %oznaczenie końca odpowiedzi końcowej (wypisanie wyniku)
\newcommand{\testStart}{\noindent \textbf{Test:}\newline} %ewentualne możliwe opcje odpowiedzi testowej: A. ? B. ? C. ? D. ? itd.
\newcommand{\testStop}{\newline} %koniec wprowadzania odpowiedzi testowych
\newcommand{\kluczStart}{\noindent \textbf{Test poprawna odpowiedź:}\newline} %klucz, poprawna odpowiedź pytania testowego (jedna literka): A lub B lub C lub D itd.
\newcommand{\kluczStop}{\newline} %koniec poprawnej odpowiedzi pytania testowego 
\newcommand{\wstawGrafike}[2]{\begin{figure}[h] \includegraphics[scale=#2] {#1} \end{figure}} %gdyby była potrzeba wstawienia obrazka, parametry: nazwa pliku, skala (jak nie wiesz co wpisać, to wpisz 1)

\begin{document}
\maketitle


\kategoria{Wikieł/Z4.15h}
\zadStart{Zadanie z Wikieł Z 4.15 h) moja wersja nr [nrWersji]}
%[a]:[25,36,49,64,81,100,121,144,169,196,225,256,289,324,361,400,441,484,529,576]
%[b]=math.isqrt([a])
%[bk]=[b]*[b]
%[bd]=2*[b]
%[bt]=3*[b]
%[bc]=4*[b]
Zbadać, czy istnieje następująca granica $\lim\limits_{x\to 0}\frac{|x|}{|sinx|\left(\sqrt{x+[a]}-[b]\right)}$. Jeśli tak, to obliczyć ją.
\zadStop
\rozwStart{Justyna Chojecka}{}
Obliczamy granice jednostronne w punkcie $x_{0}=0$.
$$\lim\limits_{x\to 0^{-}}\frac{|x|}{|sinx|\left(\sqrt{x+[a]}-[b]\right)}=\lim\limits_{x\to 0^{-}}\frac{-x}{-sinx\left(\sqrt{x+[a]}-[b]\right)}\cdot \frac{\sqrt{x+[a]}+[b]}{\sqrt{x+[a]}+[b]}=$$$$\lim\limits_{x\to 0^{-}}\frac{1}{sinx}\cdot\frac{x\left(\sqrt{x+[a]}+[b]\right)}{x+[a]-[bk]}=\lim\limits_{x\to 0^{-}}\frac{1}{sinx}\cdot\left(\sqrt{x+[a]}+[b]\right)=\left[-\infty\cdot[bd]\right]=-\infty$$
$$\lim\limits_{x\to 0^{+}}\frac{|x|}{|sinx|\left(\sqrt{x+[a]}-[b]\right)}=\lim\limits_{x\to 0^{+}}\frac{x}{sinx\left(\sqrt{x+[a]}-[b]\right)}\cdot \frac{\sqrt{x+[a]}+[b]}{\sqrt{x+[a]}+[b]}=$$$$\lim\limits_{x\to 0^{+}}\frac{1}{sinx}\cdot\frac{x\left(\sqrt{x+[a]}+[b]\right)}{x+[a]-[bk]}=\lim\limits_{x\to 0^{+}}\frac{1}{sinx}\cdot\left(\sqrt{x+[a]}+[b]\right)=\left[\infty\cdot[bd]\right]=\infty$$
Skoro 
$$\lim\limits_{x\to 0^{-}}\frac{|x|}{|sinx|\left(\sqrt{x+[a]}-[b]\right)}\neq \lim\limits_{x\to 0^{+}}\frac{|x|}{|sinx|\left(\sqrt{x+[a]}-[b]\right)},$$
to granica $\lim\limits_{x\to 0}\frac{|x|}{|sinx|\left(\sqrt{x+[a]}-[b]\right)}$ nie istnieje.
\rozwStop
\odpStart
nie istnieje
\odpStop
\testStart
A.nie istnieje
B.0
C.$[a]$
D.$-[a]$
E.$-[bd]$
F.$[bt]$
G.$[bc]$
H.$-[bc]$
I.$-[bt]$
\testStop
\kluczStart
A
\kluczStop



\end{document}
