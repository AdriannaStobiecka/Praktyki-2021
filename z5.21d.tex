\documentclass[12pt, a4paper]{article}
\usepackage[utf8]{inputenc}
\usepackage{polski}

\usepackage{amsthm}  %pakiet do tworzenia twierdzeń itp.
\usepackage{amsmath} %pakiet do niektórych symboli matematycznych
\usepackage{amssymb} %pakiet do symboli mat., np. \nsubseteq
\usepackage{amsfonts}
\usepackage{graphicx} %obsługa plików graficznych z rozszerzeniem png, jpg
\theoremstyle{definition} %styl dla definicji
\newtheorem{zad}{} 
\title{Multizestaw zadań}
\author{Robert Fidytek}
%\date{\today}
\date{}
\newcounter{liczniksekcji}
\newcommand{\kategoria}[1]{\section{#1}} %olreślamy nazwę kateforii zadań
\newcommand{\zadStart}[1]{\begin{zad}#1\newline} %oznaczenie początku zadania
\newcommand{\zadStop}{\end{zad}}   %oznaczenie końca zadania
%Makra opcjonarne (nie muszą występować):
\newcommand{\rozwStart}[2]{\noindent \textbf{Rozwiązanie (autor #1 , recenzent #2): }\newline} %oznaczenie początku rozwiązania, opcjonarnie można wprowadzić informację o autorze rozwiązania zadania i recenzencie poprawności wykonania rozwiązania zadania
\newcommand{\rozwStop}{\newline}                                            %oznaczenie końca rozwiązania
\newcommand{\odpStart}{\noindent \textbf{Odpowiedź:}\newline}    %oznaczenie początku odpowiedzi końcowej (wypisanie wyniku)
\newcommand{\odpStop}{\newline}                                             %oznaczenie końca odpowiedzi końcowej (wypisanie wyniku)
\newcommand{\testStart}{\noindent \textbf{Test:}\newline} %ewentualne możliwe opcje odpowiedzi testowej: A. ? B. ? C. ? D. ? itd.
\newcommand{\testStop}{\newline} %koniec wprowadzania odpowiedzi testowych
\newcommand{\kluczStart}{\noindent \textbf{Test poprawna odpowiedź:}\newline} %klucz, poprawna odpowiedź pytania testowego (jedna literka): A lub B lub C lub D itd.
\newcommand{\kluczStop}{\newline} %koniec poprawnej odpowiedzi pytania testowego 
\newcommand{\wstawGrafike}[2]{\begin{figure}[h] \includegraphics[scale=#2] {#1} \end{figure}} %gdyby była potrzeba wstawienia obrazka, parametry: nazwa pliku, skala (jak nie wiesz co wpisać, to wpisz 1)

\begin{document}
\maketitle


\kategoria{Wikieł/Z5.21d}
\zadStart{Zadanie z Wikieł Z 5.21 d) moja wersja nr [nrWersji]}
%[a]:[2,3,4,5,6,7,8,9]
%[a1]=[a]*5
%[b]:[2,3,4,5,6,7,8,9]
%[b1]=3*[b]
%[c]:[2,3,4,5,6,7,8,9]
%[delta]= ((-[b1])^2)-(4*[a1]*[c])
%[delta]<0
Wyznaczyć przedziały monotoniczności funkcji $f(x)=[a]x^5-[b]x^3+[c]$.
\zadStop
\rozwStart{Joanna Świerzbin}{}
$$f(x)=[a]x^5-[b]x^3+[c]x$$
$$f'(x)=\left([a]x^5-[b]x^3+[c]x\right)'= 5\cdot[a]x^4-3\cdot[b]x^2+[c]= [a1]x^4-[b1]x^2+[c]$$
$$t=x^2 \land t \geq 0  $$
$$f'(x)= [a1]t^2-[b1]t+[c]$$
$$\Delta=(-[b1])^2-4\cdot[a1]\cdot[c]$$
$$\Delta=[delta] <0 $$
$f'(x) > 0$ dla $ x\in \mathbb{R} $,
a więc $f(x)$ jest rosnąca na $\mathbb{R}$.
\rozwStop
\odpStart
 $f(x)$ jest rosnąca na $\mathbb{R}$.
\odpStop
\testStart
A. $f(x)$ jest rosnąca na $\mathbb{R}$. \\
B. $f(x)$ jest malejąca na $\mathbb{R}$.\\
C. $f(x)$ jest stałą na $\mathbb{R}$.\\
D. $f(x)$ jest rosnąca dla $x\in (-\infty, 0)$ i malejąca dla  $x\in (0, \infty)$.\\
E. $f(x)$ jest malejąca dla $x\in (-\infty, 0)$ i rosnąca dla  $x\in (0, \infty)$.\\
F. $f(x)$ jest nierosnąca na $\mathbb{R}$.\\
\testStop
\kluczStart
A
\kluczStop



\end{document}