\documentclass[12pt, a4paper]{article}
\usepackage[utf8]{inputenc}
\usepackage{polski}
\usepackage{amsthm}  %pakiet do tworzenia twierdzeń itp.
\usepackage{amsmath} %pakiet do niektórych symboli matematycznych
\usepackage{amssymb} %pakiet do symboli mat., np. \nsubseteq
\usepackage{amsfonts}
\usepackage{graphicx} %obsługa plików graficznych z rozszerzeniem png, jpg
\theoremstyle{definition} %styl dla definicji
\newtheorem{zad}{} 
\title{Multizestaw zadań}
\author{Patryk Wirkus}
%\date{\today}
\date{}
\newcommand{\kategoria}[1]{\section{#1}}
\newcommand{\zadStart}[1]{\begin{zad}#1\newline}
\newcommand{\zadStop}{\end{zad}}
\newcommand{\rozwStart}[2]{\noindent \textbf{Rozwiązanie (autor #1 , recenzent #2): }\newline}
\newcommand{\rozwStop}{\newline}                                           
\newcommand{\odpStart}{\noindent \textbf{Odpowiedź:}\newline}
\newcommand{\odpStop}{\newline}
\newcommand{\testStart}{\noindent \textbf{Test:}\newline}
\newcommand{\testStop}{\newline}
\newcommand{\kluczStart}{\noindent \textbf{Test poprawna odpowiedź:}\newline}
\newcommand{\kluczStop}{\newline}
\newcommand{\wstawGrafike}[2]{\begin{figure}[h] \includegraphics[scale=#2] {#1} \end{figure}}

\begin{document}
\maketitle

\kategoria{Wikieł/Z1.25a}


\zadStart{Zadanie z Wikieł Z 1.25 a) moja wersja nr 1}

Wyznaczyć dziedzinę funkcji okrelonej podanym wzorem $f(x) = (x^{3}-1)^{\frac{1}{3}} + |x|$.
\zadStop
\rozwStart{Patryk Wirkus}{Laura Mieczkowska}
Wzór naszej funkcji wygląda następująco:
$$f(x) = (x^{3}-1)^{\frac{1}{3}} + |x|$$.\\
Wyrażenie znajdujące się pod pierwiastkiem 3. stopnia ma dziedzinę $\mathbf{R}$.\\ A więc dla $x^{3}-1$ dziedzina to $\mathbf{R}$.\\ Także dla wyrażenia będącego w wartości bezwględnej dziedzina to $\mathbf{R}$.\\
Ostatecznie więc, dziedzina funkcji f(x) to $\mathbf{R}$.
\rozwStop
\odpStart
$x \in \mathbf{R}$
\odpStop
\testStart
A.x $\in \mathbf{R}$\\ B.$x = 1$\\ C.$x = -1$\\ D.$x \in \mathbf{R} - \{1\}$\\ E.$x \in \mathbf{R} - \{-1\}$
\testStop
\kluczStart
A
\kluczStop



\zadStart{Zadanie z Wikieł Z 1.25 a) moja wersja nr 2}

Wyznaczyć dziedzinę funkcji okrelonej podanym wzorem $f(x) = (x^{3}-2)^{\frac{1}{3}} + |x|$.
\zadStop
\rozwStart{Patryk Wirkus}{Laura Mieczkowska}
Wzór naszej funkcji wygląda następująco:
$$f(x) = (x^{3}-2)^{\frac{1}{3}} + |x|$$.\\
Wyrażenie znajdujące się pod pierwiastkiem 3. stopnia ma dziedzinę $\mathbf{R}$.\\ A więc dla $x^{3}-2$ dziedzina to $\mathbf{R}$.\\ Także dla wyrażenia będącego w wartości bezwględnej dziedzina to $\mathbf{R}$.\\
Ostatecznie więc, dziedzina funkcji f(x) to $\mathbf{R}$.
\rozwStop
\odpStart
$x \in \mathbf{R}$
\odpStop
\testStart
A.x $\in \mathbf{R}$\\ B.$x = 2$\\ C.$x = -2$\\ D.$x \in \mathbf{R} - \{2\}$\\ E.$x \in \mathbf{R} - \{-2\}$
\testStop
\kluczStart
A
\kluczStop



\zadStart{Zadanie z Wikieł Z 1.25 a) moja wersja nr 3}

Wyznaczyć dziedzinę funkcji okrelonej podanym wzorem $f(x) = (x^{3}-3)^{\frac{1}{3}} + |x|$.
\zadStop
\rozwStart{Patryk Wirkus}{Laura Mieczkowska}
Wzór naszej funkcji wygląda następująco:
$$f(x) = (x^{3}-3)^{\frac{1}{3}} + |x|$$.\\
Wyrażenie znajdujące się pod pierwiastkiem 3. stopnia ma dziedzinę $\mathbf{R}$.\\ A więc dla $x^{3}-3$ dziedzina to $\mathbf{R}$.\\ Także dla wyrażenia będącego w wartości bezwględnej dziedzina to $\mathbf{R}$.\\
Ostatecznie więc, dziedzina funkcji f(x) to $\mathbf{R}$.
\rozwStop
\odpStart
$x \in \mathbf{R}$
\odpStop
\testStart
A.x $\in \mathbf{R}$\\ B.$x = 3$\\ C.$x = -3$\\ D.$x \in \mathbf{R} - \{3\}$\\ E.$x \in \mathbf{R} - \{-3\}$
\testStop
\kluczStart
A
\kluczStop



\zadStart{Zadanie z Wikieł Z 1.25 a) moja wersja nr 4}

Wyznaczyć dziedzinę funkcji okrelonej podanym wzorem $f(x) = (x^{3}-4)^{\frac{1}{3}} + |x|$.
\zadStop
\rozwStart{Patryk Wirkus}{Laura Mieczkowska}
Wzór naszej funkcji wygląda następująco:
$$f(x) = (x^{3}-4)^{\frac{1}{3}} + |x|$$.\\
Wyrażenie znajdujące się pod pierwiastkiem 3. stopnia ma dziedzinę $\mathbf{R}$.\\ A więc dla $x^{3}-4$ dziedzina to $\mathbf{R}$.\\ Także dla wyrażenia będącego w wartości bezwględnej dziedzina to $\mathbf{R}$.\\
Ostatecznie więc, dziedzina funkcji f(x) to $\mathbf{R}$.
\rozwStop
\odpStart
$x \in \mathbf{R}$
\odpStop
\testStart
A.x $\in \mathbf{R}$\\ B.$x = 4$\\ C.$x = -4$\\ D.$x \in \mathbf{R} - \{4\}$\\ E.$x \in \mathbf{R} - \{-4\}$
\testStop
\kluczStart
A
\kluczStop



\zadStart{Zadanie z Wikieł Z 1.25 a) moja wersja nr 5}

Wyznaczyć dziedzinę funkcji okrelonej podanym wzorem $f(x) = (x^{3}-5)^{\frac{1}{3}} + |x|$.
\zadStop
\rozwStart{Patryk Wirkus}{Laura Mieczkowska}
Wzór naszej funkcji wygląda następująco:
$$f(x) = (x^{3}-5)^{\frac{1}{3}} + |x|$$.\\
Wyrażenie znajdujące się pod pierwiastkiem 3. stopnia ma dziedzinę $\mathbf{R}$.\\ A więc dla $x^{3}-5$ dziedzina to $\mathbf{R}$.\\ Także dla wyrażenia będącego w wartości bezwględnej dziedzina to $\mathbf{R}$.\\
Ostatecznie więc, dziedzina funkcji f(x) to $\mathbf{R}$.
\rozwStop
\odpStart
$x \in \mathbf{R}$
\odpStop
\testStart
A.x $\in \mathbf{R}$\\ B.$x = 5$\\ C.$x = -5$\\ D.$x \in \mathbf{R} - \{5\}$\\ E.$x \in \mathbf{R} - \{-5\}$
\testStop
\kluczStart
A
\kluczStop



\zadStart{Zadanie z Wikieł Z 1.25 a) moja wersja nr 6}

Wyznaczyć dziedzinę funkcji okrelonej podanym wzorem $f(x) = (x^{3}-6)^{\frac{1}{3}} + |x|$.
\zadStop
\rozwStart{Patryk Wirkus}{Laura Mieczkowska}
Wzór naszej funkcji wygląda następująco:
$$f(x) = (x^{3}-6)^{\frac{1}{3}} + |x|$$.\\
Wyrażenie znajdujące się pod pierwiastkiem 3. stopnia ma dziedzinę $\mathbf{R}$.\\ A więc dla $x^{3}-6$ dziedzina to $\mathbf{R}$.\\ Także dla wyrażenia będącego w wartości bezwględnej dziedzina to $\mathbf{R}$.\\
Ostatecznie więc, dziedzina funkcji f(x) to $\mathbf{R}$.
\rozwStop
\odpStart
$x \in \mathbf{R}$
\odpStop
\testStart
A.x $\in \mathbf{R}$\\ B.$x = 6$\\ C.$x = -6$\\ D.$x \in \mathbf{R} - \{6\}$\\ E.$x \in \mathbf{R} - \{-6\}$
\testStop
\kluczStart
A
\kluczStop



\zadStart{Zadanie z Wikieł Z 1.25 a) moja wersja nr 7}

Wyznaczyć dziedzinę funkcji okrelonej podanym wzorem $f(x) = (x^{3}-7)^{\frac{1}{3}} + |x|$.
\zadStop
\rozwStart{Patryk Wirkus}{Laura Mieczkowska}
Wzór naszej funkcji wygląda następująco:
$$f(x) = (x^{3}-7)^{\frac{1}{3}} + |x|$$.\\
Wyrażenie znajdujące się pod pierwiastkiem 3. stopnia ma dziedzinę $\mathbf{R}$.\\ A więc dla $x^{3}-7$ dziedzina to $\mathbf{R}$.\\ Także dla wyrażenia będącego w wartości bezwględnej dziedzina to $\mathbf{R}$.\\
Ostatecznie więc, dziedzina funkcji f(x) to $\mathbf{R}$.
\rozwStop
\odpStart
$x \in \mathbf{R}$
\odpStop
\testStart
A.x $\in \mathbf{R}$\\ B.$x = 7$\\ C.$x = -7$\\ D.$x \in \mathbf{R} - \{7\}$\\ E.$x \in \mathbf{R} - \{-7\}$
\testStop
\kluczStart
A
\kluczStop



\zadStart{Zadanie z Wikieł Z 1.25 a) moja wersja nr 8}

Wyznaczyć dziedzinę funkcji okrelonej podanym wzorem $f(x) = (x^{3}-8)^{\frac{1}{3}} + |x|$.
\zadStop
\rozwStart{Patryk Wirkus}{Laura Mieczkowska}
Wzór naszej funkcji wygląda następująco:
$$f(x) = (x^{3}-8)^{\frac{1}{3}} + |x|$$.\\
Wyrażenie znajdujące się pod pierwiastkiem 3. stopnia ma dziedzinę $\mathbf{R}$.\\ A więc dla $x^{3}-8$ dziedzina to $\mathbf{R}$.\\ Także dla wyrażenia będącego w wartości bezwględnej dziedzina to $\mathbf{R}$.\\
Ostatecznie więc, dziedzina funkcji f(x) to $\mathbf{R}$.
\rozwStop
\odpStart
$x \in \mathbf{R}$
\odpStop
\testStart
A.x $\in \mathbf{R}$\\ B.$x = 8$\\ C.$x = -8$\\ D.$x \in \mathbf{R} - \{8\}$\\ E.$x \in \mathbf{R} - \{-8\}$
\testStop
\kluczStart
A
\kluczStop



\zadStart{Zadanie z Wikieł Z 1.25 a) moja wersja nr 9}

Wyznaczyć dziedzinę funkcji okrelonej podanym wzorem $f(x) = (x^{3}-9)^{\frac{1}{3}} + |x|$.
\zadStop
\rozwStart{Patryk Wirkus}{Laura Mieczkowska}
Wzór naszej funkcji wygląda następująco:
$$f(x) = (x^{3}-9)^{\frac{1}{3}} + |x|$$.\\
Wyrażenie znajdujące się pod pierwiastkiem 3. stopnia ma dziedzinę $\mathbf{R}$.\\ A więc dla $x^{3}-9$ dziedzina to $\mathbf{R}$.\\ Także dla wyrażenia będącego w wartości bezwględnej dziedzina to $\mathbf{R}$.\\
Ostatecznie więc, dziedzina funkcji f(x) to $\mathbf{R}$.
\rozwStop
\odpStart
$x \in \mathbf{R}$
\odpStop
\testStart
A.x $\in \mathbf{R}$\\ B.$x = 9$\\ C.$x = -9$\\ D.$x \in \mathbf{R} - \{9\}$\\ E.$x \in \mathbf{R} - \{-9\}$
\testStop
\kluczStart
A
\kluczStop



\zadStart{Zadanie z Wikieł Z 1.25 a) moja wersja nr 10}

Wyznaczyć dziedzinę funkcji okrelonej podanym wzorem $f(x) = (x^{3}-10)^{\frac{1}{3}} + |x|$.
\zadStop
\rozwStart{Patryk Wirkus}{Laura Mieczkowska}
Wzór naszej funkcji wygląda następująco:
$$f(x) = (x^{3}-10)^{\frac{1}{3}} + |x|$$.\\
Wyrażenie znajdujące się pod pierwiastkiem 3. stopnia ma dziedzinę $\mathbf{R}$.\\ A więc dla $x^{3}-10$ dziedzina to $\mathbf{R}$.\\ Także dla wyrażenia będącego w wartości bezwględnej dziedzina to $\mathbf{R}$.\\
Ostatecznie więc, dziedzina funkcji f(x) to $\mathbf{R}$.
\rozwStop
\odpStart
$x \in \mathbf{R}$
\odpStop
\testStart
A.x $\in \mathbf{R}$\\ B.$x = 10$\\ C.$x = -10$\\ D.$x \in \mathbf{R} - \{10\}$\\ E.$x \in \mathbf{R} - \{-10\}$
\testStop
\kluczStart
A
\kluczStop





\end{document}
