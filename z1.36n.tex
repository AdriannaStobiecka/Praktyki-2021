\documentclass[12pt, a4paper]{article}
\usepackage[utf8]{inputenc}
\usepackage{polski}

\usepackage{amsthm}  %pakiet do tworzenia twierdzeń itp.
\usepackage{amsmath} %pakiet do niektórych symboli matematycznych
\usepackage{amssymb} %pakiet do symboli mat., np. \nsubseteq
\usepackage{amsfonts}
\usepackage{graphicx} %obsługa plików graficznych z rozszerzeniem png, jpg
\theoremstyle{definition} %styl dla definicji
\newtheorem{zad}{} 
\title{Multizestaw zadań}
\author{Mirella Narewska}
%\date{\today}
\date{}
\newcounter{liczniksekcji}
\newcommand{\kategoria}[1]{\section{#1}} %olreślamy nazwę kateforii zadań
\newcommand{\zadStart}[1]{\begin{zad}#1\newline} %oznaczenie początku zadania
\newcommand{\zadStop}{\end{zad}}   %oznaczenie końca zadania
%Makra opcjonarne (nie muszą występować):
\newcommand{\rozwStart}[2]{\noindent \textbf{Rozwiązanie (autor #1 , recenzent #2): }\newline} %oznaczenie początku rozwiązania, opcjonarnie można wprowadzić informację o autorze rozwiązania zadania i recenzencie poprawności wykonania rozwiązania zadania
\newcommand{\rozwStop}{\newline}                                            %oznaczenie końca rozwiązania
\newcommand{\odpStart}{\noindent \textbf{Odpowiedź:}\newline}    %oznaczenie początku odpowiedzi końcowej (wypisanie wyniku)
\newcommand{\odpStop}{\newline}                                             %oznaczenie końca odpowiedzi końcowej (wypisanie wyniku)
\newcommand{\testStart}{\noindent \textbf{Test:}\newline} %ewentualne możliwe opcje odpowiedzi testowej: A. ? B. ? C. ? D. ? itd.
\newcommand{\testStop}{\newline} %koniec wprowadzania odpowiedzi testowych
\newcommand{\kluczStart}{\noindent \textbf{Test poprawna odpowiedź:}\newline} %klucz, poprawna odpowiedź pytania testowego (jedna literka): A lub B lub C lub D itd.
\newcommand{\kluczStop}{\newline} %koniec poprawnej odpowiedzi pytania testowego 
\newcommand{\wstawGrafike}[2]{\begin{figure}[h] \includegraphics[scale=#2] {#1} \end{figure}} %gdyby była potrzeba wstawienia obrazka, parametry: nazwa pliku, skala (jak nie wiesz co wpisać, to wpisz 1)

\begin{document}
\maketitle


\kategoria{Wikieł/z1.36l}
\zadStart{Zadanie z Wikieł z1.36l  moja wersja nr [nrWersji]}
%[a]:[2,3,4,5,6,7,8,9,10,11,12,13,14,15]
%[b]:[2,3,4,5,6,7,8,9,10,11,12,13,14,15]
%[d]=[a]**2-4*[b]
%[p]=(pow([d],1/2))
%[p1]=[p].real
%[p2]=int([p1])
%[e]=int(([a]+[p2])/2)
%[f]=int((-[a]+[p2])/-2)
%[a]>[p2] and [a]**2>4*[b] and [p].is_integer()==True 
Rozwiązać nierówność $-x^2+[a]x-[b]<0$
\zadStop
\rozwStart{Mirella Narewska}{}
$$-x^2+[a]x-[b]<0$$
$$\triangle=[a]^2-4*[b] \Leftrightarrow \triangle=[d] \Leftrightarrow \sqrt{\triangle}=[p2]$$
$$x=\frac{-[a]-[p2]}{-2} \vee x=\frac{-[a]+[p2]}{-2}$$
$$x=[e] \vee x=[f]$$
Ramiona paraboli będą skierowane do dołu, ponieważ współczynnik liczbowy przy $x^2$ jest ujemny $\Rightarrow$ $x \in (-\infty, [f]) \cup ([e], \infty)$
\rozwStop
\odpStart
$x \in (-\infty, [f]) \cup ([e], \infty)$
\odpStop
\testStart
A.$x \in (-\infty, [f]) \cup ([e], \infty)$
\\
B.$x \in (-\infty, -[a]) \cup ([e], \infty)$
\\
C.$x \in \emptyset$
\\
D.$x \in (-\infty, [f])$
\testStop
\kluczStart
A
\kluczStop



\end{document}