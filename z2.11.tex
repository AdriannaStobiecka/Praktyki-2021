\documentclass[12pt, a4paper]{article}
\usepackage[utf8]{inputenc}
\usepackage{polski}

\usepackage{amsthm}  %pakiet do tworzenia twierdzeń itp.
\usepackage{amsmath} %pakiet do niektórych symboli matematycznych
\usepackage{amssymb} %pakiet do symboli mat., np. \nsubseteq
\usepackage{amsfonts}
\usepackage{graphicx} %obsługa plików graficznych z rozszerzeniem png, jpg
\theoremstyle{definition} %styl dla definicji
\newtheorem{zad}{} 
\title{Multizestaw zadań}
\author{Robert Fidytek}
%\date{\today}
\date{}\documentclass[12pt, a4paper]{article}
\usepackage[utf8]{inputenc}
\usepackage{polski}

\usepackage{amsthm}  %pakiet do tworzenia twierdzeń itp.
\usepackage{amsmath} %pakiet do niektórych symboli matematycznych
\usepackage{amssymb} %pakiet do symboli mat., np. \nsubseteq
\usepackage{amsfonts}
\usepackage{graphicx} %obsługa plików graficznych z rozszerzeniem png, jpg
\theoremstyle{definition} %styl dla definicji
\newtheorem{zad}{} 
\title{Multizestaw zadań}
\author{Robert Fidytek}
%\date{\today}
\date{}
\newcounter{liczniksekcji}
\newcommand{\kategoria}[1]{\section{#1}} %olreślamy nazwę kateforii zadań
\newcommand{\zadStart}[1]{\begin{zad}#1\newline} %oznaczenie początku zadania
\newcommand{\zadStop}{\end{zad}}   %oznaczenie końca zadania
%Makra opcjonarne (nie muszą występować):
\newcommand{\rozwStart}[2]{\noindent \textbf{Rozwiązanie (autor #1 , recenzent #2): }\newline} %oznaczenie początku rozwiązania, opcjonarnie można wprowadzić informację o autorze rozwiązania zadania i recenzencie poprawności wykonania rozwiązania zadania
\newcommand{\rozwStop}{\newline}                                            %oznaczenie końca rozwiązania
\newcommand{\odpStart}{\noindent \textbf{Odpowiedź:}\newline}    %oznaczenie początku odpowiedzi końcowej (wypisanie wyniku)
\newcommand{\odpStop}{\newline}                                             %oznaczenie końca odpowiedzi końcowej (wypisanie wyniku)
\newcommand{\testStart}{\noindent \textbf{Test:}\newline} %ewentualne możliwe opcje odpowiedzi testowej: A. ? B. ? C. ? D. ? itd.
\newcommand{\testStop}{\newline} %koniec wprowadzania odpowiedzi testowych
\newcommand{\kluczStart}{\noindent \textbf{Test poprawna odpowiedź:}\newline} %klucz, poprawna odpowiedź pytania testowego (jedna literka): A lub B lub C lub D itd.
\newcommand{\kluczStop}{\newline} %koniec poprawnej odpowiedzi pytania testowego 
\newcommand{\wstawGrafike}[2]{\begin{figure}[h] \includegraphics[scale=#2] {#1} \end{figure}} %gdyby była potrzeba wstawienia obrazka, parametry: nazwa pliku, skala (jak nie wiesz co wpisać, to wpisz 1)

\begin{document}
\maketitle


\kategoria{Wikieł/Z2.11}
\zadStart{Zadanie z Wikieł Z 2.11 moja wersja nr [nrWersji]}
%[p1]:[2,3,4,5,6,7,8,9,10]
%[p2]:[2,3,4,5,6,7,8,9,10]
%[p3]=random.randint(1,10)
%[p1p2]=-[p1]+[p2]
%[p1p1]=-[p1]-[p1]
%[pp1p1]=-[p1p1]
%[d]=math.gcd([pp1p1],[p1p2])
%[l]=int([pp1p1]/[d])
%[m]=int([p1p2]/[d])
%[u2]=round([l]/([m]+0.0000001),2)
%[u1]=-[u2]-1
%[u1k]=[u1]*[u1]
%[u2k]=[u2]*[u2]
%[sum]=int([u1k]+[u2k]+1)
%[d2]=math.gcd([p3],[sum])
%[np3]=int([p3]/[d2])
%[nsum]=int([sum]/[d2])
%[pu1]=-int([u1])
%[p1p2]!=0 and [p1p2]!=1

Znaleźć wektor $\vec{u}$ prostopadły do wektora $\vec{a}=[[p1],[p1],[p1]]$ i $\vec{b}=[[p1],[p2],-[p1]],$ tworzący kąt rozwarty z wektorem $\vec{c}=[0,[p1],0]$ i taki, że $|\vec{u}|=\sqrt{[p3]}$

\zadStop

\rozwStart{Maja Szabłowska}{}
Niech $\vec{u}=[u_{1}, u_{2}, u_{3}]$ oraz $\alpha$ to kąt pomiędzy wektorami $\vec{u}$ i $\vec{c}.$
\begin{enumerate}

    \item $$\vec{u}\circ\vec{a}=0$$
$$[u_{1}, u_{2}, u_{3}]\circ[[p1],[p1],[p1]]=[p1]u_{1}+[p1]u_{2}+[p1]u_{3}=u_{1}+u_{2}+u_{3}=0$$
$$u_{1}=-u_{2}-u_{3}$$

    \item $$\vec{u}\circ\vec{b}=0$$
$$[u_{1}, u_{2}, u_{3}]\circ[[p1],[p2],-[p1]]=[p1]u_{1}+[p2]u_{2}-[p1]u_{3}=0$$
$$[p1](-u_{2}-u_{3})+[p2]u_{2}-[p1]u_{3}=0$$
$$[p1p2]u_{2}[p1p1]u_{3}=0$$
$$[p1p2]u_{2}=[pp1p1]u_{3} \Rightarrow u_{2}=\frac{[l]}{[m]}u_{3}=[u2]u_{3}$$
$$u_{1}=-[u2]u_{3}-u_{3}=[u1]u_{3}$$

    \item $$\cos\alpha=\frac{\vec{u}\circ\vec{c}}{|\vec{u}|\cdot|\vec{c}|}<0 \iff \vec{u}\circ\vec{c}<0$$
$$\vec{u}\circ\vec{c}=[u_{1}, u_{2}, u_{3}]\circ[0,[p1],0]=[p1]u_{2}<0 \iff u_{2}<0$$
$$u_{2}, u_{3}<0, \quad u_{1}>0$$

    \item $$|\vec{u}|=\sqrt{u_{1}^{2}+u_{2}^{2}+u_{3}^{2}}=\sqrt{[p3]}$$
    $$u_{1}^{2}+u_{2}^{2}+u_{3}^{2}=[p3]$$
    $$([u1]u_{3})^{2}+([u2]u_{3})^{2}+u_{3}^{2}=[p3]$$
    $$[u1k]u_{3}^{2}+[u2k]u_{3}^{2}+u_{3}^{2}=[p3]$$
    $$[sum]u_{3}^{2}=[p3]$$
    $$u_{3}^{2}=\frac{[p3]}{[sum]}=\frac{[np3]}{[nsum]} \Rightarrow u_{3}=-\sqrt{\frac{[np3]}{[nsum]}}$$

$$u_{1}=[u1]u_{3}=[pu1]\sqrt{\frac{[np3]}{[nsum]}}$$
$$u_{2}=[u2]u_{3}=-[u2]\sqrt{\frac{[np3]}{[nsum]}}$$
\end{enumerate}

\rozwStop


\odpStart
$\vec{u}=\left[[pu1]\sqrt{\frac{[np3]}{[nsum]}},-[u2]\sqrt{\frac{[np3]}{[nsum]}},-\sqrt{\frac{[np3]}{[nsum]}}\right]$
\odpStop
\testStart
A.$\vec{u}=\left[[pu1]\sqrt{\frac{[np3]}{[nsum]}},-[u2]\sqrt{\frac{[np3]}{[nsum]}},-\sqrt{\frac{[np3]}{[nsum]}}\right]$
B.$\vec{u}=\left[[pu1]\sqrt{\frac{[np3]}{[nsum]}},\sqrt{\frac{[np3]}{[nsum]}},-\sqrt{\frac{[np3]}{[nsum]}}\right]$
C.$\vec{u}=\left[[pu1]\sqrt{\frac{[np3]}{[nsum]}},[u2]\sqrt{\frac{[np3]}{[nsum]}},-\sqrt{\frac{[np3]}{[nsum]}}\right]$
D.$\vec{u}=\left[-[pu1]\sqrt{\frac{[np3]}{[nsum]}},-[u2]\sqrt{\frac{[np3]}{[nsum]}},\sqrt{\frac{[np3]}{[nsum]}}\right]$
E.$\vec{u}=\left[\sqrt{\frac{[np3]}{[nsum]}},\sqrt{\frac{[np3]}{[nsum]}},-\sqrt{\frac{[np3]}{[nsum]}}\right]$
F.$\vec{u}=\left[-\sqrt{\frac{[np3]}{[nsum]}},-\sqrt{\frac{[p3]}{[nsum]}},-\sqrt{\frac{[np3]}{[nsum]}}\right]$
G.$\vec{u}=\left[[u1]\sqrt{\frac{[p3]}{[nsum]}},-[u2]\sqrt{\frac{[np3]}{[sum]}},\sqrt{\frac{[np3]}{[nsum]}}\right]$
H.$\vec{u}=[0,0,0]$
\testStop
\kluczStart
A
\kluczStop



\end{document}
