\documentclass[12pt, a4paper]{article}
\usepackage[utf8]{inputenc}
\usepackage{polski}

\usepackage{amsthm}  %pakiet do tworzenia twierdzeń itp.
\usepackage{amsmath} %pakiet do niektórych symboli matematycznych
\usepackage{amssymb} %pakiet do symboli mat., np. \nsubseteq
\usepackage{amsfonts}
\usepackage{graphicx} %obsługa plików graficznych z rozszerzeniem png, jpg
\theoremstyle{definition} %styl dla definicji
\newtheorem{zad}{} 
\title{Multizestaw zadań}
\author{Robert Fidytek}
%\date{\today}
\date{}
\newcounter{liczniksekcji}
\newcommand{\kategoria}[1]{\section{#1}} %olreślamy nazwę kateforii zadań
\newcommand{\zadStart}[1]{\begin{zad}#1\newline} %oznaczenie początku zadania
\newcommand{\zadStop}{\end{zad}}   %oznaczenie końca zadania
%Makra opcjonarne (nie muszą występować):
\newcommand{\rozwStart}[2]{\noindent \textbf{Rozwiązanie (autor #1 , recenzent #2): }\newline} %oznaczenie początku rozwiązania, opcjonarnie można wprowadzić informację o autorze rozwiązania zadania i recenzencie poprawności wykonania rozwiązania zadania
\newcommand{\rozwStop}{\newline}                                            %oznaczenie końca rozwiązania
\newcommand{\odpStart}{\noindent \textbf{Odpowiedź:}\newline}    %oznaczenie początku odpowiedzi końcowej (wypisanie wyniku)
\newcommand{\odpStop}{\newline}                                             %oznaczenie końca odpowiedzi końcowej (wypisanie wyniku)
\newcommand{\testStart}{\noindent \textbf{Test:}\newline} %ewentualne możliwe opcje odpowiedzi testowej: A. ? B. ? C. ? D. ? itd.
\newcommand{\testStop}{\newline} %koniec wprowadzania odpowiedzi testowych
\newcommand{\kluczStart}{\noindent \textbf{Test poprawna odpowiedź:}\newline} %klucz, poprawna odpowiedź pytania testowego (jedna literka): A lub B lub C lub D itd.
\newcommand{\kluczStop}{\newline} %koniec poprawnej odpowiedzi pytania testowego 
\newcommand{\wstawGrafike}[2]{\begin{figure}[h] \includegraphics[scale=#2] {#1} \end{figure}} %gdyby była potrzeba wstawienia obrazka, parametry: nazwa pliku, skala (jak nie wiesz co wpisać, to wpisz 1)

\begin{document}
\maketitle


\kategoria{Wikieł/Z3.7d}
\zadStart{Zadanie z Wikieł Z 3.7 d)  moja wersja nr [nrWersji]}
%[p1]:[2,3,4,5,6,7,8,9,10]
%[p0]:[1]
%[a1]=math.sin(math.pi/[p1])
%[a2]=round(-2*math.pi*math.sin([a1]),2)
%[a3]=round(-2*math.pi*math.sin([a2]),2)
%[a4]=round(-2*math.pi*math.sin([a3]),2)
%[a5]=round(-2*math.pi*math.sin([a4]),2)

Wypisać pięć początkowych wyrazów ciągu określonego rekurencyjnie.
$$\left\{ \begin{array}{ll}
a_{1}=\frac{\pi}{[p1]}\\
a_{n+1}=-2\pi\cdot \sin(a_{n})& \textrm{dla n$\geq$1} 
\end{array} \right.
$$
\zadStop
\rozwStart{Maja Szabłowska}{}
$$a_{1}=\frac{\pi}{[p1]}$$
$$a_{2}=-2\pi\cdot\sin(a_{1})=[a2]$$
$$a_{3}=-2\pi\cdot\sin(a_{2})=[a3]$$
$$a_{4}=-2\pi\cdot\sin(a_{3})=[a4]$$
$$a_{5}=-2\pi\cdot\sin(a_{4})=[a5]$$
\rozwStop
\odpStart
$\frac{\pi}{[p1]},[a2],[a3],[a4],[a5]$
\odpStop
\testStart
A.$\frac{\pi}{[p1]},[a2],[a3],[a4],[a5]$\\
B.$[p1],[a2],[a3],[a4],[p1]$\\
C.$[p1],[a2],[a2],[a4],[a5]$\\
D.$[p1],[a2],[a2],[a2],[a5]$\\
E.$[p1],[p1],[a3],[a4],[a5]$\\
F.$[p1],[a2],[a3],[a2],[a3]$\\
G.$[p1],[a2],[a3],[a2],[a2]$\\
H.$[a2],[a2],[a3],[a4],[a5]$\\
I.$[a5],[a2],[a3],[a4],[a5]$\\
\testStop
\kluczStart
A
\kluczStop



\end{document}