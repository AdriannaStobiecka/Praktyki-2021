\documentclass[12pt, a4paper]{article}
\usepackage[utf8]{inputenc}
\usepackage{polski}

\usepackage{amsthm}  %pakiet do tworzenia twierdzeń itp.
\usepackage{amsmath} %pakiet do niektórych symboli matematycznych
\usepackage{amssymb} %pakiet do symboli mat., np. \nsubseteq
\usepackage{amsfonts}
\usepackage{graphicx} %obsługa plików graficznych z rozszerzeniem png, jpg
\theoremstyle{definition} %styl dla definicji
\newtheorem{zad}{} 
\title{Multizestaw zadań}
\author{Robert Fidytek}
%\date{\today}
\date{}
\newcounter{liczniksekcji}
\newcommand{\kategoria}[1]{\section{#1}} %olreślamy nazwę kateforii zadań
\newcommand{\zadStart}[1]{\begin{zad}#1\newline} %oznaczenie początku zadania
\newcommand{\zadStop}{\end{zad}}   %oznaczenie końca zadania
%Makra opcjonarne (nie muszą występować):
\newcommand{\rozwStart}[2]{\noindent \textbf{Rozwiązanie (autor #1 , recenzent #2): }\newline} %oznaczenie początku rozwiązania, opcjonarnie można wprowadzić informację o autorze rozwiązania zadania i recenzencie poprawności wykonania rozwiązania zadania
\newcommand{\rozwStop}{\newline}                                            %oznaczenie końca rozwiązania
\newcommand{\odpStart}{\noindent \textbf{Odpowiedź:}\newline}    %oznaczenie początku odpowiedzi końcowej (wypisanie wyniku)
\newcommand{\odpStop}{\newline}                                             %oznaczenie końca odpowiedzi końcowej (wypisanie wyniku)
\newcommand{\testStart}{\noindent \textbf{Test:}\newline} %ewentualne możliwe opcje odpowiedzi testowej: A. ? B. ? C. ? D. ? itd.
\newcommand{\testStop}{\newline} %koniec wprowadzania odpowiedzi testowych
\newcommand{\kluczStart}{\noindent \textbf{Test poprawna odpowiedź:}\newline} %klucz, poprawna odpowiedź pytania testowego (jedna literka): A lub B lub C lub D itd.
\newcommand{\kluczStop}{\newline} %koniec poprawnej odpowiedzi pytania testowego 
\newcommand{\wstawGrafike}[2]{\begin{figure}[h] \includegraphics[scale=#2] {#1} \end{figure}} %gdyby była potrzeba wstawienia obrazka, parametry: nazwa pliku, skala (jak nie wiesz co wpisać, to wpisz 1)

\begin{document}
\maketitle


\kategoria{Wikieł/Z1.37b}
\zadStart{Zadanie z Wikieł Z 1.37 b) moja wersja nr [nrWersji]}
%[a]:[1234,12345,123456,1234567,12345678]
%[b]:[4321,54321,654321,7654321,87654321]
%[ab]=[a]+[b]
%[abm]=[a]*[b]
%[b]>[a]
Rozwiązać nierówność: $x^{2}\leq[ab]x-[a]\cdot[b]$.
\zadStop
\rozwStart{Wojciech Przybylski}{}
$$x^{2}\leq[ab]x-[a]\cdot[b]\Rightarrow x^{2}-[ab]x+[a]\cdot[b]\leq0$$
Przyglądając się naszej nierówności można się zastanowić, czy nie lepiej spróbować zaoszczędzić sobie czas i skorzystać ze wzorów Viete'a, zamiast liczenia delty.
$$\mbox{Wzory Viete'a: }$$
$$ x_{1}+x_{2}=\frac{-b}{a},\hspace{3mm} x_{1}\cdot x_{2}=\frac{c}{a}$$
$$\mbox{Podstawiamy nasze dane: }$$
$$  x_{1}+x_{2}=[ab],\hspace{3mm}x_{1}\cdot x_{2}=[a]\cdot[b]$$
$$\mbox{Sprawdzamy czy iloczyn we współczynniku 'c' będzie równy 'b': }$$
$$ [a]+[b]=[ab]$$
Rzeczywiście tak jest, więc mamy nasze pierwiastki. Teraz wystarczy zauważyć, że parabola ma ramiona skierowane ku górze, więc $x\in[[a],[b]]$.
\rozwStop
\odpStart
$x\in[[a],[b]]$
\odpStop
\testStart
A. $x\in[[a],[b]])$.\\
B. $x\in([a],[b])$.\\
C. $x\in[[ab],[abm]]$.\\
D.$x\in([a],[ab])$.\\
E. $x\in[[a],[ab]]$.\\
F. Nie ma takiego x, który by spełniał tą nierówność
\testStop
\kluczStart
A
\kluczStop



\end{document}