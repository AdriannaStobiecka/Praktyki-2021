\documentclass[12pt, a4paper]{article}
\usepackage[utf8]{inputenc}
\usepackage{polski}
\usepackage{amsthm}  %pakiet do tworzenia twierdzeń itp.
\usepackage{amsmath} %pakiet do niektórych symboli matematycznych
\usepackage{amssymb} %pakiet do symboli mat., np. \nsubseteq
\usepackage{amsfonts}
\usepackage{graphicx} %obsługa plików graficznych z rozszerzeniem png, jpg
\theoremstyle{definition} %styl dla definicji
\newtheorem{zad}{} 
\title{Multizestaw zadań}
\author{Patryk Wirkus}
%\date{\today}
\date{}
\newcommand{\kategoria}[1]{\section{#1}}
\newcommand{\zadStart}[1]{\begin{zad}#1\newline}
\newcommand{\zadStop}{\end{zad}}
\newcommand{\rozwStart}[2]{\noindent \textbf{Rozwiązanie (autor #1 , recenzent #2): }\newline}
\newcommand{\rozwStop}{\newline}                                           
\newcommand{\odpStart}{\noindent \textbf{Odpowiedź:}\newline}
\newcommand{\odpStop}{\newline}
\newcommand{\testStart}{\noindent \textbf{Test:}\newline}
\newcommand{\testStop}{\newline}
\newcommand{\kluczStart}{\noindent \textbf{Test poprawna odpowiedź:}\newline}
\newcommand{\kluczStop}{\newline}
\newcommand{\wstawGrafike}[2]{\begin{figure}[h] \includegraphics[scale=#2] {#1} \end{figure}}

\begin{document}
\maketitle

\kategoria{Wikieł/1.4h}


\zadStart{Zadanie z Wikieł Z 1.4 gh) moja wersja nr 1}

Oblicz wartość wyrażenia $[12,625+(\frac{1}{3})^{-2}\cdot(2\cdot 3^{-1} - 9^{-\frac{1}{2}})]^{-\frac{1}{3}}$.
\zadStop
\rozwStart{Patryk Wirkus}{Martyna Czarnobaj}
$$[12,625+(\frac{1}{3})^{-2}\cdot(2\cdot 3^{-1} - 9^{-\frac{1}{2}})]^{-\frac{1}{3}} = [12,625+9\cdot (\frac{2}{3}-\frac{1}{3})]^{-\frac{1}{3}} =$$
$$=15,625^{-\frac{1}{3}} = (\frac{8}{125})^\frac{1}{3}$$
\rozwStop
\odpStart
$(\frac{8}{125})^\frac{1}{3}$
\odpStop
\testStart
A.$(\frac{8}{125})^\frac{1}{3}$\\ B.$-(\frac{8}{125})^\frac{1}{3}$\\ C.$\frac{8}{125}$\\ D.$-\frac{8}{125}$\\ E.$\frac{1}{3}$
\testStop
\kluczStart
A
\kluczStop



\zadStart{Zadanie z Wikieł Z 1.4 gh) moja wersja nr 2}

Oblicz wartość wyrażenia $[12,625+(\frac{1}{3})^{-2}\cdot(2\cdot 3^{-1} - 9^{-\frac{1}{2}})]^{-\frac{1}{5}}$.
\zadStop
\rozwStart{Patryk Wirkus}{Martyna Czarnobaj}
$$[12,625+(\frac{1}{3})^{-2}\cdot(2\cdot 3^{-1} - 9^{-\frac{1}{2}})]^{-\frac{1}{5}} = [12,625+9\cdot (\frac{2}{3}-\frac{1}{3})]^{-\frac{1}{5}} =$$
$$=15,625^{-\frac{1}{5}} = (\frac{8}{125})^\frac{1}{5}$$
\rozwStop
\odpStart
$(\frac{8}{125})^\frac{1}{5}$
\odpStop
\testStart
A.$(\frac{8}{125})^\frac{1}{5}$\\ B.$-(\frac{8}{125})^\frac{1}{5}$\\ C.$\frac{8}{125}$\\ D.$-\frac{8}{125}$\\ E.$\frac{1}{5}$
\testStop
\kluczStart
A
\kluczStop



\zadStart{Zadanie z Wikieł Z 1.4 gh) moja wersja nr 3}

Oblicz wartość wyrażenia $[12,625+(\frac{1}{3})^{-2}\cdot(2\cdot 3^{-1} - 9^{-\frac{1}{2}})]^{-\frac{1}{7}}$.
\zadStop
\rozwStart{Patryk Wirkus}{Martyna Czarnobaj}
$$[12,625+(\frac{1}{3})^{-2}\cdot(2\cdot 3^{-1} - 9^{-\frac{1}{2}})]^{-\frac{1}{7}} = [12,625+9\cdot (\frac{2}{3}-\frac{1}{3})]^{-\frac{1}{7}} =$$
$$=15,625^{-\frac{1}{7}} = (\frac{8}{125})^\frac{1}{7}$$
\rozwStop
\odpStart
$(\frac{8}{125})^\frac{1}{7}$
\odpStop
\testStart
A.$(\frac{8}{125})^\frac{1}{7}$\\ B.$-(\frac{8}{125})^\frac{1}{7}$\\ C.$\frac{8}{125}$\\ D.$-\frac{8}{125}$\\ E.$\frac{1}{7}$
\testStop
\kluczStart
A
\kluczStop



\zadStart{Zadanie z Wikieł Z 1.4 gh) moja wersja nr 4}

Oblicz wartość wyrażenia $[12,625+(\frac{1}{3})^{-2}\cdot(2\cdot 3^{-1} - 9^{-\frac{1}{2}})]^{-\frac{1}{9}}$.
\zadStop
\rozwStart{Patryk Wirkus}{Martyna Czarnobaj}
$$[12,625+(\frac{1}{3})^{-2}\cdot(2\cdot 3^{-1} - 9^{-\frac{1}{2}})]^{-\frac{1}{9}} = [12,625+9\cdot (\frac{2}{3}-\frac{1}{3})]^{-\frac{1}{9}} =$$
$$=15,625^{-\frac{1}{9}} = (\frac{8}{125})^\frac{1}{9}$$
\rozwStop
\odpStart
$(\frac{8}{125})^\frac{1}{9}$
\odpStop
\testStart
A.$(\frac{8}{125})^\frac{1}{9}$\\ B.$-(\frac{8}{125})^\frac{1}{9}$\\ C.$\frac{8}{125}$\\ D.$-\frac{8}{125}$\\ E.$\frac{1}{9}$
\testStop
\kluczStart
A
\kluczStop



\zadStart{Zadanie z Wikieł Z 1.4 gh) moja wersja nr 5}

Oblicz wartość wyrażenia $[12,625+(\frac{1}{3})^{-2}\cdot(2\cdot 3^{-1} - 9^{-\frac{1}{2}})]^{-\frac{1}{11}}$.
\zadStop
\rozwStart{Patryk Wirkus}{Martyna Czarnobaj}
$$[12,625+(\frac{1}{3})^{-2}\cdot(2\cdot 3^{-1} - 9^{-\frac{1}{2}})]^{-\frac{1}{11}} = [12,625+9\cdot (\frac{2}{3}-\frac{1}{3})]^{-\frac{1}{11}} =$$
$$=15,625^{-\frac{1}{11}} = (\frac{8}{125})^\frac{1}{11}$$
\rozwStop
\odpStart
$(\frac{8}{125})^\frac{1}{11}$
\odpStop
\testStart
A.$(\frac{8}{125})^\frac{1}{11}$\\ B.$-(\frac{8}{125})^\frac{1}{11}$\\ C.$\frac{8}{125}$\\ D.$-\frac{8}{125}$\\ E.$\frac{1}{11}$
\testStop
\kluczStart
A
\kluczStop



\zadStart{Zadanie z Wikieł Z 1.4 gh) moja wersja nr 6}

Oblicz wartość wyrażenia $[12,625+(\frac{1}{3})^{-2}\cdot(2\cdot 3^{-1} - 9^{-\frac{1}{2}})]^{-\frac{1}{13}}$.
\zadStop
\rozwStart{Patryk Wirkus}{Martyna Czarnobaj}
$$[12,625+(\frac{1}{3})^{-2}\cdot(2\cdot 3^{-1} - 9^{-\frac{1}{2}})]^{-\frac{1}{13}} = [12,625+9\cdot (\frac{2}{3}-\frac{1}{3})]^{-\frac{1}{13}} =$$
$$=15,625^{-\frac{1}{13}} = (\frac{8}{125})^\frac{1}{13}$$
\rozwStop
\odpStart
$(\frac{8}{125})^\frac{1}{13}$
\odpStop
\testStart
A.$(\frac{8}{125})^\frac{1}{13}$\\ B.$-(\frac{8}{125})^\frac{1}{13}$\\ C.$\frac{8}{125}$\\ D.$-\frac{8}{125}$\\ E.$\frac{1}{13}$
\testStop
\kluczStart
A
\kluczStop



\zadStart{Zadanie z Wikieł Z 1.4 gh) moja wersja nr 7}

Oblicz wartość wyrażenia $[12,625+(\frac{1}{3})^{-2}\cdot(2\cdot 3^{-1} - 9^{-\frac{1}{2}})]^{-\frac{1}{15}}$.
\zadStop
\rozwStart{Patryk Wirkus}{Martyna Czarnobaj}
$$[12,625+(\frac{1}{3})^{-2}\cdot(2\cdot 3^{-1} - 9^{-\frac{1}{2}})]^{-\frac{1}{15}} = [12,625+9\cdot (\frac{2}{3}-\frac{1}{3})]^{-\frac{1}{15}} =$$
$$=15,625^{-\frac{1}{15}} = (\frac{8}{125})^\frac{1}{15}$$
\rozwStop
\odpStart
$(\frac{8}{125})^\frac{1}{15}$
\odpStop
\testStart
A.$(\frac{8}{125})^\frac{1}{15}$\\ B.$-(\frac{8}{125})^\frac{1}{15}$\\ C.$\frac{8}{125}$\\ D.$-\frac{8}{125}$\\ E.$\frac{1}{15}$
\testStop
\kluczStart
A
\kluczStop



\zadStart{Zadanie z Wikieł Z 1.4 gh) moja wersja nr 8}

Oblicz wartość wyrażenia $[12,625+(\frac{1}{3})^{-2}\cdot(2\cdot 3^{-1} - 9^{-\frac{1}{2}})]^{-\frac{1}{17}}$.
\zadStop
\rozwStart{Patryk Wirkus}{Martyna Czarnobaj}
$$[12,625+(\frac{1}{3})^{-2}\cdot(2\cdot 3^{-1} - 9^{-\frac{1}{2}})]^{-\frac{1}{17}} = [12,625+9\cdot (\frac{2}{3}-\frac{1}{3})]^{-\frac{1}{17}} =$$
$$=15,625^{-\frac{1}{17}} = (\frac{8}{125})^\frac{1}{17}$$
\rozwStop
\odpStart
$(\frac{8}{125})^\frac{1}{17}$
\odpStop
\testStart
A.$(\frac{8}{125})^\frac{1}{17}$\\ B.$-(\frac{8}{125})^\frac{1}{17}$\\ C.$\frac{8}{125}$\\ D.$-\frac{8}{125}$\\ E.$\frac{1}{17}$
\testStop
\kluczStart
A
\kluczStop



\zadStart{Zadanie z Wikieł Z 1.4 gh) moja wersja nr 9}

Oblicz wartość wyrażenia $[12,625+(\frac{1}{3})^{-2}\cdot(2\cdot 3^{-1} - 9^{-\frac{1}{2}})]^{-\frac{1}{19}}$.
\zadStop
\rozwStart{Patryk Wirkus}{Martyna Czarnobaj}
$$[12,625+(\frac{1}{3})^{-2}\cdot(2\cdot 3^{-1} - 9^{-\frac{1}{2}})]^{-\frac{1}{19}} = [12,625+9\cdot (\frac{2}{3}-\frac{1}{3})]^{-\frac{1}{19}} =$$
$$=15,625^{-\frac{1}{19}} = (\frac{8}{125})^\frac{1}{19}$$
\rozwStop
\odpStart
$(\frac{8}{125})^\frac{1}{19}$
\odpStop
\testStart
A.$(\frac{8}{125})^\frac{1}{19}$\\ B.$-(\frac{8}{125})^\frac{1}{19}$\\ C.$\frac{8}{125}$\\ D.$-\frac{8}{125}$\\ E.$\frac{1}{19}$
\testStop
\kluczStart
A
\kluczStop



\zadStart{Zadanie z Wikieł Z 1.4 gh) moja wersja nr 10}

Oblicz wartość wyrażenia $[12,625+(\frac{1}{3})^{-2}\cdot(2\cdot 3^{-1} - 9^{-\frac{1}{2}})]^{-\frac{1}{21}}$.
\zadStop
\rozwStart{Patryk Wirkus}{Martyna Czarnobaj}
$$[12,625+(\frac{1}{3})^{-2}\cdot(2\cdot 3^{-1} - 9^{-\frac{1}{2}})]^{-\frac{1}{21}} = [12,625+9\cdot (\frac{2}{3}-\frac{1}{3})]^{-\frac{1}{21}} =$$
$$=15,625^{-\frac{1}{21}} = (\frac{8}{125})^\frac{1}{21}$$
\rozwStop
\odpStart
$(\frac{8}{125})^\frac{1}{21}$
\odpStop
\testStart
A.$(\frac{8}{125})^\frac{1}{21}$\\ B.$-(\frac{8}{125})^\frac{1}{21}$\\ C.$\frac{8}{125}$\\ D.$-\frac{8}{125}$\\ E.$\frac{1}{21}$
\testStop
\kluczStart
A
\kluczStop





\end{document}
