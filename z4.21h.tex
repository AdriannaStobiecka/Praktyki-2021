\documentclass[12pt, a4paper]{article}
\usepackage[utf8]{inputenc}
\usepackage{polski}
\usepackage{amsthm}  %pakiet do tworzenia twierdzeń itp.
\usepackage{amsmath} %pakiet do niektórych symboli matematycznych
\usepackage{amssymb} %pakiet do symboli mat., np. \nsubseteq
\usepackage{amsfonts}
\usepackage{graphicx} %obsługa plików graficznych z rozszerzeniem png, jpg
\theoremstyle{definition} %styl dla definicji
\newtheorem{zad}{} 
\title{Multizestaw zadań}
\author{Patryk Wirkus}
%\date{\today}
\date{}
\newcommand{\kategoria}[1]{\section{#1}}
\newcommand{\zadStart}[1]{\begin{zad}#1\newline}
\newcommand{\zadStop}{\end{zad}}
\newcommand{\rozwStart}[2]{\noindent \textbf{Rozwiązanie (autor #1 , recenzent #2): }\newline}
\newcommand{\rozwStop}{\newline}                                           
\newcommand{\odpStart}{\noindent \textbf{Odpowiedź:}\newline}
\newcommand{\odpStop}{\newline}
\newcommand{\testStart}{\noindent \textbf{Test:}\newline}
\newcommand{\testStop}{\newline}
\newcommand{\kluczStart}{\noindent \textbf{Test poprawna odpowiedź:}\newline}
\newcommand{\kluczStop}{\newline}
\newcommand{\wstawGrafike}[2]{\begin{figure}[h] \includegraphics[scale=#2] {#1} \end{figure}}

\begin{document}
\maketitle

\kategoria{Wikieł/Z4.21h}


\zadStart{Zadanie z Wikieł Z 4.21 h) moja wersja nr 1}

Obliczyć granicę funkcji $\lim\limits_{x\to\ 103^{2}}\frac{103-\sqrt{x}}{x-103^{2}}$.
\zadStop
\rozwStart{Patryk Wirkus}{Szymon Tokarski}
$$\lim\limits_{x\to\ 103^{2}}\frac{103-\sqrt{x}}{x-103^{2}} = = [\frac{0}{0}] = \lim\limits_{x\to\ 103^{2}}\frac{-(\sqrt{x}-103)}{(\sqrt{x}-103)(\sqrt{x}+103)} = \lim\limits_{x\to\ 103^{2}}\frac{-1}{\sqrt{x}+103} = \frac{-1}{206}$$
\rozwStop
\odpStart
$\frac{-1}{206}$
\odpStop
\testStart
A.$\frac{-1}{206}$ B.$\frac{1}{206}$ C.$0$ D.$\infty$ E.$-\infty$
F.$\frac{2}{103}$ G.$\frac{-2}{103}$
H.$103$
I.$-103$
\testStop
\kluczStart
A
\kluczStop



\zadStart{Zadanie z Wikieł Z 4.21 h) moja wersja nr 2}

Obliczyć granicę funkcji $\lim\limits_{x\to\ 107^{2}}\frac{107-\sqrt{x}}{x-107^{2}}$.
\zadStop
\rozwStart{Patryk Wirkus}{Szymon Tokarski}
$$\lim\limits_{x\to\ 107^{2}}\frac{107-\sqrt{x}}{x-107^{2}} = = [\frac{0}{0}] = \lim\limits_{x\to\ 107^{2}}\frac{-(\sqrt{x}-107)}{(\sqrt{x}-107)(\sqrt{x}+107)} = \lim\limits_{x\to\ 107^{2}}\frac{-1}{\sqrt{x}+107} = \frac{-1}{214}$$
\rozwStop
\odpStart
$\frac{-1}{214}$
\odpStop
\testStart
A.$\frac{-1}{214}$ B.$\frac{1}{214}$ C.$0$ D.$\infty$ E.$-\infty$
F.$\frac{2}{107}$ G.$\frac{-2}{107}$
H.$107$
I.$-107$
\testStop
\kluczStart
A
\kluczStop



\zadStart{Zadanie z Wikieł Z 4.21 h) moja wersja nr 3}

Obliczyć granicę funkcji $\lim\limits_{x\to\ 109^{2}}\frac{109-\sqrt{x}}{x-109^{2}}$.
\zadStop
\rozwStart{Patryk Wirkus}{Szymon Tokarski}
$$\lim\limits_{x\to\ 109^{2}}\frac{109-\sqrt{x}}{x-109^{2}} = = [\frac{0}{0}] = \lim\limits_{x\to\ 109^{2}}\frac{-(\sqrt{x}-109)}{(\sqrt{x}-109)(\sqrt{x}+109)} = \lim\limits_{x\to\ 109^{2}}\frac{-1}{\sqrt{x}+109} = \frac{-1}{218}$$
\rozwStop
\odpStart
$\frac{-1}{218}$
\odpStop
\testStart
A.$\frac{-1}{218}$ B.$\frac{1}{218}$ C.$0$ D.$\infty$ E.$-\infty$
F.$\frac{2}{109}$ G.$\frac{-2}{109}$
H.$109$
I.$-109$
\testStop
\kluczStart
A
\kluczStop



\zadStart{Zadanie z Wikieł Z 4.21 h) moja wersja nr 4}

Obliczyć granicę funkcji $\lim\limits_{x\to\ 113^{2}}\frac{113-\sqrt{x}}{x-113^{2}}$.
\zadStop
\rozwStart{Patryk Wirkus}{Szymon Tokarski}
$$\lim\limits_{x\to\ 113^{2}}\frac{113-\sqrt{x}}{x-113^{2}} = = [\frac{0}{0}] = \lim\limits_{x\to\ 113^{2}}\frac{-(\sqrt{x}-113)}{(\sqrt{x}-113)(\sqrt{x}+113)} = \lim\limits_{x\to\ 113^{2}}\frac{-1}{\sqrt{x}+113} = \frac{-1}{226}$$
\rozwStop
\odpStart
$\frac{-1}{226}$
\odpStop
\testStart
A.$\frac{-1}{226}$ B.$\frac{1}{226}$ C.$0$ D.$\infty$ E.$-\infty$
F.$\frac{2}{113}$ G.$\frac{-2}{113}$
H.$113$
I.$-113$
\testStop
\kluczStart
A
\kluczStop



\zadStart{Zadanie z Wikieł Z 4.21 h) moja wersja nr 5}

Obliczyć granicę funkcji $\lim\limits_{x\to\ 127^{2}}\frac{127-\sqrt{x}}{x-127^{2}}$.
\zadStop
\rozwStart{Patryk Wirkus}{Szymon Tokarski}
$$\lim\limits_{x\to\ 127^{2}}\frac{127-\sqrt{x}}{x-127^{2}} = = [\frac{0}{0}] = \lim\limits_{x\to\ 127^{2}}\frac{-(\sqrt{x}-127)}{(\sqrt{x}-127)(\sqrt{x}+127)} = \lim\limits_{x\to\ 127^{2}}\frac{-1}{\sqrt{x}+127} = \frac{-1}{254}$$
\rozwStop
\odpStart
$\frac{-1}{254}$
\odpStop
\testStart
A.$\frac{-1}{254}$ B.$\frac{1}{254}$ C.$0$ D.$\infty$ E.$-\infty$
F.$\frac{2}{127}$ G.$\frac{-2}{127}$
H.$127$
I.$-127$
\testStop
\kluczStart
A
\kluczStop



\zadStart{Zadanie z Wikieł Z 4.21 h) moja wersja nr 6}

Obliczyć granicę funkcji $\lim\limits_{x\to\ 131^{2}}\frac{131-\sqrt{x}}{x-131^{2}}$.
\zadStop
\rozwStart{Patryk Wirkus}{Szymon Tokarski}
$$\lim\limits_{x\to\ 131^{2}}\frac{131-\sqrt{x}}{x-131^{2}} = = [\frac{0}{0}] = \lim\limits_{x\to\ 131^{2}}\frac{-(\sqrt{x}-131)}{(\sqrt{x}-131)(\sqrt{x}+131)} = \lim\limits_{x\to\ 131^{2}}\frac{-1}{\sqrt{x}+131} = \frac{-1}{262}$$
\rozwStop
\odpStart
$\frac{-1}{262}$
\odpStop
\testStart
A.$\frac{-1}{262}$ B.$\frac{1}{262}$ C.$0$ D.$\infty$ E.$-\infty$
F.$\frac{2}{131}$ G.$\frac{-2}{131}$
H.$131$
I.$-131$
\testStop
\kluczStart
A
\kluczStop



\zadStart{Zadanie z Wikieł Z 4.21 h) moja wersja nr 7}

Obliczyć granicę funkcji $\lim\limits_{x\to\ 137^{2}}\frac{137-\sqrt{x}}{x-137^{2}}$.
\zadStop
\rozwStart{Patryk Wirkus}{Szymon Tokarski}
$$\lim\limits_{x\to\ 137^{2}}\frac{137-\sqrt{x}}{x-137^{2}} = = [\frac{0}{0}] = \lim\limits_{x\to\ 137^{2}}\frac{-(\sqrt{x}-137)}{(\sqrt{x}-137)(\sqrt{x}+137)} = \lim\limits_{x\to\ 137^{2}}\frac{-1}{\sqrt{x}+137} = \frac{-1}{274}$$
\rozwStop
\odpStart
$\frac{-1}{274}$
\odpStop
\testStart
A.$\frac{-1}{274}$ B.$\frac{1}{274}$ C.$0$ D.$\infty$ E.$-\infty$
F.$\frac{2}{137}$ G.$\frac{-2}{137}$
H.$137$
I.$-137$
\testStop
\kluczStart
A
\kluczStop



\zadStart{Zadanie z Wikieł Z 4.21 h) moja wersja nr 8}

Obliczyć granicę funkcji $\lim\limits_{x\to\ 139^{2}}\frac{139-\sqrt{x}}{x-139^{2}}$.
\zadStop
\rozwStart{Patryk Wirkus}{Szymon Tokarski}
$$\lim\limits_{x\to\ 139^{2}}\frac{139-\sqrt{x}}{x-139^{2}} = = [\frac{0}{0}] = \lim\limits_{x\to\ 139^{2}}\frac{-(\sqrt{x}-139)}{(\sqrt{x}-139)(\sqrt{x}+139)} = \lim\limits_{x\to\ 139^{2}}\frac{-1}{\sqrt{x}+139} = \frac{-1}{278}$$
\rozwStop
\odpStart
$\frac{-1}{278}$
\odpStop
\testStart
A.$\frac{-1}{278}$ B.$\frac{1}{278}$ C.$0$ D.$\infty$ E.$-\infty$
F.$\frac{2}{139}$ G.$\frac{-2}{139}$
H.$139$
I.$-139$
\testStop
\kluczStart
A
\kluczStop



\zadStart{Zadanie z Wikieł Z 4.21 h) moja wersja nr 9}

Obliczyć granicę funkcji $\lim\limits_{x\to\ 149^{2}}\frac{149-\sqrt{x}}{x-149^{2}}$.
\zadStop
\rozwStart{Patryk Wirkus}{Szymon Tokarski}
$$\lim\limits_{x\to\ 149^{2}}\frac{149-\sqrt{x}}{x-149^{2}} = = [\frac{0}{0}] = \lim\limits_{x\to\ 149^{2}}\frac{-(\sqrt{x}-149)}{(\sqrt{x}-149)(\sqrt{x}+149)} = \lim\limits_{x\to\ 149^{2}}\frac{-1}{\sqrt{x}+149} = \frac{-1}{298}$$
\rozwStop
\odpStart
$\frac{-1}{298}$
\odpStop
\testStart
A.$\frac{-1}{298}$ B.$\frac{1}{298}$ C.$0$ D.$\infty$ E.$-\infty$
F.$\frac{2}{149}$ G.$\frac{-2}{149}$
H.$149$
I.$-149$
\testStop
\kluczStart
A
\kluczStop



\zadStart{Zadanie z Wikieł Z 4.21 h) moja wersja nr 10}

Obliczyć granicę funkcji $\lim\limits_{x\to\ 151^{2}}\frac{151-\sqrt{x}}{x-151^{2}}$.
\zadStop
\rozwStart{Patryk Wirkus}{Szymon Tokarski}
$$\lim\limits_{x\to\ 151^{2}}\frac{151-\sqrt{x}}{x-151^{2}} = = [\frac{0}{0}] = \lim\limits_{x\to\ 151^{2}}\frac{-(\sqrt{x}-151)}{(\sqrt{x}-151)(\sqrt{x}+151)} = \lim\limits_{x\to\ 151^{2}}\frac{-1}{\sqrt{x}+151} = \frac{-1}{302}$$
\rozwStop
\odpStart
$\frac{-1}{302}$
\odpStop
\testStart
A.$\frac{-1}{302}$ B.$\frac{1}{302}$ C.$0$ D.$\infty$ E.$-\infty$
F.$\frac{2}{151}$ G.$\frac{-2}{151}$
H.$151$
I.$-151$
\testStop
\kluczStart
A
\kluczStop



\zadStart{Zadanie z Wikieł Z 4.21 h) moja wersja nr 11}

Obliczyć granicę funkcji $\lim\limits_{x\to\ 157^{2}}\frac{157-\sqrt{x}}{x-157^{2}}$.
\zadStop
\rozwStart{Patryk Wirkus}{Szymon Tokarski}
$$\lim\limits_{x\to\ 157^{2}}\frac{157-\sqrt{x}}{x-157^{2}} = = [\frac{0}{0}] = \lim\limits_{x\to\ 157^{2}}\frac{-(\sqrt{x}-157)}{(\sqrt{x}-157)(\sqrt{x}+157)} = \lim\limits_{x\to\ 157^{2}}\frac{-1}{\sqrt{x}+157} = \frac{-1}{314}$$
\rozwStop
\odpStart
$\frac{-1}{314}$
\odpStop
\testStart
A.$\frac{-1}{314}$ B.$\frac{1}{314}$ C.$0$ D.$\infty$ E.$-\infty$
F.$\frac{2}{157}$ G.$\frac{-2}{157}$
H.$157$
I.$-157$
\testStop
\kluczStart
A
\kluczStop



\zadStart{Zadanie z Wikieł Z 4.21 h) moja wersja nr 12}

Obliczyć granicę funkcji $\lim\limits_{x\to\ 163^{2}}\frac{163-\sqrt{x}}{x-163^{2}}$.
\zadStop
\rozwStart{Patryk Wirkus}{Szymon Tokarski}
$$\lim\limits_{x\to\ 163^{2}}\frac{163-\sqrt{x}}{x-163^{2}} = = [\frac{0}{0}] = \lim\limits_{x\to\ 163^{2}}\frac{-(\sqrt{x}-163)}{(\sqrt{x}-163)(\sqrt{x}+163)} = \lim\limits_{x\to\ 163^{2}}\frac{-1}{\sqrt{x}+163} = \frac{-1}{326}$$
\rozwStop
\odpStart
$\frac{-1}{326}$
\odpStop
\testStart
A.$\frac{-1}{326}$ B.$\frac{1}{326}$ C.$0$ D.$\infty$ E.$-\infty$
F.$\frac{2}{163}$ G.$\frac{-2}{163}$
H.$163$
I.$-163$
\testStop
\kluczStart
A
\kluczStop



\zadStart{Zadanie z Wikieł Z 4.21 h) moja wersja nr 13}

Obliczyć granicę funkcji $\lim\limits_{x\to\ 167^{2}}\frac{167-\sqrt{x}}{x-167^{2}}$.
\zadStop
\rozwStart{Patryk Wirkus}{Szymon Tokarski}
$$\lim\limits_{x\to\ 167^{2}}\frac{167-\sqrt{x}}{x-167^{2}} = = [\frac{0}{0}] = \lim\limits_{x\to\ 167^{2}}\frac{-(\sqrt{x}-167)}{(\sqrt{x}-167)(\sqrt{x}+167)} = \lim\limits_{x\to\ 167^{2}}\frac{-1}{\sqrt{x}+167} = \frac{-1}{334}$$
\rozwStop
\odpStart
$\frac{-1}{334}$
\odpStop
\testStart
A.$\frac{-1}{334}$ B.$\frac{1}{334}$ C.$0$ D.$\infty$ E.$-\infty$
F.$\frac{2}{167}$ G.$\frac{-2}{167}$
H.$167$
I.$-167$
\testStop
\kluczStart
A
\kluczStop



\zadStart{Zadanie z Wikieł Z 4.21 h) moja wersja nr 14}

Obliczyć granicę funkcji $\lim\limits_{x\to\ 173^{2}}\frac{173-\sqrt{x}}{x-173^{2}}$.
\zadStop
\rozwStart{Patryk Wirkus}{Szymon Tokarski}
$$\lim\limits_{x\to\ 173^{2}}\frac{173-\sqrt{x}}{x-173^{2}} = = [\frac{0}{0}] = \lim\limits_{x\to\ 173^{2}}\frac{-(\sqrt{x}-173)}{(\sqrt{x}-173)(\sqrt{x}+173)} = \lim\limits_{x\to\ 173^{2}}\frac{-1}{\sqrt{x}+173} = \frac{-1}{346}$$
\rozwStop
\odpStart
$\frac{-1}{346}$
\odpStop
\testStart
A.$\frac{-1}{346}$ B.$\frac{1}{346}$ C.$0$ D.$\infty$ E.$-\infty$
F.$\frac{2}{173}$ G.$\frac{-2}{173}$
H.$173$
I.$-173$
\testStop
\kluczStart
A
\kluczStop



\zadStart{Zadanie z Wikieł Z 4.21 h) moja wersja nr 15}

Obliczyć granicę funkcji $\lim\limits_{x\to\ 179^{2}}\frac{179-\sqrt{x}}{x-179^{2}}$.
\zadStop
\rozwStart{Patryk Wirkus}{Szymon Tokarski}
$$\lim\limits_{x\to\ 179^{2}}\frac{179-\sqrt{x}}{x-179^{2}} = = [\frac{0}{0}] = \lim\limits_{x\to\ 179^{2}}\frac{-(\sqrt{x}-179)}{(\sqrt{x}-179)(\sqrt{x}+179)} = \lim\limits_{x\to\ 179^{2}}\frac{-1}{\sqrt{x}+179} = \frac{-1}{358}$$
\rozwStop
\odpStart
$\frac{-1}{358}$
\odpStop
\testStart
A.$\frac{-1}{358}$ B.$\frac{1}{358}$ C.$0$ D.$\infty$ E.$-\infty$
F.$\frac{2}{179}$ G.$\frac{-2}{179}$
H.$179$
I.$-179$
\testStop
\kluczStart
A
\kluczStop



\zadStart{Zadanie z Wikieł Z 4.21 h) moja wersja nr 16}

Obliczyć granicę funkcji $\lim\limits_{x\to\ 251^{2}}\frac{251-\sqrt{x}}{x-251^{2}}$.
\zadStop
\rozwStart{Patryk Wirkus}{Szymon Tokarski}
$$\lim\limits_{x\to\ 251^{2}}\frac{251-\sqrt{x}}{x-251^{2}} = = [\frac{0}{0}] = \lim\limits_{x\to\ 251^{2}}\frac{-(\sqrt{x}-251)}{(\sqrt{x}-251)(\sqrt{x}+251)} = \lim\limits_{x\to\ 251^{2}}\frac{-1}{\sqrt{x}+251} = \frac{-1}{502}$$
\rozwStop
\odpStart
$\frac{-1}{502}$
\odpStop
\testStart
A.$\frac{-1}{502}$ B.$\frac{1}{502}$ C.$0$ D.$\infty$ E.$-\infty$
F.$\frac{2}{251}$ G.$\frac{-2}{251}$
H.$251$
I.$-251$
\testStop
\kluczStart
A
\kluczStop



\zadStart{Zadanie z Wikieł Z 4.21 h) moja wersja nr 17}

Obliczyć granicę funkcji $\lim\limits_{x\to\ 257^{2}}\frac{257-\sqrt{x}}{x-257^{2}}$.
\zadStop
\rozwStart{Patryk Wirkus}{Szymon Tokarski}
$$\lim\limits_{x\to\ 257^{2}}\frac{257-\sqrt{x}}{x-257^{2}} = = [\frac{0}{0}] = \lim\limits_{x\to\ 257^{2}}\frac{-(\sqrt{x}-257)}{(\sqrt{x}-257)(\sqrt{x}+257)} = \lim\limits_{x\to\ 257^{2}}\frac{-1}{\sqrt{x}+257} = \frac{-1}{514}$$
\rozwStop
\odpStart
$\frac{-1}{514}$
\odpStop
\testStart
A.$\frac{-1}{514}$ B.$\frac{1}{514}$ C.$0$ D.$\infty$ E.$-\infty$
F.$\frac{2}{257}$ G.$\frac{-2}{257}$
H.$257$
I.$-257$
\testStop
\kluczStart
A
\kluczStop



\zadStart{Zadanie z Wikieł Z 4.21 h) moja wersja nr 18}

Obliczyć granicę funkcji $\lim\limits_{x\to\ 263^{2}}\frac{263-\sqrt{x}}{x-263^{2}}$.
\zadStop
\rozwStart{Patryk Wirkus}{Szymon Tokarski}
$$\lim\limits_{x\to\ 263^{2}}\frac{263-\sqrt{x}}{x-263^{2}} = = [\frac{0}{0}] = \lim\limits_{x\to\ 263^{2}}\frac{-(\sqrt{x}-263)}{(\sqrt{x}-263)(\sqrt{x}+263)} = \lim\limits_{x\to\ 263^{2}}\frac{-1}{\sqrt{x}+263} = \frac{-1}{526}$$
\rozwStop
\odpStart
$\frac{-1}{526}$
\odpStop
\testStart
A.$\frac{-1}{526}$ B.$\frac{1}{526}$ C.$0$ D.$\infty$ E.$-\infty$
F.$\frac{2}{263}$ G.$\frac{-2}{263}$
H.$263$
I.$-263$
\testStop
\kluczStart
A
\kluczStop



\zadStart{Zadanie z Wikieł Z 4.21 h) moja wersja nr 19}

Obliczyć granicę funkcji $\lim\limits_{x\to\ 269^{2}}\frac{269-\sqrt{x}}{x-269^{2}}$.
\zadStop
\rozwStart{Patryk Wirkus}{Szymon Tokarski}
$$\lim\limits_{x\to\ 269^{2}}\frac{269-\sqrt{x}}{x-269^{2}} = = [\frac{0}{0}] = \lim\limits_{x\to\ 269^{2}}\frac{-(\sqrt{x}-269)}{(\sqrt{x}-269)(\sqrt{x}+269)} = \lim\limits_{x\to\ 269^{2}}\frac{-1}{\sqrt{x}+269} = \frac{-1}{538}$$
\rozwStop
\odpStart
$\frac{-1}{538}$
\odpStop
\testStart
A.$\frac{-1}{538}$ B.$\frac{1}{538}$ C.$0$ D.$\infty$ E.$-\infty$
F.$\frac{2}{269}$ G.$\frac{-2}{269}$
H.$269$
I.$-269$
\testStop
\kluczStart
A
\kluczStop



\zadStart{Zadanie z Wikieł Z 4.21 h) moja wersja nr 20}

Obliczyć granicę funkcji $\lim\limits_{x\to\ 271^{2}}\frac{271-\sqrt{x}}{x-271^{2}}$.
\zadStop
\rozwStart{Patryk Wirkus}{Szymon Tokarski}
$$\lim\limits_{x\to\ 271^{2}}\frac{271-\sqrt{x}}{x-271^{2}} = = [\frac{0}{0}] = \lim\limits_{x\to\ 271^{2}}\frac{-(\sqrt{x}-271)}{(\sqrt{x}-271)(\sqrt{x}+271)} = \lim\limits_{x\to\ 271^{2}}\frac{-1}{\sqrt{x}+271} = \frac{-1}{542}$$
\rozwStop
\odpStart
$\frac{-1}{542}$
\odpStop
\testStart
A.$\frac{-1}{542}$ B.$\frac{1}{542}$ C.$0$ D.$\infty$ E.$-\infty$
F.$\frac{2}{271}$ G.$\frac{-2}{271}$
H.$271$
I.$-271$
\testStop
\kluczStart
A
\kluczStop



\zadStart{Zadanie z Wikieł Z 4.21 h) moja wersja nr 21}

Obliczyć granicę funkcji $\lim\limits_{x\to\ 277^{2}}\frac{277-\sqrt{x}}{x-277^{2}}$.
\zadStop
\rozwStart{Patryk Wirkus}{Szymon Tokarski}
$$\lim\limits_{x\to\ 277^{2}}\frac{277-\sqrt{x}}{x-277^{2}} = = [\frac{0}{0}] = \lim\limits_{x\to\ 277^{2}}\frac{-(\sqrt{x}-277)}{(\sqrt{x}-277)(\sqrt{x}+277)} = \lim\limits_{x\to\ 277^{2}}\frac{-1}{\sqrt{x}+277} = \frac{-1}{554}$$
\rozwStop
\odpStart
$\frac{-1}{554}$
\odpStop
\testStart
A.$\frac{-1}{554}$ B.$\frac{1}{554}$ C.$0$ D.$\infty$ E.$-\infty$
F.$\frac{2}{277}$ G.$\frac{-2}{277}$
H.$277$
I.$-277$
\testStop
\kluczStart
A
\kluczStop





\end{document}
