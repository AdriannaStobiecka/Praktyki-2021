\documentclass[12pt, a4paper]{article}
\usepackage[utf8]{inputenc}
\usepackage{polski}

\usepackage{amsthm}  %pakiet do tworzenia twierdzeń itp.
\usepackage{amsmath} %pakiet do niektórych symboli matematycznych
\usepackage{amssymb} %pakiet do symboli mat., np. \nsubseteq
\usepackage{amsfonts}
\usepackage{graphicx} %obsługa plików graficznych z rozszerzeniem png, jpg
\theoremstyle{definition} %styl dla definicji
\newtheorem{zad}{} 
\title{Multizestaw zadań}
\author{Robert Fidytek}
%\date{\today}
\date{}
\newcounter{liczniksekcji}
\newcommand{\kategoria}[1]{\section{#1}} %olreślamy nazwę kateforii zadań
\newcommand{\zadStart}[1]{\begin{zad}#1\newline} %oznaczenie początku zadania
\newcommand{\zadStop}{\end{zad}}   %oznaczenie końca zadania
%Makra opcjonarne (nie muszą występować):
\newcommand{\rozwStart}[2]{\noindent \textbf{Rozwiązanie (autor #1 , recenzent #2): }\newline} %oznaczenie początku rozwiązania, opcjonarnie można wprowadzić informację o autorze rozwiązania zadania i recenzencie poprawności wykonania rozwiązania zadania
\newcommand{\rozwStop}{\newline}                                            %oznaczenie końca rozwiązania
\newcommand{\odpStart}{\noindent \textbf{Odpowiedź:}\newline}    %oznaczenie początku odpowiedzi końcowej (wypisanie wyniku)
\newcommand{\odpStop}{\newline}                                             %oznaczenie końca odpowiedzi końcowej (wypisanie wyniku)
\newcommand{\testStart}{\noindent \textbf{Test:}\newline} %ewentualne możliwe opcje odpowiedzi testowej: A. ? B. ? C. ? D. ? itd.
\newcommand{\testStop}{\newline} %koniec wprowadzania odpowiedzi testowych
\newcommand{\kluczStart}{\noindent \textbf{Test poprawna odpowiedź:}\newline} %klucz, poprawna odpowiedź pytania testowego (jedna literka): A lub B lub C lub D itd.
\newcommand{\kluczStop}{\newline} %koniec poprawnej odpowiedzi pytania testowego 
\newcommand{\wstawGrafike}[2]{\begin{figure}[h] \includegraphics[scale=#2] {#1} \end{figure}} %gdyby była potrzeba wstawienia obrazka, parametry: nazwa pliku, skala (jak nie wiesz co wpisać, to wpisz 1)

\begin{document}
\maketitle


\kategoria{Wikieł/Z1.14i}
\zadStart{Zadanie z Wikieł Z 1.14 i) moja wersja nr [nrWersji]}
%[a]:[2,4,6,8,10]
%[b]:[3,5,7,9,11,13]
%[c]:[5,7,9,11,13,15]
%[d]:[13,15,17,19]
%[aa1]=-2*[a]
%[aa2]=2*[a]
%[w]=-[b]+[c]
%[l1]=[d]-[b]-[c]
%[l2]=[d]+[b]-[c]
%[l3]=[d]+[b]+[c]
%[aa12]=-[aa1]
%[a]<[b]<[c]<[d] and [l1]>0 and math.gcd([b],[a])==1 and math.gcd([c],[a])==1 and math.gcd([l1],[aa1])==1 and math.gcd([l3],[aa2])==1 and [w]!=[d] and [l1]/[aa1]<0
Rozwiązać równanie $|[a]x-[b]|+|[c]-[a]x|=[d]$.
\zadStop
\rozwStart{Adrianna Stobiecka}{}
$$|[a]x-[b]|+|[c]-[a]x|=[d]$$
Przypadek 1: $x\in\bigg(-\infty,\frac{[b]}{[a]}\bigg)$
$$-[a]x+[b]+[c]-[a]x=[d]$$
$$-[a]x-[a]x=[d]-[b]-[c]$$
$$[aa1]x=[l1]~~\bigg|:[aa1]$$
$$x=-\frac{[l1]}{[aa12]}$$
Mamy więc:
$$x\in\bigg(-\infty,\frac{[b]}{[a]}\bigg)\qquad\land\qquad x=-\frac{[l1]}{[aa12]}$$
Zatem z przypadku 1 otrzymujemy $x=-\frac{[l1]}{[aa12]}$.
\\Przypadek 2: $x\in\bigg[\frac{[b]}{[a]}, \frac{[c]}{[a]}\bigg]$
$$[a]x-[b]+[c]-[a]x=[d]$$
$$[a]x-[a]x=[d]+[b]-[c]$$
$$0\ne[l2]$$
Otrzymujemy sprzeczność, a więc $x\in\emptyset$.
\\Mamy więc:
$$x\in\emptyset\qquad\land\qquad x\in\bigg[\frac{[b]}{[a]}, \frac{[c]}{[a]}\bigg]$$
Zatem z przypadku 2 otrzymujemy $x\in\emptyset$.
\\Przypadek 3: $x\in\bigg(\frac{[c]}{[a]},\infty\bigg)$
$$[a]x-[b]-[c]+[a]x=[d]$$
$$[a]x+[a]x=[d]+[b]+[c]$$
$$[aa2]x=[l3]~~\bigg|:[aa2]$$
$$x=\frac{[l3]}{[aa2]}$$
Mamy więc:
$$x\in\bigg(\frac{[c]}{[a]},\infty\bigg)\qquad\land\qquad x=\frac{[l3]}{[aa2]}$$
Zatem z przypadku 3 otrzymujemy $x=\frac{[l3]}{[aa2]}$.
\\Ostatecznym rozwiązaniem wyjściowego równania jest:
$$x\in\bigg\{-\frac{[l1]}{[aa12]},\frac{[l3]}{[aa2]}\bigg\}$$
\rozwStop
\odpStart
$x\in\bigg\{-\frac{[l1]}{[aa12]},\frac{[l3]}{[aa2]}\bigg\}$
\odpStop
\testStart
A.$x\in\mathbb{R}\setminus\{0\}$
B.$x\in\mathbb{R}$
C.$x\in\bigg\{-\frac{[l1]}{[aa12]},\frac{[l3]}{[aa2]}\bigg\}$
D.$x\in\{0,1\}$
E.$x\in\emptyset$
F.$x\in\{-\frac{[l3]}{[aa2]},1\}$
G.$x\in\{[b],[c]\}$
H.$x\in\{0,[b],[c]\}$
I.$x\in\{-[c],-[b]\}$
\testStop
\kluczStart
C
\kluczStop



\end{document}