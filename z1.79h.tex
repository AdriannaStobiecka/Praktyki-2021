\documentclass[12pt, a4paper]{article}
\usepackage[utf8]{inputenc}
\usepackage{polski}

\usepackage{amsthm}  %pakiet do tworzenia twierdzeń itp.
\usepackage{amsmath} %pakiet do niektórych symboli matematycznych
\usepackage{amssymb} %pakiet do symboli mat., np. \nsubseteq
\usepackage{amsfonts}
\usepackage{graphicx} %obsługa plików graficznych z rozszerzeniem png, jpg
\theoremstyle{definition} %styl dla definicji
\newtheorem{zad}{} 
\title{Multizestaw zadań}
\author{Robert Fidytek}
%\date{\today}
\date{}
\newcounter{liczniksekcji}
\newcommand{\kategoria}[1]{\section{#1}} %olreślamy nazwę kateforii zadań
\newcommand{\zadStart}[1]{\begin{zad}#1\newline} %oznaczenie początku zadania
\newcommand{\zadStop}{\end{zad}}   %oznaczenie końca zadania
%Makra opcjonarne (nie muszą występować):
\newcommand{\rozwStart}[2]{\noindent \textbf{Rozwiązanie (autor #1 , recenzent #2): }\newline} %oznaczenie początku rozwiązania, opcjonarnie można wprowadzić informację o autorze rozwiązania zadania i recenzencie poprawności wykonania rozwiązania zadania
\newcommand{\rozwStop}{\newline}                                            %oznaczenie końca rozwiązania
\newcommand{\odpStart}{\noindent \textbf{Odpowiedź:}\newline}    %oznaczenie początku odpowiedzi końcowej (wypisanie wyniku)
\newcommand{\odpStop}{\newline}                                             %oznaczenie końca odpowiedzi końcowej (wypisanie wyniku)
\newcommand{\testStart}{\noindent \textbf{Test:}\newline} %ewentualne możliwe opcje odpowiedzi testowej: A. ? B. ? C. ? D. ? itd.
\newcommand{\testStop}{\newline} %koniec wprowadzania odpowiedzi testowych
\newcommand{\kluczStart}{\noindent \textbf{Test poprawna odpowiedź:}\newline} %klucz, poprawna odpowiedź pytania testowego (jedna literka): A lub B lub C lub D itd.
\newcommand{\kluczStop}{\newline} %koniec poprawnej odpowiedzi pytania testowego 
\newcommand{\wstawGrafike}[2]{\begin{figure}[h] \includegraphics[scale=#2] {#1} \end{figure}} %gdyby była potrzeba wstawienia obrazka, parametry: nazwa pliku, skala (jak nie wiesz co wpisać, to wpisz 1)

\begin{document}
\maketitle


\kategoria{Wikieł/Z1.79h}
\zadStart{Zadanie z Wikieł Z 1.79 h) moja wersja nr [nrWersji]}
%[a]:[42,24,2,12,10,55,30,78,6,68,60,72,80,15,95,56,4,5,20,18]
%[z]:[2,3,4,5]
%[b]=[z]*[a]
%[b1]=2*[b]
%[b2]=[b]*[b]
%[b3]=[b1]+1
%[b4]=[a]+[b2]
%[d]=[b3]*[b3]-4*[b4]
%[dep]=int(math.sqrt([d]))
%[x1]=int((-[b3]-[dep])/(-2))
%[x2]=int((-[b3]+[dep])/(-2))
%[d]>0 and [dep]-(math.sqrt([d]))==0 and [x2]<[b] and [x1]>[b]
Rozwiązać nierówność $\sqrt{x-[a]}+x>[b]$
\zadStop
\rozwStart{Barbara Bączek}{}
Zaczniemy od wyznaczenia dziedziny.
$$D:x - [a] \geq 0$$
$$D: x \in [[a], \infty)$$
Powróćmy do nierówności:
$$\sqrt{x-[a]}>[b]-x$$
\begin{enumerate}
\item Niech $x \in ([b], \infty)$, wtedy nierówność wyjściowa jest tożsamościowa, bo w zbiorze liczb rzeczywistych pierwiastek kwadratowy jest nieujemny. Zatem dla $x \in ([b], \infty)$ nierówność zachodzi.
\item  Niech $x \in [[a], [b]]$, wtedy obie strony nierówności są nieujemne.
$$\sqrt{x-[a]}>[b]-x$$
$$x-[a]>x^2 -[b1]x + [b2]$$
$$-x^2+ [b3]x -[b4]>0$$
$$\Delta= [d], \hspace{0.2 cm} \sqrt{\Delta}=[dep]$$
$$x_1= [x1] \hspace{0.2 cm} \wedge \hspace{0.2 cm} x_2=[x2]$$
Zatem uwzględniając założenie $x \in [[a], [b]]$, otrzymujemy, że dla $x \in ([x2],[b]]$ nierówność zachodzi.
\end{enumerate}
3. Podsumowując: $x \in ([x2], \infty)$
\rozwStop
\odpStart
$x \in ([x2], \infty)$
\odpStop
\testStart
A.$x \in [[a], \infty)$
B.$x \in ([a], \infty)$
C.$x \in [[x2], \infty)$
D.$x \in ([x2], \infty)$
E.$x \in [[a],[b]]$
G.$x \in (-\infty,[a]) \cup [[b], \infty)$
H.$x \in (-\infty, [x2])$
\testStop
\kluczStart
D
\kluczStop



\end{document}