\documentclass[12pt, a4paper]{article}
\usepackage[utf8]{inputenc}
\usepackage{polski}

\usepackage{amsthm}  %pakiet do tworzenia twierdzeń itp.
\usepackage{amsmath} %pakiet do niektórych symboli matematycznych
\usepackage{amssymb} %pakiet do symboli mat., np. \nsubseteq
\usepackage{amsfonts}
\usepackage{graphicx} %obsługa plików graficznych z rozszerzeniem png, jpg
\theoremstyle{definition} %styl dla definicji
\newtheorem{zad}{} 
\title{Multizestaw zadań}
\author{Radosław Grzyb}
%\date{\today}
\date{}
\newcounter{liczniksekcji}
\newcommand{\kategoria}[1]{\section{#1}} %olreślamy nazwę kateforii zadań
\newcommand{\zadStart}[1]{\begin{zad}#1\newline} %oznaczenie początku zadania
\newcommand{\zadStop}{\end{zad}}   %oznaczenie końca zadania
%Makra opcjonarne (nie muszą występować):
\newcommand{\rozwStart}[2]{\noindent \textbf{Rozwiązanie (autor #1 , recenzent #2): }\newline} %oznaczenie początku rozwiązania, opcjonarnie można wprowadzić informację o autorze rozwiązania zadania i recenzencie poprawności wykonania rozwiązania zadania
\newcommand{\rozwStop}{\newline}                                            %oznaczenie końca rozwiązania
\newcommand{\odpStart}{\noindent \textbf{Odpowiedź:}\newline}    %oznaczenie początku odpowiedzi końcowej (wypisanie wyniku)
\newcommand{\odpStop}{\newline}                                             %oznaczenie końca odpowiedzi końcowej (wypisanie wyniku)
\newcommand{\testStart}{\noindent \textbf{Test:}\newline} %ewentualne możliwe opcje odpowiedzi testowej: A. ? B. ? C. ? D. ? itd.
\newcommand{\testStop}{\newline} %koniec wprowadzania odpowiedzi testowych
\newcommand{\kluczStart}{\noindent \textbf{Test poprawna odpowiedź:}\newline} %klucz, poprawna odpowiedź pytania testowego (jedna literka): A lub B lub C lub D itd.
\newcommand{\kluczStop}{\newline} %koniec poprawnej odpowiedzi pytania testowego 
\newcommand{\wstawGrafike}[2]{\begin{figure}[h] \includegraphics[scale=#2] {#1} \end{figure}} %gdyby była potrzeba wstawienia obrazka, parametry: nazwa pliku, skala (jak nie wiesz co wpisać, to wpisz 1)

\begin{document}
\maketitle


\kategoria{Wikieł/Z5.20a}
\zadStart{Zadanie z Wikieł Z 5.20 a) moja wersja nr [nrWersji]}
%[a]:[2,3,4,5,6,7,8,9,10]
%[b]:[1,2,3,4,5,6,7,8,9,10]
%[c]:[2,3,4,5,6,7,8,9,10,11,12]
%[d]=[a]*[b]
%[f]=[c]*[b]
%math.gcd([a],[c])==1
%math.gcd([a],[f])==1
Znaleźć równania asymptot wykresu funkcji $f$ danej wzorem:
$$f(x)=\frac{[c]x^2}{[a](x-[b])}$$
\zadStop
\rozwStart{Klaudia Klejdysz}{}
\begin{enumerate}
    \item Badamy istnienie asymptot pionowych.
    Ustalamy dziedzinę funkcji: $D_f=\mathbb{R}\backslash{[b]}$.\\
    Obliczamy granicę jednostronną funkcji w punkcie $x=[b]$.
    $$\lim_{x \rightarrow [b]^{-}} \frac{[c]x^2}{[a](x-[b])}=\lim_{x \rightarrow [b]^{-}} \frac{x([c]x)}{x([a]-\frac{[d]}{x})}=\lim_{x \rightarrow [b]^{-}} \frac{[c]x}{[a]-\frac{[d]}{x}}=\frac{[f]}{0^{-}}=-\infty$$
    $$\lim_{x \rightarrow [b]^{+}} \frac{[c]x^2}{[a](x-[b])}=\lim_{x \rightarrow [b]^{+}} \frac{x([c]x)}{x([a]-\frac{[d]}{x})}=\lim_{x \rightarrow [b]^{+}} \frac{[c]x}{[a]-\frac{[d]}{x}}=\frac{[f]}{0^{+}}=\infty$$
    Nie otrzymaliśmy granicy właściwej, więc prosta o równaniu $x=[b]$ jest asymptotą pionową obustronną wykresu funkcji $f$.
    \item Badamy istnienie asymptot ukośnych.
     $$\lim_{x \rightarrow -\infty} \frac{[c]x^2}{[a](x-[b])}\frac{1}{x}=\lim_{x \rightarrow -\infty} \frac{[c]x^2}{x^2([a]-\frac{[d]}{x})}=\lim_{x \rightarrow -\infty} \frac{[c]}{[a]-\frac{[d]}{x}}=\lim_{x \rightarrow -\infty} \frac{[c]}{[a]-0}=\frac{[c]}{[a]}$$
     i
     $$\lim_{x \rightarrow -\infty} \frac{[c]x^2}{[a](x-[b])}-\frac{[c]}{[a]}x=\lim_{x \rightarrow -\infty} \frac{[c]x^2}{[a](x-[b])}-\frac{[c]x(x-[b])}{[a](x-[b])}=\lim_{x \rightarrow -\infty}\frac{[c]x^2-[c]x^2+[f]x}{[a](x-[b])}=$$
     $$=\lim_{x \rightarrow -\infty}\frac{[f]x}{[a](x-[b])}=\lim_{x \rightarrow -\infty}\frac{[f]x}{x([a]-\frac{[f]}{x})}=\lim_{x \rightarrow -\infty}\frac{[f]}{[a]-\frac{[f]}{x}}=\frac{[f]}{[a]}$$
     oraz
      $$\lim_{x \rightarrow \infty} \frac{[c]x^2}{[a](x-[b])}\frac{1}{x}=\lim_{x \rightarrow \infty} \frac{[c]x^2}{x^2([a]-\frac{[d]}{x})}=\lim_{x \rightarrow \infty} \frac{[c]}{[a]-\frac{[d]}{x}}=\lim_{x \rightarrow \infty} \frac{[c]}{[a]-0}=\frac{[c]}{[a]}$$
      i
      $$\lim_{x \rightarrow \infty} \frac{[c]x^2}{[a](x-[b])}-\frac{[c]}{[a]}x=\lim_{x \rightarrow \infty} \frac{[c]x^2}{[a](x-[b])}-\frac{[c]x(x-[b])}{[a](x-[b])}=\lim_{x \rightarrow \infty}\frac{[c]x^2-[c]x^2+[f]x}{[a](x-[b])}=$$
     $$=\lim_{x \rightarrow \infty}\frac{[f]x}{[a](x-[b])}=\lim_{x \rightarrow \infty}\frac{[f]x}{x([a]-\frac{[f]}{x})}=\lim_{x \rightarrow \infty}\frac{[f]}{[a]-\frac{[f]}{x}}=\frac{[f]}{[a]}$$
     Zatem prosta o równaniu $y=\frac{[c]}{[a]}x+\frac{[f]}{[a]}$ jest asymptotą ukośną obustronną wykresu funkcji $f$.
\end{enumerate}
\rozwStop
\odpStart
asymptota pionowa: $x=[b]$, asymptota ukośna: $y=\frac{[c]}{[a]}x+\frac{[f]}{[a]}$
\odpStop
\testStart
A. asymptota pionowa: $x=[b]$, asymptota ukośna: $y=\frac{[c]}{[a]}x+\frac{[f]}{[a]}$\\
B. asymptota pionowa: $x=[a]$, asymptota ukośna: $y=\frac{[c]}{[a]}x+\frac{[f]}{[a]}$\\
C. asymptota pionowa: $x=[b]$, asymptota ukośna: $y=\frac{[f]}{[a]}x+\frac{[b]}{[a]}$\\
D. asymptota pionowa: $y=\frac{[c]}{[a]}x+\frac{[f]}{[a]}$, asymptota ukośna: $x=[b]$\\
E. asymptota pionowa: $y=\frac{[c]}{[a]}x$, asymptota ukośna: $x=[b]$\\
F. asymptota pionowa: $y=\frac{[f]}{[a]}$, asymptota ukośna: $x=[b]$
\testStop
\kluczStart
A
\kluczStop



\end{document}

