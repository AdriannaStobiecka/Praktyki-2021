\documentclass[12pt, a4paper]{article}
\usepackage[utf8]{inputenc}
\usepackage{polski}

\usepackage{amsthm}  %pakiet do tworzenia twierdzeń itp.
\usepackage{amsmath} %pakiet do niektórych symboli matematycznych
\usepackage{amssymb} %pakiet do symboli mat., np. \nsubseteq
\usepackage{amsfonts}
\usepackage{graphicx} %obsługa plików graficznych z rozszerzeniem png, jpg
\theoremstyle{definition} %styl dla definicji
\newtheorem{zad}{} 
\title{Multizestaw zadań}
\author{Robert Fidytek}
%\date{\today}
\date{}
\newcounter{liczniksekcji}
\newcommand{\kategoria}[1]{\section{#1}} %olreślamy nazwę kateforii zadań
\newcommand{\zadStart}[1]{\begin{zad}#1\newline} %oznaczenie początku zadania
\newcommand{\zadStop}{\end{zad}}   %oznaczenie końca zadania
%Makra opcjonarne (nie muszą występować):
\newcommand{\rozwStart}[2]{\noindent \textbf{Rozwiązanie (autor #1 , recenzent #2): }\newline} %oznaczenie początku rozwiązania, opcjonarnie można wprowadzić informację o autorze rozwiązania zadania i recenzencie poprawności wykonania rozwiązania zadania
\newcommand{\rozwStop}{\newline}                                            %oznaczenie końca rozwiązania
\newcommand{\odpStart}{\noindent \textbf{Odpowiedź:}\newline}    %oznaczenie początku odpowiedzi końcowej (wypisanie wyniku)
\newcommand{\odpStop}{\newline}                                             %oznaczenie końca odpowiedzi końcowej (wypisanie wyniku)
\newcommand{\testStart}{\noindent \textbf{Test:}\newline} %ewentualne możliwe opcje odpowiedzi testowej: A. ? B. ? C. ? D. ? itd.
\newcommand{\testStop}{\newline} %koniec wprowadzania odpowiedzi testowych
\newcommand{\kluczStart}{\noindent \textbf{Test poprawna odpowiedź:}\newline} %klucz, poprawna odpowiedź pytania testowego (jedna literka): A lub B lub C lub D itd.
\newcommand{\kluczStop}{\newline} %koniec poprawnej odpowiedzi pytania testowego 
\newcommand{\wstawGrafike}[2]{\begin{figure}[h] \includegraphics[scale=#2] {#1} \end{figure}} %gdyby była potrzeba wstawienia obrazka, parametry: nazwa pliku, skala (jak nie wiesz co wpisać, to wpisz 1)

\begin{document}
\maketitle


\kategoria{Wikieł/Z1.129p}
\zadStart{Zadanie z Wikieł Z 1.129 p) moja wersja nr [nrWersji]}
%[p1]:[2,3,4,5,6,7,8,9,10]
%[p2]:[2,3,4,5,6,7,8,9,10]
%[p3]:[2,3,4,5,6,7,8,9,10]
%[p4]:[2,3,4,5,6,7,8,9,10]
%[pp1]=round(math.sqrt([p1]),2)
%[p11]=[p1]+1
%[pp11]=round(math.sqrt([p11]),2)
%[del]=[p3]*[p3]+4*[p2]*[p4]
%[pdel]=round(math.sqrt([del]),2)
%[2p2]=2*[p2]
%[x1]=round((-[p3]-[pdel])/[2p2],2)
%[x2]=round((-[p3]+[pdel])/[2p2],2)
%[x1]<[x2] and [x1]<-[pp1] and [x2]>[pp1] 

Wyznaczyć dziedzinę naturalną funkcji.
$$f(x)=\log_{x^{2}-[p1]}([p2]x^{2}+[p3]x-[p4])$$
\zadStop

\rozwStart{Maja Szabłowska}{}
$$x^{2}-[p1]>0 \quad \land \quad x^{2}-[p1]\neq 1 $$
$$(x-[pp1])(x+[pp1])>0 \quad \land \quad x^{2}-[p11]\neq0 $$
$$x\in(-\infty,-[pp1])\cup([pp1],\infty) \quad \land \quad (x-[pp11])(x+[pp11])\neq0$$
$$x\in(-\infty,-[pp1])\cup([pp1],\infty) \quad \land \quad x\neq-[pp11]\quad \land \quad x\neq[pp11]$$

$$[p2]x^{2}+[p3]x-[p4]>0$$
$$\Delta=[p3]^{2}-4\cdot[p2]\cdot(-[p4])=[del] \Rightarrow \sqrt{\Delta}=[pdel]$$
$$x_{1}=\frac{-[p3]-[pdel]}{[2p2]}=[x1], \quad x_{2}=\frac{-[p3]+[pdel]}{[2p2]}=[x2]$$
$$x\in(-\infty,[x1])\cup([x2],\infty)$$

Ostatecznie:
$$x\in(-\infty,[x1])\cup([x2],\infty)\setminus{[pp11]}$$
\rozwStop
\odpStart
$x\in(-\infty,[x1])\cup([x2],\infty)\setminus{[pp11]}$
\odpStop
\testStart
A.$x\in(-\infty,[x1])\cup([x2],\infty)\setminus{[pp11]}$
B.$x\in[e^{[p2]},\infty)$
C.$x\in(-\infty, 0)$
D.$x\in(-\infty, -[p2]] \cup [\ln[p1],\infty)$
E.$x\in[[p1],\infty)$
F.$x\in(0,\infty)$
G.$x\in\emptyset$
\testStop
\kluczStart
A
\kluczStop



\end{document}
