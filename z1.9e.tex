\documentclass[12pt, a4paper]{article}
\usepackage[utf8]{inputenc}
\usepackage{polski}

\usepackage{amsthm}  %pakiet do tworzenia twierdzeń itp.
\usepackage{amsmath} %pakiet do niektórych symboli matematycznych
\usepackage{amssymb} %pakiet do symboli mat., np. \nsubseteq
\usepackage{amsfonts}
\usepackage{graphicx} %obsługa plików graficznych z rozszerzeniem png, jpg
\theoremstyle{definition} %styl dla definicji
\newtheorem{zad}{} 
\title{Multizestaw zadań}
\author{Robert Fidytek}
%\date{\today}
\date{}\documentclass[12pt, a4paper]{article}
\usepackage[utf8]{inputenc}
\usepackage{polski}

\usepackage{amsthm}  %pakiet do tworzenia twierdzeń itp.
\usepackage{amsmath} %pakiet do niektórych symboli matematycznych
\usepackage{amssymb} %pakiet do symboli mat., np. \nsubseteq
\usepackage{amsfonts}
\usepackage{graphicx} %obsługa plików graficznych z rozszerzeniem png, jpg
\theoremstyle{definition} %styl dla definicji
\newtheorem{zad}{} 
\title{Multizestaw zadań}
\author{Robert Fidytek}
%\date{\today}
\date{}
\newcounter{liczniksekcji}
\newcommand{\kategoria}[1]{\section{#1}} %olreślamy nazwę kateforii zadań
\newcommand{\zadStart}[1]{\begin{zad}#1\newline} %oznaczenie początku zadania
\newcommand{\zadStop}{\end{zad}}   %oznaczenie końca zadania
%Makra opcjonarne (nie muszą występować):
\newcommand{\rozwStart}[2]{\noindent \textbf{Rozwiązanie (autor #1 , recenzent #2): }\newline} %oznaczenie początku rozwiązania, opcjonarnie można wprowadzić informację o autorze rozwiązania zadania i recenzencie poprawności wykonania rozwiązania zadania
\newcommand{\rozwStop}{\newline}                                            %oznaczenie końca rozwiązania
\newcommand{\odpStart}{\noindent \textbf{Odpowiedź:}\newline}    %oznaczenie początku odpowiedzi końcowej (wypisanie wyniku)
\newcommand{\odpStop}{\newline}                                             %oznaczenie końca odpowiedzi końcowej (wypisanie wyniku)
\newcommand{\testStart}{\noindent \textbf{Test:}\newline} %ewentualne możliwe opcje odpowiedzi testowej: A. ? B. ? C. ? D. ? itd.
\newcommand{\testStop}{\newline} %koniec wprowadzania odpowiedzi testowych
\newcommand{\kluczStart}{\noindent \textbf{Test poprawna odpowiedź:}\newline} %klucz, poprawna odpowiedź pytania testowego (jedna literka): A lub B lub C lub D itd.
\newcommand{\kluczStop}{\newline} %koniec poprawnej odpowiedzi pytania testowego 
\newcommand{\wstawGrafike}[2]{\begin{figure}[h] \includegraphics[scale=#2] {#1} \end{figure}} %gdyby była potrzeba wstawienia obrazka, parametry: nazwa pliku, skala (jak nie wiesz co wpisać, to wpisz 1)

\begin{document}
\maketitle


\kategoria{Wikieł/Z1.9e}
\zadStart{Zadanie z Wikieł Z 1.9e moja wersja nr [nrWersji]}
%[p1]:[2,3,5,6,7,8,10]
%[p2]:[2,3,4,5,6,7,8,9,10]
%[p3]=random.randint(2,10)
%[a]=-[p3]*0.5
%[ap]=-[a]
%[2a]=2*[a]
%[b]=-1-[2a]
%[c]=int(-3-[2a])
%[p1p2]=[p1]*[p2]
%math.gcd([c],2)==1


Uprościć wyrażenie $\sqrt{[p1]x^{-2}y^{-3}}\div[(\frac{x^{2}}{[p2]y^{-1}})^{-[p3]}]^{0,5}$

\zadStop

\rozwStart{Maja Szabłowska}{}
$$\sqrt{[p1]x^{-2}y^{-3}}\div\left[\left(\frac{x^{2}}{[p2]y^{-1}}\right)^{-[p3]}\right]^{0,5}=\sqrt{[p1]}x^{-1}y^{-\frac{3}{2}}\div\left(\frac{x^{2}}{[p2]y^{-1}}\right)^{[a]}=$$

$$=\sqrt{[p1]}x^{-1}y^{-\frac{3}{2}}\div \frac{x^{[2a]}}{[p2]y^{[ap]}}= \sqrt{[p1]}x^{-1}y^{-\frac{3}{2}}\cdot \frac{[p2]y^{[ap]}}{x^{[2a]}}=$$

$$=\sqrt{[p1p2]}x^{[b]}y^{\frac{[c]}{2}}$$
\rozwStop


\odpStart
$\sqrt{[p1p2]}x^{[b]}y^{\frac{[c]}{2}}$
\odpStop
\testStart
A.$\sqrt{[p1p2]}x^{[b]}y^{\frac{[c]}{2}}$
B.$[p1p2]x^{[b]}y^{\frac{[c]}{2}}$
C.$\sqrt{[p1p2]x^{[b]}y^{\frac{[c]}{2}}}$
D.$\sqrt{[p1p2]}x^{[b]}y^{[c]}$
E.$x^{[b]}y^{\frac{[c]}{2}}$
F.$\sqrt{[p1]}x^{[b]}y^{\frac{[c]}{2}}$


\testStop
\kluczStart
A
\kluczStop



\end{document}
