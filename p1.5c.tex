\documentclass[12pt, a4paper]{article}
\usepackage[utf8]{inputenc}
\usepackage{polski}

\usepackage{amsthm}  %pakiet do tworzenia twierdzeń itp.
\usepackage{amsmath} %pakiet do niektórych symboli matematycznych
\usepackage{amssymb} %pakiet do symboli mat., np. \nsubseteq
\usepackage{amsfonts}
\usepackage{graphicx} %obsługa plików graficznych z rozszerzeniem png, jpg
\theoremstyle{definition} %styl dla definicji
\newtheorem{zad}{} 
\title{Multizestaw zadań}
\author{Robert Fidytek}
%\date{\today}
\date{}
\newcounter{liczniksekcji}
\newcommand{\kategoria}[1]{\section{#1}} %olreślamy nazwę kateforii zadań
\newcommand{\zadStart}[1]{\begin{zad}#1\newline} %oznaczenie początku zadania
\newcommand{\zadStop}{\end{zad}}   %oznaczenie końca zadania
%Makra opcjonarne (nie muszą występować):
\newcommand{\rozwStart}[2]{\noindent \textbf{Rozwiązanie (autor #1 , recenzent #2): }\newline} %oznaczenie początku rozwiązania, opcjonarnie można wprowadzić informację o autorze rozwiązania zadania i recenzencie poprawności wykonania rozwiązania zadania
\newcommand{\rozwStop}{\newline}                                            %oznaczenie końca rozwiązania
\newcommand{\odpStart}{\noindent \textbf{Odpowiedź:}\newline}    %oznaczenie początku odpowiedzi końcowej (wypisanie wyniku)
\newcommand{\odpStop}{\newline}                                             %oznaczenie końca odpowiedzi końcowej (wypisanie wyniku)
\newcommand{\testStart}{\noindent \textbf{Test:}\newline} %ewentualne możliwe opcje odpowiedzi testowej: A. ? B. ? C. ? D. ? itd.
\newcommand{\testStop}{\newline} %koniec wprowadzania odpowiedzi testowych
\newcommand{\kluczStart}{\noindent \textbf{Test poprawna odpowiedź:}\newline} %klucz, poprawna odpowiedź pytania testowego (jedna literka): A lub B lub C lub D itd.
\newcommand{\kluczStop}{\newline} %koniec poprawnej odpowiedzi pytania testowego 
\newcommand{\wstawGrafike}[2]{\begin{figure}[h] \includegraphics[scale=#2] {#1} \end{figure}} %gdyby była potrzeba wstawienia obrazka, parametry: nazwa pliku, skala (jak nie wiesz co wpisać, to wpisz 1)

\begin{document}
\maketitle


\kategoria{Wikieł/P1.5c}
\zadStart{Zadanie z Wikieł P 1.5 c) moja wersja nr [nrWersji]}
%[p1]:[1,2,3,4,5,6,7,8,9]
%[p2]:[2,3,4,5,6,7,8,9]
%[p3]:[3,5,7,9,11,13]
%[a]=random.randint(2,10)
%[b]=[a]*[p1]
%[c]=[a]*[p2]
%[d]=[a]*[p3]
%[e]=[p2]-[p1]
%[f]=[p3]+[e]
%[g]=[p2]+[p1]
%[h]= [p3]-[g]
%[i]=[p3]-[e]
%[p1]<[p2] and [p3]-[p1]>[p2] and [p3]-[p2]>0 and [p3]-[p1]<=2[p2] and [p2]-[p3]<=3[p1] and math.gcd([p3]+[p2]-[p1],2)==1 and math.gcd([p3]-[p2]+[p1],2)==1 and [p2]+[p1]!=[p3]
Rozwiąż równanie $|[a]x + [b]| + |[c] - [a]x| = [d]$.
\zadStop
\rozwStart{Małgorzata Ugowska}{}
$$|[a]x + [b]| + |[c] - [a]x| = [d] \Leftrightarrow$$ 
$$|[a](x + [p1])| + |-[a](x - [p2])| = [d] \Leftrightarrow$$
$$[a]|x + [p1]| + [a]|x - [p2]| = [d] \Leftrightarrow$$
$$|x + [p1]| + |x - [p2]| = [p3] \Leftrightarrow$$
Sprawdzamy 4 warunki:\\
1.) $ x > -[p1], x > [p2]$\\
$$ (x + [p1]) + (x - [p2]) = [p3] $$
$$ 2x - [e] = [p3] $$
$$ 2x  = [f] $$
$$ x  = \frac{[f]}{2} $$
$x \in ([p2],\infty)$\\
2.) $ x > -[p1], x \leq [p2]$\\
$$ (x + [p1]) + (-(x - [p2])) = [p3] $$
$$ [g] \ne [p3] $$
sprzeczno\'sć, brak rozwiązań\\
3.) $ x \leq -[p1], x < [p2]$\\
$$ -(x + [p1]) + (-(x - [p2])) = [p3] $$
$$ -2x + [e] = [p3] $$
$$ -2x  = [i] $$
$$ x  = -\frac{[i]}{2} $$
$x \in (-\infty,-[p1]]$\\
4.) $ x < -[p1], x > [p2]$ sprzeczno\'sć
\rozwStop
\odpStart
dla przedziału $([p2],\infty):  x  = \frac{[f]}{2} $\\
dla przedziału $(-\infty,-[p1]):  x  = -\frac{[i]}{2} $\\
\odpStop
\testStart
A.\\
dla przedziału $([p2],\infty):  x  = -\frac{[f]}{2} $\\
dla przedziału $(-\infty,[p1]):  x  = -\frac{[i]}{2} $\\
B.\\
dla przedziału $([p2],\infty):  x  = \frac{[f]}{2} $\\
dla przedziału $(-\infty,[p1]):  x  = \frac{[i]}{2} $\\
C.\\
dla przedziału $([p1],\infty):  x  = \frac{[p1]}{2} $\\
dla przedziału $([p1],[p2]):  x  = [h] $\\
dla przedziału $(-\infty,[p2]):  x  = \frac{[p2]}{2} $\\
D.\\
dla przedziału $([p1],\infty):  x  = \frac{[f]}{3} $\\
dla przedziału $(-[p1],[p2]):  x  = [h] $\\
dla przedziału $(-\infty,[p2]):  x  = \frac{[i]}{3} $\\
E.\\
dla przedziału $([p2],\infty):  x  = \frac{[f]}{2} $\\
dla przedziału $(-\infty,-[p1]):  x  = -\frac{[i]}{2} $\\
\testStop
\kluczStart
E
\kluczStop



\end{document}