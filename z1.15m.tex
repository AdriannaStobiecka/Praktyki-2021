\documentclass[12pt, a4paper]{article}
\usepackage[utf8]{inputenc}
\usepackage{polski}

\usepackage{amsthm}  %pakiet do tworzenia twierdzeń itp.
\usepackage{amsmath} %pakiet do niektórych symboli matematycznych
\usepackage{amssymb} %pakiet do symboli mat., np. \nsubseteq
\usepackage{amsfonts}
\usepackage{graphicx} %obsługa plików graficznych z rozszerzeniem png, jpg
\theoremstyle{definition} %styl dla definicji
\newtheorem{zad}{} 
\title{Multizestaw zadań}
\author{Robert Fidytek}
%\date{\today}
\date{}
\newcounter{liczniksekcji}
\newcommand{\kategoria}[1]{\section{#1}} %olreślamy nazwę kateforii zadań
\newcommand{\zadStart}[1]{\begin{zad}#1\newline} %oznaczenie początku zadania
\newcommand{\zadStop}{\end{zad}}   %oznaczenie końca zadania
%Makra opcjonarne (nie muszą występować):
\newcommand{\rozwStart}[2]{\noindent \textbf{Rozwiązanie (autor #1 , recenzent #2): }\newline} %oznaczenie początku rozwiązania, opcjonarnie można wprowadzić informację o autorze rozwiązania zadania i recenzencie poprawności wykonania rozwiązania zadania
\newcommand{\rozwStop}{\newline}                                            %oznaczenie końca rozwiązania
\newcommand{\odpStart}{\noindent \textbf{Odpowiedź:}\newline}    %oznaczenie początku odpowiedzi końcowej (wypisanie wyniku)
\newcommand{\odpStop}{\newline}                                             %oznaczenie końca odpowiedzi końcowej (wypisanie wyniku)
\newcommand{\testStart}{\noindent \textbf{Test:}\newline} %ewentualne możliwe opcje odpowiedzi testowej: A. ? B. ? C. ? D. ? itd.
\newcommand{\testStop}{\newline} %koniec wprowadzania odpowiedzi testowych
\newcommand{\kluczStart}{\noindent \textbf{Test poprawna odpowiedź:}\newline} %klucz, poprawna odpowiedź pytania testowego (jedna literka): A lub B lub C lub D itd.
\newcommand{\kluczStop}{\newline} %koniec poprawnej odpowiedzi pytania testowego 
\newcommand{\wstawGrafike}[2]{\begin{figure}[h] \includegraphics[scale=#2] {#1} \end{figure}} %gdyby była potrzeba wstawienia obrazka, parametry: nazwa pliku, skala (jak nie wiesz co wpisać, to wpisz 1)

\begin{document}
\maketitle


\kategoria{Wikieł/Z1.15m}
\zadStart{Zadanie z Wikieł Z 1.15 m) moja wersja nr [nrWersji]}
%[a]:[-11,-9,-7,-5,-3]
%[b]:[3,5,7,11,13]
%[c]:[2,4,6,8,10]
%[d]:[2,4,6,8,10]
%[dc2]=-[d]-[c]
%[dc1]=[d]-[c]
%[odp1]=[d]-[c]
%[odp2]=[d]+[c]
%[a]!=[b]!=[c]!=[d]!=[dc1]!=[dc2]!=[odp1]!=[odp2] and [odp1]<[odp2] and math.gcd([dc1],[b])==1 and math.gcd([dc2],[b])==1
Rozwiąż nierówność $[a]\leq|[b]x+[c]|<[d]$.
\zadStop
\rozwStart{Adrianna Stobiecka}{}
$$[a]\leq|[b]x+[c]|<[d]$$
$$|[b]x+[c]|\geq[a]\qquad\land\qquad|[b]x+[c]|<[d]$$
$$x\in\mathbb{R}\qquad\land\qquad[b]x+[c]<[d]\land[b]x+[c]>-[d]$$
$$x\in\mathbb{R}\qquad\land\qquad[b]x<[d]-[c]\land[b]x>-[d]-[c]$$
$$x\in\mathbb{R}\qquad\land\qquad[b]x<[dc1]\land[b]x>[dc2]$$
$$x\in\mathbb{R}\qquad\land\qquad x<\frac{[dc1]}{[b]}\land x>\frac{[dc2]}{[b]}$$
$$x\in\mathbb{R}\qquad\land\qquad x\in\bigg(\frac{[dc2]}{[b]},\frac{[dc1]}{[b]}\bigg)$$
$$x\in\bigg(\frac{[dc2]}{[b]},\frac{[dc1]}{[b]}\bigg)$$
\rozwStop
\odpStart
$x\in\bigg(\frac{[dc2]}{[b]},\frac{[dc1]}{[b]}\bigg)$
\odpStop
\testStart
A.$x\in\mathbb{R}$
B.$x\in\bigg(\frac{[dc2]}{[b]},\frac{[dc1]}{[b]}\bigg)$
C.$x\in\emptyset$
D.$x\in\bigg(-\infty,\frac{[dc1]}{[b]}\bigg)$
E.$x\in\bigg([odp1],[odp2]\bigg)$
F.$x\in\bigg[\frac{[dc2]}{[b]},\frac{[dc1]}{[b]}\bigg]$
G.$x\in\bigg(\frac{[dc2]}{[b]},[dc1]\bigg)$
H.$x\in\bigg[[odp1],[odp2]\bigg]$
I.$x\in\bigg[\frac{[dc2]}{[b]},\infty\bigg)$
\testStop
\kluczStart
B
\kluczStop



\end{document}