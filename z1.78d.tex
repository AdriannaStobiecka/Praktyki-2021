\documentclass[12pt, a4paper]{article}
\usepackage[utf8]{inputenc}
\usepackage{polski}

\usepackage{amsthm}  %pakiet do tworzenia twierdzeń itp.
\usepackage{amsmath} %pakiet do niektórych symboli matematycznych
\usepackage{amssymb} %pakiet do symboli mat., np. \nsubseteq
\usepackage{amsfonts}
\usepackage{graphicx} %obsługa plików graficznych z rozszerzeniem png, jpg
\theoremstyle{definition} %styl dla definicji
\newtheorem{zad}{} 
\title{Multizestaw zadań}
\author{Robert Fidytek}
%\date{\today}
\date{}
\newcounter{liczniksekcji}
\newcommand{\kategoria}[1]{\section{#1}} %olreślamy nazwę kateforii zadań
\newcommand{\zadStart}[1]{\begin{zad}#1\newline} %oznaczenie początku zadania
\newcommand{\zadStop}{\end{zad}}   %oznaczenie końca zadania
%Makra opcjonarne (nie muszą występować):
\newcommand{\rozwStart}[2]{\noindent \textbf{Rozwiązanie (autor #1 , recenzent #2): }\newline} %oznaczenie początku rozwiązania, opcjonarnie można wprowadzić informację o autorze rozwiązania zadania i recenzencie poprawności wykonania rozwiązania zadania
\newcommand{\rozwStop}{\newline}                                            %oznaczenie końca rozwiązania
\newcommand{\odpStart}{\noindent \textbf{Odpowiedź:}\newline}    %oznaczenie początku odpowiedzi końcowej (wypisanie wyniku)
\newcommand{\odpStop}{\newline}                                             %oznaczenie końca odpowiedzi końcowej (wypisanie wyniku)
\newcommand{\testStart}{\noindent \textbf{Test:}\newline} %ewentualne możliwe opcje odpowiedzi testowej: A. ? B. ? C. ? D. ? itd.
\newcommand{\testStop}{\newline} %koniec wprowadzania odpowiedzi testowych
\newcommand{\kluczStart}{\noindent \textbf{Test poprawna odpowiedź:}\newline} %klucz, poprawna odpowiedź pytania testowego (jedna literka): A lub B lub C lub D itd.
\newcommand{\kluczStop}{\newline} %koniec poprawnej odpowiedzi pytania testowego 
\newcommand{\wstawGrafike}[2]{\begin{figure}[h] \includegraphics[scale=#2] {#1} \end{figure}} %gdyby była potrzeba wstawienia obrazka, parametry: nazwa pliku, skala (jak nie wiesz co wpisać, to wpisz 1)

\begin{document}
\maketitle


\kategoria{Wikieł/Z1.78d}
\zadStart{Zadanie z Wikieł Z 1.78 d) moja wersja nr [nrWersji]}
%[a]:[2,3,4,5,6,7,8,9]
%[b]:[2,3,4,5,6,7,8,9]
%[b2]=[b]**2
%[2b]=2*[b]
%[b2a]=[b2]-[a]
%[wd]=math.gcd([b2a],[a])
%[licznik]=int([b2a]/[wd])
%[mianownik]=int([2b]/[wd])
%[licznik2]=[licznik]**2
%[mianownik2]=[mianownik]**2
%[wd]!=[mianownik] and [b2a]!=0 and [2b]!=[b2a] and math.gcd([licznik2],[mianownik2])==1 and [licznik2]/[mianownik2]>=0 and [licznik2]/[mianownik2]<=[b2]
Rozwiązać równanie
$$\sqrt{x+[a]}+\sqrt{x}=[b].$$
\zadStop
\rozwStart{Adrianna Stobiecka}{}
Zakładamy, że $x+[a]\geq0$ i $x\geq0$, czyli $x\in[0,\infty)$. Powyższe równanie zapiszemy w inny sposób.
$$\sqrt{x+[a]}=[b]-\sqrt{x}$$
Lewa strona tego równania jest nieujemna, żeby zatem równanie nie było sprzeczne, musimy dodatkowo założyć $[b]-\sqrt{x}\geq0$. Stąd otrzymujemy założenie: $x\in[0,[b2]]$.
\\Wiemy, że dla $a\geq0$, $b\geq0$ oraz $n\in\mathbb{N}$ zachodzi własność
$$a=b\Leftrightarrow a^n=b^n.$$ 
Korzystając z tej własności otrzymujemy:
$$\sqrt{x+[a]}=[b]-\sqrt{x}\Leftrightarrow\big(\sqrt{x+[a]}\big)^2=([b]-\sqrt{x})^2\Leftrightarrow x+[a]=[b2]-[2b]\sqrt{x}+x$$
$$\Leftrightarrow[b2]-[a]+x-x=[2b]\sqrt{x}\Leftrightarrow[b2a]=[2b]\sqrt{x}\Leftrightarrow\sqrt{x}=\frac{[licznik]}{[mianownik]}$$
$$\Leftrightarrow x=\bigg(\frac{[licznik]}{[mianownik]}\bigg)^2\Leftrightarrow x=\frac{[licznik2]}{[mianownik2]}$$
Uwzględniając założenie $x\in[0,[b2]]$, otrzymujemy, że rozwiązaniem równania jest $x=\frac{[licznik2]}{[mianownik2]}$.
\rozwStop
\odpStart
$x=\frac{[licznik2]}{[mianownik2]}$
\odpStop
\testStart
A.$\frac{[licznik]}{[mianownik]}$
B.$x=\frac{[mianownik2]}{[licznik2]}$
C.$-1$
D.$x=\frac{[licznik2]}{[mianownik2]}$
E.$x=-\frac{[licznik2]}{[mianownik2]}$
F.$x\in\emptyset$
G.$-\frac{[licznik]}{[mianownik]}$
H.$x=-\frac{[mianownik2]}{[licznik2]}$
I.$0$
\testStop
\kluczStart
D
\kluczStop



\end{document}
