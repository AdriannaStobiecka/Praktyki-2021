\documentclass[12pt, a4paper]{article}
\usepackage[utf8]{inputenc}
\usepackage{polski}

\usepackage{amsthm}  %pakiet do tworzenia twierdzeń itp.
\usepackage{amsmath} %pakiet do niektórych symboli matematycznych
\usepackage{amssymb} %pakiet do symboli mat., np. \nsubseteq
\usepackage{amsfonts}
\usepackage{graphicx} %obsługa plików graficznych z rozszerzeniem png, jpg
\theoremstyle{definition} %styl dla definicji
\newtheorem{zad}{} 
\title{Multizestaw zadań}
\author{Robert Fidytek}
%\date{\today}
\date{}
\newcounter{liczniksekcji}
\newcommand{\kategoria}[1]{\section{#1}} %olreślamy nazwę kateforii zadań
\newcommand{\zadStart}[1]{\begin{zad}#1\newline} %oznaczenie początku zadania
\newcommand{\zadStop}{\end{zad}}   %oznaczenie końca zadania
%Makra opcjonarne (nie muszą występować):
\newcommand{\rozwStart}[2]{\noindent \textbf{Rozwiązanie (autor #1 , recenzent #2): }\newline} %oznaczenie początku rozwiązania, opcjonarnie można wprowadzić informację o autorze rozwiązania zadania i recenzencie poprawności wykonania rozwiązania zadania
\newcommand{\rozwStop}{\newline}                                            %oznaczenie końca rozwiązania
\newcommand{\odpStart}{\noindent \textbf{Odpowiedź:}\newline}    %oznaczenie początku odpowiedzi końcowej (wypisanie wyniku)
\newcommand{\odpStop}{\newline}                                             %oznaczenie końca odpowiedzi końcowej (wypisanie wyniku)
\newcommand{\testStart}{\noindent \textbf{Test:}\newline} %ewentualne możliwe opcje odpowiedzi testowej: A. ? B. ? C. ? D. ? itd.
\newcommand{\testStop}{\newline} %koniec wprowadzania odpowiedzi testowych
\newcommand{\kluczStart}{\noindent \textbf{Test poprawna odpowiedź:}\newline} %klucz, poprawna odpowiedź pytania testowego (jedna literka): A lub B lub C lub D itd.
\newcommand{\kluczStop}{\newline} %koniec poprawnej odpowiedzi pytania testowego 
\newcommand{\wstawGrafike}[2]{\begin{figure}[h] \includegraphics[scale=#2] {#1} \end{figure}} %gdyby była potrzeba wstawienia obrazka, parametry: nazwa pliku, skala (jak nie wiesz co wpisać, to wpisz 1)

\begin{document}
\maketitle


\kategoria{Wikieł/Z4.16b}
\zadStart{Zadanie z Wikieł Z 4.16 b) moja wersja nr [nrWersji]}
%[a]:[1,2,3,4,5,6,7,8]
%[b]:[2,3,4,5,6,7,8]
%[d]:[2,3,4,5,6,7,8]
%[aaa]=[a]*[a]*[a]
%[e]=[aaa]-[b]
%[aaa]!=[b]
%[g]=[a]+[d]
Zbadać, czy istnieje $\lim_{x\rightarrow x_{0}}f(x)$. Jeśli tak, to obliczyć tę granicę. $$x_{0}=[a],f(x)= \left\{ \begin{array}{ll}
x^3-[b] & \textrm{dla $x<[a]$}\\
\frac{[a]-x}{x+[d]} & \textrm{dla $x>[a]$}
\end{array} \right.$$.
\zadStop
\rozwStart{Aleksandra Pasińska}{}
$$\lim_{x\rightarrow [a]^-}x^3-[b]=[e]$$
$$\lim_{x\rightarrow [a]^+}\frac{[a]-x}{x+[d]}=\frac{0}{[g]}=0$$
$$\lim_{x\rightarrow [a]^-}x^3-[b] \neq \lim_{x\rightarrow [a]^+}\frac{[a]-x}{x+[d]}$$
\rozwStop
\odpStart
Granica nie istnieje.\\
\odpStop
\testStart
A.Granica nie istnieje
B.$\infty$
C.$-\infty$
D.$1$
E.$9$
F.$e$
G.$4$
H.$7$
I.$-7$
\testStop
\kluczStart
A
\kluczStop



\end{document}