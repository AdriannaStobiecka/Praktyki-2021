\documentclass[12pt, a4paper]{article}
\usepackage[utf8]{inputenc}
\usepackage{polski}

\usepackage{amsthm}  %pakiet do tworzenia twierdzeń itp.
\usepackage{amsmath} %pakiet do niektórych symboli matematycznych
\usepackage{amssymb} %pakiet do symboli mat., np. \nsubseteq
\usepackage{amsfonts}
\usepackage{graphicx} %obsługa plików graficznych z rozszerzeniem png, jpg
\theoremstyle{definition} %styl dla definicji
\newtheorem{zad}{} 
\title{Multizestaw zadań}
\author{Robert Fidytek}
%\date{\today}
\date{}
\newcounter{liczniksekcji}
\newcommand{\kategoria}[1]{\section{#1}} %olreślamy nazwę kateforii zadań
\newcommand{\zadStart}[1]{\begin{zad}#1\newline} %oznaczenie początku zadania
\newcommand{\zadStop}{\end{zad}}   %oznaczenie końca zadania
%Makra opcjonarne (nie muszą występować):
\newcommand{\rozwStart}[2]{\noindent \textbf{Rozwiązanie (autor #1 , recenzent #2): }\newline} %oznaczenie początku rozwiązania, opcjonarnie można wprowadzić informację o autorze rozwiązania zadania i recenzencie poprawności wykonania rozwiązania zadania
\newcommand{\rozwStop}{\newline}                                            %oznaczenie końca rozwiązania
\newcommand{\odpStart}{\noindent \textbf{Odpowiedź:}\newline}    %oznaczenie początku odpowiedzi końcowej (wypisanie wyniku)
\newcommand{\odpStop}{\newline}                                             %oznaczenie końca odpowiedzi końcowej (wypisanie wyniku)
\newcommand{\testStart}{\noindent \textbf{Test:}\newline} %ewentualne możliwe opcje odpowiedzi testowej: A. ? B. ? C. ? D. ? itd.
\newcommand{\testStop}{\newline} %koniec wprowadzania odpowiedzi testowych
\newcommand{\kluczStart}{\noindent \textbf{Test poprawna odpowiedź:}\newline} %klucz, poprawna odpowiedź pytania testowego (jedna literka): A lub B lub C lub D itd.
\newcommand{\kluczStop}{\newline} %koniec poprawnej odpowiedzi pytania testowego 
\newcommand{\wstawGrafike}[2]{\begin{figure}[h] \includegraphics[scale=#2] {#1} \end{figure}} %gdyby była potrzeba wstawienia obrazka, parametry: nazwa pliku, skala (jak nie wiesz co wpisać, to wpisz 1)

\begin{document}
\maketitle


\kategoria{Wikieł/Z1.70j}
\zadStart{Zadanie z Wikieł Z 1.70 j) moja wersja nr [nrWersji]}
%[a]:[2,3,4]
%[b]:[2,3,4]
%[c]:[2,3,4]
%[d]:[2,3,4]
%[e]:[2,3,4]
%[a]=random.randint(2,4)
%[b]=random.randint(2,4)
%[c]=random.randint(2,4)
%[d]=random.randint(2,4)
%[e]=random.randint(2,4)
%[f]=([b]-[a])
%[g]=[a]*[b]
%[ef]=[e]*[f]
%[dg]=[d]*[g]
%[x3]=[d]+[c]*[f]
%[x2]=[e]+[d]*[f]-[g]*[c]
%[x]=[ef]-[dg]
%[ge]=[g]*[e]
%[b]>[a]
Rozwiązać nierówność: $\frac{[c]x^4+[x3]x^3[x2]x^2[x]x-[ge]}{x^2+x+1}\leq0$
\zadStop
\rozwStart{Pascal Nawrocki}{Jakub Ulrych}
Zauważmy, że dziedzina $x\in\mathbb R$, ponieważ mianownik nigdy się nie wyzeruje. Korzystamy z twierdzenia o wymiernych pierwiastkach wielomianu o współczynnikach całkowitych, aby rozłożyć sobie licznik.
$$\frac{[c]x^4+[x3]x^3[x2]x^2[x]x-[ge]}{x^2+x+1}=\frac{(x-[a])(x+[b])([c]x^2+[d]x+[e])}{x^2+x+1}$$
Zauważmy, że na wpływ na znak ułamka ma tylko licznik, dlatego przy rozpatrywaniu nierówności pomijamy sobie mianownik.
$$(x-[a])(x+[b])([c]x^2+[d]x+[e])\leq0$$
Możemy teraz zauważyć, że znajdująca się funkcja kwadratowa ma deltę$<0$, co oznacza, że nie ma miejsc zerowych, więc rozpatrujemy tylko przedziały $(x-[a])(x+[b])\leq0$.
Stąd otrzymujemy rozwiązanie:
$$x\in[-[b],[a]]$$
\rozwStop
\odpStart
$x\in[-[b],[a]]$
\odpStop
\testStart
A. $x\in[-[b],[a]]$
B.$x\in([b],\infty)$
C.$x\in\emptyset$
D.$x\in(-\infty,-[a])$
\testStop
\kluczStart
A
\kluczStop
\end{document}