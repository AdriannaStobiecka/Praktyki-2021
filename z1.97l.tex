\documentclass[12pt, a4paper]{article}
\usepackage[utf8]{inputenc}
\usepackage{polski}
\usepackage{amsthm}  %pakiet do tworzenia twierdzeń itp.
\usepackage{amsmath} %pakiet do niektórych symboli matematycznych
\usepackage{amssymb} %pakiet do symboli mat., np. \nsubseteq
\usepackage{amsfonts}
\usepackage{graphicx} %obsługa plików graficznych z rozszerzeniem png, jpg
\theoremstyle{definition} %styl dla definicji
\newtheorem{zad}{} 
\title{Multizestaw zadań}
\author{Robert Fidytek}
%\date{\today}
\date{}
\newcounter{liczniksekcji}
\newcommand{\kategoria}[1]{\section{#1}} %olreślamy nazwę kateforii zadań
\newcommand{\zadStart}[1]{\begin{zad}#1\newline} %oznaczenie początku zadania
\newcommand{\zadStop}{\end{zad}}   %oznaczenie końca zadania
%Makra opcjonarne (nie muszą występować):
\newcommand{\rozwStart}[2]{\noindent \textbf{Rozwiązanie (autor #1 , recenzent #2): }\newline} %oznaczenie początku rozwiązania, opcjonarnie można wprowadzić informację o autorze rozwiązania zadania i recenzencie poprawności wykonania rozwiązania zadania
\newcommand{\rozwStop}{\newline}                                            %oznaczenie końca rozwiązania
\newcommand{\odpStart}{\noindent \textbf{Odpowiedź:}\newline}    %oznaczenie początku odpowiedzi końcowej (wypisanie wyniku)
\newcommand{\odpStop}{\newline}                                             %oznaczenie końca odpowiedzi końcowej (wypisanie wyniku)
\newcommand{\testStart}{\noindent \textbf{Test:}\newline} %ewentualne możliwe opcje odpowiedzi testowej: A. ? B. ? C. ? D. ? itd.
\newcommand{\testStop}{\newline} %koniec wprowadzania odpowiedzi testowych
\newcommand{\kluczStart}{\noindent \textbf{Test poprawna odpowiedź:}\newline} %klucz, poprawna odpowiedź pytania testowego (jedna literka): A lub B lub C lub D itd.
\newcommand{\kluczStop}{\newline} %koniec poprawnej odpowiedzi pytania testowego 
\newcommand{\wstawGrafike}[2]{\begin{figure}[h] \includegraphics[scale=#2] {#1} \end{figure}} %gdyby była potrzeba wstawienia obrazka, parametry: nazwa pliku, skala (jak nie wiesz co wpisać, to wpisz 1)

\begin{document}
\maketitle


\kategoria{Wikieł/Z1.97l}
\zadStart{Zadanie z Wikieł Z1.97 l) moja wersja nr [nrWersji]}
%[z]=2
%[y]:[1,2,3]
%[w]=8*[z]
%[o]=[y]-[w]
%[o2]=[o]*(-1)
Rozwiąż nierówność.\\
Podana nierówno $ \log_\frac{1}{2} \frac{[z]x+1}{[y]x+2} > 3$\\
\zadStop
\rozwStart{Martyna Czarnobaj}{}
Założenia:\\
\begin{center}
	$ \log_\frac{1}{2} \frac{[z]x+1}{[y]x+2} > 3$\\
\end{center}
Rozwiązanie:\\
Zamieniamy podstawe logarytmu $ \frac{1}{2} $ na $ 2 $ oraz $ 3 = \log_2 8 $
co podstawiamy.\\
\begin{center}
	$ \log_2 \frac{[y]x+2}{[z]x+1} > \log_2 8 \Rightarrow \frac{[y]x+2}{[z]x+1} < 8 $\\
	$ \frac{[y]x+2-[w]x-8}{[z]x+1}<0 \Rightarrow \frac{[o]x-6}{[z]x+1} < 0 |*(-1)$\\
	$ \frac{[o2]x+6}{[z]x+1} > 0 $\\
	$ ([o2]x + 6 > 0) $ i $ ([z]x + 1 > 0)$ lub $([o2]x + 6 < 0) $ i $ [z]x + 1 < 0 $\\	
\end{center}
Wyliczamy x:\\
\newline
$ [o2]x + 6 > 0 $\\
$ [o2]x > -6 |:[o2] $\\
$ x > -\frac{6}{[o2]} $\\
\newline
$ [z]x + 1 > 0 $\\
$ [z]x > -1 |:[z]$\\
$ x > -\frac{1}{[z]} $\\
\newline
Wiemy, że $ \frac{1}{[z]} > \frac{6}{[o2]} $, więc przyjmujemy $ x > -\frac{6}{[o2]} $\\
\newline
$ [o2]x + 6 < 0 $\\
$ [o2]x < -6 |:[o2] $\\
$ x < -\frac{6}{[o2]} $\\
\newline
$ [z]x + 1 > 0 $\\
$ [z]x > -1 |:[z]$\\
$ x > -\frac{1}{[z]} $\\
\newline
Wiemy, że $ \frac{1}{[z]} > \frac{6}{[o2]} $, więc przyjmujemy $ x < -\frac{1}{[z]} $\\
Wynik dla naszego $ x \in (-\frac{1}{[z]},-\frac{6}{[o2]}) $\\
Koniec rozwiązania.\\
\rozwStop
\odpStart
$ x \in (-\frac{1}{[z]},-\frac{6}{[o2]}) $\\
\odpStop
\testStart
A.$ x \in (-\frac{1}{[z]},-\frac{6}{[o2]}) $\\
B.$ x \in (0,-\frac{6}{[o2]}) $\\
C.$ x \in (-\frac{1}{[z]},0) $\\
\testStop
\kluczStart
A
\kluczStop



\end{document}