\documentclass[12pt, a4paper]{article}
\usepackage[utf8]{inputenc}
\usepackage{polski}

\usepackage{amsthm}  %pakiet do tworzenia twierdzeń itp.
\usepackage{amsmath} %pakiet do niektórych symboli matematycznych
\usepackage{amssymb} %pakiet do symboli mat., np. \nsubseteq
\usepackage{amsfonts}
\usepackage{graphicx} %obsługa plików graficznych z rozszerzeniem png, jpg
\theoremstyle{definition} %styl dla definicji
\newtheorem{zad}{} 
\title{Multizestaw zadań}
\author{Robert Fidytek}
%\date{\today}
\date{}
\newcounter{liczniksekcji}
\newcommand{\kategoria}[1]{\section{#1}} %olreślamy nazwę kateforii zadań
\newcommand{\zadStart}[1]{\begin{zad}#1\newline} %oznaczenie początku zadania
\newcommand{\zadStop}{\end{zad}}   %oznaczenie końca zadania
%Makra opcjonarne (nie muszą występować):
\newcommand{\rozwStart}[2]{\noindent \textbf{Rozwiązanie (autor #1 , recenzent #2): }\newline} %oznaczenie początku rozwiązania, opcjonarnie można wprowadzić informację o autorze rozwiązania zadania i recenzencie poprawności wykonania rozwiązania zadania
\newcommand{\rozwStop}{\newline}                                            %oznaczenie końca rozwiązania
\newcommand{\odpStart}{\noindent \textbf{Odpowiedź:}\newline}    %oznaczenie początku odpowiedzi końcowej (wypisanie wyniku)
\newcommand{\odpStop}{\newline}                                             %oznaczenie końca odpowiedzi końcowej (wypisanie wyniku)
\newcommand{\testStart}{\noindent \textbf{Test:}\newline} %ewentualne możliwe opcje odpowiedzi testowej: A. ? B. ? C. ? D. ? itd.
\newcommand{\testStop}{\newline} %koniec wprowadzania odpowiedzi testowych
\newcommand{\kluczStart}{\noindent \textbf{Test poprawna odpowiedź:}\newline} %klucz, poprawna odpowiedź pytania testowego (jedna literka): A lub B lub C lub D itd.
\newcommand{\kluczStop}{\newline} %koniec poprawnej odpowiedzi pytania testowego 
\newcommand{\wstawGrafike}[2]{\begin{figure}[h] \includegraphics[scale=#2] {#1} \end{figure}} %gdyby była potrzeba wstawienia obrazka, parametry: nazwa pliku, skala (jak nie wiesz co wpisać, to wpisz 1)

\begin{document}
\maketitle



\kategoria{Wikieł/Z1.15c}
\zadStart{Zadanie z Wikieł Z 1.15 c) moja wersja nr [nrWersji]}
%[a]:[2,3,4]
%[b]:[2,3,4]
%[c]:[2,3,4]
%[a]=random.randint(2,5)
%[b]=random.randint(2,5)
%[c]=random.randint(2,5)
%[ac]=round([a]/[c],2)
%[acb]=[ac]+[b]
%[macb]=-[ac]+[b]
%[b]>[ac] and [acb]>[b] and [macb]<[b]
Rozwiązać nierówność $\big|\frac{[a]}{x-[b]}\big|<[c]$
\zadStop
\rozwStart{Pascal Nawrocki}{Jakub Ulrych}
Wyznaczamy dziedzinę: $x\in\mathbb{R}\symbol{92}\{[b]\}$.
Następnie korzystamy z własności wartości bezwględnej:
$$\bigg|\frac{[a]}{x-[b]}\bigg|=\frac{[a]}{|x-[b]|}$$
I teraz rozwiązujemy:
$$\frac{[a]}{|x-[b]|}<[c]$$
$$\frac{[a]}{|x-[b]|}-[c]<0$$
$$\frac{[a]-[c]|x-[b]|}{|x-[b]|}<0$$
$$([a]-[c]|x-[b]|)|x-[b]|<0$$
Zauważmy, że ta nierówność będzie spełniona wtedy i tylko wtedy gdy \mbox{$([a]-[c]|x-[b]|)<0$}, ponieważ $|x-[b]|$ zawsze będzie dodatnia (pamiętając o dziedzinie). Stąd wystarczy, że rozwiążemy: $[a]-[c]|x-[b]|<0$.
$$[a]-[c]|x-[b]|<0$$
$$-[c]|x-[b]|<-[a]$$
$$|x-[b]|>[ac]$$
Stąd możemy rozważyć dwa przypadki:
$$x-[b]>[ac] \vee x-[b]<-[ac]$$
Stąd:
$$x>[acb] \vee x<[macb]$$
Czyli:
$$x\in(-\infty,[macb])\cup([acb],\infty)$$
\odpStart
$x\in(-\infty,[macb])\cup([acb],\infty)$
\odpStop
\testStart
A.$x\in(-\infty,[macb])\cup([acb],\infty)$
B.$x\in \emptyset$
C.$\infty$
D. $x\in (-\infty,-[a])$
\testStop
\kluczStart
A
\kluczStop


\end{document}