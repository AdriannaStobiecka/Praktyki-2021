\documentclass[12pt, a4paper]{article}
\usepackage[utf8]{inputenc}
\usepackage{polski}

\usepackage{amsthm}  %pakiet do tworzenia twierdzeń itp.
\usepackage{amsmath} %pakiet do niektórych symboli matematycznych
\usepackage{amssymb} %pakiet do symboli mat., np. \nsubseteq
\usepackage{amsfonts}
\usepackage{graphicx} %obsługa plików graficznych z rozszerzeniem png, jpg
\theoremstyle{definition} %styl dla definicji
\newtheorem{zad}{} 
\title{Multizestaw zadań}
\author{Robert Fidytek}
%\date{\today}
\date{}
\newcounter{liczniksekcji}
\newcommand{\kategoria}[1]{\section{#1}} %olreślamy nazwę kateforii zadań
\newcommand{\zadStart}[1]{\begin{zad}#1\newline} %oznaczenie początku zadania
\newcommand{\zadStop}{\end{zad}}   %oznaczenie końca zadania
%Makra opcjonarne (nie muszą występować):
\newcommand{\rozwStart}[2]{\noindent \textbf{Rozwiązanie (autor #1 , recenzent #2): }\newline} %oznaczenie początku rozwiązania, opcjonarnie można wprowadzić informację o autorze rozwiązania zadania i recenzencie poprawności wykonania rozwiązania zadania
\newcommand{\rozwStop}{\newline}                                            %oznaczenie końca rozwiązania
\newcommand{\odpStart}{\noindent \textbf{Odpowiedź:}\newline}    %oznaczenie początku odpowiedzi końcowej (wypisanie wyniku)
\newcommand{\odpStop}{\newline}                                             %oznaczenie końca odpowiedzi końcowej (wypisanie wyniku)
\newcommand{\testStart}{\noindent \textbf{Test:}\newline} %ewentualne możliwe opcje odpowiedzi testowej: A. ? B. ? C. ? D. ? itd.
\newcommand{\testStop}{\newline} %koniec wprowadzania odpowiedzi testowych
\newcommand{\kluczStart}{\noindent \textbf{Test poprawna odpowiedź:}\newline} %klucz, poprawna odpowiedź pytania testowego (jedna literka): A lub B lub C lub D itd.
\newcommand{\kluczStop}{\newline} %koniec poprawnej odpowiedzi pytania testowego 
\newcommand{\wstawGrafike}[2]{\begin{figure}[h] \includegraphics[scale=#2] {#1} \end{figure}} %gdyby była potrzeba wstawienia obrazka, parametry: nazwa pliku, skala (jak nie wiesz co wpisać, to wpisz 1)

\begin{document}
\maketitle


\kategoria{Wikieł/Z3.12e}
\zadStart{Zadanie z Wikieł Z 3.12 e) moja wersja nr [nrWersji]}
%[f]:[1,2,3,4,5,9,10,11,12,13,14,15,16,17,18,19,20,21]
%[p1]=random.randint(2001,2090)
%[p3]:[3,4,5,6]
%[p4]=[p3]-1
%[p5]=[p3]*[p1]
%[p2]=random.randint([p3]*[p1]+2, [p3]*[p1]+1000)
%[p6]=[p2]-[p5]
%[l1]=random.randint(2,60)
%[l2]=random.randint(-20,-1)
%[m1]=random.randint(1, 30)
Obliczyć granicę ciągu $a_n=\frac{{(n^{[p3]} + [l1]n [l2])}^{[p1]}}{{(n+[m1])}^{[p2]}}$.
\zadStop
\rozwStart{Barbara Bączek}{}
$$\lim_{n \rightarrow \infty} a_n= \lim_{n \rightarrow \infty} \frac{{(n^{[p3]} + [l1]n [l2])}^{[p1]}}{{(n+[m1])}^{[p2]}}= \lim_{n \rightarrow \infty} \frac{{(n^{[p3]}(1+\frac{[l1]}{n^{[p4]}}+ \frac{-1}{n^{[p3]}}))}^{[p1]}}{{(n(1+ \frac{[m1]}{n}))}^{[p2]}} =$$
$$\lim_{n \rightarrow \infty} \frac{n^{[p5]}{(1+\frac{[l1]}{n^{[p4]}}+ \frac{-1}{n^{[p3]}})}^{[p1]}}{n^{[p2]}{(1+ \frac{[m1]}{n})}^{[p2]}}=\lim_{n \rightarrow \infty} \frac{{(1+\frac{[l1]}{n^{[p4]}}+ \frac{-1}{n^{[p3]}})}^{[p1]}}{n^{[p6]}{(1+ \frac{[m1]}{n})}^{[p2]}}= \Big{[}\frac{1}{\infty}\Big{]}=0$$ 
\rozwStop
\odpStart
$0$
\odpStop
\testStart
A.$\infty$
B.$[p3]$
C.$-\infty$
D.$0$
E.$1$
G.$[p6]$
H.$[l2]$
\testStop
\kluczStart
D
\kluczStop



\end{document}