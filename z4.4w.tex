\documentclass[12pt, a4paper]{article}
\usepackage[utf8]{inputenc}
\usepackage{polski}

\usepackage{amsthm}  %pakiet do tworzenia twierdzeń itp.
\usepackage{amsmath} %pakiet do niektórych symboli matematycznych
\usepackage{amssymb} %pakiet do symboli mat., np. \nsubseteq
\usepackage{amsfonts}
\usepackage{graphicx} %obsługa plików graficznych z rozszerzeniem png, jpg
\theoremstyle{definition} %styl dla definicji
\newtheorem{zad}{} 
\title{Multizestaw zadań}
\author{Robert Fidytek}
%\date{\today}
\date{}
\newcounter{liczniksekcji}
\newcommand{\kategoria}[1]{\section{#1}} %olreślamy nazwę kateforii zadań
\newcommand{\zadStart}[1]{\begin{zad}#1\newline} %oznaczenie początku zadania
\newcommand{\zadStop}{\end{zad}}   %oznaczenie końca zadania
%Makra opcjonarne (nie muszą występować):
\newcommand{\rozwStart}[2]{\noindent \textbf{Rozwiązanie (autor #1 , recenzent #2): }\newline} %oznaczenie początku rozwiązania, opcjonarnie można wprowadzić informację o autorze rozwiązania zadania i recenzencie poprawności wykonania rozwiązania zadania
\newcommand{\rozwStop}{\newline}                                            %oznaczenie końca rozwiązania
\newcommand{\odpStart}{\noindent \textbf{Odpowiedź:}\newline}    %oznaczenie początku odpowiedzi końcowej (wypisanie wyniku)
\newcommand{\odpStop}{\newline}                                             %oznaczenie końca odpowiedzi końcowej (wypisanie wyniku)
\newcommand{\testStart}{\noindent \textbf{Test:}\newline} %ewentualne możliwe opcje odpowiedzi testowej: A. ? B. ? C. ? D. ? itd.
\newcommand{\testStop}{\newline} %koniec wprowadzania odpowiedzi testowych
\newcommand{\kluczStart}{\noindent \textbf{Test poprawna odpowiedź:}\newline} %klucz, poprawna odpowiedź pytania testowego (jedna literka): A lub B lub C lub D itd.
\newcommand{\kluczStop}{\newline} %koniec poprawnej odpowiedzi pytania testowego 
\newcommand{\wstawGrafike}[2]{\begin{figure}[h] \includegraphics[scale=#2] {#1} \end{figure}} %gdyby była potrzeba wstawienia obrazka, parametry: nazwa pliku, skala (jak nie wiesz co wpisać, to wpisz 1)

\begin{document}
\maketitle


\kategoria{Wikieł/Z4.4w}
\zadStart{Zadanie z Wikieł Z 4.4w) moja wersja nr [nrWersji]}
%[p1]:[2,3,4,5,6,7,8,9,10]
Obliczyć granicę funkcji $\lim_{x \to \frac{\Pi}{3}} \frac{[p1]\sin({x-\frac{\Pi}{3}})}{1-2\cos{x}}$
\zadStop
\rozwStart{Jakub Janik}{}
$$\lim_{x \to \frac{\Pi}{3}} \frac{[p1]\sin({x-\frac{\Pi}{3}})}{1-2\cos{x}}=\lim_{x \to \frac{\Pi}{3}} \frac{[p1](\sin{x}-\sqrt{3}\cos{x})}{2(1-2\cos{x})}=$$
$$\lim_{x \to \frac{\Pi}{3}} \frac{[p1](\sin{x}-\sqrt{3}\cos{x})(\sin{x}+\sqrt{3}\cos{x})}{2(1-2\cos{x})(\sin{x}+\sqrt{3}\cos{x})}=$$
$$\lim_{x \to \frac{\Pi}{3}} \frac{[p1](\sin^2{x}-3\cos^2{x})}{2(1-2\cos{x})(\sin{x}+\sqrt{3}\cos{x})}=$$
$$\lim_{x \to \frac{\Pi}{3}} \frac{[p1](1-4\cos^2{x})}{2(1-2\cos{x})(\sin{x}+\sqrt{3}\cos{x})}=$$
$$\lim_{x \to \frac{\Pi}{3}} \frac{[p1](1-2\cos{x})(1+2\cos{x})}{2(1-2\cos{x})(\sin{x}+\sqrt{3}\cos{x})}=$$
$$\lim_{x \to \frac{\Pi}{3}} \frac{[p1](1+2\cos{x})}{2(\sin{x}+\sqrt{3}\cos{x})}=\frac{[p1]}{\sqrt{3}}$$
\rozwStop
\odpStart
$\frac{[p1]}{\sqrt{3}}$
\odpStop
\testStart
A.$\frac{[p1]}{\sqrt{3}}$
B.$0$
C.$-\frac{[p1]}{\sqrt{3}}$
D.$\infty$
\testStop
\kluczStart
A
\kluczStop



\end{document}