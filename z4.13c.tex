\documentclass[12pt, a4paper]{article}
\usepackage[utf8]{inputenc}
\usepackage{polski}

\usepackage{amsthm}  %pakiet do tworzenia twierdzeń itp.
\usepackage{amsmath} %pakiet do niektórych symboli matematycznych
\usepackage{amssymb} %pakiet do symboli mat., np. \nsubseteq
\usepackage{amsfonts}
\usepackage{graphicx} %obsługa plików graficznych z rozszerzeniem png, jpg
\theoremstyle{definition} %styl dla definicji
\newtheorem{zad}{} 
\title{Multizestaw zadań}
\author{Robert Fidytek}
%\date{\today}
\date{}
\newcounter{liczniksekcji}
\newcommand{\kategoria}[1]{\section{#1}} %olreślamy nazwę kateforii zadań
\newcommand{\zadStart}[1]{\begin{zad}#1\newline} %oznaczenie początku zadania
\newcommand{\zadStop}{\end{zad}}   %oznaczenie końca zadania
%Makra opcjonarne (nie muszą występować):
\newcommand{\rozwStart}[2]{\noindent \textbf{Rozwiązanie (autor #1 , recenzent #2): }\newline} %oznaczenie początku rozwiązania, opcjonarnie można wprowadzić informację o autorze rozwiązania zadania i recenzencie poprawności wykonania rozwiązania zadania
\newcommand{\rozwStop}{\newline}                                            %oznaczenie końca rozwiązania
\newcommand{\odpStart}{\noindent \textbf{Odpowiedź:}\newline}    %oznaczenie początku odpowiedzi końcowej (wypisanie wyniku)
\newcommand{\odpStop}{\newline}                                             %oznaczenie końca odpowiedzi końcowej (wypisanie wyniku)
\newcommand{\testStart}{\noindent \textbf{Test:}\newline} %ewentualne możliwe opcje odpowiedzi testowej: A. ? B. ? C. ? D. ? itd.
\newcommand{\testStop}{\newline} %koniec wprowadzania odpowiedzi testowych
\newcommand{\kluczStart}{\noindent \textbf{Test poprawna odpowiedź:}\newline} %klucz, poprawna odpowiedź pytania testowego (jedna literka): A lub B lub C lub D itd.
\newcommand{\kluczStop}{\newline} %koniec poprawnej odpowiedzi pytania testowego 
\newcommand{\wstawGrafike}[2]{\begin{figure}[h] \includegraphics[scale=#2] {#1} \end{figure}} %gdyby była potrzeba wstawienia obrazka, parametry: nazwa pliku, skala (jak nie wiesz co wpisać, to wpisz 1)

\begin{document}
\maketitle


\kategoria{Wikieł/Z4.13c}
\zadStart{Zadanie z Wikieł Z 4.13 c) moja wersja nr [nrWersji]}
%[a]:[2,3,4,5,6,7]
%[b]:[2,3,4,5,6,7,8,9]
%[c]=random.randint(2,15)
%[d]:[1,2,3,4,5,6,7,8,9]
%[e]:[2,3,4,5,6,7,8]
%[xo]=[e]
%[cxo]=[c]*[xo]
%[l]=[cxo]-[d]
%[m]=[xo]-[e]
%[a]<[b] and [cxo]>[d]
Obliczyć granice jednostronne funkcji $f(x)=\left(\frac{[a]}{[b]}\right)^{\frac{[c]x-[d]}{[e]-x}}$ w punkcie $x_{0}=[xo]$. 
\zadStop
\rozwStart{Justyna Chojecka}{}
Obliczamy granice jednostronne funkcji $f$ w punkcie $x_{0}=[xo]$.
$$\lim\limits_{x\to [xo]^{-}}\left(\frac{[a]}{[b]}\right)^{\frac{[c]x-[d]}{[e]-x}}=\left(\frac{[a]}{[b]}\right)^{\frac{[c]\cdot[xo]-[d]}{[e]-[xo]}}=\left(\frac{[a]}{[b]}\right)^{\frac{[cxo]-[d]}{[e]-[xo]}}=\left(\frac{[a]}{[b]}\right)^{\frac{[l]}{[m]^{-}}}=0$$
$$\lim\limits_{x\to [xo]^{+}}\left(\frac{[a]}{[b]}\right)^{\frac{[c]x-[d]}{[e]-x}}=\left(\frac{[a]}{[b]}\right)^{\frac{[c]\cdot[xo]-[d]}{[e]-[xo]}}=\left(\frac{[a]}{[b]}\right)^{\frac{[cxo]-[d]}{[e]-[xo]}}=\left(\frac{[a]}{[b]}\right)^{\frac{[l]}{[m]^{+}}}=\infty$$
\rozwStop
\odpStart
$\lim\limits_{x\to [xo]^{-}}f(x)=0, \lim\limits_{x\to [xo]^{+}}f(x)=\infty$
\odpStop
\testStart
A.$\lim\limits_{x\to [xo]^{-}}f(x)=0, \lim\limits_{x\to [xo]^{+}}f(x)=\infty$\\
B.$\lim\limits_{x\to [xo]^{-}}f(x)=-\infty, \lim\limits_{x\to [xo]^{+}}f(x)=\infty$\\
C.$\lim\limits_{x\to [xo]^{-}}f(x)=\infty, \lim\limits_{x\to [xo]^{+}}f(x)=-\infty$\\
D.$\lim\limits_{x\to [xo]^{-}}f(x)=\infty, \lim\limits_{x\to [xo]^{+}}f(x)=\infty$\\
E.$\lim\limits_{x\to [xo]^{-}}f(x)=\infty, \lim\limits_{x\to [xo]^{+}}f(x)=0$\\
F.$\lim\limits_{x\to [xo]^{-}}f(x)=0, \lim\limits_{x\to [xo]^{+}}f(x)=0$\\
G.$\lim\limits_{x\to [xo]^{-}}f(x)=0, \lim\limits_{x\to [xo]^{+}}f(x)=-\infty$\\
H.$\lim\limits_{x\to [xo]^{-}}f(x)=-\infty, \lim\limits_{x\to [xo]^{+}}f(x)=0$\\
I.$\lim\limits_{x\to [xo]^{-}}f(x)=[xo], \lim\limits_{x\to [xo]^{+}}f(x)=\infty$
\testStop
\kluczStart
A
\kluczStop



\end{document}
