\documentclass[12pt, a4paper]{article}
\usepackage[utf8]{inputenc}
\usepackage{polski}

\usepackage{amsthm}  %pakiet do tworzenia twierdzeń itp.
\usepackage{amsmath} %pakiet do niektórych symboli matematycznych
\usepackage{amssymb} %pakiet do symboli mat., np. \nsubseteq
\usepackage{amsfonts}
\usepackage{graphicx} %obsługa plików graficznych z rozszerzeniem png, jpg
\theoremstyle{definition} %styl dla definicji
\newtheorem{zad}{} 
\title{Multizestaw zadań}
\author{Robert Fidytek}
%\date{\today}
\date{}
\newcounter{liczniksekcji}
\newcommand{\kategoria}[1]{\section{#1}} %olreślamy nazwę kateforii zadań
\newcommand{\zadStart}[1]{\begin{zad}#1\newline} %oznaczenie początku zadania
\newcommand{\zadStop}{\end{zad}}   %oznaczenie końca zadania
%Makra opcjonarne (nie muszą występować):
\newcommand{\rozwStart}[2]{\noindent \textbf{Rozwiązanie (autor #1 , recenzent #2): }\newline} %oznaczenie początku rozwiązania, opcjonarnie można wprowadzić informację o autorze rozwiązania zadania i recenzencie poprawności wykonania rozwiązania zadania
\newcommand{\rozwStop}{\newline}                                            %oznaczenie końca rozwiązania
\newcommand{\odpStart}{\noindent \textbf{Odpowiedź:}\newline}    %oznaczenie początku odpowiedzi końcowej (wypisanie wyniku)
\newcommand{\odpStop}{\newline}                                             %oznaczenie końca odpowiedzi końcowej (wypisanie wyniku)
\newcommand{\testStart}{\noindent \textbf{Test:}\newline} %ewentualne możliwe opcje odpowiedzi testowej: A. ? B. ? C. ? D. ? itd.
\newcommand{\testStop}{\newline} %koniec wprowadzania odpowiedzi testowych
\newcommand{\kluczStart}{\noindent \textbf{Test poprawna odpowiedź:}\newline} %klucz, poprawna odpowiedź pytania testowego (jedna literka): A lub B lub C lub D itd.
\newcommand{\kluczStop}{\newline} %koniec poprawnej odpowiedzi pytania testowego 
\newcommand{\wstawGrafike}[2]{\begin{figure}[h] \includegraphics[scale=#2] {#1} \end{figure}} %gdyby była potrzeba wstawienia obrazka, parametry: nazwa pliku, skala (jak nie wiesz co wpisać, to wpisz 1)

\begin{document}
\maketitle


\kategoria{Wikieł/Z1.36i}
\zadStart{Zadanie z Wikieł Z 1.36 i) moja wersja nr 1}
%[a]:[2, 3, 4, 5, 6, 7, 8, 9, 10, 11, 12, 13, 14, 15, 16, 17, 18, 19, 20, 21, 22, 23, 24, 25, 26, 27, 28, 29, 30,31,32,33,34,35]
%[b]:[2, 3, 4, 5, 6, 7, 8, 9, 10, 11, 12, 13, 14, 15, 16, 17, 18, 19, 20, 21, 22, 23, 24, 25, 26, 27, 28, 29, 30,31,32,33,34,35]
%[c]:[2, 3, 4, 5, 6, 7, 8, 9, 10, 11, 12, 13, 14, 15, 16, 17, 18, 19, 20, 21, 22, 23, 24, 25, 26, 27, 28, 29, 30,31,32,33,34,35]
%[delta]=[b]*[b]-4*[a]*[c]
%[sdelta]=pow([delta],1/2)
%[sdelta1]=int([sdelta].real)
%[t1]=([b]-[sdelta])/(2*[a])
%[t2]=([b]+[sdelta])/(2*[a])
%[x1]=pow([t1],1/2)
%[x2]=pow([t2],1/2)
%[t11]=int([t1].real)
%[t21]=int([t2].real)
%[x11]=int([x1].real)
%[x21]=int([x2].real)
%[delta]>0 and [sdelta].is_integer()==True and [t1].is_integer()==True and [t2].is_integer()==True and [x1].is_integer()==True and [x2].is_integer()==True
Rozwiązać równanie. $[a]x^{4}-[b]x^{2}+[c]=0$
\zadStop
\rozwStart{Jakub Ulrych}{}
Podstawiamy $t=x^{2}$\\
Dziedzina: $t\in[0,\infty)$
$$[a]t^{2}-[b]t+[c]=0$$
$$\Delta=[b]^{2}-4\cdot[a]\cdot[c]=[delta]$$
$$\sqrt{\Delta}=\sqrt{[delta]}=[sdelta1]$$
$$t_{1}=[t11]\in[0,\infty),t_{2}=[t21]\in[0,\infty)$$
$$x^{2}=[t11]\vee x^{2}=[t21]$$
$$x=[x11]\vee x=-[x11]\vee x=[x21]\vee x=-[x21]$$
$$x\in\{-[x11],[x11],-[x21],[x21]\}$$
\rozwStop
\odpStart
$$x\in\{-[x11],[x11],-[x21],[x21]\}$$
\odpStop
\testStart
A.$x\in\{-[x11],[x11],-[x21],[x21]\}$
B.$x\in\{[x11],[x21]\}$
C.$x\in\{-[x11],-[x21]\}$
D.$x\in\{-[x11],[x21]\}$
\testStop
\kluczStart
A
\kluczStop



\end{document}