\documentclass[12pt, a4paper]{article}
\usepackage[utf8]{inputenc}
\usepackage{polski}

\usepackage{amsthm}  %pakiet do tworzenia twierdzeń itp.
\usepackage{amsmath} %pakiet do niektórych symboli matematycznych
\usepackage{amssymb} %pakiet do symboli mat., np. \nsubseteq
\usepackage{amsfonts}
\usepackage{graphicx} %obsługa plików graficznych z rozszerzeniem png, jpg
\theoremstyle{definition} %styl dla definicji
\newtheorem{zad}{} 
\title{Multizestaw zadań}
\author{Robert Fidytek}
%\date{\today}
\date{}
\newcounter{liczniksekcji}
\newcommand{\kategoria}[1]{\section{#1}} %olreślamy nazwę kateforii zadań
\newcommand{\zadStart}[1]{\begin{zad}#1\newline} %oznaczenie początku zadania
\newcommand{\zadStop}{\end{zad}}   %oznaczenie końca zadania
%Makra opcjonarne (nie muszą występować):
\newcommand{\rozwStart}[2]{\noindent \textbf{Rozwiązanie (autor #1 , recenzent #2): }\newline} %oznaczenie początku rozwiązania, opcjonarnie można wprowadzić informację o autorze rozwiązania zadania i recenzencie poprawności wykonania rozwiązania zadania
\newcommand{\rozwStop}{\newline}                                            %oznaczenie końca rozwiązania
\newcommand{\odpStart}{\noindent \textbf{Odpowiedź:}\newline}    %oznaczenie początku odpowiedzi końcowej (wypisanie wyniku)
\newcommand{\odpStop}{\newline}                                             %oznaczenie końca odpowiedzi końcowej (wypisanie wyniku)
\newcommand{\testStart}{\noindent \textbf{Test:}\newline} %ewentualne możliwe opcje odpowiedzi testowej: A. ? B. ? C. ? D. ? itd.
\newcommand{\testStop}{\newline} %koniec wprowadzania odpowiedzi testowych
\newcommand{\kluczStart}{\noindent \textbf{Test poprawna odpowiedź:}\newline} %klucz, poprawna odpowiedź pytania testowego (jedna literka): A lub B lub C lub D itd.
\newcommand{\kluczStop}{\newline} %koniec poprawnej odpowiedzi pytania testowego 
\newcommand{\wstawGrafike}[2]{\begin{figure}[h] \includegraphics[scale=#2] {#1} \end{figure}} %gdyby była potrzeba wstawienia obrazka, parametry: nazwa pliku, skala (jak nie wiesz co wpisać, to wpisz 1)

\begin{document}
\maketitle


\kategoria{Wikieł/Z5.2c}
\zadStart{Zadanie z Wikieł Z 5.2c) moja wersja nr [nrWersji]}
%[x]:[2,3,4,5,6,7,8,9,10,11,12,15,17]
%[y]:[2,3,4,5,6,7,8,9,10,11,12,15,17]
%[a]=random.randint(2,100)
%[b]=4*[a]
%[c]=random.randint(-30,30)
%[m]=2*[a]
%[d]=[b]*pow([c],3)
Korzystając z definicji pochodnej, obliczyć wartość wskazanej pochodnej funkcji $f$.\\
$f(x)=[a]x^4 $, $f'([c])$
\zadStop
\rozwStart{Katarzyna Filipowicz}{}
Niech $h=\Delta x$
$$
f'(x)=lim_{h\rightarrow 0} \frac{[a](x+h)^4-[a]x^4}{h}=
$$ $$
=[a]\cdot lim_{h\rightarrow 0}\frac{x^4+4x^3h+6x^2h^2+4xh^3+h^4-x^4}{h}=
$$ $$
=[a]\cdot lim_{h\rightarrow 0} (4x^3+6x^2h+4xh^2+h^3)=[a] \cdot 4x^3 = [b]x^3
$$ $$
f'([c])=[b] \cdot ([c])^3=[d]
$$
\rozwStop
\odpStart
$f'([c])=[d]$
\odpStop
\testStart
A.$f'([c])=[d]$\\
B.$f'([c])=0$\\
C.$f'([c])=[b]$\\
D.$f'([c])=[a]$\\
E.$f'([c])=[c]$\\
F.$f'([c])=[x]$\\
G.$f'([c])=1$\\
H.$f'([c])=\infty$\\
I.$f'([c])=-[a]$
\testStop
\kluczStart
A
\kluczStop



\end{document}