\documentclass[12pt, a4paper]{article}
\usepackage[utf8]{inputenc}
\usepackage{polski}

\usepackage{amsthm}  %pakiet do tworzenia twierdzeń itp.
\usepackage{amsmath} %pakiet do niektórych symboli matematycznych
\usepackage{amssymb} %pakiet do symboli mat., np. \nsubseteq
\usepackage{amsfonts}
\usepackage{graphicx} %obsługa plików graficznych z rozszerzeniem png, jpg
\theoremstyle{definition} %styl dla definicji
\newtheorem{zad}{} 
\title{Multizestaw zadań}
\author{Robert Fidytek}
%\date{\today}
\date{}
\newcounter{liczniksekcji}
\newcommand{\kategoria}[1]{\section{#1}} %olreślamy nazwę kateforii zadań
\newcommand{\zadStart}[1]{\begin{zad}#1\newline} %oznaczenie początku zadania
\newcommand{\zadStop}{\end{zad}}   %oznaczenie końca zadania
%Makra opcjonarne (nie muszą występować):
\newcommand{\rozwStart}[2]{\noindent \textbf{Rozwiązanie (autor #1 , recenzent #2): }\newline} %oznaczenie początku rozwiązania, opcjonarnie można wprowadzić informację o autorze rozwiązania zadania i recenzencie poprawności wykonania rozwiązania zadania
\newcommand{\rozwStop}{\newline}                                            %oznaczenie końca rozwiązania
\newcommand{\odpStart}{\noindent \textbf{Odpowiedź:}\newline}    %oznaczenie początku odpowiedzi końcowej (wypisanie wyniku)
\newcommand{\odpStop}{\newline}                                             %oznaczenie końca odpowiedzi końcowej (wypisanie wyniku)
\newcommand{\testStart}{\noindent \textbf{Test:}\newline} %ewentualne możliwe opcje odpowiedzi testowej: A. ? B. ? C. ? D. ? itd.
\newcommand{\testStop}{\newline} %koniec wprowadzania odpowiedzi testowych
\newcommand{\kluczStart}{\noindent \textbf{Test poprawna odpowiedź:}\newline} %klucz, poprawna odpowiedź pytania testowego (jedna literka): A lub B lub C lub D itd.
\newcommand{\kluczStop}{\newline} %koniec poprawnej odpowiedzi pytania testowego 
\newcommand{\wstawGrafike}[2]{\begin{figure}[h] \includegraphics[scale=#2] {#1} \end{figure}} %gdyby była potrzeba wstawienia obrazka, parametry: nazwa pliku, skala (jak nie wiesz co wpisać, to wpisz 1)

\begin{document}
\maketitle


\kategoria{Wikieł/Z5.18 c}
\zadStart{Zadanie z Wikieł Z 5.18 c) moja wersja nr [nrWersji]}
%[a]:[2,3,4,5,6,7,8,9]
%[c]:[2,3,4,5,6,7,8,9]
%[d]=random.randint(2,10)
%[e]=[a]*[c]*[c]
%[f]=3*[d]
%math.gcd([e],[f])==1
Oblicz granicę $\lim_{x \rightarrow 0} \frac{[a]x-[a] \tg^{-1}([c]x)}{[d]x^3}$.
\zadStop
\rozwStart{Joanna Świerzbin}{}
$$ \lim_{x \rightarrow 0} \frac{[a]x-[a] \tg^{-1}([c]x)}{[d]x^3}$$
Otrzymujemy $ \left[ \frac{0}{0} \right] $ więc możemy skorzystać z twierdzenia de l'Hospitala.
$$ \lim_{x \rightarrow 0} \frac{\left([a]x-[a] \tg^{-1}([c]x)\right)'}{\left([d]x^3\right)'}=\lim_{x \rightarrow 0} \frac{[a]-\frac{[a]}{[c]^2x^2+1}}{3\cdot[d]x^2}=
\lim_{x \rightarrow 0} \frac{\frac{[a]\cdot[c]^2x^2+[a]-[a]}{[c]^2x^2+1}}{3\cdot[d]x^2}=$$
$$=\lim_{x \rightarrow 0} {\frac{[a]\cdot[c]^2x^2}{[c]^2x^2+1}}\cdot{\frac{1}{3\cdot[d]x^2}}= \lim_{x \rightarrow 0} {\frac{[a]\cdot[c]^2x^2}{3\cdot[d]\cdot [c]^2x^4+3\cdot[d]x^2}}= \lim_{x \rightarrow 0} {\frac{[a]\cdot[c]^2}{3\cdot[d]\cdot [c]^2x^2+3\cdot[d]}} =$$
 $$ = \frac{[a]\cdot[c]^2}{3\cdot[d]} = \frac{[e]}{[f]}$$
\rozwStop
\odpStart
$ \lim_{x \rightarrow 0} \frac{[a]x-[a] \tg^{-1}([c]x)}{[d]x^3} =  \frac{[e]}{[f]}$
\odpStop
\testStart
A. $ \lim_{x \rightarrow 0} \frac{[a]x-[a] \tg^{-1}([c]x)}{[d]x^3} =  \frac{[e]}{[f]}$\\
B. $ \lim_{x \rightarrow 0} \frac{[a]x-[a] \tg^{-1}([c]x)}{[d]x^3} =  -\infty$\\
C. $ \lim_{x \rightarrow 0} \frac{[a]x-[a] \tg^{-1}([c]x)}{[d]x^3} =  [f]$\\
D. $ \lim_{x \rightarrow 0} \frac{[a]x-[a] \tg^{-1}([c]x)}{[d]x^3} =  \frac{1}{[f]}$\\
E. $ \lim_{x \rightarrow 0} \frac{[a]x-[a] \tg^{-1}([c]x)}{[d]x^3} =  1$\\
F. $ \lim_{x \rightarrow 0} \frac{[a]x-[a] \tg^{-1}([c]x)}{[d]x^3} =  \infty$
\testStop
\kluczStart
A
\kluczStop



\end{document}