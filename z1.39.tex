\documentclass[12pt, a4paper]{article}
\usepackage[utf8]{inputenc}
\usepackage{polski}

\usepackage{amsthm}  %pakiet do tworzenia twierdzeń itp.
\usepackage{amsmath} %pakiet do niektórych symboli matematycznych
\usepackage{amssymb} %pakiet do symboli mat., np. \nsubseteq
\usepackage{amsfonts}
\usepackage{graphicx} %obsługa plików graficznych z rozszerzeniem png, jpg
\theoremstyle{definition} %styl dla definicji
\newtheorem{zad}{} 
\title{Multizestaw zadań}
\author{Robert Fidytek}
%\date{\today}
\date{}\documentclass[12pt, a4paper]{article}
\usepackage[utf8]{inputenc}
\usepackage{polski}

\usepackage{amsthm}  %pakiet do tworzenia twierdzeń itp.
\usepackage{amsmath} %pakiet do niektórych symboli matematycznych
\usepackage{amssymb} %pakiet do symboli mat., np. \nsubseteq
\usepackage{amsfonts}
\usepackage{graphicx} %obsługa plików graficznych z rozszerzeniem png, jpg
\theoremstyle{definition} %styl dla definicji
\newtheorem{zad}{} 
\title{Multizestaw zadań}
\author{Robert Fidytek}
%\date{\today}
\date{}
\newcounter{liczniksekcji}
\newcommand{\kategoria}[1]{\section{#1}} %olreślamy nazwę kateforii zadań
\newcommand{\zadStart}[1]{\begin{zad}#1\newline} %oznaczenie początku zadania
\newcommand{\zadStop}{\end{zad}}   %oznaczenie końca zadania
%Makra opcjonarne (nie muszą występować):
\newcommand{\rozwStart}[2]{\noindent \textbf{Rozwiązanie (autor #1 , recenzent #2): }\newline} %oznaczenie początku rozwiązania, opcjonarnie można wprowadzić informację o autorze rozwiązania zadania i recenzencie poprawności wykonania rozwiązania zadania
\newcommand{\rozwStop}{\newline}                                            %oznaczenie końca rozwiązania
\newcommand{\odpStart}{\noindent \textbf{Odpowiedź:}\newline}    %oznaczenie początku odpowiedzi końcowej (wypisanie wyniku)
\newcommand{\odpStop}{\newline}                                             %oznaczenie końca odpowiedzi końcowej (wypisanie wyniku)
\newcommand{\testStart}{\noindent \textbf{Test:}\newline} %ewentualne możliwe opcje odpowiedzi testowej: A. ? B. ? C. ? D. ? itd.
\newcommand{\testStop}{\newline} %koniec wprowadzania odpowiedzi testowych
\newcommand{\kluczStart}{\noindent \textbf{Test poprawna odpowiedź:}\newline} %klucz, poprawna odpowiedź pytania testowego (jedna literka): A lub B lub C lub D itd.
\newcommand{\kluczStop}{\newline} %koniec poprawnej odpowiedzi pytania testowego 
\newcommand{\wstawGrafike}[2]{\begin{figure}[h] \includegraphics[scale=#2] {#1} \end{figure}} %gdyby była potrzeba wstawienia obrazka, parametry: nazwa pliku, skala (jak nie wiesz co wpisać, to wpisz 1)

\begin{document}
\maketitle


\kategoria{Wikieł/Z1.39}
\zadStart{Zadanie z Wikieł Z 1.39 moja wersja nr [nrWersji]}
%[p1]:[2,3,4,5,6,7,8,9]
%[p2]:[2,3,4,5,6,7,8,9]
%[p3]=random.randint(2,10)
%[p4]:[2,3,4,5,6,7,8,9]
%[p5]=random.randint(2,10)
%[p6]=random.randint(2,10)
%[p7]=random.randint(2,10)
%[p1p4m]=[p1]-[p4]
%[p3p5m]=[p3]-[p5]
%[p5p7m]=[p5]-[p7]
%[del]=[p2]*[p2]-4*[p1p4m]*[p3p5m]
%[pdel]=round(math.sqrt(abs([del])),2)
%[2p1p4m]=2*[p1p4m]
%[del2]=[p6]*[p6]-4*[p4]*[p5p7m]
%[pdel2]=round(math.sqrt(abs([del2])),2)
%[2p4]=2*[p4]
%[x11]=round(-[p2]-[pdel],2)
%[x12]=round(-[p2]+[pdel],2)
%[x21]=round([p6]-[pdel2],2)
%[x22]=round([p6]+[pdel2],2)
%[p1]>[p4] and [del]>0 and [del2]>0 and ([x21]/[2p4])<([x12]/[2p1p4m]) and ([x12]/[2p1p4m])<([x22]/[2p4])

Rozwiązać nierówność podwójną.
$$[p1]x^{2}+[p2]x+[p3]\leq [p4]x^{2}+[p5] \leq [p6]x+[p7]$$
\zadStop

\rozwStart{Maja Szabłowska}{}
Rozdzielamy nierówność na dwie nierówności pojedyncze.
$$[p1]x^{2}+[p2]x+[p3]\leq [p4]x^{2}+[p5] $$
$$[p1p4m]x^{2}+[p2]x+[p3p5m]\leq0$$
$$\Delta_{1}=[p2]^{2}-4\cdot[p1p4m]\cdot[p3p5m]=[del] \Rightarrow \sqrt{\Delta_{1}}=[pdel]$$
$$x_{11}=\frac{-[p2]-[pdel]}{[2p1p4m]}=\frac{[x11]}{[2p1p4m]}, \quad x_{12}=\frac{-[p2]+[pdel]}{[2p1p4m]}=\frac{[x12]}{[2p1p4m]}$$
$$x\in\left[\frac{[x11]}{[2p1p4m]},\frac{[x12]}{[2p1p4m]}\right]$$

$$[p4]x^{2}+[p5] \leq [p6]x+[p7]$$
$$[p4]x^{2}-[p6]x+[p5p7m]\leq0$$
$$\Delta_{2}=(-[p6])^{2}-4\cdot[p4]\cdot[p5p7m]=[del2] \Rightarrow \sqrt{\Delta_{2}}=[pdel2]$$
$$x_{21}=\frac{[p6]-[pdel2]}{[2p4]}=\frac{[x21]}{[2p4]}, \quad x_{22}=\frac{[p6]+[pdel2]}{[2p4]}=\frac{[x22]}{[2p4]}$$
$$x\in\left[\frac{[x21]}{[2p4]}, \frac{[x22]}{[2p4]}\right]$$

Zatem odpowiedzią jest część wspólna obu zbiorów.
$$x\in\left[\frac{[x21]}{[2p4]},\frac{[x12]}{[2p1p4m]}\right]$$
\rozwStop


\odpStart
$W([np1],-\frac{[ndel]}{[m]})$
\odpStop
\testStart
A.$W([np1],-\frac{[ndel]}{[m]})$
B.$W([np1],\frac{[ndel]}{[m]})$
C.$W([np1],0)$
D.$W([p1],-[p2])$
E.$W(0,-\frac{[ndel]}{[m]})$
F.$W([np1],-\frac{[ndel]}{[p2]})$


\testStop
\kluczStart
A
\kluczStop



\end{document}
