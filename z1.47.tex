\documentclass[12pt, a4paper]{article}
\usepackage[utf8]{inputenc}
\usepackage{polski}

\usepackage{amsthm}  %pakiet do tworzenia twierdzeń itp.
\usepackage{amsmath} %pakiet do niektórych symboli matematycznych
\usepackage{amssymb} %pakiet do symboli mat., np. \nsubseteq
\usepackage{amsfonts}
\usepackage{graphicx} %obsługa plików graficznych z rozszerzeniem png, jpg
\theoremstyle{definition} %styl dla definicji
\newtheorem{zad}{} 
\title{Multizestaw zadań}
\author{Robert Fidytek}
%\date{\today}
\date{}\documentclass[12pt, a4paper]{article}
\usepackage[utf8]{inputenc}
\usepackage{polski}

\usepackage{amsthm}  %pakiet do tworzenia twierdzeń itp.
\usepackage{amsmath} %pakiet do niektórych symboli matematycznych
\usepackage{amssymb} %pakiet do symboli mat., np. \nsubseteq
\usepackage{amsfonts}
\usepackage{graphicx} %obsługa plików graficznych z rozszerzeniem png, jpg
\theoremstyle{definition} %styl dla definicji
\newtheorem{zad}{} 
\title{Multizestaw zadań}
\author{Robert Fidytek}
%\date{\today}
\date{}
\newcounter{liczniksekcji}
\newcommand{\kategoria}[1]{\section{#1}} %olreślamy nazwę kateforii zadań
\newcommand{\zadStart}[1]{\begin{zad}#1\newline} %oznaczenie początku zadania
\newcommand{\zadStop}{\end{zad}}   %oznaczenie końca zadania
%Makra opcjonarne (nie muszą występować):
\newcommand{\rozwStart}[2]{\noindent \textbf{Rozwiązanie (autor #1 , recenzent #2): }\newline} %oznaczenie początku rozwiązania, opcjonarnie można wprowadzić informację o autorze rozwiązania zadania i recenzencie poprawności wykonania rozwiązania zadania
\newcommand{\rozwStop}{\newline}                                            %oznaczenie końca rozwiązania
\newcommand{\odpStart}{\noindent \textbf{Odpowiedź:}\newline}    %oznaczenie początku odpowiedzi końcowej (wypisanie wyniku)
\newcommand{\odpStop}{\newline}                                             %oznaczenie końca odpowiedzi końcowej (wypisanie wyniku)
\newcommand{\testStart}{\noindent \textbf{Test:}\newline} %ewentualne możliwe opcje odpowiedzi testowej: A. ? B. ? C. ? D. ? itd.
\newcommand{\testStop}{\newline} %koniec wprowadzania odpowiedzi testowych
\newcommand{\kluczStart}{\noindent \textbf{Test poprawna odpowiedź:}\newline} %klucz, poprawna odpowiedź pytania testowego (jedna literka): A lub B lub C lub D itd.
\newcommand{\kluczStop}{\newline} %koniec poprawnej odpowiedzi pytania testowego 
\newcommand{\wstawGrafike}[2]{\begin{figure}[h] \includegraphics[scale=#2] {#1} \end{figure}} %gdyby była potrzeba wstawienia obrazka, parametry: nazwa pliku, skala (jak nie wiesz co wpisać, to wpisz 1)

\begin{document}
\maketitle


\kategoria{Wikieł/Z1.47}
\zadStart{Zadanie z Wikieł Z 1.47 moja wersja nr [nrWersji]}
%[p1]:[2,3,4,5,6,7,8,9]
%[p2]:[2,3,4,5,6,7,8,9]
%[p3]=random.randint(2,10)
%[2p2]=2*[p2]
%[kp2]=[p2]*[p2]
%[4p1]=4*[p1]
%[4p1p3]=4*[p1]*[p3]
%[m4p1]=1-[4p1]
%[r]=[kp2]-[4p1p3]
%[del]=[2p2]*[2p2]-4*[r]
%[pdel]=round(math.sqrt(abs([del])),2)
%[2a]=2*1
%[m1]=round((-[2p2]-[pdel])/[2a],2)
%[m2]=round((-[2p2]+[pdel])/[2a],2)
%[2p1]=2*[p1]
%[2p1p2]=[2p1]-[p2]
%[22p1p2]=2*[2p1p2]
%[2p1p2k]=[2p1p2]*[2p1p2]
%[m11]=[2p2]+[22p1p2]
%[r1]=[r]-[2p1p2k]
%[pr1]=-[r1]
%[d]=math.gcd([pr1],[m11])
%[nl]=int([pr1]/[d])
%[nm]=int([m11]/[d])
%[del]>0 and ([nl]/[nm])>[m2]

Wyznaczyć wartość parametru $m$, dla których oba pierwiastki równania $[p1]x^{2}-(m+[p2])x+[p3]=0$ są większe od 1.
\zadStop

\rozwStart{Maja Szabłowska}{}
Aby równanie posiadało dwa pierwiastki powinien być spełniony warunek $\Delta\geq0.$
$$\Delta=(m+[p2])^{2}-4\cdot[p1]\cdot[p3]=m^{2}+[2p2]m+[kp2]-[4p1p3]$$
$$=m^{2}+[2p2]m+[r]\geq0$$

$$m^{2}+[2p2]m+[r]\geq0$$
$$\Delta=[2p2]^{2}-4\cdot1\cdot[r]=[del] \Rightarrow \sqrt{\Delta}=[pdel]$$
$$m_{1}=\frac{-[2p2]-[pdel]}{[2a]}=[m1], \quad m_{2}=\frac{-[2p2]+[pdel]}{[2a]}=[m2]$$
Zatem $m\in[[m1], [m2]]$

Przejdźmy do warunku:
$$x_{1}>1 \quad \land \quad x_{2}>1$$
$$\frac{m+[p2]-\sqrt{m^{2}+[2p2]m+[r]}}{[2p1]}>1\quad \land \quad\frac{m+[p2]+\sqrt{m^{2}+[2p2]m+[r]}}{[2p1]}>1$$
$$m+[p2]-\sqrt{m^{2}+[2p2]m+[r]}>[2p1]\quad \land \quad m+[p2]+\sqrt{m^{2}+[2p2]m+[r]}>[2p1]$$

$$-\sqrt{m^{2}+[2p2]m+[r]}>-m+[2p1p2] \quad \land \quad \sqrt{m^{2}+[2p2]m+[r]}>-m+[2p1p2]$$
$$m^{2}+[2p2]m+[r]>(-m+[2p1p2])^{2} \quad \land \quad 
m^{2}+[2p2]m+[r]>(-m+[2p1p2])^{2}$$

$$m^{2}+[2p2]m+[r]-m^{2}+[22p1p2]m-[2p1p2k]>0 \quad \land \quad m^{2}+[2p2]m+[r]-m^{2}+[22p1p2]m-[2p1p2k]>0$$

$$[m11]m+[r1]>0 \quad \land \quad [m11]m+[r1]>0$$
Równania zatem sprowadzają się do jednego.
$$[m11]m-[pr1]>0$$
$$[m11]m>[pr1]$$
$$m>\frac{[pr1]}{[m11]}$$
$$m\in\left(\frac{[nl]}{[nm]}, \infty\right)$$

Odpowiedzią jest część wspólna obu wyznaczonych zbiorów, zatem jest to zbiór pusty.
\rozwStop


\odpStart
$m\in\emptyset$
\odpStop
\testStart
A.$m\in\emptyset$
B.$m\in([m2],[m1])$
C.$m\in(-\infty,[m2])$
D.$m\in([p1],[p2])$
E.$m\in([m11],[pr1])$
F.$m\in[[m1],\infty)$


\testStop
\kluczStart
A
\kluczStop



\end{document}
