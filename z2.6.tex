\documentclass[12pt, a4paper]{article}
\usepackage[utf8]{inputenc}
\usepackage{polski}

\usepackage{amsthm}  %pakiet do tworzenia twierdzeń itp.
\usepackage{amsmath} %pakiet do niektórych symboli matematycznych
\usepackage{amssymb} %pakiet do symboli mat., np. \nsubseteq
\usepackage{amsfonts}
\usepackage{graphicx} %obsługa plików graficznych z rozszerzeniem png, jpg
\theoremstyle{definition} %styl dla definicji
\newtheorem{zad}{} 
\title{Multizestaw zadań}
\author{Robert Fidytek}
%\date{\today}
\date{}
\newcounter{liczniksekcji}
\newcommand{\kategoria}[1]{\section{#1}} %olreślamy nazwę kateforii zadań
\newcommand{\zadStart}[1]{\begin{zad}#1\newline} %oznaczenie początku zadania
\newcommand{\zadStop}{\end{zad}}   %oznaczenie końca zadania
%Makra opcjonarne (nie muszą występować):
\newcommand{\rozwStart}[2]{\noindent \textbf{Rozwiązanie (autor #1 , recenzent #2): }\newline} %oznaczenie początku rozwiązania, opcjonarnie można wprowadzić informację o autorze rozwiązania zadania i recenzencie poprawności wykonania rozwiązania zadania
\newcommand{\rozwStop}{\newline}                                            %oznaczenie końca rozwiązania
\newcommand{\odpStart}{\noindent \textbf{Odpowiedź:}\newline}    %oznaczenie początku odpowiedzi końcowej (wypisanie wyniku)
\newcommand{\odpStop}{\newline}                                             %oznaczenie końca odpowiedzi końcowej (wypisanie wyniku)
\newcommand{\testStart}{\noindent \textbf{Test:}\newline} %ewentualne możliwe opcje odpowiedzi testowej: A. ? B. ? C. ? D. ? itd.
\newcommand{\testStop}{\newline} %koniec wprowadzania odpowiedzi testowych
\newcommand{\kluczStart}{\noindent \textbf{Test poprawna odpowiedź:}\newline} %klucz, poprawna odpowiedź pytania testowego (jedna literka): A lub B lub C lub D itd.
\newcommand{\kluczStop}{\newline} %koniec poprawnej odpowiedzi pytania testowego 
\newcommand{\wstawGrafike}[2]{\begin{figure}[h] \includegraphics[scale=#2] {#1} \end{figure}} %gdyby była potrzeba wstawienia obrazka, parametry: nazwa pliku, skala (jak nie wiesz co wpisać, to wpisz 1)

\begin{document}
\maketitle


\kategoria{Wikieł/Z2.6}
\zadStart{Zadanie z Wikieł Z 2.6 moja wersja nr [nrWersji]}
%[v1]:[1,2,3,4,5,6,7,8,9]
%[v3]=-[v1]
%[u1]:[1,2,3,4,5,6,7,8,9,10,11,12,13,14,15]
%[u2]:[2,3,4,5,6,7,8,9,10,11,12,13,14,15]
%[u3]=[u1]
%[ku1]=[u1]*[u1]
%[ku2]=[u2]*[u2]
%[ku3]=[u3]*[u3]
%[kv1]=[v1]*[v1]
%[kv3]=[v3]*[v3]
%[a]=[u1]*[v1]
%[b]=[u3]*[v3]
%[ab]=[a]+[b]
%[U]=[ku1]+[ku2]+[ku3]
%[V]=[kv1]+[kv3]
%[VU]=[V]*[U]
%[2u2]=2*[u2]
%[k2u2]=[2u2]*[2u2]
%[x]=[VU]-[ab]
%[y]=[k2u2]-[U]
%[xy]=[x]/[y]
%[cxy]=int([xy])
%[xy].is_integer()==True and [k2u2]>[U] and [VU]>[ab] and math.gcd([cxy],4)==2 and math.gcd([cxy],9)==3 and math.gcd([cxy],25)==1 and math.gcd([cxy],49)==1
Wyznaczyć wartość parametru k, dla których wektor [[v1],k,[v3]] tworzy z \\ wektorem [[u1],[u2],[u3]] kąt o mierze $\frac{\pi}{3}$.
\zadStop
\rozwStart{Aleksandra Pasińska}{}
$$cos\angle(v,u)=\frac{u\circ v}{|u|\cdot |v|}$$
$$u\circ v=[u1]\cdot [v1]+[u2]\cdot k+[u3]\cdot ([v3])=[a]+[u2]k+([b])=[u2]k$$
$$|u|\cdot |v|=\sqrt{[u1]^2+[u2]^2+[u3]^2}\cdot \sqrt{[v1]^2+k^2+([v3])^2}=\sqrt{[U]}\cdot\sqrt{[V]+k^2}$$
$$cos\angle(v,u)=cos(\frac{\pi}{3})=\frac{1}{2}$$
$$\frac{1}{2}=\frac{[u2]k}{\sqrt{[U]}\cdot\sqrt{[V]+k^2}}$$
$$\sqrt{[U]}\cdot\sqrt{[V]+k^2}=2\cdot[u2]k$$
$$\sqrt{[VU]+[U]k^2}=[2u2]k \bigg/^2$$
$$[VU]+[U]k^2=[k2u2]k^2$$
$$[k2u2]k^2-[U]k^2=[VU]$$
$$k^2=[cxy],k=\sqrt{[cxy]}$$
\rozwStop
\odpStart
$k=\sqrt{[cxy]}$\\
\odpStop
\testStart
A.$k=\sqrt{[cxy]}$
B.$k=\infty$
C.$k=-\infty$
D.$k=1$
E.$k=9$
F.$k=e$
G.$k=4$
H.$k=7$
I.$k=-7$
\testStop
\kluczStart
A
\kluczStop



\end{document}