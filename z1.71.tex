\documentclass[12pt, a4paper]{article}
\usepackage[utf8]{inputenc}
\usepackage{polski}

\usepackage{amsthm}  %pakiet do tworzenia twierdzeń itp.
\usepackage{amsmath} %pakiet do niektórych symboli matematycznych
\usepackage{amssymb} %pakiet do symboli mat., np. \nsubseteq
\usepackage{amsfonts}
\usepackage{graphicx} %obsługa plików graficznych z rozszerzeniem png, jpg
\theoremstyle{definition} %styl dla definicji
\newtheorem{zad}{} 
\title{Multizestaw zadań}
\author{Robert Fidytek}
%\date{\today}
\date{}
\newcounter{liczniksekcji}
\newcommand{\kategoria}[1]{\section{#1}} %olreślamy nazwę kateforii zadań
\newcommand{\zadStart}[1]{\begin{zad}#1\newline} %oznaczenie początku zadania
\newcommand{\zadStop}{\end{zad}}   %oznaczenie końca zadania
%Makra opcjonarne (nie muszą występować):
\newcommand{\rozwStart}[2]{\noindent \textbf{Rozwiązanie (autor #1 , recenzent #2): }\newline} %oznaczenie początku rozwiązania, opcjonarnie można wprowadzić informację o autorze rozwiązania zadania i recenzencie poprawności wykonania rozwiązania zadania
\newcommand{\rozwStop}{\newline}                                            %oznaczenie końca rozwiązania
\newcommand{\odpStart}{\noindent \textbf{Odpowiedź:}\newline}    %oznaczenie początku odpowiedzi końcowej (wypisanie wyniku)
\newcommand{\odpStop}{\newline}                                             %oznaczenie końca odpowiedzi końcowej (wypisanie wyniku)
\newcommand{\testStart}{\noindent \textbf{Test:}\newline} %ewentualne możliwe opcje odpowiedzi testowej: A. ? B. ? C. ? D. ? itd.
\newcommand{\testStop}{\newline} %koniec wprowadzania odpowiedzi testowych
\newcommand{\kluczStart}{\noindent \textbf{Test poprawna odpowiedź:}\newline} %klucz, poprawna odpowiedź pytania testowego (jedna literka): A lub B lub C lub D itd.
\newcommand{\kluczStop}{\newline} %koniec poprawnej odpowiedzi pytania testowego 
\newcommand{\wstawGrafike}[2]{\begin{figure}[h] \includegraphics[scale=#2] {#1} \end{figure}} %gdyby była potrzeba wstawienia obrazka, parametry: nazwa pliku, skala (jak nie wiesz co wpisać, to wpisz 1)

\begin{document}
\maketitle


\kategoria{Wikieł/Z1.71}
\zadStart{Zadanie z Wikieł Z 1.71) moja wersja nr [nrWersji]}
%[a]:[2,3,4,5,6,7]
%[b]:[2,3,4,5,6,7]
%[c]:[2,3,4,5,6,7]
%[d]:[2,3,4,5,6,7]
%[a]=random.randint(2,15)
%[b]=random.randint(2,15)
%[c]=random.randint(2,15)
%[d]=random.randint(2,15)
%[reszta1]=(([b]*[a])+[c])
%[lreszta1]=[a]+[d]
%[reszta2]=(([b]*[d])+[c])
%[l1]=[reszta1]*[lreszta1]
%[lreszta1b]=[lreszta1]*[b]
%[w1]=([reszta2]-[lreszta1b]+[reszta1])
%math.gcd([reszta2],[b])==1 and math.gcd([reszta1],[b])==1 and math.gcd([l1],[w1])==1
Dana jest funkcja $f(x)=\frac{[a]-x}{[b]x+[c]}$. Rozwiązać nierówność.
$$f(x+[a])\leq f(x-[d])$$
\zadStop
\rozwStart{Jakub Ulrych}{}
$$f(x+[a])\leq f(x-[d])=\frac{[a]-(x+[a])}{[b](x+[a])+[c]}\leq\frac{[a]-(x-[d])}{[b](x-[d])+[c]}$$
$$\frac{-x}{[b]x+[reszta1]}\leq\frac{[lreszta1]-x}{[b]x-[reszta2]}$$
założenie:$$[b]x+[reszta1]\neq0 \land [b]x-[reszta2]\neq0$$
$$x\neq\frac{-[reszta1]}{[b]} \land x\neq\frac{[reszta2]}{[b]}$$
dziedzina:$$x\in\mathbb{R}-\{\frac{-[reszta1]}{[b]},\frac{[reszta2]}{[b]}\}$$
rozwiązanie:$$\frac{-x}{[b]x+[reszta1]}\leq\frac{[lreszta1]-x}{[b]x-[reszta2]}$$
$$\frac{-x([b]x-[reszta2])-([lreszta1]-x)([b]x+[reszta1])}{([b]x+[reszta1])([b]x-[reszta2])}\leq0$$
$$\frac{[w1]x-[l1]}{([b]x+[reszta1])([b]x-[reszta2])}\leq0$$
$$\textbf{1)}[w1]x-[l1]\geq0 \land ([b]x+[reszta1])([b]x-[reszta2])\leq0$$ $$\vee$$ $$\textbf{2)}[w1]x-[l1]\leq0 \land ([b]x+[reszta1])([b]x-[reszta2])\geq0$$
$$\textbf{1)} x\geq\frac{[l1]}{[w1]}\land \frac{[reszta2]}{[b]}\leq x\leq\frac{[reszta1]}{[b]}$$ $$\vee$$
$$\textbf{2)}x\leq\frac{[l1]}{[w1]}\land \big(x\leq\frac{-[reszta1]}{[b]}\vee x\geq\frac{[reszta2]}{[b]}\big)$$
$$(\textbf{1)}\vee\textbf{2)})\land\text{dziedzina}\Rightarrow x\in(-\infty,\frac{-[reszta1]}{[b]})\cup(\frac{[reszta2]}{[b]},\frac{[l1]}{[w1]})$$
\rozwStop
\odpStart
$$x\in(-\infty,\frac{-[reszta1]}{[b]})\cup(\frac{[reszta2]}{[b]},\frac{[l1]}{[w1]})$$
\odpStop
\testStart
A.$$x\in(-\infty,\frac{-[reszta1]}{[b]})\cup(\frac{[reszta2]}{[b]},\frac{[l1]}{[w1]})$$
B.$$x\in(\frac{[reszta2]}{[b]},\frac{[l1]}{[w1]})$$
C.$$x\in(-\infty,\frac{-[reszta1]}{[b]})$$
D.$$x\in(-\infty,\frac{-[reszta1]}{[b]})\cup(\frac{[reszta2]}{[b]},\infty)$$
\testStop
\kluczStart
A
\kluczStop
\end{document}