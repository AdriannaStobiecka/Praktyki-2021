\documentclass[12pt, a4paper]{article}
\usepackage[utf8]{inputenc}
\usepackage{polski}

\usepackage{amsthm}  %pakiet do tworzenia twierdzeń itp.
\usepackage{amsmath} %pakiet do niektórych symboli matematycznych
\usepackage{amssymb} %pakiet do symboli mat., np. \nsubseteq
\usepackage{amsfonts}
\usepackage{graphicx} %obsługa plików graficznych z rozszerzeniem png, jpg
\theoremstyle{definition} %styl dla definicji
\newtheorem{zad}{} 
\title{Multizestaw zadań}
\author{Robert Fidytek}
%\date{\today}
\date{}
\newcounter{liczniksekcji}
\newcommand{\kategoria}[1]{\section{#1}} %olreślamy nazwę kateforii zadań
\newcommand{\zadStart}[1]{\begin{zad}#1\newline} %oznaczenie początku zadania
\newcommand{\zadStop}{\end{zad}}   %oznaczenie końca zadania
%Makra opcjonarne (nie muszą występować):
\newcommand{\rozwStart}[2]{\noindent \textbf{Rozwiązanie (autor #1 , recenzent #2): }\newline} %oznaczenie początku rozwiązania, opcjonarnie można wprowadzić informację o autorze rozwiązania zadania i recenzencie poprawności wykonania rozwiązania zadania
\newcommand{\rozwStop}{\newline}                                            %oznaczenie końca rozwiązania
\newcommand{\odpStart}{\noindent \textbf{Odpowiedź:}\newline}    %oznaczenie początku odpowiedzi końcowej (wypisanie wyniku)
\newcommand{\odpStop}{\newline}                                             %oznaczenie końca odpowiedzi końcowej (wypisanie wyniku)
\newcommand{\testStart}{\noindent \textbf{Test:}\newline} %ewentualne możliwe opcje odpowiedzi testowej: A. ? B. ? C. ? D. ? itd.
\newcommand{\testStop}{\newline} %koniec wprowadzania odpowiedzi testowych
\newcommand{\kluczStart}{\noindent \textbf{Test poprawna odpowiedź:}\newline} %klucz, poprawna odpowiedź pytania testowego (jedna literka): A lub B lub C lub D itd.
\newcommand{\kluczStop}{\newline} %koniec poprawnej odpowiedzi pytania testowego 
\newcommand{\wstawGrafike}[2]{\begin{figure}[h] \includegraphics[scale=#2] {#1} \end{figure}} %gdyby była potrzeba wstawienia obrazka, parametry: nazwa pliku, skala (jak nie wiesz co wpisać, to wpisz 1)

\begin{document}
\maketitle


\kategoria{Wikieł/Z1.30a}
\zadStart{Zadanie z Wikieł Z 1.30 a)  moja wersja nr [nrWersji]}
%[p1]:[4,5,6,7,8,9,10,11,12]
%[p2]:[2,3,4,5,6,7,8,9,10,11,12]
%[p3]=random.randint(1,10)
%[p1p2m]=[p1]*[p2]
Podać wzór określający złożenie funkcji $f(g(x))$ oraz $g(f(x))$ dla funkcji $f(x)=[p1]x^{2},\quad g(x)=[p2]x+[p3].$
\zadStop
\rozwStart{Maja Szabłowska}{}
$$(f\circ g)(x)=f(g(x))=f([p2]x+[p3])=([p2]x+[p3])^{2}$$
$$(g\circ f)(x)=g(f(x))=g([p1]x^{2})=[p1p2m]x^{2}+[p3]$$
\rozwStop
\odpStart
$(f\circ g)(x)=([p2]x+[p3])^{2},\quad (g\circ f)(x)=[p1p2m]x^{2}+[p3]$
\odpStop
\testStart
A.$(f\circ g)(x)=([p2]x+[p3])^{2},\quad (g\circ f)(x)=[p1p2m]x^{2}+[p3]$
B.$(f\circ g)(x)=([p2]x+[p3])^{2},\quad (g\circ f)(x)=[p1]x^{2}+[p3]$
C.$(f\circ g)(x)=([p2]x+[p3])^{2},\quad (g\circ f)(x)=[p2]x^{2}+[p3]$
D.$(f\circ g)(x)=([p2]x+[p3])^{3},\quad (g\circ f)(x)=[p1p2m]x^{2}+[p3]$
E.$(f\circ g)(x)=([p1p2m]x+[p3])^{2},\quad (g\circ f)(x)=[p1p2m]x^{2}+[p3]$
F.$(f\circ g)(x)=([p2]x+[p3])^{2},\quad (g\circ f)(x)=[p1p2m]x^{2}+[p1]$
G.$(f\circ g)(x)=([p2]x+[p3])^{2},\quad (g\circ f)(x)=[p1p2m]x^{2}+[p2]$
H.$(f\circ g)(x)=([p3]x+[p1])^{2},\quad (g\circ f)(x)=[p1p2m]x^{2}+[p3]$
I.$(f\circ g)(x)=([p2]x+[p3])^{2},\quad (g\circ f)(x)=[p2]x+[p3]$
\testStop
\kluczStart
A
\kluczStop



\end{document}