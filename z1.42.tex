\documentclass[12pt, a4paper]{article}
\usepackage[utf8]{inputenc}
\usepackage{polski}

\usepackage{amsthm}  %pakiet do tworzenia twierdzeń itp.
\usepackage{amsmath} %pakiet do niektórych symboli matematycznych
\usepackage{amssymb} %pakiet do symboli mat., np. \nsubseteq
\usepackage{amsfonts}
\usepackage{graphicx} %obsługa plików graficznych z rozszerzeniem png, jpg
\theoremstyle{definition} %styl dla definicji
\newtheorem{zad}{} 
\title{Multizestaw zadań}
\author{Laura Mieczkowska}
%\date{\today}
\date{}
\newcounter{liczniksekcji}
\newcommand{\kategoria}[1]{\section{#1}} %olreślamy nazwę kateforii zadań
\newcommand{\zadStart}[1]{\begin{zad}#1\newline} %oznaczenie początku zadania
\newcommand{\zadStop}{\end{zad}}   %oznaczenie końca zadania
%Makra opcjonarne (nie muszą występować):
\newcommand{\rozwStart}[2]{\noindent \textbf{Rozwiązanie (autor #1 , recenzent #2): }\newline} %oznaczenie początku rozwiązania, opcjonarnie można wprowadzić informację o autorze rozwiązania zadania i recenzencie poprawności wykonania rozwiązania zadania
\newcommand{\rozwStop}{\newline}                                            %oznaczenie końca rozwiązania
\newcommand{\odpStart}{\noindent \textbf{Odpowiedź:}\newline}    %oznaczenie początku odpowiedzi końcowej (wypisanie wyniku)
\newcommand{\odpStop}{\newline}                                             %oznaczenie końca odpowiedzi końcowej (wypisanie wyniku)
\newcommand{\testStart}{\noindent \textbf{Test:}\newline} %ewentualne możliwe opcje odpowiedzi testowej: A. ? B. ? C. ? D. ? itd.
\newcommand{\testStop}{\newline} %koniec wprowadzania odpowiedzi testowych
\newcommand{\kluczStart}{\noindent \textbf{Test poprawna odpowiedź:}\newline} %klucz, poprawna odpowiedź pytania testowego (jedna literka): A lub B lub C lub D itd.
\newcommand{\kluczStop}{\newline} %koniec poprawnej odpowiedzi pytania testowego 
\newcommand{\wstawGrafike}[2]{\begin{figure}[h] \includegraphics[scale=#2] {#1} \end{figure}} %gdyby była potrzeba wstawienia obrazka, parametry: nazwa pliku, skala (jak nie wiesz co wpisać, to wpisz 1)

\begin{document}
\maketitle


\kategoria{Wikieł/Z1.42}
\zadStart{Zadanie z Wikieł Z 1.42) moja wersja nr [nrWersji]}
%[a]:[2,3,4,5,6,7,8,9,10]
%[b]:[2,3,4,5,6,7,8,9,10]
%[c]:[3,4,5,6,7,8,9,10]
%[d]=[b]*[c]
%[e]=[b]-[c]
%[b]>[c]
%[f]=[e]**2+4*[d]
%[g]=[e]**2
%[i]=([g]-[f])/-4
%[h]=int([i])
Wyznaczyć wartości parametru $a$, dla których równanie $([b]-x)(x+[c])=a$ ma tylko pierwiastki dodatnie.
\zadStop
\rozwStart{Laura Mieczkowska}{}
$$([b]-x)(x+[c])=a$$
$$[b]x+[d]-x^2-[c]x-a=0$$ 
$$-x^2-[e]x+[d]-a=0$$
$$\triangle=[e]^2+4\cdot([d]-a)=[f]-4a\geq0 \Rightarrow a\leq\frac{[f]}{4}$$
$$x=\frac{[e]-\sqrt{[f]-4a}}{-2}>0 \Rightarrow [e]-\sqrt{[f]-4a}<0 \Rightarrow [e]<\sqrt{[f]-4a} \Rightarrow$$
$$[g]<[f]-4a \Rightarrow \frac{[g]-[f]}{-4}>a \Rightarrow [h]>a$$
$$a\in\bigg[[h];\frac{[f]}{4}\bigg)$$
\odpStart
$a\in[[h];\frac{[f]}{4})$
\odpStop
\testStart
A. $a=0$ \\
B. $a\in[\frac{[f]}{4};\infty)$ \\
C. $a\in\emptyset$ \\
D. $a\in[[h];\frac{[f]}{4})$ 
\testStop
\kluczStart
D
\kluczStop



\end{document}