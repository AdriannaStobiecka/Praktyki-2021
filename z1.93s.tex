\documentclass[12pt, a4paper]{article}
\usepackage[utf8]{inputenc}
\usepackage{polski}

\usepackage{amsthm}  %pakiet do tworzenia twierdzeń itp.
\usepackage{amsmath} %pakiet do niektórych symboli matematycznych
\usepackage{amssymb} %pakiet do symboli mat., np. \nsubseteq
\usepackage{amsfonts}
\usepackage{graphicx} %obsługa plików graficznych z rozszerzeniem png, jpg
\theoremstyle{definition} %styl dla definicji
\newtheorem{zad}{} 
\title{Multizestaw zadań}
\author{Robert Fidytek}
%\date{\today}
\date{}
\newcounter{liczniksekcji}
\newcommand{\kategoria}[1]{\section{#1}} %olreślamy nazwę kateforii zadań
\newcommand{\zadStart}[1]{\begin{zad}#1\newline} %oznaczenie początku zadania
\newcommand{\zadStop}{\end{zad}}   %oznaczenie końca zadania
%Makra opcjonarne (nie muszą występować):
\newcommand{\rozwStart}[2]{\noindent \textbf{Rozwiązanie (autor #1 , recenzent #2): }\newline} %oznaczenie początku rozwiązania, opcjonarnie można wprowadzić informację o autorze rozwiązania zadania i recenzencie poprawności wykonania rozwiązania zadania
\newcommand{\rozwStop}{\newline}                                            %oznaczenie końca rozwiązania
\newcommand{\odpStart}{\noindent \textbf{Odpowiedź:}\newline}    %oznaczenie początku odpowiedzi końcowej (wypisanie wyniku)
\newcommand{\odpStop}{\newline}                                             %oznaczenie końca odpowiedzi końcowej (wypisanie wyniku)
\newcommand{\testStart}{\noindent \textbf{Test:}\newline} %ewentualne możliwe opcje odpowiedzi testowej: A. ? B. ? C. ? D. ? itd.
\newcommand{\testStop}{\newline} %koniec wprowadzania odpowiedzi testowych
\newcommand{\kluczStart}{\noindent \textbf{Test poprawna odpowiedź:}\newline} %klucz, poprawna odpowiedź pytania testowego (jedna literka): A lub B lub C lub D itd.
\newcommand{\kluczStop}{\newline} %koniec poprawnej odpowiedzi pytania testowego 
\newcommand{\wstawGrafike}[2]{\begin{figure}[h] \includegraphics[scale=#2] {#1} \end{figure}} %gdyby była potrzeba wstawienia obrazka, parametry: nazwa pliku, skala (jak nie wiesz co wpisać, to wpisz 1)

\begin{document}
\maketitle


\kategoria{Wikieł/Z1.93s}
\zadStart{Zadanie z Wikieł Z 1.93 s) moja wersja nr [nrWersji]}
%[a]:[1,3,5,7,9,11,13,15,17,19]
%[b]:[1,2,3,4,5,6,7,8,9,10,11,12,13,14]
%[c]:[2,4,6,8,10,12,14,16,18]
%[d]:[1,2,3,4,5,6,7,8,9,10,11,12,13,14]
%[n]:[2,3,4,5,6,7,8]
%[2b]=(pow([n],[b]))
%[2d]=(pow([n],[d]))
%[e]=[a]*[d]
%[f]=[c]*[b]
%[g]=([a]-[c])
%[h]=([e]+[f])
%[ww]=([h]/[g])
%[w]=(int([ww]))
%[x]=(pow([n],[w]))
%[p]=(pow([n]+1,[w]))
%[ww].is_integer()==True and [x]<70 and [h]>0 and [x]!=[2d] and [x]!=(1/[2b]) and [x]>0
Rozwiązać równanie $\frac{[a]}{[b]+\log_{[n]}{x}} - \frac{[c]}{\log_{[n]}{x}-[d]} = 0$
\zadStop
\rozwStart{Małgorzata Ugowska}{}
Szukamy dziedziny:
$$x>0$$
$$[b]+\log_{[n]}{x} \ne 0 \quad \Longrightarrow \quad x \ne {[n]}^{-[b]} \quad \Longrightarrow \quad x \ne \frac{1}{[2b]}$$
$$\log_{[n]}{x} - [d] \ne 0 \quad \Longrightarrow \quad x \ne {[n]}^{[d]} \quad \Longrightarrow \quad x \ne [2d]$$
Następnie rozwiązujemy równanie:
$$\frac{[a]}{[b]+\log_{[n]}{x}} - \frac{[c]}{\log_{[n]}{x}-[d]} = 0 $$
$$ \Longleftrightarrow \quad \frac{[a](\log_{[n]}{x}-[d])-[c]([b]+\log_{[n]}{x})}{([b]+\log_{[n]}{x})(\log_{[n]}{x}-[d])}= 0 $$
$$\Longleftrightarrow \quad  [a](\log_{[n]}{x}-[d])-[c]([b]+\log_{[n]}{x}) = 0 $$
$$ \Longleftrightarrow \quad [a]\log_{[n]}{x}-[e] - [f]-[c]\log_{[n]}{x}= 0 \quad \Longleftrightarrow \quad [g]\log_{[n]}{x}-[h]=0$$
$$ \Longleftrightarrow \quad \log_{[n]}{x}=[w] \quad \Longleftrightarrow \quad x = {[n]}^{[w]} = [x]$$
\rozwStop
\odpStart
$x = [x]$
\odpStop
\testStart
A. $x = [2d]$\\
B. $x = - 1$\\
C. $x = [x]$\\
D. $x =[p]$\\
E. $x = [w]$
\testStop
\kluczStart
C
\kluczStop



\end{document}