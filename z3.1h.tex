\documentclass[12pt, a4paper]{article}
\usepackage[utf8]{inputenc}
\usepackage{polski}

\usepackage{amsthm}  %pakiet do tworzenia twierdzeń itp.
\usepackage{amsmath} %pakiet do niektórych symboli matematycznych
\usepackage{amssymb} %pakiet do symboli mat., np. \nsubseteq
\usepackage{amsfonts}
\usepackage{graphicx} %obsługa plików graficznych z rozszerzeniem png, jpg
\theoremstyle{definition} %styl dla definicji
\newtheorem{zad}{} 
\title{Multizestaw zadań}
\author{Robert Fidytek}
%\date{\today}
\date{}
\newcounter{liczniksekcji}
\newcommand{\kategoria}[1]{\section{#1}} %olreślamy nazwę kateforii zadań
\newcommand{\zadStart}[1]{\begin{zad}#1\newline} %oznaczenie początku zadania
\newcommand{\zadStop}{\end{zad}}   %oznaczenie końca zadania
%Makra opcjonarne (nie muszą występować):
\newcommand{\rozwStart}[2]{\noindent \textbf{Rozwiązanie (autor #1 , recenzent #2): }\newline} %oznaczenie początku rozwiązania, opcjonarnie można wprowadzić informację o autorze rozwiązania zadania i recenzencie poprawności wykonania rozwiązania zadania
\newcommand{\rozwStop}{\newline}                                            %oznaczenie końca rozwiązania
\newcommand{\odpStart}{\noindent \textbf{Odpowiedź:}\newline}    %oznaczenie początku odpowiedzi końcowej (wypisanie wyniku)
\newcommand{\odpStop}{\newline}                                             %oznaczenie końca odpowiedzi końcowej (wypisanie wyniku)
\newcommand{\testStart}{\noindent \textbf{Test:}\newline} %ewentualne możliwe opcje odpowiedzi testowej: A. ? B. ? C. ? D. ? itd.
\newcommand{\testStop}{\newline} %koniec wprowadzania odpowiedzi testowych
\newcommand{\kluczStart}{\noindent \textbf{Test poprawna odpowiedź:}\newline} %klucz, poprawna odpowiedź pytania testowego (jedna literka): A lub B lub C lub D itd.
\newcommand{\kluczStop}{\newline} %koniec poprawnej odpowiedzi pytania testowego 
\newcommand{\wstawGrafike}[2]{\begin{figure}[h] \includegraphics[scale=#2] {#1} \end{figure}} %gdyby była potrzeba wstawienia obrazka, parametry: nazwa pliku, skala (jak nie wiesz co wpisać, to wpisz 1)

\begin{document}
\maketitle


\kategoria{Wikieł/Z3.1h}
\zadStart{Zadanie z Wikieł Z 3.1 h) moja wersja nr 1}
%[a]:[1, 2, 3, 4, 5, 6]
%[b]:[1, 2, 3, 4, 5, 6]
%[c]:[1, 2, 3, 4, 5, 6]
%[a]=random.randint(2,30)
%[b]=random.randint(2,30)
%[c]=random.randint(2,6)
%[ab]=[a]/[b]
%[abc]=pow([ab],[c])
%[abc1]=round([abc]-1,2)
%[a]<[b] and [abc1]!=(-1)
Zbadać monotoniczność ciągu $a_{n}$.\\ $a_{n}=\big(\frac{[a]}{[b]}\big)^{[c]n}$
\zadStop
\rozwStart{Jakub Ulrych}{}
$$a_{n+1}=\bigg(\frac{[a]}{[b]}\bigg)^{[c](n+1)}=\bigg(\frac{[a]}{[b]}\bigg)^{[c]n+[c]}$$
$$a_{n+1}-a_{n}=\bigg(\frac{[a]}{[b]}\bigg)^{[c]n+[c]}-\bigg(\frac{[a]}{[b]}\bigg)^{[c]n}$$
$$\bigg(\frac{[a]}{[b]}\bigg)^{[c]n}\cdot\bigg(\frac{[a]}{[b]}\bigg)^{[c]}-\bigg(\frac{[a]}{[b]}\bigg)^{[c]n}$$
$$\bigg(\frac{[a]}{[b]}\bigg)^{[c]n}\bigg(\bigg(\frac{[a]}{[b]}\bigg)^{[c]}-1\bigg)$$
$$\bigg(\frac{[a]}{[b]}\bigg)^{[c]n}\cdot ([abc1])$$
dla każdego $n\in\mathbb{N}$ zachodzi
$$\bigg(\frac{[a]}{[b]}\bigg)^{[c]n}\cdot ([abc1])<0\Rightarrow \text{Ciąg jest malejący}$$
\rozwStop
\odpStart
$$\text{Ciąg jest malejący}$$
\odpStop
\testStart
A.$\text{Ciąg jest malejący}$
B.$\text{Ciąg jest rosnący}$
\testStop
\kluczStart
A
\kluczStop



\end{document}