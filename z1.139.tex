\documentclass[12pt, a4paper]{article}
\usepackage[utf8]{inputenc}
\usepackage{polski}

\usepackage{amsthm}  %pakiet do tworzenia twierdzeń itp.
\usepackage{amsmath} %pakiet do niektórych symboli matematycznych
\usepackage{amssymb} %pakiet do symboli mat., np. \nsubseteq
\usepackage{amsfonts}
\usepackage{graphicx} %obsługa plików graficznych z rozszerzeniem png, jpg
\theoremstyle{definition} %styl dla definicji
\newtheorem{zad}{} 
\title{Multizestaw zadań}
\author{Robert Fidytek}
%\date{\today}
\date{}
\newcounter{liczniksekcji}
\newcommand{\kategoria}[1]{\section{#1}} %olreślamy nazwę kateforii zadań
\newcommand{\zadStart}[1]{\begin{zad}#1\newline} %oznaczenie początku zadania
\newcommand{\zadStop}{\end{zad}}   %oznaczenie końca zadania
%Makra opcjonarne (nie muszą występować):
\newcommand{\rozwStart}[2]{\noindent \textbf{Rozwiązanie (autor #1 , recenzent #2): }\newline} %oznaczenie początku rozwiązania, opcjonarnie można wprowadzić informację o autorze rozwiązania zadania i recenzencie poprawności wykonania rozwiązania zadania
\newcommand{\rozwStop}{\newline}                                            %oznaczenie końca rozwiązania
\newcommand{\odpStart}{\noindent \textbf{Odpowiedź:}\newline}    %oznaczenie początku odpowiedzi końcowej (wypisanie wyniku)
\newcommand{\odpStop}{\newline}                                             %oznaczenie końca odpowiedzi końcowej (wypisanie wyniku)
\newcommand{\testStart}{\noindent \textbf{Test:}\newline} %ewentualne możliwe opcje odpowiedzi testowej: A. ? B. ? C. ? D. ? itd.
\newcommand{\testStop}{\newline} %koniec wprowadzania odpowiedzi testowych
\newcommand{\kluczStart}{\noindent \textbf{Test poprawna odpowiedź:}\newline} %klucz, poprawna odpowiedź pytania testowego (jedna literka): A lub B lub C lub D itd.
\newcommand{\kluczStop}{\newline} %koniec poprawnej odpowiedzi pytania testowego 
\newcommand{\wstawGrafike}[2]{\begin{figure}[h] \includegraphics[scale=#2] {#1} \end{figure}} %gdyby była potrzeba wstawienia obrazka, parametry: nazwa pliku, skala (jak nie wiesz co wpisać, to wpisz 1)

\begin{document}
\maketitle


\kategoria{Wikieł/Z1.139}
\zadStart{Zadanie z Wikieł Z 1.139 moja wersja nr [nrWersji]}
%[a]:[2,4,6,8,10,12,14,16,18,20]
%[c]:[1,2,3,4,5,6,7,8,9,10,11,12,13,14,15,16]
%[b]:[1,2,3,4,5,6,7,8,9,10,11,12,13,14,15,16]
%[d]=int([a]/2)
%[e]=pow([d],2)-[b]
%[f]=[c]-pow([d],2)
%[e]>0 and [f]>0 and (-([d]**2)+[a]*[d]-[b])==([d]**2-[a]*[d]+[c])
Znaleźć funkcję odwrotną do funkcji:
$$
f(x) = \left\{ \begin{array}{ll}
-x^2+[a]x-[b] & \textrm{dla $x<[d]$}\\
x^2-[a]x+[c] & \textrm{dla $x \ge [d]$}
\end{array} \right.
$$
\zadStop
\rozwStart{Małgorzata Ugowska}{}
dla $x<[d]$:
$$y=-x^2+[a]x-[b] $$
$$y=-(x^2-[a]x+[b]) $$
$$y = -((x-[d])^2-[e])$$
$$-y = (x-[d])^2-[e]$$
$$-y + [e]= ([d]-x)^2$$
$$\sqrt{-y + [e]}= [d]-x$$
$$x = [d]- \sqrt{[e]-y}$$
$$f^{-1}(x) = [d]- \sqrt{[e]-x}$$
Dziedzina: $x \in (-\infty, [e]]$\\
dla $x\ge [d]$:
$$y=x^2-[a]x+[c] $$
$$y = (x-[d])^2+[f]$$
$$y -[f]= (x-[d])^2$$
$$\sqrt{y - [f]}= x-[d]$$
$$x = [d]+\sqrt{y - [f]}$$
$$f^{-1}(x) = [d]+\sqrt{x - [f]}$$
Dziedzina: $x \in [[f], \infty)$\\
Ostatecznie:
$$
f^{-1}(x) = \left\{ \begin{array}{ll}
[d] - \sqrt{[e]-x} & \textrm{dla $x<[e]$}\\
\sqrt{x - [f]}+[d] & \textrm{dla $x \ge [f]$}
\end{array} \right.
$$
\rozwStop
\odpStart
$$
f^{-1}(x) = \left\{ \begin{array}{ll}
[d] - \sqrt{[e]-x} & \textrm{dla $x<[e]$}\\
\sqrt{x - [f]}+[d] & \textrm{dla $x \ge [f]$}
\end{array} \right.
$$
\odpStop
\testStart
A. $
f^{-1}(x) = \left\{ \begin{array}{ll}
\sqrt{-x + [d]}+[e] & \textrm{dla $x>[e]$}\\
\sqrt{x - [d]}+[f] & \textrm{dla $x \le [f]$}
\end{array} \right.
$\\
B. $
f^{-1}(x) = \left\{ \begin{array}{ll}
\sqrt{-x + [e]}-[d] & \textrm{dla $x<[e]$}\\
\sqrt{x - [f]}+[d] & \textrm{dla $x \ge [f]$}
\end{array} \right.
$\\
C. $
f^{-1}(x) = \left\{ \begin{array}{ll}
\sqrt{-x + [e]}+[d] & \textrm{dla $x<[e]$}\\
\sqrt{x - [f]}+[d] & \textrm{dla $x \ge [f]$}
\end{array} \right.
$\\
D. $
f^{-1}(x) = \left\{ \begin{array}{ll}
\sqrt{-x + [d]}+[e] & \textrm{dla $x<[e]$}\\
\sqrt{x - [d]}-[f] & \textrm{dla $x \ge [f]$}
\end{array} \right.
$\\
E. $
f^{-1}(x) = \left\{ \begin{array}{ll}
\frac{\sqrt{-x + [e]}}{[d]} & \textrm{dla $x>[e]$}\\
\frac{\sqrt{x - [f]}}{[d]} & \textrm{dla $x \le [f]$}
\end{array} \right.
$\\
F. $
f^{-1}(x) = \left\{ \begin{array}{ll}
[d] - \sqrt{[e]-x} & \textrm{dla $x<[e]$}\\
\sqrt{x - [f]}+[d] & \textrm{dla $x \ge [f]$}
\end{array} \right.
$
\testStop
\kluczStart
F
\kluczStop



\end{document}