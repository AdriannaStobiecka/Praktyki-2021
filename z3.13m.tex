\documentclass[12pt, a4paper]{article}
\usepackage[utf8]{inputenc}
\usepackage{polski}

\usepackage{amsthm}  %pakiet do tworzenia twierdzeń itp.
\usepackage{amsmath} %pakiet do niektórych symboli matematycznych
\usepackage{amssymb} %pakiet do symboli mat., np. \nsubseteq
\usepackage{amsfonts}
\usepackage{graphicx} %obsługa plików graficznych z rozszerzeniem png, jpg
\theoremstyle{definition} %styl dla definicji
\newtheorem{zad}{} 
\title{Multizestaw zadań}
\author{Robert Fidytek}
%\date{\today}
\date{}
\newcounter{liczniksekcji}
\newcommand{\kategoria}[1]{\section{#1}} %olreślamy nazwę kateforii zadań
\newcommand{\zadStart}[1]{\begin{zad}#1\newline} %oznaczenie początku zadania
\newcommand{\zadStop}{\end{zad}}   %oznaczenie końca zadania
%Makra opcjonarne (nie muszą występować):
\newcommand{\rozwStart}[2]{\noindent \textbf{Rozwiązanie (autor #1 , recenzent #2): }\newline} %oznaczenie początku rozwiązania, opcjonarnie można wprowadzić informację o autorze rozwiązania zadania i recenzencie poprawności wykonania rozwiązania zadania
\newcommand{\rozwStop}{\newline}                                            %oznaczenie końca rozwiązania
\newcommand{\odpStart}{\noindent \textbf{Odpowiedź:}\newline}    %oznaczenie początku odpowiedzi końcowej (wypisanie wyniku)
\newcommand{\odpStop}{\newline}                                             %oznaczenie końca odpowiedzi końcowej (wypisanie wyniku)
\newcommand{\testStart}{\noindent \textbf{Test:}\newline} %ewentualne możliwe opcje odpowiedzi testowej: A. ? B. ? C. ? D. ? itd.
\newcommand{\testStop}{\newline} %koniec wprowadzania odpowiedzi testowych
\newcommand{\kluczStart}{\noindent \textbf{Test poprawna odpowiedź:}\newline} %klucz, poprawna odpowiedź pytania testowego (jedna literka): A lub B lub C lub D itd.
\newcommand{\kluczStop}{\newline} %koniec poprawnej odpowiedzi pytania testowego 
\newcommand{\wstawGrafike}[2]{\begin{figure}[h] \includegraphics[scale=#2] {#1} \end{figure}} %gdyby była potrzeba wstawienia obrazka, parametry: nazwa pliku, skala (jak nie wiesz co wpisać, to wpisz 1)

\begin{document}
\maketitle


\kategoria{Wikieł/Z3.13m}
\zadStart{Zadanie z Wikieł Z 3.13 m) moja wersja nr [nrWersji]}
%[a1]:[2,3,4,5,6,7,8]
%[a]=[a1]*[a1]
%[b]:[3,5,7,9,11,13]
%[c]:[2,3,4,5,6,7,8,9,10,11,12,13,14,15]
%[d]=random.randint(1,30)
%[e]=[b]-1
%[f]=2*[a1]
%[d1]=int([d]/(math.gcd([d],[f])))
%[f1]=int([f]/(math.gcd([d],[f])))
%[a]!=[b] and [f1]!=1
Obliczyć granicę ciągu $a_n= \big{(}\sqrt{[a]n^{[b]}+[c]n+[d]} - \sqrt{[a]n^{[b]}+[c]n}\big{)}$.
\zadStop
\rozwStart{Barbara Bączek}{}
$$\lim_{n \rightarrow \infty} a_n= \lim_{n \rightarrow \infty}  \big{(}\sqrt{[a]n^{[b]}+[c]n+[d]} - \sqrt{[a]n^{[b]}+[c]n}\big{)}= $$
$$\lim_{n \rightarrow \infty}  \frac{\big{(}\sqrt{[a]n^{[b]}+[c]n+[d]} - \sqrt{[a]n^{[b]}+[c]n}\big{)}\cdot \big{(} \sqrt{[a]n^{[b]}+[c]n+[d]} + \sqrt{[a]n^{[b]}+[c]n} \big{)}}{\sqrt{[a]n^{[b]}+[c]n+[d]} + \sqrt{[a]n^{[b]}+[c]n}}=$$
$$\lim_{n \rightarrow \infty}  \frac{[a]n^{[b]}+[c]n+[d]-[a]n^{[b]}-[c]n}{\sqrt{[a]n^{[b]}+[c]n+[d]} + \sqrt{[a]n^{[b]}+[c]n}}=$$
$$ \lim_{n \rightarrow \infty}  \frac{[d]}{n^{\frac{[b]}{2}}\Big{(}\sqrt{[a]+\frac{[c]}{n^{[e]}}+\frac{[d]}{n^{[b]}}} + \sqrt{[a]+\frac{[c]}{n^{[e]}}}\Big{)}}=\lim_{n \rightarrow \infty} \frac{[d1]}{[f1]n^{\frac{[b]}{2}}}=0$$
\rozwStop
\odpStart
$0$
\odpStop
\testStart
A.$\infty$
B.$[b]$
C.$-\infty$
D.$0$
E.$-[a]$
G.$1$
H.$-1$
\testStop
\kluczStart
D
\kluczStop



\end{document}