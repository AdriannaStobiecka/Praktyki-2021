\documentclass[12pt, a4paper]{article}
\usepackage[utf8]{inputenc}
\usepackage{polski}

\usepackage{amsthm}  %pakiet do tworzenia twierdzeń itp.
\usepackage{amsmath} %pakiet do niektórych symboli matematycznych
\usepackage{amssymb} %pakiet do symboli mat., np. \nsubseteq
\usepackage{amsfonts}
\usepackage{graphicx} %obsługa plików graficznych z rozszerzeniem png, jpg
\theoremstyle{definition} %styl dla definicji
\newtheorem{zad}{} 
\title{Multizestaw zadań}
\author{Robert Fidytek}
%\date{\today}
\date{}
\newcounter{liczniksekcji}
\newcommand{\kategoria}[1]{\section{#1}} %olreślamy nazwę kateforii zadań
\newcommand{\zadStart}[1]{\begin{zad}#1\newline} %oznaczenie początku zadania
\newcommand{\zadStop}{\end{zad}}   %oznaczenie końca zadania
%Makra opcjonarne (nie muszą występować):
\newcommand{\rozwStart}[2]{\noindent \textbf{Rozwiązanie (autor #1 , recenzent #2): }\newline} %oznaczenie początku rozwiązania, opcjonarnie można wprowadzić informację o autorze rozwiązania zadania i recenzencie poprawności wykonania rozwiązania zadania
\newcommand{\rozwStop}{\newline}                                            %oznaczenie końca rozwiązania
\newcommand{\odpStart}{\noindent \textbf{Odpowiedź:}\newline}    %oznaczenie początku odpowiedzi końcowej (wypisanie wyniku)
\newcommand{\odpStop}{\newline}                                             %oznaczenie końca odpowiedzi końcowej (wypisanie wyniku)
\newcommand{\testStart}{\noindent \textbf{Test:}\newline} %ewentualne możliwe opcje odpowiedzi testowej: A. ? B. ? C. ? D. ? itd.
\newcommand{\testStop}{\newline} %koniec wprowadzania odpowiedzi testowych
\newcommand{\kluczStart}{\noindent \textbf{Test poprawna odpowiedź:}\newline} %klucz, poprawna odpowiedź pytania testowego (jedna literka): A lub B lub C lub D itd.
\newcommand{\kluczStop}{\newline} %koniec poprawnej odpowiedzi pytania testowego 
\newcommand{\wstawGrafike}[2]{\begin{figure}[h] \includegraphics[scale=#2] {#1} \end{figure}} %gdyby była potrzeba wstawienia obrazka, parametry: nazwa pliku, skala (jak nie wiesz co wpisać, to wpisz 1)

\begin{document}
\maketitle



\kategoria{Wikieł/Z1.14d}
\zadStart{Zadanie z Wikieł Z 1.14 d) moja wersja nr [nrWersji]}
%[p1]:[2,3,4,5,6,7,8]
%[p2]:[2,3,4,5,6]
%[p3]:[4,5,6,7,8,9,10,11]
%[p4]:[2,3,4,5,6]
%[a]=random.randint(2,10)
%[e]=random.randint(2,10)
%[c]=random.randint(1,10)
%[d]=random.randint(2,10)
%[b]=random.randint(2,10)
%[f]=random.randint(1,10)
%[g]=2*[a]
%[p3p1m]=[p3]-[p1]
%[p3p2m]=[p3]-[p2]
%[p3p4m]=[p3]-[p4]
%[p3]>[p1] and [p3]>[p4] and [p1]>[p2] and math.gcd([a],[d])==1 and [p3p1m]>1 and [p3p2m]>1 and [p3p4m]>1
Rozwiązać rówanie $\sqrt{{(3x-[a])}^2}=[a]-3x$
\zadStop
\rozwStart{Pascal Nawrocki}{}
Zauważmy, że dziedzina wynosi $\mathbb{R}$, ze względu na to, że wartość pod pierwiastkiem zawsze będzie nieujemna nieważne jakiego x podstawimy.
Na samym początku korzystamy z własności potęgi pod pierwiastkiem: $$\sqrt{a^2}=|a|$$
Czyli:
$$\sqrt{{(3x-[a])}^2}=[a]-3x \Leftrightarrow |3x-[a]|=[a]-3x $$
Rozwiązujemy standardowo, na dwa przypadki:
\begin{enumerate}
\item$x\in(-\infty,\frac{[a]}{3})$
\item$x\in[\frac{[a]}{3},+\infty)$
\end{enumerate}
Rozwiązujemy:
\begin{enumerate}
\item$x\in(-\infty,\frac{[a]}{3})$
$$|3x-[a]|=[a]-3x\Leftrightarrow-3x+[a]=[a]-3x\Leftrightarrow 0=0$$
A to oznacza, że mamy tożsamość dla wszystkich x z przedziału $(-\infty,\frac{[a]}{3})$, czyli rozwiązaniem w tym przypadku jest cały ten przedział.
\item$x\in[\frac{[a]}{3},+\infty)$
$$|3x-[a]|=[a]-3x\Leftrightarrow3x-[a]=[a]-3x\Leftrightarrow6x=[g]\Leftrightarrow x=\frac{[a]}{3}$$
Rozwiązaniem w tym wypadku jest $x=\frac{[a]}{3}$
\end{enumerate}
Stąd możemy podsumować czyli przypadek $1\cup2$ dają nam przedział $x\in(-\infty,\frac{[a]}{3}]$, upraszczając: $x\leq\frac{[a]}{3}$.
\odpStart
$x\leq\frac{[a]}{3}$
\odpStop
\testStart
A.$x\in\frac{[a]}{3}$
B.$x\in \emptyset$
C.$\infty$
D. $x\in (-\infty,-[a])$
E. $x\in (-\infty,-[a])\cup x\in[-[a],+\infty)$
F.$\frac{1}{[d]}$
G.$[d]$
H.$-[a]$
I.$-\infty$
\testStop
\kluczStart
A
\kluczStop


\end{document}