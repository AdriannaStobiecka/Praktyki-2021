\documentclass[12pt, a4paper]{article}
\usepackage[utf8]{inputenc}
\usepackage{polski}

\usepackage{amsthm}  %pakiet do tworzenia twierdzeń itp.
\usepackage{amsmath} %pakiet do niektórych symboli matematycznych
\usepackage{amssymb} %pakiet do symboli mat., np. \nsubseteq
\usepackage{amsfonts}
\usepackage{graphicx} %obsługa plików graficznych z rozszerzeniem png, jpg
\theoremstyle{definition} %styl dla definicji
\newtheorem{zad}{} 
\title{Multizestaw zadań}
\author{Robert Fidytek}
%\date{\today}
\date{}
\newcounter{liczniksekcji}
\newcommand{\kategoria}[1]{\section{#1}} %olreślamy nazwę kateforii zadań
\newcommand{\zadStart}[1]{\begin{zad}#1\newline} %oznaczenie początku zadania
\newcommand{\zadStop}{\end{zad}}   %oznaczenie końca zadania
%Makra opcjonarne (nie muszą występować):
\newcommand{\rozwStart}[2]{\noindent \textbf{Rozwiązanie (autor #1 , recenzent #2): }\newline} %oznaczenie początku rozwiązania, opcjonarnie można wprowadzić informację o autorze rozwiązania zadania i recenzencie poprawności wykonania rozwiązania zadania
\newcommand{\rozwStop}{\newline}                                            %oznaczenie końca rozwiązania
\newcommand{\odpStart}{\noindent \textbf{Odpowiedź:}\newline}    %oznaczenie początku odpowiedzi końcowej (wypisanie wyniku)
\newcommand{\odpStop}{\newline}                                             %oznaczenie końca odpowiedzi końcowej (wypisanie wyniku)
\newcommand{\testStart}{\noindent \textbf{Test:}\newline} %ewentualne możliwe opcje odpowiedzi testowej: A. ? B. ? C. ? D. ? itd.
\newcommand{\testStop}{\newline} %koniec wprowadzania odpowiedzi testowych
\newcommand{\kluczStart}{\noindent \textbf{Test poprawna odpowiedź:}\newline} %klucz, poprawna odpowiedź pytania testowego (jedna literka): A lub B lub C lub D itd.
\newcommand{\kluczStop}{\newline} %koniec poprawnej odpowiedzi pytania testowego 
\newcommand{\wstawGrafike}[2]{\begin{figure}[h] \includegraphics[scale=#2] {#1} \end{figure}} %gdyby była potrzeba wstawienia obrazka, parametry: nazwa pliku, skala (jak nie wiesz co wpisać, to wpisz 1)

\begin{document}
\maketitle


\kategoria{Wikieł/Z1.93a}
\zadStart{Zadanie z Wikieł Z 1.93 a) moja wersja nr [nrWersji]}
%[b]:[2,3,4,5,6,7,8,9,10,11,12,13,14,15]
%[p]=random.randint(1,9)
%[b2]=[b]+1
%[b3]=(pow([b2],1/2))
%[b3].is_integer()!=True
Rozwiązać równanie $\log_{[p]}{(x+\sqrt{[b]})}=-\log_{[p]}{(x-\sqrt{[b]})} $
\zadStop
\rozwStart{Małgorzata Ugowska}{}
$$\log_{[p]}{(x+\sqrt{[b]})}=-\log_{[p]}{(x-\sqrt{[b]})} \quad \Longleftrightarrow \quad \log_{[p]}{(x+\sqrt{[b]})}+\log_{[p]}{(x-\sqrt{[b]})}=0 $$
$$\Longleftrightarrow \quad \log_{[p]}{(x+\sqrt{[b]})(x-\sqrt{[b]})}=0 \quad \Longleftrightarrow \quad [p]^0=x^2-[b] \Longleftrightarrow \quad x^2-[b] = 1 $$
$$\quad \Longleftrightarrow \quad x^2-[b2]=0 \quad \Longleftrightarrow \quad (x-\sqrt{[b2]})(x+\sqrt{[b2]})=0$$
\rozwStop
\odpStart
$x \in \{-\sqrt{[b2]}, \sqrt{[b2]}\}$
\odpStop
\testStart
A. $x \in \{-\sqrt{[b2]}, \sqrt{[b2]}\}$
B. $x \in \{-\sqrt{[b]}, \sqrt{[b]}\}$
C. 0
D. $x \in \{-[b2], [b2]\}$
E. $[p]$
\testStop
\kluczStart
A
\kluczStop



\end{document}