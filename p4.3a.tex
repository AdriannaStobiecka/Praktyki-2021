\documentclass[12pt, a4paper]{article}
\usepackage[utf8]{inputenc}
\usepackage{polski}
\usepackage{amsthm}  %pakiet do tworzenia twierdzeń itp.
\usepackage{amsmath} %pakiet do niektórych symboli matematycznych
\usepackage{amssymb} %pakiet do symboli mat., np. \nsubseteq
\usepackage{amsfonts}
\usepackage{graphicx} %obsługa plików graficznych z rozszerzeniem png, jpg
\theoremstyle{definition} %styl dla definicji
\newtheorem{zad}{} 
\title{Multizestaw zadań}
\author{Robert Fidytek}
%\date{\today}
\date{}
\newcommand{\kategoria}[1]{\section{#1}}
\newcommand{\zadStart}[1]{\begin{zad}#1\newline}
\newcommand{\zadStop}{\end{zad}}
\newcommand{\rozwStart}[2]{\noindent \textbf{Rozwiązanie (autor #1 , recenzent #2): }\newline}
\newcommand{\rozwStop}{\newline}                                           
\newcommand{\odpStart}{\noindent \textbf{Odpowiedź:}\newline}
\newcommand{\odpStop}{\newline}
\newcommand{\testStart}{\noindent \textbf{Test:}\newline}
\newcommand{\testStop}{\newline}
\newcommand{\kluczStart}{\noindent \textbf{Test poprawna odpowiedź:}\newline}
\newcommand{\kluczStop}{\newline}
\newcommand{\wstawGrafike}[2]{\begin{figure}[h] \includegraphics[scale=#2] {#1} \end{figure}}

\begin{document}
\maketitle

\kategoria{Wikieł/P4.3a}


\zadStart{Przykład z Wikieł P 4.3a moja wersja nr 1}


Obliczyć granicę funkcji $\lim\limits_{x\to\ 0}\frac{4 \cdot x}{tan(3 \cdot x)}$.
\zadStop
\rozwStart{Patryk Wirkus}{}
$$\lim\limits_{x\to\ 0}\frac{4 \cdot x}{tan(3 \cdot x)}=\lim\limits_{x\to\ 0}\frac{4 \cdot x \cdot cos(3 \cdot x)}{sin(3 \cdot x)}=\lim\limits_{x\to\ 0}\frac{4 \cdot cos(3 \cdot x)}{\frac{sin(3 \cdot x)}{x}}=\lim\limits_{x\to\ 0}\frac{4 \cdot cos(3 \cdot x)}{3 \cdot \frac{sin(3 \cdot x)}{3 \cdot x}} = \frac{4}{3}$$
\rozwStop
\odpStart
$\frac{4}{3}$
\odpStop
\testStart
A.$\frac{4}{3}$
B.$\infty$
C.$-\infty$
D.$0$
E.$-\frac{4}{3}$
F.$\frac{3}{4}$
G.$-\frac{3}{4}$
H.$3$
I.$4$
\testStop
\kluczStart
A
\kluczStop



\zadStart{Przykład z Wikieł P 4.3a moja wersja nr 2}


Obliczyć granicę funkcji $\lim\limits_{x\to\ 0}\frac{4 \cdot x}{tan(5 \cdot x)}$.
\zadStop
\rozwStart{Patryk Wirkus}{}
$$\lim\limits_{x\to\ 0}\frac{4 \cdot x}{tan(5 \cdot x)}=\lim\limits_{x\to\ 0}\frac{4 \cdot x \cdot cos(5 \cdot x)}{sin(5 \cdot x)}=\lim\limits_{x\to\ 0}\frac{4 \cdot cos(5 \cdot x)}{\frac{sin(5 \cdot x)}{x}}=\lim\limits_{x\to\ 0}\frac{4 \cdot cos(5 \cdot x)}{5 \cdot \frac{sin(5 \cdot x)}{5 \cdot x}} = \frac{4}{5}$$
\rozwStop
\odpStart
$\frac{4}{5}$
\odpStop
\testStart
A.$\frac{4}{5}$
B.$\infty$
C.$-\infty$
D.$0$
E.$-\frac{4}{5}$
F.$\frac{5}{4}$
G.$-\frac{5}{4}$
H.$5$
I.$4$
\testStop
\kluczStart
A
\kluczStop



\zadStart{Przykład z Wikieł P 4.3a moja wersja nr 3}


Obliczyć granicę funkcji $\lim\limits_{x\to\ 0}\frac{4 \cdot x}{tan(7 \cdot x)}$.
\zadStop
\rozwStart{Patryk Wirkus}{}
$$\lim\limits_{x\to\ 0}\frac{4 \cdot x}{tan(7 \cdot x)}=\lim\limits_{x\to\ 0}\frac{4 \cdot x \cdot cos(7 \cdot x)}{sin(7 \cdot x)}=\lim\limits_{x\to\ 0}\frac{4 \cdot cos(7 \cdot x)}{\frac{sin(7 \cdot x)}{x}}=\lim\limits_{x\to\ 0}\frac{4 \cdot cos(7 \cdot x)}{7 \cdot \frac{sin(7 \cdot x)}{7 \cdot x}} = \frac{4}{7}$$
\rozwStop
\odpStart
$\frac{4}{7}$
\odpStop
\testStart
A.$\frac{4}{7}$
B.$\infty$
C.$-\infty$
D.$0$
E.$-\frac{4}{7}$
F.$\frac{7}{4}$
G.$-\frac{7}{4}$
H.$7$
I.$4$
\testStop
\kluczStart
A
\kluczStop



\zadStart{Przykład z Wikieł P 4.3a moja wersja nr 4}


Obliczyć granicę funkcji $\lim\limits_{x\to\ 0}\frac{4 \cdot x}{tan(9 \cdot x)}$.
\zadStop
\rozwStart{Patryk Wirkus}{}
$$\lim\limits_{x\to\ 0}\frac{4 \cdot x}{tan(9 \cdot x)}=\lim\limits_{x\to\ 0}\frac{4 \cdot x \cdot cos(9 \cdot x)}{sin(9 \cdot x)}=\lim\limits_{x\to\ 0}\frac{4 \cdot cos(9 \cdot x)}{\frac{sin(9 \cdot x)}{x}}=\lim\limits_{x\to\ 0}\frac{4 \cdot cos(9 \cdot x)}{9 \cdot \frac{sin(9 \cdot x)}{9 \cdot x}} = \frac{4}{9}$$
\rozwStop
\odpStart
$\frac{4}{9}$
\odpStop
\testStart
A.$\frac{4}{9}$
B.$\infty$
C.$-\infty$
D.$0$
E.$-\frac{4}{9}$
F.$\frac{9}{4}$
G.$-\frac{9}{4}$
H.$9$
I.$4$
\testStop
\kluczStart
A
\kluczStop



\zadStart{Przykład z Wikieł P 4.3a moja wersja nr 5}


Obliczyć granicę funkcji $\lim\limits_{x\to\ 0}\frac{4 \cdot x}{tan(11 \cdot x)}$.
\zadStop
\rozwStart{Patryk Wirkus}{}
$$\lim\limits_{x\to\ 0}\frac{4 \cdot x}{tan(11 \cdot x)}=\lim\limits_{x\to\ 0}\frac{4 \cdot x \cdot cos(11 \cdot x)}{sin(11 \cdot x)}=\lim\limits_{x\to\ 0}\frac{4 \cdot cos(11 \cdot x)}{\frac{sin(11 \cdot x)}{x}}=\lim\limits_{x\to\ 0}\frac{4 \cdot cos(11 \cdot x)}{11 \cdot \frac{sin(11 \cdot x)}{11 \cdot x}} = \frac{4}{11}$$
\rozwStop
\odpStart
$\frac{4}{11}$
\odpStop
\testStart
A.$\frac{4}{11}$
B.$\infty$
C.$-\infty$
D.$0$
E.$-\frac{4}{11}$
F.$\frac{11}{4}$
G.$-\frac{11}{4}$
H.$11$
I.$4$
\testStop
\kluczStart
A
\kluczStop



\zadStart{Przykład z Wikieł P 4.3a moja wersja nr 6}


Obliczyć granicę funkcji $\lim\limits_{x\to\ 0}\frac{4 \cdot x}{tan(13 \cdot x)}$.
\zadStop
\rozwStart{Patryk Wirkus}{}
$$\lim\limits_{x\to\ 0}\frac{4 \cdot x}{tan(13 \cdot x)}=\lim\limits_{x\to\ 0}\frac{4 \cdot x \cdot cos(13 \cdot x)}{sin(13 \cdot x)}=\lim\limits_{x\to\ 0}\frac{4 \cdot cos(13 \cdot x)}{\frac{sin(13 \cdot x)}{x}}=\lim\limits_{x\to\ 0}\frac{4 \cdot cos(13 \cdot x)}{13 \cdot \frac{sin(13 \cdot x)}{13 \cdot x}} = \frac{4}{13}$$
\rozwStop
\odpStart
$\frac{4}{13}$
\odpStop
\testStart
A.$\frac{4}{13}$
B.$\infty$
C.$-\infty$
D.$0$
E.$-\frac{4}{13}$
F.$\frac{13}{4}$
G.$-\frac{13}{4}$
H.$13$
I.$4$
\testStop
\kluczStart
A
\kluczStop



\zadStart{Przykład z Wikieł P 4.3a moja wersja nr 7}


Obliczyć granicę funkcji $\lim\limits_{x\to\ 0}\frac{4 \cdot x}{tan(15 \cdot x)}$.
\zadStop
\rozwStart{Patryk Wirkus}{}
$$\lim\limits_{x\to\ 0}\frac{4 \cdot x}{tan(15 \cdot x)}=\lim\limits_{x\to\ 0}\frac{4 \cdot x \cdot cos(15 \cdot x)}{sin(15 \cdot x)}=\lim\limits_{x\to\ 0}\frac{4 \cdot cos(15 \cdot x)}{\frac{sin(15 \cdot x)}{x}}=\lim\limits_{x\to\ 0}\frac{4 \cdot cos(15 \cdot x)}{15 \cdot \frac{sin(15 \cdot x)}{15 \cdot x}} = \frac{4}{15}$$
\rozwStop
\odpStart
$\frac{4}{15}$
\odpStop
\testStart
A.$\frac{4}{15}$
B.$\infty$
C.$-\infty$
D.$0$
E.$-\frac{4}{15}$
F.$\frac{15}{4}$
G.$-\frac{15}{4}$
H.$15$
I.$4$
\testStop
\kluczStart
A
\kluczStop



\zadStart{Przykład z Wikieł P 4.3a moja wersja nr 8}


Obliczyć granicę funkcji $\lim\limits_{x\to\ 0}\frac{4 \cdot x}{tan(17 \cdot x)}$.
\zadStop
\rozwStart{Patryk Wirkus}{}
$$\lim\limits_{x\to\ 0}\frac{4 \cdot x}{tan(17 \cdot x)}=\lim\limits_{x\to\ 0}\frac{4 \cdot x \cdot cos(17 \cdot x)}{sin(17 \cdot x)}=\lim\limits_{x\to\ 0}\frac{4 \cdot cos(17 \cdot x)}{\frac{sin(17 \cdot x)}{x}}=\lim\limits_{x\to\ 0}\frac{4 \cdot cos(17 \cdot x)}{17 \cdot \frac{sin(17 \cdot x)}{17 \cdot x}} = \frac{4}{17}$$
\rozwStop
\odpStart
$\frac{4}{17}$
\odpStop
\testStart
A.$\frac{4}{17}$
B.$\infty$
C.$-\infty$
D.$0$
E.$-\frac{4}{17}$
F.$\frac{17}{4}$
G.$-\frac{17}{4}$
H.$17$
I.$4$
\testStop
\kluczStart
A
\kluczStop



\zadStart{Przykład z Wikieł P 4.3a moja wersja nr 9}


Obliczyć granicę funkcji $\lim\limits_{x\to\ 0}\frac{4 \cdot x}{tan(19 \cdot x)}$.
\zadStop
\rozwStart{Patryk Wirkus}{}
$$\lim\limits_{x\to\ 0}\frac{4 \cdot x}{tan(19 \cdot x)}=\lim\limits_{x\to\ 0}\frac{4 \cdot x \cdot cos(19 \cdot x)}{sin(19 \cdot x)}=\lim\limits_{x\to\ 0}\frac{4 \cdot cos(19 \cdot x)}{\frac{sin(19 \cdot x)}{x}}=\lim\limits_{x\to\ 0}\frac{4 \cdot cos(19 \cdot x)}{19 \cdot \frac{sin(19 \cdot x)}{19 \cdot x}} = \frac{4}{19}$$
\rozwStop
\odpStart
$\frac{4}{19}$
\odpStop
\testStart
A.$\frac{4}{19}$
B.$\infty$
C.$-\infty$
D.$0$
E.$-\frac{4}{19}$
F.$\frac{19}{4}$
G.$-\frac{19}{4}$
H.$19$
I.$4$
\testStop
\kluczStart
A
\kluczStop



\zadStart{Przykład z Wikieł P 4.3a moja wersja nr 10}


Obliczyć granicę funkcji $\lim\limits_{x\to\ 0}\frac{4 \cdot x}{tan(21 \cdot x)}$.
\zadStop
\rozwStart{Patryk Wirkus}{}
$$\lim\limits_{x\to\ 0}\frac{4 \cdot x}{tan(21 \cdot x)}=\lim\limits_{x\to\ 0}\frac{4 \cdot x \cdot cos(21 \cdot x)}{sin(21 \cdot x)}=\lim\limits_{x\to\ 0}\frac{4 \cdot cos(21 \cdot x)}{\frac{sin(21 \cdot x)}{x}}=\lim\limits_{x\to\ 0}\frac{4 \cdot cos(21 \cdot x)}{21 \cdot \frac{sin(21 \cdot x)}{21 \cdot x}} = \frac{4}{21}$$
\rozwStop
\odpStart
$\frac{4}{21}$
\odpStop
\testStart
A.$\frac{4}{21}$
B.$\infty$
C.$-\infty$
D.$0$
E.$-\frac{4}{21}$
F.$\frac{21}{4}$
G.$-\frac{21}{4}$
H.$21$
I.$4$
\testStop
\kluczStart
A
\kluczStop



\zadStart{Przykład z Wikieł P 4.3a moja wersja nr 11}


Obliczyć granicę funkcji $\lim\limits_{x\to\ 0}\frac{4 \cdot x}{tan(23 \cdot x)}$.
\zadStop
\rozwStart{Patryk Wirkus}{}
$$\lim\limits_{x\to\ 0}\frac{4 \cdot x}{tan(23 \cdot x)}=\lim\limits_{x\to\ 0}\frac{4 \cdot x \cdot cos(23 \cdot x)}{sin(23 \cdot x)}=\lim\limits_{x\to\ 0}\frac{4 \cdot cos(23 \cdot x)}{\frac{sin(23 \cdot x)}{x}}=\lim\limits_{x\to\ 0}\frac{4 \cdot cos(23 \cdot x)}{23 \cdot \frac{sin(23 \cdot x)}{23 \cdot x}} = \frac{4}{23}$$
\rozwStop
\odpStart
$\frac{4}{23}$
\odpStop
\testStart
A.$\frac{4}{23}$
B.$\infty$
C.$-\infty$
D.$0$
E.$-\frac{4}{23}$
F.$\frac{23}{4}$
G.$-\frac{23}{4}$
H.$23$
I.$4$
\testStop
\kluczStart
A
\kluczStop



\zadStart{Przykład z Wikieł P 4.3a moja wersja nr 12}


Obliczyć granicę funkcji $\lim\limits_{x\to\ 0}\frac{4 \cdot x}{tan(25 \cdot x)}$.
\zadStop
\rozwStart{Patryk Wirkus}{}
$$\lim\limits_{x\to\ 0}\frac{4 \cdot x}{tan(25 \cdot x)}=\lim\limits_{x\to\ 0}\frac{4 \cdot x \cdot cos(25 \cdot x)}{sin(25 \cdot x)}=\lim\limits_{x\to\ 0}\frac{4 \cdot cos(25 \cdot x)}{\frac{sin(25 \cdot x)}{x}}=\lim\limits_{x\to\ 0}\frac{4 \cdot cos(25 \cdot x)}{25 \cdot \frac{sin(25 \cdot x)}{25 \cdot x}} = \frac{4}{25}$$
\rozwStop
\odpStart
$\frac{4}{25}$
\odpStop
\testStart
A.$\frac{4}{25}$
B.$\infty$
C.$-\infty$
D.$0$
E.$-\frac{4}{25}$
F.$\frac{25}{4}$
G.$-\frac{25}{4}$
H.$25$
I.$4$
\testStop
\kluczStart
A
\kluczStop



\zadStart{Przykład z Wikieł P 4.3a moja wersja nr 13}


Obliczyć granicę funkcji $\lim\limits_{x\to\ 0}\frac{4 \cdot x}{tan(27 \cdot x)}$.
\zadStop
\rozwStart{Patryk Wirkus}{}
$$\lim\limits_{x\to\ 0}\frac{4 \cdot x}{tan(27 \cdot x)}=\lim\limits_{x\to\ 0}\frac{4 \cdot x \cdot cos(27 \cdot x)}{sin(27 \cdot x)}=\lim\limits_{x\to\ 0}\frac{4 \cdot cos(27 \cdot x)}{\frac{sin(27 \cdot x)}{x}}=\lim\limits_{x\to\ 0}\frac{4 \cdot cos(27 \cdot x)}{27 \cdot \frac{sin(27 \cdot x)}{27 \cdot x}} = \frac{4}{27}$$
\rozwStop
\odpStart
$\frac{4}{27}$
\odpStop
\testStart
A.$\frac{4}{27}$
B.$\infty$
C.$-\infty$
D.$0$
E.$-\frac{4}{27}$
F.$\frac{27}{4}$
G.$-\frac{27}{4}$
H.$27$
I.$4$
\testStop
\kluczStart
A
\kluczStop



\zadStart{Przykład z Wikieł P 4.3a moja wersja nr 14}


Obliczyć granicę funkcji $\lim\limits_{x\to\ 0}\frac{4 \cdot x}{tan(29 \cdot x)}$.
\zadStop
\rozwStart{Patryk Wirkus}{}
$$\lim\limits_{x\to\ 0}\frac{4 \cdot x}{tan(29 \cdot x)}=\lim\limits_{x\to\ 0}\frac{4 \cdot x \cdot cos(29 \cdot x)}{sin(29 \cdot x)}=\lim\limits_{x\to\ 0}\frac{4 \cdot cos(29 \cdot x)}{\frac{sin(29 \cdot x)}{x}}=\lim\limits_{x\to\ 0}\frac{4 \cdot cos(29 \cdot x)}{29 \cdot \frac{sin(29 \cdot x)}{29 \cdot x}} = \frac{4}{29}$$
\rozwStop
\odpStart
$\frac{4}{29}$
\odpStop
\testStart
A.$\frac{4}{29}$
B.$\infty$
C.$-\infty$
D.$0$
E.$-\frac{4}{29}$
F.$\frac{29}{4}$
G.$-\frac{29}{4}$
H.$29$
I.$4$
\testStop
\kluczStart
A
\kluczStop



\zadStart{Przykład z Wikieł P 4.3a moja wersja nr 15}


Obliczyć granicę funkcji $\lim\limits_{x\to\ 0}\frac{4 \cdot x}{tan(31 \cdot x)}$.
\zadStop
\rozwStart{Patryk Wirkus}{}
$$\lim\limits_{x\to\ 0}\frac{4 \cdot x}{tan(31 \cdot x)}=\lim\limits_{x\to\ 0}\frac{4 \cdot x \cdot cos(31 \cdot x)}{sin(31 \cdot x)}=\lim\limits_{x\to\ 0}\frac{4 \cdot cos(31 \cdot x)}{\frac{sin(31 \cdot x)}{x}}=\lim\limits_{x\to\ 0}\frac{4 \cdot cos(31 \cdot x)}{31 \cdot \frac{sin(31 \cdot x)}{31 \cdot x}} = \frac{4}{31}$$
\rozwStop
\odpStart
$\frac{4}{31}$
\odpStop
\testStart
A.$\frac{4}{31}$
B.$\infty$
C.$-\infty$
D.$0$
E.$-\frac{4}{31}$
F.$\frac{31}{4}$
G.$-\frac{31}{4}$
H.$31$
I.$4$
\testStop
\kluczStart
A
\kluczStop



\zadStart{Przykład z Wikieł P 4.3a moja wersja nr 16}


Obliczyć granicę funkcji $\lim\limits_{x\to\ 0}\frac{4 \cdot x}{tan(33 \cdot x)}$.
\zadStop
\rozwStart{Patryk Wirkus}{}
$$\lim\limits_{x\to\ 0}\frac{4 \cdot x}{tan(33 \cdot x)}=\lim\limits_{x\to\ 0}\frac{4 \cdot x \cdot cos(33 \cdot x)}{sin(33 \cdot x)}=\lim\limits_{x\to\ 0}\frac{4 \cdot cos(33 \cdot x)}{\frac{sin(33 \cdot x)}{x}}=\lim\limits_{x\to\ 0}\frac{4 \cdot cos(33 \cdot x)}{33 \cdot \frac{sin(33 \cdot x)}{33 \cdot x}} = \frac{4}{33}$$
\rozwStop
\odpStart
$\frac{4}{33}$
\odpStop
\testStart
A.$\frac{4}{33}$
B.$\infty$
C.$-\infty$
D.$0$
E.$-\frac{4}{33}$
F.$\frac{33}{4}$
G.$-\frac{33}{4}$
H.$33$
I.$4$
\testStop
\kluczStart
A
\kluczStop



\zadStart{Przykład z Wikieł P 4.3a moja wersja nr 17}


Obliczyć granicę funkcji $\lim\limits_{x\to\ 0}\frac{4 \cdot x}{tan(35 \cdot x)}$.
\zadStop
\rozwStart{Patryk Wirkus}{}
$$\lim\limits_{x\to\ 0}\frac{4 \cdot x}{tan(35 \cdot x)}=\lim\limits_{x\to\ 0}\frac{4 \cdot x \cdot cos(35 \cdot x)}{sin(35 \cdot x)}=\lim\limits_{x\to\ 0}\frac{4 \cdot cos(35 \cdot x)}{\frac{sin(35 \cdot x)}{x}}=\lim\limits_{x\to\ 0}\frac{4 \cdot cos(35 \cdot x)}{35 \cdot \frac{sin(35 \cdot x)}{35 \cdot x}} = \frac{4}{35}$$
\rozwStop
\odpStart
$\frac{4}{35}$
\odpStop
\testStart
A.$\frac{4}{35}$
B.$\infty$
C.$-\infty$
D.$0$
E.$-\frac{4}{35}$
F.$\frac{35}{4}$
G.$-\frac{35}{4}$
H.$35$
I.$4$
\testStop
\kluczStart
A
\kluczStop



\zadStart{Przykład z Wikieł P 4.3a moja wersja nr 18}


Obliczyć granicę funkcji $\lim\limits_{x\to\ 0}\frac{4 \cdot x}{tan(37 \cdot x)}$.
\zadStop
\rozwStart{Patryk Wirkus}{}
$$\lim\limits_{x\to\ 0}\frac{4 \cdot x}{tan(37 \cdot x)}=\lim\limits_{x\to\ 0}\frac{4 \cdot x \cdot cos(37 \cdot x)}{sin(37 \cdot x)}=\lim\limits_{x\to\ 0}\frac{4 \cdot cos(37 \cdot x)}{\frac{sin(37 \cdot x)}{x}}=\lim\limits_{x\to\ 0}\frac{4 \cdot cos(37 \cdot x)}{37 \cdot \frac{sin(37 \cdot x)}{37 \cdot x}} = \frac{4}{37}$$
\rozwStop
\odpStart
$\frac{4}{37}$
\odpStop
\testStart
A.$\frac{4}{37}$
B.$\infty$
C.$-\infty$
D.$0$
E.$-\frac{4}{37}$
F.$\frac{37}{4}$
G.$-\frac{37}{4}$
H.$37$
I.$4$
\testStop
\kluczStart
A
\kluczStop



\zadStart{Przykład z Wikieł P 4.3a moja wersja nr 19}


Obliczyć granicę funkcji $\lim\limits_{x\to\ 0}\frac{4 \cdot x}{tan(39 \cdot x)}$.
\zadStop
\rozwStart{Patryk Wirkus}{}
$$\lim\limits_{x\to\ 0}\frac{4 \cdot x}{tan(39 \cdot x)}=\lim\limits_{x\to\ 0}\frac{4 \cdot x \cdot cos(39 \cdot x)}{sin(39 \cdot x)}=\lim\limits_{x\to\ 0}\frac{4 \cdot cos(39 \cdot x)}{\frac{sin(39 \cdot x)}{x}}=\lim\limits_{x\to\ 0}\frac{4 \cdot cos(39 \cdot x)}{39 \cdot \frac{sin(39 \cdot x)}{39 \cdot x}} = \frac{4}{39}$$
\rozwStop
\odpStart
$\frac{4}{39}$
\odpStop
\testStart
A.$\frac{4}{39}$
B.$\infty$
C.$-\infty$
D.$0$
E.$-\frac{4}{39}$
F.$\frac{39}{4}$
G.$-\frac{39}{4}$
H.$39$
I.$4$
\testStop
\kluczStart
A
\kluczStop



\zadStart{Przykład z Wikieł P 4.3a moja wersja nr 20}


Obliczyć granicę funkcji $\lim\limits_{x\to\ 0}\frac{5 \cdot x}{tan(2 \cdot x)}$.
\zadStop
\rozwStart{Patryk Wirkus}{}
$$\lim\limits_{x\to\ 0}\frac{5 \cdot x}{tan(2 \cdot x)}=\lim\limits_{x\to\ 0}\frac{5 \cdot x \cdot cos(2 \cdot x)}{sin(2 \cdot x)}=\lim\limits_{x\to\ 0}\frac{5 \cdot cos(2 \cdot x)}{\frac{sin(2 \cdot x)}{x}}=\lim\limits_{x\to\ 0}\frac{5 \cdot cos(2 \cdot x)}{2 \cdot \frac{sin(2 \cdot x)}{2 \cdot x}} = \frac{5}{2}$$
\rozwStop
\odpStart
$\frac{5}{2}$
\odpStop
\testStart
A.$\frac{5}{2}$
B.$\infty$
C.$-\infty$
D.$0$
E.$-\frac{5}{2}$
F.$\frac{2}{5}$
G.$-\frac{2}{5}$
H.$2$
I.$5$
\testStop
\kluczStart
A
\kluczStop



\zadStart{Przykład z Wikieł P 4.3a moja wersja nr 21}


Obliczyć granicę funkcji $\lim\limits_{x\to\ 0}\frac{5 \cdot x}{tan(3 \cdot x)}$.
\zadStop
\rozwStart{Patryk Wirkus}{}
$$\lim\limits_{x\to\ 0}\frac{5 \cdot x}{tan(3 \cdot x)}=\lim\limits_{x\to\ 0}\frac{5 \cdot x \cdot cos(3 \cdot x)}{sin(3 \cdot x)}=\lim\limits_{x\to\ 0}\frac{5 \cdot cos(3 \cdot x)}{\frac{sin(3 \cdot x)}{x}}=\lim\limits_{x\to\ 0}\frac{5 \cdot cos(3 \cdot x)}{3 \cdot \frac{sin(3 \cdot x)}{3 \cdot x}} = \frac{5}{3}$$
\rozwStop
\odpStart
$\frac{5}{3}$
\odpStop
\testStart
A.$\frac{5}{3}$
B.$\infty$
C.$-\infty$
D.$0$
E.$-\frac{5}{3}$
F.$\frac{3}{5}$
G.$-\frac{3}{5}$
H.$3$
I.$5$
\testStop
\kluczStart
A
\kluczStop



\zadStart{Przykład z Wikieł P 4.3a moja wersja nr 22}


Obliczyć granicę funkcji $\lim\limits_{x\to\ 0}\frac{5 \cdot x}{tan(4 \cdot x)}$.
\zadStop
\rozwStart{Patryk Wirkus}{}
$$\lim\limits_{x\to\ 0}\frac{5 \cdot x}{tan(4 \cdot x)}=\lim\limits_{x\to\ 0}\frac{5 \cdot x \cdot cos(4 \cdot x)}{sin(4 \cdot x)}=\lim\limits_{x\to\ 0}\frac{5 \cdot cos(4 \cdot x)}{\frac{sin(4 \cdot x)}{x}}=\lim\limits_{x\to\ 0}\frac{5 \cdot cos(4 \cdot x)}{4 \cdot \frac{sin(4 \cdot x)}{4 \cdot x}} = \frac{5}{4}$$
\rozwStop
\odpStart
$\frac{5}{4}$
\odpStop
\testStart
A.$\frac{5}{4}$
B.$\infty$
C.$-\infty$
D.$0$
E.$-\frac{5}{4}$
F.$\frac{4}{5}$
G.$-\frac{4}{5}$
H.$4$
I.$5$
\testStop
\kluczStart
A
\kluczStop



\zadStart{Przykład z Wikieł P 4.3a moja wersja nr 23}


Obliczyć granicę funkcji $\lim\limits_{x\to\ 0}\frac{5 \cdot x}{tan(6 \cdot x)}$.
\zadStop
\rozwStart{Patryk Wirkus}{}
$$\lim\limits_{x\to\ 0}\frac{5 \cdot x}{tan(6 \cdot x)}=\lim\limits_{x\to\ 0}\frac{5 \cdot x \cdot cos(6 \cdot x)}{sin(6 \cdot x)}=\lim\limits_{x\to\ 0}\frac{5 \cdot cos(6 \cdot x)}{\frac{sin(6 \cdot x)}{x}}=\lim\limits_{x\to\ 0}\frac{5 \cdot cos(6 \cdot x)}{6 \cdot \frac{sin(6 \cdot x)}{6 \cdot x}} = \frac{5}{6}$$
\rozwStop
\odpStart
$\frac{5}{6}$
\odpStop
\testStart
A.$\frac{5}{6}$
B.$\infty$
C.$-\infty$
D.$0$
E.$-\frac{5}{6}$
F.$\frac{6}{5}$
G.$-\frac{6}{5}$
H.$6$
I.$5$
\testStop
\kluczStart
A
\kluczStop



\zadStart{Przykład z Wikieł P 4.3a moja wersja nr 24}


Obliczyć granicę funkcji $\lim\limits_{x\to\ 0}\frac{5 \cdot x}{tan(7 \cdot x)}$.
\zadStop
\rozwStart{Patryk Wirkus}{}
$$\lim\limits_{x\to\ 0}\frac{5 \cdot x}{tan(7 \cdot x)}=\lim\limits_{x\to\ 0}\frac{5 \cdot x \cdot cos(7 \cdot x)}{sin(7 \cdot x)}=\lim\limits_{x\to\ 0}\frac{5 \cdot cos(7 \cdot x)}{\frac{sin(7 \cdot x)}{x}}=\lim\limits_{x\to\ 0}\frac{5 \cdot cos(7 \cdot x)}{7 \cdot \frac{sin(7 \cdot x)}{7 \cdot x}} = \frac{5}{7}$$
\rozwStop
\odpStart
$\frac{5}{7}$
\odpStop
\testStart
A.$\frac{5}{7}$
B.$\infty$
C.$-\infty$
D.$0$
E.$-\frac{5}{7}$
F.$\frac{7}{5}$
G.$-\frac{7}{5}$
H.$7$
I.$5$
\testStop
\kluczStart
A
\kluczStop



\zadStart{Przykład z Wikieł P 4.3a moja wersja nr 25}


Obliczyć granicę funkcji $\lim\limits_{x\to\ 0}\frac{5 \cdot x}{tan(8 \cdot x)}$.
\zadStop
\rozwStart{Patryk Wirkus}{}
$$\lim\limits_{x\to\ 0}\frac{5 \cdot x}{tan(8 \cdot x)}=\lim\limits_{x\to\ 0}\frac{5 \cdot x \cdot cos(8 \cdot x)}{sin(8 \cdot x)}=\lim\limits_{x\to\ 0}\frac{5 \cdot cos(8 \cdot x)}{\frac{sin(8 \cdot x)}{x}}=\lim\limits_{x\to\ 0}\frac{5 \cdot cos(8 \cdot x)}{8 \cdot \frac{sin(8 \cdot x)}{8 \cdot x}} = \frac{5}{8}$$
\rozwStop
\odpStart
$\frac{5}{8}$
\odpStop
\testStart
A.$\frac{5}{8}$
B.$\infty$
C.$-\infty$
D.$0$
E.$-\frac{5}{8}$
F.$\frac{8}{5}$
G.$-\frac{8}{5}$
H.$8$
I.$5$
\testStop
\kluczStart
A
\kluczStop



\zadStart{Przykład z Wikieł P 4.3a moja wersja nr 26}


Obliczyć granicę funkcji $\lim\limits_{x\to\ 0}\frac{5 \cdot x}{tan(9 \cdot x)}$.
\zadStop
\rozwStart{Patryk Wirkus}{}
$$\lim\limits_{x\to\ 0}\frac{5 \cdot x}{tan(9 \cdot x)}=\lim\limits_{x\to\ 0}\frac{5 \cdot x \cdot cos(9 \cdot x)}{sin(9 \cdot x)}=\lim\limits_{x\to\ 0}\frac{5 \cdot cos(9 \cdot x)}{\frac{sin(9 \cdot x)}{x}}=\lim\limits_{x\to\ 0}\frac{5 \cdot cos(9 \cdot x)}{9 \cdot \frac{sin(9 \cdot x)}{9 \cdot x}} = \frac{5}{9}$$
\rozwStop
\odpStart
$\frac{5}{9}$
\odpStop
\testStart
A.$\frac{5}{9}$
B.$\infty$
C.$-\infty$
D.$0$
E.$-\frac{5}{9}$
F.$\frac{9}{5}$
G.$-\frac{9}{5}$
H.$9$
I.$5$
\testStop
\kluczStart
A
\kluczStop



\zadStart{Przykład z Wikieł P 4.3a moja wersja nr 27}


Obliczyć granicę funkcji $\lim\limits_{x\to\ 0}\frac{5 \cdot x}{tan(11 \cdot x)}$.
\zadStop
\rozwStart{Patryk Wirkus}{}
$$\lim\limits_{x\to\ 0}\frac{5 \cdot x}{tan(11 \cdot x)}=\lim\limits_{x\to\ 0}\frac{5 \cdot x \cdot cos(11 \cdot x)}{sin(11 \cdot x)}=\lim\limits_{x\to\ 0}\frac{5 \cdot cos(11 \cdot x)}{\frac{sin(11 \cdot x)}{x}}=\lim\limits_{x\to\ 0}\frac{5 \cdot cos(11 \cdot x)}{11 \cdot \frac{sin(11 \cdot x)}{11 \cdot x}} = \frac{5}{11}$$
\rozwStop
\odpStart
$\frac{5}{11}$
\odpStop
\testStart
A.$\frac{5}{11}$
B.$\infty$
C.$-\infty$
D.$0$
E.$-\frac{5}{11}$
F.$\frac{11}{5}$
G.$-\frac{11}{5}$
H.$11$
I.$5$
\testStop
\kluczStart
A
\kluczStop



\zadStart{Przykład z Wikieł P 4.3a moja wersja nr 28}


Obliczyć granicę funkcji $\lim\limits_{x\to\ 0}\frac{5 \cdot x}{tan(12 \cdot x)}$.
\zadStop
\rozwStart{Patryk Wirkus}{}
$$\lim\limits_{x\to\ 0}\frac{5 \cdot x}{tan(12 \cdot x)}=\lim\limits_{x\to\ 0}\frac{5 \cdot x \cdot cos(12 \cdot x)}{sin(12 \cdot x)}=\lim\limits_{x\to\ 0}\frac{5 \cdot cos(12 \cdot x)}{\frac{sin(12 \cdot x)}{x}}=\lim\limits_{x\to\ 0}\frac{5 \cdot cos(12 \cdot x)}{12 \cdot \frac{sin(12 \cdot x)}{12 \cdot x}} = \frac{5}{12}$$
\rozwStop
\odpStart
$\frac{5}{12}$
\odpStop
\testStart
A.$\frac{5}{12}$
B.$\infty$
C.$-\infty$
D.$0$
E.$-\frac{5}{12}$
F.$\frac{12}{5}$
G.$-\frac{12}{5}$
H.$12$
I.$5$
\testStop
\kluczStart
A
\kluczStop



\zadStart{Przykład z Wikieł P 4.3a moja wersja nr 29}


Obliczyć granicę funkcji $\lim\limits_{x\to\ 0}\frac{5 \cdot x}{tan(13 \cdot x)}$.
\zadStop
\rozwStart{Patryk Wirkus}{}
$$\lim\limits_{x\to\ 0}\frac{5 \cdot x}{tan(13 \cdot x)}=\lim\limits_{x\to\ 0}\frac{5 \cdot x \cdot cos(13 \cdot x)}{sin(13 \cdot x)}=\lim\limits_{x\to\ 0}\frac{5 \cdot cos(13 \cdot x)}{\frac{sin(13 \cdot x)}{x}}=\lim\limits_{x\to\ 0}\frac{5 \cdot cos(13 \cdot x)}{13 \cdot \frac{sin(13 \cdot x)}{13 \cdot x}} = \frac{5}{13}$$
\rozwStop
\odpStart
$\frac{5}{13}$
\odpStop
\testStart
A.$\frac{5}{13}$
B.$\infty$
C.$-\infty$
D.$0$
E.$-\frac{5}{13}$
F.$\frac{13}{5}$
G.$-\frac{13}{5}$
H.$13$
I.$5$
\testStop
\kluczStart
A
\kluczStop



\zadStart{Przykład z Wikieł P 4.3a moja wersja nr 30}


Obliczyć granicę funkcji $\lim\limits_{x\to\ 0}\frac{5 \cdot x}{tan(14 \cdot x)}$.
\zadStop
\rozwStart{Patryk Wirkus}{}
$$\lim\limits_{x\to\ 0}\frac{5 \cdot x}{tan(14 \cdot x)}=\lim\limits_{x\to\ 0}\frac{5 \cdot x \cdot cos(14 \cdot x)}{sin(14 \cdot x)}=\lim\limits_{x\to\ 0}\frac{5 \cdot cos(14 \cdot x)}{\frac{sin(14 \cdot x)}{x}}=\lim\limits_{x\to\ 0}\frac{5 \cdot cos(14 \cdot x)}{14 \cdot \frac{sin(14 \cdot x)}{14 \cdot x}} = \frac{5}{14}$$
\rozwStop
\odpStart
$\frac{5}{14}$
\odpStop
\testStart
A.$\frac{5}{14}$
B.$\infty$
C.$-\infty$
D.$0$
E.$-\frac{5}{14}$
F.$\frac{14}{5}$
G.$-\frac{14}{5}$
H.$14$
I.$5$
\testStop
\kluczStart
A
\kluczStop



\zadStart{Przykład z Wikieł P 4.3a moja wersja nr 31}


Obliczyć granicę funkcji $\lim\limits_{x\to\ 0}\frac{5 \cdot x}{tan(16 \cdot x)}$.
\zadStop
\rozwStart{Patryk Wirkus}{}
$$\lim\limits_{x\to\ 0}\frac{5 \cdot x}{tan(16 \cdot x)}=\lim\limits_{x\to\ 0}\frac{5 \cdot x \cdot cos(16 \cdot x)}{sin(16 \cdot x)}=\lim\limits_{x\to\ 0}\frac{5 \cdot cos(16 \cdot x)}{\frac{sin(16 \cdot x)}{x}}=\lim\limits_{x\to\ 0}\frac{5 \cdot cos(16 \cdot x)}{16 \cdot \frac{sin(16 \cdot x)}{16 \cdot x}} = \frac{5}{16}$$
\rozwStop
\odpStart
$\frac{5}{16}$
\odpStop
\testStart
A.$\frac{5}{16}$
B.$\infty$
C.$-\infty$
D.$0$
E.$-\frac{5}{16}$
F.$\frac{16}{5}$
G.$-\frac{16}{5}$
H.$16$
I.$5$
\testStop
\kluczStart
A
\kluczStop



\zadStart{Przykład z Wikieł P 4.3a moja wersja nr 32}


Obliczyć granicę funkcji $\lim\limits_{x\to\ 0}\frac{5 \cdot x}{tan(17 \cdot x)}$.
\zadStop
\rozwStart{Patryk Wirkus}{}
$$\lim\limits_{x\to\ 0}\frac{5 \cdot x}{tan(17 \cdot x)}=\lim\limits_{x\to\ 0}\frac{5 \cdot x \cdot cos(17 \cdot x)}{sin(17 \cdot x)}=\lim\limits_{x\to\ 0}\frac{5 \cdot cos(17 \cdot x)}{\frac{sin(17 \cdot x)}{x}}=\lim\limits_{x\to\ 0}\frac{5 \cdot cos(17 \cdot x)}{17 \cdot \frac{sin(17 \cdot x)}{17 \cdot x}} = \frac{5}{17}$$
\rozwStop
\odpStart
$\frac{5}{17}$
\odpStop
\testStart
A.$\frac{5}{17}$
B.$\infty$
C.$-\infty$
D.$0$
E.$-\frac{5}{17}$
F.$\frac{17}{5}$
G.$-\frac{17}{5}$
H.$17$
I.$5$
\testStop
\kluczStart
A
\kluczStop



\zadStart{Przykład z Wikieł P 4.3a moja wersja nr 33}


Obliczyć granicę funkcji $\lim\limits_{x\to\ 0}\frac{5 \cdot x}{tan(18 \cdot x)}$.
\zadStop
\rozwStart{Patryk Wirkus}{}
$$\lim\limits_{x\to\ 0}\frac{5 \cdot x}{tan(18 \cdot x)}=\lim\limits_{x\to\ 0}\frac{5 \cdot x \cdot cos(18 \cdot x)}{sin(18 \cdot x)}=\lim\limits_{x\to\ 0}\frac{5 \cdot cos(18 \cdot x)}{\frac{sin(18 \cdot x)}{x}}=\lim\limits_{x\to\ 0}\frac{5 \cdot cos(18 \cdot x)}{18 \cdot \frac{sin(18 \cdot x)}{18 \cdot x}} = \frac{5}{18}$$
\rozwStop
\odpStart
$\frac{5}{18}$
\odpStop
\testStart
A.$\frac{5}{18}$
B.$\infty$
C.$-\infty$
D.$0$
E.$-\frac{5}{18}$
F.$\frac{18}{5}$
G.$-\frac{18}{5}$
H.$18$
I.$5$
\testStop
\kluczStart
A
\kluczStop



\zadStart{Przykład z Wikieł P 4.3a moja wersja nr 34}


Obliczyć granicę funkcji $\lim\limits_{x\to\ 0}\frac{5 \cdot x}{tan(19 \cdot x)}$.
\zadStop
\rozwStart{Patryk Wirkus}{}
$$\lim\limits_{x\to\ 0}\frac{5 \cdot x}{tan(19 \cdot x)}=\lim\limits_{x\to\ 0}\frac{5 \cdot x \cdot cos(19 \cdot x)}{sin(19 \cdot x)}=\lim\limits_{x\to\ 0}\frac{5 \cdot cos(19 \cdot x)}{\frac{sin(19 \cdot x)}{x}}=\lim\limits_{x\to\ 0}\frac{5 \cdot cos(19 \cdot x)}{19 \cdot \frac{sin(19 \cdot x)}{19 \cdot x}} = \frac{5}{19}$$
\rozwStop
\odpStart
$\frac{5}{19}$
\odpStop
\testStart
A.$\frac{5}{19}$
B.$\infty$
C.$-\infty$
D.$0$
E.$-\frac{5}{19}$
F.$\frac{19}{5}$
G.$-\frac{19}{5}$
H.$19$
I.$5$
\testStop
\kluczStart
A
\kluczStop



\zadStart{Przykład z Wikieł P 4.3a moja wersja nr 35}


Obliczyć granicę funkcji $\lim\limits_{x\to\ 0}\frac{5 \cdot x}{tan(21 \cdot x)}$.
\zadStop
\rozwStart{Patryk Wirkus}{}
$$\lim\limits_{x\to\ 0}\frac{5 \cdot x}{tan(21 \cdot x)}=\lim\limits_{x\to\ 0}\frac{5 \cdot x \cdot cos(21 \cdot x)}{sin(21 \cdot x)}=\lim\limits_{x\to\ 0}\frac{5 \cdot cos(21 \cdot x)}{\frac{sin(21 \cdot x)}{x}}=\lim\limits_{x\to\ 0}\frac{5 \cdot cos(21 \cdot x)}{21 \cdot \frac{sin(21 \cdot x)}{21 \cdot x}} = \frac{5}{21}$$
\rozwStop
\odpStart
$\frac{5}{21}$
\odpStop
\testStart
A.$\frac{5}{21}$
B.$\infty$
C.$-\infty$
D.$0$
E.$-\frac{5}{21}$
F.$\frac{21}{5}$
G.$-\frac{21}{5}$
H.$21$
I.$5$
\testStop
\kluczStart
A
\kluczStop



\zadStart{Przykład z Wikieł P 4.3a moja wersja nr 36}


Obliczyć granicę funkcji $\lim\limits_{x\to\ 0}\frac{5 \cdot x}{tan(22 \cdot x)}$.
\zadStop
\rozwStart{Patryk Wirkus}{}
$$\lim\limits_{x\to\ 0}\frac{5 \cdot x}{tan(22 \cdot x)}=\lim\limits_{x\to\ 0}\frac{5 \cdot x \cdot cos(22 \cdot x)}{sin(22 \cdot x)}=\lim\limits_{x\to\ 0}\frac{5 \cdot cos(22 \cdot x)}{\frac{sin(22 \cdot x)}{x}}=\lim\limits_{x\to\ 0}\frac{5 \cdot cos(22 \cdot x)}{22 \cdot \frac{sin(22 \cdot x)}{22 \cdot x}} = \frac{5}{22}$$
\rozwStop
\odpStart
$\frac{5}{22}$
\odpStop
\testStart
A.$\frac{5}{22}$
B.$\infty$
C.$-\infty$
D.$0$
E.$-\frac{5}{22}$
F.$\frac{22}{5}$
G.$-\frac{22}{5}$
H.$22$
I.$5$
\testStop
\kluczStart
A
\kluczStop



\zadStart{Przykład z Wikieł P 4.3a moja wersja nr 37}


Obliczyć granicę funkcji $\lim\limits_{x\to\ 0}\frac{5 \cdot x}{tan(23 \cdot x)}$.
\zadStop
\rozwStart{Patryk Wirkus}{}
$$\lim\limits_{x\to\ 0}\frac{5 \cdot x}{tan(23 \cdot x)}=\lim\limits_{x\to\ 0}\frac{5 \cdot x \cdot cos(23 \cdot x)}{sin(23 \cdot x)}=\lim\limits_{x\to\ 0}\frac{5 \cdot cos(23 \cdot x)}{\frac{sin(23 \cdot x)}{x}}=\lim\limits_{x\to\ 0}\frac{5 \cdot cos(23 \cdot x)}{23 \cdot \frac{sin(23 \cdot x)}{23 \cdot x}} = \frac{5}{23}$$
\rozwStop
\odpStart
$\frac{5}{23}$
\odpStop
\testStart
A.$\frac{5}{23}$
B.$\infty$
C.$-\infty$
D.$0$
E.$-\frac{5}{23}$
F.$\frac{23}{5}$
G.$-\frac{23}{5}$
H.$23$
I.$5$
\testStop
\kluczStart
A
\kluczStop



\zadStart{Przykład z Wikieł P 4.3a moja wersja nr 38}


Obliczyć granicę funkcji $\lim\limits_{x\to\ 0}\frac{5 \cdot x}{tan(24 \cdot x)}$.
\zadStop
\rozwStart{Patryk Wirkus}{}
$$\lim\limits_{x\to\ 0}\frac{5 \cdot x}{tan(24 \cdot x)}=\lim\limits_{x\to\ 0}\frac{5 \cdot x \cdot cos(24 \cdot x)}{sin(24 \cdot x)}=\lim\limits_{x\to\ 0}\frac{5 \cdot cos(24 \cdot x)}{\frac{sin(24 \cdot x)}{x}}=\lim\limits_{x\to\ 0}\frac{5 \cdot cos(24 \cdot x)}{24 \cdot \frac{sin(24 \cdot x)}{24 \cdot x}} = \frac{5}{24}$$
\rozwStop
\odpStart
$\frac{5}{24}$
\odpStop
\testStart
A.$\frac{5}{24}$
B.$\infty$
C.$-\infty$
D.$0$
E.$-\frac{5}{24}$
F.$\frac{24}{5}$
G.$-\frac{24}{5}$
H.$24$
I.$5$
\testStop
\kluczStart
A
\kluczStop



\zadStart{Przykład z Wikieł P 4.3a moja wersja nr 39}


Obliczyć granicę funkcji $\lim\limits_{x\to\ 0}\frac{5 \cdot x}{tan(26 \cdot x)}$.
\zadStop
\rozwStart{Patryk Wirkus}{}
$$\lim\limits_{x\to\ 0}\frac{5 \cdot x}{tan(26 \cdot x)}=\lim\limits_{x\to\ 0}\frac{5 \cdot x \cdot cos(26 \cdot x)}{sin(26 \cdot x)}=\lim\limits_{x\to\ 0}\frac{5 \cdot cos(26 \cdot x)}{\frac{sin(26 \cdot x)}{x}}=\lim\limits_{x\to\ 0}\frac{5 \cdot cos(26 \cdot x)}{26 \cdot \frac{sin(26 \cdot x)}{26 \cdot x}} = \frac{5}{26}$$
\rozwStop
\odpStart
$\frac{5}{26}$
\odpStop
\testStart
A.$\frac{5}{26}$
B.$\infty$
C.$-\infty$
D.$0$
E.$-\frac{5}{26}$
F.$\frac{26}{5}$
G.$-\frac{26}{5}$
H.$26$
I.$5$
\testStop
\kluczStart
A
\kluczStop



\zadStart{Przykład z Wikieł P 4.3a moja wersja nr 40}


Obliczyć granicę funkcji $\lim\limits_{x\to\ 0}\frac{5 \cdot x}{tan(27 \cdot x)}$.
\zadStop
\rozwStart{Patryk Wirkus}{}
$$\lim\limits_{x\to\ 0}\frac{5 \cdot x}{tan(27 \cdot x)}=\lim\limits_{x\to\ 0}\frac{5 \cdot x \cdot cos(27 \cdot x)}{sin(27 \cdot x)}=\lim\limits_{x\to\ 0}\frac{5 \cdot cos(27 \cdot x)}{\frac{sin(27 \cdot x)}{x}}=\lim\limits_{x\to\ 0}\frac{5 \cdot cos(27 \cdot x)}{27 \cdot \frac{sin(27 \cdot x)}{27 \cdot x}} = \frac{5}{27}$$
\rozwStop
\odpStart
$\frac{5}{27}$
\odpStop
\testStart
A.$\frac{5}{27}$
B.$\infty$
C.$-\infty$
D.$0$
E.$-\frac{5}{27}$
F.$\frac{27}{5}$
G.$-\frac{27}{5}$
H.$27$
I.$5$
\testStop
\kluczStart
A
\kluczStop



\zadStart{Przykład z Wikieł P 4.3a moja wersja nr 41}


Obliczyć granicę funkcji $\lim\limits_{x\to\ 0}\frac{5 \cdot x}{tan(28 \cdot x)}$.
\zadStop
\rozwStart{Patryk Wirkus}{}
$$\lim\limits_{x\to\ 0}\frac{5 \cdot x}{tan(28 \cdot x)}=\lim\limits_{x\to\ 0}\frac{5 \cdot x \cdot cos(28 \cdot x)}{sin(28 \cdot x)}=\lim\limits_{x\to\ 0}\frac{5 \cdot cos(28 \cdot x)}{\frac{sin(28 \cdot x)}{x}}=\lim\limits_{x\to\ 0}\frac{5 \cdot cos(28 \cdot x)}{28 \cdot \frac{sin(28 \cdot x)}{28 \cdot x}} = \frac{5}{28}$$
\rozwStop
\odpStart
$\frac{5}{28}$
\odpStop
\testStart
A.$\frac{5}{28}$
B.$\infty$
C.$-\infty$
D.$0$
E.$-\frac{5}{28}$
F.$\frac{28}{5}$
G.$-\frac{28}{5}$
H.$28$
I.$5$
\testStop
\kluczStart
A
\kluczStop



\zadStart{Przykład z Wikieł P 4.3a moja wersja nr 42}


Obliczyć granicę funkcji $\lim\limits_{x\to\ 0}\frac{5 \cdot x}{tan(29 \cdot x)}$.
\zadStop
\rozwStart{Patryk Wirkus}{}
$$\lim\limits_{x\to\ 0}\frac{5 \cdot x}{tan(29 \cdot x)}=\lim\limits_{x\to\ 0}\frac{5 \cdot x \cdot cos(29 \cdot x)}{sin(29 \cdot x)}=\lim\limits_{x\to\ 0}\frac{5 \cdot cos(29 \cdot x)}{\frac{sin(29 \cdot x)}{x}}=\lim\limits_{x\to\ 0}\frac{5 \cdot cos(29 \cdot x)}{29 \cdot \frac{sin(29 \cdot x)}{29 \cdot x}} = \frac{5}{29}$$
\rozwStop
\odpStart
$\frac{5}{29}$
\odpStop
\testStart
A.$\frac{5}{29}$
B.$\infty$
C.$-\infty$
D.$0$
E.$-\frac{5}{29}$
F.$\frac{29}{5}$
G.$-\frac{29}{5}$
H.$29$
I.$5$
\testStop
\kluczStart
A
\kluczStop



\zadStart{Przykład z Wikieł P 4.3a moja wersja nr 43}


Obliczyć granicę funkcji $\lim\limits_{x\to\ 0}\frac{5 \cdot x}{tan(31 \cdot x)}$.
\zadStop
\rozwStart{Patryk Wirkus}{}
$$\lim\limits_{x\to\ 0}\frac{5 \cdot x}{tan(31 \cdot x)}=\lim\limits_{x\to\ 0}\frac{5 \cdot x \cdot cos(31 \cdot x)}{sin(31 \cdot x)}=\lim\limits_{x\to\ 0}\frac{5 \cdot cos(31 \cdot x)}{\frac{sin(31 \cdot x)}{x}}=\lim\limits_{x\to\ 0}\frac{5 \cdot cos(31 \cdot x)}{31 \cdot \frac{sin(31 \cdot x)}{31 \cdot x}} = \frac{5}{31}$$
\rozwStop
\odpStart
$\frac{5}{31}$
\odpStop
\testStart
A.$\frac{5}{31}$
B.$\infty$
C.$-\infty$
D.$0$
E.$-\frac{5}{31}$
F.$\frac{31}{5}$
G.$-\frac{31}{5}$
H.$31$
I.$5$
\testStop
\kluczStart
A
\kluczStop



\zadStart{Przykład z Wikieł P 4.3a moja wersja nr 44}


Obliczyć granicę funkcji $\lim\limits_{x\to\ 0}\frac{5 \cdot x}{tan(32 \cdot x)}$.
\zadStop
\rozwStart{Patryk Wirkus}{}
$$\lim\limits_{x\to\ 0}\frac{5 \cdot x}{tan(32 \cdot x)}=\lim\limits_{x\to\ 0}\frac{5 \cdot x \cdot cos(32 \cdot x)}{sin(32 \cdot x)}=\lim\limits_{x\to\ 0}\frac{5 \cdot cos(32 \cdot x)}{\frac{sin(32 \cdot x)}{x}}=\lim\limits_{x\to\ 0}\frac{5 \cdot cos(32 \cdot x)}{32 \cdot \frac{sin(32 \cdot x)}{32 \cdot x}} = \frac{5}{32}$$
\rozwStop
\odpStart
$\frac{5}{32}$
\odpStop
\testStart
A.$\frac{5}{32}$
B.$\infty$
C.$-\infty$
D.$0$
E.$-\frac{5}{32}$
F.$\frac{32}{5}$
G.$-\frac{32}{5}$
H.$32$
I.$5$
\testStop
\kluczStart
A
\kluczStop



\zadStart{Przykład z Wikieł P 4.3a moja wersja nr 45}


Obliczyć granicę funkcji $\lim\limits_{x\to\ 0}\frac{5 \cdot x}{tan(33 \cdot x)}$.
\zadStop
\rozwStart{Patryk Wirkus}{}
$$\lim\limits_{x\to\ 0}\frac{5 \cdot x}{tan(33 \cdot x)}=\lim\limits_{x\to\ 0}\frac{5 \cdot x \cdot cos(33 \cdot x)}{sin(33 \cdot x)}=\lim\limits_{x\to\ 0}\frac{5 \cdot cos(33 \cdot x)}{\frac{sin(33 \cdot x)}{x}}=\lim\limits_{x\to\ 0}\frac{5 \cdot cos(33 \cdot x)}{33 \cdot \frac{sin(33 \cdot x)}{33 \cdot x}} = \frac{5}{33}$$
\rozwStop
\odpStart
$\frac{5}{33}$
\odpStop
\testStart
A.$\frac{5}{33}$
B.$\infty$
C.$-\infty$
D.$0$
E.$-\frac{5}{33}$
F.$\frac{33}{5}$
G.$-\frac{33}{5}$
H.$33$
I.$5$
\testStop
\kluczStart
A
\kluczStop



\zadStart{Przykład z Wikieł P 4.3a moja wersja nr 46}


Obliczyć granicę funkcji $\lim\limits_{x\to\ 0}\frac{5 \cdot x}{tan(34 \cdot x)}$.
\zadStop
\rozwStart{Patryk Wirkus}{}
$$\lim\limits_{x\to\ 0}\frac{5 \cdot x}{tan(34 \cdot x)}=\lim\limits_{x\to\ 0}\frac{5 \cdot x \cdot cos(34 \cdot x)}{sin(34 \cdot x)}=\lim\limits_{x\to\ 0}\frac{5 \cdot cos(34 \cdot x)}{\frac{sin(34 \cdot x)}{x}}=\lim\limits_{x\to\ 0}\frac{5 \cdot cos(34 \cdot x)}{34 \cdot \frac{sin(34 \cdot x)}{34 \cdot x}} = \frac{5}{34}$$
\rozwStop
\odpStart
$\frac{5}{34}$
\odpStop
\testStart
A.$\frac{5}{34}$
B.$\infty$
C.$-\infty$
D.$0$
E.$-\frac{5}{34}$
F.$\frac{34}{5}$
G.$-\frac{34}{5}$
H.$34$
I.$5$
\testStop
\kluczStart
A
\kluczStop



\zadStart{Przykład z Wikieł P 4.3a moja wersja nr 47}


Obliczyć granicę funkcji $\lim\limits_{x\to\ 0}\frac{5 \cdot x}{tan(36 \cdot x)}$.
\zadStop
\rozwStart{Patryk Wirkus}{}
$$\lim\limits_{x\to\ 0}\frac{5 \cdot x}{tan(36 \cdot x)}=\lim\limits_{x\to\ 0}\frac{5 \cdot x \cdot cos(36 \cdot x)}{sin(36 \cdot x)}=\lim\limits_{x\to\ 0}\frac{5 \cdot cos(36 \cdot x)}{\frac{sin(36 \cdot x)}{x}}=\lim\limits_{x\to\ 0}\frac{5 \cdot cos(36 \cdot x)}{36 \cdot \frac{sin(36 \cdot x)}{36 \cdot x}} = \frac{5}{36}$$
\rozwStop
\odpStart
$\frac{5}{36}$
\odpStop
\testStart
A.$\frac{5}{36}$
B.$\infty$
C.$-\infty$
D.$0$
E.$-\frac{5}{36}$
F.$\frac{36}{5}$
G.$-\frac{36}{5}$
H.$36$
I.$5$
\testStop
\kluczStart
A
\kluczStop



\zadStart{Przykład z Wikieł P 4.3a moja wersja nr 48}


Obliczyć granicę funkcji $\lim\limits_{x\to\ 0}\frac{5 \cdot x}{tan(37 \cdot x)}$.
\zadStop
\rozwStart{Patryk Wirkus}{}
$$\lim\limits_{x\to\ 0}\frac{5 \cdot x}{tan(37 \cdot x)}=\lim\limits_{x\to\ 0}\frac{5 \cdot x \cdot cos(37 \cdot x)}{sin(37 \cdot x)}=\lim\limits_{x\to\ 0}\frac{5 \cdot cos(37 \cdot x)}{\frac{sin(37 \cdot x)}{x}}=\lim\limits_{x\to\ 0}\frac{5 \cdot cos(37 \cdot x)}{37 \cdot \frac{sin(37 \cdot x)}{37 \cdot x}} = \frac{5}{37}$$
\rozwStop
\odpStart
$\frac{5}{37}$
\odpStop
\testStart
A.$\frac{5}{37}$
B.$\infty$
C.$-\infty$
D.$0$
E.$-\frac{5}{37}$
F.$\frac{37}{5}$
G.$-\frac{37}{5}$
H.$37$
I.$5$
\testStop
\kluczStart
A
\kluczStop



\zadStart{Przykład z Wikieł P 4.3a moja wersja nr 49}


Obliczyć granicę funkcji $\lim\limits_{x\to\ 0}\frac{5 \cdot x}{tan(38 \cdot x)}$.
\zadStop
\rozwStart{Patryk Wirkus}{}
$$\lim\limits_{x\to\ 0}\frac{5 \cdot x}{tan(38 \cdot x)}=\lim\limits_{x\to\ 0}\frac{5 \cdot x \cdot cos(38 \cdot x)}{sin(38 \cdot x)}=\lim\limits_{x\to\ 0}\frac{5 \cdot cos(38 \cdot x)}{\frac{sin(38 \cdot x)}{x}}=\lim\limits_{x\to\ 0}\frac{5 \cdot cos(38 \cdot x)}{38 \cdot \frac{sin(38 \cdot x)}{38 \cdot x}} = \frac{5}{38}$$
\rozwStop
\odpStart
$\frac{5}{38}$
\odpStop
\testStart
A.$\frac{5}{38}$
B.$\infty$
C.$-\infty$
D.$0$
E.$-\frac{5}{38}$
F.$\frac{38}{5}$
G.$-\frac{38}{5}$
H.$38$
I.$5$
\testStop
\kluczStart
A
\kluczStop



\zadStart{Przykład z Wikieł P 4.3a moja wersja nr 50}


Obliczyć granicę funkcji $\lim\limits_{x\to\ 0}\frac{5 \cdot x}{tan(39 \cdot x)}$.
\zadStop
\rozwStart{Patryk Wirkus}{}
$$\lim\limits_{x\to\ 0}\frac{5 \cdot x}{tan(39 \cdot x)}=\lim\limits_{x\to\ 0}\frac{5 \cdot x \cdot cos(39 \cdot x)}{sin(39 \cdot x)}=\lim\limits_{x\to\ 0}\frac{5 \cdot cos(39 \cdot x)}{\frac{sin(39 \cdot x)}{x}}=\lim\limits_{x\to\ 0}\frac{5 \cdot cos(39 \cdot x)}{39 \cdot \frac{sin(39 \cdot x)}{39 \cdot x}} = \frac{5}{39}$$
\rozwStop
\odpStart
$\frac{5}{39}$
\odpStop
\testStart
A.$\frac{5}{39}$
B.$\infty$
C.$-\infty$
D.$0$
E.$-\frac{5}{39}$
F.$\frac{39}{5}$
G.$-\frac{39}{5}$
H.$39$
I.$5$
\testStop
\kluczStart
A
\kluczStop



\zadStart{Przykład z Wikieł P 4.3a moja wersja nr 51}


Obliczyć granicę funkcji $\lim\limits_{x\to\ 0}\frac{6 \cdot x}{tan(5 \cdot x)}$.
\zadStop
\rozwStart{Patryk Wirkus}{}
$$\lim\limits_{x\to\ 0}\frac{6 \cdot x}{tan(5 \cdot x)}=\lim\limits_{x\to\ 0}\frac{6 \cdot x \cdot cos(5 \cdot x)}{sin(5 \cdot x)}=\lim\limits_{x\to\ 0}\frac{6 \cdot cos(5 \cdot x)}{\frac{sin(5 \cdot x)}{x}}=\lim\limits_{x\to\ 0}\frac{6 \cdot cos(5 \cdot x)}{5 \cdot \frac{sin(5 \cdot x)}{5 \cdot x}} = \frac{6}{5}$$
\rozwStop
\odpStart
$\frac{6}{5}$
\odpStop
\testStart
A.$\frac{6}{5}$
B.$\infty$
C.$-\infty$
D.$0$
E.$-\frac{6}{5}$
F.$\frac{5}{6}$
G.$-\frac{5}{6}$
H.$5$
I.$6$
\testStop
\kluczStart
A
\kluczStop



\zadStart{Przykład z Wikieł P 4.3a moja wersja nr 52}


Obliczyć granicę funkcji $\lim\limits_{x\to\ 0}\frac{6 \cdot x}{tan(7 \cdot x)}$.
\zadStop
\rozwStart{Patryk Wirkus}{}
$$\lim\limits_{x\to\ 0}\frac{6 \cdot x}{tan(7 \cdot x)}=\lim\limits_{x\to\ 0}\frac{6 \cdot x \cdot cos(7 \cdot x)}{sin(7 \cdot x)}=\lim\limits_{x\to\ 0}\frac{6 \cdot cos(7 \cdot x)}{\frac{sin(7 \cdot x)}{x}}=\lim\limits_{x\to\ 0}\frac{6 \cdot cos(7 \cdot x)}{7 \cdot \frac{sin(7 \cdot x)}{7 \cdot x}} = \frac{6}{7}$$
\rozwStop
\odpStart
$\frac{6}{7}$
\odpStop
\testStart
A.$\frac{6}{7}$
B.$\infty$
C.$-\infty$
D.$0$
E.$-\frac{6}{7}$
F.$\frac{7}{6}$
G.$-\frac{7}{6}$
H.$7$
I.$6$
\testStop
\kluczStart
A
\kluczStop



\zadStart{Przykład z Wikieł P 4.3a moja wersja nr 53}


Obliczyć granicę funkcji $\lim\limits_{x\to\ 0}\frac{6 \cdot x}{tan(11 \cdot x)}$.
\zadStop
\rozwStart{Patryk Wirkus}{}
$$\lim\limits_{x\to\ 0}\frac{6 \cdot x}{tan(11 \cdot x)}=\lim\limits_{x\to\ 0}\frac{6 \cdot x \cdot cos(11 \cdot x)}{sin(11 \cdot x)}=\lim\limits_{x\to\ 0}\frac{6 \cdot cos(11 \cdot x)}{\frac{sin(11 \cdot x)}{x}}=\lim\limits_{x\to\ 0}\frac{6 \cdot cos(11 \cdot x)}{11 \cdot \frac{sin(11 \cdot x)}{11 \cdot x}} = \frac{6}{11}$$
\rozwStop
\odpStart
$\frac{6}{11}$
\odpStop
\testStart
A.$\frac{6}{11}$
B.$\infty$
C.$-\infty$
D.$0$
E.$-\frac{6}{11}$
F.$\frac{11}{6}$
G.$-\frac{11}{6}$
H.$11$
I.$6$
\testStop
\kluczStart
A
\kluczStop



\zadStart{Przykład z Wikieł P 4.3a moja wersja nr 54}


Obliczyć granicę funkcji $\lim\limits_{x\to\ 0}\frac{6 \cdot x}{tan(13 \cdot x)}$.
\zadStop
\rozwStart{Patryk Wirkus}{}
$$\lim\limits_{x\to\ 0}\frac{6 \cdot x}{tan(13 \cdot x)}=\lim\limits_{x\to\ 0}\frac{6 \cdot x \cdot cos(13 \cdot x)}{sin(13 \cdot x)}=\lim\limits_{x\to\ 0}\frac{6 \cdot cos(13 \cdot x)}{\frac{sin(13 \cdot x)}{x}}=\lim\limits_{x\to\ 0}\frac{6 \cdot cos(13 \cdot x)}{13 \cdot \frac{sin(13 \cdot x)}{13 \cdot x}} = \frac{6}{13}$$
\rozwStop
\odpStart
$\frac{6}{13}$
\odpStop
\testStart
A.$\frac{6}{13}$
B.$\infty$
C.$-\infty$
D.$0$
E.$-\frac{6}{13}$
F.$\frac{13}{6}$
G.$-\frac{13}{6}$
H.$13$
I.$6$
\testStop
\kluczStart
A
\kluczStop



\zadStart{Przykład z Wikieł P 4.3a moja wersja nr 55}


Obliczyć granicę funkcji $\lim\limits_{x\to\ 0}\frac{6 \cdot x}{tan(17 \cdot x)}$.
\zadStop
\rozwStart{Patryk Wirkus}{}
$$\lim\limits_{x\to\ 0}\frac{6 \cdot x}{tan(17 \cdot x)}=\lim\limits_{x\to\ 0}\frac{6 \cdot x \cdot cos(17 \cdot x)}{sin(17 \cdot x)}=\lim\limits_{x\to\ 0}\frac{6 \cdot cos(17 \cdot x)}{\frac{sin(17 \cdot x)}{x}}=\lim\limits_{x\to\ 0}\frac{6 \cdot cos(17 \cdot x)}{17 \cdot \frac{sin(17 \cdot x)}{17 \cdot x}} = \frac{6}{17}$$
\rozwStop
\odpStart
$\frac{6}{17}$
\odpStop
\testStart
A.$\frac{6}{17}$
B.$\infty$
C.$-\infty$
D.$0$
E.$-\frac{6}{17}$
F.$\frac{17}{6}$
G.$-\frac{17}{6}$
H.$17$
I.$6$
\testStop
\kluczStart
A
\kluczStop



\zadStart{Przykład z Wikieł P 4.3a moja wersja nr 56}


Obliczyć granicę funkcji $\lim\limits_{x\to\ 0}\frac{6 \cdot x}{tan(19 \cdot x)}$.
\zadStop
\rozwStart{Patryk Wirkus}{}
$$\lim\limits_{x\to\ 0}\frac{6 \cdot x}{tan(19 \cdot x)}=\lim\limits_{x\to\ 0}\frac{6 \cdot x \cdot cos(19 \cdot x)}{sin(19 \cdot x)}=\lim\limits_{x\to\ 0}\frac{6 \cdot cos(19 \cdot x)}{\frac{sin(19 \cdot x)}{x}}=\lim\limits_{x\to\ 0}\frac{6 \cdot cos(19 \cdot x)}{19 \cdot \frac{sin(19 \cdot x)}{19 \cdot x}} = \frac{6}{19}$$
\rozwStop
\odpStart
$\frac{6}{19}$
\odpStop
\testStart
A.$\frac{6}{19}$
B.$\infty$
C.$-\infty$
D.$0$
E.$-\frac{6}{19}$
F.$\frac{19}{6}$
G.$-\frac{19}{6}$
H.$19$
I.$6$
\testStop
\kluczStart
A
\kluczStop



\zadStart{Przykład z Wikieł P 4.3a moja wersja nr 57}


Obliczyć granicę funkcji $\lim\limits_{x\to\ 0}\frac{6 \cdot x}{tan(23 \cdot x)}$.
\zadStop
\rozwStart{Patryk Wirkus}{}
$$\lim\limits_{x\to\ 0}\frac{6 \cdot x}{tan(23 \cdot x)}=\lim\limits_{x\to\ 0}\frac{6 \cdot x \cdot cos(23 \cdot x)}{sin(23 \cdot x)}=\lim\limits_{x\to\ 0}\frac{6 \cdot cos(23 \cdot x)}{\frac{sin(23 \cdot x)}{x}}=\lim\limits_{x\to\ 0}\frac{6 \cdot cos(23 \cdot x)}{23 \cdot \frac{sin(23 \cdot x)}{23 \cdot x}} = \frac{6}{23}$$
\rozwStop
\odpStart
$\frac{6}{23}$
\odpStop
\testStart
A.$\frac{6}{23}$
B.$\infty$
C.$-\infty$
D.$0$
E.$-\frac{6}{23}$
F.$\frac{23}{6}$
G.$-\frac{23}{6}$
H.$23$
I.$6$
\testStop
\kluczStart
A
\kluczStop



\zadStart{Przykład z Wikieł P 4.3a moja wersja nr 58}


Obliczyć granicę funkcji $\lim\limits_{x\to\ 0}\frac{6 \cdot x}{tan(25 \cdot x)}$.
\zadStop
\rozwStart{Patryk Wirkus}{}
$$\lim\limits_{x\to\ 0}\frac{6 \cdot x}{tan(25 \cdot x)}=\lim\limits_{x\to\ 0}\frac{6 \cdot x \cdot cos(25 \cdot x)}{sin(25 \cdot x)}=\lim\limits_{x\to\ 0}\frac{6 \cdot cos(25 \cdot x)}{\frac{sin(25 \cdot x)}{x}}=\lim\limits_{x\to\ 0}\frac{6 \cdot cos(25 \cdot x)}{25 \cdot \frac{sin(25 \cdot x)}{25 \cdot x}} = \frac{6}{25}$$
\rozwStop
\odpStart
$\frac{6}{25}$
\odpStop
\testStart
A.$\frac{6}{25}$
B.$\infty$
C.$-\infty$
D.$0$
E.$-\frac{6}{25}$
F.$\frac{25}{6}$
G.$-\frac{25}{6}$
H.$25$
I.$6$
\testStop
\kluczStart
A
\kluczStop



\zadStart{Przykład z Wikieł P 4.3a moja wersja nr 59}


Obliczyć granicę funkcji $\lim\limits_{x\to\ 0}\frac{6 \cdot x}{tan(29 \cdot x)}$.
\zadStop
\rozwStart{Patryk Wirkus}{}
$$\lim\limits_{x\to\ 0}\frac{6 \cdot x}{tan(29 \cdot x)}=\lim\limits_{x\to\ 0}\frac{6 \cdot x \cdot cos(29 \cdot x)}{sin(29 \cdot x)}=\lim\limits_{x\to\ 0}\frac{6 \cdot cos(29 \cdot x)}{\frac{sin(29 \cdot x)}{x}}=\lim\limits_{x\to\ 0}\frac{6 \cdot cos(29 \cdot x)}{29 \cdot \frac{sin(29 \cdot x)}{29 \cdot x}} = \frac{6}{29}$$
\rozwStop
\odpStart
$\frac{6}{29}$
\odpStop
\testStart
A.$\frac{6}{29}$
B.$\infty$
C.$-\infty$
D.$0$
E.$-\frac{6}{29}$
F.$\frac{29}{6}$
G.$-\frac{29}{6}$
H.$29$
I.$6$
\testStop
\kluczStart
A
\kluczStop



\zadStart{Przykład z Wikieł P 4.3a moja wersja nr 60}


Obliczyć granicę funkcji $\lim\limits_{x\to\ 0}\frac{6 \cdot x}{tan(31 \cdot x)}$.
\zadStop
\rozwStart{Patryk Wirkus}{}
$$\lim\limits_{x\to\ 0}\frac{6 \cdot x}{tan(31 \cdot x)}=\lim\limits_{x\to\ 0}\frac{6 \cdot x \cdot cos(31 \cdot x)}{sin(31 \cdot x)}=\lim\limits_{x\to\ 0}\frac{6 \cdot cos(31 \cdot x)}{\frac{sin(31 \cdot x)}{x}}=\lim\limits_{x\to\ 0}\frac{6 \cdot cos(31 \cdot x)}{31 \cdot \frac{sin(31 \cdot x)}{31 \cdot x}} = \frac{6}{31}$$
\rozwStop
\odpStart
$\frac{6}{31}$
\odpStop
\testStart
A.$\frac{6}{31}$
B.$\infty$
C.$-\infty$
D.$0$
E.$-\frac{6}{31}$
F.$\frac{31}{6}$
G.$-\frac{31}{6}$
H.$31$
I.$6$
\testStop
\kluczStart
A
\kluczStop



\zadStart{Przykład z Wikieł P 4.3a moja wersja nr 61}


Obliczyć granicę funkcji $\lim\limits_{x\to\ 0}\frac{6 \cdot x}{tan(35 \cdot x)}$.
\zadStop
\rozwStart{Patryk Wirkus}{}
$$\lim\limits_{x\to\ 0}\frac{6 \cdot x}{tan(35 \cdot x)}=\lim\limits_{x\to\ 0}\frac{6 \cdot x \cdot cos(35 \cdot x)}{sin(35 \cdot x)}=\lim\limits_{x\to\ 0}\frac{6 \cdot cos(35 \cdot x)}{\frac{sin(35 \cdot x)}{x}}=\lim\limits_{x\to\ 0}\frac{6 \cdot cos(35 \cdot x)}{35 \cdot \frac{sin(35 \cdot x)}{35 \cdot x}} = \frac{6}{35}$$
\rozwStop
\odpStart
$\frac{6}{35}$
\odpStop
\testStart
A.$\frac{6}{35}$
B.$\infty$
C.$-\infty$
D.$0$
E.$-\frac{6}{35}$
F.$\frac{35}{6}$
G.$-\frac{35}{6}$
H.$35$
I.$6$
\testStop
\kluczStart
A
\kluczStop



\zadStart{Przykład z Wikieł P 4.3a moja wersja nr 62}


Obliczyć granicę funkcji $\lim\limits_{x\to\ 0}\frac{6 \cdot x}{tan(37 \cdot x)}$.
\zadStop
\rozwStart{Patryk Wirkus}{}
$$\lim\limits_{x\to\ 0}\frac{6 \cdot x}{tan(37 \cdot x)}=\lim\limits_{x\to\ 0}\frac{6 \cdot x \cdot cos(37 \cdot x)}{sin(37 \cdot x)}=\lim\limits_{x\to\ 0}\frac{6 \cdot cos(37 \cdot x)}{\frac{sin(37 \cdot x)}{x}}=\lim\limits_{x\to\ 0}\frac{6 \cdot cos(37 \cdot x)}{37 \cdot \frac{sin(37 \cdot x)}{37 \cdot x}} = \frac{6}{37}$$
\rozwStop
\odpStart
$\frac{6}{37}$
\odpStop
\testStart
A.$\frac{6}{37}$
B.$\infty$
C.$-\infty$
D.$0$
E.$-\frac{6}{37}$
F.$\frac{37}{6}$
G.$-\frac{37}{6}$
H.$37$
I.$6$
\testStop
\kluczStart
A
\kluczStop



\zadStart{Przykład z Wikieł P 4.3a moja wersja nr 63}


Obliczyć granicę funkcji $\lim\limits_{x\to\ 0}\frac{7 \cdot x}{tan(2 \cdot x)}$.
\zadStop
\rozwStart{Patryk Wirkus}{}
$$\lim\limits_{x\to\ 0}\frac{7 \cdot x}{tan(2 \cdot x)}=\lim\limits_{x\to\ 0}\frac{7 \cdot x \cdot cos(2 \cdot x)}{sin(2 \cdot x)}=\lim\limits_{x\to\ 0}\frac{7 \cdot cos(2 \cdot x)}{\frac{sin(2 \cdot x)}{x}}=\lim\limits_{x\to\ 0}\frac{7 \cdot cos(2 \cdot x)}{2 \cdot \frac{sin(2 \cdot x)}{2 \cdot x}} = \frac{7}{2}$$
\rozwStop
\odpStart
$\frac{7}{2}$
\odpStop
\testStart
A.$\frac{7}{2}$
B.$\infty$
C.$-\infty$
D.$0$
E.$-\frac{7}{2}$
F.$\frac{2}{7}$
G.$-\frac{2}{7}$
H.$2$
I.$7$
\testStop
\kluczStart
A
\kluczStop



\zadStart{Przykład z Wikieł P 4.3a moja wersja nr 64}


Obliczyć granicę funkcji $\lim\limits_{x\to\ 0}\frac{7 \cdot x}{tan(3 \cdot x)}$.
\zadStop
\rozwStart{Patryk Wirkus}{}
$$\lim\limits_{x\to\ 0}\frac{7 \cdot x}{tan(3 \cdot x)}=\lim\limits_{x\to\ 0}\frac{7 \cdot x \cdot cos(3 \cdot x)}{sin(3 \cdot x)}=\lim\limits_{x\to\ 0}\frac{7 \cdot cos(3 \cdot x)}{\frac{sin(3 \cdot x)}{x}}=\lim\limits_{x\to\ 0}\frac{7 \cdot cos(3 \cdot x)}{3 \cdot \frac{sin(3 \cdot x)}{3 \cdot x}} = \frac{7}{3}$$
\rozwStop
\odpStart
$\frac{7}{3}$
\odpStop
\testStart
A.$\frac{7}{3}$
B.$\infty$
C.$-\infty$
D.$0$
E.$-\frac{7}{3}$
F.$\frac{3}{7}$
G.$-\frac{3}{7}$
H.$3$
I.$7$
\testStop
\kluczStart
A
\kluczStop



\zadStart{Przykład z Wikieł P 4.3a moja wersja nr 65}


Obliczyć granicę funkcji $\lim\limits_{x\to\ 0}\frac{7 \cdot x}{tan(4 \cdot x)}$.
\zadStop
\rozwStart{Patryk Wirkus}{}
$$\lim\limits_{x\to\ 0}\frac{7 \cdot x}{tan(4 \cdot x)}=\lim\limits_{x\to\ 0}\frac{7 \cdot x \cdot cos(4 \cdot x)}{sin(4 \cdot x)}=\lim\limits_{x\to\ 0}\frac{7 \cdot cos(4 \cdot x)}{\frac{sin(4 \cdot x)}{x}}=\lim\limits_{x\to\ 0}\frac{7 \cdot cos(4 \cdot x)}{4 \cdot \frac{sin(4 \cdot x)}{4 \cdot x}} = \frac{7}{4}$$
\rozwStop
\odpStart
$\frac{7}{4}$
\odpStop
\testStart
A.$\frac{7}{4}$
B.$\infty$
C.$-\infty$
D.$0$
E.$-\frac{7}{4}$
F.$\frac{4}{7}$
G.$-\frac{4}{7}$
H.$4$
I.$7$
\testStop
\kluczStart
A
\kluczStop



\zadStart{Przykład z Wikieł P 4.3a moja wersja nr 66}


Obliczyć granicę funkcji $\lim\limits_{x\to\ 0}\frac{7 \cdot x}{tan(5 \cdot x)}$.
\zadStop
\rozwStart{Patryk Wirkus}{}
$$\lim\limits_{x\to\ 0}\frac{7 \cdot x}{tan(5 \cdot x)}=\lim\limits_{x\to\ 0}\frac{7 \cdot x \cdot cos(5 \cdot x)}{sin(5 \cdot x)}=\lim\limits_{x\to\ 0}\frac{7 \cdot cos(5 \cdot x)}{\frac{sin(5 \cdot x)}{x}}=\lim\limits_{x\to\ 0}\frac{7 \cdot cos(5 \cdot x)}{5 \cdot \frac{sin(5 \cdot x)}{5 \cdot x}} = \frac{7}{5}$$
\rozwStop
\odpStart
$\frac{7}{5}$
\odpStop
\testStart
A.$\frac{7}{5}$
B.$\infty$
C.$-\infty$
D.$0$
E.$-\frac{7}{5}$
F.$\frac{5}{7}$
G.$-\frac{5}{7}$
H.$5$
I.$7$
\testStop
\kluczStart
A
\kluczStop



\zadStart{Przykład z Wikieł P 4.3a moja wersja nr 67}


Obliczyć granicę funkcji $\lim\limits_{x\to\ 0}\frac{7 \cdot x}{tan(6 \cdot x)}$.
\zadStop
\rozwStart{Patryk Wirkus}{}
$$\lim\limits_{x\to\ 0}\frac{7 \cdot x}{tan(6 \cdot x)}=\lim\limits_{x\to\ 0}\frac{7 \cdot x \cdot cos(6 \cdot x)}{sin(6 \cdot x)}=\lim\limits_{x\to\ 0}\frac{7 \cdot cos(6 \cdot x)}{\frac{sin(6 \cdot x)}{x}}=\lim\limits_{x\to\ 0}\frac{7 \cdot cos(6 \cdot x)}{6 \cdot \frac{sin(6 \cdot x)}{6 \cdot x}} = \frac{7}{6}$$
\rozwStop
\odpStart
$\frac{7}{6}$
\odpStop
\testStart
A.$\frac{7}{6}$
B.$\infty$
C.$-\infty$
D.$0$
E.$-\frac{7}{6}$
F.$\frac{6}{7}$
G.$-\frac{6}{7}$
H.$6$
I.$7$
\testStop
\kluczStart
A
\kluczStop



\zadStart{Przykład z Wikieł P 4.3a moja wersja nr 68}


Obliczyć granicę funkcji $\lim\limits_{x\to\ 0}\frac{7 \cdot x}{tan(8 \cdot x)}$.
\zadStop
\rozwStart{Patryk Wirkus}{}
$$\lim\limits_{x\to\ 0}\frac{7 \cdot x}{tan(8 \cdot x)}=\lim\limits_{x\to\ 0}\frac{7 \cdot x \cdot cos(8 \cdot x)}{sin(8 \cdot x)}=\lim\limits_{x\to\ 0}\frac{7 \cdot cos(8 \cdot x)}{\frac{sin(8 \cdot x)}{x}}=\lim\limits_{x\to\ 0}\frac{7 \cdot cos(8 \cdot x)}{8 \cdot \frac{sin(8 \cdot x)}{8 \cdot x}} = \frac{7}{8}$$
\rozwStop
\odpStart
$\frac{7}{8}$
\odpStop
\testStart
A.$\frac{7}{8}$
B.$\infty$
C.$-\infty$
D.$0$
E.$-\frac{7}{8}$
F.$\frac{8}{7}$
G.$-\frac{8}{7}$
H.$8$
I.$7$
\testStop
\kluczStart
A
\kluczStop



\zadStart{Przykład z Wikieł P 4.3a moja wersja nr 69}


Obliczyć granicę funkcji $\lim\limits_{x\to\ 0}\frac{7 \cdot x}{tan(9 \cdot x)}$.
\zadStop
\rozwStart{Patryk Wirkus}{}
$$\lim\limits_{x\to\ 0}\frac{7 \cdot x}{tan(9 \cdot x)}=\lim\limits_{x\to\ 0}\frac{7 \cdot x \cdot cos(9 \cdot x)}{sin(9 \cdot x)}=\lim\limits_{x\to\ 0}\frac{7 \cdot cos(9 \cdot x)}{\frac{sin(9 \cdot x)}{x}}=\lim\limits_{x\to\ 0}\frac{7 \cdot cos(9 \cdot x)}{9 \cdot \frac{sin(9 \cdot x)}{9 \cdot x}} = \frac{7}{9}$$
\rozwStop
\odpStart
$\frac{7}{9}$
\odpStop
\testStart
A.$\frac{7}{9}$
B.$\infty$
C.$-\infty$
D.$0$
E.$-\frac{7}{9}$
F.$\frac{9}{7}$
G.$-\frac{9}{7}$
H.$9$
I.$7$
\testStop
\kluczStart
A
\kluczStop



\zadStart{Przykład z Wikieł P 4.3a moja wersja nr 70}


Obliczyć granicę funkcji $\lim\limits_{x\to\ 0}\frac{7 \cdot x}{tan(10 \cdot x)}$.
\zadStop
\rozwStart{Patryk Wirkus}{}
$$\lim\limits_{x\to\ 0}\frac{7 \cdot x}{tan(10 \cdot x)}=\lim\limits_{x\to\ 0}\frac{7 \cdot x \cdot cos(10 \cdot x)}{sin(10 \cdot x)}=\lim\limits_{x\to\ 0}\frac{7 \cdot cos(10 \cdot x)}{\frac{sin(10 \cdot x)}{x}}=\lim\limits_{x\to\ 0}\frac{7 \cdot cos(10 \cdot x)}{10 \cdot \frac{sin(10 \cdot x)}{10 \cdot x}} = \frac{7}{10}$$
\rozwStop
\odpStart
$\frac{7}{10}$
\odpStop
\testStart
A.$\frac{7}{10}$
B.$\infty$
C.$-\infty$
D.$0$
E.$-\frac{7}{10}$
F.$\frac{10}{7}$
G.$-\frac{10}{7}$
H.$10$
I.$7$
\testStop
\kluczStart
A
\kluczStop



\zadStart{Przykład z Wikieł P 4.3a moja wersja nr 71}


Obliczyć granicę funkcji $\lim\limits_{x\to\ 0}\frac{7 \cdot x}{tan(11 \cdot x)}$.
\zadStop
\rozwStart{Patryk Wirkus}{}
$$\lim\limits_{x\to\ 0}\frac{7 \cdot x}{tan(11 \cdot x)}=\lim\limits_{x\to\ 0}\frac{7 \cdot x \cdot cos(11 \cdot x)}{sin(11 \cdot x)}=\lim\limits_{x\to\ 0}\frac{7 \cdot cos(11 \cdot x)}{\frac{sin(11 \cdot x)}{x}}=\lim\limits_{x\to\ 0}\frac{7 \cdot cos(11 \cdot x)}{11 \cdot \frac{sin(11 \cdot x)}{11 \cdot x}} = \frac{7}{11}$$
\rozwStop
\odpStart
$\frac{7}{11}$
\odpStop
\testStart
A.$\frac{7}{11}$
B.$\infty$
C.$-\infty$
D.$0$
E.$-\frac{7}{11}$
F.$\frac{11}{7}$
G.$-\frac{11}{7}$
H.$11$
I.$7$
\testStop
\kluczStart
A
\kluczStop



\zadStart{Przykład z Wikieł P 4.3a moja wersja nr 72}


Obliczyć granicę funkcji $\lim\limits_{x\to\ 0}\frac{7 \cdot x}{tan(12 \cdot x)}$.
\zadStop
\rozwStart{Patryk Wirkus}{}
$$\lim\limits_{x\to\ 0}\frac{7 \cdot x}{tan(12 \cdot x)}=\lim\limits_{x\to\ 0}\frac{7 \cdot x \cdot cos(12 \cdot x)}{sin(12 \cdot x)}=\lim\limits_{x\to\ 0}\frac{7 \cdot cos(12 \cdot x)}{\frac{sin(12 \cdot x)}{x}}=\lim\limits_{x\to\ 0}\frac{7 \cdot cos(12 \cdot x)}{12 \cdot \frac{sin(12 \cdot x)}{12 \cdot x}} = \frac{7}{12}$$
\rozwStop
\odpStart
$\frac{7}{12}$
\odpStop
\testStart
A.$\frac{7}{12}$
B.$\infty$
C.$-\infty$
D.$0$
E.$-\frac{7}{12}$
F.$\frac{12}{7}$
G.$-\frac{12}{7}$
H.$12$
I.$7$
\testStop
\kluczStart
A
\kluczStop



\zadStart{Przykład z Wikieł P 4.3a moja wersja nr 73}


Obliczyć granicę funkcji $\lim\limits_{x\to\ 0}\frac{7 \cdot x}{tan(13 \cdot x)}$.
\zadStop
\rozwStart{Patryk Wirkus}{}
$$\lim\limits_{x\to\ 0}\frac{7 \cdot x}{tan(13 \cdot x)}=\lim\limits_{x\to\ 0}\frac{7 \cdot x \cdot cos(13 \cdot x)}{sin(13 \cdot x)}=\lim\limits_{x\to\ 0}\frac{7 \cdot cos(13 \cdot x)}{\frac{sin(13 \cdot x)}{x}}=\lim\limits_{x\to\ 0}\frac{7 \cdot cos(13 \cdot x)}{13 \cdot \frac{sin(13 \cdot x)}{13 \cdot x}} = \frac{7}{13}$$
\rozwStop
\odpStart
$\frac{7}{13}$
\odpStop
\testStart
A.$\frac{7}{13}$
B.$\infty$
C.$-\infty$
D.$0$
E.$-\frac{7}{13}$
F.$\frac{13}{7}$
G.$-\frac{13}{7}$
H.$13$
I.$7$
\testStop
\kluczStart
A
\kluczStop



\zadStart{Przykład z Wikieł P 4.3a moja wersja nr 74}


Obliczyć granicę funkcji $\lim\limits_{x\to\ 0}\frac{7 \cdot x}{tan(15 \cdot x)}$.
\zadStop
\rozwStart{Patryk Wirkus}{}
$$\lim\limits_{x\to\ 0}\frac{7 \cdot x}{tan(15 \cdot x)}=\lim\limits_{x\to\ 0}\frac{7 \cdot x \cdot cos(15 \cdot x)}{sin(15 \cdot x)}=\lim\limits_{x\to\ 0}\frac{7 \cdot cos(15 \cdot x)}{\frac{sin(15 \cdot x)}{x}}=\lim\limits_{x\to\ 0}\frac{7 \cdot cos(15 \cdot x)}{15 \cdot \frac{sin(15 \cdot x)}{15 \cdot x}} = \frac{7}{15}$$
\rozwStop
\odpStart
$\frac{7}{15}$
\odpStop
\testStart
A.$\frac{7}{15}$
B.$\infty$
C.$-\infty$
D.$0$
E.$-\frac{7}{15}$
F.$\frac{15}{7}$
G.$-\frac{15}{7}$
H.$15$
I.$7$
\testStop
\kluczStart
A
\kluczStop



\zadStart{Przykład z Wikieł P 4.3a moja wersja nr 75}


Obliczyć granicę funkcji $\lim\limits_{x\to\ 0}\frac{7 \cdot x}{tan(16 \cdot x)}$.
\zadStop
\rozwStart{Patryk Wirkus}{}
$$\lim\limits_{x\to\ 0}\frac{7 \cdot x}{tan(16 \cdot x)}=\lim\limits_{x\to\ 0}\frac{7 \cdot x \cdot cos(16 \cdot x)}{sin(16 \cdot x)}=\lim\limits_{x\to\ 0}\frac{7 \cdot cos(16 \cdot x)}{\frac{sin(16 \cdot x)}{x}}=\lim\limits_{x\to\ 0}\frac{7 \cdot cos(16 \cdot x)}{16 \cdot \frac{sin(16 \cdot x)}{16 \cdot x}} = \frac{7}{16}$$
\rozwStop
\odpStart
$\frac{7}{16}$
\odpStop
\testStart
A.$\frac{7}{16}$
B.$\infty$
C.$-\infty$
D.$0$
E.$-\frac{7}{16}$
F.$\frac{16}{7}$
G.$-\frac{16}{7}$
H.$16$
I.$7$
\testStop
\kluczStart
A
\kluczStop



\zadStart{Przykład z Wikieł P 4.3a moja wersja nr 76}


Obliczyć granicę funkcji $\lim\limits_{x\to\ 0}\frac{7 \cdot x}{tan(17 \cdot x)}$.
\zadStop
\rozwStart{Patryk Wirkus}{}
$$\lim\limits_{x\to\ 0}\frac{7 \cdot x}{tan(17 \cdot x)}=\lim\limits_{x\to\ 0}\frac{7 \cdot x \cdot cos(17 \cdot x)}{sin(17 \cdot x)}=\lim\limits_{x\to\ 0}\frac{7 \cdot cos(17 \cdot x)}{\frac{sin(17 \cdot x)}{x}}=\lim\limits_{x\to\ 0}\frac{7 \cdot cos(17 \cdot x)}{17 \cdot \frac{sin(17 \cdot x)}{17 \cdot x}} = \frac{7}{17}$$
\rozwStop
\odpStart
$\frac{7}{17}$
\odpStop
\testStart
A.$\frac{7}{17}$
B.$\infty$
C.$-\infty$
D.$0$
E.$-\frac{7}{17}$
F.$\frac{17}{7}$
G.$-\frac{17}{7}$
H.$17$
I.$7$
\testStop
\kluczStart
A
\kluczStop



\zadStart{Przykład z Wikieł P 4.3a moja wersja nr 77}


Obliczyć granicę funkcji $\lim\limits_{x\to\ 0}\frac{7 \cdot x}{tan(18 \cdot x)}$.
\zadStop
\rozwStart{Patryk Wirkus}{}
$$\lim\limits_{x\to\ 0}\frac{7 \cdot x}{tan(18 \cdot x)}=\lim\limits_{x\to\ 0}\frac{7 \cdot x \cdot cos(18 \cdot x)}{sin(18 \cdot x)}=\lim\limits_{x\to\ 0}\frac{7 \cdot cos(18 \cdot x)}{\frac{sin(18 \cdot x)}{x}}=\lim\limits_{x\to\ 0}\frac{7 \cdot cos(18 \cdot x)}{18 \cdot \frac{sin(18 \cdot x)}{18 \cdot x}} = \frac{7}{18}$$
\rozwStop
\odpStart
$\frac{7}{18}$
\odpStop
\testStart
A.$\frac{7}{18}$
B.$\infty$
C.$-\infty$
D.$0$
E.$-\frac{7}{18}$
F.$\frac{18}{7}$
G.$-\frac{18}{7}$
H.$18$
I.$7$
\testStop
\kluczStart
A
\kluczStop



\zadStart{Przykład z Wikieł P 4.3a moja wersja nr 78}


Obliczyć granicę funkcji $\lim\limits_{x\to\ 0}\frac{7 \cdot x}{tan(19 \cdot x)}$.
\zadStop
\rozwStart{Patryk Wirkus}{}
$$\lim\limits_{x\to\ 0}\frac{7 \cdot x}{tan(19 \cdot x)}=\lim\limits_{x\to\ 0}\frac{7 \cdot x \cdot cos(19 \cdot x)}{sin(19 \cdot x)}=\lim\limits_{x\to\ 0}\frac{7 \cdot cos(19 \cdot x)}{\frac{sin(19 \cdot x)}{x}}=\lim\limits_{x\to\ 0}\frac{7 \cdot cos(19 \cdot x)}{19 \cdot \frac{sin(19 \cdot x)}{19 \cdot x}} = \frac{7}{19}$$
\rozwStop
\odpStart
$\frac{7}{19}$
\odpStop
\testStart
A.$\frac{7}{19}$
B.$\infty$
C.$-\infty$
D.$0$
E.$-\frac{7}{19}$
F.$\frac{19}{7}$
G.$-\frac{19}{7}$
H.$19$
I.$7$
\testStop
\kluczStart
A
\kluczStop



\zadStart{Przykład z Wikieł P 4.3a moja wersja nr 79}


Obliczyć granicę funkcji $\lim\limits_{x\to\ 0}\frac{7 \cdot x}{tan(20 \cdot x)}$.
\zadStop
\rozwStart{Patryk Wirkus}{}
$$\lim\limits_{x\to\ 0}\frac{7 \cdot x}{tan(20 \cdot x)}=\lim\limits_{x\to\ 0}\frac{7 \cdot x \cdot cos(20 \cdot x)}{sin(20 \cdot x)}=\lim\limits_{x\to\ 0}\frac{7 \cdot cos(20 \cdot x)}{\frac{sin(20 \cdot x)}{x}}=\lim\limits_{x\to\ 0}\frac{7 \cdot cos(20 \cdot x)}{20 \cdot \frac{sin(20 \cdot x)}{20 \cdot x}} = \frac{7}{20}$$
\rozwStop
\odpStart
$\frac{7}{20}$
\odpStop
\testStart
A.$\frac{7}{20}$
B.$\infty$
C.$-\infty$
D.$0$
E.$-\frac{7}{20}$
F.$\frac{20}{7}$
G.$-\frac{20}{7}$
H.$20$
I.$7$
\testStop
\kluczStart
A
\kluczStop



\zadStart{Przykład z Wikieł P 4.3a moja wersja nr 80}


Obliczyć granicę funkcji $\lim\limits_{x\to\ 0}\frac{7 \cdot x}{tan(22 \cdot x)}$.
\zadStop
\rozwStart{Patryk Wirkus}{}
$$\lim\limits_{x\to\ 0}\frac{7 \cdot x}{tan(22 \cdot x)}=\lim\limits_{x\to\ 0}\frac{7 \cdot x \cdot cos(22 \cdot x)}{sin(22 \cdot x)}=\lim\limits_{x\to\ 0}\frac{7 \cdot cos(22 \cdot x)}{\frac{sin(22 \cdot x)}{x}}=\lim\limits_{x\to\ 0}\frac{7 \cdot cos(22 \cdot x)}{22 \cdot \frac{sin(22 \cdot x)}{22 \cdot x}} = \frac{7}{22}$$
\rozwStop
\odpStart
$\frac{7}{22}$
\odpStop
\testStart
A.$\frac{7}{22}$
B.$\infty$
C.$-\infty$
D.$0$
E.$-\frac{7}{22}$
F.$\frac{22}{7}$
G.$-\frac{22}{7}$
H.$22$
I.$7$
\testStop
\kluczStart
A
\kluczStop



\zadStart{Przykład z Wikieł P 4.3a moja wersja nr 81}


Obliczyć granicę funkcji $\lim\limits_{x\to\ 0}\frac{7 \cdot x}{tan(23 \cdot x)}$.
\zadStop
\rozwStart{Patryk Wirkus}{}
$$\lim\limits_{x\to\ 0}\frac{7 \cdot x}{tan(23 \cdot x)}=\lim\limits_{x\to\ 0}\frac{7 \cdot x \cdot cos(23 \cdot x)}{sin(23 \cdot x)}=\lim\limits_{x\to\ 0}\frac{7 \cdot cos(23 \cdot x)}{\frac{sin(23 \cdot x)}{x}}=\lim\limits_{x\to\ 0}\frac{7 \cdot cos(23 \cdot x)}{23 \cdot \frac{sin(23 \cdot x)}{23 \cdot x}} = \frac{7}{23}$$
\rozwStop
\odpStart
$\frac{7}{23}$
\odpStop
\testStart
A.$\frac{7}{23}$
B.$\infty$
C.$-\infty$
D.$0$
E.$-\frac{7}{23}$
F.$\frac{23}{7}$
G.$-\frac{23}{7}$
H.$23$
I.$7$
\testStop
\kluczStart
A
\kluczStop



\zadStart{Przykład z Wikieł P 4.3a moja wersja nr 82}


Obliczyć granicę funkcji $\lim\limits_{x\to\ 0}\frac{7 \cdot x}{tan(24 \cdot x)}$.
\zadStop
\rozwStart{Patryk Wirkus}{}
$$\lim\limits_{x\to\ 0}\frac{7 \cdot x}{tan(24 \cdot x)}=\lim\limits_{x\to\ 0}\frac{7 \cdot x \cdot cos(24 \cdot x)}{sin(24 \cdot x)}=\lim\limits_{x\to\ 0}\frac{7 \cdot cos(24 \cdot x)}{\frac{sin(24 \cdot x)}{x}}=\lim\limits_{x\to\ 0}\frac{7 \cdot cos(24 \cdot x)}{24 \cdot \frac{sin(24 \cdot x)}{24 \cdot x}} = \frac{7}{24}$$
\rozwStop
\odpStart
$\frac{7}{24}$
\odpStop
\testStart
A.$\frac{7}{24}$
B.$\infty$
C.$-\infty$
D.$0$
E.$-\frac{7}{24}$
F.$\frac{24}{7}$
G.$-\frac{24}{7}$
H.$24$
I.$7$
\testStop
\kluczStart
A
\kluczStop



\zadStart{Przykład z Wikieł P 4.3a moja wersja nr 83}


Obliczyć granicę funkcji $\lim\limits_{x\to\ 0}\frac{7 \cdot x}{tan(25 \cdot x)}$.
\zadStop
\rozwStart{Patryk Wirkus}{}
$$\lim\limits_{x\to\ 0}\frac{7 \cdot x}{tan(25 \cdot x)}=\lim\limits_{x\to\ 0}\frac{7 \cdot x \cdot cos(25 \cdot x)}{sin(25 \cdot x)}=\lim\limits_{x\to\ 0}\frac{7 \cdot cos(25 \cdot x)}{\frac{sin(25 \cdot x)}{x}}=\lim\limits_{x\to\ 0}\frac{7 \cdot cos(25 \cdot x)}{25 \cdot \frac{sin(25 \cdot x)}{25 \cdot x}} = \frac{7}{25}$$
\rozwStop
\odpStart
$\frac{7}{25}$
\odpStop
\testStart
A.$\frac{7}{25}$
B.$\infty$
C.$-\infty$
D.$0$
E.$-\frac{7}{25}$
F.$\frac{25}{7}$
G.$-\frac{25}{7}$
H.$25$
I.$7$
\testStop
\kluczStart
A
\kluczStop



\zadStart{Przykład z Wikieł P 4.3a moja wersja nr 84}


Obliczyć granicę funkcji $\lim\limits_{x\to\ 0}\frac{7 \cdot x}{tan(26 \cdot x)}$.
\zadStop
\rozwStart{Patryk Wirkus}{}
$$\lim\limits_{x\to\ 0}\frac{7 \cdot x}{tan(26 \cdot x)}=\lim\limits_{x\to\ 0}\frac{7 \cdot x \cdot cos(26 \cdot x)}{sin(26 \cdot x)}=\lim\limits_{x\to\ 0}\frac{7 \cdot cos(26 \cdot x)}{\frac{sin(26 \cdot x)}{x}}=\lim\limits_{x\to\ 0}\frac{7 \cdot cos(26 \cdot x)}{26 \cdot \frac{sin(26 \cdot x)}{26 \cdot x}} = \frac{7}{26}$$
\rozwStop
\odpStart
$\frac{7}{26}$
\odpStop
\testStart
A.$\frac{7}{26}$
B.$\infty$
C.$-\infty$
D.$0$
E.$-\frac{7}{26}$
F.$\frac{26}{7}$
G.$-\frac{26}{7}$
H.$26$
I.$7$
\testStop
\kluczStart
A
\kluczStop



\zadStart{Przykład z Wikieł P 4.3a moja wersja nr 85}


Obliczyć granicę funkcji $\lim\limits_{x\to\ 0}\frac{7 \cdot x}{tan(27 \cdot x)}$.
\zadStop
\rozwStart{Patryk Wirkus}{}
$$\lim\limits_{x\to\ 0}\frac{7 \cdot x}{tan(27 \cdot x)}=\lim\limits_{x\to\ 0}\frac{7 \cdot x \cdot cos(27 \cdot x)}{sin(27 \cdot x)}=\lim\limits_{x\to\ 0}\frac{7 \cdot cos(27 \cdot x)}{\frac{sin(27 \cdot x)}{x}}=\lim\limits_{x\to\ 0}\frac{7 \cdot cos(27 \cdot x)}{27 \cdot \frac{sin(27 \cdot x)}{27 \cdot x}} = \frac{7}{27}$$
\rozwStop
\odpStart
$\frac{7}{27}$
\odpStop
\testStart
A.$\frac{7}{27}$
B.$\infty$
C.$-\infty$
D.$0$
E.$-\frac{7}{27}$
F.$\frac{27}{7}$
G.$-\frac{27}{7}$
H.$27$
I.$7$
\testStop
\kluczStart
A
\kluczStop



\zadStart{Przykład z Wikieł P 4.3a moja wersja nr 86}


Obliczyć granicę funkcji $\lim\limits_{x\to\ 0}\frac{7 \cdot x}{tan(29 \cdot x)}$.
\zadStop
\rozwStart{Patryk Wirkus}{}
$$\lim\limits_{x\to\ 0}\frac{7 \cdot x}{tan(29 \cdot x)}=\lim\limits_{x\to\ 0}\frac{7 \cdot x \cdot cos(29 \cdot x)}{sin(29 \cdot x)}=\lim\limits_{x\to\ 0}\frac{7 \cdot cos(29 \cdot x)}{\frac{sin(29 \cdot x)}{x}}=\lim\limits_{x\to\ 0}\frac{7 \cdot cos(29 \cdot x)}{29 \cdot \frac{sin(29 \cdot x)}{29 \cdot x}} = \frac{7}{29}$$
\rozwStop
\odpStart
$\frac{7}{29}$
\odpStop
\testStart
A.$\frac{7}{29}$
B.$\infty$
C.$-\infty$
D.$0$
E.$-\frac{7}{29}$
F.$\frac{29}{7}$
G.$-\frac{29}{7}$
H.$29$
I.$7$
\testStop
\kluczStart
A
\kluczStop



\zadStart{Przykład z Wikieł P 4.3a moja wersja nr 87}


Obliczyć granicę funkcji $\lim\limits_{x\to\ 0}\frac{7 \cdot x}{tan(30 \cdot x)}$.
\zadStop
\rozwStart{Patryk Wirkus}{}
$$\lim\limits_{x\to\ 0}\frac{7 \cdot x}{tan(30 \cdot x)}=\lim\limits_{x\to\ 0}\frac{7 \cdot x \cdot cos(30 \cdot x)}{sin(30 \cdot x)}=\lim\limits_{x\to\ 0}\frac{7 \cdot cos(30 \cdot x)}{\frac{sin(30 \cdot x)}{x}}=\lim\limits_{x\to\ 0}\frac{7 \cdot cos(30 \cdot x)}{30 \cdot \frac{sin(30 \cdot x)}{30 \cdot x}} = \frac{7}{30}$$
\rozwStop
\odpStart
$\frac{7}{30}$
\odpStop
\testStart
A.$\frac{7}{30}$
B.$\infty$
C.$-\infty$
D.$0$
E.$-\frac{7}{30}$
F.$\frac{30}{7}$
G.$-\frac{30}{7}$
H.$30$
I.$7$
\testStop
\kluczStart
A
\kluczStop



\zadStart{Przykład z Wikieł P 4.3a moja wersja nr 88}


Obliczyć granicę funkcji $\lim\limits_{x\to\ 0}\frac{7 \cdot x}{tan(31 \cdot x)}$.
\zadStop
\rozwStart{Patryk Wirkus}{}
$$\lim\limits_{x\to\ 0}\frac{7 \cdot x}{tan(31 \cdot x)}=\lim\limits_{x\to\ 0}\frac{7 \cdot x \cdot cos(31 \cdot x)}{sin(31 \cdot x)}=\lim\limits_{x\to\ 0}\frac{7 \cdot cos(31 \cdot x)}{\frac{sin(31 \cdot x)}{x}}=\lim\limits_{x\to\ 0}\frac{7 \cdot cos(31 \cdot x)}{31 \cdot \frac{sin(31 \cdot x)}{31 \cdot x}} = \frac{7}{31}$$
\rozwStop
\odpStart
$\frac{7}{31}$
\odpStop
\testStart
A.$\frac{7}{31}$
B.$\infty$
C.$-\infty$
D.$0$
E.$-\frac{7}{31}$
F.$\frac{31}{7}$
G.$-\frac{31}{7}$
H.$31$
I.$7$
\testStop
\kluczStart
A
\kluczStop



\zadStart{Przykład z Wikieł P 4.3a moja wersja nr 89}


Obliczyć granicę funkcji $\lim\limits_{x\to\ 0}\frac{7 \cdot x}{tan(32 \cdot x)}$.
\zadStop
\rozwStart{Patryk Wirkus}{}
$$\lim\limits_{x\to\ 0}\frac{7 \cdot x}{tan(32 \cdot x)}=\lim\limits_{x\to\ 0}\frac{7 \cdot x \cdot cos(32 \cdot x)}{sin(32 \cdot x)}=\lim\limits_{x\to\ 0}\frac{7 \cdot cos(32 \cdot x)}{\frac{sin(32 \cdot x)}{x}}=\lim\limits_{x\to\ 0}\frac{7 \cdot cos(32 \cdot x)}{32 \cdot \frac{sin(32 \cdot x)}{32 \cdot x}} = \frac{7}{32}$$
\rozwStop
\odpStart
$\frac{7}{32}$
\odpStop
\testStart
A.$\frac{7}{32}$
B.$\infty$
C.$-\infty$
D.$0$
E.$-\frac{7}{32}$
F.$\frac{32}{7}$
G.$-\frac{32}{7}$
H.$32$
I.$7$
\testStop
\kluczStart
A
\kluczStop



\zadStart{Przykład z Wikieł P 4.3a moja wersja nr 90}


Obliczyć granicę funkcji $\lim\limits_{x\to\ 0}\frac{7 \cdot x}{tan(33 \cdot x)}$.
\zadStop
\rozwStart{Patryk Wirkus}{}
$$\lim\limits_{x\to\ 0}\frac{7 \cdot x}{tan(33 \cdot x)}=\lim\limits_{x\to\ 0}\frac{7 \cdot x \cdot cos(33 \cdot x)}{sin(33 \cdot x)}=\lim\limits_{x\to\ 0}\frac{7 \cdot cos(33 \cdot x)}{\frac{sin(33 \cdot x)}{x}}=\lim\limits_{x\to\ 0}\frac{7 \cdot cos(33 \cdot x)}{33 \cdot \frac{sin(33 \cdot x)}{33 \cdot x}} = \frac{7}{33}$$
\rozwStop
\odpStart
$\frac{7}{33}$
\odpStop
\testStart
A.$\frac{7}{33}$
B.$\infty$
C.$-\infty$
D.$0$
E.$-\frac{7}{33}$
F.$\frac{33}{7}$
G.$-\frac{33}{7}$
H.$33$
I.$7$
\testStop
\kluczStart
A
\kluczStop



\zadStart{Przykład z Wikieł P 4.3a moja wersja nr 91}


Obliczyć granicę funkcji $\lim\limits_{x\to\ 0}\frac{7 \cdot x}{tan(34 \cdot x)}$.
\zadStop
\rozwStart{Patryk Wirkus}{}
$$\lim\limits_{x\to\ 0}\frac{7 \cdot x}{tan(34 \cdot x)}=\lim\limits_{x\to\ 0}\frac{7 \cdot x \cdot cos(34 \cdot x)}{sin(34 \cdot x)}=\lim\limits_{x\to\ 0}\frac{7 \cdot cos(34 \cdot x)}{\frac{sin(34 \cdot x)}{x}}=\lim\limits_{x\to\ 0}\frac{7 \cdot cos(34 \cdot x)}{34 \cdot \frac{sin(34 \cdot x)}{34 \cdot x}} = \frac{7}{34}$$
\rozwStop
\odpStart
$\frac{7}{34}$
\odpStop
\testStart
A.$\frac{7}{34}$
B.$\infty$
C.$-\infty$
D.$0$
E.$-\frac{7}{34}$
F.$\frac{34}{7}$
G.$-\frac{34}{7}$
H.$34$
I.$7$
\testStop
\kluczStart
A
\kluczStop



\zadStart{Przykład z Wikieł P 4.3a moja wersja nr 92}


Obliczyć granicę funkcji $\lim\limits_{x\to\ 0}\frac{7 \cdot x}{tan(36 \cdot x)}$.
\zadStop
\rozwStart{Patryk Wirkus}{}
$$\lim\limits_{x\to\ 0}\frac{7 \cdot x}{tan(36 \cdot x)}=\lim\limits_{x\to\ 0}\frac{7 \cdot x \cdot cos(36 \cdot x)}{sin(36 \cdot x)}=\lim\limits_{x\to\ 0}\frac{7 \cdot cos(36 \cdot x)}{\frac{sin(36 \cdot x)}{x}}=\lim\limits_{x\to\ 0}\frac{7 \cdot cos(36 \cdot x)}{36 \cdot \frac{sin(36 \cdot x)}{36 \cdot x}} = \frac{7}{36}$$
\rozwStop
\odpStart
$\frac{7}{36}$
\odpStop
\testStart
A.$\frac{7}{36}$
B.$\infty$
C.$-\infty$
D.$0$
E.$-\frac{7}{36}$
F.$\frac{36}{7}$
G.$-\frac{36}{7}$
H.$36$
I.$7$
\testStop
\kluczStart
A
\kluczStop



\zadStart{Przykład z Wikieł P 4.3a moja wersja nr 93}


Obliczyć granicę funkcji $\lim\limits_{x\to\ 0}\frac{7 \cdot x}{tan(37 \cdot x)}$.
\zadStop
\rozwStart{Patryk Wirkus}{}
$$\lim\limits_{x\to\ 0}\frac{7 \cdot x}{tan(37 \cdot x)}=\lim\limits_{x\to\ 0}\frac{7 \cdot x \cdot cos(37 \cdot x)}{sin(37 \cdot x)}=\lim\limits_{x\to\ 0}\frac{7 \cdot cos(37 \cdot x)}{\frac{sin(37 \cdot x)}{x}}=\lim\limits_{x\to\ 0}\frac{7 \cdot cos(37 \cdot x)}{37 \cdot \frac{sin(37 \cdot x)}{37 \cdot x}} = \frac{7}{37}$$
\rozwStop
\odpStart
$\frac{7}{37}$
\odpStop
\testStart
A.$\frac{7}{37}$
B.$\infty$
C.$-\infty$
D.$0$
E.$-\frac{7}{37}$
F.$\frac{37}{7}$
G.$-\frac{37}{7}$
H.$37$
I.$7$
\testStop
\kluczStart
A
\kluczStop



\zadStart{Przykład z Wikieł P 4.3a moja wersja nr 94}


Obliczyć granicę funkcji $\lim\limits_{x\to\ 0}\frac{7 \cdot x}{tan(38 \cdot x)}$.
\zadStop
\rozwStart{Patryk Wirkus}{}
$$\lim\limits_{x\to\ 0}\frac{7 \cdot x}{tan(38 \cdot x)}=\lim\limits_{x\to\ 0}\frac{7 \cdot x \cdot cos(38 \cdot x)}{sin(38 \cdot x)}=\lim\limits_{x\to\ 0}\frac{7 \cdot cos(38 \cdot x)}{\frac{sin(38 \cdot x)}{x}}=\lim\limits_{x\to\ 0}\frac{7 \cdot cos(38 \cdot x)}{38 \cdot \frac{sin(38 \cdot x)}{38 \cdot x}} = \frac{7}{38}$$
\rozwStop
\odpStart
$\frac{7}{38}$
\odpStop
\testStart
A.$\frac{7}{38}$
B.$\infty$
C.$-\infty$
D.$0$
E.$-\frac{7}{38}$
F.$\frac{38}{7}$
G.$-\frac{38}{7}$
H.$38$
I.$7$
\testStop
\kluczStart
A
\kluczStop



\zadStart{Przykład z Wikieł P 4.3a moja wersja nr 95}


Obliczyć granicę funkcji $\lim\limits_{x\to\ 0}\frac{7 \cdot x}{tan(39 \cdot x)}$.
\zadStop
\rozwStart{Patryk Wirkus}{}
$$\lim\limits_{x\to\ 0}\frac{7 \cdot x}{tan(39 \cdot x)}=\lim\limits_{x\to\ 0}\frac{7 \cdot x \cdot cos(39 \cdot x)}{sin(39 \cdot x)}=\lim\limits_{x\to\ 0}\frac{7 \cdot cos(39 \cdot x)}{\frac{sin(39 \cdot x)}{x}}=\lim\limits_{x\to\ 0}\frac{7 \cdot cos(39 \cdot x)}{39 \cdot \frac{sin(39 \cdot x)}{39 \cdot x}} = \frac{7}{39}$$
\rozwStop
\odpStart
$\frac{7}{39}$
\odpStop
\testStart
A.$\frac{7}{39}$
B.$\infty$
C.$-\infty$
D.$0$
E.$-\frac{7}{39}$
F.$\frac{39}{7}$
G.$-\frac{39}{7}$
H.$39$
I.$7$
\testStop
\kluczStart
A
\kluczStop



\zadStart{Przykład z Wikieł P 4.3a moja wersja nr 96}


Obliczyć granicę funkcji $\lim\limits_{x\to\ 0}\frac{7 \cdot x}{tan(40 \cdot x)}$.
\zadStop
\rozwStart{Patryk Wirkus}{}
$$\lim\limits_{x\to\ 0}\frac{7 \cdot x}{tan(40 \cdot x)}=\lim\limits_{x\to\ 0}\frac{7 \cdot x \cdot cos(40 \cdot x)}{sin(40 \cdot x)}=\lim\limits_{x\to\ 0}\frac{7 \cdot cos(40 \cdot x)}{\frac{sin(40 \cdot x)}{x}}=\lim\limits_{x\to\ 0}\frac{7 \cdot cos(40 \cdot x)}{40 \cdot \frac{sin(40 \cdot x)}{40 \cdot x}} = \frac{7}{40}$$
\rozwStop
\odpStart
$\frac{7}{40}$
\odpStop
\testStart
A.$\frac{7}{40}$
B.$\infty$
C.$-\infty$
D.$0$
E.$-\frac{7}{40}$
F.$\frac{40}{7}$
G.$-\frac{40}{7}$
H.$40$
I.$7$
\testStop
\kluczStart
A
\kluczStop



\zadStart{Przykład z Wikieł P 4.3a moja wersja nr 97}


Obliczyć granicę funkcji $\lim\limits_{x\to\ 0}\frac{8 \cdot x}{tan(3 \cdot x)}$.
\zadStop
\rozwStart{Patryk Wirkus}{}
$$\lim\limits_{x\to\ 0}\frac{8 \cdot x}{tan(3 \cdot x)}=\lim\limits_{x\to\ 0}\frac{8 \cdot x \cdot cos(3 \cdot x)}{sin(3 \cdot x)}=\lim\limits_{x\to\ 0}\frac{8 \cdot cos(3 \cdot x)}{\frac{sin(3 \cdot x)}{x}}=\lim\limits_{x\to\ 0}\frac{8 \cdot cos(3 \cdot x)}{3 \cdot \frac{sin(3 \cdot x)}{3 \cdot x}} = \frac{8}{3}$$
\rozwStop
\odpStart
$\frac{8}{3}$
\odpStop
\testStart
A.$\frac{8}{3}$
B.$\infty$
C.$-\infty$
D.$0$
E.$-\frac{8}{3}$
F.$\frac{3}{8}$
G.$-\frac{3}{8}$
H.$3$
I.$8$
\testStop
\kluczStart
A
\kluczStop



\zadStart{Przykład z Wikieł P 4.3a moja wersja nr 98}


Obliczyć granicę funkcji $\lim\limits_{x\to\ 0}\frac{8 \cdot x}{tan(5 \cdot x)}$.
\zadStop
\rozwStart{Patryk Wirkus}{}
$$\lim\limits_{x\to\ 0}\frac{8 \cdot x}{tan(5 \cdot x)}=\lim\limits_{x\to\ 0}\frac{8 \cdot x \cdot cos(5 \cdot x)}{sin(5 \cdot x)}=\lim\limits_{x\to\ 0}\frac{8 \cdot cos(5 \cdot x)}{\frac{sin(5 \cdot x)}{x}}=\lim\limits_{x\to\ 0}\frac{8 \cdot cos(5 \cdot x)}{5 \cdot \frac{sin(5 \cdot x)}{5 \cdot x}} = \frac{8}{5}$$
\rozwStop
\odpStart
$\frac{8}{5}$
\odpStop
\testStart
A.$\frac{8}{5}$
B.$\infty$
C.$-\infty$
D.$0$
E.$-\frac{8}{5}$
F.$\frac{5}{8}$
G.$-\frac{5}{8}$
H.$5$
I.$8$
\testStop
\kluczStart
A
\kluczStop



\zadStart{Przykład z Wikieł P 4.3a moja wersja nr 99}


Obliczyć granicę funkcji $\lim\limits_{x\to\ 0}\frac{8 \cdot x}{tan(7 \cdot x)}$.
\zadStop
\rozwStart{Patryk Wirkus}{}
$$\lim\limits_{x\to\ 0}\frac{8 \cdot x}{tan(7 \cdot x)}=\lim\limits_{x\to\ 0}\frac{8 \cdot x \cdot cos(7 \cdot x)}{sin(7 \cdot x)}=\lim\limits_{x\to\ 0}\frac{8 \cdot cos(7 \cdot x)}{\frac{sin(7 \cdot x)}{x}}=\lim\limits_{x\to\ 0}\frac{8 \cdot cos(7 \cdot x)}{7 \cdot \frac{sin(7 \cdot x)}{7 \cdot x}} = \frac{8}{7}$$
\rozwStop
\odpStart
$\frac{8}{7}$
\odpStop
\testStart
A.$\frac{8}{7}$
B.$\infty$
C.$-\infty$
D.$0$
E.$-\frac{8}{7}$
F.$\frac{7}{8}$
G.$-\frac{7}{8}$
H.$7$
I.$8$
\testStop
\kluczStart
A
\kluczStop



\zadStart{Przykład z Wikieł P 4.3a moja wersja nr 100}


Obliczyć granicę funkcji $\lim\limits_{x\to\ 0}\frac{8 \cdot x}{tan(9 \cdot x)}$.
\zadStop
\rozwStart{Patryk Wirkus}{}
$$\lim\limits_{x\to\ 0}\frac{8 \cdot x}{tan(9 \cdot x)}=\lim\limits_{x\to\ 0}\frac{8 \cdot x \cdot cos(9 \cdot x)}{sin(9 \cdot x)}=\lim\limits_{x\to\ 0}\frac{8 \cdot cos(9 \cdot x)}{\frac{sin(9 \cdot x)}{x}}=\lim\limits_{x\to\ 0}\frac{8 \cdot cos(9 \cdot x)}{9 \cdot \frac{sin(9 \cdot x)}{9 \cdot x}} = \frac{8}{9}$$
\rozwStop
\odpStart
$\frac{8}{9}$
\odpStop
\testStart
A.$\frac{8}{9}$
B.$\infty$
C.$-\infty$
D.$0$
E.$-\frac{8}{9}$
F.$\frac{9}{8}$
G.$-\frac{9}{8}$
H.$9$
I.$8$
\testStop
\kluczStart
A
\kluczStop



\zadStart{Przykład z Wikieł P 4.3a moja wersja nr 101}


Obliczyć granicę funkcji $\lim\limits_{x\to\ 0}\frac{8 \cdot x}{tan(11 \cdot x)}$.
\zadStop
\rozwStart{Patryk Wirkus}{}
$$\lim\limits_{x\to\ 0}\frac{8 \cdot x}{tan(11 \cdot x)}=\lim\limits_{x\to\ 0}\frac{8 \cdot x \cdot cos(11 \cdot x)}{sin(11 \cdot x)}=\lim\limits_{x\to\ 0}\frac{8 \cdot cos(11 \cdot x)}{\frac{sin(11 \cdot x)}{x}}=\lim\limits_{x\to\ 0}\frac{8 \cdot cos(11 \cdot x)}{11 \cdot \frac{sin(11 \cdot x)}{11 \cdot x}} = \frac{8}{11}$$
\rozwStop
\odpStart
$\frac{8}{11}$
\odpStop
\testStart
A.$\frac{8}{11}$
B.$\infty$
C.$-\infty$
D.$0$
E.$-\frac{8}{11}$
F.$\frac{11}{8}$
G.$-\frac{11}{8}$
H.$11$
I.$8$
\testStop
\kluczStart
A
\kluczStop



\zadStart{Przykład z Wikieł P 4.3a moja wersja nr 102}


Obliczyć granicę funkcji $\lim\limits_{x\to\ 0}\frac{8 \cdot x}{tan(13 \cdot x)}$.
\zadStop
\rozwStart{Patryk Wirkus}{}
$$\lim\limits_{x\to\ 0}\frac{8 \cdot x}{tan(13 \cdot x)}=\lim\limits_{x\to\ 0}\frac{8 \cdot x \cdot cos(13 \cdot x)}{sin(13 \cdot x)}=\lim\limits_{x\to\ 0}\frac{8 \cdot cos(13 \cdot x)}{\frac{sin(13 \cdot x)}{x}}=\lim\limits_{x\to\ 0}\frac{8 \cdot cos(13 \cdot x)}{13 \cdot \frac{sin(13 \cdot x)}{13 \cdot x}} = \frac{8}{13}$$
\rozwStop
\odpStart
$\frac{8}{13}$
\odpStop
\testStart
A.$\frac{8}{13}$
B.$\infty$
C.$-\infty$
D.$0$
E.$-\frac{8}{13}$
F.$\frac{13}{8}$
G.$-\frac{13}{8}$
H.$13$
I.$8$
\testStop
\kluczStart
A
\kluczStop



\zadStart{Przykład z Wikieł P 4.3a moja wersja nr 103}


Obliczyć granicę funkcji $\lim\limits_{x\to\ 0}\frac{8 \cdot x}{tan(15 \cdot x)}$.
\zadStop
\rozwStart{Patryk Wirkus}{}
$$\lim\limits_{x\to\ 0}\frac{8 \cdot x}{tan(15 \cdot x)}=\lim\limits_{x\to\ 0}\frac{8 \cdot x \cdot cos(15 \cdot x)}{sin(15 \cdot x)}=\lim\limits_{x\to\ 0}\frac{8 \cdot cos(15 \cdot x)}{\frac{sin(15 \cdot x)}{x}}=\lim\limits_{x\to\ 0}\frac{8 \cdot cos(15 \cdot x)}{15 \cdot \frac{sin(15 \cdot x)}{15 \cdot x}} = \frac{8}{15}$$
\rozwStop
\odpStart
$\frac{8}{15}$
\odpStop
\testStart
A.$\frac{8}{15}$
B.$\infty$
C.$-\infty$
D.$0$
E.$-\frac{8}{15}$
F.$\frac{15}{8}$
G.$-\frac{15}{8}$
H.$15$
I.$8$
\testStop
\kluczStart
A
\kluczStop



\zadStart{Przykład z Wikieł P 4.3a moja wersja nr 104}


Obliczyć granicę funkcji $\lim\limits_{x\to\ 0}\frac{8 \cdot x}{tan(17 \cdot x)}$.
\zadStop
\rozwStart{Patryk Wirkus}{}
$$\lim\limits_{x\to\ 0}\frac{8 \cdot x}{tan(17 \cdot x)}=\lim\limits_{x\to\ 0}\frac{8 \cdot x \cdot cos(17 \cdot x)}{sin(17 \cdot x)}=\lim\limits_{x\to\ 0}\frac{8 \cdot cos(17 \cdot x)}{\frac{sin(17 \cdot x)}{x}}=\lim\limits_{x\to\ 0}\frac{8 \cdot cos(17 \cdot x)}{17 \cdot \frac{sin(17 \cdot x)}{17 \cdot x}} = \frac{8}{17}$$
\rozwStop
\odpStart
$\frac{8}{17}$
\odpStop
\testStart
A.$\frac{8}{17}$
B.$\infty$
C.$-\infty$
D.$0$
E.$-\frac{8}{17}$
F.$\frac{17}{8}$
G.$-\frac{17}{8}$
H.$17$
I.$8$
\testStop
\kluczStart
A
\kluczStop



\zadStart{Przykład z Wikieł P 4.3a moja wersja nr 105}


Obliczyć granicę funkcji $\lim\limits_{x\to\ 0}\frac{8 \cdot x}{tan(19 \cdot x)}$.
\zadStop
\rozwStart{Patryk Wirkus}{}
$$\lim\limits_{x\to\ 0}\frac{8 \cdot x}{tan(19 \cdot x)}=\lim\limits_{x\to\ 0}\frac{8 \cdot x \cdot cos(19 \cdot x)}{sin(19 \cdot x)}=\lim\limits_{x\to\ 0}\frac{8 \cdot cos(19 \cdot x)}{\frac{sin(19 \cdot x)}{x}}=\lim\limits_{x\to\ 0}\frac{8 \cdot cos(19 \cdot x)}{19 \cdot \frac{sin(19 \cdot x)}{19 \cdot x}} = \frac{8}{19}$$
\rozwStop
\odpStart
$\frac{8}{19}$
\odpStop
\testStart
A.$\frac{8}{19}$
B.$\infty$
C.$-\infty$
D.$0$
E.$-\frac{8}{19}$
F.$\frac{19}{8}$
G.$-\frac{19}{8}$
H.$19$
I.$8$
\testStop
\kluczStart
A
\kluczStop



\zadStart{Przykład z Wikieł P 4.3a moja wersja nr 106}


Obliczyć granicę funkcji $\lim\limits_{x\to\ 0}\frac{8 \cdot x}{tan(21 \cdot x)}$.
\zadStop
\rozwStart{Patryk Wirkus}{}
$$\lim\limits_{x\to\ 0}\frac{8 \cdot x}{tan(21 \cdot x)}=\lim\limits_{x\to\ 0}\frac{8 \cdot x \cdot cos(21 \cdot x)}{sin(21 \cdot x)}=\lim\limits_{x\to\ 0}\frac{8 \cdot cos(21 \cdot x)}{\frac{sin(21 \cdot x)}{x}}=\lim\limits_{x\to\ 0}\frac{8 \cdot cos(21 \cdot x)}{21 \cdot \frac{sin(21 \cdot x)}{21 \cdot x}} = \frac{8}{21}$$
\rozwStop
\odpStart
$\frac{8}{21}$
\odpStop
\testStart
A.$\frac{8}{21}$
B.$\infty$
C.$-\infty$
D.$0$
E.$-\frac{8}{21}$
F.$\frac{21}{8}$
G.$-\frac{21}{8}$
H.$21$
I.$8$
\testStop
\kluczStart
A
\kluczStop



\zadStart{Przykład z Wikieł P 4.3a moja wersja nr 107}


Obliczyć granicę funkcji $\lim\limits_{x\to\ 0}\frac{8 \cdot x}{tan(23 \cdot x)}$.
\zadStop
\rozwStart{Patryk Wirkus}{}
$$\lim\limits_{x\to\ 0}\frac{8 \cdot x}{tan(23 \cdot x)}=\lim\limits_{x\to\ 0}\frac{8 \cdot x \cdot cos(23 \cdot x)}{sin(23 \cdot x)}=\lim\limits_{x\to\ 0}\frac{8 \cdot cos(23 \cdot x)}{\frac{sin(23 \cdot x)}{x}}=\lim\limits_{x\to\ 0}\frac{8 \cdot cos(23 \cdot x)}{23 \cdot \frac{sin(23 \cdot x)}{23 \cdot x}} = \frac{8}{23}$$
\rozwStop
\odpStart
$\frac{8}{23}$
\odpStop
\testStart
A.$\frac{8}{23}$
B.$\infty$
C.$-\infty$
D.$0$
E.$-\frac{8}{23}$
F.$\frac{23}{8}$
G.$-\frac{23}{8}$
H.$23$
I.$8$
\testStop
\kluczStart
A
\kluczStop



\zadStart{Przykład z Wikieł P 4.3a moja wersja nr 108}


Obliczyć granicę funkcji $\lim\limits_{x\to\ 0}\frac{8 \cdot x}{tan(25 \cdot x)}$.
\zadStop
\rozwStart{Patryk Wirkus}{}
$$\lim\limits_{x\to\ 0}\frac{8 \cdot x}{tan(25 \cdot x)}=\lim\limits_{x\to\ 0}\frac{8 \cdot x \cdot cos(25 \cdot x)}{sin(25 \cdot x)}=\lim\limits_{x\to\ 0}\frac{8 \cdot cos(25 \cdot x)}{\frac{sin(25 \cdot x)}{x}}=\lim\limits_{x\to\ 0}\frac{8 \cdot cos(25 \cdot x)}{25 \cdot \frac{sin(25 \cdot x)}{25 \cdot x}} = \frac{8}{25}$$
\rozwStop
\odpStart
$\frac{8}{25}$
\odpStop
\testStart
A.$\frac{8}{25}$
B.$\infty$
C.$-\infty$
D.$0$
E.$-\frac{8}{25}$
F.$\frac{25}{8}$
G.$-\frac{25}{8}$
H.$25$
I.$8$
\testStop
\kluczStart
A
\kluczStop



\zadStart{Przykład z Wikieł P 4.3a moja wersja nr 109}


Obliczyć granicę funkcji $\lim\limits_{x\to\ 0}\frac{8 \cdot x}{tan(27 \cdot x)}$.
\zadStop
\rozwStart{Patryk Wirkus}{}
$$\lim\limits_{x\to\ 0}\frac{8 \cdot x}{tan(27 \cdot x)}=\lim\limits_{x\to\ 0}\frac{8 \cdot x \cdot cos(27 \cdot x)}{sin(27 \cdot x)}=\lim\limits_{x\to\ 0}\frac{8 \cdot cos(27 \cdot x)}{\frac{sin(27 \cdot x)}{x}}=\lim\limits_{x\to\ 0}\frac{8 \cdot cos(27 \cdot x)}{27 \cdot \frac{sin(27 \cdot x)}{27 \cdot x}} = \frac{8}{27}$$
\rozwStop
\odpStart
$\frac{8}{27}$
\odpStop
\testStart
A.$\frac{8}{27}$
B.$\infty$
C.$-\infty$
D.$0$
E.$-\frac{8}{27}$
F.$\frac{27}{8}$
G.$-\frac{27}{8}$
H.$27$
I.$8$
\testStop
\kluczStart
A
\kluczStop



\zadStart{Przykład z Wikieł P 4.3a moja wersja nr 110}


Obliczyć granicę funkcji $\lim\limits_{x\to\ 0}\frac{8 \cdot x}{tan(29 \cdot x)}$.
\zadStop
\rozwStart{Patryk Wirkus}{}
$$\lim\limits_{x\to\ 0}\frac{8 \cdot x}{tan(29 \cdot x)}=\lim\limits_{x\to\ 0}\frac{8 \cdot x \cdot cos(29 \cdot x)}{sin(29 \cdot x)}=\lim\limits_{x\to\ 0}\frac{8 \cdot cos(29 \cdot x)}{\frac{sin(29 \cdot x)}{x}}=\lim\limits_{x\to\ 0}\frac{8 \cdot cos(29 \cdot x)}{29 \cdot \frac{sin(29 \cdot x)}{29 \cdot x}} = \frac{8}{29}$$
\rozwStop
\odpStart
$\frac{8}{29}$
\odpStop
\testStart
A.$\frac{8}{29}$
B.$\infty$
C.$-\infty$
D.$0$
E.$-\frac{8}{29}$
F.$\frac{29}{8}$
G.$-\frac{29}{8}$
H.$29$
I.$8$
\testStop
\kluczStart
A
\kluczStop



\zadStart{Przykład z Wikieł P 4.3a moja wersja nr 111}


Obliczyć granicę funkcji $\lim\limits_{x\to\ 0}\frac{8 \cdot x}{tan(31 \cdot x)}$.
\zadStop
\rozwStart{Patryk Wirkus}{}
$$\lim\limits_{x\to\ 0}\frac{8 \cdot x}{tan(31 \cdot x)}=\lim\limits_{x\to\ 0}\frac{8 \cdot x \cdot cos(31 \cdot x)}{sin(31 \cdot x)}=\lim\limits_{x\to\ 0}\frac{8 \cdot cos(31 \cdot x)}{\frac{sin(31 \cdot x)}{x}}=\lim\limits_{x\to\ 0}\frac{8 \cdot cos(31 \cdot x)}{31 \cdot \frac{sin(31 \cdot x)}{31 \cdot x}} = \frac{8}{31}$$
\rozwStop
\odpStart
$\frac{8}{31}$
\odpStop
\testStart
A.$\frac{8}{31}$
B.$\infty$
C.$-\infty$
D.$0$
E.$-\frac{8}{31}$
F.$\frac{31}{8}$
G.$-\frac{31}{8}$
H.$31$
I.$8$
\testStop
\kluczStart
A
\kluczStop



\zadStart{Przykład z Wikieł P 4.3a moja wersja nr 112}


Obliczyć granicę funkcji $\lim\limits_{x\to\ 0}\frac{8 \cdot x}{tan(33 \cdot x)}$.
\zadStop
\rozwStart{Patryk Wirkus}{}
$$\lim\limits_{x\to\ 0}\frac{8 \cdot x}{tan(33 \cdot x)}=\lim\limits_{x\to\ 0}\frac{8 \cdot x \cdot cos(33 \cdot x)}{sin(33 \cdot x)}=\lim\limits_{x\to\ 0}\frac{8 \cdot cos(33 \cdot x)}{\frac{sin(33 \cdot x)}{x}}=\lim\limits_{x\to\ 0}\frac{8 \cdot cos(33 \cdot x)}{33 \cdot \frac{sin(33 \cdot x)}{33 \cdot x}} = \frac{8}{33}$$
\rozwStop
\odpStart
$\frac{8}{33}$
\odpStop
\testStart
A.$\frac{8}{33}$
B.$\infty$
C.$-\infty$
D.$0$
E.$-\frac{8}{33}$
F.$\frac{33}{8}$
G.$-\frac{33}{8}$
H.$33$
I.$8$
\testStop
\kluczStart
A
\kluczStop



\zadStart{Przykład z Wikieł P 4.3a moja wersja nr 113}


Obliczyć granicę funkcji $\lim\limits_{x\to\ 0}\frac{8 \cdot x}{tan(35 \cdot x)}$.
\zadStop
\rozwStart{Patryk Wirkus}{}
$$\lim\limits_{x\to\ 0}\frac{8 \cdot x}{tan(35 \cdot x)}=\lim\limits_{x\to\ 0}\frac{8 \cdot x \cdot cos(35 \cdot x)}{sin(35 \cdot x)}=\lim\limits_{x\to\ 0}\frac{8 \cdot cos(35 \cdot x)}{\frac{sin(35 \cdot x)}{x}}=\lim\limits_{x\to\ 0}\frac{8 \cdot cos(35 \cdot x)}{35 \cdot \frac{sin(35 \cdot x)}{35 \cdot x}} = \frac{8}{35}$$
\rozwStop
\odpStart
$\frac{8}{35}$
\odpStop
\testStart
A.$\frac{8}{35}$
B.$\infty$
C.$-\infty$
D.$0$
E.$-\frac{8}{35}$
F.$\frac{35}{8}$
G.$-\frac{35}{8}$
H.$35$
I.$8$
\testStop
\kluczStart
A
\kluczStop



\zadStart{Przykład z Wikieł P 4.3a moja wersja nr 114}


Obliczyć granicę funkcji $\lim\limits_{x\to\ 0}\frac{8 \cdot x}{tan(37 \cdot x)}$.
\zadStop
\rozwStart{Patryk Wirkus}{}
$$\lim\limits_{x\to\ 0}\frac{8 \cdot x}{tan(37 \cdot x)}=\lim\limits_{x\to\ 0}\frac{8 \cdot x \cdot cos(37 \cdot x)}{sin(37 \cdot x)}=\lim\limits_{x\to\ 0}\frac{8 \cdot cos(37 \cdot x)}{\frac{sin(37 \cdot x)}{x}}=\lim\limits_{x\to\ 0}\frac{8 \cdot cos(37 \cdot x)}{37 \cdot \frac{sin(37 \cdot x)}{37 \cdot x}} = \frac{8}{37}$$
\rozwStop
\odpStart
$\frac{8}{37}$
\odpStop
\testStart
A.$\frac{8}{37}$
B.$\infty$
C.$-\infty$
D.$0$
E.$-\frac{8}{37}$
F.$\frac{37}{8}$
G.$-\frac{37}{8}$
H.$37$
I.$8$
\testStop
\kluczStart
A
\kluczStop



\zadStart{Przykład z Wikieł P 4.3a moja wersja nr 115}


Obliczyć granicę funkcji $\lim\limits_{x\to\ 0}\frac{8 \cdot x}{tan(39 \cdot x)}$.
\zadStop
\rozwStart{Patryk Wirkus}{}
$$\lim\limits_{x\to\ 0}\frac{8 \cdot x}{tan(39 \cdot x)}=\lim\limits_{x\to\ 0}\frac{8 \cdot x \cdot cos(39 \cdot x)}{sin(39 \cdot x)}=\lim\limits_{x\to\ 0}\frac{8 \cdot cos(39 \cdot x)}{\frac{sin(39 \cdot x)}{x}}=\lim\limits_{x\to\ 0}\frac{8 \cdot cos(39 \cdot x)}{39 \cdot \frac{sin(39 \cdot x)}{39 \cdot x}} = \frac{8}{39}$$
\rozwStop
\odpStart
$\frac{8}{39}$
\odpStop
\testStart
A.$\frac{8}{39}$
B.$\infty$
C.$-\infty$
D.$0$
E.$-\frac{8}{39}$
F.$\frac{39}{8}$
G.$-\frac{39}{8}$
H.$39$
I.$8$
\testStop
\kluczStart
A
\kluczStop



\zadStart{Przykład z Wikieł P 4.3a moja wersja nr 116}


Obliczyć granicę funkcji $\lim\limits_{x\to\ 0}\frac{9 \cdot x}{tan(2 \cdot x)}$.
\zadStop
\rozwStart{Patryk Wirkus}{}
$$\lim\limits_{x\to\ 0}\frac{9 \cdot x}{tan(2 \cdot x)}=\lim\limits_{x\to\ 0}\frac{9 \cdot x \cdot cos(2 \cdot x)}{sin(2 \cdot x)}=\lim\limits_{x\to\ 0}\frac{9 \cdot cos(2 \cdot x)}{\frac{sin(2 \cdot x)}{x}}=\lim\limits_{x\to\ 0}\frac{9 \cdot cos(2 \cdot x)}{2 \cdot \frac{sin(2 \cdot x)}{2 \cdot x}} = \frac{9}{2}$$
\rozwStop
\odpStart
$\frac{9}{2}$
\odpStop
\testStart
A.$\frac{9}{2}$
B.$\infty$
C.$-\infty$
D.$0$
E.$-\frac{9}{2}$
F.$\frac{2}{9}$
G.$-\frac{2}{9}$
H.$2$
I.$9$
\testStop
\kluczStart
A
\kluczStop



\zadStart{Przykład z Wikieł P 4.3a moja wersja nr 117}


Obliczyć granicę funkcji $\lim\limits_{x\to\ 0}\frac{9 \cdot x}{tan(4 \cdot x)}$.
\zadStop
\rozwStart{Patryk Wirkus}{}
$$\lim\limits_{x\to\ 0}\frac{9 \cdot x}{tan(4 \cdot x)}=\lim\limits_{x\to\ 0}\frac{9 \cdot x \cdot cos(4 \cdot x)}{sin(4 \cdot x)}=\lim\limits_{x\to\ 0}\frac{9 \cdot cos(4 \cdot x)}{\frac{sin(4 \cdot x)}{x}}=\lim\limits_{x\to\ 0}\frac{9 \cdot cos(4 \cdot x)}{4 \cdot \frac{sin(4 \cdot x)}{4 \cdot x}} = \frac{9}{4}$$
\rozwStop
\odpStart
$\frac{9}{4}$
\odpStop
\testStart
A.$\frac{9}{4}$
B.$\infty$
C.$-\infty$
D.$0$
E.$-\frac{9}{4}$
F.$\frac{4}{9}$
G.$-\frac{4}{9}$
H.$4$
I.$9$
\testStop
\kluczStart
A
\kluczStop



\zadStart{Przykład z Wikieł P 4.3a moja wersja nr 118}


Obliczyć granicę funkcji $\lim\limits_{x\to\ 0}\frac{9 \cdot x}{tan(5 \cdot x)}$.
\zadStop
\rozwStart{Patryk Wirkus}{}
$$\lim\limits_{x\to\ 0}\frac{9 \cdot x}{tan(5 \cdot x)}=\lim\limits_{x\to\ 0}\frac{9 \cdot x \cdot cos(5 \cdot x)}{sin(5 \cdot x)}=\lim\limits_{x\to\ 0}\frac{9 \cdot cos(5 \cdot x)}{\frac{sin(5 \cdot x)}{x}}=\lim\limits_{x\to\ 0}\frac{9 \cdot cos(5 \cdot x)}{5 \cdot \frac{sin(5 \cdot x)}{5 \cdot x}} = \frac{9}{5}$$
\rozwStop
\odpStart
$\frac{9}{5}$
\odpStop
\testStart
A.$\frac{9}{5}$
B.$\infty$
C.$-\infty$
D.$0$
E.$-\frac{9}{5}$
F.$\frac{5}{9}$
G.$-\frac{5}{9}$
H.$5$
I.$9$
\testStop
\kluczStart
A
\kluczStop



\zadStart{Przykład z Wikieł P 4.3a moja wersja nr 119}


Obliczyć granicę funkcji $\lim\limits_{x\to\ 0}\frac{9 \cdot x}{tan(7 \cdot x)}$.
\zadStop
\rozwStart{Patryk Wirkus}{}
$$\lim\limits_{x\to\ 0}\frac{9 \cdot x}{tan(7 \cdot x)}=\lim\limits_{x\to\ 0}\frac{9 \cdot x \cdot cos(7 \cdot x)}{sin(7 \cdot x)}=\lim\limits_{x\to\ 0}\frac{9 \cdot cos(7 \cdot x)}{\frac{sin(7 \cdot x)}{x}}=\lim\limits_{x\to\ 0}\frac{9 \cdot cos(7 \cdot x)}{7 \cdot \frac{sin(7 \cdot x)}{7 \cdot x}} = \frac{9}{7}$$
\rozwStop
\odpStart
$\frac{9}{7}$
\odpStop
\testStart
A.$\frac{9}{7}$
B.$\infty$
C.$-\infty$
D.$0$
E.$-\frac{9}{7}$
F.$\frac{7}{9}$
G.$-\frac{7}{9}$
H.$7$
I.$9$
\testStop
\kluczStart
A
\kluczStop



\zadStart{Przykład z Wikieł P 4.3a moja wersja nr 120}


Obliczyć granicę funkcji $\lim\limits_{x\to\ 0}\frac{9 \cdot x}{tan(8 \cdot x)}$.
\zadStop
\rozwStart{Patryk Wirkus}{}
$$\lim\limits_{x\to\ 0}\frac{9 \cdot x}{tan(8 \cdot x)}=\lim\limits_{x\to\ 0}\frac{9 \cdot x \cdot cos(8 \cdot x)}{sin(8 \cdot x)}=\lim\limits_{x\to\ 0}\frac{9 \cdot cos(8 \cdot x)}{\frac{sin(8 \cdot x)}{x}}=\lim\limits_{x\to\ 0}\frac{9 \cdot cos(8 \cdot x)}{8 \cdot \frac{sin(8 \cdot x)}{8 \cdot x}} = \frac{9}{8}$$
\rozwStop
\odpStart
$\frac{9}{8}$
\odpStop
\testStart
A.$\frac{9}{8}$
B.$\infty$
C.$-\infty$
D.$0$
E.$-\frac{9}{8}$
F.$\frac{8}{9}$
G.$-\frac{8}{9}$
H.$8$
I.$9$
\testStop
\kluczStart
A
\kluczStop



\zadStart{Przykład z Wikieł P 4.3a moja wersja nr 121}


Obliczyć granicę funkcji $\lim\limits_{x\to\ 0}\frac{9 \cdot x}{tan(10 \cdot x)}$.
\zadStop
\rozwStart{Patryk Wirkus}{}
$$\lim\limits_{x\to\ 0}\frac{9 \cdot x}{tan(10 \cdot x)}=\lim\limits_{x\to\ 0}\frac{9 \cdot x \cdot cos(10 \cdot x)}{sin(10 \cdot x)}=\lim\limits_{x\to\ 0}\frac{9 \cdot cos(10 \cdot x)}{\frac{sin(10 \cdot x)}{x}}=\lim\limits_{x\to\ 0}\frac{9 \cdot cos(10 \cdot x)}{10 \cdot \frac{sin(10 \cdot x)}{10 \cdot x}} = \frac{9}{10}$$
\rozwStop
\odpStart
$\frac{9}{10}$
\odpStop
\testStart
A.$\frac{9}{10}$
B.$\infty$
C.$-\infty$
D.$0$
E.$-\frac{9}{10}$
F.$\frac{10}{9}$
G.$-\frac{10}{9}$
H.$10$
I.$9$
\testStop
\kluczStart
A
\kluczStop



\zadStart{Przykład z Wikieł P 4.3a moja wersja nr 122}


Obliczyć granicę funkcji $\lim\limits_{x\to\ 0}\frac{9 \cdot x}{tan(11 \cdot x)}$.
\zadStop
\rozwStart{Patryk Wirkus}{}
$$\lim\limits_{x\to\ 0}\frac{9 \cdot x}{tan(11 \cdot x)}=\lim\limits_{x\to\ 0}\frac{9 \cdot x \cdot cos(11 \cdot x)}{sin(11 \cdot x)}=\lim\limits_{x\to\ 0}\frac{9 \cdot cos(11 \cdot x)}{\frac{sin(11 \cdot x)}{x}}=\lim\limits_{x\to\ 0}\frac{9 \cdot cos(11 \cdot x)}{11 \cdot \frac{sin(11 \cdot x)}{11 \cdot x}} = \frac{9}{11}$$
\rozwStop
\odpStart
$\frac{9}{11}$
\odpStop
\testStart
A.$\frac{9}{11}$
B.$\infty$
C.$-\infty$
D.$0$
E.$-\frac{9}{11}$
F.$\frac{11}{9}$
G.$-\frac{11}{9}$
H.$11$
I.$9$
\testStop
\kluczStart
A
\kluczStop



\zadStart{Przykład z Wikieł P 4.3a moja wersja nr 123}


Obliczyć granicę funkcji $\lim\limits_{x\to\ 0}\frac{9 \cdot x}{tan(13 \cdot x)}$.
\zadStop
\rozwStart{Patryk Wirkus}{}
$$\lim\limits_{x\to\ 0}\frac{9 \cdot x}{tan(13 \cdot x)}=\lim\limits_{x\to\ 0}\frac{9 \cdot x \cdot cos(13 \cdot x)}{sin(13 \cdot x)}=\lim\limits_{x\to\ 0}\frac{9 \cdot cos(13 \cdot x)}{\frac{sin(13 \cdot x)}{x}}=\lim\limits_{x\to\ 0}\frac{9 \cdot cos(13 \cdot x)}{13 \cdot \frac{sin(13 \cdot x)}{13 \cdot x}} = \frac{9}{13}$$
\rozwStop
\odpStart
$\frac{9}{13}$
\odpStop
\testStart
A.$\frac{9}{13}$
B.$\infty$
C.$-\infty$
D.$0$
E.$-\frac{9}{13}$
F.$\frac{13}{9}$
G.$-\frac{13}{9}$
H.$13$
I.$9$
\testStop
\kluczStart
A
\kluczStop



\zadStart{Przykład z Wikieł P 4.3a moja wersja nr 124}


Obliczyć granicę funkcji $\lim\limits_{x\to\ 0}\frac{9 \cdot x}{tan(14 \cdot x)}$.
\zadStop
\rozwStart{Patryk Wirkus}{}
$$\lim\limits_{x\to\ 0}\frac{9 \cdot x}{tan(14 \cdot x)}=\lim\limits_{x\to\ 0}\frac{9 \cdot x \cdot cos(14 \cdot x)}{sin(14 \cdot x)}=\lim\limits_{x\to\ 0}\frac{9 \cdot cos(14 \cdot x)}{\frac{sin(14 \cdot x)}{x}}=\lim\limits_{x\to\ 0}\frac{9 \cdot cos(14 \cdot x)}{14 \cdot \frac{sin(14 \cdot x)}{14 \cdot x}} = \frac{9}{14}$$
\rozwStop
\odpStart
$\frac{9}{14}$
\odpStop
\testStart
A.$\frac{9}{14}$
B.$\infty$
C.$-\infty$
D.$0$
E.$-\frac{9}{14}$
F.$\frac{14}{9}$
G.$-\frac{14}{9}$
H.$14$
I.$9$
\testStop
\kluczStart
A
\kluczStop



\zadStart{Przykład z Wikieł P 4.3a moja wersja nr 125}


Obliczyć granicę funkcji $\lim\limits_{x\to\ 0}\frac{9 \cdot x}{tan(16 \cdot x)}$.
\zadStop
\rozwStart{Patryk Wirkus}{}
$$\lim\limits_{x\to\ 0}\frac{9 \cdot x}{tan(16 \cdot x)}=\lim\limits_{x\to\ 0}\frac{9 \cdot x \cdot cos(16 \cdot x)}{sin(16 \cdot x)}=\lim\limits_{x\to\ 0}\frac{9 \cdot cos(16 \cdot x)}{\frac{sin(16 \cdot x)}{x}}=\lim\limits_{x\to\ 0}\frac{9 \cdot cos(16 \cdot x)}{16 \cdot \frac{sin(16 \cdot x)}{16 \cdot x}} = \frac{9}{16}$$
\rozwStop
\odpStart
$\frac{9}{16}$
\odpStop
\testStart
A.$\frac{9}{16}$
B.$\infty$
C.$-\infty$
D.$0$
E.$-\frac{9}{16}$
F.$\frac{16}{9}$
G.$-\frac{16}{9}$
H.$16$
I.$9$
\testStop
\kluczStart
A
\kluczStop



\zadStart{Przykład z Wikieł P 4.3a moja wersja nr 126}


Obliczyć granicę funkcji $\lim\limits_{x\to\ 0}\frac{9 \cdot x}{tan(17 \cdot x)}$.
\zadStop
\rozwStart{Patryk Wirkus}{}
$$\lim\limits_{x\to\ 0}\frac{9 \cdot x}{tan(17 \cdot x)}=\lim\limits_{x\to\ 0}\frac{9 \cdot x \cdot cos(17 \cdot x)}{sin(17 \cdot x)}=\lim\limits_{x\to\ 0}\frac{9 \cdot cos(17 \cdot x)}{\frac{sin(17 \cdot x)}{x}}=\lim\limits_{x\to\ 0}\frac{9 \cdot cos(17 \cdot x)}{17 \cdot \frac{sin(17 \cdot x)}{17 \cdot x}} = \frac{9}{17}$$
\rozwStop
\odpStart
$\frac{9}{17}$
\odpStop
\testStart
A.$\frac{9}{17}$
B.$\infty$
C.$-\infty$
D.$0$
E.$-\frac{9}{17}$
F.$\frac{17}{9}$
G.$-\frac{17}{9}$
H.$17$
I.$9$
\testStop
\kluczStart
A
\kluczStop



\zadStart{Przykład z Wikieł P 4.3a moja wersja nr 127}


Obliczyć granicę funkcji $\lim\limits_{x\to\ 0}\frac{9 \cdot x}{tan(19 \cdot x)}$.
\zadStop
\rozwStart{Patryk Wirkus}{}
$$\lim\limits_{x\to\ 0}\frac{9 \cdot x}{tan(19 \cdot x)}=\lim\limits_{x\to\ 0}\frac{9 \cdot x \cdot cos(19 \cdot x)}{sin(19 \cdot x)}=\lim\limits_{x\to\ 0}\frac{9 \cdot cos(19 \cdot x)}{\frac{sin(19 \cdot x)}{x}}=\lim\limits_{x\to\ 0}\frac{9 \cdot cos(19 \cdot x)}{19 \cdot \frac{sin(19 \cdot x)}{19 \cdot x}} = \frac{9}{19}$$
\rozwStop
\odpStart
$\frac{9}{19}$
\odpStop
\testStart
A.$\frac{9}{19}$
B.$\infty$
C.$-\infty$
D.$0$
E.$-\frac{9}{19}$
F.$\frac{19}{9}$
G.$-\frac{19}{9}$
H.$19$
I.$9$
\testStop
\kluczStart
A
\kluczStop



\zadStart{Przykład z Wikieł P 4.3a moja wersja nr 128}


Obliczyć granicę funkcji $\lim\limits_{x\to\ 0}\frac{9 \cdot x}{tan(20 \cdot x)}$.
\zadStop
\rozwStart{Patryk Wirkus}{}
$$\lim\limits_{x\to\ 0}\frac{9 \cdot x}{tan(20 \cdot x)}=\lim\limits_{x\to\ 0}\frac{9 \cdot x \cdot cos(20 \cdot x)}{sin(20 \cdot x)}=\lim\limits_{x\to\ 0}\frac{9 \cdot cos(20 \cdot x)}{\frac{sin(20 \cdot x)}{x}}=\lim\limits_{x\to\ 0}\frac{9 \cdot cos(20 \cdot x)}{20 \cdot \frac{sin(20 \cdot x)}{20 \cdot x}} = \frac{9}{20}$$
\rozwStop
\odpStart
$\frac{9}{20}$
\odpStop
\testStart
A.$\frac{9}{20}$
B.$\infty$
C.$-\infty$
D.$0$
E.$-\frac{9}{20}$
F.$\frac{20}{9}$
G.$-\frac{20}{9}$
H.$20$
I.$9$
\testStop
\kluczStart
A
\kluczStop



\zadStart{Przykład z Wikieł P 4.3a moja wersja nr 129}


Obliczyć granicę funkcji $\lim\limits_{x\to\ 0}\frac{9 \cdot x}{tan(22 \cdot x)}$.
\zadStop
\rozwStart{Patryk Wirkus}{}
$$\lim\limits_{x\to\ 0}\frac{9 \cdot x}{tan(22 \cdot x)}=\lim\limits_{x\to\ 0}\frac{9 \cdot x \cdot cos(22 \cdot x)}{sin(22 \cdot x)}=\lim\limits_{x\to\ 0}\frac{9 \cdot cos(22 \cdot x)}{\frac{sin(22 \cdot x)}{x}}=\lim\limits_{x\to\ 0}\frac{9 \cdot cos(22 \cdot x)}{22 \cdot \frac{sin(22 \cdot x)}{22 \cdot x}} = \frac{9}{22}$$
\rozwStop
\odpStart
$\frac{9}{22}$
\odpStop
\testStart
A.$\frac{9}{22}$
B.$\infty$
C.$-\infty$
D.$0$
E.$-\frac{9}{22}$
F.$\frac{22}{9}$
G.$-\frac{22}{9}$
H.$22$
I.$9$
\testStop
\kluczStart
A
\kluczStop



\zadStart{Przykład z Wikieł P 4.3a moja wersja nr 130}


Obliczyć granicę funkcji $\lim\limits_{x\to\ 0}\frac{9 \cdot x}{tan(23 \cdot x)}$.
\zadStop
\rozwStart{Patryk Wirkus}{}
$$\lim\limits_{x\to\ 0}\frac{9 \cdot x}{tan(23 \cdot x)}=\lim\limits_{x\to\ 0}\frac{9 \cdot x \cdot cos(23 \cdot x)}{sin(23 \cdot x)}=\lim\limits_{x\to\ 0}\frac{9 \cdot cos(23 \cdot x)}{\frac{sin(23 \cdot x)}{x}}=\lim\limits_{x\to\ 0}\frac{9 \cdot cos(23 \cdot x)}{23 \cdot \frac{sin(23 \cdot x)}{23 \cdot x}} = \frac{9}{23}$$
\rozwStop
\odpStart
$\frac{9}{23}$
\odpStop
\testStart
A.$\frac{9}{23}$
B.$\infty$
C.$-\infty$
D.$0$
E.$-\frac{9}{23}$
F.$\frac{23}{9}$
G.$-\frac{23}{9}$
H.$23$
I.$9$
\testStop
\kluczStart
A
\kluczStop



\zadStart{Przykład z Wikieł P 4.3a moja wersja nr 131}


Obliczyć granicę funkcji $\lim\limits_{x\to\ 0}\frac{9 \cdot x}{tan(25 \cdot x)}$.
\zadStop
\rozwStart{Patryk Wirkus}{}
$$\lim\limits_{x\to\ 0}\frac{9 \cdot x}{tan(25 \cdot x)}=\lim\limits_{x\to\ 0}\frac{9 \cdot x \cdot cos(25 \cdot x)}{sin(25 \cdot x)}=\lim\limits_{x\to\ 0}\frac{9 \cdot cos(25 \cdot x)}{\frac{sin(25 \cdot x)}{x}}=\lim\limits_{x\to\ 0}\frac{9 \cdot cos(25 \cdot x)}{25 \cdot \frac{sin(25 \cdot x)}{25 \cdot x}} = \frac{9}{25}$$
\rozwStop
\odpStart
$\frac{9}{25}$
\odpStop
\testStart
A.$\frac{9}{25}$
B.$\infty$
C.$-\infty$
D.$0$
E.$-\frac{9}{25}$
F.$\frac{25}{9}$
G.$-\frac{25}{9}$
H.$25$
I.$9$
\testStop
\kluczStart
A
\kluczStop



\zadStart{Przykład z Wikieł P 4.3a moja wersja nr 132}


Obliczyć granicę funkcji $\lim\limits_{x\to\ 0}\frac{9 \cdot x}{tan(26 \cdot x)}$.
\zadStop
\rozwStart{Patryk Wirkus}{}
$$\lim\limits_{x\to\ 0}\frac{9 \cdot x}{tan(26 \cdot x)}=\lim\limits_{x\to\ 0}\frac{9 \cdot x \cdot cos(26 \cdot x)}{sin(26 \cdot x)}=\lim\limits_{x\to\ 0}\frac{9 \cdot cos(26 \cdot x)}{\frac{sin(26 \cdot x)}{x}}=\lim\limits_{x\to\ 0}\frac{9 \cdot cos(26 \cdot x)}{26 \cdot \frac{sin(26 \cdot x)}{26 \cdot x}} = \frac{9}{26}$$
\rozwStop
\odpStart
$\frac{9}{26}$
\odpStop
\testStart
A.$\frac{9}{26}$
B.$\infty$
C.$-\infty$
D.$0$
E.$-\frac{9}{26}$
F.$\frac{26}{9}$
G.$-\frac{26}{9}$
H.$26$
I.$9$
\testStop
\kluczStart
A
\kluczStop



\zadStart{Przykład z Wikieł P 4.3a moja wersja nr 133}


Obliczyć granicę funkcji $\lim\limits_{x\to\ 0}\frac{9 \cdot x}{tan(28 \cdot x)}$.
\zadStop
\rozwStart{Patryk Wirkus}{}
$$\lim\limits_{x\to\ 0}\frac{9 \cdot x}{tan(28 \cdot x)}=\lim\limits_{x\to\ 0}\frac{9 \cdot x \cdot cos(28 \cdot x)}{sin(28 \cdot x)}=\lim\limits_{x\to\ 0}\frac{9 \cdot cos(28 \cdot x)}{\frac{sin(28 \cdot x)}{x}}=\lim\limits_{x\to\ 0}\frac{9 \cdot cos(28 \cdot x)}{28 \cdot \frac{sin(28 \cdot x)}{28 \cdot x}} = \frac{9}{28}$$
\rozwStop
\odpStart
$\frac{9}{28}$
\odpStop
\testStart
A.$\frac{9}{28}$
B.$\infty$
C.$-\infty$
D.$0$
E.$-\frac{9}{28}$
F.$\frac{28}{9}$
G.$-\frac{28}{9}$
H.$28$
I.$9$
\testStop
\kluczStart
A
\kluczStop



\zadStart{Przykład z Wikieł P 4.3a moja wersja nr 134}


Obliczyć granicę funkcji $\lim\limits_{x\to\ 0}\frac{9 \cdot x}{tan(29 \cdot x)}$.
\zadStop
\rozwStart{Patryk Wirkus}{}
$$\lim\limits_{x\to\ 0}\frac{9 \cdot x}{tan(29 \cdot x)}=\lim\limits_{x\to\ 0}\frac{9 \cdot x \cdot cos(29 \cdot x)}{sin(29 \cdot x)}=\lim\limits_{x\to\ 0}\frac{9 \cdot cos(29 \cdot x)}{\frac{sin(29 \cdot x)}{x}}=\lim\limits_{x\to\ 0}\frac{9 \cdot cos(29 \cdot x)}{29 \cdot \frac{sin(29 \cdot x)}{29 \cdot x}} = \frac{9}{29}$$
\rozwStop
\odpStart
$\frac{9}{29}$
\odpStop
\testStart
A.$\frac{9}{29}$
B.$\infty$
C.$-\infty$
D.$0$
E.$-\frac{9}{29}$
F.$\frac{29}{9}$
G.$-\frac{29}{9}$
H.$29$
I.$9$
\testStop
\kluczStart
A
\kluczStop



\zadStart{Przykład z Wikieł P 4.3a moja wersja nr 135}


Obliczyć granicę funkcji $\lim\limits_{x\to\ 0}\frac{9 \cdot x}{tan(31 \cdot x)}$.
\zadStop
\rozwStart{Patryk Wirkus}{}
$$\lim\limits_{x\to\ 0}\frac{9 \cdot x}{tan(31 \cdot x)}=\lim\limits_{x\to\ 0}\frac{9 \cdot x \cdot cos(31 \cdot x)}{sin(31 \cdot x)}=\lim\limits_{x\to\ 0}\frac{9 \cdot cos(31 \cdot x)}{\frac{sin(31 \cdot x)}{x}}=\lim\limits_{x\to\ 0}\frac{9 \cdot cos(31 \cdot x)}{31 \cdot \frac{sin(31 \cdot x)}{31 \cdot x}} = \frac{9}{31}$$
\rozwStop
\odpStart
$\frac{9}{31}$
\odpStop
\testStart
A.$\frac{9}{31}$
B.$\infty$
C.$-\infty$
D.$0$
E.$-\frac{9}{31}$
F.$\frac{31}{9}$
G.$-\frac{31}{9}$
H.$31$
I.$9$
\testStop
\kluczStart
A
\kluczStop



\zadStart{Przykład z Wikieł P 4.3a moja wersja nr 136}


Obliczyć granicę funkcji $\lim\limits_{x\to\ 0}\frac{9 \cdot x}{tan(32 \cdot x)}$.
\zadStop
\rozwStart{Patryk Wirkus}{}
$$\lim\limits_{x\to\ 0}\frac{9 \cdot x}{tan(32 \cdot x)}=\lim\limits_{x\to\ 0}\frac{9 \cdot x \cdot cos(32 \cdot x)}{sin(32 \cdot x)}=\lim\limits_{x\to\ 0}\frac{9 \cdot cos(32 \cdot x)}{\frac{sin(32 \cdot x)}{x}}=\lim\limits_{x\to\ 0}\frac{9 \cdot cos(32 \cdot x)}{32 \cdot \frac{sin(32 \cdot x)}{32 \cdot x}} = \frac{9}{32}$$
\rozwStop
\odpStart
$\frac{9}{32}$
\odpStop
\testStart
A.$\frac{9}{32}$
B.$\infty$
C.$-\infty$
D.$0$
E.$-\frac{9}{32}$
F.$\frac{32}{9}$
G.$-\frac{32}{9}$
H.$32$
I.$9$
\testStop
\kluczStart
A
\kluczStop



\zadStart{Przykład z Wikieł P 4.3a moja wersja nr 137}


Obliczyć granicę funkcji $\lim\limits_{x\to\ 0}\frac{9 \cdot x}{tan(34 \cdot x)}$.
\zadStop
\rozwStart{Patryk Wirkus}{}
$$\lim\limits_{x\to\ 0}\frac{9 \cdot x}{tan(34 \cdot x)}=\lim\limits_{x\to\ 0}\frac{9 \cdot x \cdot cos(34 \cdot x)}{sin(34 \cdot x)}=\lim\limits_{x\to\ 0}\frac{9 \cdot cos(34 \cdot x)}{\frac{sin(34 \cdot x)}{x}}=\lim\limits_{x\to\ 0}\frac{9 \cdot cos(34 \cdot x)}{34 \cdot \frac{sin(34 \cdot x)}{34 \cdot x}} = \frac{9}{34}$$
\rozwStop
\odpStart
$\frac{9}{34}$
\odpStop
\testStart
A.$\frac{9}{34}$
B.$\infty$
C.$-\infty$
D.$0$
E.$-\frac{9}{34}$
F.$\frac{34}{9}$
G.$-\frac{34}{9}$
H.$34$
I.$9$
\testStop
\kluczStart
A
\kluczStop



\zadStart{Przykład z Wikieł P 4.3a moja wersja nr 138}


Obliczyć granicę funkcji $\lim\limits_{x\to\ 0}\frac{9 \cdot x}{tan(35 \cdot x)}$.
\zadStop
\rozwStart{Patryk Wirkus}{}
$$\lim\limits_{x\to\ 0}\frac{9 \cdot x}{tan(35 \cdot x)}=\lim\limits_{x\to\ 0}\frac{9 \cdot x \cdot cos(35 \cdot x)}{sin(35 \cdot x)}=\lim\limits_{x\to\ 0}\frac{9 \cdot cos(35 \cdot x)}{\frac{sin(35 \cdot x)}{x}}=\lim\limits_{x\to\ 0}\frac{9 \cdot cos(35 \cdot x)}{35 \cdot \frac{sin(35 \cdot x)}{35 \cdot x}} = \frac{9}{35}$$
\rozwStop
\odpStart
$\frac{9}{35}$
\odpStop
\testStart
A.$\frac{9}{35}$
B.$\infty$
C.$-\infty$
D.$0$
E.$-\frac{9}{35}$
F.$\frac{35}{9}$
G.$-\frac{35}{9}$
H.$35$
I.$9$
\testStop
\kluczStart
A
\kluczStop



\zadStart{Przykład z Wikieł P 4.3a moja wersja nr 139}


Obliczyć granicę funkcji $\lim\limits_{x\to\ 0}\frac{9 \cdot x}{tan(37 \cdot x)}$.
\zadStop
\rozwStart{Patryk Wirkus}{}
$$\lim\limits_{x\to\ 0}\frac{9 \cdot x}{tan(37 \cdot x)}=\lim\limits_{x\to\ 0}\frac{9 \cdot x \cdot cos(37 \cdot x)}{sin(37 \cdot x)}=\lim\limits_{x\to\ 0}\frac{9 \cdot cos(37 \cdot x)}{\frac{sin(37 \cdot x)}{x}}=\lim\limits_{x\to\ 0}\frac{9 \cdot cos(37 \cdot x)}{37 \cdot \frac{sin(37 \cdot x)}{37 \cdot x}} = \frac{9}{37}$$
\rozwStop
\odpStart
$\frac{9}{37}$
\odpStop
\testStart
A.$\frac{9}{37}$
B.$\infty$
C.$-\infty$
D.$0$
E.$-\frac{9}{37}$
F.$\frac{37}{9}$
G.$-\frac{37}{9}$
H.$37$
I.$9$
\testStop
\kluczStart
A
\kluczStop



\zadStart{Przykład z Wikieł P 4.3a moja wersja nr 140}


Obliczyć granicę funkcji $\lim\limits_{x\to\ 0}\frac{9 \cdot x}{tan(38 \cdot x)}$.
\zadStop
\rozwStart{Patryk Wirkus}{}
$$\lim\limits_{x\to\ 0}\frac{9 \cdot x}{tan(38 \cdot x)}=\lim\limits_{x\to\ 0}\frac{9 \cdot x \cdot cos(38 \cdot x)}{sin(38 \cdot x)}=\lim\limits_{x\to\ 0}\frac{9 \cdot cos(38 \cdot x)}{\frac{sin(38 \cdot x)}{x}}=\lim\limits_{x\to\ 0}\frac{9 \cdot cos(38 \cdot x)}{38 \cdot \frac{sin(38 \cdot x)}{38 \cdot x}} = \frac{9}{38}$$
\rozwStop
\odpStart
$\frac{9}{38}$
\odpStop
\testStart
A.$\frac{9}{38}$
B.$\infty$
C.$-\infty$
D.$0$
E.$-\frac{9}{38}$
F.$\frac{38}{9}$
G.$-\frac{38}{9}$
H.$38$
I.$9$
\testStop
\kluczStart
A
\kluczStop



\zadStart{Przykład z Wikieł P 4.3a moja wersja nr 141}


Obliczyć granicę funkcji $\lim\limits_{x\to\ 0}\frac{9 \cdot x}{tan(40 \cdot x)}$.
\zadStop
\rozwStart{Patryk Wirkus}{}
$$\lim\limits_{x\to\ 0}\frac{9 \cdot x}{tan(40 \cdot x)}=\lim\limits_{x\to\ 0}\frac{9 \cdot x \cdot cos(40 \cdot x)}{sin(40 \cdot x)}=\lim\limits_{x\to\ 0}\frac{9 \cdot cos(40 \cdot x)}{\frac{sin(40 \cdot x)}{x}}=\lim\limits_{x\to\ 0}\frac{9 \cdot cos(40 \cdot x)}{40 \cdot \frac{sin(40 \cdot x)}{40 \cdot x}} = \frac{9}{40}$$
\rozwStop
\odpStart
$\frac{9}{40}$
\odpStop
\testStart
A.$\frac{9}{40}$
B.$\infty$
C.$-\infty$
D.$0$
E.$-\frac{9}{40}$
F.$\frac{40}{9}$
G.$-\frac{40}{9}$
H.$40$
I.$9$
\testStop
\kluczStart
A
\kluczStop



\zadStart{Przykład z Wikieł P 4.3a moja wersja nr 142}


Obliczyć granicę funkcji $\lim\limits_{x\to\ 0}\frac{10 \cdot x}{tan(3 \cdot x)}$.
\zadStop
\rozwStart{Patryk Wirkus}{}
$$\lim\limits_{x\to\ 0}\frac{10 \cdot x}{tan(3 \cdot x)}=\lim\limits_{x\to\ 0}\frac{10 \cdot x \cdot cos(3 \cdot x)}{sin(3 \cdot x)}=\lim\limits_{x\to\ 0}\frac{10 \cdot cos(3 \cdot x)}{\frac{sin(3 \cdot x)}{x}}=\lim\limits_{x\to\ 0}\frac{10 \cdot cos(3 \cdot x)}{3 \cdot \frac{sin(3 \cdot x)}{3 \cdot x}} = \frac{10}{3}$$
\rozwStop
\odpStart
$\frac{10}{3}$
\odpStop
\testStart
A.$\frac{10}{3}$
B.$\infty$
C.$-\infty$
D.$0$
E.$-\frac{10}{3}$
F.$\frac{3}{10}$
G.$-\frac{3}{10}$
H.$3$
I.$10$
\testStop
\kluczStart
A
\kluczStop



\zadStart{Przykład z Wikieł P 4.3a moja wersja nr 143}


Obliczyć granicę funkcji $\lim\limits_{x\to\ 0}\frac{10 \cdot x}{tan(7 \cdot x)}$.
\zadStop
\rozwStart{Patryk Wirkus}{}
$$\lim\limits_{x\to\ 0}\frac{10 \cdot x}{tan(7 \cdot x)}=\lim\limits_{x\to\ 0}\frac{10 \cdot x \cdot cos(7 \cdot x)}{sin(7 \cdot x)}=\lim\limits_{x\to\ 0}\frac{10 \cdot cos(7 \cdot x)}{\frac{sin(7 \cdot x)}{x}}=\lim\limits_{x\to\ 0}\frac{10 \cdot cos(7 \cdot x)}{7 \cdot \frac{sin(7 \cdot x)}{7 \cdot x}} = \frac{10}{7}$$
\rozwStop
\odpStart
$\frac{10}{7}$
\odpStop
\testStart
A.$\frac{10}{7}$
B.$\infty$
C.$-\infty$
D.$0$
E.$-\frac{10}{7}$
F.$\frac{7}{10}$
G.$-\frac{7}{10}$
H.$7$
I.$10$
\testStop
\kluczStart
A
\kluczStop



\zadStart{Przykład z Wikieł P 4.3a moja wersja nr 144}


Obliczyć granicę funkcji $\lim\limits_{x\to\ 0}\frac{10 \cdot x}{tan(9 \cdot x)}$.
\zadStop
\rozwStart{Patryk Wirkus}{}
$$\lim\limits_{x\to\ 0}\frac{10 \cdot x}{tan(9 \cdot x)}=\lim\limits_{x\to\ 0}\frac{10 \cdot x \cdot cos(9 \cdot x)}{sin(9 \cdot x)}=\lim\limits_{x\to\ 0}\frac{10 \cdot cos(9 \cdot x)}{\frac{sin(9 \cdot x)}{x}}=\lim\limits_{x\to\ 0}\frac{10 \cdot cos(9 \cdot x)}{9 \cdot \frac{sin(9 \cdot x)}{9 \cdot x}} = \frac{10}{9}$$
\rozwStop
\odpStart
$\frac{10}{9}$
\odpStop
\testStart
A.$\frac{10}{9}$
B.$\infty$
C.$-\infty$
D.$0$
E.$-\frac{10}{9}$
F.$\frac{9}{10}$
G.$-\frac{9}{10}$
H.$9$
I.$10$
\testStop
\kluczStart
A
\kluczStop



\zadStart{Przykład z Wikieł P 4.3a moja wersja nr 145}


Obliczyć granicę funkcji $\lim\limits_{x\to\ 0}\frac{10 \cdot x}{tan(11 \cdot x)}$.
\zadStop
\rozwStart{Patryk Wirkus}{}
$$\lim\limits_{x\to\ 0}\frac{10 \cdot x}{tan(11 \cdot x)}=\lim\limits_{x\to\ 0}\frac{10 \cdot x \cdot cos(11 \cdot x)}{sin(11 \cdot x)}=\lim\limits_{x\to\ 0}\frac{10 \cdot cos(11 \cdot x)}{\frac{sin(11 \cdot x)}{x}}=\lim\limits_{x\to\ 0}\frac{10 \cdot cos(11 \cdot x)}{11 \cdot \frac{sin(11 \cdot x)}{11 \cdot x}} = \frac{10}{11}$$
\rozwStop
\odpStart
$\frac{10}{11}$
\odpStop
\testStart
A.$\frac{10}{11}$
B.$\infty$
C.$-\infty$
D.$0$
E.$-\frac{10}{11}$
F.$\frac{11}{10}$
G.$-\frac{11}{10}$
H.$11$
I.$10$
\testStop
\kluczStart
A
\kluczStop



\zadStart{Przykład z Wikieł P 4.3a moja wersja nr 146}


Obliczyć granicę funkcji $\lim\limits_{x\to\ 0}\frac{10 \cdot x}{tan(13 \cdot x)}$.
\zadStop
\rozwStart{Patryk Wirkus}{}
$$\lim\limits_{x\to\ 0}\frac{10 \cdot x}{tan(13 \cdot x)}=\lim\limits_{x\to\ 0}\frac{10 \cdot x \cdot cos(13 \cdot x)}{sin(13 \cdot x)}=\lim\limits_{x\to\ 0}\frac{10 \cdot cos(13 \cdot x)}{\frac{sin(13 \cdot x)}{x}}=\lim\limits_{x\to\ 0}\frac{10 \cdot cos(13 \cdot x)}{13 \cdot \frac{sin(13 \cdot x)}{13 \cdot x}} = \frac{10}{13}$$
\rozwStop
\odpStart
$\frac{10}{13}$
\odpStop
\testStart
A.$\frac{10}{13}$
B.$\infty$
C.$-\infty$
D.$0$
E.$-\frac{10}{13}$
F.$\frac{13}{10}$
G.$-\frac{13}{10}$
H.$13$
I.$10$
\testStop
\kluczStart
A
\kluczStop



\zadStart{Przykład z Wikieł P 4.3a moja wersja nr 147}


Obliczyć granicę funkcji $\lim\limits_{x\to\ 0}\frac{10 \cdot x}{tan(17 \cdot x)}$.
\zadStop
\rozwStart{Patryk Wirkus}{}
$$\lim\limits_{x\to\ 0}\frac{10 \cdot x}{tan(17 \cdot x)}=\lim\limits_{x\to\ 0}\frac{10 \cdot x \cdot cos(17 \cdot x)}{sin(17 \cdot x)}=\lim\limits_{x\to\ 0}\frac{10 \cdot cos(17 \cdot x)}{\frac{sin(17 \cdot x)}{x}}=\lim\limits_{x\to\ 0}\frac{10 \cdot cos(17 \cdot x)}{17 \cdot \frac{sin(17 \cdot x)}{17 \cdot x}} = \frac{10}{17}$$
\rozwStop
\odpStart
$\frac{10}{17}$
\odpStop
\testStart
A.$\frac{10}{17}$
B.$\infty$
C.$-\infty$
D.$0$
E.$-\frac{10}{17}$
F.$\frac{17}{10}$
G.$-\frac{17}{10}$
H.$17$
I.$10$
\testStop
\kluczStart
A
\kluczStop



\zadStart{Przykład z Wikieł P 4.3a moja wersja nr 148}


Obliczyć granicę funkcji $\lim\limits_{x\to\ 0}\frac{10 \cdot x}{tan(19 \cdot x)}$.
\zadStop
\rozwStart{Patryk Wirkus}{}
$$\lim\limits_{x\to\ 0}\frac{10 \cdot x}{tan(19 \cdot x)}=\lim\limits_{x\to\ 0}\frac{10 \cdot x \cdot cos(19 \cdot x)}{sin(19 \cdot x)}=\lim\limits_{x\to\ 0}\frac{10 \cdot cos(19 \cdot x)}{\frac{sin(19 \cdot x)}{x}}=\lim\limits_{x\to\ 0}\frac{10 \cdot cos(19 \cdot x)}{19 \cdot \frac{sin(19 \cdot x)}{19 \cdot x}} = \frac{10}{19}$$
\rozwStop
\odpStart
$\frac{10}{19}$
\odpStop
\testStart
A.$\frac{10}{19}$
B.$\infty$
C.$-\infty$
D.$0$
E.$-\frac{10}{19}$
F.$\frac{19}{10}$
G.$-\frac{19}{10}$
H.$19$
I.$10$
\testStop
\kluczStart
A
\kluczStop



\zadStart{Przykład z Wikieł P 4.3a moja wersja nr 149}


Obliczyć granicę funkcji $\lim\limits_{x\to\ 0}\frac{10 \cdot x}{tan(21 \cdot x)}$.
\zadStop
\rozwStart{Patryk Wirkus}{}
$$\lim\limits_{x\to\ 0}\frac{10 \cdot x}{tan(21 \cdot x)}=\lim\limits_{x\to\ 0}\frac{10 \cdot x \cdot cos(21 \cdot x)}{sin(21 \cdot x)}=\lim\limits_{x\to\ 0}\frac{10 \cdot cos(21 \cdot x)}{\frac{sin(21 \cdot x)}{x}}=\lim\limits_{x\to\ 0}\frac{10 \cdot cos(21 \cdot x)}{21 \cdot \frac{sin(21 \cdot x)}{21 \cdot x}} = \frac{10}{21}$$
\rozwStop
\odpStart
$\frac{10}{21}$
\odpStop
\testStart
A.$\frac{10}{21}$
B.$\infty$
C.$-\infty$
D.$0$
E.$-\frac{10}{21}$
F.$\frac{21}{10}$
G.$-\frac{21}{10}$
H.$21$
I.$10$
\testStop
\kluczStart
A
\kluczStop



\zadStart{Przykład z Wikieł P 4.3a moja wersja nr 150}


Obliczyć granicę funkcji $\lim\limits_{x\to\ 0}\frac{10 \cdot x}{tan(23 \cdot x)}$.
\zadStop
\rozwStart{Patryk Wirkus}{}
$$\lim\limits_{x\to\ 0}\frac{10 \cdot x}{tan(23 \cdot x)}=\lim\limits_{x\to\ 0}\frac{10 \cdot x \cdot cos(23 \cdot x)}{sin(23 \cdot x)}=\lim\limits_{x\to\ 0}\frac{10 \cdot cos(23 \cdot x)}{\frac{sin(23 \cdot x)}{x}}=\lim\limits_{x\to\ 0}\frac{10 \cdot cos(23 \cdot x)}{23 \cdot \frac{sin(23 \cdot x)}{23 \cdot x}} = \frac{10}{23}$$
\rozwStop
\odpStart
$\frac{10}{23}$
\odpStop
\testStart
A.$\frac{10}{23}$
B.$\infty$
C.$-\infty$
D.$0$
E.$-\frac{10}{23}$
F.$\frac{23}{10}$
G.$-\frac{23}{10}$
H.$23$
I.$10$
\testStop
\kluczStart
A
\kluczStop



\zadStart{Przykład z Wikieł P 4.3a moja wersja nr 151}


Obliczyć granicę funkcji $\lim\limits_{x\to\ 0}\frac{10 \cdot x}{tan(27 \cdot x)}$.
\zadStop
\rozwStart{Patryk Wirkus}{}
$$\lim\limits_{x\to\ 0}\frac{10 \cdot x}{tan(27 \cdot x)}=\lim\limits_{x\to\ 0}\frac{10 \cdot x \cdot cos(27 \cdot x)}{sin(27 \cdot x)}=\lim\limits_{x\to\ 0}\frac{10 \cdot cos(27 \cdot x)}{\frac{sin(27 \cdot x)}{x}}=\lim\limits_{x\to\ 0}\frac{10 \cdot cos(27 \cdot x)}{27 \cdot \frac{sin(27 \cdot x)}{27 \cdot x}} = \frac{10}{27}$$
\rozwStop
\odpStart
$\frac{10}{27}$
\odpStop
\testStart
A.$\frac{10}{27}$
B.$\infty$
C.$-\infty$
D.$0$
E.$-\frac{10}{27}$
F.$\frac{27}{10}$
G.$-\frac{27}{10}$
H.$27$
I.$10$
\testStop
\kluczStart
A
\kluczStop



\zadStart{Przykład z Wikieł P 4.3a moja wersja nr 152}


Obliczyć granicę funkcji $\lim\limits_{x\to\ 0}\frac{10 \cdot x}{tan(29 \cdot x)}$.
\zadStop
\rozwStart{Patryk Wirkus}{}
$$\lim\limits_{x\to\ 0}\frac{10 \cdot x}{tan(29 \cdot x)}=\lim\limits_{x\to\ 0}\frac{10 \cdot x \cdot cos(29 \cdot x)}{sin(29 \cdot x)}=\lim\limits_{x\to\ 0}\frac{10 \cdot cos(29 \cdot x)}{\frac{sin(29 \cdot x)}{x}}=\lim\limits_{x\to\ 0}\frac{10 \cdot cos(29 \cdot x)}{29 \cdot \frac{sin(29 \cdot x)}{29 \cdot x}} = \frac{10}{29}$$
\rozwStop
\odpStart
$\frac{10}{29}$
\odpStop
\testStart
A.$\frac{10}{29}$
B.$\infty$
C.$-\infty$
D.$0$
E.$-\frac{10}{29}$
F.$\frac{29}{10}$
G.$-\frac{29}{10}$
H.$29$
I.$10$
\testStop
\kluczStart
A
\kluczStop



\zadStart{Przykład z Wikieł P 4.3a moja wersja nr 153}


Obliczyć granicę funkcji $\lim\limits_{x\to\ 0}\frac{10 \cdot x}{tan(31 \cdot x)}$.
\zadStop
\rozwStart{Patryk Wirkus}{}
$$\lim\limits_{x\to\ 0}\frac{10 \cdot x}{tan(31 \cdot x)}=\lim\limits_{x\to\ 0}\frac{10 \cdot x \cdot cos(31 \cdot x)}{sin(31 \cdot x)}=\lim\limits_{x\to\ 0}\frac{10 \cdot cos(31 \cdot x)}{\frac{sin(31 \cdot x)}{x}}=\lim\limits_{x\to\ 0}\frac{10 \cdot cos(31 \cdot x)}{31 \cdot \frac{sin(31 \cdot x)}{31 \cdot x}} = \frac{10}{31}$$
\rozwStop
\odpStart
$\frac{10}{31}$
\odpStop
\testStart
A.$\frac{10}{31}$
B.$\infty$
C.$-\infty$
D.$0$
E.$-\frac{10}{31}$
F.$\frac{31}{10}$
G.$-\frac{31}{10}$
H.$31$
I.$10$
\testStop
\kluczStart
A
\kluczStop



\zadStart{Przykład z Wikieł P 4.3a moja wersja nr 154}


Obliczyć granicę funkcji $\lim\limits_{x\to\ 0}\frac{10 \cdot x}{tan(33 \cdot x)}$.
\zadStop
\rozwStart{Patryk Wirkus}{}
$$\lim\limits_{x\to\ 0}\frac{10 \cdot x}{tan(33 \cdot x)}=\lim\limits_{x\to\ 0}\frac{10 \cdot x \cdot cos(33 \cdot x)}{sin(33 \cdot x)}=\lim\limits_{x\to\ 0}\frac{10 \cdot cos(33 \cdot x)}{\frac{sin(33 \cdot x)}{x}}=\lim\limits_{x\to\ 0}\frac{10 \cdot cos(33 \cdot x)}{33 \cdot \frac{sin(33 \cdot x)}{33 \cdot x}} = \frac{10}{33}$$
\rozwStop
\odpStart
$\frac{10}{33}$
\odpStop
\testStart
A.$\frac{10}{33}$
B.$\infty$
C.$-\infty$
D.$0$
E.$-\frac{10}{33}$
F.$\frac{33}{10}$
G.$-\frac{33}{10}$
H.$33$
I.$10$
\testStop
\kluczStart
A
\kluczStop



\zadStart{Przykład z Wikieł P 4.3a moja wersja nr 155}


Obliczyć granicę funkcji $\lim\limits_{x\to\ 0}\frac{10 \cdot x}{tan(37 \cdot x)}$.
\zadStop
\rozwStart{Patryk Wirkus}{}
$$\lim\limits_{x\to\ 0}\frac{10 \cdot x}{tan(37 \cdot x)}=\lim\limits_{x\to\ 0}\frac{10 \cdot x \cdot cos(37 \cdot x)}{sin(37 \cdot x)}=\lim\limits_{x\to\ 0}\frac{10 \cdot cos(37 \cdot x)}{\frac{sin(37 \cdot x)}{x}}=\lim\limits_{x\to\ 0}\frac{10 \cdot cos(37 \cdot x)}{37 \cdot \frac{sin(37 \cdot x)}{37 \cdot x}} = \frac{10}{37}$$
\rozwStop
\odpStart
$\frac{10}{37}$
\odpStop
\testStart
A.$\frac{10}{37}$
B.$\infty$
C.$-\infty$
D.$0$
E.$-\frac{10}{37}$
F.$\frac{37}{10}$
G.$-\frac{37}{10}$
H.$37$
I.$10$
\testStop
\kluczStart
A
\kluczStop



\zadStart{Przykład z Wikieł P 4.3a moja wersja nr 156}


Obliczyć granicę funkcji $\lim\limits_{x\to\ 0}\frac{10 \cdot x}{tan(39 \cdot x)}$.
\zadStop
\rozwStart{Patryk Wirkus}{}
$$\lim\limits_{x\to\ 0}\frac{10 \cdot x}{tan(39 \cdot x)}=\lim\limits_{x\to\ 0}\frac{10 \cdot x \cdot cos(39 \cdot x)}{sin(39 \cdot x)}=\lim\limits_{x\to\ 0}\frac{10 \cdot cos(39 \cdot x)}{\frac{sin(39 \cdot x)}{x}}=\lim\limits_{x\to\ 0}\frac{10 \cdot cos(39 \cdot x)}{39 \cdot \frac{sin(39 \cdot x)}{39 \cdot x}} = \frac{10}{39}$$
\rozwStop
\odpStart
$\frac{10}{39}$
\odpStop
\testStart
A.$\frac{10}{39}$
B.$\infty$
C.$-\infty$
D.$0$
E.$-\frac{10}{39}$
F.$\frac{39}{10}$
G.$-\frac{39}{10}$
H.$39$
I.$10$
\testStop
\kluczStart
A
\kluczStop



\zadStart{Przykład z Wikieł P 4.3a moja wersja nr 157}


Obliczyć granicę funkcji $\lim\limits_{x\to\ 0}\frac{11 \cdot x}{tan(2 \cdot x)}$.
\zadStop
\rozwStart{Patryk Wirkus}{}
$$\lim\limits_{x\to\ 0}\frac{11 \cdot x}{tan(2 \cdot x)}=\lim\limits_{x\to\ 0}\frac{11 \cdot x \cdot cos(2 \cdot x)}{sin(2 \cdot x)}=\lim\limits_{x\to\ 0}\frac{11 \cdot cos(2 \cdot x)}{\frac{sin(2 \cdot x)}{x}}=\lim\limits_{x\to\ 0}\frac{11 \cdot cos(2 \cdot x)}{2 \cdot \frac{sin(2 \cdot x)}{2 \cdot x}} = \frac{11}{2}$$
\rozwStop
\odpStart
$\frac{11}{2}$
\odpStop
\testStart
A.$\frac{11}{2}$
B.$\infty$
C.$-\infty$
D.$0$
E.$-\frac{11}{2}$
F.$\frac{2}{11}$
G.$-\frac{2}{11}$
H.$2$
I.$11$
\testStop
\kluczStart
A
\kluczStop



\zadStart{Przykład z Wikieł P 4.3a moja wersja nr 158}


Obliczyć granicę funkcji $\lim\limits_{x\to\ 0}\frac{11 \cdot x}{tan(3 \cdot x)}$.
\zadStop
\rozwStart{Patryk Wirkus}{}
$$\lim\limits_{x\to\ 0}\frac{11 \cdot x}{tan(3 \cdot x)}=\lim\limits_{x\to\ 0}\frac{11 \cdot x \cdot cos(3 \cdot x)}{sin(3 \cdot x)}=\lim\limits_{x\to\ 0}\frac{11 \cdot cos(3 \cdot x)}{\frac{sin(3 \cdot x)}{x}}=\lim\limits_{x\to\ 0}\frac{11 \cdot cos(3 \cdot x)}{3 \cdot \frac{sin(3 \cdot x)}{3 \cdot x}} = \frac{11}{3}$$
\rozwStop
\odpStart
$\frac{11}{3}$
\odpStop
\testStart
A.$\frac{11}{3}$
B.$\infty$
C.$-\infty$
D.$0$
E.$-\frac{11}{3}$
F.$\frac{3}{11}$
G.$-\frac{3}{11}$
H.$3$
I.$11$
\testStop
\kluczStart
A
\kluczStop



\zadStart{Przykład z Wikieł P 4.3a moja wersja nr 159}


Obliczyć granicę funkcji $\lim\limits_{x\to\ 0}\frac{11 \cdot x}{tan(4 \cdot x)}$.
\zadStop
\rozwStart{Patryk Wirkus}{}
$$\lim\limits_{x\to\ 0}\frac{11 \cdot x}{tan(4 \cdot x)}=\lim\limits_{x\to\ 0}\frac{11 \cdot x \cdot cos(4 \cdot x)}{sin(4 \cdot x)}=\lim\limits_{x\to\ 0}\frac{11 \cdot cos(4 \cdot x)}{\frac{sin(4 \cdot x)}{x}}=\lim\limits_{x\to\ 0}\frac{11 \cdot cos(4 \cdot x)}{4 \cdot \frac{sin(4 \cdot x)}{4 \cdot x}} = \frac{11}{4}$$
\rozwStop
\odpStart
$\frac{11}{4}$
\odpStop
\testStart
A.$\frac{11}{4}$
B.$\infty$
C.$-\infty$
D.$0$
E.$-\frac{11}{4}$
F.$\frac{4}{11}$
G.$-\frac{4}{11}$
H.$4$
I.$11$
\testStop
\kluczStart
A
\kluczStop



\zadStart{Przykład z Wikieł P 4.3a moja wersja nr 160}


Obliczyć granicę funkcji $\lim\limits_{x\to\ 0}\frac{11 \cdot x}{tan(5 \cdot x)}$.
\zadStop
\rozwStart{Patryk Wirkus}{}
$$\lim\limits_{x\to\ 0}\frac{11 \cdot x}{tan(5 \cdot x)}=\lim\limits_{x\to\ 0}\frac{11 \cdot x \cdot cos(5 \cdot x)}{sin(5 \cdot x)}=\lim\limits_{x\to\ 0}\frac{11 \cdot cos(5 \cdot x)}{\frac{sin(5 \cdot x)}{x}}=\lim\limits_{x\to\ 0}\frac{11 \cdot cos(5 \cdot x)}{5 \cdot \frac{sin(5 \cdot x)}{5 \cdot x}} = \frac{11}{5}$$
\rozwStop
\odpStart
$\frac{11}{5}$
\odpStop
\testStart
A.$\frac{11}{5}$
B.$\infty$
C.$-\infty$
D.$0$
E.$-\frac{11}{5}$
F.$\frac{5}{11}$
G.$-\frac{5}{11}$
H.$5$
I.$11$
\testStop
\kluczStart
A
\kluczStop



\zadStart{Przykład z Wikieł P 4.3a moja wersja nr 161}


Obliczyć granicę funkcji $\lim\limits_{x\to\ 0}\frac{11 \cdot x}{tan(6 \cdot x)}$.
\zadStop
\rozwStart{Patryk Wirkus}{}
$$\lim\limits_{x\to\ 0}\frac{11 \cdot x}{tan(6 \cdot x)}=\lim\limits_{x\to\ 0}\frac{11 \cdot x \cdot cos(6 \cdot x)}{sin(6 \cdot x)}=\lim\limits_{x\to\ 0}\frac{11 \cdot cos(6 \cdot x)}{\frac{sin(6 \cdot x)}{x}}=\lim\limits_{x\to\ 0}\frac{11 \cdot cos(6 \cdot x)}{6 \cdot \frac{sin(6 \cdot x)}{6 \cdot x}} = \frac{11}{6}$$
\rozwStop
\odpStart
$\frac{11}{6}$
\odpStop
\testStart
A.$\frac{11}{6}$
B.$\infty$
C.$-\infty$
D.$0$
E.$-\frac{11}{6}$
F.$\frac{6}{11}$
G.$-\frac{6}{11}$
H.$6$
I.$11$
\testStop
\kluczStart
A
\kluczStop



\zadStart{Przykład z Wikieł P 4.3a moja wersja nr 162}


Obliczyć granicę funkcji $\lim\limits_{x\to\ 0}\frac{11 \cdot x}{tan(7 \cdot x)}$.
\zadStop
\rozwStart{Patryk Wirkus}{}
$$\lim\limits_{x\to\ 0}\frac{11 \cdot x}{tan(7 \cdot x)}=\lim\limits_{x\to\ 0}\frac{11 \cdot x \cdot cos(7 \cdot x)}{sin(7 \cdot x)}=\lim\limits_{x\to\ 0}\frac{11 \cdot cos(7 \cdot x)}{\frac{sin(7 \cdot x)}{x}}=\lim\limits_{x\to\ 0}\frac{11 \cdot cos(7 \cdot x)}{7 \cdot \frac{sin(7 \cdot x)}{7 \cdot x}} = \frac{11}{7}$$
\rozwStop
\odpStart
$\frac{11}{7}$
\odpStop
\testStart
A.$\frac{11}{7}$
B.$\infty$
C.$-\infty$
D.$0$
E.$-\frac{11}{7}$
F.$\frac{7}{11}$
G.$-\frac{7}{11}$
H.$7$
I.$11$
\testStop
\kluczStart
A
\kluczStop



\zadStart{Przykład z Wikieł P 4.3a moja wersja nr 163}


Obliczyć granicę funkcji $\lim\limits_{x\to\ 0}\frac{11 \cdot x}{tan(8 \cdot x)}$.
\zadStop
\rozwStart{Patryk Wirkus}{}
$$\lim\limits_{x\to\ 0}\frac{11 \cdot x}{tan(8 \cdot x)}=\lim\limits_{x\to\ 0}\frac{11 \cdot x \cdot cos(8 \cdot x)}{sin(8 \cdot x)}=\lim\limits_{x\to\ 0}\frac{11 \cdot cos(8 \cdot x)}{\frac{sin(8 \cdot x)}{x}}=\lim\limits_{x\to\ 0}\frac{11 \cdot cos(8 \cdot x)}{8 \cdot \frac{sin(8 \cdot x)}{8 \cdot x}} = \frac{11}{8}$$
\rozwStop
\odpStart
$\frac{11}{8}$
\odpStop
\testStart
A.$\frac{11}{8}$
B.$\infty$
C.$-\infty$
D.$0$
E.$-\frac{11}{8}$
F.$\frac{8}{11}$
G.$-\frac{8}{11}$
H.$8$
I.$11$
\testStop
\kluczStart
A
\kluczStop



\zadStart{Przykład z Wikieł P 4.3a moja wersja nr 164}


Obliczyć granicę funkcji $\lim\limits_{x\to\ 0}\frac{11 \cdot x}{tan(9 \cdot x)}$.
\zadStop
\rozwStart{Patryk Wirkus}{}
$$\lim\limits_{x\to\ 0}\frac{11 \cdot x}{tan(9 \cdot x)}=\lim\limits_{x\to\ 0}\frac{11 \cdot x \cdot cos(9 \cdot x)}{sin(9 \cdot x)}=\lim\limits_{x\to\ 0}\frac{11 \cdot cos(9 \cdot x)}{\frac{sin(9 \cdot x)}{x}}=\lim\limits_{x\to\ 0}\frac{11 \cdot cos(9 \cdot x)}{9 \cdot \frac{sin(9 \cdot x)}{9 \cdot x}} = \frac{11}{9}$$
\rozwStop
\odpStart
$\frac{11}{9}$
\odpStop
\testStart
A.$\frac{11}{9}$
B.$\infty$
C.$-\infty$
D.$0$
E.$-\frac{11}{9}$
F.$\frac{9}{11}$
G.$-\frac{9}{11}$
H.$9$
I.$11$
\testStop
\kluczStart
A
\kluczStop



\zadStart{Przykład z Wikieł P 4.3a moja wersja nr 165}


Obliczyć granicę funkcji $\lim\limits_{x\to\ 0}\frac{11 \cdot x}{tan(10 \cdot x)}$.
\zadStop
\rozwStart{Patryk Wirkus}{}
$$\lim\limits_{x\to\ 0}\frac{11 \cdot x}{tan(10 \cdot x)}=\lim\limits_{x\to\ 0}\frac{11 \cdot x \cdot cos(10 \cdot x)}{sin(10 \cdot x)}=\lim\limits_{x\to\ 0}\frac{11 \cdot cos(10 \cdot x)}{\frac{sin(10 \cdot x)}{x}}=\lim\limits_{x\to\ 0}\frac{11 \cdot cos(10 \cdot x)}{10 \cdot \frac{sin(10 \cdot x)}{10 \cdot x}} = \frac{11}{10}$$
\rozwStop
\odpStart
$\frac{11}{10}$
\odpStop
\testStart
A.$\frac{11}{10}$
B.$\infty$
C.$-\infty$
D.$0$
E.$-\frac{11}{10}$
F.$\frac{10}{11}$
G.$-\frac{10}{11}$
H.$10$
I.$11$
\testStop
\kluczStart
A
\kluczStop



\zadStart{Przykład z Wikieł P 4.3a moja wersja nr 166}


Obliczyć granicę funkcji $\lim\limits_{x\to\ 0}\frac{11 \cdot x}{tan(12 \cdot x)}$.
\zadStop
\rozwStart{Patryk Wirkus}{}
$$\lim\limits_{x\to\ 0}\frac{11 \cdot x}{tan(12 \cdot x)}=\lim\limits_{x\to\ 0}\frac{11 \cdot x \cdot cos(12 \cdot x)}{sin(12 \cdot x)}=\lim\limits_{x\to\ 0}\frac{11 \cdot cos(12 \cdot x)}{\frac{sin(12 \cdot x)}{x}}=\lim\limits_{x\to\ 0}\frac{11 \cdot cos(12 \cdot x)}{12 \cdot \frac{sin(12 \cdot x)}{12 \cdot x}} = \frac{11}{12}$$
\rozwStop
\odpStart
$\frac{11}{12}$
\odpStop
\testStart
A.$\frac{11}{12}$
B.$\infty$
C.$-\infty$
D.$0$
E.$-\frac{11}{12}$
F.$\frac{12}{11}$
G.$-\frac{12}{11}$
H.$12$
I.$11$
\testStop
\kluczStart
A
\kluczStop



\zadStart{Przykład z Wikieł P 4.3a moja wersja nr 167}


Obliczyć granicę funkcji $\lim\limits_{x\to\ 0}\frac{11 \cdot x}{tan(13 \cdot x)}$.
\zadStop
\rozwStart{Patryk Wirkus}{}
$$\lim\limits_{x\to\ 0}\frac{11 \cdot x}{tan(13 \cdot x)}=\lim\limits_{x\to\ 0}\frac{11 \cdot x \cdot cos(13 \cdot x)}{sin(13 \cdot x)}=\lim\limits_{x\to\ 0}\frac{11 \cdot cos(13 \cdot x)}{\frac{sin(13 \cdot x)}{x}}=\lim\limits_{x\to\ 0}\frac{11 \cdot cos(13 \cdot x)}{13 \cdot \frac{sin(13 \cdot x)}{13 \cdot x}} = \frac{11}{13}$$
\rozwStop
\odpStart
$\frac{11}{13}$
\odpStop
\testStart
A.$\frac{11}{13}$
B.$\infty$
C.$-\infty$
D.$0$
E.$-\frac{11}{13}$
F.$\frac{13}{11}$
G.$-\frac{13}{11}$
H.$13$
I.$11$
\testStop
\kluczStart
A
\kluczStop



\zadStart{Przykład z Wikieł P 4.3a moja wersja nr 168}


Obliczyć granicę funkcji $\lim\limits_{x\to\ 0}\frac{11 \cdot x}{tan(14 \cdot x)}$.
\zadStop
\rozwStart{Patryk Wirkus}{}
$$\lim\limits_{x\to\ 0}\frac{11 \cdot x}{tan(14 \cdot x)}=\lim\limits_{x\to\ 0}\frac{11 \cdot x \cdot cos(14 \cdot x)}{sin(14 \cdot x)}=\lim\limits_{x\to\ 0}\frac{11 \cdot cos(14 \cdot x)}{\frac{sin(14 \cdot x)}{x}}=\lim\limits_{x\to\ 0}\frac{11 \cdot cos(14 \cdot x)}{14 \cdot \frac{sin(14 \cdot x)}{14 \cdot x}} = \frac{11}{14}$$
\rozwStop
\odpStart
$\frac{11}{14}$
\odpStop
\testStart
A.$\frac{11}{14}$
B.$\infty$
C.$-\infty$
D.$0$
E.$-\frac{11}{14}$
F.$\frac{14}{11}$
G.$-\frac{14}{11}$
H.$14$
I.$11$
\testStop
\kluczStart
A
\kluczStop



\zadStart{Przykład z Wikieł P 4.3a moja wersja nr 169}


Obliczyć granicę funkcji $\lim\limits_{x\to\ 0}\frac{11 \cdot x}{tan(15 \cdot x)}$.
\zadStop
\rozwStart{Patryk Wirkus}{}
$$\lim\limits_{x\to\ 0}\frac{11 \cdot x}{tan(15 \cdot x)}=\lim\limits_{x\to\ 0}\frac{11 \cdot x \cdot cos(15 \cdot x)}{sin(15 \cdot x)}=\lim\limits_{x\to\ 0}\frac{11 \cdot cos(15 \cdot x)}{\frac{sin(15 \cdot x)}{x}}=\lim\limits_{x\to\ 0}\frac{11 \cdot cos(15 \cdot x)}{15 \cdot \frac{sin(15 \cdot x)}{15 \cdot x}} = \frac{11}{15}$$
\rozwStop
\odpStart
$\frac{11}{15}$
\odpStop
\testStart
A.$\frac{11}{15}$
B.$\infty$
C.$-\infty$
D.$0$
E.$-\frac{11}{15}$
F.$\frac{15}{11}$
G.$-\frac{15}{11}$
H.$15$
I.$11$
\testStop
\kluczStart
A
\kluczStop



\zadStart{Przykład z Wikieł P 4.3a moja wersja nr 170}


Obliczyć granicę funkcji $\lim\limits_{x\to\ 0}\frac{11 \cdot x}{tan(16 \cdot x)}$.
\zadStop
\rozwStart{Patryk Wirkus}{}
$$\lim\limits_{x\to\ 0}\frac{11 \cdot x}{tan(16 \cdot x)}=\lim\limits_{x\to\ 0}\frac{11 \cdot x \cdot cos(16 \cdot x)}{sin(16 \cdot x)}=\lim\limits_{x\to\ 0}\frac{11 \cdot cos(16 \cdot x)}{\frac{sin(16 \cdot x)}{x}}=\lim\limits_{x\to\ 0}\frac{11 \cdot cos(16 \cdot x)}{16 \cdot \frac{sin(16 \cdot x)}{16 \cdot x}} = \frac{11}{16}$$
\rozwStop
\odpStart
$\frac{11}{16}$
\odpStop
\testStart
A.$\frac{11}{16}$
B.$\infty$
C.$-\infty$
D.$0$
E.$-\frac{11}{16}$
F.$\frac{16}{11}$
G.$-\frac{16}{11}$
H.$16$
I.$11$
\testStop
\kluczStart
A
\kluczStop



\zadStart{Przykład z Wikieł P 4.3a moja wersja nr 171}


Obliczyć granicę funkcji $\lim\limits_{x\to\ 0}\frac{11 \cdot x}{tan(17 \cdot x)}$.
\zadStop
\rozwStart{Patryk Wirkus}{}
$$\lim\limits_{x\to\ 0}\frac{11 \cdot x}{tan(17 \cdot x)}=\lim\limits_{x\to\ 0}\frac{11 \cdot x \cdot cos(17 \cdot x)}{sin(17 \cdot x)}=\lim\limits_{x\to\ 0}\frac{11 \cdot cos(17 \cdot x)}{\frac{sin(17 \cdot x)}{x}}=\lim\limits_{x\to\ 0}\frac{11 \cdot cos(17 \cdot x)}{17 \cdot \frac{sin(17 \cdot x)}{17 \cdot x}} = \frac{11}{17}$$
\rozwStop
\odpStart
$\frac{11}{17}$
\odpStop
\testStart
A.$\frac{11}{17}$
B.$\infty$
C.$-\infty$
D.$0$
E.$-\frac{11}{17}$
F.$\frac{17}{11}$
G.$-\frac{17}{11}$
H.$17$
I.$11$
\testStop
\kluczStart
A
\kluczStop



\zadStart{Przykład z Wikieł P 4.3a moja wersja nr 172}


Obliczyć granicę funkcji $\lim\limits_{x\to\ 0}\frac{11 \cdot x}{tan(18 \cdot x)}$.
\zadStop
\rozwStart{Patryk Wirkus}{}
$$\lim\limits_{x\to\ 0}\frac{11 \cdot x}{tan(18 \cdot x)}=\lim\limits_{x\to\ 0}\frac{11 \cdot x \cdot cos(18 \cdot x)}{sin(18 \cdot x)}=\lim\limits_{x\to\ 0}\frac{11 \cdot cos(18 \cdot x)}{\frac{sin(18 \cdot x)}{x}}=\lim\limits_{x\to\ 0}\frac{11 \cdot cos(18 \cdot x)}{18 \cdot \frac{sin(18 \cdot x)}{18 \cdot x}} = \frac{11}{18}$$
\rozwStop
\odpStart
$\frac{11}{18}$
\odpStop
\testStart
A.$\frac{11}{18}$
B.$\infty$
C.$-\infty$
D.$0$
E.$-\frac{11}{18}$
F.$\frac{18}{11}$
G.$-\frac{18}{11}$
H.$18$
I.$11$
\testStop
\kluczStart
A
\kluczStop



\zadStart{Przykład z Wikieł P 4.3a moja wersja nr 173}


Obliczyć granicę funkcji $\lim\limits_{x\to\ 0}\frac{11 \cdot x}{tan(19 \cdot x)}$.
\zadStop
\rozwStart{Patryk Wirkus}{}
$$\lim\limits_{x\to\ 0}\frac{11 \cdot x}{tan(19 \cdot x)}=\lim\limits_{x\to\ 0}\frac{11 \cdot x \cdot cos(19 \cdot x)}{sin(19 \cdot x)}=\lim\limits_{x\to\ 0}\frac{11 \cdot cos(19 \cdot x)}{\frac{sin(19 \cdot x)}{x}}=\lim\limits_{x\to\ 0}\frac{11 \cdot cos(19 \cdot x)}{19 \cdot \frac{sin(19 \cdot x)}{19 \cdot x}} = \frac{11}{19}$$
\rozwStop
\odpStart
$\frac{11}{19}$
\odpStop
\testStart
A.$\frac{11}{19}$
B.$\infty$
C.$-\infty$
D.$0$
E.$-\frac{11}{19}$
F.$\frac{19}{11}$
G.$-\frac{19}{11}$
H.$19$
I.$11$
\testStop
\kluczStart
A
\kluczStop



\zadStart{Przykład z Wikieł P 4.3a moja wersja nr 174}


Obliczyć granicę funkcji $\lim\limits_{x\to\ 0}\frac{11 \cdot x}{tan(20 \cdot x)}$.
\zadStop
\rozwStart{Patryk Wirkus}{}
$$\lim\limits_{x\to\ 0}\frac{11 \cdot x}{tan(20 \cdot x)}=\lim\limits_{x\to\ 0}\frac{11 \cdot x \cdot cos(20 \cdot x)}{sin(20 \cdot x)}=\lim\limits_{x\to\ 0}\frac{11 \cdot cos(20 \cdot x)}{\frac{sin(20 \cdot x)}{x}}=\lim\limits_{x\to\ 0}\frac{11 \cdot cos(20 \cdot x)}{20 \cdot \frac{sin(20 \cdot x)}{20 \cdot x}} = \frac{11}{20}$$
\rozwStop
\odpStart
$\frac{11}{20}$
\odpStop
\testStart
A.$\frac{11}{20}$
B.$\infty$
C.$-\infty$
D.$0$
E.$-\frac{11}{20}$
F.$\frac{20}{11}$
G.$-\frac{20}{11}$
H.$20$
I.$11$
\testStop
\kluczStart
A
\kluczStop



\zadStart{Przykład z Wikieł P 4.3a moja wersja nr 175}


Obliczyć granicę funkcji $\lim\limits_{x\to\ 0}\frac{11 \cdot x}{tan(21 \cdot x)}$.
\zadStop
\rozwStart{Patryk Wirkus}{}
$$\lim\limits_{x\to\ 0}\frac{11 \cdot x}{tan(21 \cdot x)}=\lim\limits_{x\to\ 0}\frac{11 \cdot x \cdot cos(21 \cdot x)}{sin(21 \cdot x)}=\lim\limits_{x\to\ 0}\frac{11 \cdot cos(21 \cdot x)}{\frac{sin(21 \cdot x)}{x}}=\lim\limits_{x\to\ 0}\frac{11 \cdot cos(21 \cdot x)}{21 \cdot \frac{sin(21 \cdot x)}{21 \cdot x}} = \frac{11}{21}$$
\rozwStop
\odpStart
$\frac{11}{21}$
\odpStop
\testStart
A.$\frac{11}{21}$
B.$\infty$
C.$-\infty$
D.$0$
E.$-\frac{11}{21}$
F.$\frac{21}{11}$
G.$-\frac{21}{11}$
H.$21$
I.$11$
\testStop
\kluczStart
A
\kluczStop



\zadStart{Przykład z Wikieł P 4.3a moja wersja nr 176}


Obliczyć granicę funkcji $\lim\limits_{x\to\ 0}\frac{11 \cdot x}{tan(23 \cdot x)}$.
\zadStop
\rozwStart{Patryk Wirkus}{}
$$\lim\limits_{x\to\ 0}\frac{11 \cdot x}{tan(23 \cdot x)}=\lim\limits_{x\to\ 0}\frac{11 \cdot x \cdot cos(23 \cdot x)}{sin(23 \cdot x)}=\lim\limits_{x\to\ 0}\frac{11 \cdot cos(23 \cdot x)}{\frac{sin(23 \cdot x)}{x}}=\lim\limits_{x\to\ 0}\frac{11 \cdot cos(23 \cdot x)}{23 \cdot \frac{sin(23 \cdot x)}{23 \cdot x}} = \frac{11}{23}$$
\rozwStop
\odpStart
$\frac{11}{23}$
\odpStop
\testStart
A.$\frac{11}{23}$
B.$\infty$
C.$-\infty$
D.$0$
E.$-\frac{11}{23}$
F.$\frac{23}{11}$
G.$-\frac{23}{11}$
H.$23$
I.$11$
\testStop
\kluczStart
A
\kluczStop



\zadStart{Przykład z Wikieł P 4.3a moja wersja nr 177}


Obliczyć granicę funkcji $\lim\limits_{x\to\ 0}\frac{11 \cdot x}{tan(24 \cdot x)}$.
\zadStop
\rozwStart{Patryk Wirkus}{}
$$\lim\limits_{x\to\ 0}\frac{11 \cdot x}{tan(24 \cdot x)}=\lim\limits_{x\to\ 0}\frac{11 \cdot x \cdot cos(24 \cdot x)}{sin(24 \cdot x)}=\lim\limits_{x\to\ 0}\frac{11 \cdot cos(24 \cdot x)}{\frac{sin(24 \cdot x)}{x}}=\lim\limits_{x\to\ 0}\frac{11 \cdot cos(24 \cdot x)}{24 \cdot \frac{sin(24 \cdot x)}{24 \cdot x}} = \frac{11}{24}$$
\rozwStop
\odpStart
$\frac{11}{24}$
\odpStop
\testStart
A.$\frac{11}{24}$
B.$\infty$
C.$-\infty$
D.$0$
E.$-\frac{11}{24}$
F.$\frac{24}{11}$
G.$-\frac{24}{11}$
H.$24$
I.$11$
\testStop
\kluczStart
A
\kluczStop



\zadStart{Przykład z Wikieł P 4.3a moja wersja nr 178}


Obliczyć granicę funkcji $\lim\limits_{x\to\ 0}\frac{11 \cdot x}{tan(25 \cdot x)}$.
\zadStop
\rozwStart{Patryk Wirkus}{}
$$\lim\limits_{x\to\ 0}\frac{11 \cdot x}{tan(25 \cdot x)}=\lim\limits_{x\to\ 0}\frac{11 \cdot x \cdot cos(25 \cdot x)}{sin(25 \cdot x)}=\lim\limits_{x\to\ 0}\frac{11 \cdot cos(25 \cdot x)}{\frac{sin(25 \cdot x)}{x}}=\lim\limits_{x\to\ 0}\frac{11 \cdot cos(25 \cdot x)}{25 \cdot \frac{sin(25 \cdot x)}{25 \cdot x}} = \frac{11}{25}$$
\rozwStop
\odpStart
$\frac{11}{25}$
\odpStop
\testStart
A.$\frac{11}{25}$
B.$\infty$
C.$-\infty$
D.$0$
E.$-\frac{11}{25}$
F.$\frac{25}{11}$
G.$-\frac{25}{11}$
H.$25$
I.$11$
\testStop
\kluczStart
A
\kluczStop



\zadStart{Przykład z Wikieł P 4.3a moja wersja nr 179}


Obliczyć granicę funkcji $\lim\limits_{x\to\ 0}\frac{11 \cdot x}{tan(26 \cdot x)}$.
\zadStop
\rozwStart{Patryk Wirkus}{}
$$\lim\limits_{x\to\ 0}\frac{11 \cdot x}{tan(26 \cdot x)}=\lim\limits_{x\to\ 0}\frac{11 \cdot x \cdot cos(26 \cdot x)}{sin(26 \cdot x)}=\lim\limits_{x\to\ 0}\frac{11 \cdot cos(26 \cdot x)}{\frac{sin(26 \cdot x)}{x}}=\lim\limits_{x\to\ 0}\frac{11 \cdot cos(26 \cdot x)}{26 \cdot \frac{sin(26 \cdot x)}{26 \cdot x}} = \frac{11}{26}$$
\rozwStop
\odpStart
$\frac{11}{26}$
\odpStop
\testStart
A.$\frac{11}{26}$
B.$\infty$
C.$-\infty$
D.$0$
E.$-\frac{11}{26}$
F.$\frac{26}{11}$
G.$-\frac{26}{11}$
H.$26$
I.$11$
\testStop
\kluczStart
A
\kluczStop



\zadStart{Przykład z Wikieł P 4.3a moja wersja nr 180}


Obliczyć granicę funkcji $\lim\limits_{x\to\ 0}\frac{11 \cdot x}{tan(27 \cdot x)}$.
\zadStop
\rozwStart{Patryk Wirkus}{}
$$\lim\limits_{x\to\ 0}\frac{11 \cdot x}{tan(27 \cdot x)}=\lim\limits_{x\to\ 0}\frac{11 \cdot x \cdot cos(27 \cdot x)}{sin(27 \cdot x)}=\lim\limits_{x\to\ 0}\frac{11 \cdot cos(27 \cdot x)}{\frac{sin(27 \cdot x)}{x}}=\lim\limits_{x\to\ 0}\frac{11 \cdot cos(27 \cdot x)}{27 \cdot \frac{sin(27 \cdot x)}{27 \cdot x}} = \frac{11}{27}$$
\rozwStop
\odpStart
$\frac{11}{27}$
\odpStop
\testStart
A.$\frac{11}{27}$
B.$\infty$
C.$-\infty$
D.$0$
E.$-\frac{11}{27}$
F.$\frac{27}{11}$
G.$-\frac{27}{11}$
H.$27$
I.$11$
\testStop
\kluczStart
A
\kluczStop



\zadStart{Przykład z Wikieł P 4.3a moja wersja nr 181}


Obliczyć granicę funkcji $\lim\limits_{x\to\ 0}\frac{11 \cdot x}{tan(28 \cdot x)}$.
\zadStop
\rozwStart{Patryk Wirkus}{}
$$\lim\limits_{x\to\ 0}\frac{11 \cdot x}{tan(28 \cdot x)}=\lim\limits_{x\to\ 0}\frac{11 \cdot x \cdot cos(28 \cdot x)}{sin(28 \cdot x)}=\lim\limits_{x\to\ 0}\frac{11 \cdot cos(28 \cdot x)}{\frac{sin(28 \cdot x)}{x}}=\lim\limits_{x\to\ 0}\frac{11 \cdot cos(28 \cdot x)}{28 \cdot \frac{sin(28 \cdot x)}{28 \cdot x}} = \frac{11}{28}$$
\rozwStop
\odpStart
$\frac{11}{28}$
\odpStop
\testStart
A.$\frac{11}{28}$
B.$\infty$
C.$-\infty$
D.$0$
E.$-\frac{11}{28}$
F.$\frac{28}{11}$
G.$-\frac{28}{11}$
H.$28$
I.$11$
\testStop
\kluczStart
A
\kluczStop



\zadStart{Przykład z Wikieł P 4.3a moja wersja nr 182}


Obliczyć granicę funkcji $\lim\limits_{x\to\ 0}\frac{11 \cdot x}{tan(29 \cdot x)}$.
\zadStop
\rozwStart{Patryk Wirkus}{}
$$\lim\limits_{x\to\ 0}\frac{11 \cdot x}{tan(29 \cdot x)}=\lim\limits_{x\to\ 0}\frac{11 \cdot x \cdot cos(29 \cdot x)}{sin(29 \cdot x)}=\lim\limits_{x\to\ 0}\frac{11 \cdot cos(29 \cdot x)}{\frac{sin(29 \cdot x)}{x}}=\lim\limits_{x\to\ 0}\frac{11 \cdot cos(29 \cdot x)}{29 \cdot \frac{sin(29 \cdot x)}{29 \cdot x}} = \frac{11}{29}$$
\rozwStop
\odpStart
$\frac{11}{29}$
\odpStop
\testStart
A.$\frac{11}{29}$
B.$\infty$
C.$-\infty$
D.$0$
E.$-\frac{11}{29}$
F.$\frac{29}{11}$
G.$-\frac{29}{11}$
H.$29$
I.$11$
\testStop
\kluczStart
A
\kluczStop



\zadStart{Przykład z Wikieł P 4.3a moja wersja nr 183}


Obliczyć granicę funkcji $\lim\limits_{x\to\ 0}\frac{11 \cdot x}{tan(30 \cdot x)}$.
\zadStop
\rozwStart{Patryk Wirkus}{}
$$\lim\limits_{x\to\ 0}\frac{11 \cdot x}{tan(30 \cdot x)}=\lim\limits_{x\to\ 0}\frac{11 \cdot x \cdot cos(30 \cdot x)}{sin(30 \cdot x)}=\lim\limits_{x\to\ 0}\frac{11 \cdot cos(30 \cdot x)}{\frac{sin(30 \cdot x)}{x}}=\lim\limits_{x\to\ 0}\frac{11 \cdot cos(30 \cdot x)}{30 \cdot \frac{sin(30 \cdot x)}{30 \cdot x}} = \frac{11}{30}$$
\rozwStop
\odpStart
$\frac{11}{30}$
\odpStop
\testStart
A.$\frac{11}{30}$
B.$\infty$
C.$-\infty$
D.$0$
E.$-\frac{11}{30}$
F.$\frac{30}{11}$
G.$-\frac{30}{11}$
H.$30$
I.$11$
\testStop
\kluczStart
A
\kluczStop



\zadStart{Przykład z Wikieł P 4.3a moja wersja nr 184}


Obliczyć granicę funkcji $\lim\limits_{x\to\ 0}\frac{11 \cdot x}{tan(31 \cdot x)}$.
\zadStop
\rozwStart{Patryk Wirkus}{}
$$\lim\limits_{x\to\ 0}\frac{11 \cdot x}{tan(31 \cdot x)}=\lim\limits_{x\to\ 0}\frac{11 \cdot x \cdot cos(31 \cdot x)}{sin(31 \cdot x)}=\lim\limits_{x\to\ 0}\frac{11 \cdot cos(31 \cdot x)}{\frac{sin(31 \cdot x)}{x}}=\lim\limits_{x\to\ 0}\frac{11 \cdot cos(31 \cdot x)}{31 \cdot \frac{sin(31 \cdot x)}{31 \cdot x}} = \frac{11}{31}$$
\rozwStop
\odpStart
$\frac{11}{31}$
\odpStop
\testStart
A.$\frac{11}{31}$
B.$\infty$
C.$-\infty$
D.$0$
E.$-\frac{11}{31}$
F.$\frac{31}{11}$
G.$-\frac{31}{11}$
H.$31$
I.$11$
\testStop
\kluczStart
A
\kluczStop



\zadStart{Przykład z Wikieł P 4.3a moja wersja nr 185}


Obliczyć granicę funkcji $\lim\limits_{x\to\ 0}\frac{11 \cdot x}{tan(32 \cdot x)}$.
\zadStop
\rozwStart{Patryk Wirkus}{}
$$\lim\limits_{x\to\ 0}\frac{11 \cdot x}{tan(32 \cdot x)}=\lim\limits_{x\to\ 0}\frac{11 \cdot x \cdot cos(32 \cdot x)}{sin(32 \cdot x)}=\lim\limits_{x\to\ 0}\frac{11 \cdot cos(32 \cdot x)}{\frac{sin(32 \cdot x)}{x}}=\lim\limits_{x\to\ 0}\frac{11 \cdot cos(32 \cdot x)}{32 \cdot \frac{sin(32 \cdot x)}{32 \cdot x}} = \frac{11}{32}$$
\rozwStop
\odpStart
$\frac{11}{32}$
\odpStop
\testStart
A.$\frac{11}{32}$
B.$\infty$
C.$-\infty$
D.$0$
E.$-\frac{11}{32}$
F.$\frac{32}{11}$
G.$-\frac{32}{11}$
H.$32$
I.$11$
\testStop
\kluczStart
A
\kluczStop



\zadStart{Przykład z Wikieł P 4.3a moja wersja nr 186}


Obliczyć granicę funkcji $\lim\limits_{x\to\ 0}\frac{11 \cdot x}{tan(34 \cdot x)}$.
\zadStop
\rozwStart{Patryk Wirkus}{}
$$\lim\limits_{x\to\ 0}\frac{11 \cdot x}{tan(34 \cdot x)}=\lim\limits_{x\to\ 0}\frac{11 \cdot x \cdot cos(34 \cdot x)}{sin(34 \cdot x)}=\lim\limits_{x\to\ 0}\frac{11 \cdot cos(34 \cdot x)}{\frac{sin(34 \cdot x)}{x}}=\lim\limits_{x\to\ 0}\frac{11 \cdot cos(34 \cdot x)}{34 \cdot \frac{sin(34 \cdot x)}{34 \cdot x}} = \frac{11}{34}$$
\rozwStop
\odpStart
$\frac{11}{34}$
\odpStop
\testStart
A.$\frac{11}{34}$
B.$\infty$
C.$-\infty$
D.$0$
E.$-\frac{11}{34}$
F.$\frac{34}{11}$
G.$-\frac{34}{11}$
H.$34$
I.$11$
\testStop
\kluczStart
A
\kluczStop



\zadStart{Przykład z Wikieł P 4.3a moja wersja nr 187}


Obliczyć granicę funkcji $\lim\limits_{x\to\ 0}\frac{11 \cdot x}{tan(35 \cdot x)}$.
\zadStop
\rozwStart{Patryk Wirkus}{}
$$\lim\limits_{x\to\ 0}\frac{11 \cdot x}{tan(35 \cdot x)}=\lim\limits_{x\to\ 0}\frac{11 \cdot x \cdot cos(35 \cdot x)}{sin(35 \cdot x)}=\lim\limits_{x\to\ 0}\frac{11 \cdot cos(35 \cdot x)}{\frac{sin(35 \cdot x)}{x}}=\lim\limits_{x\to\ 0}\frac{11 \cdot cos(35 \cdot x)}{35 \cdot \frac{sin(35 \cdot x)}{35 \cdot x}} = \frac{11}{35}$$
\rozwStop
\odpStart
$\frac{11}{35}$
\odpStop
\testStart
A.$\frac{11}{35}$
B.$\infty$
C.$-\infty$
D.$0$
E.$-\frac{11}{35}$
F.$\frac{35}{11}$
G.$-\frac{35}{11}$
H.$35$
I.$11$
\testStop
\kluczStart
A
\kluczStop



\zadStart{Przykład z Wikieł P 4.3a moja wersja nr 188}


Obliczyć granicę funkcji $\lim\limits_{x\to\ 0}\frac{11 \cdot x}{tan(36 \cdot x)}$.
\zadStop
\rozwStart{Patryk Wirkus}{}
$$\lim\limits_{x\to\ 0}\frac{11 \cdot x}{tan(36 \cdot x)}=\lim\limits_{x\to\ 0}\frac{11 \cdot x \cdot cos(36 \cdot x)}{sin(36 \cdot x)}=\lim\limits_{x\to\ 0}\frac{11 \cdot cos(36 \cdot x)}{\frac{sin(36 \cdot x)}{x}}=\lim\limits_{x\to\ 0}\frac{11 \cdot cos(36 \cdot x)}{36 \cdot \frac{sin(36 \cdot x)}{36 \cdot x}} = \frac{11}{36}$$
\rozwStop
\odpStart
$\frac{11}{36}$
\odpStop
\testStart
A.$\frac{11}{36}$
B.$\infty$
C.$-\infty$
D.$0$
E.$-\frac{11}{36}$
F.$\frac{36}{11}$
G.$-\frac{36}{11}$
H.$36$
I.$11$
\testStop
\kluczStart
A
\kluczStop



\zadStart{Przykład z Wikieł P 4.3a moja wersja nr 189}


Obliczyć granicę funkcji $\lim\limits_{x\to\ 0}\frac{11 \cdot x}{tan(37 \cdot x)}$.
\zadStop
\rozwStart{Patryk Wirkus}{}
$$\lim\limits_{x\to\ 0}\frac{11 \cdot x}{tan(37 \cdot x)}=\lim\limits_{x\to\ 0}\frac{11 \cdot x \cdot cos(37 \cdot x)}{sin(37 \cdot x)}=\lim\limits_{x\to\ 0}\frac{11 \cdot cos(37 \cdot x)}{\frac{sin(37 \cdot x)}{x}}=\lim\limits_{x\to\ 0}\frac{11 \cdot cos(37 \cdot x)}{37 \cdot \frac{sin(37 \cdot x)}{37 \cdot x}} = \frac{11}{37}$$
\rozwStop
\odpStart
$\frac{11}{37}$
\odpStop
\testStart
A.$\frac{11}{37}$
B.$\infty$
C.$-\infty$
D.$0$
E.$-\frac{11}{37}$
F.$\frac{37}{11}$
G.$-\frac{37}{11}$
H.$37$
I.$11$
\testStop
\kluczStart
A
\kluczStop



\zadStart{Przykład z Wikieł P 4.3a moja wersja nr 190}


Obliczyć granicę funkcji $\lim\limits_{x\to\ 0}\frac{11 \cdot x}{tan(38 \cdot x)}$.
\zadStop
\rozwStart{Patryk Wirkus}{}
$$\lim\limits_{x\to\ 0}\frac{11 \cdot x}{tan(38 \cdot x)}=\lim\limits_{x\to\ 0}\frac{11 \cdot x \cdot cos(38 \cdot x)}{sin(38 \cdot x)}=\lim\limits_{x\to\ 0}\frac{11 \cdot cos(38 \cdot x)}{\frac{sin(38 \cdot x)}{x}}=\lim\limits_{x\to\ 0}\frac{11 \cdot cos(38 \cdot x)}{38 \cdot \frac{sin(38 \cdot x)}{38 \cdot x}} = \frac{11}{38}$$
\rozwStop
\odpStart
$\frac{11}{38}$
\odpStop
\testStart
A.$\frac{11}{38}$
B.$\infty$
C.$-\infty$
D.$0$
E.$-\frac{11}{38}$
F.$\frac{38}{11}$
G.$-\frac{38}{11}$
H.$38$
I.$11$
\testStop
\kluczStart
A
\kluczStop



\zadStart{Przykład z Wikieł P 4.3a moja wersja nr 191}


Obliczyć granicę funkcji $\lim\limits_{x\to\ 0}\frac{11 \cdot x}{tan(39 \cdot x)}$.
\zadStop
\rozwStart{Patryk Wirkus}{}
$$\lim\limits_{x\to\ 0}\frac{11 \cdot x}{tan(39 \cdot x)}=\lim\limits_{x\to\ 0}\frac{11 \cdot x \cdot cos(39 \cdot x)}{sin(39 \cdot x)}=\lim\limits_{x\to\ 0}\frac{11 \cdot cos(39 \cdot x)}{\frac{sin(39 \cdot x)}{x}}=\lim\limits_{x\to\ 0}\frac{11 \cdot cos(39 \cdot x)}{39 \cdot \frac{sin(39 \cdot x)}{39 \cdot x}} = \frac{11}{39}$$
\rozwStop
\odpStart
$\frac{11}{39}$
\odpStop
\testStart
A.$\frac{11}{39}$
B.$\infty$
C.$-\infty$
D.$0$
E.$-\frac{11}{39}$
F.$\frac{39}{11}$
G.$-\frac{39}{11}$
H.$39$
I.$11$
\testStop
\kluczStart
A
\kluczStop



\zadStart{Przykład z Wikieł P 4.3a moja wersja nr 192}


Obliczyć granicę funkcji $\lim\limits_{x\to\ 0}\frac{11 \cdot x}{tan(40 \cdot x)}$.
\zadStop
\rozwStart{Patryk Wirkus}{}
$$\lim\limits_{x\to\ 0}\frac{11 \cdot x}{tan(40 \cdot x)}=\lim\limits_{x\to\ 0}\frac{11 \cdot x \cdot cos(40 \cdot x)}{sin(40 \cdot x)}=\lim\limits_{x\to\ 0}\frac{11 \cdot cos(40 \cdot x)}{\frac{sin(40 \cdot x)}{x}}=\lim\limits_{x\to\ 0}\frac{11 \cdot cos(40 \cdot x)}{40 \cdot \frac{sin(40 \cdot x)}{40 \cdot x}} = \frac{11}{40}$$
\rozwStop
\odpStart
$\frac{11}{40}$
\odpStop
\testStart
A.$\frac{11}{40}$
B.$\infty$
C.$-\infty$
D.$0$
E.$-\frac{11}{40}$
F.$\frac{40}{11}$
G.$-\frac{40}{11}$
H.$40$
I.$11$
\testStop
\kluczStart
A
\kluczStop



\zadStart{Przykład z Wikieł P 4.3a moja wersja nr 193}


Obliczyć granicę funkcji $\lim\limits_{x\to\ 0}\frac{12 \cdot x}{tan(5 \cdot x)}$.
\zadStop
\rozwStart{Patryk Wirkus}{}
$$\lim\limits_{x\to\ 0}\frac{12 \cdot x}{tan(5 \cdot x)}=\lim\limits_{x\to\ 0}\frac{12 \cdot x \cdot cos(5 \cdot x)}{sin(5 \cdot x)}=\lim\limits_{x\to\ 0}\frac{12 \cdot cos(5 \cdot x)}{\frac{sin(5 \cdot x)}{x}}=\lim\limits_{x\to\ 0}\frac{12 \cdot cos(5 \cdot x)}{5 \cdot \frac{sin(5 \cdot x)}{5 \cdot x}} = \frac{12}{5}$$
\rozwStop
\odpStart
$\frac{12}{5}$
\odpStop
\testStart
A.$\frac{12}{5}$
B.$\infty$
C.$-\infty$
D.$0$
E.$-\frac{12}{5}$
F.$\frac{5}{12}$
G.$-\frac{5}{12}$
H.$5$
I.$12$
\testStop
\kluczStart
A
\kluczStop



\zadStart{Przykład z Wikieł P 4.3a moja wersja nr 194}


Obliczyć granicę funkcji $\lim\limits_{x\to\ 0}\frac{12 \cdot x}{tan(7 \cdot x)}$.
\zadStop
\rozwStart{Patryk Wirkus}{}
$$\lim\limits_{x\to\ 0}\frac{12 \cdot x}{tan(7 \cdot x)}=\lim\limits_{x\to\ 0}\frac{12 \cdot x \cdot cos(7 \cdot x)}{sin(7 \cdot x)}=\lim\limits_{x\to\ 0}\frac{12 \cdot cos(7 \cdot x)}{\frac{sin(7 \cdot x)}{x}}=\lim\limits_{x\to\ 0}\frac{12 \cdot cos(7 \cdot x)}{7 \cdot \frac{sin(7 \cdot x)}{7 \cdot x}} = \frac{12}{7}$$
\rozwStop
\odpStart
$\frac{12}{7}$
\odpStop
\testStart
A.$\frac{12}{7}$
B.$\infty$
C.$-\infty$
D.$0$
E.$-\frac{12}{7}$
F.$\frac{7}{12}$
G.$-\frac{7}{12}$
H.$7$
I.$12$
\testStop
\kluczStart
A
\kluczStop



\zadStart{Przykład z Wikieł P 4.3a moja wersja nr 195}


Obliczyć granicę funkcji $\lim\limits_{x\to\ 0}\frac{12 \cdot x}{tan(11 \cdot x)}$.
\zadStop
\rozwStart{Patryk Wirkus}{}
$$\lim\limits_{x\to\ 0}\frac{12 \cdot x}{tan(11 \cdot x)}=\lim\limits_{x\to\ 0}\frac{12 \cdot x \cdot cos(11 \cdot x)}{sin(11 \cdot x)}=\lim\limits_{x\to\ 0}\frac{12 \cdot cos(11 \cdot x)}{\frac{sin(11 \cdot x)}{x}}=\lim\limits_{x\to\ 0}\frac{12 \cdot cos(11 \cdot x)}{11 \cdot \frac{sin(11 \cdot x)}{11 \cdot x}} = \frac{12}{11}$$
\rozwStop
\odpStart
$\frac{12}{11}$
\odpStop
\testStart
A.$\frac{12}{11}$
B.$\infty$
C.$-\infty$
D.$0$
E.$-\frac{12}{11}$
F.$\frac{11}{12}$
G.$-\frac{11}{12}$
H.$11$
I.$12$
\testStop
\kluczStart
A
\kluczStop



\zadStart{Przykład z Wikieł P 4.3a moja wersja nr 196}


Obliczyć granicę funkcji $\lim\limits_{x\to\ 0}\frac{12 \cdot x}{tan(13 \cdot x)}$.
\zadStop
\rozwStart{Patryk Wirkus}{}
$$\lim\limits_{x\to\ 0}\frac{12 \cdot x}{tan(13 \cdot x)}=\lim\limits_{x\to\ 0}\frac{12 \cdot x \cdot cos(13 \cdot x)}{sin(13 \cdot x)}=\lim\limits_{x\to\ 0}\frac{12 \cdot cos(13 \cdot x)}{\frac{sin(13 \cdot x)}{x}}=\lim\limits_{x\to\ 0}\frac{12 \cdot cos(13 \cdot x)}{13 \cdot \frac{sin(13 \cdot x)}{13 \cdot x}} = \frac{12}{13}$$
\rozwStop
\odpStart
$\frac{12}{13}$
\odpStop
\testStart
A.$\frac{12}{13}$
B.$\infty$
C.$-\infty$
D.$0$
E.$-\frac{12}{13}$
F.$\frac{13}{12}$
G.$-\frac{13}{12}$
H.$13$
I.$12$
\testStop
\kluczStart
A
\kluczStop



\zadStart{Przykład z Wikieł P 4.3a moja wersja nr 197}


Obliczyć granicę funkcji $\lim\limits_{x\to\ 0}\frac{12 \cdot x}{tan(17 \cdot x)}$.
\zadStop
\rozwStart{Patryk Wirkus}{}
$$\lim\limits_{x\to\ 0}\frac{12 \cdot x}{tan(17 \cdot x)}=\lim\limits_{x\to\ 0}\frac{12 \cdot x \cdot cos(17 \cdot x)}{sin(17 \cdot x)}=\lim\limits_{x\to\ 0}\frac{12 \cdot cos(17 \cdot x)}{\frac{sin(17 \cdot x)}{x}}=\lim\limits_{x\to\ 0}\frac{12 \cdot cos(17 \cdot x)}{17 \cdot \frac{sin(17 \cdot x)}{17 \cdot x}} = \frac{12}{17}$$
\rozwStop
\odpStart
$\frac{12}{17}$
\odpStop
\testStart
A.$\frac{12}{17}$
B.$\infty$
C.$-\infty$
D.$0$
E.$-\frac{12}{17}$
F.$\frac{17}{12}$
G.$-\frac{17}{12}$
H.$17$
I.$12$
\testStop
\kluczStart
A
\kluczStop



\zadStart{Przykład z Wikieł P 4.3a moja wersja nr 198}


Obliczyć granicę funkcji $\lim\limits_{x\to\ 0}\frac{12 \cdot x}{tan(19 \cdot x)}$.
\zadStop
\rozwStart{Patryk Wirkus}{}
$$\lim\limits_{x\to\ 0}\frac{12 \cdot x}{tan(19 \cdot x)}=\lim\limits_{x\to\ 0}\frac{12 \cdot x \cdot cos(19 \cdot x)}{sin(19 \cdot x)}=\lim\limits_{x\to\ 0}\frac{12 \cdot cos(19 \cdot x)}{\frac{sin(19 \cdot x)}{x}}=\lim\limits_{x\to\ 0}\frac{12 \cdot cos(19 \cdot x)}{19 \cdot \frac{sin(19 \cdot x)}{19 \cdot x}} = \frac{12}{19}$$
\rozwStop
\odpStart
$\frac{12}{19}$
\odpStop
\testStart
A.$\frac{12}{19}$
B.$\infty$
C.$-\infty$
D.$0$
E.$-\frac{12}{19}$
F.$\frac{19}{12}$
G.$-\frac{19}{12}$
H.$19$
I.$12$
\testStop
\kluczStart
A
\kluczStop



\zadStart{Przykład z Wikieł P 4.3a moja wersja nr 199}


Obliczyć granicę funkcji $\lim\limits_{x\to\ 0}\frac{12 \cdot x}{tan(23 \cdot x)}$.
\zadStop
\rozwStart{Patryk Wirkus}{}
$$\lim\limits_{x\to\ 0}\frac{12 \cdot x}{tan(23 \cdot x)}=\lim\limits_{x\to\ 0}\frac{12 \cdot x \cdot cos(23 \cdot x)}{sin(23 \cdot x)}=\lim\limits_{x\to\ 0}\frac{12 \cdot cos(23 \cdot x)}{\frac{sin(23 \cdot x)}{x}}=\lim\limits_{x\to\ 0}\frac{12 \cdot cos(23 \cdot x)}{23 \cdot \frac{sin(23 \cdot x)}{23 \cdot x}} = \frac{12}{23}$$
\rozwStop
\odpStart
$\frac{12}{23}$
\odpStop
\testStart
A.$\frac{12}{23}$
B.$\infty$
C.$-\infty$
D.$0$
E.$-\frac{12}{23}$
F.$\frac{23}{12}$
G.$-\frac{23}{12}$
H.$23$
I.$12$
\testStop
\kluczStart
A
\kluczStop



\zadStart{Przykład z Wikieł P 4.3a moja wersja nr 200}


Obliczyć granicę funkcji $\lim\limits_{x\to\ 0}\frac{12 \cdot x}{tan(25 \cdot x)}$.
\zadStop
\rozwStart{Patryk Wirkus}{}
$$\lim\limits_{x\to\ 0}\frac{12 \cdot x}{tan(25 \cdot x)}=\lim\limits_{x\to\ 0}\frac{12 \cdot x \cdot cos(25 \cdot x)}{sin(25 \cdot x)}=\lim\limits_{x\to\ 0}\frac{12 \cdot cos(25 \cdot x)}{\frac{sin(25 \cdot x)}{x}}=\lim\limits_{x\to\ 0}\frac{12 \cdot cos(25 \cdot x)}{25 \cdot \frac{sin(25 \cdot x)}{25 \cdot x}} = \frac{12}{25}$$
\rozwStop
\odpStart
$\frac{12}{25}$
\odpStop
\testStart
A.$\frac{12}{25}$
B.$\infty$
C.$-\infty$
D.$0$
E.$-\frac{12}{25}$
F.$\frac{25}{12}$
G.$-\frac{25}{12}$
H.$25$
I.$12$
\testStop
\kluczStart
A
\kluczStop



\zadStart{Przykład z Wikieł P 4.3a moja wersja nr 201}


Obliczyć granicę funkcji $\lim\limits_{x\to\ 0}\frac{12 \cdot x}{tan(29 \cdot x)}$.
\zadStop
\rozwStart{Patryk Wirkus}{}
$$\lim\limits_{x\to\ 0}\frac{12 \cdot x}{tan(29 \cdot x)}=\lim\limits_{x\to\ 0}\frac{12 \cdot x \cdot cos(29 \cdot x)}{sin(29 \cdot x)}=\lim\limits_{x\to\ 0}\frac{12 \cdot cos(29 \cdot x)}{\frac{sin(29 \cdot x)}{x}}=\lim\limits_{x\to\ 0}\frac{12 \cdot cos(29 \cdot x)}{29 \cdot \frac{sin(29 \cdot x)}{29 \cdot x}} = \frac{12}{29}$$
\rozwStop
\odpStart
$\frac{12}{29}$
\odpStop
\testStart
A.$\frac{12}{29}$
B.$\infty$
C.$-\infty$
D.$0$
E.$-\frac{12}{29}$
F.$\frac{29}{12}$
G.$-\frac{29}{12}$
H.$29$
I.$12$
\testStop
\kluczStart
A
\kluczStop



\zadStart{Przykład z Wikieł P 4.3a moja wersja nr 202}


Obliczyć granicę funkcji $\lim\limits_{x\to\ 0}\frac{12 \cdot x}{tan(31 \cdot x)}$.
\zadStop
\rozwStart{Patryk Wirkus}{}
$$\lim\limits_{x\to\ 0}\frac{12 \cdot x}{tan(31 \cdot x)}=\lim\limits_{x\to\ 0}\frac{12 \cdot x \cdot cos(31 \cdot x)}{sin(31 \cdot x)}=\lim\limits_{x\to\ 0}\frac{12 \cdot cos(31 \cdot x)}{\frac{sin(31 \cdot x)}{x}}=\lim\limits_{x\to\ 0}\frac{12 \cdot cos(31 \cdot x)}{31 \cdot \frac{sin(31 \cdot x)}{31 \cdot x}} = \frac{12}{31}$$
\rozwStop
\odpStart
$\frac{12}{31}$
\odpStop
\testStart
A.$\frac{12}{31}$
B.$\infty$
C.$-\infty$
D.$0$
E.$-\frac{12}{31}$
F.$\frac{31}{12}$
G.$-\frac{31}{12}$
H.$31$
I.$12$
\testStop
\kluczStart
A
\kluczStop



\zadStart{Przykład z Wikieł P 4.3a moja wersja nr 203}


Obliczyć granicę funkcji $\lim\limits_{x\to\ 0}\frac{12 \cdot x}{tan(35 \cdot x)}$.
\zadStop
\rozwStart{Patryk Wirkus}{}
$$\lim\limits_{x\to\ 0}\frac{12 \cdot x}{tan(35 \cdot x)}=\lim\limits_{x\to\ 0}\frac{12 \cdot x \cdot cos(35 \cdot x)}{sin(35 \cdot x)}=\lim\limits_{x\to\ 0}\frac{12 \cdot cos(35 \cdot x)}{\frac{sin(35 \cdot x)}{x}}=\lim\limits_{x\to\ 0}\frac{12 \cdot cos(35 \cdot x)}{35 \cdot \frac{sin(35 \cdot x)}{35 \cdot x}} = \frac{12}{35}$$
\rozwStop
\odpStart
$\frac{12}{35}$
\odpStop
\testStart
A.$\frac{12}{35}$
B.$\infty$
C.$-\infty$
D.$0$
E.$-\frac{12}{35}$
F.$\frac{35}{12}$
G.$-\frac{35}{12}$
H.$35$
I.$12$
\testStop
\kluczStart
A
\kluczStop



\zadStart{Przykład z Wikieł P 4.3a moja wersja nr 204}


Obliczyć granicę funkcji $\lim\limits_{x\to\ 0}\frac{12 \cdot x}{tan(37 \cdot x)}$.
\zadStop
\rozwStart{Patryk Wirkus}{}
$$\lim\limits_{x\to\ 0}\frac{12 \cdot x}{tan(37 \cdot x)}=\lim\limits_{x\to\ 0}\frac{12 \cdot x \cdot cos(37 \cdot x)}{sin(37 \cdot x)}=\lim\limits_{x\to\ 0}\frac{12 \cdot cos(37 \cdot x)}{\frac{sin(37 \cdot x)}{x}}=\lim\limits_{x\to\ 0}\frac{12 \cdot cos(37 \cdot x)}{37 \cdot \frac{sin(37 \cdot x)}{37 \cdot x}} = \frac{12}{37}$$
\rozwStop
\odpStart
$\frac{12}{37}$
\odpStop
\testStart
A.$\frac{12}{37}$
B.$\infty$
C.$-\infty$
D.$0$
E.$-\frac{12}{37}$
F.$\frac{37}{12}$
G.$-\frac{37}{12}$
H.$37$
I.$12$
\testStop
\kluczStart
A
\kluczStop



\zadStart{Przykład z Wikieł P 4.3a moja wersja nr 205}


Obliczyć granicę funkcji $\lim\limits_{x\to\ 0}\frac{13 \cdot x}{tan(2 \cdot x)}$.
\zadStop
\rozwStart{Patryk Wirkus}{}
$$\lim\limits_{x\to\ 0}\frac{13 \cdot x}{tan(2 \cdot x)}=\lim\limits_{x\to\ 0}\frac{13 \cdot x \cdot cos(2 \cdot x)}{sin(2 \cdot x)}=\lim\limits_{x\to\ 0}\frac{13 \cdot cos(2 \cdot x)}{\frac{sin(2 \cdot x)}{x}}=\lim\limits_{x\to\ 0}\frac{13 \cdot cos(2 \cdot x)}{2 \cdot \frac{sin(2 \cdot x)}{2 \cdot x}} = \frac{13}{2}$$
\rozwStop
\odpStart
$\frac{13}{2}$
\odpStop
\testStart
A.$\frac{13}{2}$
B.$\infty$
C.$-\infty$
D.$0$
E.$-\frac{13}{2}$
F.$\frac{2}{13}$
G.$-\frac{2}{13}$
H.$2$
I.$13$
\testStop
\kluczStart
A
\kluczStop



\zadStart{Przykład z Wikieł P 4.3a moja wersja nr 206}


Obliczyć granicę funkcji $\lim\limits_{x\to\ 0}\frac{13 \cdot x}{tan(3 \cdot x)}$.
\zadStop
\rozwStart{Patryk Wirkus}{}
$$\lim\limits_{x\to\ 0}\frac{13 \cdot x}{tan(3 \cdot x)}=\lim\limits_{x\to\ 0}\frac{13 \cdot x \cdot cos(3 \cdot x)}{sin(3 \cdot x)}=\lim\limits_{x\to\ 0}\frac{13 \cdot cos(3 \cdot x)}{\frac{sin(3 \cdot x)}{x}}=\lim\limits_{x\to\ 0}\frac{13 \cdot cos(3 \cdot x)}{3 \cdot \frac{sin(3 \cdot x)}{3 \cdot x}} = \frac{13}{3}$$
\rozwStop
\odpStart
$\frac{13}{3}$
\odpStop
\testStart
A.$\frac{13}{3}$
B.$\infty$
C.$-\infty$
D.$0$
E.$-\frac{13}{3}$
F.$\frac{3}{13}$
G.$-\frac{3}{13}$
H.$3$
I.$13$
\testStop
\kluczStart
A
\kluczStop



\zadStart{Przykład z Wikieł P 4.3a moja wersja nr 207}


Obliczyć granicę funkcji $\lim\limits_{x\to\ 0}\frac{13 \cdot x}{tan(4 \cdot x)}$.
\zadStop
\rozwStart{Patryk Wirkus}{}
$$\lim\limits_{x\to\ 0}\frac{13 \cdot x}{tan(4 \cdot x)}=\lim\limits_{x\to\ 0}\frac{13 \cdot x \cdot cos(4 \cdot x)}{sin(4 \cdot x)}=\lim\limits_{x\to\ 0}\frac{13 \cdot cos(4 \cdot x)}{\frac{sin(4 \cdot x)}{x}}=\lim\limits_{x\to\ 0}\frac{13 \cdot cos(4 \cdot x)}{4 \cdot \frac{sin(4 \cdot x)}{4 \cdot x}} = \frac{13}{4}$$
\rozwStop
\odpStart
$\frac{13}{4}$
\odpStop
\testStart
A.$\frac{13}{4}$
B.$\infty$
C.$-\infty$
D.$0$
E.$-\frac{13}{4}$
F.$\frac{4}{13}$
G.$-\frac{4}{13}$
H.$4$
I.$13$
\testStop
\kluczStart
A
\kluczStop



\zadStart{Przykład z Wikieł P 4.3a moja wersja nr 208}


Obliczyć granicę funkcji $\lim\limits_{x\to\ 0}\frac{13 \cdot x}{tan(5 \cdot x)}$.
\zadStop
\rozwStart{Patryk Wirkus}{}
$$\lim\limits_{x\to\ 0}\frac{13 \cdot x}{tan(5 \cdot x)}=\lim\limits_{x\to\ 0}\frac{13 \cdot x \cdot cos(5 \cdot x)}{sin(5 \cdot x)}=\lim\limits_{x\to\ 0}\frac{13 \cdot cos(5 \cdot x)}{\frac{sin(5 \cdot x)}{x}}=\lim\limits_{x\to\ 0}\frac{13 \cdot cos(5 \cdot x)}{5 \cdot \frac{sin(5 \cdot x)}{5 \cdot x}} = \frac{13}{5}$$
\rozwStop
\odpStart
$\frac{13}{5}$
\odpStop
\testStart
A.$\frac{13}{5}$
B.$\infty$
C.$-\infty$
D.$0$
E.$-\frac{13}{5}$
F.$\frac{5}{13}$
G.$-\frac{5}{13}$
H.$5$
I.$13$
\testStop
\kluczStart
A
\kluczStop



\zadStart{Przykład z Wikieł P 4.3a moja wersja nr 209}


Obliczyć granicę funkcji $\lim\limits_{x\to\ 0}\frac{13 \cdot x}{tan(6 \cdot x)}$.
\zadStop
\rozwStart{Patryk Wirkus}{}
$$\lim\limits_{x\to\ 0}\frac{13 \cdot x}{tan(6 \cdot x)}=\lim\limits_{x\to\ 0}\frac{13 \cdot x \cdot cos(6 \cdot x)}{sin(6 \cdot x)}=\lim\limits_{x\to\ 0}\frac{13 \cdot cos(6 \cdot x)}{\frac{sin(6 \cdot x)}{x}}=\lim\limits_{x\to\ 0}\frac{13 \cdot cos(6 \cdot x)}{6 \cdot \frac{sin(6 \cdot x)}{6 \cdot x}} = \frac{13}{6}$$
\rozwStop
\odpStart
$\frac{13}{6}$
\odpStop
\testStart
A.$\frac{13}{6}$
B.$\infty$
C.$-\infty$
D.$0$
E.$-\frac{13}{6}$
F.$\frac{6}{13}$
G.$-\frac{6}{13}$
H.$6$
I.$13$
\testStop
\kluczStart
A
\kluczStop



\zadStart{Przykład z Wikieł P 4.3a moja wersja nr 210}


Obliczyć granicę funkcji $\lim\limits_{x\to\ 0}\frac{13 \cdot x}{tan(7 \cdot x)}$.
\zadStop
\rozwStart{Patryk Wirkus}{}
$$\lim\limits_{x\to\ 0}\frac{13 \cdot x}{tan(7 \cdot x)}=\lim\limits_{x\to\ 0}\frac{13 \cdot x \cdot cos(7 \cdot x)}{sin(7 \cdot x)}=\lim\limits_{x\to\ 0}\frac{13 \cdot cos(7 \cdot x)}{\frac{sin(7 \cdot x)}{x}}=\lim\limits_{x\to\ 0}\frac{13 \cdot cos(7 \cdot x)}{7 \cdot \frac{sin(7 \cdot x)}{7 \cdot x}} = \frac{13}{7}$$
\rozwStop
\odpStart
$\frac{13}{7}$
\odpStop
\testStart
A.$\frac{13}{7}$
B.$\infty$
C.$-\infty$
D.$0$
E.$-\frac{13}{7}$
F.$\frac{7}{13}$
G.$-\frac{7}{13}$
H.$7$
I.$13$
\testStop
\kluczStart
A
\kluczStop



\zadStart{Przykład z Wikieł P 4.3a moja wersja nr 211}


Obliczyć granicę funkcji $\lim\limits_{x\to\ 0}\frac{13 \cdot x}{tan(8 \cdot x)}$.
\zadStop
\rozwStart{Patryk Wirkus}{}
$$\lim\limits_{x\to\ 0}\frac{13 \cdot x}{tan(8 \cdot x)}=\lim\limits_{x\to\ 0}\frac{13 \cdot x \cdot cos(8 \cdot x)}{sin(8 \cdot x)}=\lim\limits_{x\to\ 0}\frac{13 \cdot cos(8 \cdot x)}{\frac{sin(8 \cdot x)}{x}}=\lim\limits_{x\to\ 0}\frac{13 \cdot cos(8 \cdot x)}{8 \cdot \frac{sin(8 \cdot x)}{8 \cdot x}} = \frac{13}{8}$$
\rozwStop
\odpStart
$\frac{13}{8}$
\odpStop
\testStart
A.$\frac{13}{8}$
B.$\infty$
C.$-\infty$
D.$0$
E.$-\frac{13}{8}$
F.$\frac{8}{13}$
G.$-\frac{8}{13}$
H.$8$
I.$13$
\testStop
\kluczStart
A
\kluczStop



\zadStart{Przykład z Wikieł P 4.3a moja wersja nr 212}


Obliczyć granicę funkcji $\lim\limits_{x\to\ 0}\frac{13 \cdot x}{tan(9 \cdot x)}$.
\zadStop
\rozwStart{Patryk Wirkus}{}
$$\lim\limits_{x\to\ 0}\frac{13 \cdot x}{tan(9 \cdot x)}=\lim\limits_{x\to\ 0}\frac{13 \cdot x \cdot cos(9 \cdot x)}{sin(9 \cdot x)}=\lim\limits_{x\to\ 0}\frac{13 \cdot cos(9 \cdot x)}{\frac{sin(9 \cdot x)}{x}}=\lim\limits_{x\to\ 0}\frac{13 \cdot cos(9 \cdot x)}{9 \cdot \frac{sin(9 \cdot x)}{9 \cdot x}} = \frac{13}{9}$$
\rozwStop
\odpStart
$\frac{13}{9}$
\odpStop
\testStart
A.$\frac{13}{9}$
B.$\infty$
C.$-\infty$
D.$0$
E.$-\frac{13}{9}$
F.$\frac{9}{13}$
G.$-\frac{9}{13}$
H.$9$
I.$13$
\testStop
\kluczStart
A
\kluczStop



\zadStart{Przykład z Wikieł P 4.3a moja wersja nr 213}


Obliczyć granicę funkcji $\lim\limits_{x\to\ 0}\frac{13 \cdot x}{tan(10 \cdot x)}$.
\zadStop
\rozwStart{Patryk Wirkus}{}
$$\lim\limits_{x\to\ 0}\frac{13 \cdot x}{tan(10 \cdot x)}=\lim\limits_{x\to\ 0}\frac{13 \cdot x \cdot cos(10 \cdot x)}{sin(10 \cdot x)}=\lim\limits_{x\to\ 0}\frac{13 \cdot cos(10 \cdot x)}{\frac{sin(10 \cdot x)}{x}}=\lim\limits_{x\to\ 0}\frac{13 \cdot cos(10 \cdot x)}{10 \cdot \frac{sin(10 \cdot x)}{10 \cdot x}} = \frac{13}{10}$$
\rozwStop
\odpStart
$\frac{13}{10}$
\odpStop
\testStart
A.$\frac{13}{10}$
B.$\infty$
C.$-\infty$
D.$0$
E.$-\frac{13}{10}$
F.$\frac{10}{13}$
G.$-\frac{10}{13}$
H.$10$
I.$13$
\testStop
\kluczStart
A
\kluczStop



\zadStart{Przykład z Wikieł P 4.3a moja wersja nr 214}


Obliczyć granicę funkcji $\lim\limits_{x\to\ 0}\frac{13 \cdot x}{tan(11 \cdot x)}$.
\zadStop
\rozwStart{Patryk Wirkus}{}
$$\lim\limits_{x\to\ 0}\frac{13 \cdot x}{tan(11 \cdot x)}=\lim\limits_{x\to\ 0}\frac{13 \cdot x \cdot cos(11 \cdot x)}{sin(11 \cdot x)}=\lim\limits_{x\to\ 0}\frac{13 \cdot cos(11 \cdot x)}{\frac{sin(11 \cdot x)}{x}}=\lim\limits_{x\to\ 0}\frac{13 \cdot cos(11 \cdot x)}{11 \cdot \frac{sin(11 \cdot x)}{11 \cdot x}} = \frac{13}{11}$$
\rozwStop
\odpStart
$\frac{13}{11}$
\odpStop
\testStart
A.$\frac{13}{11}$
B.$\infty$
C.$-\infty$
D.$0$
E.$-\frac{13}{11}$
F.$\frac{11}{13}$
G.$-\frac{11}{13}$
H.$11$
I.$13$
\testStop
\kluczStart
A
\kluczStop



\zadStart{Przykład z Wikieł P 4.3a moja wersja nr 215}


Obliczyć granicę funkcji $\lim\limits_{x\to\ 0}\frac{13 \cdot x}{tan(12 \cdot x)}$.
\zadStop
\rozwStart{Patryk Wirkus}{}
$$\lim\limits_{x\to\ 0}\frac{13 \cdot x}{tan(12 \cdot x)}=\lim\limits_{x\to\ 0}\frac{13 \cdot x \cdot cos(12 \cdot x)}{sin(12 \cdot x)}=\lim\limits_{x\to\ 0}\frac{13 \cdot cos(12 \cdot x)}{\frac{sin(12 \cdot x)}{x}}=\lim\limits_{x\to\ 0}\frac{13 \cdot cos(12 \cdot x)}{12 \cdot \frac{sin(12 \cdot x)}{12 \cdot x}} = \frac{13}{12}$$
\rozwStop
\odpStart
$\frac{13}{12}$
\odpStop
\testStart
A.$\frac{13}{12}$
B.$\infty$
C.$-\infty$
D.$0$
E.$-\frac{13}{12}$
F.$\frac{12}{13}$
G.$-\frac{12}{13}$
H.$12$
I.$13$
\testStop
\kluczStart
A
\kluczStop



\zadStart{Przykład z Wikieł P 4.3a moja wersja nr 216}


Obliczyć granicę funkcji $\lim\limits_{x\to\ 0}\frac{13 \cdot x}{tan(14 \cdot x)}$.
\zadStop
\rozwStart{Patryk Wirkus}{}
$$\lim\limits_{x\to\ 0}\frac{13 \cdot x}{tan(14 \cdot x)}=\lim\limits_{x\to\ 0}\frac{13 \cdot x \cdot cos(14 \cdot x)}{sin(14 \cdot x)}=\lim\limits_{x\to\ 0}\frac{13 \cdot cos(14 \cdot x)}{\frac{sin(14 \cdot x)}{x}}=\lim\limits_{x\to\ 0}\frac{13 \cdot cos(14 \cdot x)}{14 \cdot \frac{sin(14 \cdot x)}{14 \cdot x}} = \frac{13}{14}$$
\rozwStop
\odpStart
$\frac{13}{14}$
\odpStop
\testStart
A.$\frac{13}{14}$
B.$\infty$
C.$-\infty$
D.$0$
E.$-\frac{13}{14}$
F.$\frac{14}{13}$
G.$-\frac{14}{13}$
H.$14$
I.$13$
\testStop
\kluczStart
A
\kluczStop



\zadStart{Przykład z Wikieł P 4.3a moja wersja nr 217}


Obliczyć granicę funkcji $\lim\limits_{x\to\ 0}\frac{13 \cdot x}{tan(15 \cdot x)}$.
\zadStop
\rozwStart{Patryk Wirkus}{}
$$\lim\limits_{x\to\ 0}\frac{13 \cdot x}{tan(15 \cdot x)}=\lim\limits_{x\to\ 0}\frac{13 \cdot x \cdot cos(15 \cdot x)}{sin(15 \cdot x)}=\lim\limits_{x\to\ 0}\frac{13 \cdot cos(15 \cdot x)}{\frac{sin(15 \cdot x)}{x}}=\lim\limits_{x\to\ 0}\frac{13 \cdot cos(15 \cdot x)}{15 \cdot \frac{sin(15 \cdot x)}{15 \cdot x}} = \frac{13}{15}$$
\rozwStop
\odpStart
$\frac{13}{15}$
\odpStop
\testStart
A.$\frac{13}{15}$
B.$\infty$
C.$-\infty$
D.$0$
E.$-\frac{13}{15}$
F.$\frac{15}{13}$
G.$-\frac{15}{13}$
H.$15$
I.$13$
\testStop
\kluczStart
A
\kluczStop



\zadStart{Przykład z Wikieł P 4.3a moja wersja nr 218}


Obliczyć granicę funkcji $\lim\limits_{x\to\ 0}\frac{13 \cdot x}{tan(16 \cdot x)}$.
\zadStop
\rozwStart{Patryk Wirkus}{}
$$\lim\limits_{x\to\ 0}\frac{13 \cdot x}{tan(16 \cdot x)}=\lim\limits_{x\to\ 0}\frac{13 \cdot x \cdot cos(16 \cdot x)}{sin(16 \cdot x)}=\lim\limits_{x\to\ 0}\frac{13 \cdot cos(16 \cdot x)}{\frac{sin(16 \cdot x)}{x}}=\lim\limits_{x\to\ 0}\frac{13 \cdot cos(16 \cdot x)}{16 \cdot \frac{sin(16 \cdot x)}{16 \cdot x}} = \frac{13}{16}$$
\rozwStop
\odpStart
$\frac{13}{16}$
\odpStop
\testStart
A.$\frac{13}{16}$
B.$\infty$
C.$-\infty$
D.$0$
E.$-\frac{13}{16}$
F.$\frac{16}{13}$
G.$-\frac{16}{13}$
H.$16$
I.$13$
\testStop
\kluczStart
A
\kluczStop



\zadStart{Przykład z Wikieł P 4.3a moja wersja nr 219}


Obliczyć granicę funkcji $\lim\limits_{x\to\ 0}\frac{13 \cdot x}{tan(17 \cdot x)}$.
\zadStop
\rozwStart{Patryk Wirkus}{}
$$\lim\limits_{x\to\ 0}\frac{13 \cdot x}{tan(17 \cdot x)}=\lim\limits_{x\to\ 0}\frac{13 \cdot x \cdot cos(17 \cdot x)}{sin(17 \cdot x)}=\lim\limits_{x\to\ 0}\frac{13 \cdot cos(17 \cdot x)}{\frac{sin(17 \cdot x)}{x}}=\lim\limits_{x\to\ 0}\frac{13 \cdot cos(17 \cdot x)}{17 \cdot \frac{sin(17 \cdot x)}{17 \cdot x}} = \frac{13}{17}$$
\rozwStop
\odpStart
$\frac{13}{17}$
\odpStop
\testStart
A.$\frac{13}{17}$
B.$\infty$
C.$-\infty$
D.$0$
E.$-\frac{13}{17}$
F.$\frac{17}{13}$
G.$-\frac{17}{13}$
H.$17$
I.$13$
\testStop
\kluczStart
A
\kluczStop



\zadStart{Przykład z Wikieł P 4.3a moja wersja nr 220}


Obliczyć granicę funkcji $\lim\limits_{x\to\ 0}\frac{13 \cdot x}{tan(18 \cdot x)}$.
\zadStop
\rozwStart{Patryk Wirkus}{}
$$\lim\limits_{x\to\ 0}\frac{13 \cdot x}{tan(18 \cdot x)}=\lim\limits_{x\to\ 0}\frac{13 \cdot x \cdot cos(18 \cdot x)}{sin(18 \cdot x)}=\lim\limits_{x\to\ 0}\frac{13 \cdot cos(18 \cdot x)}{\frac{sin(18 \cdot x)}{x}}=\lim\limits_{x\to\ 0}\frac{13 \cdot cos(18 \cdot x)}{18 \cdot \frac{sin(18 \cdot x)}{18 \cdot x}} = \frac{13}{18}$$
\rozwStop
\odpStart
$\frac{13}{18}$
\odpStop
\testStart
A.$\frac{13}{18}$
B.$\infty$
C.$-\infty$
D.$0$
E.$-\frac{13}{18}$
F.$\frac{18}{13}$
G.$-\frac{18}{13}$
H.$18$
I.$13$
\testStop
\kluczStart
A
\kluczStop



\zadStart{Przykład z Wikieł P 4.3a moja wersja nr 221}


Obliczyć granicę funkcji $\lim\limits_{x\to\ 0}\frac{13 \cdot x}{tan(19 \cdot x)}$.
\zadStop
\rozwStart{Patryk Wirkus}{}
$$\lim\limits_{x\to\ 0}\frac{13 \cdot x}{tan(19 \cdot x)}=\lim\limits_{x\to\ 0}\frac{13 \cdot x \cdot cos(19 \cdot x)}{sin(19 \cdot x)}=\lim\limits_{x\to\ 0}\frac{13 \cdot cos(19 \cdot x)}{\frac{sin(19 \cdot x)}{x}}=\lim\limits_{x\to\ 0}\frac{13 \cdot cos(19 \cdot x)}{19 \cdot \frac{sin(19 \cdot x)}{19 \cdot x}} = \frac{13}{19}$$
\rozwStop
\odpStart
$\frac{13}{19}$
\odpStop
\testStart
A.$\frac{13}{19}$
B.$\infty$
C.$-\infty$
D.$0$
E.$-\frac{13}{19}$
F.$\frac{19}{13}$
G.$-\frac{19}{13}$
H.$19$
I.$13$
\testStop
\kluczStart
A
\kluczStop



\zadStart{Przykład z Wikieł P 4.3a moja wersja nr 222}


Obliczyć granicę funkcji $\lim\limits_{x\to\ 0}\frac{13 \cdot x}{tan(20 \cdot x)}$.
\zadStop
\rozwStart{Patryk Wirkus}{}
$$\lim\limits_{x\to\ 0}\frac{13 \cdot x}{tan(20 \cdot x)}=\lim\limits_{x\to\ 0}\frac{13 \cdot x \cdot cos(20 \cdot x)}{sin(20 \cdot x)}=\lim\limits_{x\to\ 0}\frac{13 \cdot cos(20 \cdot x)}{\frac{sin(20 \cdot x)}{x}}=\lim\limits_{x\to\ 0}\frac{13 \cdot cos(20 \cdot x)}{20 \cdot \frac{sin(20 \cdot x)}{20 \cdot x}} = \frac{13}{20}$$
\rozwStop
\odpStart
$\frac{13}{20}$
\odpStop
\testStart
A.$\frac{13}{20}$
B.$\infty$
C.$-\infty$
D.$0$
E.$-\frac{13}{20}$
F.$\frac{20}{13}$
G.$-\frac{20}{13}$
H.$20$
I.$13$
\testStop
\kluczStart
A
\kluczStop



\zadStart{Przykład z Wikieł P 4.3a moja wersja nr 223}


Obliczyć granicę funkcji $\lim\limits_{x\to\ 0}\frac{13 \cdot x}{tan(21 \cdot x)}$.
\zadStop
\rozwStart{Patryk Wirkus}{}
$$\lim\limits_{x\to\ 0}\frac{13 \cdot x}{tan(21 \cdot x)}=\lim\limits_{x\to\ 0}\frac{13 \cdot x \cdot cos(21 \cdot x)}{sin(21 \cdot x)}=\lim\limits_{x\to\ 0}\frac{13 \cdot cos(21 \cdot x)}{\frac{sin(21 \cdot x)}{x}}=\lim\limits_{x\to\ 0}\frac{13 \cdot cos(21 \cdot x)}{21 \cdot \frac{sin(21 \cdot x)}{21 \cdot x}} = \frac{13}{21}$$
\rozwStop
\odpStart
$\frac{13}{21}$
\odpStop
\testStart
A.$\frac{13}{21}$
B.$\infty$
C.$-\infty$
D.$0$
E.$-\frac{13}{21}$
F.$\frac{21}{13}$
G.$-\frac{21}{13}$
H.$21$
I.$13$
\testStop
\kluczStart
A
\kluczStop



\zadStart{Przykład z Wikieł P 4.3a moja wersja nr 224}


Obliczyć granicę funkcji $\lim\limits_{x\to\ 0}\frac{13 \cdot x}{tan(22 \cdot x)}$.
\zadStop
\rozwStart{Patryk Wirkus}{}
$$\lim\limits_{x\to\ 0}\frac{13 \cdot x}{tan(22 \cdot x)}=\lim\limits_{x\to\ 0}\frac{13 \cdot x \cdot cos(22 \cdot x)}{sin(22 \cdot x)}=\lim\limits_{x\to\ 0}\frac{13 \cdot cos(22 \cdot x)}{\frac{sin(22 \cdot x)}{x}}=\lim\limits_{x\to\ 0}\frac{13 \cdot cos(22 \cdot x)}{22 \cdot \frac{sin(22 \cdot x)}{22 \cdot x}} = \frac{13}{22}$$
\rozwStop
\odpStart
$\frac{13}{22}$
\odpStop
\testStart
A.$\frac{13}{22}$
B.$\infty$
C.$-\infty$
D.$0$
E.$-\frac{13}{22}$
F.$\frac{22}{13}$
G.$-\frac{22}{13}$
H.$22$
I.$13$
\testStop
\kluczStart
A
\kluczStop



\zadStart{Przykład z Wikieł P 4.3a moja wersja nr 225}


Obliczyć granicę funkcji $\lim\limits_{x\to\ 0}\frac{13 \cdot x}{tan(23 \cdot x)}$.
\zadStop
\rozwStart{Patryk Wirkus}{}
$$\lim\limits_{x\to\ 0}\frac{13 \cdot x}{tan(23 \cdot x)}=\lim\limits_{x\to\ 0}\frac{13 \cdot x \cdot cos(23 \cdot x)}{sin(23 \cdot x)}=\lim\limits_{x\to\ 0}\frac{13 \cdot cos(23 \cdot x)}{\frac{sin(23 \cdot x)}{x}}=\lim\limits_{x\to\ 0}\frac{13 \cdot cos(23 \cdot x)}{23 \cdot \frac{sin(23 \cdot x)}{23 \cdot x}} = \frac{13}{23}$$
\rozwStop
\odpStart
$\frac{13}{23}$
\odpStop
\testStart
A.$\frac{13}{23}$
B.$\infty$
C.$-\infty$
D.$0$
E.$-\frac{13}{23}$
F.$\frac{23}{13}$
G.$-\frac{23}{13}$
H.$23$
I.$13$
\testStop
\kluczStart
A
\kluczStop



\zadStart{Przykład z Wikieł P 4.3a moja wersja nr 226}


Obliczyć granicę funkcji $\lim\limits_{x\to\ 0}\frac{13 \cdot x}{tan(24 \cdot x)}$.
\zadStop
\rozwStart{Patryk Wirkus}{}
$$\lim\limits_{x\to\ 0}\frac{13 \cdot x}{tan(24 \cdot x)}=\lim\limits_{x\to\ 0}\frac{13 \cdot x \cdot cos(24 \cdot x)}{sin(24 \cdot x)}=\lim\limits_{x\to\ 0}\frac{13 \cdot cos(24 \cdot x)}{\frac{sin(24 \cdot x)}{x}}=\lim\limits_{x\to\ 0}\frac{13 \cdot cos(24 \cdot x)}{24 \cdot \frac{sin(24 \cdot x)}{24 \cdot x}} = \frac{13}{24}$$
\rozwStop
\odpStart
$\frac{13}{24}$
\odpStop
\testStart
A.$\frac{13}{24}$
B.$\infty$
C.$-\infty$
D.$0$
E.$-\frac{13}{24}$
F.$\frac{24}{13}$
G.$-\frac{24}{13}$
H.$24$
I.$13$
\testStop
\kluczStart
A
\kluczStop



\zadStart{Przykład z Wikieł P 4.3a moja wersja nr 227}


Obliczyć granicę funkcji $\lim\limits_{x\to\ 0}\frac{13 \cdot x}{tan(25 \cdot x)}$.
\zadStop
\rozwStart{Patryk Wirkus}{}
$$\lim\limits_{x\to\ 0}\frac{13 \cdot x}{tan(25 \cdot x)}=\lim\limits_{x\to\ 0}\frac{13 \cdot x \cdot cos(25 \cdot x)}{sin(25 \cdot x)}=\lim\limits_{x\to\ 0}\frac{13 \cdot cos(25 \cdot x)}{\frac{sin(25 \cdot x)}{x}}=\lim\limits_{x\to\ 0}\frac{13 \cdot cos(25 \cdot x)}{25 \cdot \frac{sin(25 \cdot x)}{25 \cdot x}} = \frac{13}{25}$$
\rozwStop
\odpStart
$\frac{13}{25}$
\odpStop
\testStart
A.$\frac{13}{25}$
B.$\infty$
C.$-\infty$
D.$0$
E.$-\frac{13}{25}$
F.$\frac{25}{13}$
G.$-\frac{25}{13}$
H.$25$
I.$13$
\testStop
\kluczStart
A
\kluczStop



\zadStart{Przykład z Wikieł P 4.3a moja wersja nr 228}


Obliczyć granicę funkcji $\lim\limits_{x\to\ 0}\frac{13 \cdot x}{tan(27 \cdot x)}$.
\zadStop
\rozwStart{Patryk Wirkus}{}
$$\lim\limits_{x\to\ 0}\frac{13 \cdot x}{tan(27 \cdot x)}=\lim\limits_{x\to\ 0}\frac{13 \cdot x \cdot cos(27 \cdot x)}{sin(27 \cdot x)}=\lim\limits_{x\to\ 0}\frac{13 \cdot cos(27 \cdot x)}{\frac{sin(27 \cdot x)}{x}}=\lim\limits_{x\to\ 0}\frac{13 \cdot cos(27 \cdot x)}{27 \cdot \frac{sin(27 \cdot x)}{27 \cdot x}} = \frac{13}{27}$$
\rozwStop
\odpStart
$\frac{13}{27}$
\odpStop
\testStart
A.$\frac{13}{27}$
B.$\infty$
C.$-\infty$
D.$0$
E.$-\frac{13}{27}$
F.$\frac{27}{13}$
G.$-\frac{27}{13}$
H.$27$
I.$13$
\testStop
\kluczStart
A
\kluczStop



\zadStart{Przykład z Wikieł P 4.3a moja wersja nr 229}


Obliczyć granicę funkcji $\lim\limits_{x\to\ 0}\frac{13 \cdot x}{tan(28 \cdot x)}$.
\zadStop
\rozwStart{Patryk Wirkus}{}
$$\lim\limits_{x\to\ 0}\frac{13 \cdot x}{tan(28 \cdot x)}=\lim\limits_{x\to\ 0}\frac{13 \cdot x \cdot cos(28 \cdot x)}{sin(28 \cdot x)}=\lim\limits_{x\to\ 0}\frac{13 \cdot cos(28 \cdot x)}{\frac{sin(28 \cdot x)}{x}}=\lim\limits_{x\to\ 0}\frac{13 \cdot cos(28 \cdot x)}{28 \cdot \frac{sin(28 \cdot x)}{28 \cdot x}} = \frac{13}{28}$$
\rozwStop
\odpStart
$\frac{13}{28}$
\odpStop
\testStart
A.$\frac{13}{28}$
B.$\infty$
C.$-\infty$
D.$0$
E.$-\frac{13}{28}$
F.$\frac{28}{13}$
G.$-\frac{28}{13}$
H.$28$
I.$13$
\testStop
\kluczStart
A
\kluczStop



\zadStart{Przykład z Wikieł P 4.3a moja wersja nr 230}


Obliczyć granicę funkcji $\lim\limits_{x\to\ 0}\frac{13 \cdot x}{tan(29 \cdot x)}$.
\zadStop
\rozwStart{Patryk Wirkus}{}
$$\lim\limits_{x\to\ 0}\frac{13 \cdot x}{tan(29 \cdot x)}=\lim\limits_{x\to\ 0}\frac{13 \cdot x \cdot cos(29 \cdot x)}{sin(29 \cdot x)}=\lim\limits_{x\to\ 0}\frac{13 \cdot cos(29 \cdot x)}{\frac{sin(29 \cdot x)}{x}}=\lim\limits_{x\to\ 0}\frac{13 \cdot cos(29 \cdot x)}{29 \cdot \frac{sin(29 \cdot x)}{29 \cdot x}} = \frac{13}{29}$$
\rozwStop
\odpStart
$\frac{13}{29}$
\odpStop
\testStart
A.$\frac{13}{29}$
B.$\infty$
C.$-\infty$
D.$0$
E.$-\frac{13}{29}$
F.$\frac{29}{13}$
G.$-\frac{29}{13}$
H.$29$
I.$13$
\testStop
\kluczStart
A
\kluczStop



\zadStart{Przykład z Wikieł P 4.3a moja wersja nr 231}


Obliczyć granicę funkcji $\lim\limits_{x\to\ 0}\frac{13 \cdot x}{tan(30 \cdot x)}$.
\zadStop
\rozwStart{Patryk Wirkus}{}
$$\lim\limits_{x\to\ 0}\frac{13 \cdot x}{tan(30 \cdot x)}=\lim\limits_{x\to\ 0}\frac{13 \cdot x \cdot cos(30 \cdot x)}{sin(30 \cdot x)}=\lim\limits_{x\to\ 0}\frac{13 \cdot cos(30 \cdot x)}{\frac{sin(30 \cdot x)}{x}}=\lim\limits_{x\to\ 0}\frac{13 \cdot cos(30 \cdot x)}{30 \cdot \frac{sin(30 \cdot x)}{30 \cdot x}} = \frac{13}{30}$$
\rozwStop
\odpStart
$\frac{13}{30}$
\odpStop
\testStart
A.$\frac{13}{30}$
B.$\infty$
C.$-\infty$
D.$0$
E.$-\frac{13}{30}$
F.$\frac{30}{13}$
G.$-\frac{30}{13}$
H.$30$
I.$13$
\testStop
\kluczStart
A
\kluczStop



\zadStart{Przykład z Wikieł P 4.3a moja wersja nr 232}


Obliczyć granicę funkcji $\lim\limits_{x\to\ 0}\frac{13 \cdot x}{tan(31 \cdot x)}$.
\zadStop
\rozwStart{Patryk Wirkus}{}
$$\lim\limits_{x\to\ 0}\frac{13 \cdot x}{tan(31 \cdot x)}=\lim\limits_{x\to\ 0}\frac{13 \cdot x \cdot cos(31 \cdot x)}{sin(31 \cdot x)}=\lim\limits_{x\to\ 0}\frac{13 \cdot cos(31 \cdot x)}{\frac{sin(31 \cdot x)}{x}}=\lim\limits_{x\to\ 0}\frac{13 \cdot cos(31 \cdot x)}{31 \cdot \frac{sin(31 \cdot x)}{31 \cdot x}} = \frac{13}{31}$$
\rozwStop
\odpStart
$\frac{13}{31}$
\odpStop
\testStart
A.$\frac{13}{31}$
B.$\infty$
C.$-\infty$
D.$0$
E.$-\frac{13}{31}$
F.$\frac{31}{13}$
G.$-\frac{31}{13}$
H.$31$
I.$13$
\testStop
\kluczStart
A
\kluczStop



\zadStart{Przykład z Wikieł P 4.3a moja wersja nr 233}


Obliczyć granicę funkcji $\lim\limits_{x\to\ 0}\frac{13 \cdot x}{tan(32 \cdot x)}$.
\zadStop
\rozwStart{Patryk Wirkus}{}
$$\lim\limits_{x\to\ 0}\frac{13 \cdot x}{tan(32 \cdot x)}=\lim\limits_{x\to\ 0}\frac{13 \cdot x \cdot cos(32 \cdot x)}{sin(32 \cdot x)}=\lim\limits_{x\to\ 0}\frac{13 \cdot cos(32 \cdot x)}{\frac{sin(32 \cdot x)}{x}}=\lim\limits_{x\to\ 0}\frac{13 \cdot cos(32 \cdot x)}{32 \cdot \frac{sin(32 \cdot x)}{32 \cdot x}} = \frac{13}{32}$$
\rozwStop
\odpStart
$\frac{13}{32}$
\odpStop
\testStart
A.$\frac{13}{32}$
B.$\infty$
C.$-\infty$
D.$0$
E.$-\frac{13}{32}$
F.$\frac{32}{13}$
G.$-\frac{32}{13}$
H.$32$
I.$13$
\testStop
\kluczStart
A
\kluczStop



\zadStart{Przykład z Wikieł P 4.3a moja wersja nr 234}


Obliczyć granicę funkcji $\lim\limits_{x\to\ 0}\frac{13 \cdot x}{tan(33 \cdot x)}$.
\zadStop
\rozwStart{Patryk Wirkus}{}
$$\lim\limits_{x\to\ 0}\frac{13 \cdot x}{tan(33 \cdot x)}=\lim\limits_{x\to\ 0}\frac{13 \cdot x \cdot cos(33 \cdot x)}{sin(33 \cdot x)}=\lim\limits_{x\to\ 0}\frac{13 \cdot cos(33 \cdot x)}{\frac{sin(33 \cdot x)}{x}}=\lim\limits_{x\to\ 0}\frac{13 \cdot cos(33 \cdot x)}{33 \cdot \frac{sin(33 \cdot x)}{33 \cdot x}} = \frac{13}{33}$$
\rozwStop
\odpStart
$\frac{13}{33}$
\odpStop
\testStart
A.$\frac{13}{33}$
B.$\infty$
C.$-\infty$
D.$0$
E.$-\frac{13}{33}$
F.$\frac{33}{13}$
G.$-\frac{33}{13}$
H.$33$
I.$13$
\testStop
\kluczStart
A
\kluczStop



\zadStart{Przykład z Wikieł P 4.3a moja wersja nr 235}


Obliczyć granicę funkcji $\lim\limits_{x\to\ 0}\frac{13 \cdot x}{tan(34 \cdot x)}$.
\zadStop
\rozwStart{Patryk Wirkus}{}
$$\lim\limits_{x\to\ 0}\frac{13 \cdot x}{tan(34 \cdot x)}=\lim\limits_{x\to\ 0}\frac{13 \cdot x \cdot cos(34 \cdot x)}{sin(34 \cdot x)}=\lim\limits_{x\to\ 0}\frac{13 \cdot cos(34 \cdot x)}{\frac{sin(34 \cdot x)}{x}}=\lim\limits_{x\to\ 0}\frac{13 \cdot cos(34 \cdot x)}{34 \cdot \frac{sin(34 \cdot x)}{34 \cdot x}} = \frac{13}{34}$$
\rozwStop
\odpStart
$\frac{13}{34}$
\odpStop
\testStart
A.$\frac{13}{34}$
B.$\infty$
C.$-\infty$
D.$0$
E.$-\frac{13}{34}$
F.$\frac{34}{13}$
G.$-\frac{34}{13}$
H.$34$
I.$13$
\testStop
\kluczStart
A
\kluczStop



\zadStart{Przykład z Wikieł P 4.3a moja wersja nr 236}


Obliczyć granicę funkcji $\lim\limits_{x\to\ 0}\frac{13 \cdot x}{tan(35 \cdot x)}$.
\zadStop
\rozwStart{Patryk Wirkus}{}
$$\lim\limits_{x\to\ 0}\frac{13 \cdot x}{tan(35 \cdot x)}=\lim\limits_{x\to\ 0}\frac{13 \cdot x \cdot cos(35 \cdot x)}{sin(35 \cdot x)}=\lim\limits_{x\to\ 0}\frac{13 \cdot cos(35 \cdot x)}{\frac{sin(35 \cdot x)}{x}}=\lim\limits_{x\to\ 0}\frac{13 \cdot cos(35 \cdot x)}{35 \cdot \frac{sin(35 \cdot x)}{35 \cdot x}} = \frac{13}{35}$$
\rozwStop
\odpStart
$\frac{13}{35}$
\odpStop
\testStart
A.$\frac{13}{35}$
B.$\infty$
C.$-\infty$
D.$0$
E.$-\frac{13}{35}$
F.$\frac{35}{13}$
G.$-\frac{35}{13}$
H.$35$
I.$13$
\testStop
\kluczStart
A
\kluczStop



\zadStart{Przykład z Wikieł P 4.3a moja wersja nr 237}


Obliczyć granicę funkcji $\lim\limits_{x\to\ 0}\frac{13 \cdot x}{tan(36 \cdot x)}$.
\zadStop
\rozwStart{Patryk Wirkus}{}
$$\lim\limits_{x\to\ 0}\frac{13 \cdot x}{tan(36 \cdot x)}=\lim\limits_{x\to\ 0}\frac{13 \cdot x \cdot cos(36 \cdot x)}{sin(36 \cdot x)}=\lim\limits_{x\to\ 0}\frac{13 \cdot cos(36 \cdot x)}{\frac{sin(36 \cdot x)}{x}}=\lim\limits_{x\to\ 0}\frac{13 \cdot cos(36 \cdot x)}{36 \cdot \frac{sin(36 \cdot x)}{36 \cdot x}} = \frac{13}{36}$$
\rozwStop
\odpStart
$\frac{13}{36}$
\odpStop
\testStart
A.$\frac{13}{36}$
B.$\infty$
C.$-\infty$
D.$0$
E.$-\frac{13}{36}$
F.$\frac{36}{13}$
G.$-\frac{36}{13}$
H.$36$
I.$13$
\testStop
\kluczStart
A
\kluczStop



\zadStart{Przykład z Wikieł P 4.3a moja wersja nr 238}


Obliczyć granicę funkcji $\lim\limits_{x\to\ 0}\frac{13 \cdot x}{tan(37 \cdot x)}$.
\zadStop
\rozwStart{Patryk Wirkus}{}
$$\lim\limits_{x\to\ 0}\frac{13 \cdot x}{tan(37 \cdot x)}=\lim\limits_{x\to\ 0}\frac{13 \cdot x \cdot cos(37 \cdot x)}{sin(37 \cdot x)}=\lim\limits_{x\to\ 0}\frac{13 \cdot cos(37 \cdot x)}{\frac{sin(37 \cdot x)}{x}}=\lim\limits_{x\to\ 0}\frac{13 \cdot cos(37 \cdot x)}{37 \cdot \frac{sin(37 \cdot x)}{37 \cdot x}} = \frac{13}{37}$$
\rozwStop
\odpStart
$\frac{13}{37}$
\odpStop
\testStart
A.$\frac{13}{37}$
B.$\infty$
C.$-\infty$
D.$0$
E.$-\frac{13}{37}$
F.$\frac{37}{13}$
G.$-\frac{37}{13}$
H.$37$
I.$13$
\testStop
\kluczStart
A
\kluczStop



\zadStart{Przykład z Wikieł P 4.3a moja wersja nr 239}


Obliczyć granicę funkcji $\lim\limits_{x\to\ 0}\frac{13 \cdot x}{tan(38 \cdot x)}$.
\zadStop
\rozwStart{Patryk Wirkus}{}
$$\lim\limits_{x\to\ 0}\frac{13 \cdot x}{tan(38 \cdot x)}=\lim\limits_{x\to\ 0}\frac{13 \cdot x \cdot cos(38 \cdot x)}{sin(38 \cdot x)}=\lim\limits_{x\to\ 0}\frac{13 \cdot cos(38 \cdot x)}{\frac{sin(38 \cdot x)}{x}}=\lim\limits_{x\to\ 0}\frac{13 \cdot cos(38 \cdot x)}{38 \cdot \frac{sin(38 \cdot x)}{38 \cdot x}} = \frac{13}{38}$$
\rozwStop
\odpStart
$\frac{13}{38}$
\odpStop
\testStart
A.$\frac{13}{38}$
B.$\infty$
C.$-\infty$
D.$0$
E.$-\frac{13}{38}$
F.$\frac{38}{13}$
G.$-\frac{38}{13}$
H.$38$
I.$13$
\testStop
\kluczStart
A
\kluczStop



\zadStart{Przykład z Wikieł P 4.3a moja wersja nr 240}


Obliczyć granicę funkcji $\lim\limits_{x\to\ 0}\frac{13 \cdot x}{tan(40 \cdot x)}$.
\zadStop
\rozwStart{Patryk Wirkus}{}
$$\lim\limits_{x\to\ 0}\frac{13 \cdot x}{tan(40 \cdot x)}=\lim\limits_{x\to\ 0}\frac{13 \cdot x \cdot cos(40 \cdot x)}{sin(40 \cdot x)}=\lim\limits_{x\to\ 0}\frac{13 \cdot cos(40 \cdot x)}{\frac{sin(40 \cdot x)}{x}}=\lim\limits_{x\to\ 0}\frac{13 \cdot cos(40 \cdot x)}{40 \cdot \frac{sin(40 \cdot x)}{40 \cdot x}} = \frac{13}{40}$$
\rozwStop
\odpStart
$\frac{13}{40}$
\odpStop
\testStart
A.$\frac{13}{40}$
B.$\infty$
C.$-\infty$
D.$0$
E.$-\frac{13}{40}$
F.$\frac{40}{13}$
G.$-\frac{40}{13}$
H.$40$
I.$13$
\testStop
\kluczStart
A
\kluczStop



\zadStart{Przykład z Wikieł P 4.3a moja wersja nr 241}


Obliczyć granicę funkcji $\lim\limits_{x\to\ 0}\frac{14 \cdot x}{tan(3 \cdot x)}$.
\zadStop
\rozwStart{Patryk Wirkus}{}
$$\lim\limits_{x\to\ 0}\frac{14 \cdot x}{tan(3 \cdot x)}=\lim\limits_{x\to\ 0}\frac{14 \cdot x \cdot cos(3 \cdot x)}{sin(3 \cdot x)}=\lim\limits_{x\to\ 0}\frac{14 \cdot cos(3 \cdot x)}{\frac{sin(3 \cdot x)}{x}}=\lim\limits_{x\to\ 0}\frac{14 \cdot cos(3 \cdot x)}{3 \cdot \frac{sin(3 \cdot x)}{3 \cdot x}} = \frac{14}{3}$$
\rozwStop
\odpStart
$\frac{14}{3}$
\odpStop
\testStart
A.$\frac{14}{3}$
B.$\infty$
C.$-\infty$
D.$0$
E.$-\frac{14}{3}$
F.$\frac{3}{14}$
G.$-\frac{3}{14}$
H.$3$
I.$14$
\testStop
\kluczStart
A
\kluczStop



\zadStart{Przykład z Wikieł P 4.3a moja wersja nr 242}


Obliczyć granicę funkcji $\lim\limits_{x\to\ 0}\frac{14 \cdot x}{tan(5 \cdot x)}$.
\zadStop
\rozwStart{Patryk Wirkus}{}
$$\lim\limits_{x\to\ 0}\frac{14 \cdot x}{tan(5 \cdot x)}=\lim\limits_{x\to\ 0}\frac{14 \cdot x \cdot cos(5 \cdot x)}{sin(5 \cdot x)}=\lim\limits_{x\to\ 0}\frac{14 \cdot cos(5 \cdot x)}{\frac{sin(5 \cdot x)}{x}}=\lim\limits_{x\to\ 0}\frac{14 \cdot cos(5 \cdot x)}{5 \cdot \frac{sin(5 \cdot x)}{5 \cdot x}} = \frac{14}{5}$$
\rozwStop
\odpStart
$\frac{14}{5}$
\odpStop
\testStart
A.$\frac{14}{5}$
B.$\infty$
C.$-\infty$
D.$0$
E.$-\frac{14}{5}$
F.$\frac{5}{14}$
G.$-\frac{5}{14}$
H.$5$
I.$14$
\testStop
\kluczStart
A
\kluczStop



\zadStart{Przykład z Wikieł P 4.3a moja wersja nr 243}


Obliczyć granicę funkcji $\lim\limits_{x\to\ 0}\frac{14 \cdot x}{tan(9 \cdot x)}$.
\zadStop
\rozwStart{Patryk Wirkus}{}
$$\lim\limits_{x\to\ 0}\frac{14 \cdot x}{tan(9 \cdot x)}=\lim\limits_{x\to\ 0}\frac{14 \cdot x \cdot cos(9 \cdot x)}{sin(9 \cdot x)}=\lim\limits_{x\to\ 0}\frac{14 \cdot cos(9 \cdot x)}{\frac{sin(9 \cdot x)}{x}}=\lim\limits_{x\to\ 0}\frac{14 \cdot cos(9 \cdot x)}{9 \cdot \frac{sin(9 \cdot x)}{9 \cdot x}} = \frac{14}{9}$$
\rozwStop
\odpStart
$\frac{14}{9}$
\odpStop
\testStart
A.$\frac{14}{9}$
B.$\infty$
C.$-\infty$
D.$0$
E.$-\frac{14}{9}$
F.$\frac{9}{14}$
G.$-\frac{9}{14}$
H.$9$
I.$14$
\testStop
\kluczStart
A
\kluczStop



\zadStart{Przykład z Wikieł P 4.3a moja wersja nr 244}


Obliczyć granicę funkcji $\lim\limits_{x\to\ 0}\frac{14 \cdot x}{tan(11 \cdot x)}$.
\zadStop
\rozwStart{Patryk Wirkus}{}
$$\lim\limits_{x\to\ 0}\frac{14 \cdot x}{tan(11 \cdot x)}=\lim\limits_{x\to\ 0}\frac{14 \cdot x \cdot cos(11 \cdot x)}{sin(11 \cdot x)}=\lim\limits_{x\to\ 0}\frac{14 \cdot cos(11 \cdot x)}{\frac{sin(11 \cdot x)}{x}}=\lim\limits_{x\to\ 0}\frac{14 \cdot cos(11 \cdot x)}{11 \cdot \frac{sin(11 \cdot x)}{11 \cdot x}} = \frac{14}{11}$$
\rozwStop
\odpStart
$\frac{14}{11}$
\odpStop
\testStart
A.$\frac{14}{11}$
B.$\infty$
C.$-\infty$
D.$0$
E.$-\frac{14}{11}$
F.$\frac{11}{14}$
G.$-\frac{11}{14}$
H.$11$
I.$14$
\testStop
\kluczStart
A
\kluczStop



\zadStart{Przykład z Wikieł P 4.3a moja wersja nr 245}


Obliczyć granicę funkcji $\lim\limits_{x\to\ 0}\frac{14 \cdot x}{tan(13 \cdot x)}$.
\zadStop
\rozwStart{Patryk Wirkus}{}
$$\lim\limits_{x\to\ 0}\frac{14 \cdot x}{tan(13 \cdot x)}=\lim\limits_{x\to\ 0}\frac{14 \cdot x \cdot cos(13 \cdot x)}{sin(13 \cdot x)}=\lim\limits_{x\to\ 0}\frac{14 \cdot cos(13 \cdot x)}{\frac{sin(13 \cdot x)}{x}}=\lim\limits_{x\to\ 0}\frac{14 \cdot cos(13 \cdot x)}{13 \cdot \frac{sin(13 \cdot x)}{13 \cdot x}} = \frac{14}{13}$$
\rozwStop
\odpStart
$\frac{14}{13}$
\odpStop
\testStart
A.$\frac{14}{13}$
B.$\infty$
C.$-\infty$
D.$0$
E.$-\frac{14}{13}$
F.$\frac{13}{14}$
G.$-\frac{13}{14}$
H.$13$
I.$14$
\testStop
\kluczStart
A
\kluczStop



\zadStart{Przykład z Wikieł P 4.3a moja wersja nr 246}


Obliczyć granicę funkcji $\lim\limits_{x\to\ 0}\frac{14 \cdot x}{tan(15 \cdot x)}$.
\zadStop
\rozwStart{Patryk Wirkus}{}
$$\lim\limits_{x\to\ 0}\frac{14 \cdot x}{tan(15 \cdot x)}=\lim\limits_{x\to\ 0}\frac{14 \cdot x \cdot cos(15 \cdot x)}{sin(15 \cdot x)}=\lim\limits_{x\to\ 0}\frac{14 \cdot cos(15 \cdot x)}{\frac{sin(15 \cdot x)}{x}}=\lim\limits_{x\to\ 0}\frac{14 \cdot cos(15 \cdot x)}{15 \cdot \frac{sin(15 \cdot x)}{15 \cdot x}} = \frac{14}{15}$$
\rozwStop
\odpStart
$\frac{14}{15}$
\odpStop
\testStart
A.$\frac{14}{15}$
B.$\infty$
C.$-\infty$
D.$0$
E.$-\frac{14}{15}$
F.$\frac{15}{14}$
G.$-\frac{15}{14}$
H.$15$
I.$14$
\testStop
\kluczStart
A
\kluczStop



\zadStart{Przykład z Wikieł P 4.3a moja wersja nr 247}


Obliczyć granicę funkcji $\lim\limits_{x\to\ 0}\frac{14 \cdot x}{tan(17 \cdot x)}$.
\zadStop
\rozwStart{Patryk Wirkus}{}
$$\lim\limits_{x\to\ 0}\frac{14 \cdot x}{tan(17 \cdot x)}=\lim\limits_{x\to\ 0}\frac{14 \cdot x \cdot cos(17 \cdot x)}{sin(17 \cdot x)}=\lim\limits_{x\to\ 0}\frac{14 \cdot cos(17 \cdot x)}{\frac{sin(17 \cdot x)}{x}}=\lim\limits_{x\to\ 0}\frac{14 \cdot cos(17 \cdot x)}{17 \cdot \frac{sin(17 \cdot x)}{17 \cdot x}} = \frac{14}{17}$$
\rozwStop
\odpStart
$\frac{14}{17}$
\odpStop
\testStart
A.$\frac{14}{17}$
B.$\infty$
C.$-\infty$
D.$0$
E.$-\frac{14}{17}$
F.$\frac{17}{14}$
G.$-\frac{17}{14}$
H.$17$
I.$14$
\testStop
\kluczStart
A
\kluczStop



\zadStart{Przykład z Wikieł P 4.3a moja wersja nr 248}


Obliczyć granicę funkcji $\lim\limits_{x\to\ 0}\frac{14 \cdot x}{tan(19 \cdot x)}$.
\zadStop
\rozwStart{Patryk Wirkus}{}
$$\lim\limits_{x\to\ 0}\frac{14 \cdot x}{tan(19 \cdot x)}=\lim\limits_{x\to\ 0}\frac{14 \cdot x \cdot cos(19 \cdot x)}{sin(19 \cdot x)}=\lim\limits_{x\to\ 0}\frac{14 \cdot cos(19 \cdot x)}{\frac{sin(19 \cdot x)}{x}}=\lim\limits_{x\to\ 0}\frac{14 \cdot cos(19 \cdot x)}{19 \cdot \frac{sin(19 \cdot x)}{19 \cdot x}} = \frac{14}{19}$$
\rozwStop
\odpStart
$\frac{14}{19}$
\odpStop
\testStart
A.$\frac{14}{19}$
B.$\infty$
C.$-\infty$
D.$0$
E.$-\frac{14}{19}$
F.$\frac{19}{14}$
G.$-\frac{19}{14}$
H.$19$
I.$14$
\testStop
\kluczStart
A
\kluczStop



\zadStart{Przykład z Wikieł P 4.3a moja wersja nr 249}


Obliczyć granicę funkcji $\lim\limits_{x\to\ 0}\frac{14 \cdot x}{tan(23 \cdot x)}$.
\zadStop
\rozwStart{Patryk Wirkus}{}
$$\lim\limits_{x\to\ 0}\frac{14 \cdot x}{tan(23 \cdot x)}=\lim\limits_{x\to\ 0}\frac{14 \cdot x \cdot cos(23 \cdot x)}{sin(23 \cdot x)}=\lim\limits_{x\to\ 0}\frac{14 \cdot cos(23 \cdot x)}{\frac{sin(23 \cdot x)}{x}}=\lim\limits_{x\to\ 0}\frac{14 \cdot cos(23 \cdot x)}{23 \cdot \frac{sin(23 \cdot x)}{23 \cdot x}} = \frac{14}{23}$$
\rozwStop
\odpStart
$\frac{14}{23}$
\odpStop
\testStart
A.$\frac{14}{23}$
B.$\infty$
C.$-\infty$
D.$0$
E.$-\frac{14}{23}$
F.$\frac{23}{14}$
G.$-\frac{23}{14}$
H.$23$
I.$14$
\testStop
\kluczStart
A
\kluczStop



\zadStart{Przykład z Wikieł P 4.3a moja wersja nr 250}


Obliczyć granicę funkcji $\lim\limits_{x\to\ 0}\frac{14 \cdot x}{tan(25 \cdot x)}$.
\zadStop
\rozwStart{Patryk Wirkus}{}
$$\lim\limits_{x\to\ 0}\frac{14 \cdot x}{tan(25 \cdot x)}=\lim\limits_{x\to\ 0}\frac{14 \cdot x \cdot cos(25 \cdot x)}{sin(25 \cdot x)}=\lim\limits_{x\to\ 0}\frac{14 \cdot cos(25 \cdot x)}{\frac{sin(25 \cdot x)}{x}}=\lim\limits_{x\to\ 0}\frac{14 \cdot cos(25 \cdot x)}{25 \cdot \frac{sin(25 \cdot x)}{25 \cdot x}} = \frac{14}{25}$$
\rozwStop
\odpStart
$\frac{14}{25}$
\odpStop
\testStart
A.$\frac{14}{25}$
B.$\infty$
C.$-\infty$
D.$0$
E.$-\frac{14}{25}$
F.$\frac{25}{14}$
G.$-\frac{25}{14}$
H.$25$
I.$14$
\testStop
\kluczStart
A
\kluczStop



\zadStart{Przykład z Wikieł P 4.3a moja wersja nr 251}


Obliczyć granicę funkcji $\lim\limits_{x\to\ 0}\frac{14 \cdot x}{tan(27 \cdot x)}$.
\zadStop
\rozwStart{Patryk Wirkus}{}
$$\lim\limits_{x\to\ 0}\frac{14 \cdot x}{tan(27 \cdot x)}=\lim\limits_{x\to\ 0}\frac{14 \cdot x \cdot cos(27 \cdot x)}{sin(27 \cdot x)}=\lim\limits_{x\to\ 0}\frac{14 \cdot cos(27 \cdot x)}{\frac{sin(27 \cdot x)}{x}}=\lim\limits_{x\to\ 0}\frac{14 \cdot cos(27 \cdot x)}{27 \cdot \frac{sin(27 \cdot x)}{27 \cdot x}} = \frac{14}{27}$$
\rozwStop
\odpStart
$\frac{14}{27}$
\odpStop
\testStart
A.$\frac{14}{27}$
B.$\infty$
C.$-\infty$
D.$0$
E.$-\frac{14}{27}$
F.$\frac{27}{14}$
G.$-\frac{27}{14}$
H.$27$
I.$14$
\testStop
\kluczStart
A
\kluczStop



\zadStart{Przykład z Wikieł P 4.3a moja wersja nr 252}


Obliczyć granicę funkcji $\lim\limits_{x\to\ 0}\frac{14 \cdot x}{tan(29 \cdot x)}$.
\zadStop
\rozwStart{Patryk Wirkus}{}
$$\lim\limits_{x\to\ 0}\frac{14 \cdot x}{tan(29 \cdot x)}=\lim\limits_{x\to\ 0}\frac{14 \cdot x \cdot cos(29 \cdot x)}{sin(29 \cdot x)}=\lim\limits_{x\to\ 0}\frac{14 \cdot cos(29 \cdot x)}{\frac{sin(29 \cdot x)}{x}}=\lim\limits_{x\to\ 0}\frac{14 \cdot cos(29 \cdot x)}{29 \cdot \frac{sin(29 \cdot x)}{29 \cdot x}} = \frac{14}{29}$$
\rozwStop
\odpStart
$\frac{14}{29}$
\odpStop
\testStart
A.$\frac{14}{29}$
B.$\infty$
C.$-\infty$
D.$0$
E.$-\frac{14}{29}$
F.$\frac{29}{14}$
G.$-\frac{29}{14}$
H.$29$
I.$14$
\testStop
\kluczStart
A
\kluczStop



\zadStart{Przykład z Wikieł P 4.3a moja wersja nr 253}


Obliczyć granicę funkcji $\lim\limits_{x\to\ 0}\frac{14 \cdot x}{tan(31 \cdot x)}$.
\zadStop
\rozwStart{Patryk Wirkus}{}
$$\lim\limits_{x\to\ 0}\frac{14 \cdot x}{tan(31 \cdot x)}=\lim\limits_{x\to\ 0}\frac{14 \cdot x \cdot cos(31 \cdot x)}{sin(31 \cdot x)}=\lim\limits_{x\to\ 0}\frac{14 \cdot cos(31 \cdot x)}{\frac{sin(31 \cdot x)}{x}}=\lim\limits_{x\to\ 0}\frac{14 \cdot cos(31 \cdot x)}{31 \cdot \frac{sin(31 \cdot x)}{31 \cdot x}} = \frac{14}{31}$$
\rozwStop
\odpStart
$\frac{14}{31}$
\odpStop
\testStart
A.$\frac{14}{31}$
B.$\infty$
C.$-\infty$
D.$0$
E.$-\frac{14}{31}$
F.$\frac{31}{14}$
G.$-\frac{31}{14}$
H.$31$
I.$14$
\testStop
\kluczStart
A
\kluczStop



\zadStart{Przykład z Wikieł P 4.3a moja wersja nr 254}


Obliczyć granicę funkcji $\lim\limits_{x\to\ 0}\frac{14 \cdot x}{tan(33 \cdot x)}$.
\zadStop
\rozwStart{Patryk Wirkus}{}
$$\lim\limits_{x\to\ 0}\frac{14 \cdot x}{tan(33 \cdot x)}=\lim\limits_{x\to\ 0}\frac{14 \cdot x \cdot cos(33 \cdot x)}{sin(33 \cdot x)}=\lim\limits_{x\to\ 0}\frac{14 \cdot cos(33 \cdot x)}{\frac{sin(33 \cdot x)}{x}}=\lim\limits_{x\to\ 0}\frac{14 \cdot cos(33 \cdot x)}{33 \cdot \frac{sin(33 \cdot x)}{33 \cdot x}} = \frac{14}{33}$$
\rozwStop
\odpStart
$\frac{14}{33}$
\odpStop
\testStart
A.$\frac{14}{33}$
B.$\infty$
C.$-\infty$
D.$0$
E.$-\frac{14}{33}$
F.$\frac{33}{14}$
G.$-\frac{33}{14}$
H.$33$
I.$14$
\testStop
\kluczStart
A
\kluczStop



\zadStart{Przykład z Wikieł P 4.3a moja wersja nr 255}


Obliczyć granicę funkcji $\lim\limits_{x\to\ 0}\frac{14 \cdot x}{tan(37 \cdot x)}$.
\zadStop
\rozwStart{Patryk Wirkus}{}
$$\lim\limits_{x\to\ 0}\frac{14 \cdot x}{tan(37 \cdot x)}=\lim\limits_{x\to\ 0}\frac{14 \cdot x \cdot cos(37 \cdot x)}{sin(37 \cdot x)}=\lim\limits_{x\to\ 0}\frac{14 \cdot cos(37 \cdot x)}{\frac{sin(37 \cdot x)}{x}}=\lim\limits_{x\to\ 0}\frac{14 \cdot cos(37 \cdot x)}{37 \cdot \frac{sin(37 \cdot x)}{37 \cdot x}} = \frac{14}{37}$$
\rozwStop
\odpStart
$\frac{14}{37}$
\odpStop
\testStart
A.$\frac{14}{37}$
B.$\infty$
C.$-\infty$
D.$0$
E.$-\frac{14}{37}$
F.$\frac{37}{14}$
G.$-\frac{37}{14}$
H.$37$
I.$14$
\testStop
\kluczStart
A
\kluczStop



\zadStart{Przykład z Wikieł P 4.3a moja wersja nr 256}


Obliczyć granicę funkcji $\lim\limits_{x\to\ 0}\frac{14 \cdot x}{tan(39 \cdot x)}$.
\zadStop
\rozwStart{Patryk Wirkus}{}
$$\lim\limits_{x\to\ 0}\frac{14 \cdot x}{tan(39 \cdot x)}=\lim\limits_{x\to\ 0}\frac{14 \cdot x \cdot cos(39 \cdot x)}{sin(39 \cdot x)}=\lim\limits_{x\to\ 0}\frac{14 \cdot cos(39 \cdot x)}{\frac{sin(39 \cdot x)}{x}}=\lim\limits_{x\to\ 0}\frac{14 \cdot cos(39 \cdot x)}{39 \cdot \frac{sin(39 \cdot x)}{39 \cdot x}} = \frac{14}{39}$$
\rozwStop
\odpStart
$\frac{14}{39}$
\odpStop
\testStart
A.$\frac{14}{39}$
B.$\infty$
C.$-\infty$
D.$0$
E.$-\frac{14}{39}$
F.$\frac{39}{14}$
G.$-\frac{39}{14}$
H.$39$
I.$14$
\testStop
\kluczStart
A
\kluczStop



\zadStart{Przykład z Wikieł P 4.3a moja wersja nr 257}


Obliczyć granicę funkcji $\lim\limits_{x\to\ 0}\frac{15 \cdot x}{tan(2 \cdot x)}$.
\zadStop
\rozwStart{Patryk Wirkus}{}
$$\lim\limits_{x\to\ 0}\frac{15 \cdot x}{tan(2 \cdot x)}=\lim\limits_{x\to\ 0}\frac{15 \cdot x \cdot cos(2 \cdot x)}{sin(2 \cdot x)}=\lim\limits_{x\to\ 0}\frac{15 \cdot cos(2 \cdot x)}{\frac{sin(2 \cdot x)}{x}}=\lim\limits_{x\to\ 0}\frac{15 \cdot cos(2 \cdot x)}{2 \cdot \frac{sin(2 \cdot x)}{2 \cdot x}} = \frac{15}{2}$$
\rozwStop
\odpStart
$\frac{15}{2}$
\odpStop
\testStart
A.$\frac{15}{2}$
B.$\infty$
C.$-\infty$
D.$0$
E.$-\frac{15}{2}$
F.$\frac{2}{15}$
G.$-\frac{2}{15}$
H.$2$
I.$15$
\testStop
\kluczStart
A
\kluczStop



\zadStart{Przykład z Wikieł P 4.3a moja wersja nr 258}


Obliczyć granicę funkcji $\lim\limits_{x\to\ 0}\frac{15 \cdot x}{tan(4 \cdot x)}$.
\zadStop
\rozwStart{Patryk Wirkus}{}
$$\lim\limits_{x\to\ 0}\frac{15 \cdot x}{tan(4 \cdot x)}=\lim\limits_{x\to\ 0}\frac{15 \cdot x \cdot cos(4 \cdot x)}{sin(4 \cdot x)}=\lim\limits_{x\to\ 0}\frac{15 \cdot cos(4 \cdot x)}{\frac{sin(4 \cdot x)}{x}}=\lim\limits_{x\to\ 0}\frac{15 \cdot cos(4 \cdot x)}{4 \cdot \frac{sin(4 \cdot x)}{4 \cdot x}} = \frac{15}{4}$$
\rozwStop
\odpStart
$\frac{15}{4}$
\odpStop
\testStart
A.$\frac{15}{4}$
B.$\infty$
C.$-\infty$
D.$0$
E.$-\frac{15}{4}$
F.$\frac{4}{15}$
G.$-\frac{4}{15}$
H.$4$
I.$15$
\testStop
\kluczStart
A
\kluczStop



\zadStart{Przykład z Wikieł P 4.3a moja wersja nr 259}


Obliczyć granicę funkcji $\lim\limits_{x\to\ 0}\frac{15 \cdot x}{tan(7 \cdot x)}$.
\zadStop
\rozwStart{Patryk Wirkus}{}
$$\lim\limits_{x\to\ 0}\frac{15 \cdot x}{tan(7 \cdot x)}=\lim\limits_{x\to\ 0}\frac{15 \cdot x \cdot cos(7 \cdot x)}{sin(7 \cdot x)}=\lim\limits_{x\to\ 0}\frac{15 \cdot cos(7 \cdot x)}{\frac{sin(7 \cdot x)}{x}}=\lim\limits_{x\to\ 0}\frac{15 \cdot cos(7 \cdot x)}{7 \cdot \frac{sin(7 \cdot x)}{7 \cdot x}} = \frac{15}{7}$$
\rozwStop
\odpStart
$\frac{15}{7}$
\odpStop
\testStart
A.$\frac{15}{7}$
B.$\infty$
C.$-\infty$
D.$0$
E.$-\frac{15}{7}$
F.$\frac{7}{15}$
G.$-\frac{7}{15}$
H.$7$
I.$15$
\testStop
\kluczStart
A
\kluczStop



\zadStart{Przykład z Wikieł P 4.3a moja wersja nr 260}


Obliczyć granicę funkcji $\lim\limits_{x\to\ 0}\frac{15 \cdot x}{tan(8 \cdot x)}$.
\zadStop
\rozwStart{Patryk Wirkus}{}
$$\lim\limits_{x\to\ 0}\frac{15 \cdot x}{tan(8 \cdot x)}=\lim\limits_{x\to\ 0}\frac{15 \cdot x \cdot cos(8 \cdot x)}{sin(8 \cdot x)}=\lim\limits_{x\to\ 0}\frac{15 \cdot cos(8 \cdot x)}{\frac{sin(8 \cdot x)}{x}}=\lim\limits_{x\to\ 0}\frac{15 \cdot cos(8 \cdot x)}{8 \cdot \frac{sin(8 \cdot x)}{8 \cdot x}} = \frac{15}{8}$$
\rozwStop
\odpStart
$\frac{15}{8}$
\odpStop
\testStart
A.$\frac{15}{8}$
B.$\infty$
C.$-\infty$
D.$0$
E.$-\frac{15}{8}$
F.$\frac{8}{15}$
G.$-\frac{8}{15}$
H.$8$
I.$15$
\testStop
\kluczStart
A
\kluczStop



\zadStart{Przykład z Wikieł P 4.3a moja wersja nr 261}


Obliczyć granicę funkcji $\lim\limits_{x\to\ 0}\frac{15 \cdot x}{tan(11 \cdot x)}$.
\zadStop
\rozwStart{Patryk Wirkus}{}
$$\lim\limits_{x\to\ 0}\frac{15 \cdot x}{tan(11 \cdot x)}=\lim\limits_{x\to\ 0}\frac{15 \cdot x \cdot cos(11 \cdot x)}{sin(11 \cdot x)}=\lim\limits_{x\to\ 0}\frac{15 \cdot cos(11 \cdot x)}{\frac{sin(11 \cdot x)}{x}}=\lim\limits_{x\to\ 0}\frac{15 \cdot cos(11 \cdot x)}{11 \cdot \frac{sin(11 \cdot x)}{11 \cdot x}} = \frac{15}{11}$$
\rozwStop
\odpStart
$\frac{15}{11}$
\odpStop
\testStart
A.$\frac{15}{11}$
B.$\infty$
C.$-\infty$
D.$0$
E.$-\frac{15}{11}$
F.$\frac{11}{15}$
G.$-\frac{11}{15}$
H.$11$
I.$15$
\testStop
\kluczStart
A
\kluczStop



\zadStart{Przykład z Wikieł P 4.3a moja wersja nr 262}


Obliczyć granicę funkcji $\lim\limits_{x\to\ 0}\frac{15 \cdot x}{tan(13 \cdot x)}$.
\zadStop
\rozwStart{Patryk Wirkus}{}
$$\lim\limits_{x\to\ 0}\frac{15 \cdot x}{tan(13 \cdot x)}=\lim\limits_{x\to\ 0}\frac{15 \cdot x \cdot cos(13 \cdot x)}{sin(13 \cdot x)}=\lim\limits_{x\to\ 0}\frac{15 \cdot cos(13 \cdot x)}{\frac{sin(13 \cdot x)}{x}}=\lim\limits_{x\to\ 0}\frac{15 \cdot cos(13 \cdot x)}{13 \cdot \frac{sin(13 \cdot x)}{13 \cdot x}} = \frac{15}{13}$$
\rozwStop
\odpStart
$\frac{15}{13}$
\odpStop
\testStart
A.$\frac{15}{13}$
B.$\infty$
C.$-\infty$
D.$0$
E.$-\frac{15}{13}$
F.$\frac{13}{15}$
G.$-\frac{13}{15}$
H.$13$
I.$15$
\testStop
\kluczStart
A
\kluczStop



\zadStart{Przykład z Wikieł P 4.3a moja wersja nr 263}


Obliczyć granicę funkcji $\lim\limits_{x\to\ 0}\frac{15 \cdot x}{tan(14 \cdot x)}$.
\zadStop
\rozwStart{Patryk Wirkus}{}
$$\lim\limits_{x\to\ 0}\frac{15 \cdot x}{tan(14 \cdot x)}=\lim\limits_{x\to\ 0}\frac{15 \cdot x \cdot cos(14 \cdot x)}{sin(14 \cdot x)}=\lim\limits_{x\to\ 0}\frac{15 \cdot cos(14 \cdot x)}{\frac{sin(14 \cdot x)}{x}}=\lim\limits_{x\to\ 0}\frac{15 \cdot cos(14 \cdot x)}{14 \cdot \frac{sin(14 \cdot x)}{14 \cdot x}} = \frac{15}{14}$$
\rozwStop
\odpStart
$\frac{15}{14}$
\odpStop
\testStart
A.$\frac{15}{14}$
B.$\infty$
C.$-\infty$
D.$0$
E.$-\frac{15}{14}$
F.$\frac{14}{15}$
G.$-\frac{14}{15}$
H.$14$
I.$15$
\testStop
\kluczStart
A
\kluczStop



\zadStart{Przykład z Wikieł P 4.3a moja wersja nr 264}


Obliczyć granicę funkcji $\lim\limits_{x\to\ 0}\frac{15 \cdot x}{tan(16 \cdot x)}$.
\zadStop
\rozwStart{Patryk Wirkus}{}
$$\lim\limits_{x\to\ 0}\frac{15 \cdot x}{tan(16 \cdot x)}=\lim\limits_{x\to\ 0}\frac{15 \cdot x \cdot cos(16 \cdot x)}{sin(16 \cdot x)}=\lim\limits_{x\to\ 0}\frac{15 \cdot cos(16 \cdot x)}{\frac{sin(16 \cdot x)}{x}}=\lim\limits_{x\to\ 0}\frac{15 \cdot cos(16 \cdot x)}{16 \cdot \frac{sin(16 \cdot x)}{16 \cdot x}} = \frac{15}{16}$$
\rozwStop
\odpStart
$\frac{15}{16}$
\odpStop
\testStart
A.$\frac{15}{16}$
B.$\infty$
C.$-\infty$
D.$0$
E.$-\frac{15}{16}$
F.$\frac{16}{15}$
G.$-\frac{16}{15}$
H.$16$
I.$15$
\testStop
\kluczStart
A
\kluczStop



\zadStart{Przykład z Wikieł P 4.3a moja wersja nr 265}


Obliczyć granicę funkcji $\lim\limits_{x\to\ 0}\frac{15 \cdot x}{tan(17 \cdot x)}$.
\zadStop
\rozwStart{Patryk Wirkus}{}
$$\lim\limits_{x\to\ 0}\frac{15 \cdot x}{tan(17 \cdot x)}=\lim\limits_{x\to\ 0}\frac{15 \cdot x \cdot cos(17 \cdot x)}{sin(17 \cdot x)}=\lim\limits_{x\to\ 0}\frac{15 \cdot cos(17 \cdot x)}{\frac{sin(17 \cdot x)}{x}}=\lim\limits_{x\to\ 0}\frac{15 \cdot cos(17 \cdot x)}{17 \cdot \frac{sin(17 \cdot x)}{17 \cdot x}} = \frac{15}{17}$$
\rozwStop
\odpStart
$\frac{15}{17}$
\odpStop
\testStart
A.$\frac{15}{17}$
B.$\infty$
C.$-\infty$
D.$0$
E.$-\frac{15}{17}$
F.$\frac{17}{15}$
G.$-\frac{17}{15}$
H.$17$
I.$15$
\testStop
\kluczStart
A
\kluczStop



\zadStart{Przykład z Wikieł P 4.3a moja wersja nr 266}


Obliczyć granicę funkcji $\lim\limits_{x\to\ 0}\frac{15 \cdot x}{tan(19 \cdot x)}$.
\zadStop
\rozwStart{Patryk Wirkus}{}
$$\lim\limits_{x\to\ 0}\frac{15 \cdot x}{tan(19 \cdot x)}=\lim\limits_{x\to\ 0}\frac{15 \cdot x \cdot cos(19 \cdot x)}{sin(19 \cdot x)}=\lim\limits_{x\to\ 0}\frac{15 \cdot cos(19 \cdot x)}{\frac{sin(19 \cdot x)}{x}}=\lim\limits_{x\to\ 0}\frac{15 \cdot cos(19 \cdot x)}{19 \cdot \frac{sin(19 \cdot x)}{19 \cdot x}} = \frac{15}{19}$$
\rozwStop
\odpStart
$\frac{15}{19}$
\odpStop
\testStart
A.$\frac{15}{19}$
B.$\infty$
C.$-\infty$
D.$0$
E.$-\frac{15}{19}$
F.$\frac{19}{15}$
G.$-\frac{19}{15}$
H.$19$
I.$15$
\testStop
\kluczStart
A
\kluczStop



\zadStart{Przykład z Wikieł P 4.3a moja wersja nr 267}


Obliczyć granicę funkcji $\lim\limits_{x\to\ 0}\frac{15 \cdot x}{tan(22 \cdot x)}$.
\zadStop
\rozwStart{Patryk Wirkus}{}
$$\lim\limits_{x\to\ 0}\frac{15 \cdot x}{tan(22 \cdot x)}=\lim\limits_{x\to\ 0}\frac{15 \cdot x \cdot cos(22 \cdot x)}{sin(22 \cdot x)}=\lim\limits_{x\to\ 0}\frac{15 \cdot cos(22 \cdot x)}{\frac{sin(22 \cdot x)}{x}}=\lim\limits_{x\to\ 0}\frac{15 \cdot cos(22 \cdot x)}{22 \cdot \frac{sin(22 \cdot x)}{22 \cdot x}} = \frac{15}{22}$$
\rozwStop
\odpStart
$\frac{15}{22}$
\odpStop
\testStart
A.$\frac{15}{22}$
B.$\infty$
C.$-\infty$
D.$0$
E.$-\frac{15}{22}$
F.$\frac{22}{15}$
G.$-\frac{22}{15}$
H.$22$
I.$15$
\testStop
\kluczStart
A
\kluczStop



\zadStart{Przykład z Wikieł P 4.3a moja wersja nr 268}


Obliczyć granicę funkcji $\lim\limits_{x\to\ 0}\frac{15 \cdot x}{tan(23 \cdot x)}$.
\zadStop
\rozwStart{Patryk Wirkus}{}
$$\lim\limits_{x\to\ 0}\frac{15 \cdot x}{tan(23 \cdot x)}=\lim\limits_{x\to\ 0}\frac{15 \cdot x \cdot cos(23 \cdot x)}{sin(23 \cdot x)}=\lim\limits_{x\to\ 0}\frac{15 \cdot cos(23 \cdot x)}{\frac{sin(23 \cdot x)}{x}}=\lim\limits_{x\to\ 0}\frac{15 \cdot cos(23 \cdot x)}{23 \cdot \frac{sin(23 \cdot x)}{23 \cdot x}} = \frac{15}{23}$$
\rozwStop
\odpStart
$\frac{15}{23}$
\odpStop
\testStart
A.$\frac{15}{23}$
B.$\infty$
C.$-\infty$
D.$0$
E.$-\frac{15}{23}$
F.$\frac{23}{15}$
G.$-\frac{23}{15}$
H.$23$
I.$15$
\testStop
\kluczStart
A
\kluczStop



\zadStart{Przykład z Wikieł P 4.3a moja wersja nr 269}


Obliczyć granicę funkcji $\lim\limits_{x\to\ 0}\frac{15 \cdot x}{tan(26 \cdot x)}$.
\zadStop
\rozwStart{Patryk Wirkus}{}
$$\lim\limits_{x\to\ 0}\frac{15 \cdot x}{tan(26 \cdot x)}=\lim\limits_{x\to\ 0}\frac{15 \cdot x \cdot cos(26 \cdot x)}{sin(26 \cdot x)}=\lim\limits_{x\to\ 0}\frac{15 \cdot cos(26 \cdot x)}{\frac{sin(26 \cdot x)}{x}}=\lim\limits_{x\to\ 0}\frac{15 \cdot cos(26 \cdot x)}{26 \cdot \frac{sin(26 \cdot x)}{26 \cdot x}} = \frac{15}{26}$$
\rozwStop
\odpStart
$\frac{15}{26}$
\odpStop
\testStart
A.$\frac{15}{26}$
B.$\infty$
C.$-\infty$
D.$0$
E.$-\frac{15}{26}$
F.$\frac{26}{15}$
G.$-\frac{26}{15}$
H.$26$
I.$15$
\testStop
\kluczStart
A
\kluczStop



\zadStart{Przykład z Wikieł P 4.3a moja wersja nr 270}


Obliczyć granicę funkcji $\lim\limits_{x\to\ 0}\frac{15 \cdot x}{tan(28 \cdot x)}$.
\zadStop
\rozwStart{Patryk Wirkus}{}
$$\lim\limits_{x\to\ 0}\frac{15 \cdot x}{tan(28 \cdot x)}=\lim\limits_{x\to\ 0}\frac{15 \cdot x \cdot cos(28 \cdot x)}{sin(28 \cdot x)}=\lim\limits_{x\to\ 0}\frac{15 \cdot cos(28 \cdot x)}{\frac{sin(28 \cdot x)}{x}}=\lim\limits_{x\to\ 0}\frac{15 \cdot cos(28 \cdot x)}{28 \cdot \frac{sin(28 \cdot x)}{28 \cdot x}} = \frac{15}{28}$$
\rozwStop
\odpStart
$\frac{15}{28}$
\odpStop
\testStart
A.$\frac{15}{28}$
B.$\infty$
C.$-\infty$
D.$0$
E.$-\frac{15}{28}$
F.$\frac{28}{15}$
G.$-\frac{28}{15}$
H.$28$
I.$15$
\testStop
\kluczStart
A
\kluczStop



\zadStart{Przykład z Wikieł P 4.3a moja wersja nr 271}


Obliczyć granicę funkcji $\lim\limits_{x\to\ 0}\frac{15 \cdot x}{tan(29 \cdot x)}$.
\zadStop
\rozwStart{Patryk Wirkus}{}
$$\lim\limits_{x\to\ 0}\frac{15 \cdot x}{tan(29 \cdot x)}=\lim\limits_{x\to\ 0}\frac{15 \cdot x \cdot cos(29 \cdot x)}{sin(29 \cdot x)}=\lim\limits_{x\to\ 0}\frac{15 \cdot cos(29 \cdot x)}{\frac{sin(29 \cdot x)}{x}}=\lim\limits_{x\to\ 0}\frac{15 \cdot cos(29 \cdot x)}{29 \cdot \frac{sin(29 \cdot x)}{29 \cdot x}} = \frac{15}{29}$$
\rozwStop
\odpStart
$\frac{15}{29}$
\odpStop
\testStart
A.$\frac{15}{29}$
B.$\infty$
C.$-\infty$
D.$0$
E.$-\frac{15}{29}$
F.$\frac{29}{15}$
G.$-\frac{29}{15}$
H.$29$
I.$15$
\testStop
\kluczStart
A
\kluczStop



\zadStart{Przykład z Wikieł P 4.3a moja wersja nr 272}


Obliczyć granicę funkcji $\lim\limits_{x\to\ 0}\frac{15 \cdot x}{tan(31 \cdot x)}$.
\zadStop
\rozwStart{Patryk Wirkus}{}
$$\lim\limits_{x\to\ 0}\frac{15 \cdot x}{tan(31 \cdot x)}=\lim\limits_{x\to\ 0}\frac{15 \cdot x \cdot cos(31 \cdot x)}{sin(31 \cdot x)}=\lim\limits_{x\to\ 0}\frac{15 \cdot cos(31 \cdot x)}{\frac{sin(31 \cdot x)}{x}}=\lim\limits_{x\to\ 0}\frac{15 \cdot cos(31 \cdot x)}{31 \cdot \frac{sin(31 \cdot x)}{31 \cdot x}} = \frac{15}{31}$$
\rozwStop
\odpStart
$\frac{15}{31}$
\odpStop
\testStart
A.$\frac{15}{31}$
B.$\infty$
C.$-\infty$
D.$0$
E.$-\frac{15}{31}$
F.$\frac{31}{15}$
G.$-\frac{31}{15}$
H.$31$
I.$15$
\testStop
\kluczStart
A
\kluczStop



\zadStart{Przykład z Wikieł P 4.3a moja wersja nr 273}


Obliczyć granicę funkcji $\lim\limits_{x\to\ 0}\frac{15 \cdot x}{tan(32 \cdot x)}$.
\zadStop
\rozwStart{Patryk Wirkus}{}
$$\lim\limits_{x\to\ 0}\frac{15 \cdot x}{tan(32 \cdot x)}=\lim\limits_{x\to\ 0}\frac{15 \cdot x \cdot cos(32 \cdot x)}{sin(32 \cdot x)}=\lim\limits_{x\to\ 0}\frac{15 \cdot cos(32 \cdot x)}{\frac{sin(32 \cdot x)}{x}}=\lim\limits_{x\to\ 0}\frac{15 \cdot cos(32 \cdot x)}{32 \cdot \frac{sin(32 \cdot x)}{32 \cdot x}} = \frac{15}{32}$$
\rozwStop
\odpStart
$\frac{15}{32}$
\odpStop
\testStart
A.$\frac{15}{32}$
B.$\infty$
C.$-\infty$
D.$0$
E.$-\frac{15}{32}$
F.$\frac{32}{15}$
G.$-\frac{32}{15}$
H.$32$
I.$15$
\testStop
\kluczStart
A
\kluczStop



\zadStart{Przykład z Wikieł P 4.3a moja wersja nr 274}


Obliczyć granicę funkcji $\lim\limits_{x\to\ 0}\frac{15 \cdot x}{tan(34 \cdot x)}$.
\zadStop
\rozwStart{Patryk Wirkus}{}
$$\lim\limits_{x\to\ 0}\frac{15 \cdot x}{tan(34 \cdot x)}=\lim\limits_{x\to\ 0}\frac{15 \cdot x \cdot cos(34 \cdot x)}{sin(34 \cdot x)}=\lim\limits_{x\to\ 0}\frac{15 \cdot cos(34 \cdot x)}{\frac{sin(34 \cdot x)}{x}}=\lim\limits_{x\to\ 0}\frac{15 \cdot cos(34 \cdot x)}{34 \cdot \frac{sin(34 \cdot x)}{34 \cdot x}} = \frac{15}{34}$$
\rozwStop
\odpStart
$\frac{15}{34}$
\odpStop
\testStart
A.$\frac{15}{34}$
B.$\infty$
C.$-\infty$
D.$0$
E.$-\frac{15}{34}$
F.$\frac{34}{15}$
G.$-\frac{34}{15}$
H.$34$
I.$15$
\testStop
\kluczStart
A
\kluczStop



\zadStart{Przykład z Wikieł P 4.3a moja wersja nr 275}


Obliczyć granicę funkcji $\lim\limits_{x\to\ 0}\frac{15 \cdot x}{tan(37 \cdot x)}$.
\zadStop
\rozwStart{Patryk Wirkus}{}
$$\lim\limits_{x\to\ 0}\frac{15 \cdot x}{tan(37 \cdot x)}=\lim\limits_{x\to\ 0}\frac{15 \cdot x \cdot cos(37 \cdot x)}{sin(37 \cdot x)}=\lim\limits_{x\to\ 0}\frac{15 \cdot cos(37 \cdot x)}{\frac{sin(37 \cdot x)}{x}}=\lim\limits_{x\to\ 0}\frac{15 \cdot cos(37 \cdot x)}{37 \cdot \frac{sin(37 \cdot x)}{37 \cdot x}} = \frac{15}{37}$$
\rozwStop
\odpStart
$\frac{15}{37}$
\odpStop
\testStart
A.$\frac{15}{37}$
B.$\infty$
C.$-\infty$
D.$0$
E.$-\frac{15}{37}$
F.$\frac{37}{15}$
G.$-\frac{37}{15}$
H.$37$
I.$15$
\testStop
\kluczStart
A
\kluczStop



\zadStart{Przykład z Wikieł P 4.3a moja wersja nr 276}


Obliczyć granicę funkcji $\lim\limits_{x\to\ 0}\frac{15 \cdot x}{tan(38 \cdot x)}$.
\zadStop
\rozwStart{Patryk Wirkus}{}
$$\lim\limits_{x\to\ 0}\frac{15 \cdot x}{tan(38 \cdot x)}=\lim\limits_{x\to\ 0}\frac{15 \cdot x \cdot cos(38 \cdot x)}{sin(38 \cdot x)}=\lim\limits_{x\to\ 0}\frac{15 \cdot cos(38 \cdot x)}{\frac{sin(38 \cdot x)}{x}}=\lim\limits_{x\to\ 0}\frac{15 \cdot cos(38 \cdot x)}{38 \cdot \frac{sin(38 \cdot x)}{38 \cdot x}} = \frac{15}{38}$$
\rozwStop
\odpStart
$\frac{15}{38}$
\odpStop
\testStart
A.$\frac{15}{38}$
B.$\infty$
C.$-\infty$
D.$0$
E.$-\frac{15}{38}$
F.$\frac{38}{15}$
G.$-\frac{38}{15}$
H.$38$
I.$15$
\testStop
\kluczStart
A
\kluczStop



\zadStart{Przykład z Wikieł P 4.3a moja wersja nr 277}


Obliczyć granicę funkcji $\lim\limits_{x\to\ 0}\frac{16 \cdot x}{tan(3 \cdot x)}$.
\zadStop
\rozwStart{Patryk Wirkus}{}
$$\lim\limits_{x\to\ 0}\frac{16 \cdot x}{tan(3 \cdot x)}=\lim\limits_{x\to\ 0}\frac{16 \cdot x \cdot cos(3 \cdot x)}{sin(3 \cdot x)}=\lim\limits_{x\to\ 0}\frac{16 \cdot cos(3 \cdot x)}{\frac{sin(3 \cdot x)}{x}}=\lim\limits_{x\to\ 0}\frac{16 \cdot cos(3 \cdot x)}{3 \cdot \frac{sin(3 \cdot x)}{3 \cdot x}} = \frac{16}{3}$$
\rozwStop
\odpStart
$\frac{16}{3}$
\odpStop
\testStart
A.$\frac{16}{3}$
B.$\infty$
C.$-\infty$
D.$0$
E.$-\frac{16}{3}$
F.$\frac{3}{16}$
G.$-\frac{3}{16}$
H.$3$
I.$16$
\testStop
\kluczStart
A
\kluczStop



\zadStart{Przykład z Wikieł P 4.3a moja wersja nr 278}


Obliczyć granicę funkcji $\lim\limits_{x\to\ 0}\frac{16 \cdot x}{tan(5 \cdot x)}$.
\zadStop
\rozwStart{Patryk Wirkus}{}
$$\lim\limits_{x\to\ 0}\frac{16 \cdot x}{tan(5 \cdot x)}=\lim\limits_{x\to\ 0}\frac{16 \cdot x \cdot cos(5 \cdot x)}{sin(5 \cdot x)}=\lim\limits_{x\to\ 0}\frac{16 \cdot cos(5 \cdot x)}{\frac{sin(5 \cdot x)}{x}}=\lim\limits_{x\to\ 0}\frac{16 \cdot cos(5 \cdot x)}{5 \cdot \frac{sin(5 \cdot x)}{5 \cdot x}} = \frac{16}{5}$$
\rozwStop
\odpStart
$\frac{16}{5}$
\odpStop
\testStart
A.$\frac{16}{5}$
B.$\infty$
C.$-\infty$
D.$0$
E.$-\frac{16}{5}$
F.$\frac{5}{16}$
G.$-\frac{5}{16}$
H.$5$
I.$16$
\testStop
\kluczStart
A
\kluczStop



\zadStart{Przykład z Wikieł P 4.3a moja wersja nr 279}


Obliczyć granicę funkcji $\lim\limits_{x\to\ 0}\frac{16 \cdot x}{tan(7 \cdot x)}$.
\zadStop
\rozwStart{Patryk Wirkus}{}
$$\lim\limits_{x\to\ 0}\frac{16 \cdot x}{tan(7 \cdot x)}=\lim\limits_{x\to\ 0}\frac{16 \cdot x \cdot cos(7 \cdot x)}{sin(7 \cdot x)}=\lim\limits_{x\to\ 0}\frac{16 \cdot cos(7 \cdot x)}{\frac{sin(7 \cdot x)}{x}}=\lim\limits_{x\to\ 0}\frac{16 \cdot cos(7 \cdot x)}{7 \cdot \frac{sin(7 \cdot x)}{7 \cdot x}} = \frac{16}{7}$$
\rozwStop
\odpStart
$\frac{16}{7}$
\odpStop
\testStart
A.$\frac{16}{7}$
B.$\infty$
C.$-\infty$
D.$0$
E.$-\frac{16}{7}$
F.$\frac{7}{16}$
G.$-\frac{7}{16}$
H.$7$
I.$16$
\testStop
\kluczStart
A
\kluczStop



\zadStart{Przykład z Wikieł P 4.3a moja wersja nr 280}


Obliczyć granicę funkcji $\lim\limits_{x\to\ 0}\frac{16 \cdot x}{tan(9 \cdot x)}$.
\zadStop
\rozwStart{Patryk Wirkus}{}
$$\lim\limits_{x\to\ 0}\frac{16 \cdot x}{tan(9 \cdot x)}=\lim\limits_{x\to\ 0}\frac{16 \cdot x \cdot cos(9 \cdot x)}{sin(9 \cdot x)}=\lim\limits_{x\to\ 0}\frac{16 \cdot cos(9 \cdot x)}{\frac{sin(9 \cdot x)}{x}}=\lim\limits_{x\to\ 0}\frac{16 \cdot cos(9 \cdot x)}{9 \cdot \frac{sin(9 \cdot x)}{9 \cdot x}} = \frac{16}{9}$$
\rozwStop
\odpStart
$\frac{16}{9}$
\odpStop
\testStart
A.$\frac{16}{9}$
B.$\infty$
C.$-\infty$
D.$0$
E.$-\frac{16}{9}$
F.$\frac{9}{16}$
G.$-\frac{9}{16}$
H.$9$
I.$16$
\testStop
\kluczStart
A
\kluczStop



\zadStart{Przykład z Wikieł P 4.3a moja wersja nr 281}


Obliczyć granicę funkcji $\lim\limits_{x\to\ 0}\frac{16 \cdot x}{tan(11 \cdot x)}$.
\zadStop
\rozwStart{Patryk Wirkus}{}
$$\lim\limits_{x\to\ 0}\frac{16 \cdot x}{tan(11 \cdot x)}=\lim\limits_{x\to\ 0}\frac{16 \cdot x \cdot cos(11 \cdot x)}{sin(11 \cdot x)}=\lim\limits_{x\to\ 0}\frac{16 \cdot cos(11 \cdot x)}{\frac{sin(11 \cdot x)}{x}}=\lim\limits_{x\to\ 0}\frac{16 \cdot cos(11 \cdot x)}{11 \cdot \frac{sin(11 \cdot x)}{11 \cdot x}} = \frac{16}{11}$$
\rozwStop
\odpStart
$\frac{16}{11}$
\odpStop
\testStart
A.$\frac{16}{11}$
B.$\infty$
C.$-\infty$
D.$0$
E.$-\frac{16}{11}$
F.$\frac{11}{16}$
G.$-\frac{11}{16}$
H.$11$
I.$16$
\testStop
\kluczStart
A
\kluczStop



\zadStart{Przykład z Wikieł P 4.3a moja wersja nr 282}


Obliczyć granicę funkcji $\lim\limits_{x\to\ 0}\frac{16 \cdot x}{tan(13 \cdot x)}$.
\zadStop
\rozwStart{Patryk Wirkus}{}
$$\lim\limits_{x\to\ 0}\frac{16 \cdot x}{tan(13 \cdot x)}=\lim\limits_{x\to\ 0}\frac{16 \cdot x \cdot cos(13 \cdot x)}{sin(13 \cdot x)}=\lim\limits_{x\to\ 0}\frac{16 \cdot cos(13 \cdot x)}{\frac{sin(13 \cdot x)}{x}}=\lim\limits_{x\to\ 0}\frac{16 \cdot cos(13 \cdot x)}{13 \cdot \frac{sin(13 \cdot x)}{13 \cdot x}} = \frac{16}{13}$$
\rozwStop
\odpStart
$\frac{16}{13}$
\odpStop
\testStart
A.$\frac{16}{13}$
B.$\infty$
C.$-\infty$
D.$0$
E.$-\frac{16}{13}$
F.$\frac{13}{16}$
G.$-\frac{13}{16}$
H.$13$
I.$16$
\testStop
\kluczStart
A
\kluczStop



\zadStart{Przykład z Wikieł P 4.3a moja wersja nr 283}


Obliczyć granicę funkcji $\lim\limits_{x\to\ 0}\frac{16 \cdot x}{tan(15 \cdot x)}$.
\zadStop
\rozwStart{Patryk Wirkus}{}
$$\lim\limits_{x\to\ 0}\frac{16 \cdot x}{tan(15 \cdot x)}=\lim\limits_{x\to\ 0}\frac{16 \cdot x \cdot cos(15 \cdot x)}{sin(15 \cdot x)}=\lim\limits_{x\to\ 0}\frac{16 \cdot cos(15 \cdot x)}{\frac{sin(15 \cdot x)}{x}}=\lim\limits_{x\to\ 0}\frac{16 \cdot cos(15 \cdot x)}{15 \cdot \frac{sin(15 \cdot x)}{15 \cdot x}} = \frac{16}{15}$$
\rozwStop
\odpStart
$\frac{16}{15}$
\odpStop
\testStart
A.$\frac{16}{15}$
B.$\infty$
C.$-\infty$
D.$0$
E.$-\frac{16}{15}$
F.$\frac{15}{16}$
G.$-\frac{15}{16}$
H.$15$
I.$16$
\testStop
\kluczStart
A
\kluczStop



\zadStart{Przykład z Wikieł P 4.3a moja wersja nr 284}


Obliczyć granicę funkcji $\lim\limits_{x\to\ 0}\frac{16 \cdot x}{tan(17 \cdot x)}$.
\zadStop
\rozwStart{Patryk Wirkus}{}
$$\lim\limits_{x\to\ 0}\frac{16 \cdot x}{tan(17 \cdot x)}=\lim\limits_{x\to\ 0}\frac{16 \cdot x \cdot cos(17 \cdot x)}{sin(17 \cdot x)}=\lim\limits_{x\to\ 0}\frac{16 \cdot cos(17 \cdot x)}{\frac{sin(17 \cdot x)}{x}}=\lim\limits_{x\to\ 0}\frac{16 \cdot cos(17 \cdot x)}{17 \cdot \frac{sin(17 \cdot x)}{17 \cdot x}} = \frac{16}{17}$$
\rozwStop
\odpStart
$\frac{16}{17}$
\odpStop
\testStart
A.$\frac{16}{17}$
B.$\infty$
C.$-\infty$
D.$0$
E.$-\frac{16}{17}$
F.$\frac{17}{16}$
G.$-\frac{17}{16}$
H.$17$
I.$16$
\testStop
\kluczStart
A
\kluczStop



\zadStart{Przykład z Wikieł P 4.3a moja wersja nr 285}


Obliczyć granicę funkcji $\lim\limits_{x\to\ 0}\frac{16 \cdot x}{tan(19 \cdot x)}$.
\zadStop
\rozwStart{Patryk Wirkus}{}
$$\lim\limits_{x\to\ 0}\frac{16 \cdot x}{tan(19 \cdot x)}=\lim\limits_{x\to\ 0}\frac{16 \cdot x \cdot cos(19 \cdot x)}{sin(19 \cdot x)}=\lim\limits_{x\to\ 0}\frac{16 \cdot cos(19 \cdot x)}{\frac{sin(19 \cdot x)}{x}}=\lim\limits_{x\to\ 0}\frac{16 \cdot cos(19 \cdot x)}{19 \cdot \frac{sin(19 \cdot x)}{19 \cdot x}} = \frac{16}{19}$$
\rozwStop
\odpStart
$\frac{16}{19}$
\odpStop
\testStart
A.$\frac{16}{19}$
B.$\infty$
C.$-\infty$
D.$0$
E.$-\frac{16}{19}$
F.$\frac{19}{16}$
G.$-\frac{19}{16}$
H.$19$
I.$16$
\testStop
\kluczStart
A
\kluczStop



\zadStart{Przykład z Wikieł P 4.3a moja wersja nr 286}


Obliczyć granicę funkcji $\lim\limits_{x\to\ 0}\frac{16 \cdot x}{tan(21 \cdot x)}$.
\zadStop
\rozwStart{Patryk Wirkus}{}
$$\lim\limits_{x\to\ 0}\frac{16 \cdot x}{tan(21 \cdot x)}=\lim\limits_{x\to\ 0}\frac{16 \cdot x \cdot cos(21 \cdot x)}{sin(21 \cdot x)}=\lim\limits_{x\to\ 0}\frac{16 \cdot cos(21 \cdot x)}{\frac{sin(21 \cdot x)}{x}}=\lim\limits_{x\to\ 0}\frac{16 \cdot cos(21 \cdot x)}{21 \cdot \frac{sin(21 \cdot x)}{21 \cdot x}} = \frac{16}{21}$$
\rozwStop
\odpStart
$\frac{16}{21}$
\odpStop
\testStart
A.$\frac{16}{21}$
B.$\infty$
C.$-\infty$
D.$0$
E.$-\frac{16}{21}$
F.$\frac{21}{16}$
G.$-\frac{21}{16}$
H.$21$
I.$16$
\testStop
\kluczStart
A
\kluczStop



\zadStart{Przykład z Wikieł P 4.3a moja wersja nr 287}


Obliczyć granicę funkcji $\lim\limits_{x\to\ 0}\frac{16 \cdot x}{tan(23 \cdot x)}$.
\zadStop
\rozwStart{Patryk Wirkus}{}
$$\lim\limits_{x\to\ 0}\frac{16 \cdot x}{tan(23 \cdot x)}=\lim\limits_{x\to\ 0}\frac{16 \cdot x \cdot cos(23 \cdot x)}{sin(23 \cdot x)}=\lim\limits_{x\to\ 0}\frac{16 \cdot cos(23 \cdot x)}{\frac{sin(23 \cdot x)}{x}}=\lim\limits_{x\to\ 0}\frac{16 \cdot cos(23 \cdot x)}{23 \cdot \frac{sin(23 \cdot x)}{23 \cdot x}} = \frac{16}{23}$$
\rozwStop
\odpStart
$\frac{16}{23}$
\odpStop
\testStart
A.$\frac{16}{23}$
B.$\infty$
C.$-\infty$
D.$0$
E.$-\frac{16}{23}$
F.$\frac{23}{16}$
G.$-\frac{23}{16}$
H.$23$
I.$16$
\testStop
\kluczStart
A
\kluczStop



\zadStart{Przykład z Wikieł P 4.3a moja wersja nr 288}


Obliczyć granicę funkcji $\lim\limits_{x\to\ 0}\frac{16 \cdot x}{tan(25 \cdot x)}$.
\zadStop
\rozwStart{Patryk Wirkus}{}
$$\lim\limits_{x\to\ 0}\frac{16 \cdot x}{tan(25 \cdot x)}=\lim\limits_{x\to\ 0}\frac{16 \cdot x \cdot cos(25 \cdot x)}{sin(25 \cdot x)}=\lim\limits_{x\to\ 0}\frac{16 \cdot cos(25 \cdot x)}{\frac{sin(25 \cdot x)}{x}}=\lim\limits_{x\to\ 0}\frac{16 \cdot cos(25 \cdot x)}{25 \cdot \frac{sin(25 \cdot x)}{25 \cdot x}} = \frac{16}{25}$$
\rozwStop
\odpStart
$\frac{16}{25}$
\odpStop
\testStart
A.$\frac{16}{25}$
B.$\infty$
C.$-\infty$
D.$0$
E.$-\frac{16}{25}$
F.$\frac{25}{16}$
G.$-\frac{25}{16}$
H.$25$
I.$16$
\testStop
\kluczStart
A
\kluczStop



\zadStart{Przykład z Wikieł P 4.3a moja wersja nr 289}


Obliczyć granicę funkcji $\lim\limits_{x\to\ 0}\frac{16 \cdot x}{tan(27 \cdot x)}$.
\zadStop
\rozwStart{Patryk Wirkus}{}
$$\lim\limits_{x\to\ 0}\frac{16 \cdot x}{tan(27 \cdot x)}=\lim\limits_{x\to\ 0}\frac{16 \cdot x \cdot cos(27 \cdot x)}{sin(27 \cdot x)}=\lim\limits_{x\to\ 0}\frac{16 \cdot cos(27 \cdot x)}{\frac{sin(27 \cdot x)}{x}}=\lim\limits_{x\to\ 0}\frac{16 \cdot cos(27 \cdot x)}{27 \cdot \frac{sin(27 \cdot x)}{27 \cdot x}} = \frac{16}{27}$$
\rozwStop
\odpStart
$\frac{16}{27}$
\odpStop
\testStart
A.$\frac{16}{27}$
B.$\infty$
C.$-\infty$
D.$0$
E.$-\frac{16}{27}$
F.$\frac{27}{16}$
G.$-\frac{27}{16}$
H.$27$
I.$16$
\testStop
\kluczStart
A
\kluczStop



\zadStart{Przykład z Wikieł P 4.3a moja wersja nr 290}


Obliczyć granicę funkcji $\lim\limits_{x\to\ 0}\frac{16 \cdot x}{tan(29 \cdot x)}$.
\zadStop
\rozwStart{Patryk Wirkus}{}
$$\lim\limits_{x\to\ 0}\frac{16 \cdot x}{tan(29 \cdot x)}=\lim\limits_{x\to\ 0}\frac{16 \cdot x \cdot cos(29 \cdot x)}{sin(29 \cdot x)}=\lim\limits_{x\to\ 0}\frac{16 \cdot cos(29 \cdot x)}{\frac{sin(29 \cdot x)}{x}}=\lim\limits_{x\to\ 0}\frac{16 \cdot cos(29 \cdot x)}{29 \cdot \frac{sin(29 \cdot x)}{29 \cdot x}} = \frac{16}{29}$$
\rozwStop
\odpStart
$\frac{16}{29}$
\odpStop
\testStart
A.$\frac{16}{29}$
B.$\infty$
C.$-\infty$
D.$0$
E.$-\frac{16}{29}$
F.$\frac{29}{16}$
G.$-\frac{29}{16}$
H.$29$
I.$16$
\testStop
\kluczStart
A
\kluczStop



\zadStart{Przykład z Wikieł P 4.3a moja wersja nr 291}


Obliczyć granicę funkcji $\lim\limits_{x\to\ 0}\frac{16 \cdot x}{tan(31 \cdot x)}$.
\zadStop
\rozwStart{Patryk Wirkus}{}
$$\lim\limits_{x\to\ 0}\frac{16 \cdot x}{tan(31 \cdot x)}=\lim\limits_{x\to\ 0}\frac{16 \cdot x \cdot cos(31 \cdot x)}{sin(31 \cdot x)}=\lim\limits_{x\to\ 0}\frac{16 \cdot cos(31 \cdot x)}{\frac{sin(31 \cdot x)}{x}}=\lim\limits_{x\to\ 0}\frac{16 \cdot cos(31 \cdot x)}{31 \cdot \frac{sin(31 \cdot x)}{31 \cdot x}} = \frac{16}{31}$$
\rozwStop
\odpStart
$\frac{16}{31}$
\odpStop
\testStart
A.$\frac{16}{31}$
B.$\infty$
C.$-\infty$
D.$0$
E.$-\frac{16}{31}$
F.$\frac{31}{16}$
G.$-\frac{31}{16}$
H.$31$
I.$16$
\testStop
\kluczStart
A
\kluczStop



\zadStart{Przykład z Wikieł P 4.3a moja wersja nr 292}


Obliczyć granicę funkcji $\lim\limits_{x\to\ 0}\frac{16 \cdot x}{tan(33 \cdot x)}$.
\zadStop
\rozwStart{Patryk Wirkus}{}
$$\lim\limits_{x\to\ 0}\frac{16 \cdot x}{tan(33 \cdot x)}=\lim\limits_{x\to\ 0}\frac{16 \cdot x \cdot cos(33 \cdot x)}{sin(33 \cdot x)}=\lim\limits_{x\to\ 0}\frac{16 \cdot cos(33 \cdot x)}{\frac{sin(33 \cdot x)}{x}}=\lim\limits_{x\to\ 0}\frac{16 \cdot cos(33 \cdot x)}{33 \cdot \frac{sin(33 \cdot x)}{33 \cdot x}} = \frac{16}{33}$$
\rozwStop
\odpStart
$\frac{16}{33}$
\odpStop
\testStart
A.$\frac{16}{33}$
B.$\infty$
C.$-\infty$
D.$0$
E.$-\frac{16}{33}$
F.$\frac{33}{16}$
G.$-\frac{33}{16}$
H.$33$
I.$16$
\testStop
\kluczStart
A
\kluczStop



\zadStart{Przykład z Wikieł P 4.3a moja wersja nr 293}


Obliczyć granicę funkcji $\lim\limits_{x\to\ 0}\frac{16 \cdot x}{tan(35 \cdot x)}$.
\zadStop
\rozwStart{Patryk Wirkus}{}
$$\lim\limits_{x\to\ 0}\frac{16 \cdot x}{tan(35 \cdot x)}=\lim\limits_{x\to\ 0}\frac{16 \cdot x \cdot cos(35 \cdot x)}{sin(35 \cdot x)}=\lim\limits_{x\to\ 0}\frac{16 \cdot cos(35 \cdot x)}{\frac{sin(35 \cdot x)}{x}}=\lim\limits_{x\to\ 0}\frac{16 \cdot cos(35 \cdot x)}{35 \cdot \frac{sin(35 \cdot x)}{35 \cdot x}} = \frac{16}{35}$$
\rozwStop
\odpStart
$\frac{16}{35}$
\odpStop
\testStart
A.$\frac{16}{35}$
B.$\infty$
C.$-\infty$
D.$0$
E.$-\frac{16}{35}$
F.$\frac{35}{16}$
G.$-\frac{35}{16}$
H.$35$
I.$16$
\testStop
\kluczStart
A
\kluczStop



\zadStart{Przykład z Wikieł P 4.3a moja wersja nr 294}


Obliczyć granicę funkcji $\lim\limits_{x\to\ 0}\frac{16 \cdot x}{tan(37 \cdot x)}$.
\zadStop
\rozwStart{Patryk Wirkus}{}
$$\lim\limits_{x\to\ 0}\frac{16 \cdot x}{tan(37 \cdot x)}=\lim\limits_{x\to\ 0}\frac{16 \cdot x \cdot cos(37 \cdot x)}{sin(37 \cdot x)}=\lim\limits_{x\to\ 0}\frac{16 \cdot cos(37 \cdot x)}{\frac{sin(37 \cdot x)}{x}}=\lim\limits_{x\to\ 0}\frac{16 \cdot cos(37 \cdot x)}{37 \cdot \frac{sin(37 \cdot x)}{37 \cdot x}} = \frac{16}{37}$$
\rozwStop
\odpStart
$\frac{16}{37}$
\odpStop
\testStart
A.$\frac{16}{37}$
B.$\infty$
C.$-\infty$
D.$0$
E.$-\frac{16}{37}$
F.$\frac{37}{16}$
G.$-\frac{37}{16}$
H.$37$
I.$16$
\testStop
\kluczStart
A
\kluczStop



\zadStart{Przykład z Wikieł P 4.3a moja wersja nr 295}


Obliczyć granicę funkcji $\lim\limits_{x\to\ 0}\frac{16 \cdot x}{tan(39 \cdot x)}$.
\zadStop
\rozwStart{Patryk Wirkus}{}
$$\lim\limits_{x\to\ 0}\frac{16 \cdot x}{tan(39 \cdot x)}=\lim\limits_{x\to\ 0}\frac{16 \cdot x \cdot cos(39 \cdot x)}{sin(39 \cdot x)}=\lim\limits_{x\to\ 0}\frac{16 \cdot cos(39 \cdot x)}{\frac{sin(39 \cdot x)}{x}}=\lim\limits_{x\to\ 0}\frac{16 \cdot cos(39 \cdot x)}{39 \cdot \frac{sin(39 \cdot x)}{39 \cdot x}} = \frac{16}{39}$$
\rozwStop
\odpStart
$\frac{16}{39}$
\odpStop
\testStart
A.$\frac{16}{39}$
B.$\infty$
C.$-\infty$
D.$0$
E.$-\frac{16}{39}$
F.$\frac{39}{16}$
G.$-\frac{39}{16}$
H.$39$
I.$16$
\testStop
\kluczStart
A
\kluczStop



\zadStart{Przykład z Wikieł P 4.3a moja wersja nr 296}


Obliczyć granicę funkcji $\lim\limits_{x\to\ 0}\frac{17 \cdot x}{tan(2 \cdot x)}$.
\zadStop
\rozwStart{Patryk Wirkus}{}
$$\lim\limits_{x\to\ 0}\frac{17 \cdot x}{tan(2 \cdot x)}=\lim\limits_{x\to\ 0}\frac{17 \cdot x \cdot cos(2 \cdot x)}{sin(2 \cdot x)}=\lim\limits_{x\to\ 0}\frac{17 \cdot cos(2 \cdot x)}{\frac{sin(2 \cdot x)}{x}}=\lim\limits_{x\to\ 0}\frac{17 \cdot cos(2 \cdot x)}{2 \cdot \frac{sin(2 \cdot x)}{2 \cdot x}} = \frac{17}{2}$$
\rozwStop
\odpStart
$\frac{17}{2}$
\odpStop
\testStart
A.$\frac{17}{2}$
B.$\infty$
C.$-\infty$
D.$0$
E.$-\frac{17}{2}$
F.$\frac{2}{17}$
G.$-\frac{2}{17}$
H.$2$
I.$17$
\testStop
\kluczStart
A
\kluczStop



\zadStart{Przykład z Wikieł P 4.3a moja wersja nr 297}


Obliczyć granicę funkcji $\lim\limits_{x\to\ 0}\frac{17 \cdot x}{tan(3 \cdot x)}$.
\zadStop
\rozwStart{Patryk Wirkus}{}
$$\lim\limits_{x\to\ 0}\frac{17 \cdot x}{tan(3 \cdot x)}=\lim\limits_{x\to\ 0}\frac{17 \cdot x \cdot cos(3 \cdot x)}{sin(3 \cdot x)}=\lim\limits_{x\to\ 0}\frac{17 \cdot cos(3 \cdot x)}{\frac{sin(3 \cdot x)}{x}}=\lim\limits_{x\to\ 0}\frac{17 \cdot cos(3 \cdot x)}{3 \cdot \frac{sin(3 \cdot x)}{3 \cdot x}} = \frac{17}{3}$$
\rozwStop
\odpStart
$\frac{17}{3}$
\odpStop
\testStart
A.$\frac{17}{3}$
B.$\infty$
C.$-\infty$
D.$0$
E.$-\frac{17}{3}$
F.$\frac{3}{17}$
G.$-\frac{3}{17}$
H.$3$
I.$17$
\testStop
\kluczStart
A
\kluczStop



\zadStart{Przykład z Wikieł P 4.3a moja wersja nr 298}


Obliczyć granicę funkcji $\lim\limits_{x\to\ 0}\frac{17 \cdot x}{tan(4 \cdot x)}$.
\zadStop
\rozwStart{Patryk Wirkus}{}
$$\lim\limits_{x\to\ 0}\frac{17 \cdot x}{tan(4 \cdot x)}=\lim\limits_{x\to\ 0}\frac{17 \cdot x \cdot cos(4 \cdot x)}{sin(4 \cdot x)}=\lim\limits_{x\to\ 0}\frac{17 \cdot cos(4 \cdot x)}{\frac{sin(4 \cdot x)}{x}}=\lim\limits_{x\to\ 0}\frac{17 \cdot cos(4 \cdot x)}{4 \cdot \frac{sin(4 \cdot x)}{4 \cdot x}} = \frac{17}{4}$$
\rozwStop
\odpStart
$\frac{17}{4}$
\odpStop
\testStart
A.$\frac{17}{4}$
B.$\infty$
C.$-\infty$
D.$0$
E.$-\frac{17}{4}$
F.$\frac{4}{17}$
G.$-\frac{4}{17}$
H.$4$
I.$17$
\testStop
\kluczStart
A
\kluczStop



\zadStart{Przykład z Wikieł P 4.3a moja wersja nr 299}


Obliczyć granicę funkcji $\lim\limits_{x\to\ 0}\frac{17 \cdot x}{tan(5 \cdot x)}$.
\zadStop
\rozwStart{Patryk Wirkus}{}
$$\lim\limits_{x\to\ 0}\frac{17 \cdot x}{tan(5 \cdot x)}=\lim\limits_{x\to\ 0}\frac{17 \cdot x \cdot cos(5 \cdot x)}{sin(5 \cdot x)}=\lim\limits_{x\to\ 0}\frac{17 \cdot cos(5 \cdot x)}{\frac{sin(5 \cdot x)}{x}}=\lim\limits_{x\to\ 0}\frac{17 \cdot cos(5 \cdot x)}{5 \cdot \frac{sin(5 \cdot x)}{5 \cdot x}} = \frac{17}{5}$$
\rozwStop
\odpStart
$\frac{17}{5}$
\odpStop
\testStart
A.$\frac{17}{5}$
B.$\infty$
C.$-\infty$
D.$0$
E.$-\frac{17}{5}$
F.$\frac{5}{17}$
G.$-\frac{5}{17}$
H.$5$
I.$17$
\testStop
\kluczStart
A
\kluczStop



\zadStart{Przykład z Wikieł P 4.3a moja wersja nr 300}


Obliczyć granicę funkcji $\lim\limits_{x\to\ 0}\frac{17 \cdot x}{tan(6 \cdot x)}$.
\zadStop
\rozwStart{Patryk Wirkus}{}
$$\lim\limits_{x\to\ 0}\frac{17 \cdot x}{tan(6 \cdot x)}=\lim\limits_{x\to\ 0}\frac{17 \cdot x \cdot cos(6 \cdot x)}{sin(6 \cdot x)}=\lim\limits_{x\to\ 0}\frac{17 \cdot cos(6 \cdot x)}{\frac{sin(6 \cdot x)}{x}}=\lim\limits_{x\to\ 0}\frac{17 \cdot cos(6 \cdot x)}{6 \cdot \frac{sin(6 \cdot x)}{6 \cdot x}} = \frac{17}{6}$$
\rozwStop
\odpStart
$\frac{17}{6}$
\odpStop
\testStart
A.$\frac{17}{6}$
B.$\infty$
C.$-\infty$
D.$0$
E.$-\frac{17}{6}$
F.$\frac{6}{17}$
G.$-\frac{6}{17}$
H.$6$
I.$17$
\testStop
\kluczStart
A
\kluczStop



\zadStart{Przykład z Wikieł P 4.3a moja wersja nr 301}


Obliczyć granicę funkcji $\lim\limits_{x\to\ 0}\frac{17 \cdot x}{tan(7 \cdot x)}$.
\zadStop
\rozwStart{Patryk Wirkus}{}
$$\lim\limits_{x\to\ 0}\frac{17 \cdot x}{tan(7 \cdot x)}=\lim\limits_{x\to\ 0}\frac{17 \cdot x \cdot cos(7 \cdot x)}{sin(7 \cdot x)}=\lim\limits_{x\to\ 0}\frac{17 \cdot cos(7 \cdot x)}{\frac{sin(7 \cdot x)}{x}}=\lim\limits_{x\to\ 0}\frac{17 \cdot cos(7 \cdot x)}{7 \cdot \frac{sin(7 \cdot x)}{7 \cdot x}} = \frac{17}{7}$$
\rozwStop
\odpStart
$\frac{17}{7}$
\odpStop
\testStart
A.$\frac{17}{7}$
B.$\infty$
C.$-\infty$
D.$0$
E.$-\frac{17}{7}$
F.$\frac{7}{17}$
G.$-\frac{7}{17}$
H.$7$
I.$17$
\testStop
\kluczStart
A
\kluczStop



\zadStart{Przykład z Wikieł P 4.3a moja wersja nr 302}


Obliczyć granicę funkcji $\lim\limits_{x\to\ 0}\frac{17 \cdot x}{tan(8 \cdot x)}$.
\zadStop
\rozwStart{Patryk Wirkus}{}
$$\lim\limits_{x\to\ 0}\frac{17 \cdot x}{tan(8 \cdot x)}=\lim\limits_{x\to\ 0}\frac{17 \cdot x \cdot cos(8 \cdot x)}{sin(8 \cdot x)}=\lim\limits_{x\to\ 0}\frac{17 \cdot cos(8 \cdot x)}{\frac{sin(8 \cdot x)}{x}}=\lim\limits_{x\to\ 0}\frac{17 \cdot cos(8 \cdot x)}{8 \cdot \frac{sin(8 \cdot x)}{8 \cdot x}} = \frac{17}{8}$$
\rozwStop
\odpStart
$\frac{17}{8}$
\odpStop
\testStart
A.$\frac{17}{8}$
B.$\infty$
C.$-\infty$
D.$0$
E.$-\frac{17}{8}$
F.$\frac{8}{17}$
G.$-\frac{8}{17}$
H.$8$
I.$17$
\testStop
\kluczStart
A
\kluczStop



\zadStart{Przykład z Wikieł P 4.3a moja wersja nr 303}


Obliczyć granicę funkcji $\lim\limits_{x\to\ 0}\frac{17 \cdot x}{tan(9 \cdot x)}$.
\zadStop
\rozwStart{Patryk Wirkus}{}
$$\lim\limits_{x\to\ 0}\frac{17 \cdot x}{tan(9 \cdot x)}=\lim\limits_{x\to\ 0}\frac{17 \cdot x \cdot cos(9 \cdot x)}{sin(9 \cdot x)}=\lim\limits_{x\to\ 0}\frac{17 \cdot cos(9 \cdot x)}{\frac{sin(9 \cdot x)}{x}}=\lim\limits_{x\to\ 0}\frac{17 \cdot cos(9 \cdot x)}{9 \cdot \frac{sin(9 \cdot x)}{9 \cdot x}} = \frac{17}{9}$$
\rozwStop
\odpStart
$\frac{17}{9}$
\odpStop
\testStart
A.$\frac{17}{9}$
B.$\infty$
C.$-\infty$
D.$0$
E.$-\frac{17}{9}$
F.$\frac{9}{17}$
G.$-\frac{9}{17}$
H.$9$
I.$17$
\testStop
\kluczStart
A
\kluczStop



\zadStart{Przykład z Wikieł P 4.3a moja wersja nr 304}


Obliczyć granicę funkcji $\lim\limits_{x\to\ 0}\frac{17 \cdot x}{tan(10 \cdot x)}$.
\zadStop
\rozwStart{Patryk Wirkus}{}
$$\lim\limits_{x\to\ 0}\frac{17 \cdot x}{tan(10 \cdot x)}=\lim\limits_{x\to\ 0}\frac{17 \cdot x \cdot cos(10 \cdot x)}{sin(10 \cdot x)}=\lim\limits_{x\to\ 0}\frac{17 \cdot cos(10 \cdot x)}{\frac{sin(10 \cdot x)}{x}}=\lim\limits_{x\to\ 0}\frac{17 \cdot cos(10 \cdot x)}{10 \cdot \frac{sin(10 \cdot x)}{10 \cdot x}} = \frac{17}{10}$$
\rozwStop
\odpStart
$\frac{17}{10}$
\odpStop
\testStart
A.$\frac{17}{10}$
B.$\infty$
C.$-\infty$
D.$0$
E.$-\frac{17}{10}$
F.$\frac{10}{17}$
G.$-\frac{10}{17}$
H.$10$
I.$17$
\testStop
\kluczStart
A
\kluczStop



\zadStart{Przykład z Wikieł P 4.3a moja wersja nr 305}


Obliczyć granicę funkcji $\lim\limits_{x\to\ 0}\frac{17 \cdot x}{tan(11 \cdot x)}$.
\zadStop
\rozwStart{Patryk Wirkus}{}
$$\lim\limits_{x\to\ 0}\frac{17 \cdot x}{tan(11 \cdot x)}=\lim\limits_{x\to\ 0}\frac{17 \cdot x \cdot cos(11 \cdot x)}{sin(11 \cdot x)}=\lim\limits_{x\to\ 0}\frac{17 \cdot cos(11 \cdot x)}{\frac{sin(11 \cdot x)}{x}}=\lim\limits_{x\to\ 0}\frac{17 \cdot cos(11 \cdot x)}{11 \cdot \frac{sin(11 \cdot x)}{11 \cdot x}} = \frac{17}{11}$$
\rozwStop
\odpStart
$\frac{17}{11}$
\odpStop
\testStart
A.$\frac{17}{11}$
B.$\infty$
C.$-\infty$
D.$0$
E.$-\frac{17}{11}$
F.$\frac{11}{17}$
G.$-\frac{11}{17}$
H.$11$
I.$17$
\testStop
\kluczStart
A
\kluczStop



\zadStart{Przykład z Wikieł P 4.3a moja wersja nr 306}


Obliczyć granicę funkcji $\lim\limits_{x\to\ 0}\frac{17 \cdot x}{tan(12 \cdot x)}$.
\zadStop
\rozwStart{Patryk Wirkus}{}
$$\lim\limits_{x\to\ 0}\frac{17 \cdot x}{tan(12 \cdot x)}=\lim\limits_{x\to\ 0}\frac{17 \cdot x \cdot cos(12 \cdot x)}{sin(12 \cdot x)}=\lim\limits_{x\to\ 0}\frac{17 \cdot cos(12 \cdot x)}{\frac{sin(12 \cdot x)}{x}}=\lim\limits_{x\to\ 0}\frac{17 \cdot cos(12 \cdot x)}{12 \cdot \frac{sin(12 \cdot x)}{12 \cdot x}} = \frac{17}{12}$$
\rozwStop
\odpStart
$\frac{17}{12}$
\odpStop
\testStart
A.$\frac{17}{12}$
B.$\infty$
C.$-\infty$
D.$0$
E.$-\frac{17}{12}$
F.$\frac{12}{17}$
G.$-\frac{12}{17}$
H.$12$
I.$17$
\testStop
\kluczStart
A
\kluczStop



\zadStart{Przykład z Wikieł P 4.3a moja wersja nr 307}


Obliczyć granicę funkcji $\lim\limits_{x\to\ 0}\frac{17 \cdot x}{tan(13 \cdot x)}$.
\zadStop
\rozwStart{Patryk Wirkus}{}
$$\lim\limits_{x\to\ 0}\frac{17 \cdot x}{tan(13 \cdot x)}=\lim\limits_{x\to\ 0}\frac{17 \cdot x \cdot cos(13 \cdot x)}{sin(13 \cdot x)}=\lim\limits_{x\to\ 0}\frac{17 \cdot cos(13 \cdot x)}{\frac{sin(13 \cdot x)}{x}}=\lim\limits_{x\to\ 0}\frac{17 \cdot cos(13 \cdot x)}{13 \cdot \frac{sin(13 \cdot x)}{13 \cdot x}} = \frac{17}{13}$$
\rozwStop
\odpStart
$\frac{17}{13}$
\odpStop
\testStart
A.$\frac{17}{13}$
B.$\infty$
C.$-\infty$
D.$0$
E.$-\frac{17}{13}$
F.$\frac{13}{17}$
G.$-\frac{13}{17}$
H.$13$
I.$17$
\testStop
\kluczStart
A
\kluczStop



\zadStart{Przykład z Wikieł P 4.3a moja wersja nr 308}


Obliczyć granicę funkcji $\lim\limits_{x\to\ 0}\frac{17 \cdot x}{tan(14 \cdot x)}$.
\zadStop
\rozwStart{Patryk Wirkus}{}
$$\lim\limits_{x\to\ 0}\frac{17 \cdot x}{tan(14 \cdot x)}=\lim\limits_{x\to\ 0}\frac{17 \cdot x \cdot cos(14 \cdot x)}{sin(14 \cdot x)}=\lim\limits_{x\to\ 0}\frac{17 \cdot cos(14 \cdot x)}{\frac{sin(14 \cdot x)}{x}}=\lim\limits_{x\to\ 0}\frac{17 \cdot cos(14 \cdot x)}{14 \cdot \frac{sin(14 \cdot x)}{14 \cdot x}} = \frac{17}{14}$$
\rozwStop
\odpStart
$\frac{17}{14}$
\odpStop
\testStart
A.$\frac{17}{14}$
B.$\infty$
C.$-\infty$
D.$0$
E.$-\frac{17}{14}$
F.$\frac{14}{17}$
G.$-\frac{14}{17}$
H.$14$
I.$17$
\testStop
\kluczStart
A
\kluczStop



\zadStart{Przykład z Wikieł P 4.3a moja wersja nr 309}


Obliczyć granicę funkcji $\lim\limits_{x\to\ 0}\frac{17 \cdot x}{tan(15 \cdot x)}$.
\zadStop
\rozwStart{Patryk Wirkus}{}
$$\lim\limits_{x\to\ 0}\frac{17 \cdot x}{tan(15 \cdot x)}=\lim\limits_{x\to\ 0}\frac{17 \cdot x \cdot cos(15 \cdot x)}{sin(15 \cdot x)}=\lim\limits_{x\to\ 0}\frac{17 \cdot cos(15 \cdot x)}{\frac{sin(15 \cdot x)}{x}}=\lim\limits_{x\to\ 0}\frac{17 \cdot cos(15 \cdot x)}{15 \cdot \frac{sin(15 \cdot x)}{15 \cdot x}} = \frac{17}{15}$$
\rozwStop
\odpStart
$\frac{17}{15}$
\odpStop
\testStart
A.$\frac{17}{15}$
B.$\infty$
C.$-\infty$
D.$0$
E.$-\frac{17}{15}$
F.$\frac{15}{17}$
G.$-\frac{15}{17}$
H.$15$
I.$17$
\testStop
\kluczStart
A
\kluczStop



\zadStart{Przykład z Wikieł P 4.3a moja wersja nr 310}


Obliczyć granicę funkcji $\lim\limits_{x\to\ 0}\frac{17 \cdot x}{tan(16 \cdot x)}$.
\zadStop
\rozwStart{Patryk Wirkus}{}
$$\lim\limits_{x\to\ 0}\frac{17 \cdot x}{tan(16 \cdot x)}=\lim\limits_{x\to\ 0}\frac{17 \cdot x \cdot cos(16 \cdot x)}{sin(16 \cdot x)}=\lim\limits_{x\to\ 0}\frac{17 \cdot cos(16 \cdot x)}{\frac{sin(16 \cdot x)}{x}}=\lim\limits_{x\to\ 0}\frac{17 \cdot cos(16 \cdot x)}{16 \cdot \frac{sin(16 \cdot x)}{16 \cdot x}} = \frac{17}{16}$$
\rozwStop
\odpStart
$\frac{17}{16}$
\odpStop
\testStart
A.$\frac{17}{16}$
B.$\infty$
C.$-\infty$
D.$0$
E.$-\frac{17}{16}$
F.$\frac{16}{17}$
G.$-\frac{16}{17}$
H.$16$
I.$17$
\testStop
\kluczStart
A
\kluczStop



\zadStart{Przykład z Wikieł P 4.3a moja wersja nr 311}


Obliczyć granicę funkcji $\lim\limits_{x\to\ 0}\frac{17 \cdot x}{tan(18 \cdot x)}$.
\zadStop
\rozwStart{Patryk Wirkus}{}
$$\lim\limits_{x\to\ 0}\frac{17 \cdot x}{tan(18 \cdot x)}=\lim\limits_{x\to\ 0}\frac{17 \cdot x \cdot cos(18 \cdot x)}{sin(18 \cdot x)}=\lim\limits_{x\to\ 0}\frac{17 \cdot cos(18 \cdot x)}{\frac{sin(18 \cdot x)}{x}}=\lim\limits_{x\to\ 0}\frac{17 \cdot cos(18 \cdot x)}{18 \cdot \frac{sin(18 \cdot x)}{18 \cdot x}} = \frac{17}{18}$$
\rozwStop
\odpStart
$\frac{17}{18}$
\odpStop
\testStart
A.$\frac{17}{18}$
B.$\infty$
C.$-\infty$
D.$0$
E.$-\frac{17}{18}$
F.$\frac{18}{17}$
G.$-\frac{18}{17}$
H.$18$
I.$17$
\testStop
\kluczStart
A
\kluczStop



\zadStart{Przykład z Wikieł P 4.3a moja wersja nr 312}


Obliczyć granicę funkcji $\lim\limits_{x\to\ 0}\frac{17 \cdot x}{tan(19 \cdot x)}$.
\zadStop
\rozwStart{Patryk Wirkus}{}
$$\lim\limits_{x\to\ 0}\frac{17 \cdot x}{tan(19 \cdot x)}=\lim\limits_{x\to\ 0}\frac{17 \cdot x \cdot cos(19 \cdot x)}{sin(19 \cdot x)}=\lim\limits_{x\to\ 0}\frac{17 \cdot cos(19 \cdot x)}{\frac{sin(19 \cdot x)}{x}}=\lim\limits_{x\to\ 0}\frac{17 \cdot cos(19 \cdot x)}{19 \cdot \frac{sin(19 \cdot x)}{19 \cdot x}} = \frac{17}{19}$$
\rozwStop
\odpStart
$\frac{17}{19}$
\odpStop
\testStart
A.$\frac{17}{19}$
B.$\infty$
C.$-\infty$
D.$0$
E.$-\frac{17}{19}$
F.$\frac{19}{17}$
G.$-\frac{19}{17}$
H.$19$
I.$17$
\testStop
\kluczStart
A
\kluczStop



\zadStart{Przykład z Wikieł P 4.3a moja wersja nr 313}


Obliczyć granicę funkcji $\lim\limits_{x\to\ 0}\frac{17 \cdot x}{tan(20 \cdot x)}$.
\zadStop
\rozwStart{Patryk Wirkus}{}
$$\lim\limits_{x\to\ 0}\frac{17 \cdot x}{tan(20 \cdot x)}=\lim\limits_{x\to\ 0}\frac{17 \cdot x \cdot cos(20 \cdot x)}{sin(20 \cdot x)}=\lim\limits_{x\to\ 0}\frac{17 \cdot cos(20 \cdot x)}{\frac{sin(20 \cdot x)}{x}}=\lim\limits_{x\to\ 0}\frac{17 \cdot cos(20 \cdot x)}{20 \cdot \frac{sin(20 \cdot x)}{20 \cdot x}} = \frac{17}{20}$$
\rozwStop
\odpStart
$\frac{17}{20}$
\odpStop
\testStart
A.$\frac{17}{20}$
B.$\infty$
C.$-\infty$
D.$0$
E.$-\frac{17}{20}$
F.$\frac{20}{17}$
G.$-\frac{20}{17}$
H.$20$
I.$17$
\testStop
\kluczStart
A
\kluczStop



\zadStart{Przykład z Wikieł P 4.3a moja wersja nr 314}


Obliczyć granicę funkcji $\lim\limits_{x\to\ 0}\frac{17 \cdot x}{tan(21 \cdot x)}$.
\zadStop
\rozwStart{Patryk Wirkus}{}
$$\lim\limits_{x\to\ 0}\frac{17 \cdot x}{tan(21 \cdot x)}=\lim\limits_{x\to\ 0}\frac{17 \cdot x \cdot cos(21 \cdot x)}{sin(21 \cdot x)}=\lim\limits_{x\to\ 0}\frac{17 \cdot cos(21 \cdot x)}{\frac{sin(21 \cdot x)}{x}}=\lim\limits_{x\to\ 0}\frac{17 \cdot cos(21 \cdot x)}{21 \cdot \frac{sin(21 \cdot x)}{21 \cdot x}} = \frac{17}{21}$$
\rozwStop
\odpStart
$\frac{17}{21}$
\odpStop
\testStart
A.$\frac{17}{21}$
B.$\infty$
C.$-\infty$
D.$0$
E.$-\frac{17}{21}$
F.$\frac{21}{17}$
G.$-\frac{21}{17}$
H.$21$
I.$17$
\testStop
\kluczStart
A
\kluczStop



\zadStart{Przykład z Wikieł P 4.3a moja wersja nr 315}


Obliczyć granicę funkcji $\lim\limits_{x\to\ 0}\frac{17 \cdot x}{tan(22 \cdot x)}$.
\zadStop
\rozwStart{Patryk Wirkus}{}
$$\lim\limits_{x\to\ 0}\frac{17 \cdot x}{tan(22 \cdot x)}=\lim\limits_{x\to\ 0}\frac{17 \cdot x \cdot cos(22 \cdot x)}{sin(22 \cdot x)}=\lim\limits_{x\to\ 0}\frac{17 \cdot cos(22 \cdot x)}{\frac{sin(22 \cdot x)}{x}}=\lim\limits_{x\to\ 0}\frac{17 \cdot cos(22 \cdot x)}{22 \cdot \frac{sin(22 \cdot x)}{22 \cdot x}} = \frac{17}{22}$$
\rozwStop
\odpStart
$\frac{17}{22}$
\odpStop
\testStart
A.$\frac{17}{22}$
B.$\infty$
C.$-\infty$
D.$0$
E.$-\frac{17}{22}$
F.$\frac{22}{17}$
G.$-\frac{22}{17}$
H.$22$
I.$17$
\testStop
\kluczStart
A
\kluczStop



\zadStart{Przykład z Wikieł P 4.3a moja wersja nr 316}


Obliczyć granicę funkcji $\lim\limits_{x\to\ 0}\frac{17 \cdot x}{tan(23 \cdot x)}$.
\zadStop
\rozwStart{Patryk Wirkus}{}
$$\lim\limits_{x\to\ 0}\frac{17 \cdot x}{tan(23 \cdot x)}=\lim\limits_{x\to\ 0}\frac{17 \cdot x \cdot cos(23 \cdot x)}{sin(23 \cdot x)}=\lim\limits_{x\to\ 0}\frac{17 \cdot cos(23 \cdot x)}{\frac{sin(23 \cdot x)}{x}}=\lim\limits_{x\to\ 0}\frac{17 \cdot cos(23 \cdot x)}{23 \cdot \frac{sin(23 \cdot x)}{23 \cdot x}} = \frac{17}{23}$$
\rozwStop
\odpStart
$\frac{17}{23}$
\odpStop
\testStart
A.$\frac{17}{23}$
B.$\infty$
C.$-\infty$
D.$0$
E.$-\frac{17}{23}$
F.$\frac{23}{17}$
G.$-\frac{23}{17}$
H.$23$
I.$17$
\testStop
\kluczStart
A
\kluczStop



\zadStart{Przykład z Wikieł P 4.3a moja wersja nr 317}


Obliczyć granicę funkcji $\lim\limits_{x\to\ 0}\frac{17 \cdot x}{tan(24 \cdot x)}$.
\zadStop
\rozwStart{Patryk Wirkus}{}
$$\lim\limits_{x\to\ 0}\frac{17 \cdot x}{tan(24 \cdot x)}=\lim\limits_{x\to\ 0}\frac{17 \cdot x \cdot cos(24 \cdot x)}{sin(24 \cdot x)}=\lim\limits_{x\to\ 0}\frac{17 \cdot cos(24 \cdot x)}{\frac{sin(24 \cdot x)}{x}}=\lim\limits_{x\to\ 0}\frac{17 \cdot cos(24 \cdot x)}{24 \cdot \frac{sin(24 \cdot x)}{24 \cdot x}} = \frac{17}{24}$$
\rozwStop
\odpStart
$\frac{17}{24}$
\odpStop
\testStart
A.$\frac{17}{24}$
B.$\infty$
C.$-\infty$
D.$0$
E.$-\frac{17}{24}$
F.$\frac{24}{17}$
G.$-\frac{24}{17}$
H.$24$
I.$17$
\testStop
\kluczStart
A
\kluczStop



\zadStart{Przykład z Wikieł P 4.3a moja wersja nr 318}


Obliczyć granicę funkcji $\lim\limits_{x\to\ 0}\frac{17 \cdot x}{tan(25 \cdot x)}$.
\zadStop
\rozwStart{Patryk Wirkus}{}
$$\lim\limits_{x\to\ 0}\frac{17 \cdot x}{tan(25 \cdot x)}=\lim\limits_{x\to\ 0}\frac{17 \cdot x \cdot cos(25 \cdot x)}{sin(25 \cdot x)}=\lim\limits_{x\to\ 0}\frac{17 \cdot cos(25 \cdot x)}{\frac{sin(25 \cdot x)}{x}}=\lim\limits_{x\to\ 0}\frac{17 \cdot cos(25 \cdot x)}{25 \cdot \frac{sin(25 \cdot x)}{25 \cdot x}} = \frac{17}{25}$$
\rozwStop
\odpStart
$\frac{17}{25}$
\odpStop
\testStart
A.$\frac{17}{25}$
B.$\infty$
C.$-\infty$
D.$0$
E.$-\frac{17}{25}$
F.$\frac{25}{17}$
G.$-\frac{25}{17}$
H.$25$
I.$17$
\testStop
\kluczStart
A
\kluczStop



\zadStart{Przykład z Wikieł P 4.3a moja wersja nr 319}


Obliczyć granicę funkcji $\lim\limits_{x\to\ 0}\frac{17 \cdot x}{tan(26 \cdot x)}$.
\zadStop
\rozwStart{Patryk Wirkus}{}
$$\lim\limits_{x\to\ 0}\frac{17 \cdot x}{tan(26 \cdot x)}=\lim\limits_{x\to\ 0}\frac{17 \cdot x \cdot cos(26 \cdot x)}{sin(26 \cdot x)}=\lim\limits_{x\to\ 0}\frac{17 \cdot cos(26 \cdot x)}{\frac{sin(26 \cdot x)}{x}}=\lim\limits_{x\to\ 0}\frac{17 \cdot cos(26 \cdot x)}{26 \cdot \frac{sin(26 \cdot x)}{26 \cdot x}} = \frac{17}{26}$$
\rozwStop
\odpStart
$\frac{17}{26}$
\odpStop
\testStart
A.$\frac{17}{26}$
B.$\infty$
C.$-\infty$
D.$0$
E.$-\frac{17}{26}$
F.$\frac{26}{17}$
G.$-\frac{26}{17}$
H.$26$
I.$17$
\testStop
\kluczStart
A
\kluczStop



\zadStart{Przykład z Wikieł P 4.3a moja wersja nr 320}


Obliczyć granicę funkcji $\lim\limits_{x\to\ 0}\frac{17 \cdot x}{tan(27 \cdot x)}$.
\zadStop
\rozwStart{Patryk Wirkus}{}
$$\lim\limits_{x\to\ 0}\frac{17 \cdot x}{tan(27 \cdot x)}=\lim\limits_{x\to\ 0}\frac{17 \cdot x \cdot cos(27 \cdot x)}{sin(27 \cdot x)}=\lim\limits_{x\to\ 0}\frac{17 \cdot cos(27 \cdot x)}{\frac{sin(27 \cdot x)}{x}}=\lim\limits_{x\to\ 0}\frac{17 \cdot cos(27 \cdot x)}{27 \cdot \frac{sin(27 \cdot x)}{27 \cdot x}} = \frac{17}{27}$$
\rozwStop
\odpStart
$\frac{17}{27}$
\odpStop
\testStart
A.$\frac{17}{27}$
B.$\infty$
C.$-\infty$
D.$0$
E.$-\frac{17}{27}$
F.$\frac{27}{17}$
G.$-\frac{27}{17}$
H.$27$
I.$17$
\testStop
\kluczStart
A
\kluczStop



\zadStart{Przykład z Wikieł P 4.3a moja wersja nr 321}


Obliczyć granicę funkcji $\lim\limits_{x\to\ 0}\frac{17 \cdot x}{tan(28 \cdot x)}$.
\zadStop
\rozwStart{Patryk Wirkus}{}
$$\lim\limits_{x\to\ 0}\frac{17 \cdot x}{tan(28 \cdot x)}=\lim\limits_{x\to\ 0}\frac{17 \cdot x \cdot cos(28 \cdot x)}{sin(28 \cdot x)}=\lim\limits_{x\to\ 0}\frac{17 \cdot cos(28 \cdot x)}{\frac{sin(28 \cdot x)}{x}}=\lim\limits_{x\to\ 0}\frac{17 \cdot cos(28 \cdot x)}{28 \cdot \frac{sin(28 \cdot x)}{28 \cdot x}} = \frac{17}{28}$$
\rozwStop
\odpStart
$\frac{17}{28}$
\odpStop
\testStart
A.$\frac{17}{28}$
B.$\infty$
C.$-\infty$
D.$0$
E.$-\frac{17}{28}$
F.$\frac{28}{17}$
G.$-\frac{28}{17}$
H.$28$
I.$17$
\testStop
\kluczStart
A
\kluczStop



\zadStart{Przykład z Wikieł P 4.3a moja wersja nr 322}


Obliczyć granicę funkcji $\lim\limits_{x\to\ 0}\frac{17 \cdot x}{tan(29 \cdot x)}$.
\zadStop
\rozwStart{Patryk Wirkus}{}
$$\lim\limits_{x\to\ 0}\frac{17 \cdot x}{tan(29 \cdot x)}=\lim\limits_{x\to\ 0}\frac{17 \cdot x \cdot cos(29 \cdot x)}{sin(29 \cdot x)}=\lim\limits_{x\to\ 0}\frac{17 \cdot cos(29 \cdot x)}{\frac{sin(29 \cdot x)}{x}}=\lim\limits_{x\to\ 0}\frac{17 \cdot cos(29 \cdot x)}{29 \cdot \frac{sin(29 \cdot x)}{29 \cdot x}} = \frac{17}{29}$$
\rozwStop
\odpStart
$\frac{17}{29}$
\odpStop
\testStart
A.$\frac{17}{29}$
B.$\infty$
C.$-\infty$
D.$0$
E.$-\frac{17}{29}$
F.$\frac{29}{17}$
G.$-\frac{29}{17}$
H.$29$
I.$17$
\testStop
\kluczStart
A
\kluczStop



\zadStart{Przykład z Wikieł P 4.3a moja wersja nr 323}


Obliczyć granicę funkcji $\lim\limits_{x\to\ 0}\frac{17 \cdot x}{tan(30 \cdot x)}$.
\zadStop
\rozwStart{Patryk Wirkus}{}
$$\lim\limits_{x\to\ 0}\frac{17 \cdot x}{tan(30 \cdot x)}=\lim\limits_{x\to\ 0}\frac{17 \cdot x \cdot cos(30 \cdot x)}{sin(30 \cdot x)}=\lim\limits_{x\to\ 0}\frac{17 \cdot cos(30 \cdot x)}{\frac{sin(30 \cdot x)}{x}}=\lim\limits_{x\to\ 0}\frac{17 \cdot cos(30 \cdot x)}{30 \cdot \frac{sin(30 \cdot x)}{30 \cdot x}} = \frac{17}{30}$$
\rozwStop
\odpStart
$\frac{17}{30}$
\odpStop
\testStart
A.$\frac{17}{30}$
B.$\infty$
C.$-\infty$
D.$0$
E.$-\frac{17}{30}$
F.$\frac{30}{17}$
G.$-\frac{30}{17}$
H.$30$
I.$17$
\testStop
\kluczStart
A
\kluczStop



\zadStart{Przykład z Wikieł P 4.3a moja wersja nr 324}


Obliczyć granicę funkcji $\lim\limits_{x\to\ 0}\frac{17 \cdot x}{tan(31 \cdot x)}$.
\zadStop
\rozwStart{Patryk Wirkus}{}
$$\lim\limits_{x\to\ 0}\frac{17 \cdot x}{tan(31 \cdot x)}=\lim\limits_{x\to\ 0}\frac{17 \cdot x \cdot cos(31 \cdot x)}{sin(31 \cdot x)}=\lim\limits_{x\to\ 0}\frac{17 \cdot cos(31 \cdot x)}{\frac{sin(31 \cdot x)}{x}}=\lim\limits_{x\to\ 0}\frac{17 \cdot cos(31 \cdot x)}{31 \cdot \frac{sin(31 \cdot x)}{31 \cdot x}} = \frac{17}{31}$$
\rozwStop
\odpStart
$\frac{17}{31}$
\odpStop
\testStart
A.$\frac{17}{31}$
B.$\infty$
C.$-\infty$
D.$0$
E.$-\frac{17}{31}$
F.$\frac{31}{17}$
G.$-\frac{31}{17}$
H.$31$
I.$17$
\testStop
\kluczStart
A
\kluczStop



\zadStart{Przykład z Wikieł P 4.3a moja wersja nr 325}


Obliczyć granicę funkcji $\lim\limits_{x\to\ 0}\frac{17 \cdot x}{tan(32 \cdot x)}$.
\zadStop
\rozwStart{Patryk Wirkus}{}
$$\lim\limits_{x\to\ 0}\frac{17 \cdot x}{tan(32 \cdot x)}=\lim\limits_{x\to\ 0}\frac{17 \cdot x \cdot cos(32 \cdot x)}{sin(32 \cdot x)}=\lim\limits_{x\to\ 0}\frac{17 \cdot cos(32 \cdot x)}{\frac{sin(32 \cdot x)}{x}}=\lim\limits_{x\to\ 0}\frac{17 \cdot cos(32 \cdot x)}{32 \cdot \frac{sin(32 \cdot x)}{32 \cdot x}} = \frac{17}{32}$$
\rozwStop
\odpStart
$\frac{17}{32}$
\odpStop
\testStart
A.$\frac{17}{32}$
B.$\infty$
C.$-\infty$
D.$0$
E.$-\frac{17}{32}$
F.$\frac{32}{17}$
G.$-\frac{32}{17}$
H.$32$
I.$17$
\testStop
\kluczStart
A
\kluczStop



\zadStart{Przykład z Wikieł P 4.3a moja wersja nr 326}


Obliczyć granicę funkcji $\lim\limits_{x\to\ 0}\frac{17 \cdot x}{tan(33 \cdot x)}$.
\zadStop
\rozwStart{Patryk Wirkus}{}
$$\lim\limits_{x\to\ 0}\frac{17 \cdot x}{tan(33 \cdot x)}=\lim\limits_{x\to\ 0}\frac{17 \cdot x \cdot cos(33 \cdot x)}{sin(33 \cdot x)}=\lim\limits_{x\to\ 0}\frac{17 \cdot cos(33 \cdot x)}{\frac{sin(33 \cdot x)}{x}}=\lim\limits_{x\to\ 0}\frac{17 \cdot cos(33 \cdot x)}{33 \cdot \frac{sin(33 \cdot x)}{33 \cdot x}} = \frac{17}{33}$$
\rozwStop
\odpStart
$\frac{17}{33}$
\odpStop
\testStart
A.$\frac{17}{33}$
B.$\infty$
C.$-\infty$
D.$0$
E.$-\frac{17}{33}$
F.$\frac{33}{17}$
G.$-\frac{33}{17}$
H.$33$
I.$17$
\testStop
\kluczStart
A
\kluczStop



\zadStart{Przykład z Wikieł P 4.3a moja wersja nr 327}


Obliczyć granicę funkcji $\lim\limits_{x\to\ 0}\frac{17 \cdot x}{tan(35 \cdot x)}$.
\zadStop
\rozwStart{Patryk Wirkus}{}
$$\lim\limits_{x\to\ 0}\frac{17 \cdot x}{tan(35 \cdot x)}=\lim\limits_{x\to\ 0}\frac{17 \cdot x \cdot cos(35 \cdot x)}{sin(35 \cdot x)}=\lim\limits_{x\to\ 0}\frac{17 \cdot cos(35 \cdot x)}{\frac{sin(35 \cdot x)}{x}}=\lim\limits_{x\to\ 0}\frac{17 \cdot cos(35 \cdot x)}{35 \cdot \frac{sin(35 \cdot x)}{35 \cdot x}} = \frac{17}{35}$$
\rozwStop
\odpStart
$\frac{17}{35}$
\odpStop
\testStart
A.$\frac{17}{35}$
B.$\infty$
C.$-\infty$
D.$0$
E.$-\frac{17}{35}$
F.$\frac{35}{17}$
G.$-\frac{35}{17}$
H.$35$
I.$17$
\testStop
\kluczStart
A
\kluczStop



\zadStart{Przykład z Wikieł P 4.3a moja wersja nr 328}


Obliczyć granicę funkcji $\lim\limits_{x\to\ 0}\frac{17 \cdot x}{tan(36 \cdot x)}$.
\zadStop
\rozwStart{Patryk Wirkus}{}
$$\lim\limits_{x\to\ 0}\frac{17 \cdot x}{tan(36 \cdot x)}=\lim\limits_{x\to\ 0}\frac{17 \cdot x \cdot cos(36 \cdot x)}{sin(36 \cdot x)}=\lim\limits_{x\to\ 0}\frac{17 \cdot cos(36 \cdot x)}{\frac{sin(36 \cdot x)}{x}}=\lim\limits_{x\to\ 0}\frac{17 \cdot cos(36 \cdot x)}{36 \cdot \frac{sin(36 \cdot x)}{36 \cdot x}} = \frac{17}{36}$$
\rozwStop
\odpStart
$\frac{17}{36}$
\odpStop
\testStart
A.$\frac{17}{36}$
B.$\infty$
C.$-\infty$
D.$0$
E.$-\frac{17}{36}$
F.$\frac{36}{17}$
G.$-\frac{36}{17}$
H.$36$
I.$17$
\testStop
\kluczStart
A
\kluczStop



\zadStart{Przykład z Wikieł P 4.3a moja wersja nr 329}


Obliczyć granicę funkcji $\lim\limits_{x\to\ 0}\frac{17 \cdot x}{tan(37 \cdot x)}$.
\zadStop
\rozwStart{Patryk Wirkus}{}
$$\lim\limits_{x\to\ 0}\frac{17 \cdot x}{tan(37 \cdot x)}=\lim\limits_{x\to\ 0}\frac{17 \cdot x \cdot cos(37 \cdot x)}{sin(37 \cdot x)}=\lim\limits_{x\to\ 0}\frac{17 \cdot cos(37 \cdot x)}{\frac{sin(37 \cdot x)}{x}}=\lim\limits_{x\to\ 0}\frac{17 \cdot cos(37 \cdot x)}{37 \cdot \frac{sin(37 \cdot x)}{37 \cdot x}} = \frac{17}{37}$$
\rozwStop
\odpStart
$\frac{17}{37}$
\odpStop
\testStart
A.$\frac{17}{37}$
B.$\infty$
C.$-\infty$
D.$0$
E.$-\frac{17}{37}$
F.$\frac{37}{17}$
G.$-\frac{37}{17}$
H.$37$
I.$17$
\testStop
\kluczStart
A
\kluczStop



\zadStart{Przykład z Wikieł P 4.3a moja wersja nr 330}


Obliczyć granicę funkcji $\lim\limits_{x\to\ 0}\frac{17 \cdot x}{tan(38 \cdot x)}$.
\zadStop
\rozwStart{Patryk Wirkus}{}
$$\lim\limits_{x\to\ 0}\frac{17 \cdot x}{tan(38 \cdot x)}=\lim\limits_{x\to\ 0}\frac{17 \cdot x \cdot cos(38 \cdot x)}{sin(38 \cdot x)}=\lim\limits_{x\to\ 0}\frac{17 \cdot cos(38 \cdot x)}{\frac{sin(38 \cdot x)}{x}}=\lim\limits_{x\to\ 0}\frac{17 \cdot cos(38 \cdot x)}{38 \cdot \frac{sin(38 \cdot x)}{38 \cdot x}} = \frac{17}{38}$$
\rozwStop
\odpStart
$\frac{17}{38}$
\odpStop
\testStart
A.$\frac{17}{38}$
B.$\infty$
C.$-\infty$
D.$0$
E.$-\frac{17}{38}$
F.$\frac{38}{17}$
G.$-\frac{38}{17}$
H.$38$
I.$17$
\testStop
\kluczStart
A
\kluczStop



\zadStart{Przykład z Wikieł P 4.3a moja wersja nr 331}


Obliczyć granicę funkcji $\lim\limits_{x\to\ 0}\frac{17 \cdot x}{tan(39 \cdot x)}$.
\zadStop
\rozwStart{Patryk Wirkus}{}
$$\lim\limits_{x\to\ 0}\frac{17 \cdot x}{tan(39 \cdot x)}=\lim\limits_{x\to\ 0}\frac{17 \cdot x \cdot cos(39 \cdot x)}{sin(39 \cdot x)}=\lim\limits_{x\to\ 0}\frac{17 \cdot cos(39 \cdot x)}{\frac{sin(39 \cdot x)}{x}}=\lim\limits_{x\to\ 0}\frac{17 \cdot cos(39 \cdot x)}{39 \cdot \frac{sin(39 \cdot x)}{39 \cdot x}} = \frac{17}{39}$$
\rozwStop
\odpStart
$\frac{17}{39}$
\odpStop
\testStart
A.$\frac{17}{39}$
B.$\infty$
C.$-\infty$
D.$0$
E.$-\frac{17}{39}$
F.$\frac{39}{17}$
G.$-\frac{39}{17}$
H.$39$
I.$17$
\testStop
\kluczStart
A
\kluczStop



\zadStart{Przykład z Wikieł P 4.3a moja wersja nr 332}


Obliczyć granicę funkcji $\lim\limits_{x\to\ 0}\frac{17 \cdot x}{tan(40 \cdot x)}$.
\zadStop
\rozwStart{Patryk Wirkus}{}
$$\lim\limits_{x\to\ 0}\frac{17 \cdot x}{tan(40 \cdot x)}=\lim\limits_{x\to\ 0}\frac{17 \cdot x \cdot cos(40 \cdot x)}{sin(40 \cdot x)}=\lim\limits_{x\to\ 0}\frac{17 \cdot cos(40 \cdot x)}{\frac{sin(40 \cdot x)}{x}}=\lim\limits_{x\to\ 0}\frac{17 \cdot cos(40 \cdot x)}{40 \cdot \frac{sin(40 \cdot x)}{40 \cdot x}} = \frac{17}{40}$$
\rozwStop
\odpStart
$\frac{17}{40}$
\odpStop
\testStart
A.$\frac{17}{40}$
B.$\infty$
C.$-\infty$
D.$0$
E.$-\frac{17}{40}$
F.$\frac{40}{17}$
G.$-\frac{40}{17}$
H.$40$
I.$17$
\testStop
\kluczStart
A
\kluczStop



\zadStart{Przykład z Wikieł P 4.3a moja wersja nr 333}


Obliczyć granicę funkcji $\lim\limits_{x\to\ 0}\frac{18 \cdot x}{tan(5 \cdot x)}$.
\zadStop
\rozwStart{Patryk Wirkus}{}
$$\lim\limits_{x\to\ 0}\frac{18 \cdot x}{tan(5 \cdot x)}=\lim\limits_{x\to\ 0}\frac{18 \cdot x \cdot cos(5 \cdot x)}{sin(5 \cdot x)}=\lim\limits_{x\to\ 0}\frac{18 \cdot cos(5 \cdot x)}{\frac{sin(5 \cdot x)}{x}}=\lim\limits_{x\to\ 0}\frac{18 \cdot cos(5 \cdot x)}{5 \cdot \frac{sin(5 \cdot x)}{5 \cdot x}} = \frac{18}{5}$$
\rozwStop
\odpStart
$\frac{18}{5}$
\odpStop
\testStart
A.$\frac{18}{5}$
B.$\infty$
C.$-\infty$
D.$0$
E.$-\frac{18}{5}$
F.$\frac{5}{18}$
G.$-\frac{5}{18}$
H.$5$
I.$18$
\testStop
\kluczStart
A
\kluczStop



\zadStart{Przykład z Wikieł P 4.3a moja wersja nr 334}


Obliczyć granicę funkcji $\lim\limits_{x\to\ 0}\frac{18 \cdot x}{tan(7 \cdot x)}$.
\zadStop
\rozwStart{Patryk Wirkus}{}
$$\lim\limits_{x\to\ 0}\frac{18 \cdot x}{tan(7 \cdot x)}=\lim\limits_{x\to\ 0}\frac{18 \cdot x \cdot cos(7 \cdot x)}{sin(7 \cdot x)}=\lim\limits_{x\to\ 0}\frac{18 \cdot cos(7 \cdot x)}{\frac{sin(7 \cdot x)}{x}}=\lim\limits_{x\to\ 0}\frac{18 \cdot cos(7 \cdot x)}{7 \cdot \frac{sin(7 \cdot x)}{7 \cdot x}} = \frac{18}{7}$$
\rozwStop
\odpStart
$\frac{18}{7}$
\odpStop
\testStart
A.$\frac{18}{7}$
B.$\infty$
C.$-\infty$
D.$0$
E.$-\frac{18}{7}$
F.$\frac{7}{18}$
G.$-\frac{7}{18}$
H.$7$
I.$18$
\testStop
\kluczStart
A
\kluczStop



\zadStart{Przykład z Wikieł P 4.3a moja wersja nr 335}


Obliczyć granicę funkcji $\lim\limits_{x\to\ 0}\frac{18 \cdot x}{tan(11 \cdot x)}$.
\zadStop
\rozwStart{Patryk Wirkus}{}
$$\lim\limits_{x\to\ 0}\frac{18 \cdot x}{tan(11 \cdot x)}=\lim\limits_{x\to\ 0}\frac{18 \cdot x \cdot cos(11 \cdot x)}{sin(11 \cdot x)}=\lim\limits_{x\to\ 0}\frac{18 \cdot cos(11 \cdot x)}{\frac{sin(11 \cdot x)}{x}}=\lim\limits_{x\to\ 0}\frac{18 \cdot cos(11 \cdot x)}{11 \cdot \frac{sin(11 \cdot x)}{11 \cdot x}} = \frac{18}{11}$$
\rozwStop
\odpStart
$\frac{18}{11}$
\odpStop
\testStart
A.$\frac{18}{11}$
B.$\infty$
C.$-\infty$
D.$0$
E.$-\frac{18}{11}$
F.$\frac{11}{18}$
G.$-\frac{11}{18}$
H.$11$
I.$18$
\testStop
\kluczStart
A
\kluczStop



\zadStart{Przykład z Wikieł P 4.3a moja wersja nr 336}


Obliczyć granicę funkcji $\lim\limits_{x\to\ 0}\frac{18 \cdot x}{tan(13 \cdot x)}$.
\zadStop
\rozwStart{Patryk Wirkus}{}
$$\lim\limits_{x\to\ 0}\frac{18 \cdot x}{tan(13 \cdot x)}=\lim\limits_{x\to\ 0}\frac{18 \cdot x \cdot cos(13 \cdot x)}{sin(13 \cdot x)}=\lim\limits_{x\to\ 0}\frac{18 \cdot cos(13 \cdot x)}{\frac{sin(13 \cdot x)}{x}}=\lim\limits_{x\to\ 0}\frac{18 \cdot cos(13 \cdot x)}{13 \cdot \frac{sin(13 \cdot x)}{13 \cdot x}} = \frac{18}{13}$$
\rozwStop
\odpStart
$\frac{18}{13}$
\odpStop
\testStart
A.$\frac{18}{13}$
B.$\infty$
C.$-\infty$
D.$0$
E.$-\frac{18}{13}$
F.$\frac{13}{18}$
G.$-\frac{13}{18}$
H.$13$
I.$18$
\testStop
\kluczStart
A
\kluczStop



\zadStart{Przykład z Wikieł P 4.3a moja wersja nr 337}


Obliczyć granicę funkcji $\lim\limits_{x\to\ 0}\frac{18 \cdot x}{tan(17 \cdot x)}$.
\zadStop
\rozwStart{Patryk Wirkus}{}
$$\lim\limits_{x\to\ 0}\frac{18 \cdot x}{tan(17 \cdot x)}=\lim\limits_{x\to\ 0}\frac{18 \cdot x \cdot cos(17 \cdot x)}{sin(17 \cdot x)}=\lim\limits_{x\to\ 0}\frac{18 \cdot cos(17 \cdot x)}{\frac{sin(17 \cdot x)}{x}}=\lim\limits_{x\to\ 0}\frac{18 \cdot cos(17 \cdot x)}{17 \cdot \frac{sin(17 \cdot x)}{17 \cdot x}} = \frac{18}{17}$$
\rozwStop
\odpStart
$\frac{18}{17}$
\odpStop
\testStart
A.$\frac{18}{17}$
B.$\infty$
C.$-\infty$
D.$0$
E.$-\frac{18}{17}$
F.$\frac{17}{18}$
G.$-\frac{17}{18}$
H.$17$
I.$18$
\testStop
\kluczStart
A
\kluczStop



\zadStart{Przykład z Wikieł P 4.3a moja wersja nr 338}


Obliczyć granicę funkcji $\lim\limits_{x\to\ 0}\frac{18 \cdot x}{tan(19 \cdot x)}$.
\zadStop
\rozwStart{Patryk Wirkus}{}
$$\lim\limits_{x\to\ 0}\frac{18 \cdot x}{tan(19 \cdot x)}=\lim\limits_{x\to\ 0}\frac{18 \cdot x \cdot cos(19 \cdot x)}{sin(19 \cdot x)}=\lim\limits_{x\to\ 0}\frac{18 \cdot cos(19 \cdot x)}{\frac{sin(19 \cdot x)}{x}}=\lim\limits_{x\to\ 0}\frac{18 \cdot cos(19 \cdot x)}{19 \cdot \frac{sin(19 \cdot x)}{19 \cdot x}} = \frac{18}{19}$$
\rozwStop
\odpStart
$\frac{18}{19}$
\odpStop
\testStart
A.$\frac{18}{19}$
B.$\infty$
C.$-\infty$
D.$0$
E.$-\frac{18}{19}$
F.$\frac{19}{18}$
G.$-\frac{19}{18}$
H.$19$
I.$18$
\testStop
\kluczStart
A
\kluczStop



\zadStart{Przykład z Wikieł P 4.3a moja wersja nr 339}


Obliczyć granicę funkcji $\lim\limits_{x\to\ 0}\frac{18 \cdot x}{tan(23 \cdot x)}$.
\zadStop
\rozwStart{Patryk Wirkus}{}
$$\lim\limits_{x\to\ 0}\frac{18 \cdot x}{tan(23 \cdot x)}=\lim\limits_{x\to\ 0}\frac{18 \cdot x \cdot cos(23 \cdot x)}{sin(23 \cdot x)}=\lim\limits_{x\to\ 0}\frac{18 \cdot cos(23 \cdot x)}{\frac{sin(23 \cdot x)}{x}}=\lim\limits_{x\to\ 0}\frac{18 \cdot cos(23 \cdot x)}{23 \cdot \frac{sin(23 \cdot x)}{23 \cdot x}} = \frac{18}{23}$$
\rozwStop
\odpStart
$\frac{18}{23}$
\odpStop
\testStart
A.$\frac{18}{23}$
B.$\infty$
C.$-\infty$
D.$0$
E.$-\frac{18}{23}$
F.$\frac{23}{18}$
G.$-\frac{23}{18}$
H.$23$
I.$18$
\testStop
\kluczStart
A
\kluczStop



\zadStart{Przykład z Wikieł P 4.3a moja wersja nr 340}


Obliczyć granicę funkcji $\lim\limits_{x\to\ 0}\frac{18 \cdot x}{tan(25 \cdot x)}$.
\zadStop
\rozwStart{Patryk Wirkus}{}
$$\lim\limits_{x\to\ 0}\frac{18 \cdot x}{tan(25 \cdot x)}=\lim\limits_{x\to\ 0}\frac{18 \cdot x \cdot cos(25 \cdot x)}{sin(25 \cdot x)}=\lim\limits_{x\to\ 0}\frac{18 \cdot cos(25 \cdot x)}{\frac{sin(25 \cdot x)}{x}}=\lim\limits_{x\to\ 0}\frac{18 \cdot cos(25 \cdot x)}{25 \cdot \frac{sin(25 \cdot x)}{25 \cdot x}} = \frac{18}{25}$$
\rozwStop
\odpStart
$\frac{18}{25}$
\odpStop
\testStart
A.$\frac{18}{25}$
B.$\infty$
C.$-\infty$
D.$0$
E.$-\frac{18}{25}$
F.$\frac{25}{18}$
G.$-\frac{25}{18}$
H.$25$
I.$18$
\testStop
\kluczStart
A
\kluczStop



\zadStart{Przykład z Wikieł P 4.3a moja wersja nr 341}


Obliczyć granicę funkcji $\lim\limits_{x\to\ 0}\frac{18 \cdot x}{tan(29 \cdot x)}$.
\zadStop
\rozwStart{Patryk Wirkus}{}
$$\lim\limits_{x\to\ 0}\frac{18 \cdot x}{tan(29 \cdot x)}=\lim\limits_{x\to\ 0}\frac{18 \cdot x \cdot cos(29 \cdot x)}{sin(29 \cdot x)}=\lim\limits_{x\to\ 0}\frac{18 \cdot cos(29 \cdot x)}{\frac{sin(29 \cdot x)}{x}}=\lim\limits_{x\to\ 0}\frac{18 \cdot cos(29 \cdot x)}{29 \cdot \frac{sin(29 \cdot x)}{29 \cdot x}} = \frac{18}{29}$$
\rozwStop
\odpStart
$\frac{18}{29}$
\odpStop
\testStart
A.$\frac{18}{29}$
B.$\infty$
C.$-\infty$
D.$0$
E.$-\frac{18}{29}$
F.$\frac{29}{18}$
G.$-\frac{29}{18}$
H.$29$
I.$18$
\testStop
\kluczStart
A
\kluczStop



\zadStart{Przykład z Wikieł P 4.3a moja wersja nr 342}


Obliczyć granicę funkcji $\lim\limits_{x\to\ 0}\frac{18 \cdot x}{tan(31 \cdot x)}$.
\zadStop
\rozwStart{Patryk Wirkus}{}
$$\lim\limits_{x\to\ 0}\frac{18 \cdot x}{tan(31 \cdot x)}=\lim\limits_{x\to\ 0}\frac{18 \cdot x \cdot cos(31 \cdot x)}{sin(31 \cdot x)}=\lim\limits_{x\to\ 0}\frac{18 \cdot cos(31 \cdot x)}{\frac{sin(31 \cdot x)}{x}}=\lim\limits_{x\to\ 0}\frac{18 \cdot cos(31 \cdot x)}{31 \cdot \frac{sin(31 \cdot x)}{31 \cdot x}} = \frac{18}{31}$$
\rozwStop
\odpStart
$\frac{18}{31}$
\odpStop
\testStart
A.$\frac{18}{31}$
B.$\infty$
C.$-\infty$
D.$0$
E.$-\frac{18}{31}$
F.$\frac{31}{18}$
G.$-\frac{31}{18}$
H.$31$
I.$18$
\testStop
\kluczStart
A
\kluczStop



\zadStart{Przykład z Wikieł P 4.3a moja wersja nr 343}


Obliczyć granicę funkcji $\lim\limits_{x\to\ 0}\frac{18 \cdot x}{tan(35 \cdot x)}$.
\zadStop
\rozwStart{Patryk Wirkus}{}
$$\lim\limits_{x\to\ 0}\frac{18 \cdot x}{tan(35 \cdot x)}=\lim\limits_{x\to\ 0}\frac{18 \cdot x \cdot cos(35 \cdot x)}{sin(35 \cdot x)}=\lim\limits_{x\to\ 0}\frac{18 \cdot cos(35 \cdot x)}{\frac{sin(35 \cdot x)}{x}}=\lim\limits_{x\to\ 0}\frac{18 \cdot cos(35 \cdot x)}{35 \cdot \frac{sin(35 \cdot x)}{35 \cdot x}} = \frac{18}{35}$$
\rozwStop
\odpStart
$\frac{18}{35}$
\odpStop
\testStart
A.$\frac{18}{35}$
B.$\infty$
C.$-\infty$
D.$0$
E.$-\frac{18}{35}$
F.$\frac{35}{18}$
G.$-\frac{35}{18}$
H.$35$
I.$18$
\testStop
\kluczStart
A
\kluczStop



\zadStart{Przykład z Wikieł P 4.3a moja wersja nr 344}


Obliczyć granicę funkcji $\lim\limits_{x\to\ 0}\frac{18 \cdot x}{tan(37 \cdot x)}$.
\zadStop
\rozwStart{Patryk Wirkus}{}
$$\lim\limits_{x\to\ 0}\frac{18 \cdot x}{tan(37 \cdot x)}=\lim\limits_{x\to\ 0}\frac{18 \cdot x \cdot cos(37 \cdot x)}{sin(37 \cdot x)}=\lim\limits_{x\to\ 0}\frac{18 \cdot cos(37 \cdot x)}{\frac{sin(37 \cdot x)}{x}}=\lim\limits_{x\to\ 0}\frac{18 \cdot cos(37 \cdot x)}{37 \cdot \frac{sin(37 \cdot x)}{37 \cdot x}} = \frac{18}{37}$$
\rozwStop
\odpStart
$\frac{18}{37}$
\odpStop
\testStart
A.$\frac{18}{37}$
B.$\infty$
C.$-\infty$
D.$0$
E.$-\frac{18}{37}$
F.$\frac{37}{18}$
G.$-\frac{37}{18}$
H.$37$
I.$18$
\testStop
\kluczStart
A
\kluczStop



\zadStart{Przykład z Wikieł P 4.3a moja wersja nr 345}


Obliczyć granicę funkcji $\lim\limits_{x\to\ 0}\frac{19 \cdot x}{tan(2 \cdot x)}$.
\zadStop
\rozwStart{Patryk Wirkus}{}
$$\lim\limits_{x\to\ 0}\frac{19 \cdot x}{tan(2 \cdot x)}=\lim\limits_{x\to\ 0}\frac{19 \cdot x \cdot cos(2 \cdot x)}{sin(2 \cdot x)}=\lim\limits_{x\to\ 0}\frac{19 \cdot cos(2 \cdot x)}{\frac{sin(2 \cdot x)}{x}}=\lim\limits_{x\to\ 0}\frac{19 \cdot cos(2 \cdot x)}{2 \cdot \frac{sin(2 \cdot x)}{2 \cdot x}} = \frac{19}{2}$$
\rozwStop
\odpStart
$\frac{19}{2}$
\odpStop
\testStart
A.$\frac{19}{2}$
B.$\infty$
C.$-\infty$
D.$0$
E.$-\frac{19}{2}$
F.$\frac{2}{19}$
G.$-\frac{2}{19}$
H.$2$
I.$19$
\testStop
\kluczStart
A
\kluczStop



\zadStart{Przykład z Wikieł P 4.3a moja wersja nr 346}


Obliczyć granicę funkcji $\lim\limits_{x\to\ 0}\frac{19 \cdot x}{tan(3 \cdot x)}$.
\zadStop
\rozwStart{Patryk Wirkus}{}
$$\lim\limits_{x\to\ 0}\frac{19 \cdot x}{tan(3 \cdot x)}=\lim\limits_{x\to\ 0}\frac{19 \cdot x \cdot cos(3 \cdot x)}{sin(3 \cdot x)}=\lim\limits_{x\to\ 0}\frac{19 \cdot cos(3 \cdot x)}{\frac{sin(3 \cdot x)}{x}}=\lim\limits_{x\to\ 0}\frac{19 \cdot cos(3 \cdot x)}{3 \cdot \frac{sin(3 \cdot x)}{3 \cdot x}} = \frac{19}{3}$$
\rozwStop
\odpStart
$\frac{19}{3}$
\odpStop
\testStart
A.$\frac{19}{3}$
B.$\infty$
C.$-\infty$
D.$0$
E.$-\frac{19}{3}$
F.$\frac{3}{19}$
G.$-\frac{3}{19}$
H.$3$
I.$19$
\testStop
\kluczStart
A
\kluczStop



\zadStart{Przykład z Wikieł P 4.3a moja wersja nr 347}


Obliczyć granicę funkcji $\lim\limits_{x\to\ 0}\frac{19 \cdot x}{tan(4 \cdot x)}$.
\zadStop
\rozwStart{Patryk Wirkus}{}
$$\lim\limits_{x\to\ 0}\frac{19 \cdot x}{tan(4 \cdot x)}=\lim\limits_{x\to\ 0}\frac{19 \cdot x \cdot cos(4 \cdot x)}{sin(4 \cdot x)}=\lim\limits_{x\to\ 0}\frac{19 \cdot cos(4 \cdot x)}{\frac{sin(4 \cdot x)}{x}}=\lim\limits_{x\to\ 0}\frac{19 \cdot cos(4 \cdot x)}{4 \cdot \frac{sin(4 \cdot x)}{4 \cdot x}} = \frac{19}{4}$$
\rozwStop
\odpStart
$\frac{19}{4}$
\odpStop
\testStart
A.$\frac{19}{4}$
B.$\infty$
C.$-\infty$
D.$0$
E.$-\frac{19}{4}$
F.$\frac{4}{19}$
G.$-\frac{4}{19}$
H.$4$
I.$19$
\testStop
\kluczStart
A
\kluczStop



\zadStart{Przykład z Wikieł P 4.3a moja wersja nr 348}


Obliczyć granicę funkcji $\lim\limits_{x\to\ 0}\frac{19 \cdot x}{tan(5 \cdot x)}$.
\zadStop
\rozwStart{Patryk Wirkus}{}
$$\lim\limits_{x\to\ 0}\frac{19 \cdot x}{tan(5 \cdot x)}=\lim\limits_{x\to\ 0}\frac{19 \cdot x \cdot cos(5 \cdot x)}{sin(5 \cdot x)}=\lim\limits_{x\to\ 0}\frac{19 \cdot cos(5 \cdot x)}{\frac{sin(5 \cdot x)}{x}}=\lim\limits_{x\to\ 0}\frac{19 \cdot cos(5 \cdot x)}{5 \cdot \frac{sin(5 \cdot x)}{5 \cdot x}} = \frac{19}{5}$$
\rozwStop
\odpStart
$\frac{19}{5}$
\odpStop
\testStart
A.$\frac{19}{5}$
B.$\infty$
C.$-\infty$
D.$0$
E.$-\frac{19}{5}$
F.$\frac{5}{19}$
G.$-\frac{5}{19}$
H.$5$
I.$19$
\testStop
\kluczStart
A
\kluczStop



\zadStart{Przykład z Wikieł P 4.3a moja wersja nr 349}


Obliczyć granicę funkcji $\lim\limits_{x\to\ 0}\frac{19 \cdot x}{tan(6 \cdot x)}$.
\zadStop
\rozwStart{Patryk Wirkus}{}
$$\lim\limits_{x\to\ 0}\frac{19 \cdot x}{tan(6 \cdot x)}=\lim\limits_{x\to\ 0}\frac{19 \cdot x \cdot cos(6 \cdot x)}{sin(6 \cdot x)}=\lim\limits_{x\to\ 0}\frac{19 \cdot cos(6 \cdot x)}{\frac{sin(6 \cdot x)}{x}}=\lim\limits_{x\to\ 0}\frac{19 \cdot cos(6 \cdot x)}{6 \cdot \frac{sin(6 \cdot x)}{6 \cdot x}} = \frac{19}{6}$$
\rozwStop
\odpStart
$\frac{19}{6}$
\odpStop
\testStart
A.$\frac{19}{6}$
B.$\infty$
C.$-\infty$
D.$0$
E.$-\frac{19}{6}$
F.$\frac{6}{19}$
G.$-\frac{6}{19}$
H.$6$
I.$19$
\testStop
\kluczStart
A
\kluczStop



\zadStart{Przykład z Wikieł P 4.3a moja wersja nr 350}


Obliczyć granicę funkcji $\lim\limits_{x\to\ 0}\frac{19 \cdot x}{tan(7 \cdot x)}$.
\zadStop
\rozwStart{Patryk Wirkus}{}
$$\lim\limits_{x\to\ 0}\frac{19 \cdot x}{tan(7 \cdot x)}=\lim\limits_{x\to\ 0}\frac{19 \cdot x \cdot cos(7 \cdot x)}{sin(7 \cdot x)}=\lim\limits_{x\to\ 0}\frac{19 \cdot cos(7 \cdot x)}{\frac{sin(7 \cdot x)}{x}}=\lim\limits_{x\to\ 0}\frac{19 \cdot cos(7 \cdot x)}{7 \cdot \frac{sin(7 \cdot x)}{7 \cdot x}} = \frac{19}{7}$$
\rozwStop
\odpStart
$\frac{19}{7}$
\odpStop
\testStart
A.$\frac{19}{7}$
B.$\infty$
C.$-\infty$
D.$0$
E.$-\frac{19}{7}$
F.$\frac{7}{19}$
G.$-\frac{7}{19}$
H.$7$
I.$19$
\testStop
\kluczStart
A
\kluczStop



\zadStart{Przykład z Wikieł P 4.3a moja wersja nr 351}


Obliczyć granicę funkcji $\lim\limits_{x\to\ 0}\frac{19 \cdot x}{tan(8 \cdot x)}$.
\zadStop
\rozwStart{Patryk Wirkus}{}
$$\lim\limits_{x\to\ 0}\frac{19 \cdot x}{tan(8 \cdot x)}=\lim\limits_{x\to\ 0}\frac{19 \cdot x \cdot cos(8 \cdot x)}{sin(8 \cdot x)}=\lim\limits_{x\to\ 0}\frac{19 \cdot cos(8 \cdot x)}{\frac{sin(8 \cdot x)}{x}}=\lim\limits_{x\to\ 0}\frac{19 \cdot cos(8 \cdot x)}{8 \cdot \frac{sin(8 \cdot x)}{8 \cdot x}} = \frac{19}{8}$$
\rozwStop
\odpStart
$\frac{19}{8}$
\odpStop
\testStart
A.$\frac{19}{8}$
B.$\infty$
C.$-\infty$
D.$0$
E.$-\frac{19}{8}$
F.$\frac{8}{19}$
G.$-\frac{8}{19}$
H.$8$
I.$19$
\testStop
\kluczStart
A
\kluczStop



\zadStart{Przykład z Wikieł P 4.3a moja wersja nr 352}


Obliczyć granicę funkcji $\lim\limits_{x\to\ 0}\frac{19 \cdot x}{tan(9 \cdot x)}$.
\zadStop
\rozwStart{Patryk Wirkus}{}
$$\lim\limits_{x\to\ 0}\frac{19 \cdot x}{tan(9 \cdot x)}=\lim\limits_{x\to\ 0}\frac{19 \cdot x \cdot cos(9 \cdot x)}{sin(9 \cdot x)}=\lim\limits_{x\to\ 0}\frac{19 \cdot cos(9 \cdot x)}{\frac{sin(9 \cdot x)}{x}}=\lim\limits_{x\to\ 0}\frac{19 \cdot cos(9 \cdot x)}{9 \cdot \frac{sin(9 \cdot x)}{9 \cdot x}} = \frac{19}{9}$$
\rozwStop
\odpStart
$\frac{19}{9}$
\odpStop
\testStart
A.$\frac{19}{9}$
B.$\infty$
C.$-\infty$
D.$0$
E.$-\frac{19}{9}$
F.$\frac{9}{19}$
G.$-\frac{9}{19}$
H.$9$
I.$19$
\testStop
\kluczStart
A
\kluczStop



\zadStart{Przykład z Wikieł P 4.3a moja wersja nr 353}


Obliczyć granicę funkcji $\lim\limits_{x\to\ 0}\frac{19 \cdot x}{tan(10 \cdot x)}$.
\zadStop
\rozwStart{Patryk Wirkus}{}
$$\lim\limits_{x\to\ 0}\frac{19 \cdot x}{tan(10 \cdot x)}=\lim\limits_{x\to\ 0}\frac{19 \cdot x \cdot cos(10 \cdot x)}{sin(10 \cdot x)}=\lim\limits_{x\to\ 0}\frac{19 \cdot cos(10 \cdot x)}{\frac{sin(10 \cdot x)}{x}}=\lim\limits_{x\to\ 0}\frac{19 \cdot cos(10 \cdot x)}{10 \cdot \frac{sin(10 \cdot x)}{10 \cdot x}} = \frac{19}{10}$$
\rozwStop
\odpStart
$\frac{19}{10}$
\odpStop
\testStart
A.$\frac{19}{10}$
B.$\infty$
C.$-\infty$
D.$0$
E.$-\frac{19}{10}$
F.$\frac{10}{19}$
G.$-\frac{10}{19}$
H.$10$
I.$19$
\testStop
\kluczStart
A
\kluczStop



\zadStart{Przykład z Wikieł P 4.3a moja wersja nr 354}


Obliczyć granicę funkcji $\lim\limits_{x\to\ 0}\frac{19 \cdot x}{tan(11 \cdot x)}$.
\zadStop
\rozwStart{Patryk Wirkus}{}
$$\lim\limits_{x\to\ 0}\frac{19 \cdot x}{tan(11 \cdot x)}=\lim\limits_{x\to\ 0}\frac{19 \cdot x \cdot cos(11 \cdot x)}{sin(11 \cdot x)}=\lim\limits_{x\to\ 0}\frac{19 \cdot cos(11 \cdot x)}{\frac{sin(11 \cdot x)}{x}}=\lim\limits_{x\to\ 0}\frac{19 \cdot cos(11 \cdot x)}{11 \cdot \frac{sin(11 \cdot x)}{11 \cdot x}} = \frac{19}{11}$$
\rozwStop
\odpStart
$\frac{19}{11}$
\odpStop
\testStart
A.$\frac{19}{11}$
B.$\infty$
C.$-\infty$
D.$0$
E.$-\frac{19}{11}$
F.$\frac{11}{19}$
G.$-\frac{11}{19}$
H.$11$
I.$19$
\testStop
\kluczStart
A
\kluczStop



\zadStart{Przykład z Wikieł P 4.3a moja wersja nr 355}


Obliczyć granicę funkcji $\lim\limits_{x\to\ 0}\frac{19 \cdot x}{tan(12 \cdot x)}$.
\zadStop
\rozwStart{Patryk Wirkus}{}
$$\lim\limits_{x\to\ 0}\frac{19 \cdot x}{tan(12 \cdot x)}=\lim\limits_{x\to\ 0}\frac{19 \cdot x \cdot cos(12 \cdot x)}{sin(12 \cdot x)}=\lim\limits_{x\to\ 0}\frac{19 \cdot cos(12 \cdot x)}{\frac{sin(12 \cdot x)}{x}}=\lim\limits_{x\to\ 0}\frac{19 \cdot cos(12 \cdot x)}{12 \cdot \frac{sin(12 \cdot x)}{12 \cdot x}} = \frac{19}{12}$$
\rozwStop
\odpStart
$\frac{19}{12}$
\odpStop
\testStart
A.$\frac{19}{12}$
B.$\infty$
C.$-\infty$
D.$0$
E.$-\frac{19}{12}$
F.$\frac{12}{19}$
G.$-\frac{12}{19}$
H.$12$
I.$19$
\testStop
\kluczStart
A
\kluczStop



\zadStart{Przykład z Wikieł P 4.3a moja wersja nr 356}


Obliczyć granicę funkcji $\lim\limits_{x\to\ 0}\frac{19 \cdot x}{tan(13 \cdot x)}$.
\zadStop
\rozwStart{Patryk Wirkus}{}
$$\lim\limits_{x\to\ 0}\frac{19 \cdot x}{tan(13 \cdot x)}=\lim\limits_{x\to\ 0}\frac{19 \cdot x \cdot cos(13 \cdot x)}{sin(13 \cdot x)}=\lim\limits_{x\to\ 0}\frac{19 \cdot cos(13 \cdot x)}{\frac{sin(13 \cdot x)}{x}}=\lim\limits_{x\to\ 0}\frac{19 \cdot cos(13 \cdot x)}{13 \cdot \frac{sin(13 \cdot x)}{13 \cdot x}} = \frac{19}{13}$$
\rozwStop
\odpStart
$\frac{19}{13}$
\odpStop
\testStart
A.$\frac{19}{13}$
B.$\infty$
C.$-\infty$
D.$0$
E.$-\frac{19}{13}$
F.$\frac{13}{19}$
G.$-\frac{13}{19}$
H.$13$
I.$19$
\testStop
\kluczStart
A
\kluczStop



\zadStart{Przykład z Wikieł P 4.3a moja wersja nr 357}


Obliczyć granicę funkcji $\lim\limits_{x\to\ 0}\frac{19 \cdot x}{tan(14 \cdot x)}$.
\zadStop
\rozwStart{Patryk Wirkus}{}
$$\lim\limits_{x\to\ 0}\frac{19 \cdot x}{tan(14 \cdot x)}=\lim\limits_{x\to\ 0}\frac{19 \cdot x \cdot cos(14 \cdot x)}{sin(14 \cdot x)}=\lim\limits_{x\to\ 0}\frac{19 \cdot cos(14 \cdot x)}{\frac{sin(14 \cdot x)}{x}}=\lim\limits_{x\to\ 0}\frac{19 \cdot cos(14 \cdot x)}{14 \cdot \frac{sin(14 \cdot x)}{14 \cdot x}} = \frac{19}{14}$$
\rozwStop
\odpStart
$\frac{19}{14}$
\odpStop
\testStart
A.$\frac{19}{14}$
B.$\infty$
C.$-\infty$
D.$0$
E.$-\frac{19}{14}$
F.$\frac{14}{19}$
G.$-\frac{14}{19}$
H.$14$
I.$19$
\testStop
\kluczStart
A
\kluczStop



\zadStart{Przykład z Wikieł P 4.3a moja wersja nr 358}


Obliczyć granicę funkcji $\lim\limits_{x\to\ 0}\frac{19 \cdot x}{tan(15 \cdot x)}$.
\zadStop
\rozwStart{Patryk Wirkus}{}
$$\lim\limits_{x\to\ 0}\frac{19 \cdot x}{tan(15 \cdot x)}=\lim\limits_{x\to\ 0}\frac{19 \cdot x \cdot cos(15 \cdot x)}{sin(15 \cdot x)}=\lim\limits_{x\to\ 0}\frac{19 \cdot cos(15 \cdot x)}{\frac{sin(15 \cdot x)}{x}}=\lim\limits_{x\to\ 0}\frac{19 \cdot cos(15 \cdot x)}{15 \cdot \frac{sin(15 \cdot x)}{15 \cdot x}} = \frac{19}{15}$$
\rozwStop
\odpStart
$\frac{19}{15}$
\odpStop
\testStart
A.$\frac{19}{15}$
B.$\infty$
C.$-\infty$
D.$0$
E.$-\frac{19}{15}$
F.$\frac{15}{19}$
G.$-\frac{15}{19}$
H.$15$
I.$19$
\testStop
\kluczStart
A
\kluczStop



\zadStart{Przykład z Wikieł P 4.3a moja wersja nr 359}


Obliczyć granicę funkcji $\lim\limits_{x\to\ 0}\frac{19 \cdot x}{tan(16 \cdot x)}$.
\zadStop
\rozwStart{Patryk Wirkus}{}
$$\lim\limits_{x\to\ 0}\frac{19 \cdot x}{tan(16 \cdot x)}=\lim\limits_{x\to\ 0}\frac{19 \cdot x \cdot cos(16 \cdot x)}{sin(16 \cdot x)}=\lim\limits_{x\to\ 0}\frac{19 \cdot cos(16 \cdot x)}{\frac{sin(16 \cdot x)}{x}}=\lim\limits_{x\to\ 0}\frac{19 \cdot cos(16 \cdot x)}{16 \cdot \frac{sin(16 \cdot x)}{16 \cdot x}} = \frac{19}{16}$$
\rozwStop
\odpStart
$\frac{19}{16}$
\odpStop
\testStart
A.$\frac{19}{16}$
B.$\infty$
C.$-\infty$
D.$0$
E.$-\frac{19}{16}$
F.$\frac{16}{19}$
G.$-\frac{16}{19}$
H.$16$
I.$19$
\testStop
\kluczStart
A
\kluczStop



\zadStart{Przykład z Wikieł P 4.3a moja wersja nr 360}


Obliczyć granicę funkcji $\lim\limits_{x\to\ 0}\frac{19 \cdot x}{tan(17 \cdot x)}$.
\zadStop
\rozwStart{Patryk Wirkus}{}
$$\lim\limits_{x\to\ 0}\frac{19 \cdot x}{tan(17 \cdot x)}=\lim\limits_{x\to\ 0}\frac{19 \cdot x \cdot cos(17 \cdot x)}{sin(17 \cdot x)}=\lim\limits_{x\to\ 0}\frac{19 \cdot cos(17 \cdot x)}{\frac{sin(17 \cdot x)}{x}}=\lim\limits_{x\to\ 0}\frac{19 \cdot cos(17 \cdot x)}{17 \cdot \frac{sin(17 \cdot x)}{17 \cdot x}} = \frac{19}{17}$$
\rozwStop
\odpStart
$\frac{19}{17}$
\odpStop
\testStart
A.$\frac{19}{17}$
B.$\infty$
C.$-\infty$
D.$0$
E.$-\frac{19}{17}$
F.$\frac{17}{19}$
G.$-\frac{17}{19}$
H.$17$
I.$19$
\testStop
\kluczStart
A
\kluczStop



\zadStart{Przykład z Wikieł P 4.3a moja wersja nr 361}


Obliczyć granicę funkcji $\lim\limits_{x\to\ 0}\frac{19 \cdot x}{tan(18 \cdot x)}$.
\zadStop
\rozwStart{Patryk Wirkus}{}
$$\lim\limits_{x\to\ 0}\frac{19 \cdot x}{tan(18 \cdot x)}=\lim\limits_{x\to\ 0}\frac{19 \cdot x \cdot cos(18 \cdot x)}{sin(18 \cdot x)}=\lim\limits_{x\to\ 0}\frac{19 \cdot cos(18 \cdot x)}{\frac{sin(18 \cdot x)}{x}}=\lim\limits_{x\to\ 0}\frac{19 \cdot cos(18 \cdot x)}{18 \cdot \frac{sin(18 \cdot x)}{18 \cdot x}} = \frac{19}{18}$$
\rozwStop
\odpStart
$\frac{19}{18}$
\odpStop
\testStart
A.$\frac{19}{18}$
B.$\infty$
C.$-\infty$
D.$0$
E.$-\frac{19}{18}$
F.$\frac{18}{19}$
G.$-\frac{18}{19}$
H.$18$
I.$19$
\testStop
\kluczStart
A
\kluczStop



\zadStart{Przykład z Wikieł P 4.3a moja wersja nr 362}


Obliczyć granicę funkcji $\lim\limits_{x\to\ 0}\frac{19 \cdot x}{tan(20 \cdot x)}$.
\zadStop
\rozwStart{Patryk Wirkus}{}
$$\lim\limits_{x\to\ 0}\frac{19 \cdot x}{tan(20 \cdot x)}=\lim\limits_{x\to\ 0}\frac{19 \cdot x \cdot cos(20 \cdot x)}{sin(20 \cdot x)}=\lim\limits_{x\to\ 0}\frac{19 \cdot cos(20 \cdot x)}{\frac{sin(20 \cdot x)}{x}}=\lim\limits_{x\to\ 0}\frac{19 \cdot cos(20 \cdot x)}{20 \cdot \frac{sin(20 \cdot x)}{20 \cdot x}} = \frac{19}{20}$$
\rozwStop
\odpStart
$\frac{19}{20}$
\odpStop
\testStart
A.$\frac{19}{20}$
B.$\infty$
C.$-\infty$
D.$0$
E.$-\frac{19}{20}$
F.$\frac{20}{19}$
G.$-\frac{20}{19}$
H.$20$
I.$19$
\testStop
\kluczStart
A
\kluczStop



\zadStart{Przykład z Wikieł P 4.3a moja wersja nr 363}


Obliczyć granicę funkcji $\lim\limits_{x\to\ 0}\frac{19 \cdot x}{tan(21 \cdot x)}$.
\zadStop
\rozwStart{Patryk Wirkus}{}
$$\lim\limits_{x\to\ 0}\frac{19 \cdot x}{tan(21 \cdot x)}=\lim\limits_{x\to\ 0}\frac{19 \cdot x \cdot cos(21 \cdot x)}{sin(21 \cdot x)}=\lim\limits_{x\to\ 0}\frac{19 \cdot cos(21 \cdot x)}{\frac{sin(21 \cdot x)}{x}}=\lim\limits_{x\to\ 0}\frac{19 \cdot cos(21 \cdot x)}{21 \cdot \frac{sin(21 \cdot x)}{21 \cdot x}} = \frac{19}{21}$$
\rozwStop
\odpStart
$\frac{19}{21}$
\odpStop
\testStart
A.$\frac{19}{21}$
B.$\infty$
C.$-\infty$
D.$0$
E.$-\frac{19}{21}$
F.$\frac{21}{19}$
G.$-\frac{21}{19}$
H.$21$
I.$19$
\testStop
\kluczStart
A
\kluczStop



\zadStart{Przykład z Wikieł P 4.3a moja wersja nr 364}


Obliczyć granicę funkcji $\lim\limits_{x\to\ 0}\frac{19 \cdot x}{tan(22 \cdot x)}$.
\zadStop
\rozwStart{Patryk Wirkus}{}
$$\lim\limits_{x\to\ 0}\frac{19 \cdot x}{tan(22 \cdot x)}=\lim\limits_{x\to\ 0}\frac{19 \cdot x \cdot cos(22 \cdot x)}{sin(22 \cdot x)}=\lim\limits_{x\to\ 0}\frac{19 \cdot cos(22 \cdot x)}{\frac{sin(22 \cdot x)}{x}}=\lim\limits_{x\to\ 0}\frac{19 \cdot cos(22 \cdot x)}{22 \cdot \frac{sin(22 \cdot x)}{22 \cdot x}} = \frac{19}{22}$$
\rozwStop
\odpStart
$\frac{19}{22}$
\odpStop
\testStart
A.$\frac{19}{22}$
B.$\infty$
C.$-\infty$
D.$0$
E.$-\frac{19}{22}$
F.$\frac{22}{19}$
G.$-\frac{22}{19}$
H.$22$
I.$19$
\testStop
\kluczStart
A
\kluczStop



\zadStart{Przykład z Wikieł P 4.3a moja wersja nr 365}


Obliczyć granicę funkcji $\lim\limits_{x\to\ 0}\frac{19 \cdot x}{tan(23 \cdot x)}$.
\zadStop
\rozwStart{Patryk Wirkus}{}
$$\lim\limits_{x\to\ 0}\frac{19 \cdot x}{tan(23 \cdot x)}=\lim\limits_{x\to\ 0}\frac{19 \cdot x \cdot cos(23 \cdot x)}{sin(23 \cdot x)}=\lim\limits_{x\to\ 0}\frac{19 \cdot cos(23 \cdot x)}{\frac{sin(23 \cdot x)}{x}}=\lim\limits_{x\to\ 0}\frac{19 \cdot cos(23 \cdot x)}{23 \cdot \frac{sin(23 \cdot x)}{23 \cdot x}} = \frac{19}{23}$$
\rozwStop
\odpStart
$\frac{19}{23}$
\odpStop
\testStart
A.$\frac{19}{23}$
B.$\infty$
C.$-\infty$
D.$0$
E.$-\frac{19}{23}$
F.$\frac{23}{19}$
G.$-\frac{23}{19}$
H.$23$
I.$19$
\testStop
\kluczStart
A
\kluczStop



\zadStart{Przykład z Wikieł P 4.3a moja wersja nr 366}


Obliczyć granicę funkcji $\lim\limits_{x\to\ 0}\frac{19 \cdot x}{tan(24 \cdot x)}$.
\zadStop
\rozwStart{Patryk Wirkus}{}
$$\lim\limits_{x\to\ 0}\frac{19 \cdot x}{tan(24 \cdot x)}=\lim\limits_{x\to\ 0}\frac{19 \cdot x \cdot cos(24 \cdot x)}{sin(24 \cdot x)}=\lim\limits_{x\to\ 0}\frac{19 \cdot cos(24 \cdot x)}{\frac{sin(24 \cdot x)}{x}}=\lim\limits_{x\to\ 0}\frac{19 \cdot cos(24 \cdot x)}{24 \cdot \frac{sin(24 \cdot x)}{24 \cdot x}} = \frac{19}{24}$$
\rozwStop
\odpStart
$\frac{19}{24}$
\odpStop
\testStart
A.$\frac{19}{24}$
B.$\infty$
C.$-\infty$
D.$0$
E.$-\frac{19}{24}$
F.$\frac{24}{19}$
G.$-\frac{24}{19}$
H.$24$
I.$19$
\testStop
\kluczStart
A
\kluczStop



\zadStart{Przykład z Wikieł P 4.3a moja wersja nr 367}


Obliczyć granicę funkcji $\lim\limits_{x\to\ 0}\frac{19 \cdot x}{tan(25 \cdot x)}$.
\zadStop
\rozwStart{Patryk Wirkus}{}
$$\lim\limits_{x\to\ 0}\frac{19 \cdot x}{tan(25 \cdot x)}=\lim\limits_{x\to\ 0}\frac{19 \cdot x \cdot cos(25 \cdot x)}{sin(25 \cdot x)}=\lim\limits_{x\to\ 0}\frac{19 \cdot cos(25 \cdot x)}{\frac{sin(25 \cdot x)}{x}}=\lim\limits_{x\to\ 0}\frac{19 \cdot cos(25 \cdot x)}{25 \cdot \frac{sin(25 \cdot x)}{25 \cdot x}} = \frac{19}{25}$$
\rozwStop
\odpStart
$\frac{19}{25}$
\odpStop
\testStart
A.$\frac{19}{25}$
B.$\infty$
C.$-\infty$
D.$0$
E.$-\frac{19}{25}$
F.$\frac{25}{19}$
G.$-\frac{25}{19}$
H.$25$
I.$19$
\testStop
\kluczStart
A
\kluczStop



\zadStart{Przykład z Wikieł P 4.3a moja wersja nr 368}


Obliczyć granicę funkcji $\lim\limits_{x\to\ 0}\frac{19 \cdot x}{tan(26 \cdot x)}$.
\zadStop
\rozwStart{Patryk Wirkus}{}
$$\lim\limits_{x\to\ 0}\frac{19 \cdot x}{tan(26 \cdot x)}=\lim\limits_{x\to\ 0}\frac{19 \cdot x \cdot cos(26 \cdot x)}{sin(26 \cdot x)}=\lim\limits_{x\to\ 0}\frac{19 \cdot cos(26 \cdot x)}{\frac{sin(26 \cdot x)}{x}}=\lim\limits_{x\to\ 0}\frac{19 \cdot cos(26 \cdot x)}{26 \cdot \frac{sin(26 \cdot x)}{26 \cdot x}} = \frac{19}{26}$$
\rozwStop
\odpStart
$\frac{19}{26}$
\odpStop
\testStart
A.$\frac{19}{26}$
B.$\infty$
C.$-\infty$
D.$0$
E.$-\frac{19}{26}$
F.$\frac{26}{19}$
G.$-\frac{26}{19}$
H.$26$
I.$19$
\testStop
\kluczStart
A
\kluczStop



\zadStart{Przykład z Wikieł P 4.3a moja wersja nr 369}


Obliczyć granicę funkcji $\lim\limits_{x\to\ 0}\frac{19 \cdot x}{tan(27 \cdot x)}$.
\zadStop
\rozwStart{Patryk Wirkus}{}
$$\lim\limits_{x\to\ 0}\frac{19 \cdot x}{tan(27 \cdot x)}=\lim\limits_{x\to\ 0}\frac{19 \cdot x \cdot cos(27 \cdot x)}{sin(27 \cdot x)}=\lim\limits_{x\to\ 0}\frac{19 \cdot cos(27 \cdot x)}{\frac{sin(27 \cdot x)}{x}}=\lim\limits_{x\to\ 0}\frac{19 \cdot cos(27 \cdot x)}{27 \cdot \frac{sin(27 \cdot x)}{27 \cdot x}} = \frac{19}{27}$$
\rozwStop
\odpStart
$\frac{19}{27}$
\odpStop
\testStart
A.$\frac{19}{27}$
B.$\infty$
C.$-\infty$
D.$0$
E.$-\frac{19}{27}$
F.$\frac{27}{19}$
G.$-\frac{27}{19}$
H.$27$
I.$19$
\testStop
\kluczStart
A
\kluczStop



\zadStart{Przykład z Wikieł P 4.3a moja wersja nr 370}


Obliczyć granicę funkcji $\lim\limits_{x\to\ 0}\frac{19 \cdot x}{tan(28 \cdot x)}$.
\zadStop
\rozwStart{Patryk Wirkus}{}
$$\lim\limits_{x\to\ 0}\frac{19 \cdot x}{tan(28 \cdot x)}=\lim\limits_{x\to\ 0}\frac{19 \cdot x \cdot cos(28 \cdot x)}{sin(28 \cdot x)}=\lim\limits_{x\to\ 0}\frac{19 \cdot cos(28 \cdot x)}{\frac{sin(28 \cdot x)}{x}}=\lim\limits_{x\to\ 0}\frac{19 \cdot cos(28 \cdot x)}{28 \cdot \frac{sin(28 \cdot x)}{28 \cdot x}} = \frac{19}{28}$$
\rozwStop
\odpStart
$\frac{19}{28}$
\odpStop
\testStart
A.$\frac{19}{28}$
B.$\infty$
C.$-\infty$
D.$0$
E.$-\frac{19}{28}$
F.$\frac{28}{19}$
G.$-\frac{28}{19}$
H.$28$
I.$19$
\testStop
\kluczStart
A
\kluczStop



\zadStart{Przykład z Wikieł P 4.3a moja wersja nr 371}


Obliczyć granicę funkcji $\lim\limits_{x\to\ 0}\frac{19 \cdot x}{tan(29 \cdot x)}$.
\zadStop
\rozwStart{Patryk Wirkus}{}
$$\lim\limits_{x\to\ 0}\frac{19 \cdot x}{tan(29 \cdot x)}=\lim\limits_{x\to\ 0}\frac{19 \cdot x \cdot cos(29 \cdot x)}{sin(29 \cdot x)}=\lim\limits_{x\to\ 0}\frac{19 \cdot cos(29 \cdot x)}{\frac{sin(29 \cdot x)}{x}}=\lim\limits_{x\to\ 0}\frac{19 \cdot cos(29 \cdot x)}{29 \cdot \frac{sin(29 \cdot x)}{29 \cdot x}} = \frac{19}{29}$$
\rozwStop
\odpStart
$\frac{19}{29}$
\odpStop
\testStart
A.$\frac{19}{29}$
B.$\infty$
C.$-\infty$
D.$0$
E.$-\frac{19}{29}$
F.$\frac{29}{19}$
G.$-\frac{29}{19}$
H.$29$
I.$19$
\testStop
\kluczStart
A
\kluczStop



\zadStart{Przykład z Wikieł P 4.3a moja wersja nr 372}


Obliczyć granicę funkcji $\lim\limits_{x\to\ 0}\frac{19 \cdot x}{tan(30 \cdot x)}$.
\zadStop
\rozwStart{Patryk Wirkus}{}
$$\lim\limits_{x\to\ 0}\frac{19 \cdot x}{tan(30 \cdot x)}=\lim\limits_{x\to\ 0}\frac{19 \cdot x \cdot cos(30 \cdot x)}{sin(30 \cdot x)}=\lim\limits_{x\to\ 0}\frac{19 \cdot cos(30 \cdot x)}{\frac{sin(30 \cdot x)}{x}}=\lim\limits_{x\to\ 0}\frac{19 \cdot cos(30 \cdot x)}{30 \cdot \frac{sin(30 \cdot x)}{30 \cdot x}} = \frac{19}{30}$$
\rozwStop
\odpStart
$\frac{19}{30}$
\odpStop
\testStart
A.$\frac{19}{30}$
B.$\infty$
C.$-\infty$
D.$0$
E.$-\frac{19}{30}$
F.$\frac{30}{19}$
G.$-\frac{30}{19}$
H.$30$
I.$19$
\testStop
\kluczStart
A
\kluczStop



\zadStart{Przykład z Wikieł P 4.3a moja wersja nr 373}


Obliczyć granicę funkcji $\lim\limits_{x\to\ 0}\frac{19 \cdot x}{tan(31 \cdot x)}$.
\zadStop
\rozwStart{Patryk Wirkus}{}
$$\lim\limits_{x\to\ 0}\frac{19 \cdot x}{tan(31 \cdot x)}=\lim\limits_{x\to\ 0}\frac{19 \cdot x \cdot cos(31 \cdot x)}{sin(31 \cdot x)}=\lim\limits_{x\to\ 0}\frac{19 \cdot cos(31 \cdot x)}{\frac{sin(31 \cdot x)}{x}}=\lim\limits_{x\to\ 0}\frac{19 \cdot cos(31 \cdot x)}{31 \cdot \frac{sin(31 \cdot x)}{31 \cdot x}} = \frac{19}{31}$$
\rozwStop
\odpStart
$\frac{19}{31}$
\odpStop
\testStart
A.$\frac{19}{31}$
B.$\infty$
C.$-\infty$
D.$0$
E.$-\frac{19}{31}$
F.$\frac{31}{19}$
G.$-\frac{31}{19}$
H.$31$
I.$19$
\testStop
\kluczStart
A
\kluczStop



\zadStart{Przykład z Wikieł P 4.3a moja wersja nr 374}


Obliczyć granicę funkcji $\lim\limits_{x\to\ 0}\frac{19 \cdot x}{tan(32 \cdot x)}$.
\zadStop
\rozwStart{Patryk Wirkus}{}
$$\lim\limits_{x\to\ 0}\frac{19 \cdot x}{tan(32 \cdot x)}=\lim\limits_{x\to\ 0}\frac{19 \cdot x \cdot cos(32 \cdot x)}{sin(32 \cdot x)}=\lim\limits_{x\to\ 0}\frac{19 \cdot cos(32 \cdot x)}{\frac{sin(32 \cdot x)}{x}}=\lim\limits_{x\to\ 0}\frac{19 \cdot cos(32 \cdot x)}{32 \cdot \frac{sin(32 \cdot x)}{32 \cdot x}} = \frac{19}{32}$$
\rozwStop
\odpStart
$\frac{19}{32}$
\odpStop
\testStart
A.$\frac{19}{32}$
B.$\infty$
C.$-\infty$
D.$0$
E.$-\frac{19}{32}$
F.$\frac{32}{19}$
G.$-\frac{32}{19}$
H.$32$
I.$19$
\testStop
\kluczStart
A
\kluczStop



\zadStart{Przykład z Wikieł P 4.3a moja wersja nr 375}


Obliczyć granicę funkcji $\lim\limits_{x\to\ 0}\frac{19 \cdot x}{tan(33 \cdot x)}$.
\zadStop
\rozwStart{Patryk Wirkus}{}
$$\lim\limits_{x\to\ 0}\frac{19 \cdot x}{tan(33 \cdot x)}=\lim\limits_{x\to\ 0}\frac{19 \cdot x \cdot cos(33 \cdot x)}{sin(33 \cdot x)}=\lim\limits_{x\to\ 0}\frac{19 \cdot cos(33 \cdot x)}{\frac{sin(33 \cdot x)}{x}}=\lim\limits_{x\to\ 0}\frac{19 \cdot cos(33 \cdot x)}{33 \cdot \frac{sin(33 \cdot x)}{33 \cdot x}} = \frac{19}{33}$$
\rozwStop
\odpStart
$\frac{19}{33}$
\odpStop
\testStart
A.$\frac{19}{33}$
B.$\infty$
C.$-\infty$
D.$0$
E.$-\frac{19}{33}$
F.$\frac{33}{19}$
G.$-\frac{33}{19}$
H.$33$
I.$19$
\testStop
\kluczStart
A
\kluczStop



\zadStart{Przykład z Wikieł P 4.3a moja wersja nr 376}


Obliczyć granicę funkcji $\lim\limits_{x\to\ 0}\frac{19 \cdot x}{tan(34 \cdot x)}$.
\zadStop
\rozwStart{Patryk Wirkus}{}
$$\lim\limits_{x\to\ 0}\frac{19 \cdot x}{tan(34 \cdot x)}=\lim\limits_{x\to\ 0}\frac{19 \cdot x \cdot cos(34 \cdot x)}{sin(34 \cdot x)}=\lim\limits_{x\to\ 0}\frac{19 \cdot cos(34 \cdot x)}{\frac{sin(34 \cdot x)}{x}}=\lim\limits_{x\to\ 0}\frac{19 \cdot cos(34 \cdot x)}{34 \cdot \frac{sin(34 \cdot x)}{34 \cdot x}} = \frac{19}{34}$$
\rozwStop
\odpStart
$\frac{19}{34}$
\odpStop
\testStart
A.$\frac{19}{34}$
B.$\infty$
C.$-\infty$
D.$0$
E.$-\frac{19}{34}$
F.$\frac{34}{19}$
G.$-\frac{34}{19}$
H.$34$
I.$19$
\testStop
\kluczStart
A
\kluczStop



\zadStart{Przykład z Wikieł P 4.3a moja wersja nr 377}


Obliczyć granicę funkcji $\lim\limits_{x\to\ 0}\frac{19 \cdot x}{tan(35 \cdot x)}$.
\zadStop
\rozwStart{Patryk Wirkus}{}
$$\lim\limits_{x\to\ 0}\frac{19 \cdot x}{tan(35 \cdot x)}=\lim\limits_{x\to\ 0}\frac{19 \cdot x \cdot cos(35 \cdot x)}{sin(35 \cdot x)}=\lim\limits_{x\to\ 0}\frac{19 \cdot cos(35 \cdot x)}{\frac{sin(35 \cdot x)}{x}}=\lim\limits_{x\to\ 0}\frac{19 \cdot cos(35 \cdot x)}{35 \cdot \frac{sin(35 \cdot x)}{35 \cdot x}} = \frac{19}{35}$$
\rozwStop
\odpStart
$\frac{19}{35}$
\odpStop
\testStart
A.$\frac{19}{35}$
B.$\infty$
C.$-\infty$
D.$0$
E.$-\frac{19}{35}$
F.$\frac{35}{19}$
G.$-\frac{35}{19}$
H.$35$
I.$19$
\testStop
\kluczStart
A
\kluczStop



\zadStart{Przykład z Wikieł P 4.3a moja wersja nr 378}


Obliczyć granicę funkcji $\lim\limits_{x\to\ 0}\frac{19 \cdot x}{tan(36 \cdot x)}$.
\zadStop
\rozwStart{Patryk Wirkus}{}
$$\lim\limits_{x\to\ 0}\frac{19 \cdot x}{tan(36 \cdot x)}=\lim\limits_{x\to\ 0}\frac{19 \cdot x \cdot cos(36 \cdot x)}{sin(36 \cdot x)}=\lim\limits_{x\to\ 0}\frac{19 \cdot cos(36 \cdot x)}{\frac{sin(36 \cdot x)}{x}}=\lim\limits_{x\to\ 0}\frac{19 \cdot cos(36 \cdot x)}{36 \cdot \frac{sin(36 \cdot x)}{36 \cdot x}} = \frac{19}{36}$$
\rozwStop
\odpStart
$\frac{19}{36}$
\odpStop
\testStart
A.$\frac{19}{36}$
B.$\infty$
C.$-\infty$
D.$0$
E.$-\frac{19}{36}$
F.$\frac{36}{19}$
G.$-\frac{36}{19}$
H.$36$
I.$19$
\testStop
\kluczStart
A
\kluczStop



\zadStart{Przykład z Wikieł P 4.3a moja wersja nr 379}


Obliczyć granicę funkcji $\lim\limits_{x\to\ 0}\frac{19 \cdot x}{tan(37 \cdot x)}$.
\zadStop
\rozwStart{Patryk Wirkus}{}
$$\lim\limits_{x\to\ 0}\frac{19 \cdot x}{tan(37 \cdot x)}=\lim\limits_{x\to\ 0}\frac{19 \cdot x \cdot cos(37 \cdot x)}{sin(37 \cdot x)}=\lim\limits_{x\to\ 0}\frac{19 \cdot cos(37 \cdot x)}{\frac{sin(37 \cdot x)}{x}}=\lim\limits_{x\to\ 0}\frac{19 \cdot cos(37 \cdot x)}{37 \cdot \frac{sin(37 \cdot x)}{37 \cdot x}} = \frac{19}{37}$$
\rozwStop
\odpStart
$\frac{19}{37}$
\odpStop
\testStart
A.$\frac{19}{37}$
B.$\infty$
C.$-\infty$
D.$0$
E.$-\frac{19}{37}$
F.$\frac{37}{19}$
G.$-\frac{37}{19}$
H.$37$
I.$19$
\testStop
\kluczStart
A
\kluczStop



\zadStart{Przykład z Wikieł P 4.3a moja wersja nr 380}


Obliczyć granicę funkcji $\lim\limits_{x\to\ 0}\frac{19 \cdot x}{tan(39 \cdot x)}$.
\zadStop
\rozwStart{Patryk Wirkus}{}
$$\lim\limits_{x\to\ 0}\frac{19 \cdot x}{tan(39 \cdot x)}=\lim\limits_{x\to\ 0}\frac{19 \cdot x \cdot cos(39 \cdot x)}{sin(39 \cdot x)}=\lim\limits_{x\to\ 0}\frac{19 \cdot cos(39 \cdot x)}{\frac{sin(39 \cdot x)}{x}}=\lim\limits_{x\to\ 0}\frac{19 \cdot cos(39 \cdot x)}{39 \cdot \frac{sin(39 \cdot x)}{39 \cdot x}} = \frac{19}{39}$$
\rozwStop
\odpStart
$\frac{19}{39}$
\odpStop
\testStart
A.$\frac{19}{39}$
B.$\infty$
C.$-\infty$
D.$0$
E.$-\frac{19}{39}$
F.$\frac{39}{19}$
G.$-\frac{39}{19}$
H.$39$
I.$19$
\testStop
\kluczStart
A
\kluczStop



\zadStart{Przykład z Wikieł P 4.3a moja wersja nr 381}


Obliczyć granicę funkcji $\lim\limits_{x\to\ 0}\frac{19 \cdot x}{tan(40 \cdot x)}$.
\zadStop
\rozwStart{Patryk Wirkus}{}
$$\lim\limits_{x\to\ 0}\frac{19 \cdot x}{tan(40 \cdot x)}=\lim\limits_{x\to\ 0}\frac{19 \cdot x \cdot cos(40 \cdot x)}{sin(40 \cdot x)}=\lim\limits_{x\to\ 0}\frac{19 \cdot cos(40 \cdot x)}{\frac{sin(40 \cdot x)}{x}}=\lim\limits_{x\to\ 0}\frac{19 \cdot cos(40 \cdot x)}{40 \cdot \frac{sin(40 \cdot x)}{40 \cdot x}} = \frac{19}{40}$$
\rozwStop
\odpStart
$\frac{19}{40}$
\odpStop
\testStart
A.$\frac{19}{40}$
B.$\infty$
C.$-\infty$
D.$0$
E.$-\frac{19}{40}$
F.$\frac{40}{19}$
G.$-\frac{40}{19}$
H.$40$
I.$19$
\testStop
\kluczStart
A
\kluczStop



\zadStart{Przykład z Wikieł P 4.3a moja wersja nr 382}


Obliczyć granicę funkcji $\lim\limits_{x\to\ 0}\frac{20 \cdot x}{tan(3 \cdot x)}$.
\zadStop
\rozwStart{Patryk Wirkus}{}
$$\lim\limits_{x\to\ 0}\frac{20 \cdot x}{tan(3 \cdot x)}=\lim\limits_{x\to\ 0}\frac{20 \cdot x \cdot cos(3 \cdot x)}{sin(3 \cdot x)}=\lim\limits_{x\to\ 0}\frac{20 \cdot cos(3 \cdot x)}{\frac{sin(3 \cdot x)}{x}}=\lim\limits_{x\to\ 0}\frac{20 \cdot cos(3 \cdot x)}{3 \cdot \frac{sin(3 \cdot x)}{3 \cdot x}} = \frac{20}{3}$$
\rozwStop
\odpStart
$\frac{20}{3}$
\odpStop
\testStart
A.$\frac{20}{3}$
B.$\infty$
C.$-\infty$
D.$0$
E.$-\frac{20}{3}$
F.$\frac{3}{20}$
G.$-\frac{3}{20}$
H.$3$
I.$20$
\testStop
\kluczStart
A
\kluczStop



\zadStart{Przykład z Wikieł P 4.3a moja wersja nr 383}


Obliczyć granicę funkcji $\lim\limits_{x\to\ 0}\frac{20 \cdot x}{tan(7 \cdot x)}$.
\zadStop
\rozwStart{Patryk Wirkus}{}
$$\lim\limits_{x\to\ 0}\frac{20 \cdot x}{tan(7 \cdot x)}=\lim\limits_{x\to\ 0}\frac{20 \cdot x \cdot cos(7 \cdot x)}{sin(7 \cdot x)}=\lim\limits_{x\to\ 0}\frac{20 \cdot cos(7 \cdot x)}{\frac{sin(7 \cdot x)}{x}}=\lim\limits_{x\to\ 0}\frac{20 \cdot cos(7 \cdot x)}{7 \cdot \frac{sin(7 \cdot x)}{7 \cdot x}} = \frac{20}{7}$$
\rozwStop
\odpStart
$\frac{20}{7}$
\odpStop
\testStart
A.$\frac{20}{7}$
B.$\infty$
C.$-\infty$
D.$0$
E.$-\frac{20}{7}$
F.$\frac{7}{20}$
G.$-\frac{7}{20}$
H.$7$
I.$20$
\testStop
\kluczStart
A
\kluczStop



\zadStart{Przykład z Wikieł P 4.3a moja wersja nr 384}


Obliczyć granicę funkcji $\lim\limits_{x\to\ 0}\frac{20 \cdot x}{tan(9 \cdot x)}$.
\zadStop
\rozwStart{Patryk Wirkus}{}
$$\lim\limits_{x\to\ 0}\frac{20 \cdot x}{tan(9 \cdot x)}=\lim\limits_{x\to\ 0}\frac{20 \cdot x \cdot cos(9 \cdot x)}{sin(9 \cdot x)}=\lim\limits_{x\to\ 0}\frac{20 \cdot cos(9 \cdot x)}{\frac{sin(9 \cdot x)}{x}}=\lim\limits_{x\to\ 0}\frac{20 \cdot cos(9 \cdot x)}{9 \cdot \frac{sin(9 \cdot x)}{9 \cdot x}} = \frac{20}{9}$$
\rozwStop
\odpStart
$\frac{20}{9}$
\odpStop
\testStart
A.$\frac{20}{9}$
B.$\infty$
C.$-\infty$
D.$0$
E.$-\frac{20}{9}$
F.$\frac{9}{20}$
G.$-\frac{9}{20}$
H.$9$
I.$20$
\testStop
\kluczStart
A
\kluczStop



\zadStart{Przykład z Wikieł P 4.3a moja wersja nr 385}


Obliczyć granicę funkcji $\lim\limits_{x\to\ 0}\frac{20 \cdot x}{tan(11 \cdot x)}$.
\zadStop
\rozwStart{Patryk Wirkus}{}
$$\lim\limits_{x\to\ 0}\frac{20 \cdot x}{tan(11 \cdot x)}=\lim\limits_{x\to\ 0}\frac{20 \cdot x \cdot cos(11 \cdot x)}{sin(11 \cdot x)}=\lim\limits_{x\to\ 0}\frac{20 \cdot cos(11 \cdot x)}{\frac{sin(11 \cdot x)}{x}}=\lim\limits_{x\to\ 0}\frac{20 \cdot cos(11 \cdot x)}{11 \cdot \frac{sin(11 \cdot x)}{11 \cdot x}} = \frac{20}{11}$$
\rozwStop
\odpStart
$\frac{20}{11}$
\odpStop
\testStart
A.$\frac{20}{11}$
B.$\infty$
C.$-\infty$
D.$0$
E.$-\frac{20}{11}$
F.$\frac{11}{20}$
G.$-\frac{11}{20}$
H.$11$
I.$20$
\testStop
\kluczStart
A
\kluczStop



\zadStart{Przykład z Wikieł P 4.3a moja wersja nr 386}


Obliczyć granicę funkcji $\lim\limits_{x\to\ 0}\frac{20 \cdot x}{tan(13 \cdot x)}$.
\zadStop
\rozwStart{Patryk Wirkus}{}
$$\lim\limits_{x\to\ 0}\frac{20 \cdot x}{tan(13 \cdot x)}=\lim\limits_{x\to\ 0}\frac{20 \cdot x \cdot cos(13 \cdot x)}{sin(13 \cdot x)}=\lim\limits_{x\to\ 0}\frac{20 \cdot cos(13 \cdot x)}{\frac{sin(13 \cdot x)}{x}}=\lim\limits_{x\to\ 0}\frac{20 \cdot cos(13 \cdot x)}{13 \cdot \frac{sin(13 \cdot x)}{13 \cdot x}} = \frac{20}{13}$$
\rozwStop
\odpStart
$\frac{20}{13}$
\odpStop
\testStart
A.$\frac{20}{13}$
B.$\infty$
C.$-\infty$
D.$0$
E.$-\frac{20}{13}$
F.$\frac{13}{20}$
G.$-\frac{13}{20}$
H.$13$
I.$20$
\testStop
\kluczStart
A
\kluczStop



\zadStart{Przykład z Wikieł P 4.3a moja wersja nr 387}


Obliczyć granicę funkcji $\lim\limits_{x\to\ 0}\frac{20 \cdot x}{tan(17 \cdot x)}$.
\zadStop
\rozwStart{Patryk Wirkus}{}
$$\lim\limits_{x\to\ 0}\frac{20 \cdot x}{tan(17 \cdot x)}=\lim\limits_{x\to\ 0}\frac{20 \cdot x \cdot cos(17 \cdot x)}{sin(17 \cdot x)}=\lim\limits_{x\to\ 0}\frac{20 \cdot cos(17 \cdot x)}{\frac{sin(17 \cdot x)}{x}}=\lim\limits_{x\to\ 0}\frac{20 \cdot cos(17 \cdot x)}{17 \cdot \frac{sin(17 \cdot x)}{17 \cdot x}} = \frac{20}{17}$$
\rozwStop
\odpStart
$\frac{20}{17}$
\odpStop
\testStart
A.$\frac{20}{17}$
B.$\infty$
C.$-\infty$
D.$0$
E.$-\frac{20}{17}$
F.$\frac{17}{20}$
G.$-\frac{17}{20}$
H.$17$
I.$20$
\testStop
\kluczStart
A
\kluczStop



\zadStart{Przykład z Wikieł P 4.3a moja wersja nr 388}


Obliczyć granicę funkcji $\lim\limits_{x\to\ 0}\frac{20 \cdot x}{tan(19 \cdot x)}$.
\zadStop
\rozwStart{Patryk Wirkus}{}
$$\lim\limits_{x\to\ 0}\frac{20 \cdot x}{tan(19 \cdot x)}=\lim\limits_{x\to\ 0}\frac{20 \cdot x \cdot cos(19 \cdot x)}{sin(19 \cdot x)}=\lim\limits_{x\to\ 0}\frac{20 \cdot cos(19 \cdot x)}{\frac{sin(19 \cdot x)}{x}}=\lim\limits_{x\to\ 0}\frac{20 \cdot cos(19 \cdot x)}{19 \cdot \frac{sin(19 \cdot x)}{19 \cdot x}} = \frac{20}{19}$$
\rozwStop
\odpStart
$\frac{20}{19}$
\odpStop
\testStart
A.$\frac{20}{19}$
B.$\infty$
C.$-\infty$
D.$0$
E.$-\frac{20}{19}$
F.$\frac{19}{20}$
G.$-\frac{19}{20}$
H.$19$
I.$20$
\testStop
\kluczStart
A
\kluczStop



\zadStart{Przykład z Wikieł P 4.3a moja wersja nr 389}


Obliczyć granicę funkcji $\lim\limits_{x\to\ 0}\frac{20 \cdot x}{tan(21 \cdot x)}$.
\zadStop
\rozwStart{Patryk Wirkus}{}
$$\lim\limits_{x\to\ 0}\frac{20 \cdot x}{tan(21 \cdot x)}=\lim\limits_{x\to\ 0}\frac{20 \cdot x \cdot cos(21 \cdot x)}{sin(21 \cdot x)}=\lim\limits_{x\to\ 0}\frac{20 \cdot cos(21 \cdot x)}{\frac{sin(21 \cdot x)}{x}}=\lim\limits_{x\to\ 0}\frac{20 \cdot cos(21 \cdot x)}{21 \cdot \frac{sin(21 \cdot x)}{21 \cdot x}} = \frac{20}{21}$$
\rozwStop
\odpStart
$\frac{20}{21}$
\odpStop
\testStart
A.$\frac{20}{21}$
B.$\infty$
C.$-\infty$
D.$0$
E.$-\frac{20}{21}$
F.$\frac{21}{20}$
G.$-\frac{21}{20}$
H.$21$
I.$20$
\testStop
\kluczStart
A
\kluczStop



\zadStart{Przykład z Wikieł P 4.3a moja wersja nr 390}


Obliczyć granicę funkcji $\lim\limits_{x\to\ 0}\frac{20 \cdot x}{tan(23 \cdot x)}$.
\zadStop
\rozwStart{Patryk Wirkus}{}
$$\lim\limits_{x\to\ 0}\frac{20 \cdot x}{tan(23 \cdot x)}=\lim\limits_{x\to\ 0}\frac{20 \cdot x \cdot cos(23 \cdot x)}{sin(23 \cdot x)}=\lim\limits_{x\to\ 0}\frac{20 \cdot cos(23 \cdot x)}{\frac{sin(23 \cdot x)}{x}}=\lim\limits_{x\to\ 0}\frac{20 \cdot cos(23 \cdot x)}{23 \cdot \frac{sin(23 \cdot x)}{23 \cdot x}} = \frac{20}{23}$$
\rozwStop
\odpStart
$\frac{20}{23}$
\odpStop
\testStart
A.$\frac{20}{23}$
B.$\infty$
C.$-\infty$
D.$0$
E.$-\frac{20}{23}$
F.$\frac{23}{20}$
G.$-\frac{23}{20}$
H.$23$
I.$20$
\testStop
\kluczStart
A
\kluczStop



\zadStart{Przykład z Wikieł P 4.3a moja wersja nr 391}


Obliczyć granicę funkcji $\lim\limits_{x\to\ 0}\frac{20 \cdot x}{tan(27 \cdot x)}$.
\zadStop
\rozwStart{Patryk Wirkus}{}
$$\lim\limits_{x\to\ 0}\frac{20 \cdot x}{tan(27 \cdot x)}=\lim\limits_{x\to\ 0}\frac{20 \cdot x \cdot cos(27 \cdot x)}{sin(27 \cdot x)}=\lim\limits_{x\to\ 0}\frac{20 \cdot cos(27 \cdot x)}{\frac{sin(27 \cdot x)}{x}}=\lim\limits_{x\to\ 0}\frac{20 \cdot cos(27 \cdot x)}{27 \cdot \frac{sin(27 \cdot x)}{27 \cdot x}} = \frac{20}{27}$$
\rozwStop
\odpStart
$\frac{20}{27}$
\odpStop
\testStart
A.$\frac{20}{27}$
B.$\infty$
C.$-\infty$
D.$0$
E.$-\frac{20}{27}$
F.$\frac{27}{20}$
G.$-\frac{27}{20}$
H.$27$
I.$20$
\testStop
\kluczStart
A
\kluczStop



\zadStart{Przykład z Wikieł P 4.3a moja wersja nr 392}


Obliczyć granicę funkcji $\lim\limits_{x\to\ 0}\frac{20 \cdot x}{tan(29 \cdot x)}$.
\zadStop
\rozwStart{Patryk Wirkus}{}
$$\lim\limits_{x\to\ 0}\frac{20 \cdot x}{tan(29 \cdot x)}=\lim\limits_{x\to\ 0}\frac{20 \cdot x \cdot cos(29 \cdot x)}{sin(29 \cdot x)}=\lim\limits_{x\to\ 0}\frac{20 \cdot cos(29 \cdot x)}{\frac{sin(29 \cdot x)}{x}}=\lim\limits_{x\to\ 0}\frac{20 \cdot cos(29 \cdot x)}{29 \cdot \frac{sin(29 \cdot x)}{29 \cdot x}} = \frac{20}{29}$$
\rozwStop
\odpStart
$\frac{20}{29}$
\odpStop
\testStart
A.$\frac{20}{29}$
B.$\infty$
C.$-\infty$
D.$0$
E.$-\frac{20}{29}$
F.$\frac{29}{20}$
G.$-\frac{29}{20}$
H.$29$
I.$20$
\testStop
\kluczStart
A
\kluczStop



\zadStart{Przykład z Wikieł P 4.3a moja wersja nr 393}


Obliczyć granicę funkcji $\lim\limits_{x\to\ 0}\frac{20 \cdot x}{tan(31 \cdot x)}$.
\zadStop
\rozwStart{Patryk Wirkus}{}
$$\lim\limits_{x\to\ 0}\frac{20 \cdot x}{tan(31 \cdot x)}=\lim\limits_{x\to\ 0}\frac{20 \cdot x \cdot cos(31 \cdot x)}{sin(31 \cdot x)}=\lim\limits_{x\to\ 0}\frac{20 \cdot cos(31 \cdot x)}{\frac{sin(31 \cdot x)}{x}}=\lim\limits_{x\to\ 0}\frac{20 \cdot cos(31 \cdot x)}{31 \cdot \frac{sin(31 \cdot x)}{31 \cdot x}} = \frac{20}{31}$$
\rozwStop
\odpStart
$\frac{20}{31}$
\odpStop
\testStart
A.$\frac{20}{31}$
B.$\infty$
C.$-\infty$
D.$0$
E.$-\frac{20}{31}$
F.$\frac{31}{20}$
G.$-\frac{31}{20}$
H.$31$
I.$20$
\testStop
\kluczStart
A
\kluczStop



\zadStart{Przykład z Wikieł P 4.3a moja wersja nr 394}


Obliczyć granicę funkcji $\lim\limits_{x\to\ 0}\frac{20 \cdot x}{tan(33 \cdot x)}$.
\zadStop
\rozwStart{Patryk Wirkus}{}
$$\lim\limits_{x\to\ 0}\frac{20 \cdot x}{tan(33 \cdot x)}=\lim\limits_{x\to\ 0}\frac{20 \cdot x \cdot cos(33 \cdot x)}{sin(33 \cdot x)}=\lim\limits_{x\to\ 0}\frac{20 \cdot cos(33 \cdot x)}{\frac{sin(33 \cdot x)}{x}}=\lim\limits_{x\to\ 0}\frac{20 \cdot cos(33 \cdot x)}{33 \cdot \frac{sin(33 \cdot x)}{33 \cdot x}} = \frac{20}{33}$$
\rozwStop
\odpStart
$\frac{20}{33}$
\odpStop
\testStart
A.$\frac{20}{33}$
B.$\infty$
C.$-\infty$
D.$0$
E.$-\frac{20}{33}$
F.$\frac{33}{20}$
G.$-\frac{33}{20}$
H.$33$
I.$20$
\testStop
\kluczStart
A
\kluczStop



\zadStart{Przykład z Wikieł P 4.3a moja wersja nr 395}


Obliczyć granicę funkcji $\lim\limits_{x\to\ 0}\frac{20 \cdot x}{tan(37 \cdot x)}$.
\zadStop
\rozwStart{Patryk Wirkus}{}
$$\lim\limits_{x\to\ 0}\frac{20 \cdot x}{tan(37 \cdot x)}=\lim\limits_{x\to\ 0}\frac{20 \cdot x \cdot cos(37 \cdot x)}{sin(37 \cdot x)}=\lim\limits_{x\to\ 0}\frac{20 \cdot cos(37 \cdot x)}{\frac{sin(37 \cdot x)}{x}}=\lim\limits_{x\to\ 0}\frac{20 \cdot cos(37 \cdot x)}{37 \cdot \frac{sin(37 \cdot x)}{37 \cdot x}} = \frac{20}{37}$$
\rozwStop
\odpStart
$\frac{20}{37}$
\odpStop
\testStart
A.$\frac{20}{37}$
B.$\infty$
C.$-\infty$
D.$0$
E.$-\frac{20}{37}$
F.$\frac{37}{20}$
G.$-\frac{37}{20}$
H.$37$
I.$20$
\testStop
\kluczStart
A
\kluczStop



\zadStart{Przykład z Wikieł P 4.3a moja wersja nr 396}


Obliczyć granicę funkcji $\lim\limits_{x\to\ 0}\frac{20 \cdot x}{tan(39 \cdot x)}$.
\zadStop
\rozwStart{Patryk Wirkus}{}
$$\lim\limits_{x\to\ 0}\frac{20 \cdot x}{tan(39 \cdot x)}=\lim\limits_{x\to\ 0}\frac{20 \cdot x \cdot cos(39 \cdot x)}{sin(39 \cdot x)}=\lim\limits_{x\to\ 0}\frac{20 \cdot cos(39 \cdot x)}{\frac{sin(39 \cdot x)}{x}}=\lim\limits_{x\to\ 0}\frac{20 \cdot cos(39 \cdot x)}{39 \cdot \frac{sin(39 \cdot x)}{39 \cdot x}} = \frac{20}{39}$$
\rozwStop
\odpStart
$\frac{20}{39}$
\odpStop
\testStart
A.$\frac{20}{39}$
B.$\infty$
C.$-\infty$
D.$0$
E.$-\frac{20}{39}$
F.$\frac{39}{20}$
G.$-\frac{39}{20}$
H.$39$
I.$20$
\testStop
\kluczStart
A
\kluczStop



\zadStart{Przykład z Wikieł P 4.3a moja wersja nr 397}


Obliczyć granicę funkcji $\lim\limits_{x\to\ 0}\frac{21 \cdot x}{tan(2 \cdot x)}$.
\zadStop
\rozwStart{Patryk Wirkus}{}
$$\lim\limits_{x\to\ 0}\frac{21 \cdot x}{tan(2 \cdot x)}=\lim\limits_{x\to\ 0}\frac{21 \cdot x \cdot cos(2 \cdot x)}{sin(2 \cdot x)}=\lim\limits_{x\to\ 0}\frac{21 \cdot cos(2 \cdot x)}{\frac{sin(2 \cdot x)}{x}}=\lim\limits_{x\to\ 0}\frac{21 \cdot cos(2 \cdot x)}{2 \cdot \frac{sin(2 \cdot x)}{2 \cdot x}} = \frac{21}{2}$$
\rozwStop
\odpStart
$\frac{21}{2}$
\odpStop
\testStart
A.$\frac{21}{2}$
B.$\infty$
C.$-\infty$
D.$0$
E.$-\frac{21}{2}$
F.$\frac{2}{21}$
G.$-\frac{2}{21}$
H.$2$
I.$21$
\testStop
\kluczStart
A
\kluczStop



\zadStart{Przykład z Wikieł P 4.3a moja wersja nr 398}


Obliczyć granicę funkcji $\lim\limits_{x\to\ 0}\frac{21 \cdot x}{tan(4 \cdot x)}$.
\zadStop
\rozwStart{Patryk Wirkus}{}
$$\lim\limits_{x\to\ 0}\frac{21 \cdot x}{tan(4 \cdot x)}=\lim\limits_{x\to\ 0}\frac{21 \cdot x \cdot cos(4 \cdot x)}{sin(4 \cdot x)}=\lim\limits_{x\to\ 0}\frac{21 \cdot cos(4 \cdot x)}{\frac{sin(4 \cdot x)}{x}}=\lim\limits_{x\to\ 0}\frac{21 \cdot cos(4 \cdot x)}{4 \cdot \frac{sin(4 \cdot x)}{4 \cdot x}} = \frac{21}{4}$$
\rozwStop
\odpStart
$\frac{21}{4}$
\odpStop
\testStart
A.$\frac{21}{4}$
B.$\infty$
C.$-\infty$
D.$0$
E.$-\frac{21}{4}$
F.$\frac{4}{21}$
G.$-\frac{4}{21}$
H.$4$
I.$21$
\testStop
\kluczStart
A
\kluczStop



\zadStart{Przykład z Wikieł P 4.3a moja wersja nr 399}


Obliczyć granicę funkcji $\lim\limits_{x\to\ 0}\frac{21 \cdot x}{tan(5 \cdot x)}$.
\zadStop
\rozwStart{Patryk Wirkus}{}
$$\lim\limits_{x\to\ 0}\frac{21 \cdot x}{tan(5 \cdot x)}=\lim\limits_{x\to\ 0}\frac{21 \cdot x \cdot cos(5 \cdot x)}{sin(5 \cdot x)}=\lim\limits_{x\to\ 0}\frac{21 \cdot cos(5 \cdot x)}{\frac{sin(5 \cdot x)}{x}}=\lim\limits_{x\to\ 0}\frac{21 \cdot cos(5 \cdot x)}{5 \cdot \frac{sin(5 \cdot x)}{5 \cdot x}} = \frac{21}{5}$$
\rozwStop
\odpStart
$\frac{21}{5}$
\odpStop
\testStart
A.$\frac{21}{5}$
B.$\infty$
C.$-\infty$
D.$0$
E.$-\frac{21}{5}$
F.$\frac{5}{21}$
G.$-\frac{5}{21}$
H.$5$
I.$21$
\testStop
\kluczStart
A
\kluczStop



\zadStart{Przykład z Wikieł P 4.3a moja wersja nr 400}


Obliczyć granicę funkcji $\lim\limits_{x\to\ 0}\frac{21 \cdot x}{tan(8 \cdot x)}$.
\zadStop
\rozwStart{Patryk Wirkus}{}
$$\lim\limits_{x\to\ 0}\frac{21 \cdot x}{tan(8 \cdot x)}=\lim\limits_{x\to\ 0}\frac{21 \cdot x \cdot cos(8 \cdot x)}{sin(8 \cdot x)}=\lim\limits_{x\to\ 0}\frac{21 \cdot cos(8 \cdot x)}{\frac{sin(8 \cdot x)}{x}}=\lim\limits_{x\to\ 0}\frac{21 \cdot cos(8 \cdot x)}{8 \cdot \frac{sin(8 \cdot x)}{8 \cdot x}} = \frac{21}{8}$$
\rozwStop
\odpStart
$\frac{21}{8}$
\odpStop
\testStart
A.$\frac{21}{8}$
B.$\infty$
C.$-\infty$
D.$0$
E.$-\frac{21}{8}$
F.$\frac{8}{21}$
G.$-\frac{8}{21}$
H.$8$
I.$21$
\testStop
\kluczStart
A
\kluczStop



\zadStart{Przykład z Wikieł P 4.3a moja wersja nr 401}


Obliczyć granicę funkcji $\lim\limits_{x\to\ 0}\frac{21 \cdot x}{tan(10 \cdot x)}$.
\zadStop
\rozwStart{Patryk Wirkus}{}
$$\lim\limits_{x\to\ 0}\frac{21 \cdot x}{tan(10 \cdot x)}=\lim\limits_{x\to\ 0}\frac{21 \cdot x \cdot cos(10 \cdot x)}{sin(10 \cdot x)}=\lim\limits_{x\to\ 0}\frac{21 \cdot cos(10 \cdot x)}{\frac{sin(10 \cdot x)}{x}}=\lim\limits_{x\to\ 0}\frac{21 \cdot cos(10 \cdot x)}{10 \cdot \frac{sin(10 \cdot x)}{10 \cdot x}} = \frac{21}{10}$$
\rozwStop
\odpStart
$\frac{21}{10}$
\odpStop
\testStart
A.$\frac{21}{10}$
B.$\infty$
C.$-\infty$
D.$0$
E.$-\frac{21}{10}$
F.$\frac{10}{21}$
G.$-\frac{10}{21}$
H.$10$
I.$21$
\testStop
\kluczStart
A
\kluczStop



\zadStart{Przykład z Wikieł P 4.3a moja wersja nr 402}


Obliczyć granicę funkcji $\lim\limits_{x\to\ 0}\frac{21 \cdot x}{tan(11 \cdot x)}$.
\zadStop
\rozwStart{Patryk Wirkus}{}
$$\lim\limits_{x\to\ 0}\frac{21 \cdot x}{tan(11 \cdot x)}=\lim\limits_{x\to\ 0}\frac{21 \cdot x \cdot cos(11 \cdot x)}{sin(11 \cdot x)}=\lim\limits_{x\to\ 0}\frac{21 \cdot cos(11 \cdot x)}{\frac{sin(11 \cdot x)}{x}}=\lim\limits_{x\to\ 0}\frac{21 \cdot cos(11 \cdot x)}{11 \cdot \frac{sin(11 \cdot x)}{11 \cdot x}} = \frac{21}{11}$$
\rozwStop
\odpStart
$\frac{21}{11}$
\odpStop
\testStart
A.$\frac{21}{11}$
B.$\infty$
C.$-\infty$
D.$0$
E.$-\frac{21}{11}$
F.$\frac{11}{21}$
G.$-\frac{11}{21}$
H.$11$
I.$21$
\testStop
\kluczStart
A
\kluczStop



\zadStart{Przykład z Wikieł P 4.3a moja wersja nr 403}


Obliczyć granicę funkcji $\lim\limits_{x\to\ 0}\frac{21 \cdot x}{tan(13 \cdot x)}$.
\zadStop
\rozwStart{Patryk Wirkus}{}
$$\lim\limits_{x\to\ 0}\frac{21 \cdot x}{tan(13 \cdot x)}=\lim\limits_{x\to\ 0}\frac{21 \cdot x \cdot cos(13 \cdot x)}{sin(13 \cdot x)}=\lim\limits_{x\to\ 0}\frac{21 \cdot cos(13 \cdot x)}{\frac{sin(13 \cdot x)}{x}}=\lim\limits_{x\to\ 0}\frac{21 \cdot cos(13 \cdot x)}{13 \cdot \frac{sin(13 \cdot x)}{13 \cdot x}} = \frac{21}{13}$$
\rozwStop
\odpStart
$\frac{21}{13}$
\odpStop
\testStart
A.$\frac{21}{13}$
B.$\infty$
C.$-\infty$
D.$0$
E.$-\frac{21}{13}$
F.$\frac{13}{21}$
G.$-\frac{13}{21}$
H.$13$
I.$21$
\testStop
\kluczStart
A
\kluczStop



\zadStart{Przykład z Wikieł P 4.3a moja wersja nr 404}


Obliczyć granicę funkcji $\lim\limits_{x\to\ 0}\frac{21 \cdot x}{tan(16 \cdot x)}$.
\zadStop
\rozwStart{Patryk Wirkus}{}
$$\lim\limits_{x\to\ 0}\frac{21 \cdot x}{tan(16 \cdot x)}=\lim\limits_{x\to\ 0}\frac{21 \cdot x \cdot cos(16 \cdot x)}{sin(16 \cdot x)}=\lim\limits_{x\to\ 0}\frac{21 \cdot cos(16 \cdot x)}{\frac{sin(16 \cdot x)}{x}}=\lim\limits_{x\to\ 0}\frac{21 \cdot cos(16 \cdot x)}{16 \cdot \frac{sin(16 \cdot x)}{16 \cdot x}} = \frac{21}{16}$$
\rozwStop
\odpStart
$\frac{21}{16}$
\odpStop
\testStart
A.$\frac{21}{16}$
B.$\infty$
C.$-\infty$
D.$0$
E.$-\frac{21}{16}$
F.$\frac{16}{21}$
G.$-\frac{16}{21}$
H.$16$
I.$21$
\testStop
\kluczStart
A
\kluczStop



\zadStart{Przykład z Wikieł P 4.3a moja wersja nr 405}


Obliczyć granicę funkcji $\lim\limits_{x\to\ 0}\frac{21 \cdot x}{tan(17 \cdot x)}$.
\zadStop
\rozwStart{Patryk Wirkus}{}
$$\lim\limits_{x\to\ 0}\frac{21 \cdot x}{tan(17 \cdot x)}=\lim\limits_{x\to\ 0}\frac{21 \cdot x \cdot cos(17 \cdot x)}{sin(17 \cdot x)}=\lim\limits_{x\to\ 0}\frac{21 \cdot cos(17 \cdot x)}{\frac{sin(17 \cdot x)}{x}}=\lim\limits_{x\to\ 0}\frac{21 \cdot cos(17 \cdot x)}{17 \cdot \frac{sin(17 \cdot x)}{17 \cdot x}} = \frac{21}{17}$$
\rozwStop
\odpStart
$\frac{21}{17}$
\odpStop
\testStart
A.$\frac{21}{17}$
B.$\infty$
C.$-\infty$
D.$0$
E.$-\frac{21}{17}$
F.$\frac{17}{21}$
G.$-\frac{17}{21}$
H.$17$
I.$21$
\testStop
\kluczStart
A
\kluczStop



\zadStart{Przykład z Wikieł P 4.3a moja wersja nr 406}


Obliczyć granicę funkcji $\lim\limits_{x\to\ 0}\frac{21 \cdot x}{tan(19 \cdot x)}$.
\zadStop
\rozwStart{Patryk Wirkus}{}
$$\lim\limits_{x\to\ 0}\frac{21 \cdot x}{tan(19 \cdot x)}=\lim\limits_{x\to\ 0}\frac{21 \cdot x \cdot cos(19 \cdot x)}{sin(19 \cdot x)}=\lim\limits_{x\to\ 0}\frac{21 \cdot cos(19 \cdot x)}{\frac{sin(19 \cdot x)}{x}}=\lim\limits_{x\to\ 0}\frac{21 \cdot cos(19 \cdot x)}{19 \cdot \frac{sin(19 \cdot x)}{19 \cdot x}} = \frac{21}{19}$$
\rozwStop
\odpStart
$\frac{21}{19}$
\odpStop
\testStart
A.$\frac{21}{19}$
B.$\infty$
C.$-\infty$
D.$0$
E.$-\frac{21}{19}$
F.$\frac{19}{21}$
G.$-\frac{19}{21}$
H.$19$
I.$21$
\testStop
\kluczStart
A
\kluczStop



\zadStart{Przykład z Wikieł P 4.3a moja wersja nr 407}


Obliczyć granicę funkcji $\lim\limits_{x\to\ 0}\frac{21 \cdot x}{tan(20 \cdot x)}$.
\zadStop
\rozwStart{Patryk Wirkus}{}
$$\lim\limits_{x\to\ 0}\frac{21 \cdot x}{tan(20 \cdot x)}=\lim\limits_{x\to\ 0}\frac{21 \cdot x \cdot cos(20 \cdot x)}{sin(20 \cdot x)}=\lim\limits_{x\to\ 0}\frac{21 \cdot cos(20 \cdot x)}{\frac{sin(20 \cdot x)}{x}}=\lim\limits_{x\to\ 0}\frac{21 \cdot cos(20 \cdot x)}{20 \cdot \frac{sin(20 \cdot x)}{20 \cdot x}} = \frac{21}{20}$$
\rozwStop
\odpStart
$\frac{21}{20}$
\odpStop
\testStart
A.$\frac{21}{20}$
B.$\infty$
C.$-\infty$
D.$0$
E.$-\frac{21}{20}$
F.$\frac{20}{21}$
G.$-\frac{20}{21}$
H.$20$
I.$21$
\testStop
\kluczStart
A
\kluczStop



\zadStart{Przykład z Wikieł P 4.3a moja wersja nr 408}


Obliczyć granicę funkcji $\lim\limits_{x\to\ 0}\frac{21 \cdot x}{tan(22 \cdot x)}$.
\zadStop
\rozwStart{Patryk Wirkus}{}
$$\lim\limits_{x\to\ 0}\frac{21 \cdot x}{tan(22 \cdot x)}=\lim\limits_{x\to\ 0}\frac{21 \cdot x \cdot cos(22 \cdot x)}{sin(22 \cdot x)}=\lim\limits_{x\to\ 0}\frac{21 \cdot cos(22 \cdot x)}{\frac{sin(22 \cdot x)}{x}}=\lim\limits_{x\to\ 0}\frac{21 \cdot cos(22 \cdot x)}{22 \cdot \frac{sin(22 \cdot x)}{22 \cdot x}} = \frac{21}{22}$$
\rozwStop
\odpStart
$\frac{21}{22}$
\odpStop
\testStart
A.$\frac{21}{22}$
B.$\infty$
C.$-\infty$
D.$0$
E.$-\frac{21}{22}$
F.$\frac{22}{21}$
G.$-\frac{22}{21}$
H.$22$
I.$21$
\testStop
\kluczStart
A
\kluczStop



\zadStart{Przykład z Wikieł P 4.3a moja wersja nr 409}


Obliczyć granicę funkcji $\lim\limits_{x\to\ 0}\frac{21 \cdot x}{tan(23 \cdot x)}$.
\zadStop
\rozwStart{Patryk Wirkus}{}
$$\lim\limits_{x\to\ 0}\frac{21 \cdot x}{tan(23 \cdot x)}=\lim\limits_{x\to\ 0}\frac{21 \cdot x \cdot cos(23 \cdot x)}{sin(23 \cdot x)}=\lim\limits_{x\to\ 0}\frac{21 \cdot cos(23 \cdot x)}{\frac{sin(23 \cdot x)}{x}}=\lim\limits_{x\to\ 0}\frac{21 \cdot cos(23 \cdot x)}{23 \cdot \frac{sin(23 \cdot x)}{23 \cdot x}} = \frac{21}{23}$$
\rozwStop
\odpStart
$\frac{21}{23}$
\odpStop
\testStart
A.$\frac{21}{23}$
B.$\infty$
C.$-\infty$
D.$0$
E.$-\frac{21}{23}$
F.$\frac{23}{21}$
G.$-\frac{23}{21}$
H.$23$
I.$21$
\testStop
\kluczStart
A
\kluczStop



\zadStart{Przykład z Wikieł P 4.3a moja wersja nr 410}


Obliczyć granicę funkcji $\lim\limits_{x\to\ 0}\frac{21 \cdot x}{tan(25 \cdot x)}$.
\zadStop
\rozwStart{Patryk Wirkus}{}
$$\lim\limits_{x\to\ 0}\frac{21 \cdot x}{tan(25 \cdot x)}=\lim\limits_{x\to\ 0}\frac{21 \cdot x \cdot cos(25 \cdot x)}{sin(25 \cdot x)}=\lim\limits_{x\to\ 0}\frac{21 \cdot cos(25 \cdot x)}{\frac{sin(25 \cdot x)}{x}}=\lim\limits_{x\to\ 0}\frac{21 \cdot cos(25 \cdot x)}{25 \cdot \frac{sin(25 \cdot x)}{25 \cdot x}} = \frac{21}{25}$$
\rozwStop
\odpStart
$\frac{21}{25}$
\odpStop
\testStart
A.$\frac{21}{25}$
B.$\infty$
C.$-\infty$
D.$0$
E.$-\frac{21}{25}$
F.$\frac{25}{21}$
G.$-\frac{25}{21}$
H.$25$
I.$21$
\testStop
\kluczStart
A
\kluczStop



\zadStart{Przykład z Wikieł P 4.3a moja wersja nr 411}


Obliczyć granicę funkcji $\lim\limits_{x\to\ 0}\frac{21 \cdot x}{tan(26 \cdot x)}$.
\zadStop
\rozwStart{Patryk Wirkus}{}
$$\lim\limits_{x\to\ 0}\frac{21 \cdot x}{tan(26 \cdot x)}=\lim\limits_{x\to\ 0}\frac{21 \cdot x \cdot cos(26 \cdot x)}{sin(26 \cdot x)}=\lim\limits_{x\to\ 0}\frac{21 \cdot cos(26 \cdot x)}{\frac{sin(26 \cdot x)}{x}}=\lim\limits_{x\to\ 0}\frac{21 \cdot cos(26 \cdot x)}{26 \cdot \frac{sin(26 \cdot x)}{26 \cdot x}} = \frac{21}{26}$$
\rozwStop
\odpStart
$\frac{21}{26}$
\odpStop
\testStart
A.$\frac{21}{26}$
B.$\infty$
C.$-\infty$
D.$0$
E.$-\frac{21}{26}$
F.$\frac{26}{21}$
G.$-\frac{26}{21}$
H.$26$
I.$21$
\testStop
\kluczStart
A
\kluczStop



\zadStart{Przykład z Wikieł P 4.3a moja wersja nr 412}


Obliczyć granicę funkcji $\lim\limits_{x\to\ 0}\frac{21 \cdot x}{tan(29 \cdot x)}$.
\zadStop
\rozwStart{Patryk Wirkus}{}
$$\lim\limits_{x\to\ 0}\frac{21 \cdot x}{tan(29 \cdot x)}=\lim\limits_{x\to\ 0}\frac{21 \cdot x \cdot cos(29 \cdot x)}{sin(29 \cdot x)}=\lim\limits_{x\to\ 0}\frac{21 \cdot cos(29 \cdot x)}{\frac{sin(29 \cdot x)}{x}}=\lim\limits_{x\to\ 0}\frac{21 \cdot cos(29 \cdot x)}{29 \cdot \frac{sin(29 \cdot x)}{29 \cdot x}} = \frac{21}{29}$$
\rozwStop
\odpStart
$\frac{21}{29}$
\odpStop
\testStart
A.$\frac{21}{29}$
B.$\infty$
C.$-\infty$
D.$0$
E.$-\frac{21}{29}$
F.$\frac{29}{21}$
G.$-\frac{29}{21}$
H.$29$
I.$21$
\testStop
\kluczStart
A
\kluczStop



\zadStart{Przykład z Wikieł P 4.3a moja wersja nr 413}


Obliczyć granicę funkcji $\lim\limits_{x\to\ 0}\frac{21 \cdot x}{tan(31 \cdot x)}$.
\zadStop
\rozwStart{Patryk Wirkus}{}
$$\lim\limits_{x\to\ 0}\frac{21 \cdot x}{tan(31 \cdot x)}=\lim\limits_{x\to\ 0}\frac{21 \cdot x \cdot cos(31 \cdot x)}{sin(31 \cdot x)}=\lim\limits_{x\to\ 0}\frac{21 \cdot cos(31 \cdot x)}{\frac{sin(31 \cdot x)}{x}}=\lim\limits_{x\to\ 0}\frac{21 \cdot cos(31 \cdot x)}{31 \cdot \frac{sin(31 \cdot x)}{31 \cdot x}} = \frac{21}{31}$$
\rozwStop
\odpStart
$\frac{21}{31}$
\odpStop
\testStart
A.$\frac{21}{31}$
B.$\infty$
C.$-\infty$
D.$0$
E.$-\frac{21}{31}$
F.$\frac{31}{21}$
G.$-\frac{31}{21}$
H.$31$
I.$21$
\testStop
\kluczStart
A
\kluczStop



\zadStart{Przykład z Wikieł P 4.3a moja wersja nr 414}


Obliczyć granicę funkcji $\lim\limits_{x\to\ 0}\frac{21 \cdot x}{tan(32 \cdot x)}$.
\zadStop
\rozwStart{Patryk Wirkus}{}
$$\lim\limits_{x\to\ 0}\frac{21 \cdot x}{tan(32 \cdot x)}=\lim\limits_{x\to\ 0}\frac{21 \cdot x \cdot cos(32 \cdot x)}{sin(32 \cdot x)}=\lim\limits_{x\to\ 0}\frac{21 \cdot cos(32 \cdot x)}{\frac{sin(32 \cdot x)}{x}}=\lim\limits_{x\to\ 0}\frac{21 \cdot cos(32 \cdot x)}{32 \cdot \frac{sin(32 \cdot x)}{32 \cdot x}} = \frac{21}{32}$$
\rozwStop
\odpStart
$\frac{21}{32}$
\odpStop
\testStart
A.$\frac{21}{32}$
B.$\infty$
C.$-\infty$
D.$0$
E.$-\frac{21}{32}$
F.$\frac{32}{21}$
G.$-\frac{32}{21}$
H.$32$
I.$21$
\testStop
\kluczStart
A
\kluczStop



\zadStart{Przykład z Wikieł P 4.3a moja wersja nr 415}


Obliczyć granicę funkcji $\lim\limits_{x\to\ 0}\frac{21 \cdot x}{tan(34 \cdot x)}$.
\zadStop
\rozwStart{Patryk Wirkus}{}
$$\lim\limits_{x\to\ 0}\frac{21 \cdot x}{tan(34 \cdot x)}=\lim\limits_{x\to\ 0}\frac{21 \cdot x \cdot cos(34 \cdot x)}{sin(34 \cdot x)}=\lim\limits_{x\to\ 0}\frac{21 \cdot cos(34 \cdot x)}{\frac{sin(34 \cdot x)}{x}}=\lim\limits_{x\to\ 0}\frac{21 \cdot cos(34 \cdot x)}{34 \cdot \frac{sin(34 \cdot x)}{34 \cdot x}} = \frac{21}{34}$$
\rozwStop
\odpStart
$\frac{21}{34}$
\odpStop
\testStart
A.$\frac{21}{34}$
B.$\infty$
C.$-\infty$
D.$0$
E.$-\frac{21}{34}$
F.$\frac{34}{21}$
G.$-\frac{34}{21}$
H.$34$
I.$21$
\testStop
\kluczStart
A
\kluczStop



\zadStart{Przykład z Wikieł P 4.3a moja wersja nr 416}


Obliczyć granicę funkcji $\lim\limits_{x\to\ 0}\frac{21 \cdot x}{tan(37 \cdot x)}$.
\zadStop
\rozwStart{Patryk Wirkus}{}
$$\lim\limits_{x\to\ 0}\frac{21 \cdot x}{tan(37 \cdot x)}=\lim\limits_{x\to\ 0}\frac{21 \cdot x \cdot cos(37 \cdot x)}{sin(37 \cdot x)}=\lim\limits_{x\to\ 0}\frac{21 \cdot cos(37 \cdot x)}{\frac{sin(37 \cdot x)}{x}}=\lim\limits_{x\to\ 0}\frac{21 \cdot cos(37 \cdot x)}{37 \cdot \frac{sin(37 \cdot x)}{37 \cdot x}} = \frac{21}{37}$$
\rozwStop
\odpStart
$\frac{21}{37}$
\odpStop
\testStart
A.$\frac{21}{37}$
B.$\infty$
C.$-\infty$
D.$0$
E.$-\frac{21}{37}$
F.$\frac{37}{21}$
G.$-\frac{37}{21}$
H.$37$
I.$21$
\testStop
\kluczStart
A
\kluczStop



\zadStart{Przykład z Wikieł P 4.3a moja wersja nr 417}


Obliczyć granicę funkcji $\lim\limits_{x\to\ 0}\frac{21 \cdot x}{tan(38 \cdot x)}$.
\zadStop
\rozwStart{Patryk Wirkus}{}
$$\lim\limits_{x\to\ 0}\frac{21 \cdot x}{tan(38 \cdot x)}=\lim\limits_{x\to\ 0}\frac{21 \cdot x \cdot cos(38 \cdot x)}{sin(38 \cdot x)}=\lim\limits_{x\to\ 0}\frac{21 \cdot cos(38 \cdot x)}{\frac{sin(38 \cdot x)}{x}}=\lim\limits_{x\to\ 0}\frac{21 \cdot cos(38 \cdot x)}{38 \cdot \frac{sin(38 \cdot x)}{38 \cdot x}} = \frac{21}{38}$$
\rozwStop
\odpStart
$\frac{21}{38}$
\odpStop
\testStart
A.$\frac{21}{38}$
B.$\infty$
C.$-\infty$
D.$0$
E.$-\frac{21}{38}$
F.$\frac{38}{21}$
G.$-\frac{38}{21}$
H.$38$
I.$21$
\testStop
\kluczStart
A
\kluczStop



\zadStart{Przykład z Wikieł P 4.3a moja wersja nr 418}


Obliczyć granicę funkcji $\lim\limits_{x\to\ 0}\frac{21 \cdot x}{tan(40 \cdot x)}$.
\zadStop
\rozwStart{Patryk Wirkus}{}
$$\lim\limits_{x\to\ 0}\frac{21 \cdot x}{tan(40 \cdot x)}=\lim\limits_{x\to\ 0}\frac{21 \cdot x \cdot cos(40 \cdot x)}{sin(40 \cdot x)}=\lim\limits_{x\to\ 0}\frac{21 \cdot cos(40 \cdot x)}{\frac{sin(40 \cdot x)}{x}}=\lim\limits_{x\to\ 0}\frac{21 \cdot cos(40 \cdot x)}{40 \cdot \frac{sin(40 \cdot x)}{40 \cdot x}} = \frac{21}{40}$$
\rozwStop
\odpStart
$\frac{21}{40}$
\odpStop
\testStart
A.$\frac{21}{40}$
B.$\infty$
C.$-\infty$
D.$0$
E.$-\frac{21}{40}$
F.$\frac{40}{21}$
G.$-\frac{40}{21}$
H.$40$
I.$21$
\testStop
\kluczStart
A
\kluczStop



\zadStart{Przykład z Wikieł P 4.3a moja wersja nr 419}


Obliczyć granicę funkcji $\lim\limits_{x\to\ 0}\frac{22 \cdot x}{tan(3 \cdot x)}$.
\zadStop
\rozwStart{Patryk Wirkus}{}
$$\lim\limits_{x\to\ 0}\frac{22 \cdot x}{tan(3 \cdot x)}=\lim\limits_{x\to\ 0}\frac{22 \cdot x \cdot cos(3 \cdot x)}{sin(3 \cdot x)}=\lim\limits_{x\to\ 0}\frac{22 \cdot cos(3 \cdot x)}{\frac{sin(3 \cdot x)}{x}}=\lim\limits_{x\to\ 0}\frac{22 \cdot cos(3 \cdot x)}{3 \cdot \frac{sin(3 \cdot x)}{3 \cdot x}} = \frac{22}{3}$$
\rozwStop
\odpStart
$\frac{22}{3}$
\odpStop
\testStart
A.$\frac{22}{3}$
B.$\infty$
C.$-\infty$
D.$0$
E.$-\frac{22}{3}$
F.$\frac{3}{22}$
G.$-\frac{3}{22}$
H.$3$
I.$22$
\testStop
\kluczStart
A
\kluczStop



\zadStart{Przykład z Wikieł P 4.3a moja wersja nr 420}


Obliczyć granicę funkcji $\lim\limits_{x\to\ 0}\frac{22 \cdot x}{tan(5 \cdot x)}$.
\zadStop
\rozwStart{Patryk Wirkus}{}
$$\lim\limits_{x\to\ 0}\frac{22 \cdot x}{tan(5 \cdot x)}=\lim\limits_{x\to\ 0}\frac{22 \cdot x \cdot cos(5 \cdot x)}{sin(5 \cdot x)}=\lim\limits_{x\to\ 0}\frac{22 \cdot cos(5 \cdot x)}{\frac{sin(5 \cdot x)}{x}}=\lim\limits_{x\to\ 0}\frac{22 \cdot cos(5 \cdot x)}{5 \cdot \frac{sin(5 \cdot x)}{5 \cdot x}} = \frac{22}{5}$$
\rozwStop
\odpStart
$\frac{22}{5}$
\odpStop
\testStart
A.$\frac{22}{5}$
B.$\infty$
C.$-\infty$
D.$0$
E.$-\frac{22}{5}$
F.$\frac{5}{22}$
G.$-\frac{5}{22}$
H.$5$
I.$22$
\testStop
\kluczStart
A
\kluczStop



\zadStart{Przykład z Wikieł P 4.3a moja wersja nr 421}


Obliczyć granicę funkcji $\lim\limits_{x\to\ 0}\frac{22 \cdot x}{tan(7 \cdot x)}$.
\zadStop
\rozwStart{Patryk Wirkus}{}
$$\lim\limits_{x\to\ 0}\frac{22 \cdot x}{tan(7 \cdot x)}=\lim\limits_{x\to\ 0}\frac{22 \cdot x \cdot cos(7 \cdot x)}{sin(7 \cdot x)}=\lim\limits_{x\to\ 0}\frac{22 \cdot cos(7 \cdot x)}{\frac{sin(7 \cdot x)}{x}}=\lim\limits_{x\to\ 0}\frac{22 \cdot cos(7 \cdot x)}{7 \cdot \frac{sin(7 \cdot x)}{7 \cdot x}} = \frac{22}{7}$$
\rozwStop
\odpStart
$\frac{22}{7}$
\odpStop
\testStart
A.$\frac{22}{7}$
B.$\infty$
C.$-\infty$
D.$0$
E.$-\frac{22}{7}$
F.$\frac{7}{22}$
G.$-\frac{7}{22}$
H.$7$
I.$22$
\testStop
\kluczStart
A
\kluczStop



\zadStart{Przykład z Wikieł P 4.3a moja wersja nr 422}


Obliczyć granicę funkcji $\lim\limits_{x\to\ 0}\frac{22 \cdot x}{tan(9 \cdot x)}$.
\zadStop
\rozwStart{Patryk Wirkus}{}
$$\lim\limits_{x\to\ 0}\frac{22 \cdot x}{tan(9 \cdot x)}=\lim\limits_{x\to\ 0}\frac{22 \cdot x \cdot cos(9 \cdot x)}{sin(9 \cdot x)}=\lim\limits_{x\to\ 0}\frac{22 \cdot cos(9 \cdot x)}{\frac{sin(9 \cdot x)}{x}}=\lim\limits_{x\to\ 0}\frac{22 \cdot cos(9 \cdot x)}{9 \cdot \frac{sin(9 \cdot x)}{9 \cdot x}} = \frac{22}{9}$$
\rozwStop
\odpStart
$\frac{22}{9}$
\odpStop
\testStart
A.$\frac{22}{9}$
B.$\infty$
C.$-\infty$
D.$0$
E.$-\frac{22}{9}$
F.$\frac{9}{22}$
G.$-\frac{9}{22}$
H.$9$
I.$22$
\testStop
\kluczStart
A
\kluczStop



\zadStart{Przykład z Wikieł P 4.3a moja wersja nr 423}


Obliczyć granicę funkcji $\lim\limits_{x\to\ 0}\frac{22 \cdot x}{tan(13 \cdot x)}$.
\zadStop
\rozwStart{Patryk Wirkus}{}
$$\lim\limits_{x\to\ 0}\frac{22 \cdot x}{tan(13 \cdot x)}=\lim\limits_{x\to\ 0}\frac{22 \cdot x \cdot cos(13 \cdot x)}{sin(13 \cdot x)}=\lim\limits_{x\to\ 0}\frac{22 \cdot cos(13 \cdot x)}{\frac{sin(13 \cdot x)}{x}}=\lim\limits_{x\to\ 0}\frac{22 \cdot cos(13 \cdot x)}{13 \cdot \frac{sin(13 \cdot x)}{13 \cdot x}} = \frac{22}{13}$$
\rozwStop
\odpStart
$\frac{22}{13}$
\odpStop
\testStart
A.$\frac{22}{13}$
B.$\infty$
C.$-\infty$
D.$0$
E.$-\frac{22}{13}$
F.$\frac{13}{22}$
G.$-\frac{13}{22}$
H.$13$
I.$22$
\testStop
\kluczStart
A
\kluczStop



\zadStart{Przykład z Wikieł P 4.3a moja wersja nr 424}


Obliczyć granicę funkcji $\lim\limits_{x\to\ 0}\frac{22 \cdot x}{tan(15 \cdot x)}$.
\zadStop
\rozwStart{Patryk Wirkus}{}
$$\lim\limits_{x\to\ 0}\frac{22 \cdot x}{tan(15 \cdot x)}=\lim\limits_{x\to\ 0}\frac{22 \cdot x \cdot cos(15 \cdot x)}{sin(15 \cdot x)}=\lim\limits_{x\to\ 0}\frac{22 \cdot cos(15 \cdot x)}{\frac{sin(15 \cdot x)}{x}}=\lim\limits_{x\to\ 0}\frac{22 \cdot cos(15 \cdot x)}{15 \cdot \frac{sin(15 \cdot x)}{15 \cdot x}} = \frac{22}{15}$$
\rozwStop
\odpStart
$\frac{22}{15}$
\odpStop
\testStart
A.$\frac{22}{15}$
B.$\infty$
C.$-\infty$
D.$0$
E.$-\frac{22}{15}$
F.$\frac{15}{22}$
G.$-\frac{15}{22}$
H.$15$
I.$22$
\testStop
\kluczStart
A
\kluczStop



\zadStart{Przykład z Wikieł P 4.3a moja wersja nr 425}


Obliczyć granicę funkcji $\lim\limits_{x\to\ 0}\frac{22 \cdot x}{tan(17 \cdot x)}$.
\zadStop
\rozwStart{Patryk Wirkus}{}
$$\lim\limits_{x\to\ 0}\frac{22 \cdot x}{tan(17 \cdot x)}=\lim\limits_{x\to\ 0}\frac{22 \cdot x \cdot cos(17 \cdot x)}{sin(17 \cdot x)}=\lim\limits_{x\to\ 0}\frac{22 \cdot cos(17 \cdot x)}{\frac{sin(17 \cdot x)}{x}}=\lim\limits_{x\to\ 0}\frac{22 \cdot cos(17 \cdot x)}{17 \cdot \frac{sin(17 \cdot x)}{17 \cdot x}} = \frac{22}{17}$$
\rozwStop
\odpStart
$\frac{22}{17}$
\odpStop
\testStart
A.$\frac{22}{17}$
B.$\infty$
C.$-\infty$
D.$0$
E.$-\frac{22}{17}$
F.$\frac{17}{22}$
G.$-\frac{17}{22}$
H.$17$
I.$22$
\testStop
\kluczStart
A
\kluczStop



\zadStart{Przykład z Wikieł P 4.3a moja wersja nr 426}


Obliczyć granicę funkcji $\lim\limits_{x\to\ 0}\frac{22 \cdot x}{tan(19 \cdot x)}$.
\zadStop
\rozwStart{Patryk Wirkus}{}
$$\lim\limits_{x\to\ 0}\frac{22 \cdot x}{tan(19 \cdot x)}=\lim\limits_{x\to\ 0}\frac{22 \cdot x \cdot cos(19 \cdot x)}{sin(19 \cdot x)}=\lim\limits_{x\to\ 0}\frac{22 \cdot cos(19 \cdot x)}{\frac{sin(19 \cdot x)}{x}}=\lim\limits_{x\to\ 0}\frac{22 \cdot cos(19 \cdot x)}{19 \cdot \frac{sin(19 \cdot x)}{19 \cdot x}} = \frac{22}{19}$$
\rozwStop
\odpStart
$\frac{22}{19}$
\odpStop
\testStart
A.$\frac{22}{19}$
B.$\infty$
C.$-\infty$
D.$0$
E.$-\frac{22}{19}$
F.$\frac{19}{22}$
G.$-\frac{19}{22}$
H.$19$
I.$22$
\testStop
\kluczStart
A
\kluczStop



\zadStart{Przykład z Wikieł P 4.3a moja wersja nr 427}


Obliczyć granicę funkcji $\lim\limits_{x\to\ 0}\frac{22 \cdot x}{tan(21 \cdot x)}$.
\zadStop
\rozwStart{Patryk Wirkus}{}
$$\lim\limits_{x\to\ 0}\frac{22 \cdot x}{tan(21 \cdot x)}=\lim\limits_{x\to\ 0}\frac{22 \cdot x \cdot cos(21 \cdot x)}{sin(21 \cdot x)}=\lim\limits_{x\to\ 0}\frac{22 \cdot cos(21 \cdot x)}{\frac{sin(21 \cdot x)}{x}}=\lim\limits_{x\to\ 0}\frac{22 \cdot cos(21 \cdot x)}{21 \cdot \frac{sin(21 \cdot x)}{21 \cdot x}} = \frac{22}{21}$$
\rozwStop
\odpStart
$\frac{22}{21}$
\odpStop
\testStart
A.$\frac{22}{21}$
B.$\infty$
C.$-\infty$
D.$0$
E.$-\frac{22}{21}$
F.$\frac{21}{22}$
G.$-\frac{21}{22}$
H.$21$
I.$22$
\testStop
\kluczStart
A
\kluczStop



\zadStart{Przykład z Wikieł P 4.3a moja wersja nr 428}


Obliczyć granicę funkcji $\lim\limits_{x\to\ 0}\frac{22 \cdot x}{tan(23 \cdot x)}$.
\zadStop
\rozwStart{Patryk Wirkus}{}
$$\lim\limits_{x\to\ 0}\frac{22 \cdot x}{tan(23 \cdot x)}=\lim\limits_{x\to\ 0}\frac{22 \cdot x \cdot cos(23 \cdot x)}{sin(23 \cdot x)}=\lim\limits_{x\to\ 0}\frac{22 \cdot cos(23 \cdot x)}{\frac{sin(23 \cdot x)}{x}}=\lim\limits_{x\to\ 0}\frac{22 \cdot cos(23 \cdot x)}{23 \cdot \frac{sin(23 \cdot x)}{23 \cdot x}} = \frac{22}{23}$$
\rozwStop
\odpStart
$\frac{22}{23}$
\odpStop
\testStart
A.$\frac{22}{23}$
B.$\infty$
C.$-\infty$
D.$0$
E.$-\frac{22}{23}$
F.$\frac{23}{22}$
G.$-\frac{23}{22}$
H.$23$
I.$22$
\testStop
\kluczStart
A
\kluczStop



\zadStart{Przykład z Wikieł P 4.3a moja wersja nr 429}


Obliczyć granicę funkcji $\lim\limits_{x\to\ 0}\frac{22 \cdot x}{tan(25 \cdot x)}$.
\zadStop
\rozwStart{Patryk Wirkus}{}
$$\lim\limits_{x\to\ 0}\frac{22 \cdot x}{tan(25 \cdot x)}=\lim\limits_{x\to\ 0}\frac{22 \cdot x \cdot cos(25 \cdot x)}{sin(25 \cdot x)}=\lim\limits_{x\to\ 0}\frac{22 \cdot cos(25 \cdot x)}{\frac{sin(25 \cdot x)}{x}}=\lim\limits_{x\to\ 0}\frac{22 \cdot cos(25 \cdot x)}{25 \cdot \frac{sin(25 \cdot x)}{25 \cdot x}} = \frac{22}{25}$$
\rozwStop
\odpStart
$\frac{22}{25}$
\odpStop
\testStart
A.$\frac{22}{25}$
B.$\infty$
C.$-\infty$
D.$0$
E.$-\frac{22}{25}$
F.$\frac{25}{22}$
G.$-\frac{25}{22}$
H.$25$
I.$22$
\testStop
\kluczStart
A
\kluczStop



\zadStart{Przykład z Wikieł P 4.3a moja wersja nr 430}


Obliczyć granicę funkcji $\lim\limits_{x\to\ 0}\frac{22 \cdot x}{tan(27 \cdot x)}$.
\zadStop
\rozwStart{Patryk Wirkus}{}
$$\lim\limits_{x\to\ 0}\frac{22 \cdot x}{tan(27 \cdot x)}=\lim\limits_{x\to\ 0}\frac{22 \cdot x \cdot cos(27 \cdot x)}{sin(27 \cdot x)}=\lim\limits_{x\to\ 0}\frac{22 \cdot cos(27 \cdot x)}{\frac{sin(27 \cdot x)}{x}}=\lim\limits_{x\to\ 0}\frac{22 \cdot cos(27 \cdot x)}{27 \cdot \frac{sin(27 \cdot x)}{27 \cdot x}} = \frac{22}{27}$$
\rozwStop
\odpStart
$\frac{22}{27}$
\odpStop
\testStart
A.$\frac{22}{27}$
B.$\infty$
C.$-\infty$
D.$0$
E.$-\frac{22}{27}$
F.$\frac{27}{22}$
G.$-\frac{27}{22}$
H.$27$
I.$22$
\testStop
\kluczStart
A
\kluczStop



\zadStart{Przykład z Wikieł P 4.3a moja wersja nr 431}


Obliczyć granicę funkcji $\lim\limits_{x\to\ 0}\frac{22 \cdot x}{tan(29 \cdot x)}$.
\zadStop
\rozwStart{Patryk Wirkus}{}
$$\lim\limits_{x\to\ 0}\frac{22 \cdot x}{tan(29 \cdot x)}=\lim\limits_{x\to\ 0}\frac{22 \cdot x \cdot cos(29 \cdot x)}{sin(29 \cdot x)}=\lim\limits_{x\to\ 0}\frac{22 \cdot cos(29 \cdot x)}{\frac{sin(29 \cdot x)}{x}}=\lim\limits_{x\to\ 0}\frac{22 \cdot cos(29 \cdot x)}{29 \cdot \frac{sin(29 \cdot x)}{29 \cdot x}} = \frac{22}{29}$$
\rozwStop
\odpStart
$\frac{22}{29}$
\odpStop
\testStart
A.$\frac{22}{29}$
B.$\infty$
C.$-\infty$
D.$0$
E.$-\frac{22}{29}$
F.$\frac{29}{22}$
G.$-\frac{29}{22}$
H.$29$
I.$22$
\testStop
\kluczStart
A
\kluczStop



\zadStart{Przykład z Wikieł P 4.3a moja wersja nr 432}


Obliczyć granicę funkcji $\lim\limits_{x\to\ 0}\frac{22 \cdot x}{tan(31 \cdot x)}$.
\zadStop
\rozwStart{Patryk Wirkus}{}
$$\lim\limits_{x\to\ 0}\frac{22 \cdot x}{tan(31 \cdot x)}=\lim\limits_{x\to\ 0}\frac{22 \cdot x \cdot cos(31 \cdot x)}{sin(31 \cdot x)}=\lim\limits_{x\to\ 0}\frac{22 \cdot cos(31 \cdot x)}{\frac{sin(31 \cdot x)}{x}}=\lim\limits_{x\to\ 0}\frac{22 \cdot cos(31 \cdot x)}{31 \cdot \frac{sin(31 \cdot x)}{31 \cdot x}} = \frac{22}{31}$$
\rozwStop
\odpStart
$\frac{22}{31}$
\odpStop
\testStart
A.$\frac{22}{31}$
B.$\infty$
C.$-\infty$
D.$0$
E.$-\frac{22}{31}$
F.$\frac{31}{22}$
G.$-\frac{31}{22}$
H.$31$
I.$22$
\testStop
\kluczStart
A
\kluczStop



\zadStart{Przykład z Wikieł P 4.3a moja wersja nr 433}


Obliczyć granicę funkcji $\lim\limits_{x\to\ 0}\frac{22 \cdot x}{tan(35 \cdot x)}$.
\zadStop
\rozwStart{Patryk Wirkus}{}
$$\lim\limits_{x\to\ 0}\frac{22 \cdot x}{tan(35 \cdot x)}=\lim\limits_{x\to\ 0}\frac{22 \cdot x \cdot cos(35 \cdot x)}{sin(35 \cdot x)}=\lim\limits_{x\to\ 0}\frac{22 \cdot cos(35 \cdot x)}{\frac{sin(35 \cdot x)}{x}}=\lim\limits_{x\to\ 0}\frac{22 \cdot cos(35 \cdot x)}{35 \cdot \frac{sin(35 \cdot x)}{35 \cdot x}} = \frac{22}{35}$$
\rozwStop
\odpStart
$\frac{22}{35}$
\odpStop
\testStart
A.$\frac{22}{35}$
B.$\infty$
C.$-\infty$
D.$0$
E.$-\frac{22}{35}$
F.$\frac{35}{22}$
G.$-\frac{35}{22}$
H.$35$
I.$22$
\testStop
\kluczStart
A
\kluczStop



\zadStart{Przykład z Wikieł P 4.3a moja wersja nr 434}


Obliczyć granicę funkcji $\lim\limits_{x\to\ 0}\frac{22 \cdot x}{tan(37 \cdot x)}$.
\zadStop
\rozwStart{Patryk Wirkus}{}
$$\lim\limits_{x\to\ 0}\frac{22 \cdot x}{tan(37 \cdot x)}=\lim\limits_{x\to\ 0}\frac{22 \cdot x \cdot cos(37 \cdot x)}{sin(37 \cdot x)}=\lim\limits_{x\to\ 0}\frac{22 \cdot cos(37 \cdot x)}{\frac{sin(37 \cdot x)}{x}}=\lim\limits_{x\to\ 0}\frac{22 \cdot cos(37 \cdot x)}{37 \cdot \frac{sin(37 \cdot x)}{37 \cdot x}} = \frac{22}{37}$$
\rozwStop
\odpStart
$\frac{22}{37}$
\odpStop
\testStart
A.$\frac{22}{37}$
B.$\infty$
C.$-\infty$
D.$0$
E.$-\frac{22}{37}$
F.$\frac{37}{22}$
G.$-\frac{37}{22}$
H.$37$
I.$22$
\testStop
\kluczStart
A
\kluczStop



\zadStart{Przykład z Wikieł P 4.3a moja wersja nr 435}


Obliczyć granicę funkcji $\lim\limits_{x\to\ 0}\frac{22 \cdot x}{tan(39 \cdot x)}$.
\zadStop
\rozwStart{Patryk Wirkus}{}
$$\lim\limits_{x\to\ 0}\frac{22 \cdot x}{tan(39 \cdot x)}=\lim\limits_{x\to\ 0}\frac{22 \cdot x \cdot cos(39 \cdot x)}{sin(39 \cdot x)}=\lim\limits_{x\to\ 0}\frac{22 \cdot cos(39 \cdot x)}{\frac{sin(39 \cdot x)}{x}}=\lim\limits_{x\to\ 0}\frac{22 \cdot cos(39 \cdot x)}{39 \cdot \frac{sin(39 \cdot x)}{39 \cdot x}} = \frac{22}{39}$$
\rozwStop
\odpStart
$\frac{22}{39}$
\odpStop
\testStart
A.$\frac{22}{39}$
B.$\infty$
C.$-\infty$
D.$0$
E.$-\frac{22}{39}$
F.$\frac{39}{22}$
G.$-\frac{39}{22}$
H.$39$
I.$22$
\testStop
\kluczStart
A
\kluczStop



\zadStart{Przykład z Wikieł P 4.3a moja wersja nr 436}


Obliczyć granicę funkcji $\lim\limits_{x\to\ 0}\frac{23 \cdot x}{tan(2 \cdot x)}$.
\zadStop
\rozwStart{Patryk Wirkus}{}
$$\lim\limits_{x\to\ 0}\frac{23 \cdot x}{tan(2 \cdot x)}=\lim\limits_{x\to\ 0}\frac{23 \cdot x \cdot cos(2 \cdot x)}{sin(2 \cdot x)}=\lim\limits_{x\to\ 0}\frac{23 \cdot cos(2 \cdot x)}{\frac{sin(2 \cdot x)}{x}}=\lim\limits_{x\to\ 0}\frac{23 \cdot cos(2 \cdot x)}{2 \cdot \frac{sin(2 \cdot x)}{2 \cdot x}} = \frac{23}{2}$$
\rozwStop
\odpStart
$\frac{23}{2}$
\odpStop
\testStart
A.$\frac{23}{2}$
B.$\infty$
C.$-\infty$
D.$0$
E.$-\frac{23}{2}$
F.$\frac{2}{23}$
G.$-\frac{2}{23}$
H.$2$
I.$23$
\testStop
\kluczStart
A
\kluczStop



\zadStart{Przykład z Wikieł P 4.3a moja wersja nr 437}


Obliczyć granicę funkcji $\lim\limits_{x\to\ 0}\frac{23 \cdot x}{tan(3 \cdot x)}$.
\zadStop
\rozwStart{Patryk Wirkus}{}
$$\lim\limits_{x\to\ 0}\frac{23 \cdot x}{tan(3 \cdot x)}=\lim\limits_{x\to\ 0}\frac{23 \cdot x \cdot cos(3 \cdot x)}{sin(3 \cdot x)}=\lim\limits_{x\to\ 0}\frac{23 \cdot cos(3 \cdot x)}{\frac{sin(3 \cdot x)}{x}}=\lim\limits_{x\to\ 0}\frac{23 \cdot cos(3 \cdot x)}{3 \cdot \frac{sin(3 \cdot x)}{3 \cdot x}} = \frac{23}{3}$$
\rozwStop
\odpStart
$\frac{23}{3}$
\odpStop
\testStart
A.$\frac{23}{3}$
B.$\infty$
C.$-\infty$
D.$0$
E.$-\frac{23}{3}$
F.$\frac{3}{23}$
G.$-\frac{3}{23}$
H.$3$
I.$23$
\testStop
\kluczStart
A
\kluczStop



\zadStart{Przykład z Wikieł P 4.3a moja wersja nr 438}


Obliczyć granicę funkcji $\lim\limits_{x\to\ 0}\frac{23 \cdot x}{tan(4 \cdot x)}$.
\zadStop
\rozwStart{Patryk Wirkus}{}
$$\lim\limits_{x\to\ 0}\frac{23 \cdot x}{tan(4 \cdot x)}=\lim\limits_{x\to\ 0}\frac{23 \cdot x \cdot cos(4 \cdot x)}{sin(4 \cdot x)}=\lim\limits_{x\to\ 0}\frac{23 \cdot cos(4 \cdot x)}{\frac{sin(4 \cdot x)}{x}}=\lim\limits_{x\to\ 0}\frac{23 \cdot cos(4 \cdot x)}{4 \cdot \frac{sin(4 \cdot x)}{4 \cdot x}} = \frac{23}{4}$$
\rozwStop
\odpStart
$\frac{23}{4}$
\odpStop
\testStart
A.$\frac{23}{4}$
B.$\infty$
C.$-\infty$
D.$0$
E.$-\frac{23}{4}$
F.$\frac{4}{23}$
G.$-\frac{4}{23}$
H.$4$
I.$23$
\testStop
\kluczStart
A
\kluczStop



\zadStart{Przykład z Wikieł P 4.3a moja wersja nr 439}


Obliczyć granicę funkcji $\lim\limits_{x\to\ 0}\frac{23 \cdot x}{tan(5 \cdot x)}$.
\zadStop
\rozwStart{Patryk Wirkus}{}
$$\lim\limits_{x\to\ 0}\frac{23 \cdot x}{tan(5 \cdot x)}=\lim\limits_{x\to\ 0}\frac{23 \cdot x \cdot cos(5 \cdot x)}{sin(5 \cdot x)}=\lim\limits_{x\to\ 0}\frac{23 \cdot cos(5 \cdot x)}{\frac{sin(5 \cdot x)}{x}}=\lim\limits_{x\to\ 0}\frac{23 \cdot cos(5 \cdot x)}{5 \cdot \frac{sin(5 \cdot x)}{5 \cdot x}} = \frac{23}{5}$$
\rozwStop
\odpStart
$\frac{23}{5}$
\odpStop
\testStart
A.$\frac{23}{5}$
B.$\infty$
C.$-\infty$
D.$0$
E.$-\frac{23}{5}$
F.$\frac{5}{23}$
G.$-\frac{5}{23}$
H.$5$
I.$23$
\testStop
\kluczStart
A
\kluczStop



\zadStart{Przykład z Wikieł P 4.3a moja wersja nr 440}


Obliczyć granicę funkcji $\lim\limits_{x\to\ 0}\frac{23 \cdot x}{tan(6 \cdot x)}$.
\zadStop
\rozwStart{Patryk Wirkus}{}
$$\lim\limits_{x\to\ 0}\frac{23 \cdot x}{tan(6 \cdot x)}=\lim\limits_{x\to\ 0}\frac{23 \cdot x \cdot cos(6 \cdot x)}{sin(6 \cdot x)}=\lim\limits_{x\to\ 0}\frac{23 \cdot cos(6 \cdot x)}{\frac{sin(6 \cdot x)}{x}}=\lim\limits_{x\to\ 0}\frac{23 \cdot cos(6 \cdot x)}{6 \cdot \frac{sin(6 \cdot x)}{6 \cdot x}} = \frac{23}{6}$$
\rozwStop
\odpStart
$\frac{23}{6}$
\odpStop
\testStart
A.$\frac{23}{6}$
B.$\infty$
C.$-\infty$
D.$0$
E.$-\frac{23}{6}$
F.$\frac{6}{23}$
G.$-\frac{6}{23}$
H.$6$
I.$23$
\testStop
\kluczStart
A
\kluczStop



\zadStart{Przykład z Wikieł P 4.3a moja wersja nr 441}


Obliczyć granicę funkcji $\lim\limits_{x\to\ 0}\frac{23 \cdot x}{tan(7 \cdot x)}$.
\zadStop
\rozwStart{Patryk Wirkus}{}
$$\lim\limits_{x\to\ 0}\frac{23 \cdot x}{tan(7 \cdot x)}=\lim\limits_{x\to\ 0}\frac{23 \cdot x \cdot cos(7 \cdot x)}{sin(7 \cdot x)}=\lim\limits_{x\to\ 0}\frac{23 \cdot cos(7 \cdot x)}{\frac{sin(7 \cdot x)}{x}}=\lim\limits_{x\to\ 0}\frac{23 \cdot cos(7 \cdot x)}{7 \cdot \frac{sin(7 \cdot x)}{7 \cdot x}} = \frac{23}{7}$$
\rozwStop
\odpStart
$\frac{23}{7}$
\odpStop
\testStart
A.$\frac{23}{7}$
B.$\infty$
C.$-\infty$
D.$0$
E.$-\frac{23}{7}$
F.$\frac{7}{23}$
G.$-\frac{7}{23}$
H.$7$
I.$23$
\testStop
\kluczStart
A
\kluczStop



\zadStart{Przykład z Wikieł P 4.3a moja wersja nr 442}


Obliczyć granicę funkcji $\lim\limits_{x\to\ 0}\frac{23 \cdot x}{tan(8 \cdot x)}$.
\zadStop
\rozwStart{Patryk Wirkus}{}
$$\lim\limits_{x\to\ 0}\frac{23 \cdot x}{tan(8 \cdot x)}=\lim\limits_{x\to\ 0}\frac{23 \cdot x \cdot cos(8 \cdot x)}{sin(8 \cdot x)}=\lim\limits_{x\to\ 0}\frac{23 \cdot cos(8 \cdot x)}{\frac{sin(8 \cdot x)}{x}}=\lim\limits_{x\to\ 0}\frac{23 \cdot cos(8 \cdot x)}{8 \cdot \frac{sin(8 \cdot x)}{8 \cdot x}} = \frac{23}{8}$$
\rozwStop
\odpStart
$\frac{23}{8}$
\odpStop
\testStart
A.$\frac{23}{8}$
B.$\infty$
C.$-\infty$
D.$0$
E.$-\frac{23}{8}$
F.$\frac{8}{23}$
G.$-\frac{8}{23}$
H.$8$
I.$23$
\testStop
\kluczStart
A
\kluczStop



\zadStart{Przykład z Wikieł P 4.3a moja wersja nr 443}


Obliczyć granicę funkcji $\lim\limits_{x\to\ 0}\frac{23 \cdot x}{tan(9 \cdot x)}$.
\zadStop
\rozwStart{Patryk Wirkus}{}
$$\lim\limits_{x\to\ 0}\frac{23 \cdot x}{tan(9 \cdot x)}=\lim\limits_{x\to\ 0}\frac{23 \cdot x \cdot cos(9 \cdot x)}{sin(9 \cdot x)}=\lim\limits_{x\to\ 0}\frac{23 \cdot cos(9 \cdot x)}{\frac{sin(9 \cdot x)}{x}}=\lim\limits_{x\to\ 0}\frac{23 \cdot cos(9 \cdot x)}{9 \cdot \frac{sin(9 \cdot x)}{9 \cdot x}} = \frac{23}{9}$$
\rozwStop
\odpStart
$\frac{23}{9}$
\odpStop
\testStart
A.$\frac{23}{9}$
B.$\infty$
C.$-\infty$
D.$0$
E.$-\frac{23}{9}$
F.$\frac{9}{23}$
G.$-\frac{9}{23}$
H.$9$
I.$23$
\testStop
\kluczStart
A
\kluczStop



\zadStart{Przykład z Wikieł P 4.3a moja wersja nr 444}


Obliczyć granicę funkcji $\lim\limits_{x\to\ 0}\frac{23 \cdot x}{tan(10 \cdot x)}$.
\zadStop
\rozwStart{Patryk Wirkus}{}
$$\lim\limits_{x\to\ 0}\frac{23 \cdot x}{tan(10 \cdot x)}=\lim\limits_{x\to\ 0}\frac{23 \cdot x \cdot cos(10 \cdot x)}{sin(10 \cdot x)}=\lim\limits_{x\to\ 0}\frac{23 \cdot cos(10 \cdot x)}{\frac{sin(10 \cdot x)}{x}}=\lim\limits_{x\to\ 0}\frac{23 \cdot cos(10 \cdot x)}{10 \cdot \frac{sin(10 \cdot x)}{10 \cdot x}} = \frac{23}{10}$$
\rozwStop
\odpStart
$\frac{23}{10}$
\odpStop
\testStart
A.$\frac{23}{10}$
B.$\infty$
C.$-\infty$
D.$0$
E.$-\frac{23}{10}$
F.$\frac{10}{23}$
G.$-\frac{10}{23}$
H.$10$
I.$23$
\testStop
\kluczStart
A
\kluczStop



\zadStart{Przykład z Wikieł P 4.3a moja wersja nr 445}


Obliczyć granicę funkcji $\lim\limits_{x\to\ 0}\frac{23 \cdot x}{tan(11 \cdot x)}$.
\zadStop
\rozwStart{Patryk Wirkus}{}
$$\lim\limits_{x\to\ 0}\frac{23 \cdot x}{tan(11 \cdot x)}=\lim\limits_{x\to\ 0}\frac{23 \cdot x \cdot cos(11 \cdot x)}{sin(11 \cdot x)}=\lim\limits_{x\to\ 0}\frac{23 \cdot cos(11 \cdot x)}{\frac{sin(11 \cdot x)}{x}}=\lim\limits_{x\to\ 0}\frac{23 \cdot cos(11 \cdot x)}{11 \cdot \frac{sin(11 \cdot x)}{11 \cdot x}} = \frac{23}{11}$$
\rozwStop
\odpStart
$\frac{23}{11}$
\odpStop
\testStart
A.$\frac{23}{11}$
B.$\infty$
C.$-\infty$
D.$0$
E.$-\frac{23}{11}$
F.$\frac{11}{23}$
G.$-\frac{11}{23}$
H.$11$
I.$23$
\testStop
\kluczStart
A
\kluczStop



\zadStart{Przykład z Wikieł P 4.3a moja wersja nr 446}


Obliczyć granicę funkcji $\lim\limits_{x\to\ 0}\frac{23 \cdot x}{tan(12 \cdot x)}$.
\zadStop
\rozwStart{Patryk Wirkus}{}
$$\lim\limits_{x\to\ 0}\frac{23 \cdot x}{tan(12 \cdot x)}=\lim\limits_{x\to\ 0}\frac{23 \cdot x \cdot cos(12 \cdot x)}{sin(12 \cdot x)}=\lim\limits_{x\to\ 0}\frac{23 \cdot cos(12 \cdot x)}{\frac{sin(12 \cdot x)}{x}}=\lim\limits_{x\to\ 0}\frac{23 \cdot cos(12 \cdot x)}{12 \cdot \frac{sin(12 \cdot x)}{12 \cdot x}} = \frac{23}{12}$$
\rozwStop
\odpStart
$\frac{23}{12}$
\odpStop
\testStart
A.$\frac{23}{12}$
B.$\infty$
C.$-\infty$
D.$0$
E.$-\frac{23}{12}$
F.$\frac{12}{23}$
G.$-\frac{12}{23}$
H.$12$
I.$23$
\testStop
\kluczStart
A
\kluczStop



\zadStart{Przykład z Wikieł P 4.3a moja wersja nr 447}


Obliczyć granicę funkcji $\lim\limits_{x\to\ 0}\frac{23 \cdot x}{tan(13 \cdot x)}$.
\zadStop
\rozwStart{Patryk Wirkus}{}
$$\lim\limits_{x\to\ 0}\frac{23 \cdot x}{tan(13 \cdot x)}=\lim\limits_{x\to\ 0}\frac{23 \cdot x \cdot cos(13 \cdot x)}{sin(13 \cdot x)}=\lim\limits_{x\to\ 0}\frac{23 \cdot cos(13 \cdot x)}{\frac{sin(13 \cdot x)}{x}}=\lim\limits_{x\to\ 0}\frac{23 \cdot cos(13 \cdot x)}{13 \cdot \frac{sin(13 \cdot x)}{13 \cdot x}} = \frac{23}{13}$$
\rozwStop
\odpStart
$\frac{23}{13}$
\odpStop
\testStart
A.$\frac{23}{13}$
B.$\infty$
C.$-\infty$
D.$0$
E.$-\frac{23}{13}$
F.$\frac{13}{23}$
G.$-\frac{13}{23}$
H.$13$
I.$23$
\testStop
\kluczStart
A
\kluczStop



\zadStart{Przykład z Wikieł P 4.3a moja wersja nr 448}


Obliczyć granicę funkcji $\lim\limits_{x\to\ 0}\frac{23 \cdot x}{tan(14 \cdot x)}$.
\zadStop
\rozwStart{Patryk Wirkus}{}
$$\lim\limits_{x\to\ 0}\frac{23 \cdot x}{tan(14 \cdot x)}=\lim\limits_{x\to\ 0}\frac{23 \cdot x \cdot cos(14 \cdot x)}{sin(14 \cdot x)}=\lim\limits_{x\to\ 0}\frac{23 \cdot cos(14 \cdot x)}{\frac{sin(14 \cdot x)}{x}}=\lim\limits_{x\to\ 0}\frac{23 \cdot cos(14 \cdot x)}{14 \cdot \frac{sin(14 \cdot x)}{14 \cdot x}} = \frac{23}{14}$$
\rozwStop
\odpStart
$\frac{23}{14}$
\odpStop
\testStart
A.$\frac{23}{14}$
B.$\infty$
C.$-\infty$
D.$0$
E.$-\frac{23}{14}$
F.$\frac{14}{23}$
G.$-\frac{14}{23}$
H.$14$
I.$23$
\testStop
\kluczStart
A
\kluczStop



\zadStart{Przykład z Wikieł P 4.3a moja wersja nr 449}


Obliczyć granicę funkcji $\lim\limits_{x\to\ 0}\frac{23 \cdot x}{tan(15 \cdot x)}$.
\zadStop
\rozwStart{Patryk Wirkus}{}
$$\lim\limits_{x\to\ 0}\frac{23 \cdot x}{tan(15 \cdot x)}=\lim\limits_{x\to\ 0}\frac{23 \cdot x \cdot cos(15 \cdot x)}{sin(15 \cdot x)}=\lim\limits_{x\to\ 0}\frac{23 \cdot cos(15 \cdot x)}{\frac{sin(15 \cdot x)}{x}}=\lim\limits_{x\to\ 0}\frac{23 \cdot cos(15 \cdot x)}{15 \cdot \frac{sin(15 \cdot x)}{15 \cdot x}} = \frac{23}{15}$$
\rozwStop
\odpStart
$\frac{23}{15}$
\odpStop
\testStart
A.$\frac{23}{15}$
B.$\infty$
C.$-\infty$
D.$0$
E.$-\frac{23}{15}$
F.$\frac{15}{23}$
G.$-\frac{15}{23}$
H.$15$
I.$23$
\testStop
\kluczStart
A
\kluczStop



\zadStart{Przykład z Wikieł P 4.3a moja wersja nr 450}


Obliczyć granicę funkcji $\lim\limits_{x\to\ 0}\frac{23 \cdot x}{tan(16 \cdot x)}$.
\zadStop
\rozwStart{Patryk Wirkus}{}
$$\lim\limits_{x\to\ 0}\frac{23 \cdot x}{tan(16 \cdot x)}=\lim\limits_{x\to\ 0}\frac{23 \cdot x \cdot cos(16 \cdot x)}{sin(16 \cdot x)}=\lim\limits_{x\to\ 0}\frac{23 \cdot cos(16 \cdot x)}{\frac{sin(16 \cdot x)}{x}}=\lim\limits_{x\to\ 0}\frac{23 \cdot cos(16 \cdot x)}{16 \cdot \frac{sin(16 \cdot x)}{16 \cdot x}} = \frac{23}{16}$$
\rozwStop
\odpStart
$\frac{23}{16}$
\odpStop
\testStart
A.$\frac{23}{16}$
B.$\infty$
C.$-\infty$
D.$0$
E.$-\frac{23}{16}$
F.$\frac{16}{23}$
G.$-\frac{16}{23}$
H.$16$
I.$23$
\testStop
\kluczStart
A
\kluczStop



\zadStart{Przykład z Wikieł P 4.3a moja wersja nr 451}


Obliczyć granicę funkcji $\lim\limits_{x\to\ 0}\frac{23 \cdot x}{tan(17 \cdot x)}$.
\zadStop
\rozwStart{Patryk Wirkus}{}
$$\lim\limits_{x\to\ 0}\frac{23 \cdot x}{tan(17 \cdot x)}=\lim\limits_{x\to\ 0}\frac{23 \cdot x \cdot cos(17 \cdot x)}{sin(17 \cdot x)}=\lim\limits_{x\to\ 0}\frac{23 \cdot cos(17 \cdot x)}{\frac{sin(17 \cdot x)}{x}}=\lim\limits_{x\to\ 0}\frac{23 \cdot cos(17 \cdot x)}{17 \cdot \frac{sin(17 \cdot x)}{17 \cdot x}} = \frac{23}{17}$$
\rozwStop
\odpStart
$\frac{23}{17}$
\odpStop
\testStart
A.$\frac{23}{17}$
B.$\infty$
C.$-\infty$
D.$0$
E.$-\frac{23}{17}$
F.$\frac{17}{23}$
G.$-\frac{17}{23}$
H.$17$
I.$23$
\testStop
\kluczStart
A
\kluczStop



\zadStart{Przykład z Wikieł P 4.3a moja wersja nr 452}


Obliczyć granicę funkcji $\lim\limits_{x\to\ 0}\frac{23 \cdot x}{tan(18 \cdot x)}$.
\zadStop
\rozwStart{Patryk Wirkus}{}
$$\lim\limits_{x\to\ 0}\frac{23 \cdot x}{tan(18 \cdot x)}=\lim\limits_{x\to\ 0}\frac{23 \cdot x \cdot cos(18 \cdot x)}{sin(18 \cdot x)}=\lim\limits_{x\to\ 0}\frac{23 \cdot cos(18 \cdot x)}{\frac{sin(18 \cdot x)}{x}}=\lim\limits_{x\to\ 0}\frac{23 \cdot cos(18 \cdot x)}{18 \cdot \frac{sin(18 \cdot x)}{18 \cdot x}} = \frac{23}{18}$$
\rozwStop
\odpStart
$\frac{23}{18}$
\odpStop
\testStart
A.$\frac{23}{18}$
B.$\infty$
C.$-\infty$
D.$0$
E.$-\frac{23}{18}$
F.$\frac{18}{23}$
G.$-\frac{18}{23}$
H.$18$
I.$23$
\testStop
\kluczStart
A
\kluczStop



\zadStart{Przykład z Wikieł P 4.3a moja wersja nr 453}


Obliczyć granicę funkcji $\lim\limits_{x\to\ 0}\frac{23 \cdot x}{tan(19 \cdot x)}$.
\zadStop
\rozwStart{Patryk Wirkus}{}
$$\lim\limits_{x\to\ 0}\frac{23 \cdot x}{tan(19 \cdot x)}=\lim\limits_{x\to\ 0}\frac{23 \cdot x \cdot cos(19 \cdot x)}{sin(19 \cdot x)}=\lim\limits_{x\to\ 0}\frac{23 \cdot cos(19 \cdot x)}{\frac{sin(19 \cdot x)}{x}}=\lim\limits_{x\to\ 0}\frac{23 \cdot cos(19 \cdot x)}{19 \cdot \frac{sin(19 \cdot x)}{19 \cdot x}} = \frac{23}{19}$$
\rozwStop
\odpStart
$\frac{23}{19}$
\odpStop
\testStart
A.$\frac{23}{19}$
B.$\infty$
C.$-\infty$
D.$0$
E.$-\frac{23}{19}$
F.$\frac{19}{23}$
G.$-\frac{19}{23}$
H.$19$
I.$23$
\testStop
\kluczStart
A
\kluczStop



\zadStart{Przykład z Wikieł P 4.3a moja wersja nr 454}


Obliczyć granicę funkcji $\lim\limits_{x\to\ 0}\frac{23 \cdot x}{tan(20 \cdot x)}$.
\zadStop
\rozwStart{Patryk Wirkus}{}
$$\lim\limits_{x\to\ 0}\frac{23 \cdot x}{tan(20 \cdot x)}=\lim\limits_{x\to\ 0}\frac{23 \cdot x \cdot cos(20 \cdot x)}{sin(20 \cdot x)}=\lim\limits_{x\to\ 0}\frac{23 \cdot cos(20 \cdot x)}{\frac{sin(20 \cdot x)}{x}}=\lim\limits_{x\to\ 0}\frac{23 \cdot cos(20 \cdot x)}{20 \cdot \frac{sin(20 \cdot x)}{20 \cdot x}} = \frac{23}{20}$$
\rozwStop
\odpStart
$\frac{23}{20}$
\odpStop
\testStart
A.$\frac{23}{20}$
B.$\infty$
C.$-\infty$
D.$0$
E.$-\frac{23}{20}$
F.$\frac{20}{23}$
G.$-\frac{20}{23}$
H.$20$
I.$23$
\testStop
\kluczStart
A
\kluczStop



\zadStart{Przykład z Wikieł P 4.3a moja wersja nr 455}


Obliczyć granicę funkcji $\lim\limits_{x\to\ 0}\frac{23 \cdot x}{tan(21 \cdot x)}$.
\zadStop
\rozwStart{Patryk Wirkus}{}
$$\lim\limits_{x\to\ 0}\frac{23 \cdot x}{tan(21 \cdot x)}=\lim\limits_{x\to\ 0}\frac{23 \cdot x \cdot cos(21 \cdot x)}{sin(21 \cdot x)}=\lim\limits_{x\to\ 0}\frac{23 \cdot cos(21 \cdot x)}{\frac{sin(21 \cdot x)}{x}}=\lim\limits_{x\to\ 0}\frac{23 \cdot cos(21 \cdot x)}{21 \cdot \frac{sin(21 \cdot x)}{21 \cdot x}} = \frac{23}{21}$$
\rozwStop
\odpStart
$\frac{23}{21}$
\odpStop
\testStart
A.$\frac{23}{21}$
B.$\infty$
C.$-\infty$
D.$0$
E.$-\frac{23}{21}$
F.$\frac{21}{23}$
G.$-\frac{21}{23}$
H.$21$
I.$23$
\testStop
\kluczStart
A
\kluczStop



\zadStart{Przykład z Wikieł P 4.3a moja wersja nr 456}


Obliczyć granicę funkcji $\lim\limits_{x\to\ 0}\frac{23 \cdot x}{tan(22 \cdot x)}$.
\zadStop
\rozwStart{Patryk Wirkus}{}
$$\lim\limits_{x\to\ 0}\frac{23 \cdot x}{tan(22 \cdot x)}=\lim\limits_{x\to\ 0}\frac{23 \cdot x \cdot cos(22 \cdot x)}{sin(22 \cdot x)}=\lim\limits_{x\to\ 0}\frac{23 \cdot cos(22 \cdot x)}{\frac{sin(22 \cdot x)}{x}}=\lim\limits_{x\to\ 0}\frac{23 \cdot cos(22 \cdot x)}{22 \cdot \frac{sin(22 \cdot x)}{22 \cdot x}} = \frac{23}{22}$$
\rozwStop
\odpStart
$\frac{23}{22}$
\odpStop
\testStart
A.$\frac{23}{22}$
B.$\infty$
C.$-\infty$
D.$0$
E.$-\frac{23}{22}$
F.$\frac{22}{23}$
G.$-\frac{22}{23}$
H.$22$
I.$23$
\testStop
\kluczStart
A
\kluczStop



\zadStart{Przykład z Wikieł P 4.3a moja wersja nr 457}


Obliczyć granicę funkcji $\lim\limits_{x\to\ 0}\frac{23 \cdot x}{tan(24 \cdot x)}$.
\zadStop
\rozwStart{Patryk Wirkus}{}
$$\lim\limits_{x\to\ 0}\frac{23 \cdot x}{tan(24 \cdot x)}=\lim\limits_{x\to\ 0}\frac{23 \cdot x \cdot cos(24 \cdot x)}{sin(24 \cdot x)}=\lim\limits_{x\to\ 0}\frac{23 \cdot cos(24 \cdot x)}{\frac{sin(24 \cdot x)}{x}}=\lim\limits_{x\to\ 0}\frac{23 \cdot cos(24 \cdot x)}{24 \cdot \frac{sin(24 \cdot x)}{24 \cdot x}} = \frac{23}{24}$$
\rozwStop
\odpStart
$\frac{23}{24}$
\odpStop
\testStart
A.$\frac{23}{24}$
B.$\infty$
C.$-\infty$
D.$0$
E.$-\frac{23}{24}$
F.$\frac{24}{23}$
G.$-\frac{24}{23}$
H.$24$
I.$23$
\testStop
\kluczStart
A
\kluczStop



\zadStart{Przykład z Wikieł P 4.3a moja wersja nr 458}


Obliczyć granicę funkcji $\lim\limits_{x\to\ 0}\frac{23 \cdot x}{tan(25 \cdot x)}$.
\zadStop
\rozwStart{Patryk Wirkus}{}
$$\lim\limits_{x\to\ 0}\frac{23 \cdot x}{tan(25 \cdot x)}=\lim\limits_{x\to\ 0}\frac{23 \cdot x \cdot cos(25 \cdot x)}{sin(25 \cdot x)}=\lim\limits_{x\to\ 0}\frac{23 \cdot cos(25 \cdot x)}{\frac{sin(25 \cdot x)}{x}}=\lim\limits_{x\to\ 0}\frac{23 \cdot cos(25 \cdot x)}{25 \cdot \frac{sin(25 \cdot x)}{25 \cdot x}} = \frac{23}{25}$$
\rozwStop
\odpStart
$\frac{23}{25}$
\odpStop
\testStart
A.$\frac{23}{25}$
B.$\infty$
C.$-\infty$
D.$0$
E.$-\frac{23}{25}$
F.$\frac{25}{23}$
G.$-\frac{25}{23}$
H.$25$
I.$23$
\testStop
\kluczStart
A
\kluczStop



\zadStart{Przykład z Wikieł P 4.3a moja wersja nr 459}


Obliczyć granicę funkcji $\lim\limits_{x\to\ 0}\frac{23 \cdot x}{tan(26 \cdot x)}$.
\zadStop
\rozwStart{Patryk Wirkus}{}
$$\lim\limits_{x\to\ 0}\frac{23 \cdot x}{tan(26 \cdot x)}=\lim\limits_{x\to\ 0}\frac{23 \cdot x \cdot cos(26 \cdot x)}{sin(26 \cdot x)}=\lim\limits_{x\to\ 0}\frac{23 \cdot cos(26 \cdot x)}{\frac{sin(26 \cdot x)}{x}}=\lim\limits_{x\to\ 0}\frac{23 \cdot cos(26 \cdot x)}{26 \cdot \frac{sin(26 \cdot x)}{26 \cdot x}} = \frac{23}{26}$$
\rozwStop
\odpStart
$\frac{23}{26}$
\odpStop
\testStart
A.$\frac{23}{26}$
B.$\infty$
C.$-\infty$
D.$0$
E.$-\frac{23}{26}$
F.$\frac{26}{23}$
G.$-\frac{26}{23}$
H.$26$
I.$23$
\testStop
\kluczStart
A
\kluczStop



\zadStart{Przykład z Wikieł P 4.3a moja wersja nr 460}


Obliczyć granicę funkcji $\lim\limits_{x\to\ 0}\frac{23 \cdot x}{tan(27 \cdot x)}$.
\zadStop
\rozwStart{Patryk Wirkus}{}
$$\lim\limits_{x\to\ 0}\frac{23 \cdot x}{tan(27 \cdot x)}=\lim\limits_{x\to\ 0}\frac{23 \cdot x \cdot cos(27 \cdot x)}{sin(27 \cdot x)}=\lim\limits_{x\to\ 0}\frac{23 \cdot cos(27 \cdot x)}{\frac{sin(27 \cdot x)}{x}}=\lim\limits_{x\to\ 0}\frac{23 \cdot cos(27 \cdot x)}{27 \cdot \frac{sin(27 \cdot x)}{27 \cdot x}} = \frac{23}{27}$$
\rozwStop
\odpStart
$\frac{23}{27}$
\odpStop
\testStart
A.$\frac{23}{27}$
B.$\infty$
C.$-\infty$
D.$0$
E.$-\frac{23}{27}$
F.$\frac{27}{23}$
G.$-\frac{27}{23}$
H.$27$
I.$23$
\testStop
\kluczStart
A
\kluczStop



\zadStart{Przykład z Wikieł P 4.3a moja wersja nr 461}


Obliczyć granicę funkcji $\lim\limits_{x\to\ 0}\frac{23 \cdot x}{tan(28 \cdot x)}$.
\zadStop
\rozwStart{Patryk Wirkus}{}
$$\lim\limits_{x\to\ 0}\frac{23 \cdot x}{tan(28 \cdot x)}=\lim\limits_{x\to\ 0}\frac{23 \cdot x \cdot cos(28 \cdot x)}{sin(28 \cdot x)}=\lim\limits_{x\to\ 0}\frac{23 \cdot cos(28 \cdot x)}{\frac{sin(28 \cdot x)}{x}}=\lim\limits_{x\to\ 0}\frac{23 \cdot cos(28 \cdot x)}{28 \cdot \frac{sin(28 \cdot x)}{28 \cdot x}} = \frac{23}{28}$$
\rozwStop
\odpStart
$\frac{23}{28}$
\odpStop
\testStart
A.$\frac{23}{28}$
B.$\infty$
C.$-\infty$
D.$0$
E.$-\frac{23}{28}$
F.$\frac{28}{23}$
G.$-\frac{28}{23}$
H.$28$
I.$23$
\testStop
\kluczStart
A
\kluczStop



\zadStart{Przykład z Wikieł P 4.3a moja wersja nr 462}


Obliczyć granicę funkcji $\lim\limits_{x\to\ 0}\frac{23 \cdot x}{tan(29 \cdot x)}$.
\zadStop
\rozwStart{Patryk Wirkus}{}
$$\lim\limits_{x\to\ 0}\frac{23 \cdot x}{tan(29 \cdot x)}=\lim\limits_{x\to\ 0}\frac{23 \cdot x \cdot cos(29 \cdot x)}{sin(29 \cdot x)}=\lim\limits_{x\to\ 0}\frac{23 \cdot cos(29 \cdot x)}{\frac{sin(29 \cdot x)}{x}}=\lim\limits_{x\to\ 0}\frac{23 \cdot cos(29 \cdot x)}{29 \cdot \frac{sin(29 \cdot x)}{29 \cdot x}} = \frac{23}{29}$$
\rozwStop
\odpStart
$\frac{23}{29}$
\odpStop
\testStart
A.$\frac{23}{29}$
B.$\infty$
C.$-\infty$
D.$0$
E.$-\frac{23}{29}$
F.$\frac{29}{23}$
G.$-\frac{29}{23}$
H.$29$
I.$23$
\testStop
\kluczStart
A
\kluczStop



\zadStart{Przykład z Wikieł P 4.3a moja wersja nr 463}


Obliczyć granicę funkcji $\lim\limits_{x\to\ 0}\frac{23 \cdot x}{tan(30 \cdot x)}$.
\zadStop
\rozwStart{Patryk Wirkus}{}
$$\lim\limits_{x\to\ 0}\frac{23 \cdot x}{tan(30 \cdot x)}=\lim\limits_{x\to\ 0}\frac{23 \cdot x \cdot cos(30 \cdot x)}{sin(30 \cdot x)}=\lim\limits_{x\to\ 0}\frac{23 \cdot cos(30 \cdot x)}{\frac{sin(30 \cdot x)}{x}}=\lim\limits_{x\to\ 0}\frac{23 \cdot cos(30 \cdot x)}{30 \cdot \frac{sin(30 \cdot x)}{30 \cdot x}} = \frac{23}{30}$$
\rozwStop
\odpStart
$\frac{23}{30}$
\odpStop
\testStart
A.$\frac{23}{30}$
B.$\infty$
C.$-\infty$
D.$0$
E.$-\frac{23}{30}$
F.$\frac{30}{23}$
G.$-\frac{30}{23}$
H.$30$
I.$23$
\testStop
\kluczStart
A
\kluczStop



\zadStart{Przykład z Wikieł P 4.3a moja wersja nr 464}


Obliczyć granicę funkcji $\lim\limits_{x\to\ 0}\frac{23 \cdot x}{tan(31 \cdot x)}$.
\zadStop
\rozwStart{Patryk Wirkus}{}
$$\lim\limits_{x\to\ 0}\frac{23 \cdot x}{tan(31 \cdot x)}=\lim\limits_{x\to\ 0}\frac{23 \cdot x \cdot cos(31 \cdot x)}{sin(31 \cdot x)}=\lim\limits_{x\to\ 0}\frac{23 \cdot cos(31 \cdot x)}{\frac{sin(31 \cdot x)}{x}}=\lim\limits_{x\to\ 0}\frac{23 \cdot cos(31 \cdot x)}{31 \cdot \frac{sin(31 \cdot x)}{31 \cdot x}} = \frac{23}{31}$$
\rozwStop
\odpStart
$\frac{23}{31}$
\odpStop
\testStart
A.$\frac{23}{31}$
B.$\infty$
C.$-\infty$
D.$0$
E.$-\frac{23}{31}$
F.$\frac{31}{23}$
G.$-\frac{31}{23}$
H.$31$
I.$23$
\testStop
\kluczStart
A
\kluczStop



\zadStart{Przykład z Wikieł P 4.3a moja wersja nr 465}


Obliczyć granicę funkcji $\lim\limits_{x\to\ 0}\frac{23 \cdot x}{tan(32 \cdot x)}$.
\zadStop
\rozwStart{Patryk Wirkus}{}
$$\lim\limits_{x\to\ 0}\frac{23 \cdot x}{tan(32 \cdot x)}=\lim\limits_{x\to\ 0}\frac{23 \cdot x \cdot cos(32 \cdot x)}{sin(32 \cdot x)}=\lim\limits_{x\to\ 0}\frac{23 \cdot cos(32 \cdot x)}{\frac{sin(32 \cdot x)}{x}}=\lim\limits_{x\to\ 0}\frac{23 \cdot cos(32 \cdot x)}{32 \cdot \frac{sin(32 \cdot x)}{32 \cdot x}} = \frac{23}{32}$$
\rozwStop
\odpStart
$\frac{23}{32}$
\odpStop
\testStart
A.$\frac{23}{32}$
B.$\infty$
C.$-\infty$
D.$0$
E.$-\frac{23}{32}$
F.$\frac{32}{23}$
G.$-\frac{32}{23}$
H.$32$
I.$23$
\testStop
\kluczStart
A
\kluczStop



\zadStart{Przykład z Wikieł P 4.3a moja wersja nr 466}


Obliczyć granicę funkcji $\lim\limits_{x\to\ 0}\frac{23 \cdot x}{tan(33 \cdot x)}$.
\zadStop
\rozwStart{Patryk Wirkus}{}
$$\lim\limits_{x\to\ 0}\frac{23 \cdot x}{tan(33 \cdot x)}=\lim\limits_{x\to\ 0}\frac{23 \cdot x \cdot cos(33 \cdot x)}{sin(33 \cdot x)}=\lim\limits_{x\to\ 0}\frac{23 \cdot cos(33 \cdot x)}{\frac{sin(33 \cdot x)}{x}}=\lim\limits_{x\to\ 0}\frac{23 \cdot cos(33 \cdot x)}{33 \cdot \frac{sin(33 \cdot x)}{33 \cdot x}} = \frac{23}{33}$$
\rozwStop
\odpStart
$\frac{23}{33}$
\odpStop
\testStart
A.$\frac{23}{33}$
B.$\infty$
C.$-\infty$
D.$0$
E.$-\frac{23}{33}$
F.$\frac{33}{23}$
G.$-\frac{33}{23}$
H.$33$
I.$23$
\testStop
\kluczStart
A
\kluczStop



\zadStart{Przykład z Wikieł P 4.3a moja wersja nr 467}


Obliczyć granicę funkcji $\lim\limits_{x\to\ 0}\frac{23 \cdot x}{tan(34 \cdot x)}$.
\zadStop
\rozwStart{Patryk Wirkus}{}
$$\lim\limits_{x\to\ 0}\frac{23 \cdot x}{tan(34 \cdot x)}=\lim\limits_{x\to\ 0}\frac{23 \cdot x \cdot cos(34 \cdot x)}{sin(34 \cdot x)}=\lim\limits_{x\to\ 0}\frac{23 \cdot cos(34 \cdot x)}{\frac{sin(34 \cdot x)}{x}}=\lim\limits_{x\to\ 0}\frac{23 \cdot cos(34 \cdot x)}{34 \cdot \frac{sin(34 \cdot x)}{34 \cdot x}} = \frac{23}{34}$$
\rozwStop
\odpStart
$\frac{23}{34}$
\odpStop
\testStart
A.$\frac{23}{34}$
B.$\infty$
C.$-\infty$
D.$0$
E.$-\frac{23}{34}$
F.$\frac{34}{23}$
G.$-\frac{34}{23}$
H.$34$
I.$23$
\testStop
\kluczStart
A
\kluczStop



\zadStart{Przykład z Wikieł P 4.3a moja wersja nr 468}


Obliczyć granicę funkcji $\lim\limits_{x\to\ 0}\frac{23 \cdot x}{tan(35 \cdot x)}$.
\zadStop
\rozwStart{Patryk Wirkus}{}
$$\lim\limits_{x\to\ 0}\frac{23 \cdot x}{tan(35 \cdot x)}=\lim\limits_{x\to\ 0}\frac{23 \cdot x \cdot cos(35 \cdot x)}{sin(35 \cdot x)}=\lim\limits_{x\to\ 0}\frac{23 \cdot cos(35 \cdot x)}{\frac{sin(35 \cdot x)}{x}}=\lim\limits_{x\to\ 0}\frac{23 \cdot cos(35 \cdot x)}{35 \cdot \frac{sin(35 \cdot x)}{35 \cdot x}} = \frac{23}{35}$$
\rozwStop
\odpStart
$\frac{23}{35}$
\odpStop
\testStart
A.$\frac{23}{35}$
B.$\infty$
C.$-\infty$
D.$0$
E.$-\frac{23}{35}$
F.$\frac{35}{23}$
G.$-\frac{35}{23}$
H.$35$
I.$23$
\testStop
\kluczStart
A
\kluczStop



\zadStart{Przykład z Wikieł P 4.3a moja wersja nr 469}


Obliczyć granicę funkcji $\lim\limits_{x\to\ 0}\frac{23 \cdot x}{tan(36 \cdot x)}$.
\zadStop
\rozwStart{Patryk Wirkus}{}
$$\lim\limits_{x\to\ 0}\frac{23 \cdot x}{tan(36 \cdot x)}=\lim\limits_{x\to\ 0}\frac{23 \cdot x \cdot cos(36 \cdot x)}{sin(36 \cdot x)}=\lim\limits_{x\to\ 0}\frac{23 \cdot cos(36 \cdot x)}{\frac{sin(36 \cdot x)}{x}}=\lim\limits_{x\to\ 0}\frac{23 \cdot cos(36 \cdot x)}{36 \cdot \frac{sin(36 \cdot x)}{36 \cdot x}} = \frac{23}{36}$$
\rozwStop
\odpStart
$\frac{23}{36}$
\odpStop
\testStart
A.$\frac{23}{36}$
B.$\infty$
C.$-\infty$
D.$0$
E.$-\frac{23}{36}$
F.$\frac{36}{23}$
G.$-\frac{36}{23}$
H.$36$
I.$23$
\testStop
\kluczStart
A
\kluczStop



\zadStart{Przykład z Wikieł P 4.3a moja wersja nr 470}


Obliczyć granicę funkcji $\lim\limits_{x\to\ 0}\frac{23 \cdot x}{tan(37 \cdot x)}$.
\zadStop
\rozwStart{Patryk Wirkus}{}
$$\lim\limits_{x\to\ 0}\frac{23 \cdot x}{tan(37 \cdot x)}=\lim\limits_{x\to\ 0}\frac{23 \cdot x \cdot cos(37 \cdot x)}{sin(37 \cdot x)}=\lim\limits_{x\to\ 0}\frac{23 \cdot cos(37 \cdot x)}{\frac{sin(37 \cdot x)}{x}}=\lim\limits_{x\to\ 0}\frac{23 \cdot cos(37 \cdot x)}{37 \cdot \frac{sin(37 \cdot x)}{37 \cdot x}} = \frac{23}{37}$$
\rozwStop
\odpStart
$\frac{23}{37}$
\odpStop
\testStart
A.$\frac{23}{37}$
B.$\infty$
C.$-\infty$
D.$0$
E.$-\frac{23}{37}$
F.$\frac{37}{23}$
G.$-\frac{37}{23}$
H.$37$
I.$23$
\testStop
\kluczStart
A
\kluczStop



\zadStart{Przykład z Wikieł P 4.3a moja wersja nr 471}


Obliczyć granicę funkcji $\lim\limits_{x\to\ 0}\frac{23 \cdot x}{tan(38 \cdot x)}$.
\zadStop
\rozwStart{Patryk Wirkus}{}
$$\lim\limits_{x\to\ 0}\frac{23 \cdot x}{tan(38 \cdot x)}=\lim\limits_{x\to\ 0}\frac{23 \cdot x \cdot cos(38 \cdot x)}{sin(38 \cdot x)}=\lim\limits_{x\to\ 0}\frac{23 \cdot cos(38 \cdot x)}{\frac{sin(38 \cdot x)}{x}}=\lim\limits_{x\to\ 0}\frac{23 \cdot cos(38 \cdot x)}{38 \cdot \frac{sin(38 \cdot x)}{38 \cdot x}} = \frac{23}{38}$$
\rozwStop
\odpStart
$\frac{23}{38}$
\odpStop
\testStart
A.$\frac{23}{38}$
B.$\infty$
C.$-\infty$
D.$0$
E.$-\frac{23}{38}$
F.$\frac{38}{23}$
G.$-\frac{38}{23}$
H.$38$
I.$23$
\testStop
\kluczStart
A
\kluczStop



\zadStart{Przykład z Wikieł P 4.3a moja wersja nr 472}


Obliczyć granicę funkcji $\lim\limits_{x\to\ 0}\frac{23 \cdot x}{tan(39 \cdot x)}$.
\zadStop
\rozwStart{Patryk Wirkus}{}
$$\lim\limits_{x\to\ 0}\frac{23 \cdot x}{tan(39 \cdot x)}=\lim\limits_{x\to\ 0}\frac{23 \cdot x \cdot cos(39 \cdot x)}{sin(39 \cdot x)}=\lim\limits_{x\to\ 0}\frac{23 \cdot cos(39 \cdot x)}{\frac{sin(39 \cdot x)}{x}}=\lim\limits_{x\to\ 0}\frac{23 \cdot cos(39 \cdot x)}{39 \cdot \frac{sin(39 \cdot x)}{39 \cdot x}} = \frac{23}{39}$$
\rozwStop
\odpStart
$\frac{23}{39}$
\odpStop
\testStart
A.$\frac{23}{39}$
B.$\infty$
C.$-\infty$
D.$0$
E.$-\frac{23}{39}$
F.$\frac{39}{23}$
G.$-\frac{39}{23}$
H.$39$
I.$23$
\testStop
\kluczStart
A
\kluczStop



\zadStart{Przykład z Wikieł P 4.3a moja wersja nr 473}


Obliczyć granicę funkcji $\lim\limits_{x\to\ 0}\frac{23 \cdot x}{tan(40 \cdot x)}$.
\zadStop
\rozwStart{Patryk Wirkus}{}
$$\lim\limits_{x\to\ 0}\frac{23 \cdot x}{tan(40 \cdot x)}=\lim\limits_{x\to\ 0}\frac{23 \cdot x \cdot cos(40 \cdot x)}{sin(40 \cdot x)}=\lim\limits_{x\to\ 0}\frac{23 \cdot cos(40 \cdot x)}{\frac{sin(40 \cdot x)}{x}}=\lim\limits_{x\to\ 0}\frac{23 \cdot cos(40 \cdot x)}{40 \cdot \frac{sin(40 \cdot x)}{40 \cdot x}} = \frac{23}{40}$$
\rozwStop
\odpStart
$\frac{23}{40}$
\odpStop
\testStart
A.$\frac{23}{40}$
B.$\infty$
C.$-\infty$
D.$0$
E.$-\frac{23}{40}$
F.$\frac{40}{23}$
G.$-\frac{40}{23}$
H.$40$
I.$23$
\testStop
\kluczStart
A
\kluczStop



\zadStart{Przykład z Wikieł P 4.3a moja wersja nr 474}


Obliczyć granicę funkcji $\lim\limits_{x\to\ 0}\frac{24 \cdot x}{tan(5 \cdot x)}$.
\zadStop
\rozwStart{Patryk Wirkus}{}
$$\lim\limits_{x\to\ 0}\frac{24 \cdot x}{tan(5 \cdot x)}=\lim\limits_{x\to\ 0}\frac{24 \cdot x \cdot cos(5 \cdot x)}{sin(5 \cdot x)}=\lim\limits_{x\to\ 0}\frac{24 \cdot cos(5 \cdot x)}{\frac{sin(5 \cdot x)}{x}}=\lim\limits_{x\to\ 0}\frac{24 \cdot cos(5 \cdot x)}{5 \cdot \frac{sin(5 \cdot x)}{5 \cdot x}} = \frac{24}{5}$$
\rozwStop
\odpStart
$\frac{24}{5}$
\odpStop
\testStart
A.$\frac{24}{5}$
B.$\infty$
C.$-\infty$
D.$0$
E.$-\frac{24}{5}$
F.$\frac{5}{24}$
G.$-\frac{5}{24}$
H.$5$
I.$24$
\testStop
\kluczStart
A
\kluczStop



\zadStart{Przykład z Wikieł P 4.3a moja wersja nr 475}


Obliczyć granicę funkcji $\lim\limits_{x\to\ 0}\frac{24 \cdot x}{tan(7 \cdot x)}$.
\zadStop
\rozwStart{Patryk Wirkus}{}
$$\lim\limits_{x\to\ 0}\frac{24 \cdot x}{tan(7 \cdot x)}=\lim\limits_{x\to\ 0}\frac{24 \cdot x \cdot cos(7 \cdot x)}{sin(7 \cdot x)}=\lim\limits_{x\to\ 0}\frac{24 \cdot cos(7 \cdot x)}{\frac{sin(7 \cdot x)}{x}}=\lim\limits_{x\to\ 0}\frac{24 \cdot cos(7 \cdot x)}{7 \cdot \frac{sin(7 \cdot x)}{7 \cdot x}} = \frac{24}{7}$$
\rozwStop
\odpStart
$\frac{24}{7}$
\odpStop
\testStart
A.$\frac{24}{7}$
B.$\infty$
C.$-\infty$
D.$0$
E.$-\frac{24}{7}$
F.$\frac{7}{24}$
G.$-\frac{7}{24}$
H.$7$
I.$24$
\testStop
\kluczStart
A
\kluczStop



\zadStart{Przykład z Wikieł P 4.3a moja wersja nr 476}


Obliczyć granicę funkcji $\lim\limits_{x\to\ 0}\frac{24 \cdot x}{tan(11 \cdot x)}$.
\zadStop
\rozwStart{Patryk Wirkus}{}
$$\lim\limits_{x\to\ 0}\frac{24 \cdot x}{tan(11 \cdot x)}=\lim\limits_{x\to\ 0}\frac{24 \cdot x \cdot cos(11 \cdot x)}{sin(11 \cdot x)}=\lim\limits_{x\to\ 0}\frac{24 \cdot cos(11 \cdot x)}{\frac{sin(11 \cdot x)}{x}}=\lim\limits_{x\to\ 0}\frac{24 \cdot cos(11 \cdot x)}{11 \cdot \frac{sin(11 \cdot x)}{11 \cdot x}} = \frac{24}{11}$$
\rozwStop
\odpStart
$\frac{24}{11}$
\odpStop
\testStart
A.$\frac{24}{11}$
B.$\infty$
C.$-\infty$
D.$0$
E.$-\frac{24}{11}$
F.$\frac{11}{24}$
G.$-\frac{11}{24}$
H.$11$
I.$24$
\testStop
\kluczStart
A
\kluczStop



\zadStart{Przykład z Wikieł P 4.3a moja wersja nr 477}


Obliczyć granicę funkcji $\lim\limits_{x\to\ 0}\frac{24 \cdot x}{tan(13 \cdot x)}$.
\zadStop
\rozwStart{Patryk Wirkus}{}
$$\lim\limits_{x\to\ 0}\frac{24 \cdot x}{tan(13 \cdot x)}=\lim\limits_{x\to\ 0}\frac{24 \cdot x \cdot cos(13 \cdot x)}{sin(13 \cdot x)}=\lim\limits_{x\to\ 0}\frac{24 \cdot cos(13 \cdot x)}{\frac{sin(13 \cdot x)}{x}}=\lim\limits_{x\to\ 0}\frac{24 \cdot cos(13 \cdot x)}{13 \cdot \frac{sin(13 \cdot x)}{13 \cdot x}} = \frac{24}{13}$$
\rozwStop
\odpStart
$\frac{24}{13}$
\odpStop
\testStart
A.$\frac{24}{13}$
B.$\infty$
C.$-\infty$
D.$0$
E.$-\frac{24}{13}$
F.$\frac{13}{24}$
G.$-\frac{13}{24}$
H.$13$
I.$24$
\testStop
\kluczStart
A
\kluczStop



\zadStart{Przykład z Wikieł P 4.3a moja wersja nr 478}


Obliczyć granicę funkcji $\lim\limits_{x\to\ 0}\frac{24 \cdot x}{tan(17 \cdot x)}$.
\zadStop
\rozwStart{Patryk Wirkus}{}
$$\lim\limits_{x\to\ 0}\frac{24 \cdot x}{tan(17 \cdot x)}=\lim\limits_{x\to\ 0}\frac{24 \cdot x \cdot cos(17 \cdot x)}{sin(17 \cdot x)}=\lim\limits_{x\to\ 0}\frac{24 \cdot cos(17 \cdot x)}{\frac{sin(17 \cdot x)}{x}}=\lim\limits_{x\to\ 0}\frac{24 \cdot cos(17 \cdot x)}{17 \cdot \frac{sin(17 \cdot x)}{17 \cdot x}} = \frac{24}{17}$$
\rozwStop
\odpStart
$\frac{24}{17}$
\odpStop
\testStart
A.$\frac{24}{17}$
B.$\infty$
C.$-\infty$
D.$0$
E.$-\frac{24}{17}$
F.$\frac{17}{24}$
G.$-\frac{17}{24}$
H.$17$
I.$24$
\testStop
\kluczStart
A
\kluczStop



\zadStart{Przykład z Wikieł P 4.3a moja wersja nr 479}


Obliczyć granicę funkcji $\lim\limits_{x\to\ 0}\frac{24 \cdot x}{tan(19 \cdot x)}$.
\zadStop
\rozwStart{Patryk Wirkus}{}
$$\lim\limits_{x\to\ 0}\frac{24 \cdot x}{tan(19 \cdot x)}=\lim\limits_{x\to\ 0}\frac{24 \cdot x \cdot cos(19 \cdot x)}{sin(19 \cdot x)}=\lim\limits_{x\to\ 0}\frac{24 \cdot cos(19 \cdot x)}{\frac{sin(19 \cdot x)}{x}}=\lim\limits_{x\to\ 0}\frac{24 \cdot cos(19 \cdot x)}{19 \cdot \frac{sin(19 \cdot x)}{19 \cdot x}} = \frac{24}{19}$$
\rozwStop
\odpStart
$\frac{24}{19}$
\odpStop
\testStart
A.$\frac{24}{19}$
B.$\infty$
C.$-\infty$
D.$0$
E.$-\frac{24}{19}$
F.$\frac{19}{24}$
G.$-\frac{19}{24}$
H.$19$
I.$24$
\testStop
\kluczStart
A
\kluczStop



\zadStart{Przykład z Wikieł P 4.3a moja wersja nr 480}


Obliczyć granicę funkcji $\lim\limits_{x\to\ 0}\frac{24 \cdot x}{tan(23 \cdot x)}$.
\zadStop
\rozwStart{Patryk Wirkus}{}
$$\lim\limits_{x\to\ 0}\frac{24 \cdot x}{tan(23 \cdot x)}=\lim\limits_{x\to\ 0}\frac{24 \cdot x \cdot cos(23 \cdot x)}{sin(23 \cdot x)}=\lim\limits_{x\to\ 0}\frac{24 \cdot cos(23 \cdot x)}{\frac{sin(23 \cdot x)}{x}}=\lim\limits_{x\to\ 0}\frac{24 \cdot cos(23 \cdot x)}{23 \cdot \frac{sin(23 \cdot x)}{23 \cdot x}} = \frac{24}{23}$$
\rozwStop
\odpStart
$\frac{24}{23}$
\odpStop
\testStart
A.$\frac{24}{23}$
B.$\infty$
C.$-\infty$
D.$0$
E.$-\frac{24}{23}$
F.$\frac{23}{24}$
G.$-\frac{23}{24}$
H.$23$
I.$24$
\testStop
\kluczStart
A
\kluczStop



\zadStart{Przykład z Wikieł P 4.3a moja wersja nr 481}


Obliczyć granicę funkcji $\lim\limits_{x\to\ 0}\frac{24 \cdot x}{tan(25 \cdot x)}$.
\zadStop
\rozwStart{Patryk Wirkus}{}
$$\lim\limits_{x\to\ 0}\frac{24 \cdot x}{tan(25 \cdot x)}=\lim\limits_{x\to\ 0}\frac{24 \cdot x \cdot cos(25 \cdot x)}{sin(25 \cdot x)}=\lim\limits_{x\to\ 0}\frac{24 \cdot cos(25 \cdot x)}{\frac{sin(25 \cdot x)}{x}}=\lim\limits_{x\to\ 0}\frac{24 \cdot cos(25 \cdot x)}{25 \cdot \frac{sin(25 \cdot x)}{25 \cdot x}} = \frac{24}{25}$$
\rozwStop
\odpStart
$\frac{24}{25}$
\odpStop
\testStart
A.$\frac{24}{25}$
B.$\infty$
C.$-\infty$
D.$0$
E.$-\frac{24}{25}$
F.$\frac{25}{24}$
G.$-\frac{25}{24}$
H.$25$
I.$24$
\testStop
\kluczStart
A
\kluczStop



\zadStart{Przykład z Wikieł P 4.3a moja wersja nr 482}


Obliczyć granicę funkcji $\lim\limits_{x\to\ 0}\frac{24 \cdot x}{tan(29 \cdot x)}$.
\zadStop
\rozwStart{Patryk Wirkus}{}
$$\lim\limits_{x\to\ 0}\frac{24 \cdot x}{tan(29 \cdot x)}=\lim\limits_{x\to\ 0}\frac{24 \cdot x \cdot cos(29 \cdot x)}{sin(29 \cdot x)}=\lim\limits_{x\to\ 0}\frac{24 \cdot cos(29 \cdot x)}{\frac{sin(29 \cdot x)}{x}}=\lim\limits_{x\to\ 0}\frac{24 \cdot cos(29 \cdot x)}{29 \cdot \frac{sin(29 \cdot x)}{29 \cdot x}} = \frac{24}{29}$$
\rozwStop
\odpStart
$\frac{24}{29}$
\odpStop
\testStart
A.$\frac{24}{29}$
B.$\infty$
C.$-\infty$
D.$0$
E.$-\frac{24}{29}$
F.$\frac{29}{24}$
G.$-\frac{29}{24}$
H.$29$
I.$24$
\testStop
\kluczStart
A
\kluczStop



\zadStart{Przykład z Wikieł P 4.3a moja wersja nr 483}


Obliczyć granicę funkcji $\lim\limits_{x\to\ 0}\frac{24 \cdot x}{tan(31 \cdot x)}$.
\zadStop
\rozwStart{Patryk Wirkus}{}
$$\lim\limits_{x\to\ 0}\frac{24 \cdot x}{tan(31 \cdot x)}=\lim\limits_{x\to\ 0}\frac{24 \cdot x \cdot cos(31 \cdot x)}{sin(31 \cdot x)}=\lim\limits_{x\to\ 0}\frac{24 \cdot cos(31 \cdot x)}{\frac{sin(31 \cdot x)}{x}}=\lim\limits_{x\to\ 0}\frac{24 \cdot cos(31 \cdot x)}{31 \cdot \frac{sin(31 \cdot x)}{31 \cdot x}} = \frac{24}{31}$$
\rozwStop
\odpStart
$\frac{24}{31}$
\odpStop
\testStart
A.$\frac{24}{31}$
B.$\infty$
C.$-\infty$
D.$0$
E.$-\frac{24}{31}$
F.$\frac{31}{24}$
G.$-\frac{31}{24}$
H.$31$
I.$24$
\testStop
\kluczStart
A
\kluczStop



\zadStart{Przykład z Wikieł P 4.3a moja wersja nr 484}


Obliczyć granicę funkcji $\lim\limits_{x\to\ 0}\frac{24 \cdot x}{tan(35 \cdot x)}$.
\zadStop
\rozwStart{Patryk Wirkus}{}
$$\lim\limits_{x\to\ 0}\frac{24 \cdot x}{tan(35 \cdot x)}=\lim\limits_{x\to\ 0}\frac{24 \cdot x \cdot cos(35 \cdot x)}{sin(35 \cdot x)}=\lim\limits_{x\to\ 0}\frac{24 \cdot cos(35 \cdot x)}{\frac{sin(35 \cdot x)}{x}}=\lim\limits_{x\to\ 0}\frac{24 \cdot cos(35 \cdot x)}{35 \cdot \frac{sin(35 \cdot x)}{35 \cdot x}} = \frac{24}{35}$$
\rozwStop
\odpStart
$\frac{24}{35}$
\odpStop
\testStart
A.$\frac{24}{35}$
B.$\infty$
C.$-\infty$
D.$0$
E.$-\frac{24}{35}$
F.$\frac{35}{24}$
G.$-\frac{35}{24}$
H.$35$
I.$24$
\testStop
\kluczStart
A
\kluczStop



\zadStart{Przykład z Wikieł P 4.3a moja wersja nr 485}


Obliczyć granicę funkcji $\lim\limits_{x\to\ 0}\frac{24 \cdot x}{tan(37 \cdot x)}$.
\zadStop
\rozwStart{Patryk Wirkus}{}
$$\lim\limits_{x\to\ 0}\frac{24 \cdot x}{tan(37 \cdot x)}=\lim\limits_{x\to\ 0}\frac{24 \cdot x \cdot cos(37 \cdot x)}{sin(37 \cdot x)}=\lim\limits_{x\to\ 0}\frac{24 \cdot cos(37 \cdot x)}{\frac{sin(37 \cdot x)}{x}}=\lim\limits_{x\to\ 0}\frac{24 \cdot cos(37 \cdot x)}{37 \cdot \frac{sin(37 \cdot x)}{37 \cdot x}} = \frac{24}{37}$$
\rozwStop
\odpStart
$\frac{24}{37}$
\odpStop
\testStart
A.$\frac{24}{37}$
B.$\infty$
C.$-\infty$
D.$0$
E.$-\frac{24}{37}$
F.$\frac{37}{24}$
G.$-\frac{37}{24}$
H.$37$
I.$24$
\testStop
\kluczStart
A
\kluczStop



\zadStart{Przykład z Wikieł P 4.3a moja wersja nr 486}


Obliczyć granicę funkcji $\lim\limits_{x\to\ 0}\frac{25 \cdot x}{tan(2 \cdot x)}$.
\zadStop
\rozwStart{Patryk Wirkus}{}
$$\lim\limits_{x\to\ 0}\frac{25 \cdot x}{tan(2 \cdot x)}=\lim\limits_{x\to\ 0}\frac{25 \cdot x \cdot cos(2 \cdot x)}{sin(2 \cdot x)}=\lim\limits_{x\to\ 0}\frac{25 \cdot cos(2 \cdot x)}{\frac{sin(2 \cdot x)}{x}}=\lim\limits_{x\to\ 0}\frac{25 \cdot cos(2 \cdot x)}{2 \cdot \frac{sin(2 \cdot x)}{2 \cdot x}} = \frac{25}{2}$$
\rozwStop
\odpStart
$\frac{25}{2}$
\odpStop
\testStart
A.$\frac{25}{2}$
B.$\infty$
C.$-\infty$
D.$0$
E.$-\frac{25}{2}$
F.$\frac{2}{25}$
G.$-\frac{2}{25}$
H.$2$
I.$25$
\testStop
\kluczStart
A
\kluczStop



\zadStart{Przykład z Wikieł P 4.3a moja wersja nr 487}


Obliczyć granicę funkcji $\lim\limits_{x\to\ 0}\frac{25 \cdot x}{tan(3 \cdot x)}$.
\zadStop
\rozwStart{Patryk Wirkus}{}
$$\lim\limits_{x\to\ 0}\frac{25 \cdot x}{tan(3 \cdot x)}=\lim\limits_{x\to\ 0}\frac{25 \cdot x \cdot cos(3 \cdot x)}{sin(3 \cdot x)}=\lim\limits_{x\to\ 0}\frac{25 \cdot cos(3 \cdot x)}{\frac{sin(3 \cdot x)}{x}}=\lim\limits_{x\to\ 0}\frac{25 \cdot cos(3 \cdot x)}{3 \cdot \frac{sin(3 \cdot x)}{3 \cdot x}} = \frac{25}{3}$$
\rozwStop
\odpStart
$\frac{25}{3}$
\odpStop
\testStart
A.$\frac{25}{3}$
B.$\infty$
C.$-\infty$
D.$0$
E.$-\frac{25}{3}$
F.$\frac{3}{25}$
G.$-\frac{3}{25}$
H.$3$
I.$25$
\testStop
\kluczStart
A
\kluczStop



\zadStart{Przykład z Wikieł P 4.3a moja wersja nr 488}


Obliczyć granicę funkcji $\lim\limits_{x\to\ 0}\frac{25 \cdot x}{tan(4 \cdot x)}$.
\zadStop
\rozwStart{Patryk Wirkus}{}
$$\lim\limits_{x\to\ 0}\frac{25 \cdot x}{tan(4 \cdot x)}=\lim\limits_{x\to\ 0}\frac{25 \cdot x \cdot cos(4 \cdot x)}{sin(4 \cdot x)}=\lim\limits_{x\to\ 0}\frac{25 \cdot cos(4 \cdot x)}{\frac{sin(4 \cdot x)}{x}}=\lim\limits_{x\to\ 0}\frac{25 \cdot cos(4 \cdot x)}{4 \cdot \frac{sin(4 \cdot x)}{4 \cdot x}} = \frac{25}{4}$$
\rozwStop
\odpStart
$\frac{25}{4}$
\odpStop
\testStart
A.$\frac{25}{4}$
B.$\infty$
C.$-\infty$
D.$0$
E.$-\frac{25}{4}$
F.$\frac{4}{25}$
G.$-\frac{4}{25}$
H.$4$
I.$25$
\testStop
\kluczStart
A
\kluczStop



\zadStart{Przykład z Wikieł P 4.3a moja wersja nr 489}


Obliczyć granicę funkcji $\lim\limits_{x\to\ 0}\frac{25 \cdot x}{tan(6 \cdot x)}$.
\zadStop
\rozwStart{Patryk Wirkus}{}
$$\lim\limits_{x\to\ 0}\frac{25 \cdot x}{tan(6 \cdot x)}=\lim\limits_{x\to\ 0}\frac{25 \cdot x \cdot cos(6 \cdot x)}{sin(6 \cdot x)}=\lim\limits_{x\to\ 0}\frac{25 \cdot cos(6 \cdot x)}{\frac{sin(6 \cdot x)}{x}}=\lim\limits_{x\to\ 0}\frac{25 \cdot cos(6 \cdot x)}{6 \cdot \frac{sin(6 \cdot x)}{6 \cdot x}} = \frac{25}{6}$$
\rozwStop
\odpStart
$\frac{25}{6}$
\odpStop
\testStart
A.$\frac{25}{6}$
B.$\infty$
C.$-\infty$
D.$0$
E.$-\frac{25}{6}$
F.$\frac{6}{25}$
G.$-\frac{6}{25}$
H.$6$
I.$25$
\testStop
\kluczStart
A
\kluczStop



\zadStart{Przykład z Wikieł P 4.3a moja wersja nr 490}


Obliczyć granicę funkcji $\lim\limits_{x\to\ 0}\frac{25 \cdot x}{tan(7 \cdot x)}$.
\zadStop
\rozwStart{Patryk Wirkus}{}
$$\lim\limits_{x\to\ 0}\frac{25 \cdot x}{tan(7 \cdot x)}=\lim\limits_{x\to\ 0}\frac{25 \cdot x \cdot cos(7 \cdot x)}{sin(7 \cdot x)}=\lim\limits_{x\to\ 0}\frac{25 \cdot cos(7 \cdot x)}{\frac{sin(7 \cdot x)}{x}}=\lim\limits_{x\to\ 0}\frac{25 \cdot cos(7 \cdot x)}{7 \cdot \frac{sin(7 \cdot x)}{7 \cdot x}} = \frac{25}{7}$$
\rozwStop
\odpStart
$\frac{25}{7}$
\odpStop
\testStart
A.$\frac{25}{7}$
B.$\infty$
C.$-\infty$
D.$0$
E.$-\frac{25}{7}$
F.$\frac{7}{25}$
G.$-\frac{7}{25}$
H.$7$
I.$25$
\testStop
\kluczStart
A
\kluczStop



\zadStart{Przykład z Wikieł P 4.3a moja wersja nr 491}


Obliczyć granicę funkcji $\lim\limits_{x\to\ 0}\frac{25 \cdot x}{tan(8 \cdot x)}$.
\zadStop
\rozwStart{Patryk Wirkus}{}
$$\lim\limits_{x\to\ 0}\frac{25 \cdot x}{tan(8 \cdot x)}=\lim\limits_{x\to\ 0}\frac{25 \cdot x \cdot cos(8 \cdot x)}{sin(8 \cdot x)}=\lim\limits_{x\to\ 0}\frac{25 \cdot cos(8 \cdot x)}{\frac{sin(8 \cdot x)}{x}}=\lim\limits_{x\to\ 0}\frac{25 \cdot cos(8 \cdot x)}{8 \cdot \frac{sin(8 \cdot x)}{8 \cdot x}} = \frac{25}{8}$$
\rozwStop
\odpStart
$\frac{25}{8}$
\odpStop
\testStart
A.$\frac{25}{8}$
B.$\infty$
C.$-\infty$
D.$0$
E.$-\frac{25}{8}$
F.$\frac{8}{25}$
G.$-\frac{8}{25}$
H.$8$
I.$25$
\testStop
\kluczStart
A
\kluczStop



\zadStart{Przykład z Wikieł P 4.3a moja wersja nr 492}


Obliczyć granicę funkcji $\lim\limits_{x\to\ 0}\frac{25 \cdot x}{tan(9 \cdot x)}$.
\zadStop
\rozwStart{Patryk Wirkus}{}
$$\lim\limits_{x\to\ 0}\frac{25 \cdot x}{tan(9 \cdot x)}=\lim\limits_{x\to\ 0}\frac{25 \cdot x \cdot cos(9 \cdot x)}{sin(9 \cdot x)}=\lim\limits_{x\to\ 0}\frac{25 \cdot cos(9 \cdot x)}{\frac{sin(9 \cdot x)}{x}}=\lim\limits_{x\to\ 0}\frac{25 \cdot cos(9 \cdot x)}{9 \cdot \frac{sin(9 \cdot x)}{9 \cdot x}} = \frac{25}{9}$$
\rozwStop
\odpStart
$\frac{25}{9}$
\odpStop
\testStart
A.$\frac{25}{9}$
B.$\infty$
C.$-\infty$
D.$0$
E.$-\frac{25}{9}$
F.$\frac{9}{25}$
G.$-\frac{9}{25}$
H.$9$
I.$25$
\testStop
\kluczStart
A
\kluczStop



\zadStart{Przykład z Wikieł P 4.3a moja wersja nr 493}


Obliczyć granicę funkcji $\lim\limits_{x\to\ 0}\frac{25 \cdot x}{tan(11 \cdot x)}$.
\zadStop
\rozwStart{Patryk Wirkus}{}
$$\lim\limits_{x\to\ 0}\frac{25 \cdot x}{tan(11 \cdot x)}=\lim\limits_{x\to\ 0}\frac{25 \cdot x \cdot cos(11 \cdot x)}{sin(11 \cdot x)}=\lim\limits_{x\to\ 0}\frac{25 \cdot cos(11 \cdot x)}{\frac{sin(11 \cdot x)}{x}}=\lim\limits_{x\to\ 0}\frac{25 \cdot cos(11 \cdot x)}{11 \cdot \frac{sin(11 \cdot x)}{11 \cdot x}} = \frac{25}{11}$$
\rozwStop
\odpStart
$\frac{25}{11}$
\odpStop
\testStart
A.$\frac{25}{11}$
B.$\infty$
C.$-\infty$
D.$0$
E.$-\frac{25}{11}$
F.$\frac{11}{25}$
G.$-\frac{11}{25}$
H.$11$
I.$25$
\testStop
\kluczStart
A
\kluczStop



\zadStart{Przykład z Wikieł P 4.3a moja wersja nr 494}


Obliczyć granicę funkcji $\lim\limits_{x\to\ 0}\frac{25 \cdot x}{tan(12 \cdot x)}$.
\zadStop
\rozwStart{Patryk Wirkus}{}
$$\lim\limits_{x\to\ 0}\frac{25 \cdot x}{tan(12 \cdot x)}=\lim\limits_{x\to\ 0}\frac{25 \cdot x \cdot cos(12 \cdot x)}{sin(12 \cdot x)}=\lim\limits_{x\to\ 0}\frac{25 \cdot cos(12 \cdot x)}{\frac{sin(12 \cdot x)}{x}}=\lim\limits_{x\to\ 0}\frac{25 \cdot cos(12 \cdot x)}{12 \cdot \frac{sin(12 \cdot x)}{12 \cdot x}} = \frac{25}{12}$$
\rozwStop
\odpStart
$\frac{25}{12}$
\odpStop
\testStart
A.$\frac{25}{12}$
B.$\infty$
C.$-\infty$
D.$0$
E.$-\frac{25}{12}$
F.$\frac{12}{25}$
G.$-\frac{12}{25}$
H.$12$
I.$25$
\testStop
\kluczStart
A
\kluczStop



\zadStart{Przykład z Wikieł P 4.3a moja wersja nr 495}


Obliczyć granicę funkcji $\lim\limits_{x\to\ 0}\frac{25 \cdot x}{tan(13 \cdot x)}$.
\zadStop
\rozwStart{Patryk Wirkus}{}
$$\lim\limits_{x\to\ 0}\frac{25 \cdot x}{tan(13 \cdot x)}=\lim\limits_{x\to\ 0}\frac{25 \cdot x \cdot cos(13 \cdot x)}{sin(13 \cdot x)}=\lim\limits_{x\to\ 0}\frac{25 \cdot cos(13 \cdot x)}{\frac{sin(13 \cdot x)}{x}}=\lim\limits_{x\to\ 0}\frac{25 \cdot cos(13 \cdot x)}{13 \cdot \frac{sin(13 \cdot x)}{13 \cdot x}} = \frac{25}{13}$$
\rozwStop
\odpStart
$\frac{25}{13}$
\odpStop
\testStart
A.$\frac{25}{13}$
B.$\infty$
C.$-\infty$
D.$0$
E.$-\frac{25}{13}$
F.$\frac{13}{25}$
G.$-\frac{13}{25}$
H.$13$
I.$25$
\testStop
\kluczStart
A
\kluczStop



\zadStart{Przykład z Wikieł P 4.3a moja wersja nr 496}


Obliczyć granicę funkcji $\lim\limits_{x\to\ 0}\frac{25 \cdot x}{tan(14 \cdot x)}$.
\zadStop
\rozwStart{Patryk Wirkus}{}
$$\lim\limits_{x\to\ 0}\frac{25 \cdot x}{tan(14 \cdot x)}=\lim\limits_{x\to\ 0}\frac{25 \cdot x \cdot cos(14 \cdot x)}{sin(14 \cdot x)}=\lim\limits_{x\to\ 0}\frac{25 \cdot cos(14 \cdot x)}{\frac{sin(14 \cdot x)}{x}}=\lim\limits_{x\to\ 0}\frac{25 \cdot cos(14 \cdot x)}{14 \cdot \frac{sin(14 \cdot x)}{14 \cdot x}} = \frac{25}{14}$$
\rozwStop
\odpStart
$\frac{25}{14}$
\odpStop
\testStart
A.$\frac{25}{14}$
B.$\infty$
C.$-\infty$
D.$0$
E.$-\frac{25}{14}$
F.$\frac{14}{25}$
G.$-\frac{14}{25}$
H.$14$
I.$25$
\testStop
\kluczStart
A
\kluczStop



\zadStart{Przykład z Wikieł P 4.3a moja wersja nr 497}


Obliczyć granicę funkcji $\lim\limits_{x\to\ 0}\frac{25 \cdot x}{tan(16 \cdot x)}$.
\zadStop
\rozwStart{Patryk Wirkus}{}
$$\lim\limits_{x\to\ 0}\frac{25 \cdot x}{tan(16 \cdot x)}=\lim\limits_{x\to\ 0}\frac{25 \cdot x \cdot cos(16 \cdot x)}{sin(16 \cdot x)}=\lim\limits_{x\to\ 0}\frac{25 \cdot cos(16 \cdot x)}{\frac{sin(16 \cdot x)}{x}}=\lim\limits_{x\to\ 0}\frac{25 \cdot cos(16 \cdot x)}{16 \cdot \frac{sin(16 \cdot x)}{16 \cdot x}} = \frac{25}{16}$$
\rozwStop
\odpStart
$\frac{25}{16}$
\odpStop
\testStart
A.$\frac{25}{16}$
B.$\infty$
C.$-\infty$
D.$0$
E.$-\frac{25}{16}$
F.$\frac{16}{25}$
G.$-\frac{16}{25}$
H.$16$
I.$25$
\testStop
\kluczStart
A
\kluczStop



\zadStart{Przykład z Wikieł P 4.3a moja wersja nr 498}


Obliczyć granicę funkcji $\lim\limits_{x\to\ 0}\frac{25 \cdot x}{tan(17 \cdot x)}$.
\zadStop
\rozwStart{Patryk Wirkus}{}
$$\lim\limits_{x\to\ 0}\frac{25 \cdot x}{tan(17 \cdot x)}=\lim\limits_{x\to\ 0}\frac{25 \cdot x \cdot cos(17 \cdot x)}{sin(17 \cdot x)}=\lim\limits_{x\to\ 0}\frac{25 \cdot cos(17 \cdot x)}{\frac{sin(17 \cdot x)}{x}}=\lim\limits_{x\to\ 0}\frac{25 \cdot cos(17 \cdot x)}{17 \cdot \frac{sin(17 \cdot x)}{17 \cdot x}} = \frac{25}{17}$$
\rozwStop
\odpStart
$\frac{25}{17}$
\odpStop
\testStart
A.$\frac{25}{17}$
B.$\infty$
C.$-\infty$
D.$0$
E.$-\frac{25}{17}$
F.$\frac{17}{25}$
G.$-\frac{17}{25}$
H.$17$
I.$25$
\testStop
\kluczStart
A
\kluczStop



\zadStart{Przykład z Wikieł P 4.3a moja wersja nr 499}


Obliczyć granicę funkcji $\lim\limits_{x\to\ 0}\frac{25 \cdot x}{tan(18 \cdot x)}$.
\zadStop
\rozwStart{Patryk Wirkus}{}
$$\lim\limits_{x\to\ 0}\frac{25 \cdot x}{tan(18 \cdot x)}=\lim\limits_{x\to\ 0}\frac{25 \cdot x \cdot cos(18 \cdot x)}{sin(18 \cdot x)}=\lim\limits_{x\to\ 0}\frac{25 \cdot cos(18 \cdot x)}{\frac{sin(18 \cdot x)}{x}}=\lim\limits_{x\to\ 0}\frac{25 \cdot cos(18 \cdot x)}{18 \cdot \frac{sin(18 \cdot x)}{18 \cdot x}} = \frac{25}{18}$$
\rozwStop
\odpStart
$\frac{25}{18}$
\odpStop
\testStart
A.$\frac{25}{18}$
B.$\infty$
C.$-\infty$
D.$0$
E.$-\frac{25}{18}$
F.$\frac{18}{25}$
G.$-\frac{18}{25}$
H.$18$
I.$25$
\testStop
\kluczStart
A
\kluczStop



\zadStart{Przykład z Wikieł P 4.3a moja wersja nr 500}


Obliczyć granicę funkcji $\lim\limits_{x\to\ 0}\frac{25 \cdot x}{tan(19 \cdot x)}$.
\zadStop
\rozwStart{Patryk Wirkus}{}
$$\lim\limits_{x\to\ 0}\frac{25 \cdot x}{tan(19 \cdot x)}=\lim\limits_{x\to\ 0}\frac{25 \cdot x \cdot cos(19 \cdot x)}{sin(19 \cdot x)}=\lim\limits_{x\to\ 0}\frac{25 \cdot cos(19 \cdot x)}{\frac{sin(19 \cdot x)}{x}}=\lim\limits_{x\to\ 0}\frac{25 \cdot cos(19 \cdot x)}{19 \cdot \frac{sin(19 \cdot x)}{19 \cdot x}} = \frac{25}{19}$$
\rozwStop
\odpStart
$\frac{25}{19}$
\odpStop
\testStart
A.$\frac{25}{19}$
B.$\infty$
C.$-\infty$
D.$0$
E.$-\frac{25}{19}$
F.$\frac{19}{25}$
G.$-\frac{19}{25}$
H.$19$
I.$25$
\testStop
\kluczStart
A
\kluczStop



\zadStart{Przykład z Wikieł P 4.3a moja wersja nr 501}


Obliczyć granicę funkcji $\lim\limits_{x\to\ 0}\frac{25 \cdot x}{tan(21 \cdot x)}$.
\zadStop
\rozwStart{Patryk Wirkus}{}
$$\lim\limits_{x\to\ 0}\frac{25 \cdot x}{tan(21 \cdot x)}=\lim\limits_{x\to\ 0}\frac{25 \cdot x \cdot cos(21 \cdot x)}{sin(21 \cdot x)}=\lim\limits_{x\to\ 0}\frac{25 \cdot cos(21 \cdot x)}{\frac{sin(21 \cdot x)}{x}}=\lim\limits_{x\to\ 0}\frac{25 \cdot cos(21 \cdot x)}{21 \cdot \frac{sin(21 \cdot x)}{21 \cdot x}} = \frac{25}{21}$$
\rozwStop
\odpStart
$\frac{25}{21}$
\odpStop
\testStart
A.$\frac{25}{21}$
B.$\infty$
C.$-\infty$
D.$0$
E.$-\frac{25}{21}$
F.$\frac{21}{25}$
G.$-\frac{21}{25}$
H.$21$
I.$25$
\testStop
\kluczStart
A
\kluczStop



\zadStart{Przykład z Wikieł P 4.3a moja wersja nr 502}


Obliczyć granicę funkcji $\lim\limits_{x\to\ 0}\frac{25 \cdot x}{tan(22 \cdot x)}$.
\zadStop
\rozwStart{Patryk Wirkus}{}
$$\lim\limits_{x\to\ 0}\frac{25 \cdot x}{tan(22 \cdot x)}=\lim\limits_{x\to\ 0}\frac{25 \cdot x \cdot cos(22 \cdot x)}{sin(22 \cdot x)}=\lim\limits_{x\to\ 0}\frac{25 \cdot cos(22 \cdot x)}{\frac{sin(22 \cdot x)}{x}}=\lim\limits_{x\to\ 0}\frac{25 \cdot cos(22 \cdot x)}{22 \cdot \frac{sin(22 \cdot x)}{22 \cdot x}} = \frac{25}{22}$$
\rozwStop
\odpStart
$\frac{25}{22}$
\odpStop
\testStart
A.$\frac{25}{22}$
B.$\infty$
C.$-\infty$
D.$0$
E.$-\frac{25}{22}$
F.$\frac{22}{25}$
G.$-\frac{22}{25}$
H.$22$
I.$25$
\testStop
\kluczStart
A
\kluczStop



\zadStart{Przykład z Wikieł P 4.3a moja wersja nr 503}


Obliczyć granicę funkcji $\lim\limits_{x\to\ 0}\frac{25 \cdot x}{tan(23 \cdot x)}$.
\zadStop
\rozwStart{Patryk Wirkus}{}
$$\lim\limits_{x\to\ 0}\frac{25 \cdot x}{tan(23 \cdot x)}=\lim\limits_{x\to\ 0}\frac{25 \cdot x \cdot cos(23 \cdot x)}{sin(23 \cdot x)}=\lim\limits_{x\to\ 0}\frac{25 \cdot cos(23 \cdot x)}{\frac{sin(23 \cdot x)}{x}}=\lim\limits_{x\to\ 0}\frac{25 \cdot cos(23 \cdot x)}{23 \cdot \frac{sin(23 \cdot x)}{23 \cdot x}} = \frac{25}{23}$$
\rozwStop
\odpStart
$\frac{25}{23}$
\odpStop
\testStart
A.$\frac{25}{23}$
B.$\infty$
C.$-\infty$
D.$0$
E.$-\frac{25}{23}$
F.$\frac{23}{25}$
G.$-\frac{23}{25}$
H.$23$
I.$25$
\testStop
\kluczStart
A
\kluczStop



\zadStart{Przykład z Wikieł P 4.3a moja wersja nr 504}


Obliczyć granicę funkcji $\lim\limits_{x\to\ 0}\frac{25 \cdot x}{tan(24 \cdot x)}$.
\zadStop
\rozwStart{Patryk Wirkus}{}
$$\lim\limits_{x\to\ 0}\frac{25 \cdot x}{tan(24 \cdot x)}=\lim\limits_{x\to\ 0}\frac{25 \cdot x \cdot cos(24 \cdot x)}{sin(24 \cdot x)}=\lim\limits_{x\to\ 0}\frac{25 \cdot cos(24 \cdot x)}{\frac{sin(24 \cdot x)}{x}}=\lim\limits_{x\to\ 0}\frac{25 \cdot cos(24 \cdot x)}{24 \cdot \frac{sin(24 \cdot x)}{24 \cdot x}} = \frac{25}{24}$$
\rozwStop
\odpStart
$\frac{25}{24}$
\odpStop
\testStart
A.$\frac{25}{24}$
B.$\infty$
C.$-\infty$
D.$0$
E.$-\frac{25}{24}$
F.$\frac{24}{25}$
G.$-\frac{24}{25}$
H.$24$
I.$25$
\testStop
\kluczStart
A
\kluczStop



\zadStart{Przykład z Wikieł P 4.3a moja wersja nr 505}


Obliczyć granicę funkcji $\lim\limits_{x\to\ 0}\frac{25 \cdot x}{tan(26 \cdot x)}$.
\zadStop
\rozwStart{Patryk Wirkus}{}
$$\lim\limits_{x\to\ 0}\frac{25 \cdot x}{tan(26 \cdot x)}=\lim\limits_{x\to\ 0}\frac{25 \cdot x \cdot cos(26 \cdot x)}{sin(26 \cdot x)}=\lim\limits_{x\to\ 0}\frac{25 \cdot cos(26 \cdot x)}{\frac{sin(26 \cdot x)}{x}}=\lim\limits_{x\to\ 0}\frac{25 \cdot cos(26 \cdot x)}{26 \cdot \frac{sin(26 \cdot x)}{26 \cdot x}} = \frac{25}{26}$$
\rozwStop
\odpStart
$\frac{25}{26}$
\odpStop
\testStart
A.$\frac{25}{26}$
B.$\infty$
C.$-\infty$
D.$0$
E.$-\frac{25}{26}$
F.$\frac{26}{25}$
G.$-\frac{26}{25}$
H.$26$
I.$25$
\testStop
\kluczStart
A
\kluczStop



\zadStart{Przykład z Wikieł P 4.3a moja wersja nr 506}


Obliczyć granicę funkcji $\lim\limits_{x\to\ 0}\frac{25 \cdot x}{tan(27 \cdot x)}$.
\zadStop
\rozwStart{Patryk Wirkus}{}
$$\lim\limits_{x\to\ 0}\frac{25 \cdot x}{tan(27 \cdot x)}=\lim\limits_{x\to\ 0}\frac{25 \cdot x \cdot cos(27 \cdot x)}{sin(27 \cdot x)}=\lim\limits_{x\to\ 0}\frac{25 \cdot cos(27 \cdot x)}{\frac{sin(27 \cdot x)}{x}}=\lim\limits_{x\to\ 0}\frac{25 \cdot cos(27 \cdot x)}{27 \cdot \frac{sin(27 \cdot x)}{27 \cdot x}} = \frac{25}{27}$$
\rozwStop
\odpStart
$\frac{25}{27}$
\odpStop
\testStart
A.$\frac{25}{27}$
B.$\infty$
C.$-\infty$
D.$0$
E.$-\frac{25}{27}$
F.$\frac{27}{25}$
G.$-\frac{27}{25}$
H.$27$
I.$25$
\testStop
\kluczStart
A
\kluczStop



\zadStart{Przykład z Wikieł P 4.3a moja wersja nr 507}


Obliczyć granicę funkcji $\lim\limits_{x\to\ 0}\frac{25 \cdot x}{tan(28 \cdot x)}$.
\zadStop
\rozwStart{Patryk Wirkus}{}
$$\lim\limits_{x\to\ 0}\frac{25 \cdot x}{tan(28 \cdot x)}=\lim\limits_{x\to\ 0}\frac{25 \cdot x \cdot cos(28 \cdot x)}{sin(28 \cdot x)}=\lim\limits_{x\to\ 0}\frac{25 \cdot cos(28 \cdot x)}{\frac{sin(28 \cdot x)}{x}}=\lim\limits_{x\to\ 0}\frac{25 \cdot cos(28 \cdot x)}{28 \cdot \frac{sin(28 \cdot x)}{28 \cdot x}} = \frac{25}{28}$$
\rozwStop
\odpStart
$\frac{25}{28}$
\odpStop
\testStart
A.$\frac{25}{28}$
B.$\infty$
C.$-\infty$
D.$0$
E.$-\frac{25}{28}$
F.$\frac{28}{25}$
G.$-\frac{28}{25}$
H.$28$
I.$25$
\testStop
\kluczStart
A
\kluczStop



\zadStart{Przykład z Wikieł P 4.3a moja wersja nr 508}


Obliczyć granicę funkcji $\lim\limits_{x\to\ 0}\frac{25 \cdot x}{tan(29 \cdot x)}$.
\zadStop
\rozwStart{Patryk Wirkus}{}
$$\lim\limits_{x\to\ 0}\frac{25 \cdot x}{tan(29 \cdot x)}=\lim\limits_{x\to\ 0}\frac{25 \cdot x \cdot cos(29 \cdot x)}{sin(29 \cdot x)}=\lim\limits_{x\to\ 0}\frac{25 \cdot cos(29 \cdot x)}{\frac{sin(29 \cdot x)}{x}}=\lim\limits_{x\to\ 0}\frac{25 \cdot cos(29 \cdot x)}{29 \cdot \frac{sin(29 \cdot x)}{29 \cdot x}} = \frac{25}{29}$$
\rozwStop
\odpStart
$\frac{25}{29}$
\odpStop
\testStart
A.$\frac{25}{29}$
B.$\infty$
C.$-\infty$
D.$0$
E.$-\frac{25}{29}$
F.$\frac{29}{25}$
G.$-\frac{29}{25}$
H.$29$
I.$25$
\testStop
\kluczStart
A
\kluczStop



\zadStart{Przykład z Wikieł P 4.3a moja wersja nr 509}


Obliczyć granicę funkcji $\lim\limits_{x\to\ 0}\frac{25 \cdot x}{tan(31 \cdot x)}$.
\zadStop
\rozwStart{Patryk Wirkus}{}
$$\lim\limits_{x\to\ 0}\frac{25 \cdot x}{tan(31 \cdot x)}=\lim\limits_{x\to\ 0}\frac{25 \cdot x \cdot cos(31 \cdot x)}{sin(31 \cdot x)}=\lim\limits_{x\to\ 0}\frac{25 \cdot cos(31 \cdot x)}{\frac{sin(31 \cdot x)}{x}}=\lim\limits_{x\to\ 0}\frac{25 \cdot cos(31 \cdot x)}{31 \cdot \frac{sin(31 \cdot x)}{31 \cdot x}} = \frac{25}{31}$$
\rozwStop
\odpStart
$\frac{25}{31}$
\odpStop
\testStart
A.$\frac{25}{31}$
B.$\infty$
C.$-\infty$
D.$0$
E.$-\frac{25}{31}$
F.$\frac{31}{25}$
G.$-\frac{31}{25}$
H.$31$
I.$25$
\testStop
\kluczStart
A
\kluczStop



\zadStart{Przykład z Wikieł P 4.3a moja wersja nr 510}


Obliczyć granicę funkcji $\lim\limits_{x\to\ 0}\frac{25 \cdot x}{tan(32 \cdot x)}$.
\zadStop
\rozwStart{Patryk Wirkus}{}
$$\lim\limits_{x\to\ 0}\frac{25 \cdot x}{tan(32 \cdot x)}=\lim\limits_{x\to\ 0}\frac{25 \cdot x \cdot cos(32 \cdot x)}{sin(32 \cdot x)}=\lim\limits_{x\to\ 0}\frac{25 \cdot cos(32 \cdot x)}{\frac{sin(32 \cdot x)}{x}}=\lim\limits_{x\to\ 0}\frac{25 \cdot cos(32 \cdot x)}{32 \cdot \frac{sin(32 \cdot x)}{32 \cdot x}} = \frac{25}{32}$$
\rozwStop
\odpStart
$\frac{25}{32}$
\odpStop
\testStart
A.$\frac{25}{32}$
B.$\infty$
C.$-\infty$
D.$0$
E.$-\frac{25}{32}$
F.$\frac{32}{25}$
G.$-\frac{32}{25}$
H.$32$
I.$25$
\testStop
\kluczStart
A
\kluczStop



\zadStart{Przykład z Wikieł P 4.3a moja wersja nr 511}


Obliczyć granicę funkcji $\lim\limits_{x\to\ 0}\frac{25 \cdot x}{tan(33 \cdot x)}$.
\zadStop
\rozwStart{Patryk Wirkus}{}
$$\lim\limits_{x\to\ 0}\frac{25 \cdot x}{tan(33 \cdot x)}=\lim\limits_{x\to\ 0}\frac{25 \cdot x \cdot cos(33 \cdot x)}{sin(33 \cdot x)}=\lim\limits_{x\to\ 0}\frac{25 \cdot cos(33 \cdot x)}{\frac{sin(33 \cdot x)}{x}}=\lim\limits_{x\to\ 0}\frac{25 \cdot cos(33 \cdot x)}{33 \cdot \frac{sin(33 \cdot x)}{33 \cdot x}} = \frac{25}{33}$$
\rozwStop
\odpStart
$\frac{25}{33}$
\odpStop
\testStart
A.$\frac{25}{33}$
B.$\infty$
C.$-\infty$
D.$0$
E.$-\frac{25}{33}$
F.$\frac{33}{25}$
G.$-\frac{33}{25}$
H.$33$
I.$25$
\testStop
\kluczStart
A
\kluczStop



\zadStart{Przykład z Wikieł P 4.3a moja wersja nr 512}


Obliczyć granicę funkcji $\lim\limits_{x\to\ 0}\frac{25 \cdot x}{tan(34 \cdot x)}$.
\zadStop
\rozwStart{Patryk Wirkus}{}
$$\lim\limits_{x\to\ 0}\frac{25 \cdot x}{tan(34 \cdot x)}=\lim\limits_{x\to\ 0}\frac{25 \cdot x \cdot cos(34 \cdot x)}{sin(34 \cdot x)}=\lim\limits_{x\to\ 0}\frac{25 \cdot cos(34 \cdot x)}{\frac{sin(34 \cdot x)}{x}}=\lim\limits_{x\to\ 0}\frac{25 \cdot cos(34 \cdot x)}{34 \cdot \frac{sin(34 \cdot x)}{34 \cdot x}} = \frac{25}{34}$$
\rozwStop
\odpStart
$\frac{25}{34}$
\odpStop
\testStart
A.$\frac{25}{34}$
B.$\infty$
C.$-\infty$
D.$0$
E.$-\frac{25}{34}$
F.$\frac{34}{25}$
G.$-\frac{34}{25}$
H.$34$
I.$25$
\testStop
\kluczStart
A
\kluczStop



\zadStart{Przykład z Wikieł P 4.3a moja wersja nr 513}


Obliczyć granicę funkcji $\lim\limits_{x\to\ 0}\frac{25 \cdot x}{tan(36 \cdot x)}$.
\zadStop
\rozwStart{Patryk Wirkus}{}
$$\lim\limits_{x\to\ 0}\frac{25 \cdot x}{tan(36 \cdot x)}=\lim\limits_{x\to\ 0}\frac{25 \cdot x \cdot cos(36 \cdot x)}{sin(36 \cdot x)}=\lim\limits_{x\to\ 0}\frac{25 \cdot cos(36 \cdot x)}{\frac{sin(36 \cdot x)}{x}}=\lim\limits_{x\to\ 0}\frac{25 \cdot cos(36 \cdot x)}{36 \cdot \frac{sin(36 \cdot x)}{36 \cdot x}} = \frac{25}{36}$$
\rozwStop
\odpStart
$\frac{25}{36}$
\odpStop
\testStart
A.$\frac{25}{36}$
B.$\infty$
C.$-\infty$
D.$0$
E.$-\frac{25}{36}$
F.$\frac{36}{25}$
G.$-\frac{36}{25}$
H.$36$
I.$25$
\testStop
\kluczStart
A
\kluczStop



\zadStart{Przykład z Wikieł P 4.3a moja wersja nr 514}


Obliczyć granicę funkcji $\lim\limits_{x\to\ 0}\frac{25 \cdot x}{tan(37 \cdot x)}$.
\zadStop
\rozwStart{Patryk Wirkus}{}
$$\lim\limits_{x\to\ 0}\frac{25 \cdot x}{tan(37 \cdot x)}=\lim\limits_{x\to\ 0}\frac{25 \cdot x \cdot cos(37 \cdot x)}{sin(37 \cdot x)}=\lim\limits_{x\to\ 0}\frac{25 \cdot cos(37 \cdot x)}{\frac{sin(37 \cdot x)}{x}}=\lim\limits_{x\to\ 0}\frac{25 \cdot cos(37 \cdot x)}{37 \cdot \frac{sin(37 \cdot x)}{37 \cdot x}} = \frac{25}{37}$$
\rozwStop
\odpStart
$\frac{25}{37}$
\odpStop
\testStart
A.$\frac{25}{37}$
B.$\infty$
C.$-\infty$
D.$0$
E.$-\frac{25}{37}$
F.$\frac{37}{25}$
G.$-\frac{37}{25}$
H.$37$
I.$25$
\testStop
\kluczStart
A
\kluczStop



\zadStart{Przykład z Wikieł P 4.3a moja wersja nr 515}


Obliczyć granicę funkcji $\lim\limits_{x\to\ 0}\frac{25 \cdot x}{tan(38 \cdot x)}$.
\zadStop
\rozwStart{Patryk Wirkus}{}
$$\lim\limits_{x\to\ 0}\frac{25 \cdot x}{tan(38 \cdot x)}=\lim\limits_{x\to\ 0}\frac{25 \cdot x \cdot cos(38 \cdot x)}{sin(38 \cdot x)}=\lim\limits_{x\to\ 0}\frac{25 \cdot cos(38 \cdot x)}{\frac{sin(38 \cdot x)}{x}}=\lim\limits_{x\to\ 0}\frac{25 \cdot cos(38 \cdot x)}{38 \cdot \frac{sin(38 \cdot x)}{38 \cdot x}} = \frac{25}{38}$$
\rozwStop
\odpStart
$\frac{25}{38}$
\odpStop
\testStart
A.$\frac{25}{38}$
B.$\infty$
C.$-\infty$
D.$0$
E.$-\frac{25}{38}$
F.$\frac{38}{25}$
G.$-\frac{38}{25}$
H.$38$
I.$25$
\testStop
\kluczStart
A
\kluczStop



\zadStart{Przykład z Wikieł P 4.3a moja wersja nr 516}


Obliczyć granicę funkcji $\lim\limits_{x\to\ 0}\frac{25 \cdot x}{tan(39 \cdot x)}$.
\zadStop
\rozwStart{Patryk Wirkus}{}
$$\lim\limits_{x\to\ 0}\frac{25 \cdot x}{tan(39 \cdot x)}=\lim\limits_{x\to\ 0}\frac{25 \cdot x \cdot cos(39 \cdot x)}{sin(39 \cdot x)}=\lim\limits_{x\to\ 0}\frac{25 \cdot cos(39 \cdot x)}{\frac{sin(39 \cdot x)}{x}}=\lim\limits_{x\to\ 0}\frac{25 \cdot cos(39 \cdot x)}{39 \cdot \frac{sin(39 \cdot x)}{39 \cdot x}} = \frac{25}{39}$$
\rozwStop
\odpStart
$\frac{25}{39}$
\odpStop
\testStart
A.$\frac{25}{39}$
B.$\infty$
C.$-\infty$
D.$0$
E.$-\frac{25}{39}$
F.$\frac{39}{25}$
G.$-\frac{39}{25}$
H.$39$
I.$25$
\testStop
\kluczStart
A
\kluczStop



\zadStart{Przykład z Wikieł P 4.3a moja wersja nr 517}


Obliczyć granicę funkcji $\lim\limits_{x\to\ 0}\frac{26 \cdot x}{tan(3 \cdot x)}$.
\zadStop
\rozwStart{Patryk Wirkus}{}
$$\lim\limits_{x\to\ 0}\frac{26 \cdot x}{tan(3 \cdot x)}=\lim\limits_{x\to\ 0}\frac{26 \cdot x \cdot cos(3 \cdot x)}{sin(3 \cdot x)}=\lim\limits_{x\to\ 0}\frac{26 \cdot cos(3 \cdot x)}{\frac{sin(3 \cdot x)}{x}}=\lim\limits_{x\to\ 0}\frac{26 \cdot cos(3 \cdot x)}{3 \cdot \frac{sin(3 \cdot x)}{3 \cdot x}} = \frac{26}{3}$$
\rozwStop
\odpStart
$\frac{26}{3}$
\odpStop
\testStart
A.$\frac{26}{3}$
B.$\infty$
C.$-\infty$
D.$0$
E.$-\frac{26}{3}$
F.$\frac{3}{26}$
G.$-\frac{3}{26}$
H.$3$
I.$26$
\testStop
\kluczStart
A
\kluczStop



\zadStart{Przykład z Wikieł P 4.3a moja wersja nr 518}


Obliczyć granicę funkcji $\lim\limits_{x\to\ 0}\frac{26 \cdot x}{tan(5 \cdot x)}$.
\zadStop
\rozwStart{Patryk Wirkus}{}
$$\lim\limits_{x\to\ 0}\frac{26 \cdot x}{tan(5 \cdot x)}=\lim\limits_{x\to\ 0}\frac{26 \cdot x \cdot cos(5 \cdot x)}{sin(5 \cdot x)}=\lim\limits_{x\to\ 0}\frac{26 \cdot cos(5 \cdot x)}{\frac{sin(5 \cdot x)}{x}}=\lim\limits_{x\to\ 0}\frac{26 \cdot cos(5 \cdot x)}{5 \cdot \frac{sin(5 \cdot x)}{5 \cdot x}} = \frac{26}{5}$$
\rozwStop
\odpStart
$\frac{26}{5}$
\odpStop
\testStart
A.$\frac{26}{5}$
B.$\infty$
C.$-\infty$
D.$0$
E.$-\frac{26}{5}$
F.$\frac{5}{26}$
G.$-\frac{5}{26}$
H.$5$
I.$26$
\testStop
\kluczStart
A
\kluczStop



\zadStart{Przykład z Wikieł P 4.3a moja wersja nr 519}


Obliczyć granicę funkcji $\lim\limits_{x\to\ 0}\frac{26 \cdot x}{tan(7 \cdot x)}$.
\zadStop
\rozwStart{Patryk Wirkus}{}
$$\lim\limits_{x\to\ 0}\frac{26 \cdot x}{tan(7 \cdot x)}=\lim\limits_{x\to\ 0}\frac{26 \cdot x \cdot cos(7 \cdot x)}{sin(7 \cdot x)}=\lim\limits_{x\to\ 0}\frac{26 \cdot cos(7 \cdot x)}{\frac{sin(7 \cdot x)}{x}}=\lim\limits_{x\to\ 0}\frac{26 \cdot cos(7 \cdot x)}{7 \cdot \frac{sin(7 \cdot x)}{7 \cdot x}} = \frac{26}{7}$$
\rozwStop
\odpStart
$\frac{26}{7}$
\odpStop
\testStart
A.$\frac{26}{7}$
B.$\infty$
C.$-\infty$
D.$0$
E.$-\frac{26}{7}$
F.$\frac{7}{26}$
G.$-\frac{7}{26}$
H.$7$
I.$26$
\testStop
\kluczStart
A
\kluczStop



\zadStart{Przykład z Wikieł P 4.3a moja wersja nr 520}


Obliczyć granicę funkcji $\lim\limits_{x\to\ 0}\frac{26 \cdot x}{tan(9 \cdot x)}$.
\zadStop
\rozwStart{Patryk Wirkus}{}
$$\lim\limits_{x\to\ 0}\frac{26 \cdot x}{tan(9 \cdot x)}=\lim\limits_{x\to\ 0}\frac{26 \cdot x \cdot cos(9 \cdot x)}{sin(9 \cdot x)}=\lim\limits_{x\to\ 0}\frac{26 \cdot cos(9 \cdot x)}{\frac{sin(9 \cdot x)}{x}}=\lim\limits_{x\to\ 0}\frac{26 \cdot cos(9 \cdot x)}{9 \cdot \frac{sin(9 \cdot x)}{9 \cdot x}} = \frac{26}{9}$$
\rozwStop
\odpStart
$\frac{26}{9}$
\odpStop
\testStart
A.$\frac{26}{9}$
B.$\infty$
C.$-\infty$
D.$0$
E.$-\frac{26}{9}$
F.$\frac{9}{26}$
G.$-\frac{9}{26}$
H.$9$
I.$26$
\testStop
\kluczStart
A
\kluczStop



\zadStart{Przykład z Wikieł P 4.3a moja wersja nr 521}


Obliczyć granicę funkcji $\lim\limits_{x\to\ 0}\frac{26 \cdot x}{tan(11 \cdot x)}$.
\zadStop
\rozwStart{Patryk Wirkus}{}
$$\lim\limits_{x\to\ 0}\frac{26 \cdot x}{tan(11 \cdot x)}=\lim\limits_{x\to\ 0}\frac{26 \cdot x \cdot cos(11 \cdot x)}{sin(11 \cdot x)}=\lim\limits_{x\to\ 0}\frac{26 \cdot cos(11 \cdot x)}{\frac{sin(11 \cdot x)}{x}}=\lim\limits_{x\to\ 0}\frac{26 \cdot cos(11 \cdot x)}{11 \cdot \frac{sin(11 \cdot x)}{11 \cdot x}} = \frac{26}{11}$$
\rozwStop
\odpStart
$\frac{26}{11}$
\odpStop
\testStart
A.$\frac{26}{11}$
B.$\infty$
C.$-\infty$
D.$0$
E.$-\frac{26}{11}$
F.$\frac{11}{26}$
G.$-\frac{11}{26}$
H.$11$
I.$26$
\testStop
\kluczStart
A
\kluczStop



\zadStart{Przykład z Wikieł P 4.3a moja wersja nr 522}


Obliczyć granicę funkcji $\lim\limits_{x\to\ 0}\frac{26 \cdot x}{tan(15 \cdot x)}$.
\zadStop
\rozwStart{Patryk Wirkus}{}
$$\lim\limits_{x\to\ 0}\frac{26 \cdot x}{tan(15 \cdot x)}=\lim\limits_{x\to\ 0}\frac{26 \cdot x \cdot cos(15 \cdot x)}{sin(15 \cdot x)}=\lim\limits_{x\to\ 0}\frac{26 \cdot cos(15 \cdot x)}{\frac{sin(15 \cdot x)}{x}}=\lim\limits_{x\to\ 0}\frac{26 \cdot cos(15 \cdot x)}{15 \cdot \frac{sin(15 \cdot x)}{15 \cdot x}} = \frac{26}{15}$$
\rozwStop
\odpStart
$\frac{26}{15}$
\odpStop
\testStart
A.$\frac{26}{15}$
B.$\infty$
C.$-\infty$
D.$0$
E.$-\frac{26}{15}$
F.$\frac{15}{26}$
G.$-\frac{15}{26}$
H.$15$
I.$26$
\testStop
\kluczStart
A
\kluczStop



\zadStart{Przykład z Wikieł P 4.3a moja wersja nr 523}


Obliczyć granicę funkcji $\lim\limits_{x\to\ 0}\frac{26 \cdot x}{tan(17 \cdot x)}$.
\zadStop
\rozwStart{Patryk Wirkus}{}
$$\lim\limits_{x\to\ 0}\frac{26 \cdot x}{tan(17 \cdot x)}=\lim\limits_{x\to\ 0}\frac{26 \cdot x \cdot cos(17 \cdot x)}{sin(17 \cdot x)}=\lim\limits_{x\to\ 0}\frac{26 \cdot cos(17 \cdot x)}{\frac{sin(17 \cdot x)}{x}}=\lim\limits_{x\to\ 0}\frac{26 \cdot cos(17 \cdot x)}{17 \cdot \frac{sin(17 \cdot x)}{17 \cdot x}} = \frac{26}{17}$$
\rozwStop
\odpStart
$\frac{26}{17}$
\odpStop
\testStart
A.$\frac{26}{17}$
B.$\infty$
C.$-\infty$
D.$0$
E.$-\frac{26}{17}$
F.$\frac{17}{26}$
G.$-\frac{17}{26}$
H.$17$
I.$26$
\testStop
\kluczStart
A
\kluczStop



\zadStart{Przykład z Wikieł P 4.3a moja wersja nr 524}


Obliczyć granicę funkcji $\lim\limits_{x\to\ 0}\frac{26 \cdot x}{tan(19 \cdot x)}$.
\zadStop
\rozwStart{Patryk Wirkus}{}
$$\lim\limits_{x\to\ 0}\frac{26 \cdot x}{tan(19 \cdot x)}=\lim\limits_{x\to\ 0}\frac{26 \cdot x \cdot cos(19 \cdot x)}{sin(19 \cdot x)}=\lim\limits_{x\to\ 0}\frac{26 \cdot cos(19 \cdot x)}{\frac{sin(19 \cdot x)}{x}}=\lim\limits_{x\to\ 0}\frac{26 \cdot cos(19 \cdot x)}{19 \cdot \frac{sin(19 \cdot x)}{19 \cdot x}} = \frac{26}{19}$$
\rozwStop
\odpStart
$\frac{26}{19}$
\odpStop
\testStart
A.$\frac{26}{19}$
B.$\infty$
C.$-\infty$
D.$0$
E.$-\frac{26}{19}$
F.$\frac{19}{26}$
G.$-\frac{19}{26}$
H.$19$
I.$26$
\testStop
\kluczStart
A
\kluczStop



\zadStart{Przykład z Wikieł P 4.3a moja wersja nr 525}


Obliczyć granicę funkcji $\lim\limits_{x\to\ 0}\frac{26 \cdot x}{tan(21 \cdot x)}$.
\zadStop
\rozwStart{Patryk Wirkus}{}
$$\lim\limits_{x\to\ 0}\frac{26 \cdot x}{tan(21 \cdot x)}=\lim\limits_{x\to\ 0}\frac{26 \cdot x \cdot cos(21 \cdot x)}{sin(21 \cdot x)}=\lim\limits_{x\to\ 0}\frac{26 \cdot cos(21 \cdot x)}{\frac{sin(21 \cdot x)}{x}}=\lim\limits_{x\to\ 0}\frac{26 \cdot cos(21 \cdot x)}{21 \cdot \frac{sin(21 \cdot x)}{21 \cdot x}} = \frac{26}{21}$$
\rozwStop
\odpStart
$\frac{26}{21}$
\odpStop
\testStart
A.$\frac{26}{21}$
B.$\infty$
C.$-\infty$
D.$0$
E.$-\frac{26}{21}$
F.$\frac{21}{26}$
G.$-\frac{21}{26}$
H.$21$
I.$26$
\testStop
\kluczStart
A
\kluczStop



\zadStart{Przykład z Wikieł P 4.3a moja wersja nr 526}


Obliczyć granicę funkcji $\lim\limits_{x\to\ 0}\frac{26 \cdot x}{tan(23 \cdot x)}$.
\zadStop
\rozwStart{Patryk Wirkus}{}
$$\lim\limits_{x\to\ 0}\frac{26 \cdot x}{tan(23 \cdot x)}=\lim\limits_{x\to\ 0}\frac{26 \cdot x \cdot cos(23 \cdot x)}{sin(23 \cdot x)}=\lim\limits_{x\to\ 0}\frac{26 \cdot cos(23 \cdot x)}{\frac{sin(23 \cdot x)}{x}}=\lim\limits_{x\to\ 0}\frac{26 \cdot cos(23 \cdot x)}{23 \cdot \frac{sin(23 \cdot x)}{23 \cdot x}} = \frac{26}{23}$$
\rozwStop
\odpStart
$\frac{26}{23}$
\odpStop
\testStart
A.$\frac{26}{23}$
B.$\infty$
C.$-\infty$
D.$0$
E.$-\frac{26}{23}$
F.$\frac{23}{26}$
G.$-\frac{23}{26}$
H.$23$
I.$26$
\testStop
\kluczStart
A
\kluczStop



\zadStart{Przykład z Wikieł P 4.3a moja wersja nr 527}


Obliczyć granicę funkcji $\lim\limits_{x\to\ 0}\frac{26 \cdot x}{tan(25 \cdot x)}$.
\zadStop
\rozwStart{Patryk Wirkus}{}
$$\lim\limits_{x\to\ 0}\frac{26 \cdot x}{tan(25 \cdot x)}=\lim\limits_{x\to\ 0}\frac{26 \cdot x \cdot cos(25 \cdot x)}{sin(25 \cdot x)}=\lim\limits_{x\to\ 0}\frac{26 \cdot cos(25 \cdot x)}{\frac{sin(25 \cdot x)}{x}}=\lim\limits_{x\to\ 0}\frac{26 \cdot cos(25 \cdot x)}{25 \cdot \frac{sin(25 \cdot x)}{25 \cdot x}} = \frac{26}{25}$$
\rozwStop
\odpStart
$\frac{26}{25}$
\odpStop
\testStart
A.$\frac{26}{25}$
B.$\infty$
C.$-\infty$
D.$0$
E.$-\frac{26}{25}$
F.$\frac{25}{26}$
G.$-\frac{25}{26}$
H.$25$
I.$26$
\testStop
\kluczStart
A
\kluczStop



\zadStart{Przykład z Wikieł P 4.3a moja wersja nr 528}


Obliczyć granicę funkcji $\lim\limits_{x\to\ 0}\frac{26 \cdot x}{tan(27 \cdot x)}$.
\zadStop
\rozwStart{Patryk Wirkus}{}
$$\lim\limits_{x\to\ 0}\frac{26 \cdot x}{tan(27 \cdot x)}=\lim\limits_{x\to\ 0}\frac{26 \cdot x \cdot cos(27 \cdot x)}{sin(27 \cdot x)}=\lim\limits_{x\to\ 0}\frac{26 \cdot cos(27 \cdot x)}{\frac{sin(27 \cdot x)}{x}}=\lim\limits_{x\to\ 0}\frac{26 \cdot cos(27 \cdot x)}{27 \cdot \frac{sin(27 \cdot x)}{27 \cdot x}} = \frac{26}{27}$$
\rozwStop
\odpStart
$\frac{26}{27}$
\odpStop
\testStart
A.$\frac{26}{27}$
B.$\infty$
C.$-\infty$
D.$0$
E.$-\frac{26}{27}$
F.$\frac{27}{26}$
G.$-\frac{27}{26}$
H.$27$
I.$26$
\testStop
\kluczStart
A
\kluczStop



\zadStart{Przykład z Wikieł P 4.3a moja wersja nr 529}


Obliczyć granicę funkcji $\lim\limits_{x\to\ 0}\frac{26 \cdot x}{tan(29 \cdot x)}$.
\zadStop
\rozwStart{Patryk Wirkus}{}
$$\lim\limits_{x\to\ 0}\frac{26 \cdot x}{tan(29 \cdot x)}=\lim\limits_{x\to\ 0}\frac{26 \cdot x \cdot cos(29 \cdot x)}{sin(29 \cdot x)}=\lim\limits_{x\to\ 0}\frac{26 \cdot cos(29 \cdot x)}{\frac{sin(29 \cdot x)}{x}}=\lim\limits_{x\to\ 0}\frac{26 \cdot cos(29 \cdot x)}{29 \cdot \frac{sin(29 \cdot x)}{29 \cdot x}} = \frac{26}{29}$$
\rozwStop
\odpStart
$\frac{26}{29}$
\odpStop
\testStart
A.$\frac{26}{29}$
B.$\infty$
C.$-\infty$
D.$0$
E.$-\frac{26}{29}$
F.$\frac{29}{26}$
G.$-\frac{29}{26}$
H.$29$
I.$26$
\testStop
\kluczStart
A
\kluczStop



\zadStart{Przykład z Wikieł P 4.3a moja wersja nr 530}


Obliczyć granicę funkcji $\lim\limits_{x\to\ 0}\frac{26 \cdot x}{tan(31 \cdot x)}$.
\zadStop
\rozwStart{Patryk Wirkus}{}
$$\lim\limits_{x\to\ 0}\frac{26 \cdot x}{tan(31 \cdot x)}=\lim\limits_{x\to\ 0}\frac{26 \cdot x \cdot cos(31 \cdot x)}{sin(31 \cdot x)}=\lim\limits_{x\to\ 0}\frac{26 \cdot cos(31 \cdot x)}{\frac{sin(31 \cdot x)}{x}}=\lim\limits_{x\to\ 0}\frac{26 \cdot cos(31 \cdot x)}{31 \cdot \frac{sin(31 \cdot x)}{31 \cdot x}} = \frac{26}{31}$$
\rozwStop
\odpStart
$\frac{26}{31}$
\odpStop
\testStart
A.$\frac{26}{31}$
B.$\infty$
C.$-\infty$
D.$0$
E.$-\frac{26}{31}$
F.$\frac{31}{26}$
G.$-\frac{31}{26}$
H.$31$
I.$26$
\testStop
\kluczStart
A
\kluczStop



\zadStart{Przykład z Wikieł P 4.3a moja wersja nr 531}


Obliczyć granicę funkcji $\lim\limits_{x\to\ 0}\frac{26 \cdot x}{tan(33 \cdot x)}$.
\zadStop
\rozwStart{Patryk Wirkus}{}
$$\lim\limits_{x\to\ 0}\frac{26 \cdot x}{tan(33 \cdot x)}=\lim\limits_{x\to\ 0}\frac{26 \cdot x \cdot cos(33 \cdot x)}{sin(33 \cdot x)}=\lim\limits_{x\to\ 0}\frac{26 \cdot cos(33 \cdot x)}{\frac{sin(33 \cdot x)}{x}}=\lim\limits_{x\to\ 0}\frac{26 \cdot cos(33 \cdot x)}{33 \cdot \frac{sin(33 \cdot x)}{33 \cdot x}} = \frac{26}{33}$$
\rozwStop
\odpStart
$\frac{26}{33}$
\odpStop
\testStart
A.$\frac{26}{33}$
B.$\infty$
C.$-\infty$
D.$0$
E.$-\frac{26}{33}$
F.$\frac{33}{26}$
G.$-\frac{33}{26}$
H.$33$
I.$26$
\testStop
\kluczStart
A
\kluczStop



\zadStart{Przykład z Wikieł P 4.3a moja wersja nr 532}


Obliczyć granicę funkcji $\lim\limits_{x\to\ 0}\frac{26 \cdot x}{tan(35 \cdot x)}$.
\zadStop
\rozwStart{Patryk Wirkus}{}
$$\lim\limits_{x\to\ 0}\frac{26 \cdot x}{tan(35 \cdot x)}=\lim\limits_{x\to\ 0}\frac{26 \cdot x \cdot cos(35 \cdot x)}{sin(35 \cdot x)}=\lim\limits_{x\to\ 0}\frac{26 \cdot cos(35 \cdot x)}{\frac{sin(35 \cdot x)}{x}}=\lim\limits_{x\to\ 0}\frac{26 \cdot cos(35 \cdot x)}{35 \cdot \frac{sin(35 \cdot x)}{35 \cdot x}} = \frac{26}{35}$$
\rozwStop
\odpStart
$\frac{26}{35}$
\odpStop
\testStart
A.$\frac{26}{35}$
B.$\infty$
C.$-\infty$
D.$0$
E.$-\frac{26}{35}$
F.$\frac{35}{26}$
G.$-\frac{35}{26}$
H.$35$
I.$26$
\testStop
\kluczStart
A
\kluczStop



\zadStart{Przykład z Wikieł P 4.3a moja wersja nr 533}


Obliczyć granicę funkcji $\lim\limits_{x\to\ 0}\frac{26 \cdot x}{tan(37 \cdot x)}$.
\zadStop
\rozwStart{Patryk Wirkus}{}
$$\lim\limits_{x\to\ 0}\frac{26 \cdot x}{tan(37 \cdot x)}=\lim\limits_{x\to\ 0}\frac{26 \cdot x \cdot cos(37 \cdot x)}{sin(37 \cdot x)}=\lim\limits_{x\to\ 0}\frac{26 \cdot cos(37 \cdot x)}{\frac{sin(37 \cdot x)}{x}}=\lim\limits_{x\to\ 0}\frac{26 \cdot cos(37 \cdot x)}{37 \cdot \frac{sin(37 \cdot x)}{37 \cdot x}} = \frac{26}{37}$$
\rozwStop
\odpStart
$\frac{26}{37}$
\odpStop
\testStart
A.$\frac{26}{37}$
B.$\infty$
C.$-\infty$
D.$0$
E.$-\frac{26}{37}$
F.$\frac{37}{26}$
G.$-\frac{37}{26}$
H.$37$
I.$26$
\testStop
\kluczStart
A
\kluczStop



\zadStart{Przykład z Wikieł P 4.3a moja wersja nr 534}


Obliczyć granicę funkcji $\lim\limits_{x\to\ 0}\frac{27 \cdot x}{tan(2 \cdot x)}$.
\zadStop
\rozwStart{Patryk Wirkus}{}
$$\lim\limits_{x\to\ 0}\frac{27 \cdot x}{tan(2 \cdot x)}=\lim\limits_{x\to\ 0}\frac{27 \cdot x \cdot cos(2 \cdot x)}{sin(2 \cdot x)}=\lim\limits_{x\to\ 0}\frac{27 \cdot cos(2 \cdot x)}{\frac{sin(2 \cdot x)}{x}}=\lim\limits_{x\to\ 0}\frac{27 \cdot cos(2 \cdot x)}{2 \cdot \frac{sin(2 \cdot x)}{2 \cdot x}} = \frac{27}{2}$$
\rozwStop
\odpStart
$\frac{27}{2}$
\odpStop
\testStart
A.$\frac{27}{2}$
B.$\infty$
C.$-\infty$
D.$0$
E.$-\frac{27}{2}$
F.$\frac{2}{27}$
G.$-\frac{2}{27}$
H.$2$
I.$27$
\testStop
\kluczStart
A
\kluczStop



\zadStart{Przykład z Wikieł P 4.3a moja wersja nr 535}


Obliczyć granicę funkcji $\lim\limits_{x\to\ 0}\frac{27 \cdot x}{tan(4 \cdot x)}$.
\zadStop
\rozwStart{Patryk Wirkus}{}
$$\lim\limits_{x\to\ 0}\frac{27 \cdot x}{tan(4 \cdot x)}=\lim\limits_{x\to\ 0}\frac{27 \cdot x \cdot cos(4 \cdot x)}{sin(4 \cdot x)}=\lim\limits_{x\to\ 0}\frac{27 \cdot cos(4 \cdot x)}{\frac{sin(4 \cdot x)}{x}}=\lim\limits_{x\to\ 0}\frac{27 \cdot cos(4 \cdot x)}{4 \cdot \frac{sin(4 \cdot x)}{4 \cdot x}} = \frac{27}{4}$$
\rozwStop
\odpStart
$\frac{27}{4}$
\odpStop
\testStart
A.$\frac{27}{4}$
B.$\infty$
C.$-\infty$
D.$0$
E.$-\frac{27}{4}$
F.$\frac{4}{27}$
G.$-\frac{4}{27}$
H.$4$
I.$27$
\testStop
\kluczStart
A
\kluczStop



\zadStart{Przykład z Wikieł P 4.3a moja wersja nr 536}


Obliczyć granicę funkcji $\lim\limits_{x\to\ 0}\frac{27 \cdot x}{tan(5 \cdot x)}$.
\zadStop
\rozwStart{Patryk Wirkus}{}
$$\lim\limits_{x\to\ 0}\frac{27 \cdot x}{tan(5 \cdot x)}=\lim\limits_{x\to\ 0}\frac{27 \cdot x \cdot cos(5 \cdot x)}{sin(5 \cdot x)}=\lim\limits_{x\to\ 0}\frac{27 \cdot cos(5 \cdot x)}{\frac{sin(5 \cdot x)}{x}}=\lim\limits_{x\to\ 0}\frac{27 \cdot cos(5 \cdot x)}{5 \cdot \frac{sin(5 \cdot x)}{5 \cdot x}} = \frac{27}{5}$$
\rozwStop
\odpStart
$\frac{27}{5}$
\odpStop
\testStart
A.$\frac{27}{5}$
B.$\infty$
C.$-\infty$
D.$0$
E.$-\frac{27}{5}$
F.$\frac{5}{27}$
G.$-\frac{5}{27}$
H.$5$
I.$27$
\testStop
\kluczStart
A
\kluczStop



\zadStart{Przykład z Wikieł P 4.3a moja wersja nr 537}


Obliczyć granicę funkcji $\lim\limits_{x\to\ 0}\frac{27 \cdot x}{tan(7 \cdot x)}$.
\zadStop
\rozwStart{Patryk Wirkus}{}
$$\lim\limits_{x\to\ 0}\frac{27 \cdot x}{tan(7 \cdot x)}=\lim\limits_{x\to\ 0}\frac{27 \cdot x \cdot cos(7 \cdot x)}{sin(7 \cdot x)}=\lim\limits_{x\to\ 0}\frac{27 \cdot cos(7 \cdot x)}{\frac{sin(7 \cdot x)}{x}}=\lim\limits_{x\to\ 0}\frac{27 \cdot cos(7 \cdot x)}{7 \cdot \frac{sin(7 \cdot x)}{7 \cdot x}} = \frac{27}{7}$$
\rozwStop
\odpStart
$\frac{27}{7}$
\odpStop
\testStart
A.$\frac{27}{7}$
B.$\infty$
C.$-\infty$
D.$0$
E.$-\frac{27}{7}$
F.$\frac{7}{27}$
G.$-\frac{7}{27}$
H.$7$
I.$27$
\testStop
\kluczStart
A
\kluczStop



\zadStart{Przykład z Wikieł P 4.3a moja wersja nr 538}


Obliczyć granicę funkcji $\lim\limits_{x\to\ 0}\frac{27 \cdot x}{tan(8 \cdot x)}$.
\zadStop
\rozwStart{Patryk Wirkus}{}
$$\lim\limits_{x\to\ 0}\frac{27 \cdot x}{tan(8 \cdot x)}=\lim\limits_{x\to\ 0}\frac{27 \cdot x \cdot cos(8 \cdot x)}{sin(8 \cdot x)}=\lim\limits_{x\to\ 0}\frac{27 \cdot cos(8 \cdot x)}{\frac{sin(8 \cdot x)}{x}}=\lim\limits_{x\to\ 0}\frac{27 \cdot cos(8 \cdot x)}{8 \cdot \frac{sin(8 \cdot x)}{8 \cdot x}} = \frac{27}{8}$$
\rozwStop
\odpStart
$\frac{27}{8}$
\odpStop
\testStart
A.$\frac{27}{8}$
B.$\infty$
C.$-\infty$
D.$0$
E.$-\frac{27}{8}$
F.$\frac{8}{27}$
G.$-\frac{8}{27}$
H.$8$
I.$27$
\testStop
\kluczStart
A
\kluczStop



\zadStart{Przykład z Wikieł P 4.3a moja wersja nr 539}


Obliczyć granicę funkcji $\lim\limits_{x\to\ 0}\frac{27 \cdot x}{tan(10 \cdot x)}$.
\zadStop
\rozwStart{Patryk Wirkus}{}
$$\lim\limits_{x\to\ 0}\frac{27 \cdot x}{tan(10 \cdot x)}=\lim\limits_{x\to\ 0}\frac{27 \cdot x \cdot cos(10 \cdot x)}{sin(10 \cdot x)}=\lim\limits_{x\to\ 0}\frac{27 \cdot cos(10 \cdot x)}{\frac{sin(10 \cdot x)}{x}}=\lim\limits_{x\to\ 0}\frac{27 \cdot cos(10 \cdot x)}{10 \cdot \frac{sin(10 \cdot x)}{10 \cdot x}} = \frac{27}{10}$$
\rozwStop
\odpStart
$\frac{27}{10}$
\odpStop
\testStart
A.$\frac{27}{10}$
B.$\infty$
C.$-\infty$
D.$0$
E.$-\frac{27}{10}$
F.$\frac{10}{27}$
G.$-\frac{10}{27}$
H.$10$
I.$27$
\testStop
\kluczStart
A
\kluczStop



\zadStart{Przykład z Wikieł P 4.3a moja wersja nr 540}


Obliczyć granicę funkcji $\lim\limits_{x\to\ 0}\frac{27 \cdot x}{tan(11 \cdot x)}$.
\zadStop
\rozwStart{Patryk Wirkus}{}
$$\lim\limits_{x\to\ 0}\frac{27 \cdot x}{tan(11 \cdot x)}=\lim\limits_{x\to\ 0}\frac{27 \cdot x \cdot cos(11 \cdot x)}{sin(11 \cdot x)}=\lim\limits_{x\to\ 0}\frac{27 \cdot cos(11 \cdot x)}{\frac{sin(11 \cdot x)}{x}}=\lim\limits_{x\to\ 0}\frac{27 \cdot cos(11 \cdot x)}{11 \cdot \frac{sin(11 \cdot x)}{11 \cdot x}} = \frac{27}{11}$$
\rozwStop
\odpStart
$\frac{27}{11}$
\odpStop
\testStart
A.$\frac{27}{11}$
B.$\infty$
C.$-\infty$
D.$0$
E.$-\frac{27}{11}$
F.$\frac{11}{27}$
G.$-\frac{11}{27}$
H.$11$
I.$27$
\testStop
\kluczStart
A
\kluczStop



\zadStart{Przykład z Wikieł P 4.3a moja wersja nr 541}


Obliczyć granicę funkcji $\lim\limits_{x\to\ 0}\frac{27 \cdot x}{tan(13 \cdot x)}$.
\zadStop
\rozwStart{Patryk Wirkus}{}
$$\lim\limits_{x\to\ 0}\frac{27 \cdot x}{tan(13 \cdot x)}=\lim\limits_{x\to\ 0}\frac{27 \cdot x \cdot cos(13 \cdot x)}{sin(13 \cdot x)}=\lim\limits_{x\to\ 0}\frac{27 \cdot cos(13 \cdot x)}{\frac{sin(13 \cdot x)}{x}}=\lim\limits_{x\to\ 0}\frac{27 \cdot cos(13 \cdot x)}{13 \cdot \frac{sin(13 \cdot x)}{13 \cdot x}} = \frac{27}{13}$$
\rozwStop
\odpStart
$\frac{27}{13}$
\odpStop
\testStart
A.$\frac{27}{13}$
B.$\infty$
C.$-\infty$
D.$0$
E.$-\frac{27}{13}$
F.$\frac{13}{27}$
G.$-\frac{13}{27}$
H.$13$
I.$27$
\testStop
\kluczStart
A
\kluczStop



\zadStart{Przykład z Wikieł P 4.3a moja wersja nr 542}


Obliczyć granicę funkcji $\lim\limits_{x\to\ 0}\frac{27 \cdot x}{tan(14 \cdot x)}$.
\zadStop
\rozwStart{Patryk Wirkus}{}
$$\lim\limits_{x\to\ 0}\frac{27 \cdot x}{tan(14 \cdot x)}=\lim\limits_{x\to\ 0}\frac{27 \cdot x \cdot cos(14 \cdot x)}{sin(14 \cdot x)}=\lim\limits_{x\to\ 0}\frac{27 \cdot cos(14 \cdot x)}{\frac{sin(14 \cdot x)}{x}}=\lim\limits_{x\to\ 0}\frac{27 \cdot cos(14 \cdot x)}{14 \cdot \frac{sin(14 \cdot x)}{14 \cdot x}} = \frac{27}{14}$$
\rozwStop
\odpStart
$\frac{27}{14}$
\odpStop
\testStart
A.$\frac{27}{14}$
B.$\infty$
C.$-\infty$
D.$0$
E.$-\frac{27}{14}$
F.$\frac{14}{27}$
G.$-\frac{14}{27}$
H.$14$
I.$27$
\testStop
\kluczStart
A
\kluczStop



\zadStart{Przykład z Wikieł P 4.3a moja wersja nr 543}


Obliczyć granicę funkcji $\lim\limits_{x\to\ 0}\frac{27 \cdot x}{tan(16 \cdot x)}$.
\zadStop
\rozwStart{Patryk Wirkus}{}
$$\lim\limits_{x\to\ 0}\frac{27 \cdot x}{tan(16 \cdot x)}=\lim\limits_{x\to\ 0}\frac{27 \cdot x \cdot cos(16 \cdot x)}{sin(16 \cdot x)}=\lim\limits_{x\to\ 0}\frac{27 \cdot cos(16 \cdot x)}{\frac{sin(16 \cdot x)}{x}}=\lim\limits_{x\to\ 0}\frac{27 \cdot cos(16 \cdot x)}{16 \cdot \frac{sin(16 \cdot x)}{16 \cdot x}} = \frac{27}{16}$$
\rozwStop
\odpStart
$\frac{27}{16}$
\odpStop
\testStart
A.$\frac{27}{16}$
B.$\infty$
C.$-\infty$
D.$0$
E.$-\frac{27}{16}$
F.$\frac{16}{27}$
G.$-\frac{16}{27}$
H.$16$
I.$27$
\testStop
\kluczStart
A
\kluczStop



\zadStart{Przykład z Wikieł P 4.3a moja wersja nr 544}


Obliczyć granicę funkcji $\lim\limits_{x\to\ 0}\frac{27 \cdot x}{tan(17 \cdot x)}$.
\zadStop
\rozwStart{Patryk Wirkus}{}
$$\lim\limits_{x\to\ 0}\frac{27 \cdot x}{tan(17 \cdot x)}=\lim\limits_{x\to\ 0}\frac{27 \cdot x \cdot cos(17 \cdot x)}{sin(17 \cdot x)}=\lim\limits_{x\to\ 0}\frac{27 \cdot cos(17 \cdot x)}{\frac{sin(17 \cdot x)}{x}}=\lim\limits_{x\to\ 0}\frac{27 \cdot cos(17 \cdot x)}{17 \cdot \frac{sin(17 \cdot x)}{17 \cdot x}} = \frac{27}{17}$$
\rozwStop
\odpStart
$\frac{27}{17}$
\odpStop
\testStart
A.$\frac{27}{17}$
B.$\infty$
C.$-\infty$
D.$0$
E.$-\frac{27}{17}$
F.$\frac{17}{27}$
G.$-\frac{17}{27}$
H.$17$
I.$27$
\testStop
\kluczStart
A
\kluczStop



\zadStart{Przykład z Wikieł P 4.3a moja wersja nr 545}


Obliczyć granicę funkcji $\lim\limits_{x\to\ 0}\frac{27 \cdot x}{tan(19 \cdot x)}$.
\zadStop
\rozwStart{Patryk Wirkus}{}
$$\lim\limits_{x\to\ 0}\frac{27 \cdot x}{tan(19 \cdot x)}=\lim\limits_{x\to\ 0}\frac{27 \cdot x \cdot cos(19 \cdot x)}{sin(19 \cdot x)}=\lim\limits_{x\to\ 0}\frac{27 \cdot cos(19 \cdot x)}{\frac{sin(19 \cdot x)}{x}}=\lim\limits_{x\to\ 0}\frac{27 \cdot cos(19 \cdot x)}{19 \cdot \frac{sin(19 \cdot x)}{19 \cdot x}} = \frac{27}{19}$$
\rozwStop
\odpStart
$\frac{27}{19}$
\odpStop
\testStart
A.$\frac{27}{19}$
B.$\infty$
C.$-\infty$
D.$0$
E.$-\frac{27}{19}$
F.$\frac{19}{27}$
G.$-\frac{19}{27}$
H.$19$
I.$27$
\testStop
\kluczStart
A
\kluczStop



\zadStart{Przykład z Wikieł P 4.3a moja wersja nr 546}


Obliczyć granicę funkcji $\lim\limits_{x\to\ 0}\frac{27 \cdot x}{tan(20 \cdot x)}$.
\zadStop
\rozwStart{Patryk Wirkus}{}
$$\lim\limits_{x\to\ 0}\frac{27 \cdot x}{tan(20 \cdot x)}=\lim\limits_{x\to\ 0}\frac{27 \cdot x \cdot cos(20 \cdot x)}{sin(20 \cdot x)}=\lim\limits_{x\to\ 0}\frac{27 \cdot cos(20 \cdot x)}{\frac{sin(20 \cdot x)}{x}}=\lim\limits_{x\to\ 0}\frac{27 \cdot cos(20 \cdot x)}{20 \cdot \frac{sin(20 \cdot x)}{20 \cdot x}} = \frac{27}{20}$$
\rozwStop
\odpStart
$\frac{27}{20}$
\odpStop
\testStart
A.$\frac{27}{20}$
B.$\infty$
C.$-\infty$
D.$0$
E.$-\frac{27}{20}$
F.$\frac{20}{27}$
G.$-\frac{20}{27}$
H.$20$
I.$27$
\testStop
\kluczStart
A
\kluczStop



\zadStart{Przykład z Wikieł P 4.3a moja wersja nr 547}


Obliczyć granicę funkcji $\lim\limits_{x\to\ 0}\frac{27 \cdot x}{tan(22 \cdot x)}$.
\zadStop
\rozwStart{Patryk Wirkus}{}
$$\lim\limits_{x\to\ 0}\frac{27 \cdot x}{tan(22 \cdot x)}=\lim\limits_{x\to\ 0}\frac{27 \cdot x \cdot cos(22 \cdot x)}{sin(22 \cdot x)}=\lim\limits_{x\to\ 0}\frac{27 \cdot cos(22 \cdot x)}{\frac{sin(22 \cdot x)}{x}}=\lim\limits_{x\to\ 0}\frac{27 \cdot cos(22 \cdot x)}{22 \cdot \frac{sin(22 \cdot x)}{22 \cdot x}} = \frac{27}{22}$$
\rozwStop
\odpStart
$\frac{27}{22}$
\odpStop
\testStart
A.$\frac{27}{22}$
B.$\infty$
C.$-\infty$
D.$0$
E.$-\frac{27}{22}$
F.$\frac{22}{27}$
G.$-\frac{22}{27}$
H.$22$
I.$27$
\testStop
\kluczStart
A
\kluczStop



\zadStart{Przykład z Wikieł P 4.3a moja wersja nr 548}


Obliczyć granicę funkcji $\lim\limits_{x\to\ 0}\frac{27 \cdot x}{tan(23 \cdot x)}$.
\zadStop
\rozwStart{Patryk Wirkus}{}
$$\lim\limits_{x\to\ 0}\frac{27 \cdot x}{tan(23 \cdot x)}=\lim\limits_{x\to\ 0}\frac{27 \cdot x \cdot cos(23 \cdot x)}{sin(23 \cdot x)}=\lim\limits_{x\to\ 0}\frac{27 \cdot cos(23 \cdot x)}{\frac{sin(23 \cdot x)}{x}}=\lim\limits_{x\to\ 0}\frac{27 \cdot cos(23 \cdot x)}{23 \cdot \frac{sin(23 \cdot x)}{23 \cdot x}} = \frac{27}{23}$$
\rozwStop
\odpStart
$\frac{27}{23}$
\odpStop
\testStart
A.$\frac{27}{23}$
B.$\infty$
C.$-\infty$
D.$0$
E.$-\frac{27}{23}$
F.$\frac{23}{27}$
G.$-\frac{23}{27}$
H.$23$
I.$27$
\testStop
\kluczStart
A
\kluczStop



\zadStart{Przykład z Wikieł P 4.3a moja wersja nr 549}


Obliczyć granicę funkcji $\lim\limits_{x\to\ 0}\frac{27 \cdot x}{tan(25 \cdot x)}$.
\zadStop
\rozwStart{Patryk Wirkus}{}
$$\lim\limits_{x\to\ 0}\frac{27 \cdot x}{tan(25 \cdot x)}=\lim\limits_{x\to\ 0}\frac{27 \cdot x \cdot cos(25 \cdot x)}{sin(25 \cdot x)}=\lim\limits_{x\to\ 0}\frac{27 \cdot cos(25 \cdot x)}{\frac{sin(25 \cdot x)}{x}}=\lim\limits_{x\to\ 0}\frac{27 \cdot cos(25 \cdot x)}{25 \cdot \frac{sin(25 \cdot x)}{25 \cdot x}} = \frac{27}{25}$$
\rozwStop
\odpStart
$\frac{27}{25}$
\odpStop
\testStart
A.$\frac{27}{25}$
B.$\infty$
C.$-\infty$
D.$0$
E.$-\frac{27}{25}$
F.$\frac{25}{27}$
G.$-\frac{25}{27}$
H.$25$
I.$27$
\testStop
\kluczStart
A
\kluczStop



\zadStart{Przykład z Wikieł P 4.3a moja wersja nr 550}


Obliczyć granicę funkcji $\lim\limits_{x\to\ 0}\frac{27 \cdot x}{tan(26 \cdot x)}$.
\zadStop
\rozwStart{Patryk Wirkus}{}
$$\lim\limits_{x\to\ 0}\frac{27 \cdot x}{tan(26 \cdot x)}=\lim\limits_{x\to\ 0}\frac{27 \cdot x \cdot cos(26 \cdot x)}{sin(26 \cdot x)}=\lim\limits_{x\to\ 0}\frac{27 \cdot cos(26 \cdot x)}{\frac{sin(26 \cdot x)}{x}}=\lim\limits_{x\to\ 0}\frac{27 \cdot cos(26 \cdot x)}{26 \cdot \frac{sin(26 \cdot x)}{26 \cdot x}} = \frac{27}{26}$$
\rozwStop
\odpStart
$\frac{27}{26}$
\odpStop
\testStart
A.$\frac{27}{26}$
B.$\infty$
C.$-\infty$
D.$0$
E.$-\frac{27}{26}$
F.$\frac{26}{27}$
G.$-\frac{26}{27}$
H.$26$
I.$27$
\testStop
\kluczStart
A
\kluczStop



\zadStart{Przykład z Wikieł P 4.3a moja wersja nr 551}


Obliczyć granicę funkcji $\lim\limits_{x\to\ 0}\frac{27 \cdot x}{tan(28 \cdot x)}$.
\zadStop
\rozwStart{Patryk Wirkus}{}
$$\lim\limits_{x\to\ 0}\frac{27 \cdot x}{tan(28 \cdot x)}=\lim\limits_{x\to\ 0}\frac{27 \cdot x \cdot cos(28 \cdot x)}{sin(28 \cdot x)}=\lim\limits_{x\to\ 0}\frac{27 \cdot cos(28 \cdot x)}{\frac{sin(28 \cdot x)}{x}}=\lim\limits_{x\to\ 0}\frac{27 \cdot cos(28 \cdot x)}{28 \cdot \frac{sin(28 \cdot x)}{28 \cdot x}} = \frac{27}{28}$$
\rozwStop
\odpStart
$\frac{27}{28}$
\odpStop
\testStart
A.$\frac{27}{28}$
B.$\infty$
C.$-\infty$
D.$0$
E.$-\frac{27}{28}$
F.$\frac{28}{27}$
G.$-\frac{28}{27}$
H.$28$
I.$27$
\testStop
\kluczStart
A
\kluczStop



\zadStart{Przykład z Wikieł P 4.3a moja wersja nr 552}


Obliczyć granicę funkcji $\lim\limits_{x\to\ 0}\frac{27 \cdot x}{tan(29 \cdot x)}$.
\zadStop
\rozwStart{Patryk Wirkus}{}
$$\lim\limits_{x\to\ 0}\frac{27 \cdot x}{tan(29 \cdot x)}=\lim\limits_{x\to\ 0}\frac{27 \cdot x \cdot cos(29 \cdot x)}{sin(29 \cdot x)}=\lim\limits_{x\to\ 0}\frac{27 \cdot cos(29 \cdot x)}{\frac{sin(29 \cdot x)}{x}}=\lim\limits_{x\to\ 0}\frac{27 \cdot cos(29 \cdot x)}{29 \cdot \frac{sin(29 \cdot x)}{29 \cdot x}} = \frac{27}{29}$$
\rozwStop
\odpStart
$\frac{27}{29}$
\odpStop
\testStart
A.$\frac{27}{29}$
B.$\infty$
C.$-\infty$
D.$0$
E.$-\frac{27}{29}$
F.$\frac{29}{27}$
G.$-\frac{29}{27}$
H.$29$
I.$27$
\testStop
\kluczStart
A
\kluczStop



\zadStart{Przykład z Wikieł P 4.3a moja wersja nr 553}


Obliczyć granicę funkcji $\lim\limits_{x\to\ 0}\frac{27 \cdot x}{tan(31 \cdot x)}$.
\zadStop
\rozwStart{Patryk Wirkus}{}
$$\lim\limits_{x\to\ 0}\frac{27 \cdot x}{tan(31 \cdot x)}=\lim\limits_{x\to\ 0}\frac{27 \cdot x \cdot cos(31 \cdot x)}{sin(31 \cdot x)}=\lim\limits_{x\to\ 0}\frac{27 \cdot cos(31 \cdot x)}{\frac{sin(31 \cdot x)}{x}}=\lim\limits_{x\to\ 0}\frac{27 \cdot cos(31 \cdot x)}{31 \cdot \frac{sin(31 \cdot x)}{31 \cdot x}} = \frac{27}{31}$$
\rozwStop
\odpStart
$\frac{27}{31}$
\odpStop
\testStart
A.$\frac{27}{31}$
B.$\infty$
C.$-\infty$
D.$0$
E.$-\frac{27}{31}$
F.$\frac{31}{27}$
G.$-\frac{31}{27}$
H.$31$
I.$27$
\testStop
\kluczStart
A
\kluczStop



\zadStart{Przykład z Wikieł P 4.3a moja wersja nr 554}


Obliczyć granicę funkcji $\lim\limits_{x\to\ 0}\frac{27 \cdot x}{tan(32 \cdot x)}$.
\zadStop
\rozwStart{Patryk Wirkus}{}
$$\lim\limits_{x\to\ 0}\frac{27 \cdot x}{tan(32 \cdot x)}=\lim\limits_{x\to\ 0}\frac{27 \cdot x \cdot cos(32 \cdot x)}{sin(32 \cdot x)}=\lim\limits_{x\to\ 0}\frac{27 \cdot cos(32 \cdot x)}{\frac{sin(32 \cdot x)}{x}}=\lim\limits_{x\to\ 0}\frac{27 \cdot cos(32 \cdot x)}{32 \cdot \frac{sin(32 \cdot x)}{32 \cdot x}} = \frac{27}{32}$$
\rozwStop
\odpStart
$\frac{27}{32}$
\odpStop
\testStart
A.$\frac{27}{32}$
B.$\infty$
C.$-\infty$
D.$0$
E.$-\frac{27}{32}$
F.$\frac{32}{27}$
G.$-\frac{32}{27}$
H.$32$
I.$27$
\testStop
\kluczStart
A
\kluczStop



\zadStart{Przykład z Wikieł P 4.3a moja wersja nr 555}


Obliczyć granicę funkcji $\lim\limits_{x\to\ 0}\frac{27 \cdot x}{tan(34 \cdot x)}$.
\zadStop
\rozwStart{Patryk Wirkus}{}
$$\lim\limits_{x\to\ 0}\frac{27 \cdot x}{tan(34 \cdot x)}=\lim\limits_{x\to\ 0}\frac{27 \cdot x \cdot cos(34 \cdot x)}{sin(34 \cdot x)}=\lim\limits_{x\to\ 0}\frac{27 \cdot cos(34 \cdot x)}{\frac{sin(34 \cdot x)}{x}}=\lim\limits_{x\to\ 0}\frac{27 \cdot cos(34 \cdot x)}{34 \cdot \frac{sin(34 \cdot x)}{34 \cdot x}} = \frac{27}{34}$$
\rozwStop
\odpStart
$\frac{27}{34}$
\odpStop
\testStart
A.$\frac{27}{34}$
B.$\infty$
C.$-\infty$
D.$0$
E.$-\frac{27}{34}$
F.$\frac{34}{27}$
G.$-\frac{34}{27}$
H.$34$
I.$27$
\testStop
\kluczStart
A
\kluczStop



\zadStart{Przykład z Wikieł P 4.3a moja wersja nr 556}


Obliczyć granicę funkcji $\lim\limits_{x\to\ 0}\frac{27 \cdot x}{tan(35 \cdot x)}$.
\zadStop
\rozwStart{Patryk Wirkus}{}
$$\lim\limits_{x\to\ 0}\frac{27 \cdot x}{tan(35 \cdot x)}=\lim\limits_{x\to\ 0}\frac{27 \cdot x \cdot cos(35 \cdot x)}{sin(35 \cdot x)}=\lim\limits_{x\to\ 0}\frac{27 \cdot cos(35 \cdot x)}{\frac{sin(35 \cdot x)}{x}}=\lim\limits_{x\to\ 0}\frac{27 \cdot cos(35 \cdot x)}{35 \cdot \frac{sin(35 \cdot x)}{35 \cdot x}} = \frac{27}{35}$$
\rozwStop
\odpStart
$\frac{27}{35}$
\odpStop
\testStart
A.$\frac{27}{35}$
B.$\infty$
C.$-\infty$
D.$0$
E.$-\frac{27}{35}$
F.$\frac{35}{27}$
G.$-\frac{35}{27}$
H.$35$
I.$27$
\testStop
\kluczStart
A
\kluczStop



\zadStart{Przykład z Wikieł P 4.3a moja wersja nr 557}


Obliczyć granicę funkcji $\lim\limits_{x\to\ 0}\frac{27 \cdot x}{tan(37 \cdot x)}$.
\zadStop
\rozwStart{Patryk Wirkus}{}
$$\lim\limits_{x\to\ 0}\frac{27 \cdot x}{tan(37 \cdot x)}=\lim\limits_{x\to\ 0}\frac{27 \cdot x \cdot cos(37 \cdot x)}{sin(37 \cdot x)}=\lim\limits_{x\to\ 0}\frac{27 \cdot cos(37 \cdot x)}{\frac{sin(37 \cdot x)}{x}}=\lim\limits_{x\to\ 0}\frac{27 \cdot cos(37 \cdot x)}{37 \cdot \frac{sin(37 \cdot x)}{37 \cdot x}} = \frac{27}{37}$$
\rozwStop
\odpStart
$\frac{27}{37}$
\odpStop
\testStart
A.$\frac{27}{37}$
B.$\infty$
C.$-\infty$
D.$0$
E.$-\frac{27}{37}$
F.$\frac{37}{27}$
G.$-\frac{37}{27}$
H.$37$
I.$27$
\testStop
\kluczStart
A
\kluczStop



\zadStart{Przykład z Wikieł P 4.3a moja wersja nr 558}


Obliczyć granicę funkcji $\lim\limits_{x\to\ 0}\frac{27 \cdot x}{tan(38 \cdot x)}$.
\zadStop
\rozwStart{Patryk Wirkus}{}
$$\lim\limits_{x\to\ 0}\frac{27 \cdot x}{tan(38 \cdot x)}=\lim\limits_{x\to\ 0}\frac{27 \cdot x \cdot cos(38 \cdot x)}{sin(38 \cdot x)}=\lim\limits_{x\to\ 0}\frac{27 \cdot cos(38 \cdot x)}{\frac{sin(38 \cdot x)}{x}}=\lim\limits_{x\to\ 0}\frac{27 \cdot cos(38 \cdot x)}{38 \cdot \frac{sin(38 \cdot x)}{38 \cdot x}} = \frac{27}{38}$$
\rozwStop
\odpStart
$\frac{27}{38}$
\odpStop
\testStart
A.$\frac{27}{38}$
B.$\infty$
C.$-\infty$
D.$0$
E.$-\frac{27}{38}$
F.$\frac{38}{27}$
G.$-\frac{38}{27}$
H.$38$
I.$27$
\testStop
\kluczStart
A
\kluczStop



\zadStart{Przykład z Wikieł P 4.3a moja wersja nr 559}


Obliczyć granicę funkcji $\lim\limits_{x\to\ 0}\frac{27 \cdot x}{tan(40 \cdot x)}$.
\zadStop
\rozwStart{Patryk Wirkus}{}
$$\lim\limits_{x\to\ 0}\frac{27 \cdot x}{tan(40 \cdot x)}=\lim\limits_{x\to\ 0}\frac{27 \cdot x \cdot cos(40 \cdot x)}{sin(40 \cdot x)}=\lim\limits_{x\to\ 0}\frac{27 \cdot cos(40 \cdot x)}{\frac{sin(40 \cdot x)}{x}}=\lim\limits_{x\to\ 0}\frac{27 \cdot cos(40 \cdot x)}{40 \cdot \frac{sin(40 \cdot x)}{40 \cdot x}} = \frac{27}{40}$$
\rozwStop
\odpStart
$\frac{27}{40}$
\odpStop
\testStart
A.$\frac{27}{40}$
B.$\infty$
C.$-\infty$
D.$0$
E.$-\frac{27}{40}$
F.$\frac{40}{27}$
G.$-\frac{40}{27}$
H.$40$
I.$27$
\testStop
\kluczStart
A
\kluczStop



\zadStart{Przykład z Wikieł P 4.3a moja wersja nr 560}


Obliczyć granicę funkcji $\lim\limits_{x\to\ 0}\frac{28 \cdot x}{tan(3 \cdot x)}$.
\zadStop
\rozwStart{Patryk Wirkus}{}
$$\lim\limits_{x\to\ 0}\frac{28 \cdot x}{tan(3 \cdot x)}=\lim\limits_{x\to\ 0}\frac{28 \cdot x \cdot cos(3 \cdot x)}{sin(3 \cdot x)}=\lim\limits_{x\to\ 0}\frac{28 \cdot cos(3 \cdot x)}{\frac{sin(3 \cdot x)}{x}}=\lim\limits_{x\to\ 0}\frac{28 \cdot cos(3 \cdot x)}{3 \cdot \frac{sin(3 \cdot x)}{3 \cdot x}} = \frac{28}{3}$$
\rozwStop
\odpStart
$\frac{28}{3}$
\odpStop
\testStart
A.$\frac{28}{3}$
B.$\infty$
C.$-\infty$
D.$0$
E.$-\frac{28}{3}$
F.$\frac{3}{28}$
G.$-\frac{3}{28}$
H.$3$
I.$28$
\testStop
\kluczStart
A
\kluczStop



\zadStart{Przykład z Wikieł P 4.3a moja wersja nr 561}


Obliczyć granicę funkcji $\lim\limits_{x\to\ 0}\frac{28 \cdot x}{tan(5 \cdot x)}$.
\zadStop
\rozwStart{Patryk Wirkus}{}
$$\lim\limits_{x\to\ 0}\frac{28 \cdot x}{tan(5 \cdot x)}=\lim\limits_{x\to\ 0}\frac{28 \cdot x \cdot cos(5 \cdot x)}{sin(5 \cdot x)}=\lim\limits_{x\to\ 0}\frac{28 \cdot cos(5 \cdot x)}{\frac{sin(5 \cdot x)}{x}}=\lim\limits_{x\to\ 0}\frac{28 \cdot cos(5 \cdot x)}{5 \cdot \frac{sin(5 \cdot x)}{5 \cdot x}} = \frac{28}{5}$$
\rozwStop
\odpStart
$\frac{28}{5}$
\odpStop
\testStart
A.$\frac{28}{5}$
B.$\infty$
C.$-\infty$
D.$0$
E.$-\frac{28}{5}$
F.$\frac{5}{28}$
G.$-\frac{5}{28}$
H.$5$
I.$28$
\testStop
\kluczStart
A
\kluczStop



\zadStart{Przykład z Wikieł P 4.3a moja wersja nr 562}


Obliczyć granicę funkcji $\lim\limits_{x\to\ 0}\frac{28 \cdot x}{tan(9 \cdot x)}$.
\zadStop
\rozwStart{Patryk Wirkus}{}
$$\lim\limits_{x\to\ 0}\frac{28 \cdot x}{tan(9 \cdot x)}=\lim\limits_{x\to\ 0}\frac{28 \cdot x \cdot cos(9 \cdot x)}{sin(9 \cdot x)}=\lim\limits_{x\to\ 0}\frac{28 \cdot cos(9 \cdot x)}{\frac{sin(9 \cdot x)}{x}}=\lim\limits_{x\to\ 0}\frac{28 \cdot cos(9 \cdot x)}{9 \cdot \frac{sin(9 \cdot x)}{9 \cdot x}} = \frac{28}{9}$$
\rozwStop
\odpStart
$\frac{28}{9}$
\odpStop
\testStart
A.$\frac{28}{9}$
B.$\infty$
C.$-\infty$
D.$0$
E.$-\frac{28}{9}$
F.$\frac{9}{28}$
G.$-\frac{9}{28}$
H.$9$
I.$28$
\testStop
\kluczStart
A
\kluczStop



\zadStart{Przykład z Wikieł P 4.3a moja wersja nr 563}


Obliczyć granicę funkcji $\lim\limits_{x\to\ 0}\frac{28 \cdot x}{tan(11 \cdot x)}$.
\zadStop
\rozwStart{Patryk Wirkus}{}
$$\lim\limits_{x\to\ 0}\frac{28 \cdot x}{tan(11 \cdot x)}=\lim\limits_{x\to\ 0}\frac{28 \cdot x \cdot cos(11 \cdot x)}{sin(11 \cdot x)}=\lim\limits_{x\to\ 0}\frac{28 \cdot cos(11 \cdot x)}{\frac{sin(11 \cdot x)}{x}}=\lim\limits_{x\to\ 0}\frac{28 \cdot cos(11 \cdot x)}{11 \cdot \frac{sin(11 \cdot x)}{11 \cdot x}} = \frac{28}{11}$$
\rozwStop
\odpStart
$\frac{28}{11}$
\odpStop
\testStart
A.$\frac{28}{11}$
B.$\infty$
C.$-\infty$
D.$0$
E.$-\frac{28}{11}$
F.$\frac{11}{28}$
G.$-\frac{11}{28}$
H.$11$
I.$28$
\testStop
\kluczStart
A
\kluczStop



\zadStart{Przykład z Wikieł P 4.3a moja wersja nr 564}


Obliczyć granicę funkcji $\lim\limits_{x\to\ 0}\frac{28 \cdot x}{tan(13 \cdot x)}$.
\zadStop
\rozwStart{Patryk Wirkus}{}
$$\lim\limits_{x\to\ 0}\frac{28 \cdot x}{tan(13 \cdot x)}=\lim\limits_{x\to\ 0}\frac{28 \cdot x \cdot cos(13 \cdot x)}{sin(13 \cdot x)}=\lim\limits_{x\to\ 0}\frac{28 \cdot cos(13 \cdot x)}{\frac{sin(13 \cdot x)}{x}}=\lim\limits_{x\to\ 0}\frac{28 \cdot cos(13 \cdot x)}{13 \cdot \frac{sin(13 \cdot x)}{13 \cdot x}} = \frac{28}{13}$$
\rozwStop
\odpStart
$\frac{28}{13}$
\odpStop
\testStart
A.$\frac{28}{13}$
B.$\infty$
C.$-\infty$
D.$0$
E.$-\frac{28}{13}$
F.$\frac{13}{28}$
G.$-\frac{13}{28}$
H.$13$
I.$28$
\testStop
\kluczStart
A
\kluczStop



\zadStart{Przykład z Wikieł P 4.3a moja wersja nr 565}


Obliczyć granicę funkcji $\lim\limits_{x\to\ 0}\frac{28 \cdot x}{tan(15 \cdot x)}$.
\zadStop
\rozwStart{Patryk Wirkus}{}
$$\lim\limits_{x\to\ 0}\frac{28 \cdot x}{tan(15 \cdot x)}=\lim\limits_{x\to\ 0}\frac{28 \cdot x \cdot cos(15 \cdot x)}{sin(15 \cdot x)}=\lim\limits_{x\to\ 0}\frac{28 \cdot cos(15 \cdot x)}{\frac{sin(15 \cdot x)}{x}}=\lim\limits_{x\to\ 0}\frac{28 \cdot cos(15 \cdot x)}{15 \cdot \frac{sin(15 \cdot x)}{15 \cdot x}} = \frac{28}{15}$$
\rozwStop
\odpStart
$\frac{28}{15}$
\odpStop
\testStart
A.$\frac{28}{15}$
B.$\infty$
C.$-\infty$
D.$0$
E.$-\frac{28}{15}$
F.$\frac{15}{28}$
G.$-\frac{15}{28}$
H.$15$
I.$28$
\testStop
\kluczStart
A
\kluczStop



\zadStart{Przykład z Wikieł P 4.3a moja wersja nr 566}


Obliczyć granicę funkcji $\lim\limits_{x\to\ 0}\frac{28 \cdot x}{tan(17 \cdot x)}$.
\zadStop
\rozwStart{Patryk Wirkus}{}
$$\lim\limits_{x\to\ 0}\frac{28 \cdot x}{tan(17 \cdot x)}=\lim\limits_{x\to\ 0}\frac{28 \cdot x \cdot cos(17 \cdot x)}{sin(17 \cdot x)}=\lim\limits_{x\to\ 0}\frac{28 \cdot cos(17 \cdot x)}{\frac{sin(17 \cdot x)}{x}}=\lim\limits_{x\to\ 0}\frac{28 \cdot cos(17 \cdot x)}{17 \cdot \frac{sin(17 \cdot x)}{17 \cdot x}} = \frac{28}{17}$$
\rozwStop
\odpStart
$\frac{28}{17}$
\odpStop
\testStart
A.$\frac{28}{17}$
B.$\infty$
C.$-\infty$
D.$0$
E.$-\frac{28}{17}$
F.$\frac{17}{28}$
G.$-\frac{17}{28}$
H.$17$
I.$28$
\testStop
\kluczStart
A
\kluczStop



\zadStart{Przykład z Wikieł P 4.3a moja wersja nr 567}


Obliczyć granicę funkcji $\lim\limits_{x\to\ 0}\frac{28 \cdot x}{tan(19 \cdot x)}$.
\zadStop
\rozwStart{Patryk Wirkus}{}
$$\lim\limits_{x\to\ 0}\frac{28 \cdot x}{tan(19 \cdot x)}=\lim\limits_{x\to\ 0}\frac{28 \cdot x \cdot cos(19 \cdot x)}{sin(19 \cdot x)}=\lim\limits_{x\to\ 0}\frac{28 \cdot cos(19 \cdot x)}{\frac{sin(19 \cdot x)}{x}}=\lim\limits_{x\to\ 0}\frac{28 \cdot cos(19 \cdot x)}{19 \cdot \frac{sin(19 \cdot x)}{19 \cdot x}} = \frac{28}{19}$$
\rozwStop
\odpStart
$\frac{28}{19}$
\odpStop
\testStart
A.$\frac{28}{19}$
B.$\infty$
C.$-\infty$
D.$0$
E.$-\frac{28}{19}$
F.$\frac{19}{28}$
G.$-\frac{19}{28}$
H.$19$
I.$28$
\testStop
\kluczStart
A
\kluczStop



\zadStart{Przykład z Wikieł P 4.3a moja wersja nr 568}


Obliczyć granicę funkcji $\lim\limits_{x\to\ 0}\frac{28 \cdot x}{tan(23 \cdot x)}$.
\zadStop
\rozwStart{Patryk Wirkus}{}
$$\lim\limits_{x\to\ 0}\frac{28 \cdot x}{tan(23 \cdot x)}=\lim\limits_{x\to\ 0}\frac{28 \cdot x \cdot cos(23 \cdot x)}{sin(23 \cdot x)}=\lim\limits_{x\to\ 0}\frac{28 \cdot cos(23 \cdot x)}{\frac{sin(23 \cdot x)}{x}}=\lim\limits_{x\to\ 0}\frac{28 \cdot cos(23 \cdot x)}{23 \cdot \frac{sin(23 \cdot x)}{23 \cdot x}} = \frac{28}{23}$$
\rozwStop
\odpStart
$\frac{28}{23}$
\odpStop
\testStart
A.$\frac{28}{23}$
B.$\infty$
C.$-\infty$
D.$0$
E.$-\frac{28}{23}$
F.$\frac{23}{28}$
G.$-\frac{23}{28}$
H.$23$
I.$28$
\testStop
\kluczStart
A
\kluczStop



\zadStart{Przykład z Wikieł P 4.3a moja wersja nr 569}


Obliczyć granicę funkcji $\lim\limits_{x\to\ 0}\frac{28 \cdot x}{tan(25 \cdot x)}$.
\zadStop
\rozwStart{Patryk Wirkus}{}
$$\lim\limits_{x\to\ 0}\frac{28 \cdot x}{tan(25 \cdot x)}=\lim\limits_{x\to\ 0}\frac{28 \cdot x \cdot cos(25 \cdot x)}{sin(25 \cdot x)}=\lim\limits_{x\to\ 0}\frac{28 \cdot cos(25 \cdot x)}{\frac{sin(25 \cdot x)}{x}}=\lim\limits_{x\to\ 0}\frac{28 \cdot cos(25 \cdot x)}{25 \cdot \frac{sin(25 \cdot x)}{25 \cdot x}} = \frac{28}{25}$$
\rozwStop
\odpStart
$\frac{28}{25}$
\odpStop
\testStart
A.$\frac{28}{25}$
B.$\infty$
C.$-\infty$
D.$0$
E.$-\frac{28}{25}$
F.$\frac{25}{28}$
G.$-\frac{25}{28}$
H.$25$
I.$28$
\testStop
\kluczStart
A
\kluczStop



\zadStart{Przykład z Wikieł P 4.3a moja wersja nr 570}


Obliczyć granicę funkcji $\lim\limits_{x\to\ 0}\frac{28 \cdot x}{tan(27 \cdot x)}$.
\zadStop
\rozwStart{Patryk Wirkus}{}
$$\lim\limits_{x\to\ 0}\frac{28 \cdot x}{tan(27 \cdot x)}=\lim\limits_{x\to\ 0}\frac{28 \cdot x \cdot cos(27 \cdot x)}{sin(27 \cdot x)}=\lim\limits_{x\to\ 0}\frac{28 \cdot cos(27 \cdot x)}{\frac{sin(27 \cdot x)}{x}}=\lim\limits_{x\to\ 0}\frac{28 \cdot cos(27 \cdot x)}{27 \cdot \frac{sin(27 \cdot x)}{27 \cdot x}} = \frac{28}{27}$$
\rozwStop
\odpStart
$\frac{28}{27}$
\odpStop
\testStart
A.$\frac{28}{27}$
B.$\infty$
C.$-\infty$
D.$0$
E.$-\frac{28}{27}$
F.$\frac{27}{28}$
G.$-\frac{27}{28}$
H.$27$
I.$28$
\testStop
\kluczStart
A
\kluczStop



\zadStart{Przykład z Wikieł P 4.3a moja wersja nr 571}


Obliczyć granicę funkcji $\lim\limits_{x\to\ 0}\frac{28 \cdot x}{tan(29 \cdot x)}$.
\zadStop
\rozwStart{Patryk Wirkus}{}
$$\lim\limits_{x\to\ 0}\frac{28 \cdot x}{tan(29 \cdot x)}=\lim\limits_{x\to\ 0}\frac{28 \cdot x \cdot cos(29 \cdot x)}{sin(29 \cdot x)}=\lim\limits_{x\to\ 0}\frac{28 \cdot cos(29 \cdot x)}{\frac{sin(29 \cdot x)}{x}}=\lim\limits_{x\to\ 0}\frac{28 \cdot cos(29 \cdot x)}{29 \cdot \frac{sin(29 \cdot x)}{29 \cdot x}} = \frac{28}{29}$$
\rozwStop
\odpStart
$\frac{28}{29}$
\odpStop
\testStart
A.$\frac{28}{29}$
B.$\infty$
C.$-\infty$
D.$0$
E.$-\frac{28}{29}$
F.$\frac{29}{28}$
G.$-\frac{29}{28}$
H.$29$
I.$28$
\testStop
\kluczStart
A
\kluczStop



\zadStart{Przykład z Wikieł P 4.3a moja wersja nr 572}


Obliczyć granicę funkcji $\lim\limits_{x\to\ 0}\frac{28 \cdot x}{tan(31 \cdot x)}$.
\zadStop
\rozwStart{Patryk Wirkus}{}
$$\lim\limits_{x\to\ 0}\frac{28 \cdot x}{tan(31 \cdot x)}=\lim\limits_{x\to\ 0}\frac{28 \cdot x \cdot cos(31 \cdot x)}{sin(31 \cdot x)}=\lim\limits_{x\to\ 0}\frac{28 \cdot cos(31 \cdot x)}{\frac{sin(31 \cdot x)}{x}}=\lim\limits_{x\to\ 0}\frac{28 \cdot cos(31 \cdot x)}{31 \cdot \frac{sin(31 \cdot x)}{31 \cdot x}} = \frac{28}{31}$$
\rozwStop
\odpStart
$\frac{28}{31}$
\odpStop
\testStart
A.$\frac{28}{31}$
B.$\infty$
C.$-\infty$
D.$0$
E.$-\frac{28}{31}$
F.$\frac{31}{28}$
G.$-\frac{31}{28}$
H.$31$
I.$28$
\testStop
\kluczStart
A
\kluczStop



\zadStart{Przykład z Wikieł P 4.3a moja wersja nr 573}


Obliczyć granicę funkcji $\lim\limits_{x\to\ 0}\frac{28 \cdot x}{tan(33 \cdot x)}$.
\zadStop
\rozwStart{Patryk Wirkus}{}
$$\lim\limits_{x\to\ 0}\frac{28 \cdot x}{tan(33 \cdot x)}=\lim\limits_{x\to\ 0}\frac{28 \cdot x \cdot cos(33 \cdot x)}{sin(33 \cdot x)}=\lim\limits_{x\to\ 0}\frac{28 \cdot cos(33 \cdot x)}{\frac{sin(33 \cdot x)}{x}}=\lim\limits_{x\to\ 0}\frac{28 \cdot cos(33 \cdot x)}{33 \cdot \frac{sin(33 \cdot x)}{33 \cdot x}} = \frac{28}{33}$$
\rozwStop
\odpStart
$\frac{28}{33}$
\odpStop
\testStart
A.$\frac{28}{33}$
B.$\infty$
C.$-\infty$
D.$0$
E.$-\frac{28}{33}$
F.$\frac{33}{28}$
G.$-\frac{33}{28}$
H.$33$
I.$28$
\testStop
\kluczStart
A
\kluczStop



\zadStart{Przykład z Wikieł P 4.3a moja wersja nr 574}


Obliczyć granicę funkcji $\lim\limits_{x\to\ 0}\frac{28 \cdot x}{tan(37 \cdot x)}$.
\zadStop
\rozwStart{Patryk Wirkus}{}
$$\lim\limits_{x\to\ 0}\frac{28 \cdot x}{tan(37 \cdot x)}=\lim\limits_{x\to\ 0}\frac{28 \cdot x \cdot cos(37 \cdot x)}{sin(37 \cdot x)}=\lim\limits_{x\to\ 0}\frac{28 \cdot cos(37 \cdot x)}{\frac{sin(37 \cdot x)}{x}}=\lim\limits_{x\to\ 0}\frac{28 \cdot cos(37 \cdot x)}{37 \cdot \frac{sin(37 \cdot x)}{37 \cdot x}} = \frac{28}{37}$$
\rozwStop
\odpStart
$\frac{28}{37}$
\odpStop
\testStart
A.$\frac{28}{37}$
B.$\infty$
C.$-\infty$
D.$0$
E.$-\frac{28}{37}$
F.$\frac{37}{28}$
G.$-\frac{37}{28}$
H.$37$
I.$28$
\testStop
\kluczStart
A
\kluczStop



\zadStart{Przykład z Wikieł P 4.3a moja wersja nr 575}


Obliczyć granicę funkcji $\lim\limits_{x\to\ 0}\frac{28 \cdot x}{tan(39 \cdot x)}$.
\zadStop
\rozwStart{Patryk Wirkus}{}
$$\lim\limits_{x\to\ 0}\frac{28 \cdot x}{tan(39 \cdot x)}=\lim\limits_{x\to\ 0}\frac{28 \cdot x \cdot cos(39 \cdot x)}{sin(39 \cdot x)}=\lim\limits_{x\to\ 0}\frac{28 \cdot cos(39 \cdot x)}{\frac{sin(39 \cdot x)}{x}}=\lim\limits_{x\to\ 0}\frac{28 \cdot cos(39 \cdot x)}{39 \cdot \frac{sin(39 \cdot x)}{39 \cdot x}} = \frac{28}{39}$$
\rozwStop
\odpStart
$\frac{28}{39}$
\odpStop
\testStart
A.$\frac{28}{39}$
B.$\infty$
C.$-\infty$
D.$0$
E.$-\frac{28}{39}$
F.$\frac{39}{28}$
G.$-\frac{39}{28}$
H.$39$
I.$28$
\testStop
\kluczStart
A
\kluczStop



\zadStart{Przykład z Wikieł P 4.3a moja wersja nr 576}


Obliczyć granicę funkcji $\lim\limits_{x\to\ 0}\frac{29 \cdot x}{tan(2 \cdot x)}$.
\zadStop
\rozwStart{Patryk Wirkus}{}
$$\lim\limits_{x\to\ 0}\frac{29 \cdot x}{tan(2 \cdot x)}=\lim\limits_{x\to\ 0}\frac{29 \cdot x \cdot cos(2 \cdot x)}{sin(2 \cdot x)}=\lim\limits_{x\to\ 0}\frac{29 \cdot cos(2 \cdot x)}{\frac{sin(2 \cdot x)}{x}}=\lim\limits_{x\to\ 0}\frac{29 \cdot cos(2 \cdot x)}{2 \cdot \frac{sin(2 \cdot x)}{2 \cdot x}} = \frac{29}{2}$$
\rozwStop
\odpStart
$\frac{29}{2}$
\odpStop
\testStart
A.$\frac{29}{2}$
B.$\infty$
C.$-\infty$
D.$0$
E.$-\frac{29}{2}$
F.$\frac{2}{29}$
G.$-\frac{2}{29}$
H.$2$
I.$29$
\testStop
\kluczStart
A
\kluczStop



\zadStart{Przykład z Wikieł P 4.3a moja wersja nr 577}


Obliczyć granicę funkcji $\lim\limits_{x\to\ 0}\frac{29 \cdot x}{tan(3 \cdot x)}$.
\zadStop
\rozwStart{Patryk Wirkus}{}
$$\lim\limits_{x\to\ 0}\frac{29 \cdot x}{tan(3 \cdot x)}=\lim\limits_{x\to\ 0}\frac{29 \cdot x \cdot cos(3 \cdot x)}{sin(3 \cdot x)}=\lim\limits_{x\to\ 0}\frac{29 \cdot cos(3 \cdot x)}{\frac{sin(3 \cdot x)}{x}}=\lim\limits_{x\to\ 0}\frac{29 \cdot cos(3 \cdot x)}{3 \cdot \frac{sin(3 \cdot x)}{3 \cdot x}} = \frac{29}{3}$$
\rozwStop
\odpStart
$\frac{29}{3}$
\odpStop
\testStart
A.$\frac{29}{3}$
B.$\infty$
C.$-\infty$
D.$0$
E.$-\frac{29}{3}$
F.$\frac{3}{29}$
G.$-\frac{3}{29}$
H.$3$
I.$29$
\testStop
\kluczStart
A
\kluczStop



\zadStart{Przykład z Wikieł P 4.3a moja wersja nr 578}


Obliczyć granicę funkcji $\lim\limits_{x\to\ 0}\frac{29 \cdot x}{tan(4 \cdot x)}$.
\zadStop
\rozwStart{Patryk Wirkus}{}
$$\lim\limits_{x\to\ 0}\frac{29 \cdot x}{tan(4 \cdot x)}=\lim\limits_{x\to\ 0}\frac{29 \cdot x \cdot cos(4 \cdot x)}{sin(4 \cdot x)}=\lim\limits_{x\to\ 0}\frac{29 \cdot cos(4 \cdot x)}{\frac{sin(4 \cdot x)}{x}}=\lim\limits_{x\to\ 0}\frac{29 \cdot cos(4 \cdot x)}{4 \cdot \frac{sin(4 \cdot x)}{4 \cdot x}} = \frac{29}{4}$$
\rozwStop
\odpStart
$\frac{29}{4}$
\odpStop
\testStart
A.$\frac{29}{4}$
B.$\infty$
C.$-\infty$
D.$0$
E.$-\frac{29}{4}$
F.$\frac{4}{29}$
G.$-\frac{4}{29}$
H.$4$
I.$29$
\testStop
\kluczStart
A
\kluczStop



\zadStart{Przykład z Wikieł P 4.3a moja wersja nr 579}


Obliczyć granicę funkcji $\lim\limits_{x\to\ 0}\frac{29 \cdot x}{tan(5 \cdot x)}$.
\zadStop
\rozwStart{Patryk Wirkus}{}
$$\lim\limits_{x\to\ 0}\frac{29 \cdot x}{tan(5 \cdot x)}=\lim\limits_{x\to\ 0}\frac{29 \cdot x \cdot cos(5 \cdot x)}{sin(5 \cdot x)}=\lim\limits_{x\to\ 0}\frac{29 \cdot cos(5 \cdot x)}{\frac{sin(5 \cdot x)}{x}}=\lim\limits_{x\to\ 0}\frac{29 \cdot cos(5 \cdot x)}{5 \cdot \frac{sin(5 \cdot x)}{5 \cdot x}} = \frac{29}{5}$$
\rozwStop
\odpStart
$\frac{29}{5}$
\odpStop
\testStart
A.$\frac{29}{5}$
B.$\infty$
C.$-\infty$
D.$0$
E.$-\frac{29}{5}$
F.$\frac{5}{29}$
G.$-\frac{5}{29}$
H.$5$
I.$29$
\testStop
\kluczStart
A
\kluczStop



\zadStart{Przykład z Wikieł P 4.3a moja wersja nr 580}


Obliczyć granicę funkcji $\lim\limits_{x\to\ 0}\frac{29 \cdot x}{tan(6 \cdot x)}$.
\zadStop
\rozwStart{Patryk Wirkus}{}
$$\lim\limits_{x\to\ 0}\frac{29 \cdot x}{tan(6 \cdot x)}=\lim\limits_{x\to\ 0}\frac{29 \cdot x \cdot cos(6 \cdot x)}{sin(6 \cdot x)}=\lim\limits_{x\to\ 0}\frac{29 \cdot cos(6 \cdot x)}{\frac{sin(6 \cdot x)}{x}}=\lim\limits_{x\to\ 0}\frac{29 \cdot cos(6 \cdot x)}{6 \cdot \frac{sin(6 \cdot x)}{6 \cdot x}} = \frac{29}{6}$$
\rozwStop
\odpStart
$\frac{29}{6}$
\odpStop
\testStart
A.$\frac{29}{6}$
B.$\infty$
C.$-\infty$
D.$0$
E.$-\frac{29}{6}$
F.$\frac{6}{29}$
G.$-\frac{6}{29}$
H.$6$
I.$29$
\testStop
\kluczStart
A
\kluczStop



\zadStart{Przykład z Wikieł P 4.3a moja wersja nr 581}


Obliczyć granicę funkcji $\lim\limits_{x\to\ 0}\frac{29 \cdot x}{tan(7 \cdot x)}$.
\zadStop
\rozwStart{Patryk Wirkus}{}
$$\lim\limits_{x\to\ 0}\frac{29 \cdot x}{tan(7 \cdot x)}=\lim\limits_{x\to\ 0}\frac{29 \cdot x \cdot cos(7 \cdot x)}{sin(7 \cdot x)}=\lim\limits_{x\to\ 0}\frac{29 \cdot cos(7 \cdot x)}{\frac{sin(7 \cdot x)}{x}}=\lim\limits_{x\to\ 0}\frac{29 \cdot cos(7 \cdot x)}{7 \cdot \frac{sin(7 \cdot x)}{7 \cdot x}} = \frac{29}{7}$$
\rozwStop
\odpStart
$\frac{29}{7}$
\odpStop
\testStart
A.$\frac{29}{7}$
B.$\infty$
C.$-\infty$
D.$0$
E.$-\frac{29}{7}$
F.$\frac{7}{29}$
G.$-\frac{7}{29}$
H.$7$
I.$29$
\testStop
\kluczStart
A
\kluczStop



\zadStart{Przykład z Wikieł P 4.3a moja wersja nr 582}


Obliczyć granicę funkcji $\lim\limits_{x\to\ 0}\frac{29 \cdot x}{tan(8 \cdot x)}$.
\zadStop
\rozwStart{Patryk Wirkus}{}
$$\lim\limits_{x\to\ 0}\frac{29 \cdot x}{tan(8 \cdot x)}=\lim\limits_{x\to\ 0}\frac{29 \cdot x \cdot cos(8 \cdot x)}{sin(8 \cdot x)}=\lim\limits_{x\to\ 0}\frac{29 \cdot cos(8 \cdot x)}{\frac{sin(8 \cdot x)}{x}}=\lim\limits_{x\to\ 0}\frac{29 \cdot cos(8 \cdot x)}{8 \cdot \frac{sin(8 \cdot x)}{8 \cdot x}} = \frac{29}{8}$$
\rozwStop
\odpStart
$\frac{29}{8}$
\odpStop
\testStart
A.$\frac{29}{8}$
B.$\infty$
C.$-\infty$
D.$0$
E.$-\frac{29}{8}$
F.$\frac{8}{29}$
G.$-\frac{8}{29}$
H.$8$
I.$29$
\testStop
\kluczStart
A
\kluczStop



\zadStart{Przykład z Wikieł P 4.3a moja wersja nr 583}


Obliczyć granicę funkcji $\lim\limits_{x\to\ 0}\frac{29 \cdot x}{tan(9 \cdot x)}$.
\zadStop
\rozwStart{Patryk Wirkus}{}
$$\lim\limits_{x\to\ 0}\frac{29 \cdot x}{tan(9 \cdot x)}=\lim\limits_{x\to\ 0}\frac{29 \cdot x \cdot cos(9 \cdot x)}{sin(9 \cdot x)}=\lim\limits_{x\to\ 0}\frac{29 \cdot cos(9 \cdot x)}{\frac{sin(9 \cdot x)}{x}}=\lim\limits_{x\to\ 0}\frac{29 \cdot cos(9 \cdot x)}{9 \cdot \frac{sin(9 \cdot x)}{9 \cdot x}} = \frac{29}{9}$$
\rozwStop
\odpStart
$\frac{29}{9}$
\odpStop
\testStart
A.$\frac{29}{9}$
B.$\infty$
C.$-\infty$
D.$0$
E.$-\frac{29}{9}$
F.$\frac{9}{29}$
G.$-\frac{9}{29}$
H.$9$
I.$29$
\testStop
\kluczStart
A
\kluczStop



\zadStart{Przykład z Wikieł P 4.3a moja wersja nr 584}


Obliczyć granicę funkcji $\lim\limits_{x\to\ 0}\frac{29 \cdot x}{tan(10 \cdot x)}$.
\zadStop
\rozwStart{Patryk Wirkus}{}
$$\lim\limits_{x\to\ 0}\frac{29 \cdot x}{tan(10 \cdot x)}=\lim\limits_{x\to\ 0}\frac{29 \cdot x \cdot cos(10 \cdot x)}{sin(10 \cdot x)}=\lim\limits_{x\to\ 0}\frac{29 \cdot cos(10 \cdot x)}{\frac{sin(10 \cdot x)}{x}}=\lim\limits_{x\to\ 0}\frac{29 \cdot cos(10 \cdot x)}{10 \cdot \frac{sin(10 \cdot x)}{10 \cdot x}} = \frac{29}{10}$$
\rozwStop
\odpStart
$\frac{29}{10}$
\odpStop
\testStart
A.$\frac{29}{10}$
B.$\infty$
C.$-\infty$
D.$0$
E.$-\frac{29}{10}$
F.$\frac{10}{29}$
G.$-\frac{10}{29}$
H.$10$
I.$29$
\testStop
\kluczStart
A
\kluczStop



\zadStart{Przykład z Wikieł P 4.3a moja wersja nr 585}


Obliczyć granicę funkcji $\lim\limits_{x\to\ 0}\frac{29 \cdot x}{tan(11 \cdot x)}$.
\zadStop
\rozwStart{Patryk Wirkus}{}
$$\lim\limits_{x\to\ 0}\frac{29 \cdot x}{tan(11 \cdot x)}=\lim\limits_{x\to\ 0}\frac{29 \cdot x \cdot cos(11 \cdot x)}{sin(11 \cdot x)}=\lim\limits_{x\to\ 0}\frac{29 \cdot cos(11 \cdot x)}{\frac{sin(11 \cdot x)}{x}}=\lim\limits_{x\to\ 0}\frac{29 \cdot cos(11 \cdot x)}{11 \cdot \frac{sin(11 \cdot x)}{11 \cdot x}} = \frac{29}{11}$$
\rozwStop
\odpStart
$\frac{29}{11}$
\odpStop
\testStart
A.$\frac{29}{11}$
B.$\infty$
C.$-\infty$
D.$0$
E.$-\frac{29}{11}$
F.$\frac{11}{29}$
G.$-\frac{11}{29}$
H.$11$
I.$29$
\testStop
\kluczStart
A
\kluczStop



\zadStart{Przykład z Wikieł P 4.3a moja wersja nr 586}


Obliczyć granicę funkcji $\lim\limits_{x\to\ 0}\frac{29 \cdot x}{tan(12 \cdot x)}$.
\zadStop
\rozwStart{Patryk Wirkus}{}
$$\lim\limits_{x\to\ 0}\frac{29 \cdot x}{tan(12 \cdot x)}=\lim\limits_{x\to\ 0}\frac{29 \cdot x \cdot cos(12 \cdot x)}{sin(12 \cdot x)}=\lim\limits_{x\to\ 0}\frac{29 \cdot cos(12 \cdot x)}{\frac{sin(12 \cdot x)}{x}}=\lim\limits_{x\to\ 0}\frac{29 \cdot cos(12 \cdot x)}{12 \cdot \frac{sin(12 \cdot x)}{12 \cdot x}} = \frac{29}{12}$$
\rozwStop
\odpStart
$\frac{29}{12}$
\odpStop
\testStart
A.$\frac{29}{12}$
B.$\infty$
C.$-\infty$
D.$0$
E.$-\frac{29}{12}$
F.$\frac{12}{29}$
G.$-\frac{12}{29}$
H.$12$
I.$29$
\testStop
\kluczStart
A
\kluczStop



\zadStart{Przykład z Wikieł P 4.3a moja wersja nr 587}


Obliczyć granicę funkcji $\lim\limits_{x\to\ 0}\frac{29 \cdot x}{tan(13 \cdot x)}$.
\zadStop
\rozwStart{Patryk Wirkus}{}
$$\lim\limits_{x\to\ 0}\frac{29 \cdot x}{tan(13 \cdot x)}=\lim\limits_{x\to\ 0}\frac{29 \cdot x \cdot cos(13 \cdot x)}{sin(13 \cdot x)}=\lim\limits_{x\to\ 0}\frac{29 \cdot cos(13 \cdot x)}{\frac{sin(13 \cdot x)}{x}}=\lim\limits_{x\to\ 0}\frac{29 \cdot cos(13 \cdot x)}{13 \cdot \frac{sin(13 \cdot x)}{13 \cdot x}} = \frac{29}{13}$$
\rozwStop
\odpStart
$\frac{29}{13}$
\odpStop
\testStart
A.$\frac{29}{13}$
B.$\infty$
C.$-\infty$
D.$0$
E.$-\frac{29}{13}$
F.$\frac{13}{29}$
G.$-\frac{13}{29}$
H.$13$
I.$29$
\testStop
\kluczStart
A
\kluczStop



\zadStart{Przykład z Wikieł P 4.3a moja wersja nr 588}


Obliczyć granicę funkcji $\lim\limits_{x\to\ 0}\frac{29 \cdot x}{tan(14 \cdot x)}$.
\zadStop
\rozwStart{Patryk Wirkus}{}
$$\lim\limits_{x\to\ 0}\frac{29 \cdot x}{tan(14 \cdot x)}=\lim\limits_{x\to\ 0}\frac{29 \cdot x \cdot cos(14 \cdot x)}{sin(14 \cdot x)}=\lim\limits_{x\to\ 0}\frac{29 \cdot cos(14 \cdot x)}{\frac{sin(14 \cdot x)}{x}}=\lim\limits_{x\to\ 0}\frac{29 \cdot cos(14 \cdot x)}{14 \cdot \frac{sin(14 \cdot x)}{14 \cdot x}} = \frac{29}{14}$$
\rozwStop
\odpStart
$\frac{29}{14}$
\odpStop
\testStart
A.$\frac{29}{14}$
B.$\infty$
C.$-\infty$
D.$0$
E.$-\frac{29}{14}$
F.$\frac{14}{29}$
G.$-\frac{14}{29}$
H.$14$
I.$29$
\testStop
\kluczStart
A
\kluczStop



\zadStart{Przykład z Wikieł P 4.3a moja wersja nr 589}


Obliczyć granicę funkcji $\lim\limits_{x\to\ 0}\frac{29 \cdot x}{tan(15 \cdot x)}$.
\zadStop
\rozwStart{Patryk Wirkus}{}
$$\lim\limits_{x\to\ 0}\frac{29 \cdot x}{tan(15 \cdot x)}=\lim\limits_{x\to\ 0}\frac{29 \cdot x \cdot cos(15 \cdot x)}{sin(15 \cdot x)}=\lim\limits_{x\to\ 0}\frac{29 \cdot cos(15 \cdot x)}{\frac{sin(15 \cdot x)}{x}}=\lim\limits_{x\to\ 0}\frac{29 \cdot cos(15 \cdot x)}{15 \cdot \frac{sin(15 \cdot x)}{15 \cdot x}} = \frac{29}{15}$$
\rozwStop
\odpStart
$\frac{29}{15}$
\odpStop
\testStart
A.$\frac{29}{15}$
B.$\infty$
C.$-\infty$
D.$0$
E.$-\frac{29}{15}$
F.$\frac{15}{29}$
G.$-\frac{15}{29}$
H.$15$
I.$29$
\testStop
\kluczStart
A
\kluczStop



\zadStart{Przykład z Wikieł P 4.3a moja wersja nr 590}


Obliczyć granicę funkcji $\lim\limits_{x\to\ 0}\frac{29 \cdot x}{tan(16 \cdot x)}$.
\zadStop
\rozwStart{Patryk Wirkus}{}
$$\lim\limits_{x\to\ 0}\frac{29 \cdot x}{tan(16 \cdot x)}=\lim\limits_{x\to\ 0}\frac{29 \cdot x \cdot cos(16 \cdot x)}{sin(16 \cdot x)}=\lim\limits_{x\to\ 0}\frac{29 \cdot cos(16 \cdot x)}{\frac{sin(16 \cdot x)}{x}}=\lim\limits_{x\to\ 0}\frac{29 \cdot cos(16 \cdot x)}{16 \cdot \frac{sin(16 \cdot x)}{16 \cdot x}} = \frac{29}{16}$$
\rozwStop
\odpStart
$\frac{29}{16}$
\odpStop
\testStart
A.$\frac{29}{16}$
B.$\infty$
C.$-\infty$
D.$0$
E.$-\frac{29}{16}$
F.$\frac{16}{29}$
G.$-\frac{16}{29}$
H.$16$
I.$29$
\testStop
\kluczStart
A
\kluczStop



\zadStart{Przykład z Wikieł P 4.3a moja wersja nr 591}


Obliczyć granicę funkcji $\lim\limits_{x\to\ 0}\frac{29 \cdot x}{tan(17 \cdot x)}$.
\zadStop
\rozwStart{Patryk Wirkus}{}
$$\lim\limits_{x\to\ 0}\frac{29 \cdot x}{tan(17 \cdot x)}=\lim\limits_{x\to\ 0}\frac{29 \cdot x \cdot cos(17 \cdot x)}{sin(17 \cdot x)}=\lim\limits_{x\to\ 0}\frac{29 \cdot cos(17 \cdot x)}{\frac{sin(17 \cdot x)}{x}}=\lim\limits_{x\to\ 0}\frac{29 \cdot cos(17 \cdot x)}{17 \cdot \frac{sin(17 \cdot x)}{17 \cdot x}} = \frac{29}{17}$$
\rozwStop
\odpStart
$\frac{29}{17}$
\odpStop
\testStart
A.$\frac{29}{17}$
B.$\infty$
C.$-\infty$
D.$0$
E.$-\frac{29}{17}$
F.$\frac{17}{29}$
G.$-\frac{17}{29}$
H.$17$
I.$29$
\testStop
\kluczStart
A
\kluczStop



\zadStart{Przykład z Wikieł P 4.3a moja wersja nr 592}


Obliczyć granicę funkcji $\lim\limits_{x\to\ 0}\frac{29 \cdot x}{tan(18 \cdot x)}$.
\zadStop
\rozwStart{Patryk Wirkus}{}
$$\lim\limits_{x\to\ 0}\frac{29 \cdot x}{tan(18 \cdot x)}=\lim\limits_{x\to\ 0}\frac{29 \cdot x \cdot cos(18 \cdot x)}{sin(18 \cdot x)}=\lim\limits_{x\to\ 0}\frac{29 \cdot cos(18 \cdot x)}{\frac{sin(18 \cdot x)}{x}}=\lim\limits_{x\to\ 0}\frac{29 \cdot cos(18 \cdot x)}{18 \cdot \frac{sin(18 \cdot x)}{18 \cdot x}} = \frac{29}{18}$$
\rozwStop
\odpStart
$\frac{29}{18}$
\odpStop
\testStart
A.$\frac{29}{18}$
B.$\infty$
C.$-\infty$
D.$0$
E.$-\frac{29}{18}$
F.$\frac{18}{29}$
G.$-\frac{18}{29}$
H.$18$
I.$29$
\testStop
\kluczStart
A
\kluczStop



\zadStart{Przykład z Wikieł P 4.3a moja wersja nr 593}


Obliczyć granicę funkcji $\lim\limits_{x\to\ 0}\frac{29 \cdot x}{tan(19 \cdot x)}$.
\zadStop
\rozwStart{Patryk Wirkus}{}
$$\lim\limits_{x\to\ 0}\frac{29 \cdot x}{tan(19 \cdot x)}=\lim\limits_{x\to\ 0}\frac{29 \cdot x \cdot cos(19 \cdot x)}{sin(19 \cdot x)}=\lim\limits_{x\to\ 0}\frac{29 \cdot cos(19 \cdot x)}{\frac{sin(19 \cdot x)}{x}}=\lim\limits_{x\to\ 0}\frac{29 \cdot cos(19 \cdot x)}{19 \cdot \frac{sin(19 \cdot x)}{19 \cdot x}} = \frac{29}{19}$$
\rozwStop
\odpStart
$\frac{29}{19}$
\odpStop
\testStart
A.$\frac{29}{19}$
B.$\infty$
C.$-\infty$
D.$0$
E.$-\frac{29}{19}$
F.$\frac{19}{29}$
G.$-\frac{19}{29}$
H.$19$
I.$29$
\testStop
\kluczStart
A
\kluczStop



\zadStart{Przykład z Wikieł P 4.3a moja wersja nr 594}


Obliczyć granicę funkcji $\lim\limits_{x\to\ 0}\frac{29 \cdot x}{tan(20 \cdot x)}$.
\zadStop
\rozwStart{Patryk Wirkus}{}
$$\lim\limits_{x\to\ 0}\frac{29 \cdot x}{tan(20 \cdot x)}=\lim\limits_{x\to\ 0}\frac{29 \cdot x \cdot cos(20 \cdot x)}{sin(20 \cdot x)}=\lim\limits_{x\to\ 0}\frac{29 \cdot cos(20 \cdot x)}{\frac{sin(20 \cdot x)}{x}}=\lim\limits_{x\to\ 0}\frac{29 \cdot cos(20 \cdot x)}{20 \cdot \frac{sin(20 \cdot x)}{20 \cdot x}} = \frac{29}{20}$$
\rozwStop
\odpStart
$\frac{29}{20}$
\odpStop
\testStart
A.$\frac{29}{20}$
B.$\infty$
C.$-\infty$
D.$0$
E.$-\frac{29}{20}$
F.$\frac{20}{29}$
G.$-\frac{20}{29}$
H.$20$
I.$29$
\testStop
\kluczStart
A
\kluczStop



\zadStart{Przykład z Wikieł P 4.3a moja wersja nr 595}


Obliczyć granicę funkcji $\lim\limits_{x\to\ 0}\frac{29 \cdot x}{tan(21 \cdot x)}$.
\zadStop
\rozwStart{Patryk Wirkus}{}
$$\lim\limits_{x\to\ 0}\frac{29 \cdot x}{tan(21 \cdot x)}=\lim\limits_{x\to\ 0}\frac{29 \cdot x \cdot cos(21 \cdot x)}{sin(21 \cdot x)}=\lim\limits_{x\to\ 0}\frac{29 \cdot cos(21 \cdot x)}{\frac{sin(21 \cdot x)}{x}}=\lim\limits_{x\to\ 0}\frac{29 \cdot cos(21 \cdot x)}{21 \cdot \frac{sin(21 \cdot x)}{21 \cdot x}} = \frac{29}{21}$$
\rozwStop
\odpStart
$\frac{29}{21}$
\odpStop
\testStart
A.$\frac{29}{21}$
B.$\infty$
C.$-\infty$
D.$0$
E.$-\frac{29}{21}$
F.$\frac{21}{29}$
G.$-\frac{21}{29}$
H.$21$
I.$29$
\testStop
\kluczStart
A
\kluczStop



\zadStart{Przykład z Wikieł P 4.3a moja wersja nr 596}


Obliczyć granicę funkcji $\lim\limits_{x\to\ 0}\frac{29 \cdot x}{tan(22 \cdot x)}$.
\zadStop
\rozwStart{Patryk Wirkus}{}
$$\lim\limits_{x\to\ 0}\frac{29 \cdot x}{tan(22 \cdot x)}=\lim\limits_{x\to\ 0}\frac{29 \cdot x \cdot cos(22 \cdot x)}{sin(22 \cdot x)}=\lim\limits_{x\to\ 0}\frac{29 \cdot cos(22 \cdot x)}{\frac{sin(22 \cdot x)}{x}}=\lim\limits_{x\to\ 0}\frac{29 \cdot cos(22 \cdot x)}{22 \cdot \frac{sin(22 \cdot x)}{22 \cdot x}} = \frac{29}{22}$$
\rozwStop
\odpStart
$\frac{29}{22}$
\odpStop
\testStart
A.$\frac{29}{22}$
B.$\infty$
C.$-\infty$
D.$0$
E.$-\frac{29}{22}$
F.$\frac{22}{29}$
G.$-\frac{22}{29}$
H.$22$
I.$29$
\testStop
\kluczStart
A
\kluczStop



\zadStart{Przykład z Wikieł P 4.3a moja wersja nr 597}


Obliczyć granicę funkcji $\lim\limits_{x\to\ 0}\frac{29 \cdot x}{tan(23 \cdot x)}$.
\zadStop
\rozwStart{Patryk Wirkus}{}
$$\lim\limits_{x\to\ 0}\frac{29 \cdot x}{tan(23 \cdot x)}=\lim\limits_{x\to\ 0}\frac{29 \cdot x \cdot cos(23 \cdot x)}{sin(23 \cdot x)}=\lim\limits_{x\to\ 0}\frac{29 \cdot cos(23 \cdot x)}{\frac{sin(23 \cdot x)}{x}}=\lim\limits_{x\to\ 0}\frac{29 \cdot cos(23 \cdot x)}{23 \cdot \frac{sin(23 \cdot x)}{23 \cdot x}} = \frac{29}{23}$$
\rozwStop
\odpStart
$\frac{29}{23}$
\odpStop
\testStart
A.$\frac{29}{23}$
B.$\infty$
C.$-\infty$
D.$0$
E.$-\frac{29}{23}$
F.$\frac{23}{29}$
G.$-\frac{23}{29}$
H.$23$
I.$29$
\testStop
\kluczStart
A
\kluczStop



\zadStart{Przykład z Wikieł P 4.3a moja wersja nr 598}


Obliczyć granicę funkcji $\lim\limits_{x\to\ 0}\frac{29 \cdot x}{tan(24 \cdot x)}$.
\zadStop
\rozwStart{Patryk Wirkus}{}
$$\lim\limits_{x\to\ 0}\frac{29 \cdot x}{tan(24 \cdot x)}=\lim\limits_{x\to\ 0}\frac{29 \cdot x \cdot cos(24 \cdot x)}{sin(24 \cdot x)}=\lim\limits_{x\to\ 0}\frac{29 \cdot cos(24 \cdot x)}{\frac{sin(24 \cdot x)}{x}}=\lim\limits_{x\to\ 0}\frac{29 \cdot cos(24 \cdot x)}{24 \cdot \frac{sin(24 \cdot x)}{24 \cdot x}} = \frac{29}{24}$$
\rozwStop
\odpStart
$\frac{29}{24}$
\odpStop
\testStart
A.$\frac{29}{24}$
B.$\infty$
C.$-\infty$
D.$0$
E.$-\frac{29}{24}$
F.$\frac{24}{29}$
G.$-\frac{24}{29}$
H.$24$
I.$29$
\testStop
\kluczStart
A
\kluczStop



\zadStart{Przykład z Wikieł P 4.3a moja wersja nr 599}


Obliczyć granicę funkcji $\lim\limits_{x\to\ 0}\frac{29 \cdot x}{tan(25 \cdot x)}$.
\zadStop
\rozwStart{Patryk Wirkus}{}
$$\lim\limits_{x\to\ 0}\frac{29 \cdot x}{tan(25 \cdot x)}=\lim\limits_{x\to\ 0}\frac{29 \cdot x \cdot cos(25 \cdot x)}{sin(25 \cdot x)}=\lim\limits_{x\to\ 0}\frac{29 \cdot cos(25 \cdot x)}{\frac{sin(25 \cdot x)}{x}}=\lim\limits_{x\to\ 0}\frac{29 \cdot cos(25 \cdot x)}{25 \cdot \frac{sin(25 \cdot x)}{25 \cdot x}} = \frac{29}{25}$$
\rozwStop
\odpStart
$\frac{29}{25}$
\odpStop
\testStart
A.$\frac{29}{25}$
B.$\infty$
C.$-\infty$
D.$0$
E.$-\frac{29}{25}$
F.$\frac{25}{29}$
G.$-\frac{25}{29}$
H.$25$
I.$29$
\testStop
\kluczStart
A
\kluczStop



\zadStart{Przykład z Wikieł P 4.3a moja wersja nr 600}


Obliczyć granicę funkcji $\lim\limits_{x\to\ 0}\frac{29 \cdot x}{tan(26 \cdot x)}$.
\zadStop
\rozwStart{Patryk Wirkus}{}
$$\lim\limits_{x\to\ 0}\frac{29 \cdot x}{tan(26 \cdot x)}=\lim\limits_{x\to\ 0}\frac{29 \cdot x \cdot cos(26 \cdot x)}{sin(26 \cdot x)}=\lim\limits_{x\to\ 0}\frac{29 \cdot cos(26 \cdot x)}{\frac{sin(26 \cdot x)}{x}}=\lim\limits_{x\to\ 0}\frac{29 \cdot cos(26 \cdot x)}{26 \cdot \frac{sin(26 \cdot x)}{26 \cdot x}} = \frac{29}{26}$$
\rozwStop
\odpStart
$\frac{29}{26}$
\odpStop
\testStart
A.$\frac{29}{26}$
B.$\infty$
C.$-\infty$
D.$0$
E.$-\frac{29}{26}$
F.$\frac{26}{29}$
G.$-\frac{26}{29}$
H.$26$
I.$29$
\testStop
\kluczStart
A
\kluczStop



\zadStart{Przykład z Wikieł P 4.3a moja wersja nr 601}


Obliczyć granicę funkcji $\lim\limits_{x\to\ 0}\frac{29 \cdot x}{tan(27 \cdot x)}$.
\zadStop
\rozwStart{Patryk Wirkus}{}
$$\lim\limits_{x\to\ 0}\frac{29 \cdot x}{tan(27 \cdot x)}=\lim\limits_{x\to\ 0}\frac{29 \cdot x \cdot cos(27 \cdot x)}{sin(27 \cdot x)}=\lim\limits_{x\to\ 0}\frac{29 \cdot cos(27 \cdot x)}{\frac{sin(27 \cdot x)}{x}}=\lim\limits_{x\to\ 0}\frac{29 \cdot cos(27 \cdot x)}{27 \cdot \frac{sin(27 \cdot x)}{27 \cdot x}} = \frac{29}{27}$$
\rozwStop
\odpStart
$\frac{29}{27}$
\odpStop
\testStart
A.$\frac{29}{27}$
B.$\infty$
C.$-\infty$
D.$0$
E.$-\frac{29}{27}$
F.$\frac{27}{29}$
G.$-\frac{27}{29}$
H.$27$
I.$29$
\testStop
\kluczStart
A
\kluczStop



\zadStart{Przykład z Wikieł P 4.3a moja wersja nr 602}


Obliczyć granicę funkcji $\lim\limits_{x\to\ 0}\frac{29 \cdot x}{tan(28 \cdot x)}$.
\zadStop
\rozwStart{Patryk Wirkus}{}
$$\lim\limits_{x\to\ 0}\frac{29 \cdot x}{tan(28 \cdot x)}=\lim\limits_{x\to\ 0}\frac{29 \cdot x \cdot cos(28 \cdot x)}{sin(28 \cdot x)}=\lim\limits_{x\to\ 0}\frac{29 \cdot cos(28 \cdot x)}{\frac{sin(28 \cdot x)}{x}}=\lim\limits_{x\to\ 0}\frac{29 \cdot cos(28 \cdot x)}{28 \cdot \frac{sin(28 \cdot x)}{28 \cdot x}} = \frac{29}{28}$$
\rozwStop
\odpStart
$\frac{29}{28}$
\odpStop
\testStart
A.$\frac{29}{28}$
B.$\infty$
C.$-\infty$
D.$0$
E.$-\frac{29}{28}$
F.$\frac{28}{29}$
G.$-\frac{28}{29}$
H.$28$
I.$29$
\testStop
\kluczStart
A
\kluczStop



\zadStart{Przykład z Wikieł P 4.3a moja wersja nr 603}


Obliczyć granicę funkcji $\lim\limits_{x\to\ 0}\frac{29 \cdot x}{tan(30 \cdot x)}$.
\zadStop
\rozwStart{Patryk Wirkus}{}
$$\lim\limits_{x\to\ 0}\frac{29 \cdot x}{tan(30 \cdot x)}=\lim\limits_{x\to\ 0}\frac{29 \cdot x \cdot cos(30 \cdot x)}{sin(30 \cdot x)}=\lim\limits_{x\to\ 0}\frac{29 \cdot cos(30 \cdot x)}{\frac{sin(30 \cdot x)}{x}}=\lim\limits_{x\to\ 0}\frac{29 \cdot cos(30 \cdot x)}{30 \cdot \frac{sin(30 \cdot x)}{30 \cdot x}} = \frac{29}{30}$$
\rozwStop
\odpStart
$\frac{29}{30}$
\odpStop
\testStart
A.$\frac{29}{30}$
B.$\infty$
C.$-\infty$
D.$0$
E.$-\frac{29}{30}$
F.$\frac{30}{29}$
G.$-\frac{30}{29}$
H.$30$
I.$29$
\testStop
\kluczStart
A
\kluczStop



\zadStart{Przykład z Wikieł P 4.3a moja wersja nr 604}


Obliczyć granicę funkcji $\lim\limits_{x\to\ 0}\frac{29 \cdot x}{tan(31 \cdot x)}$.
\zadStop
\rozwStart{Patryk Wirkus}{}
$$\lim\limits_{x\to\ 0}\frac{29 \cdot x}{tan(31 \cdot x)}=\lim\limits_{x\to\ 0}\frac{29 \cdot x \cdot cos(31 \cdot x)}{sin(31 \cdot x)}=\lim\limits_{x\to\ 0}\frac{29 \cdot cos(31 \cdot x)}{\frac{sin(31 \cdot x)}{x}}=\lim\limits_{x\to\ 0}\frac{29 \cdot cos(31 \cdot x)}{31 \cdot \frac{sin(31 \cdot x)}{31 \cdot x}} = \frac{29}{31}$$
\rozwStop
\odpStart
$\frac{29}{31}$
\odpStop
\testStart
A.$\frac{29}{31}$
B.$\infty$
C.$-\infty$
D.$0$
E.$-\frac{29}{31}$
F.$\frac{31}{29}$
G.$-\frac{31}{29}$
H.$31$
I.$29$
\testStop
\kluczStart
A
\kluczStop



\zadStart{Przykład z Wikieł P 4.3a moja wersja nr 605}


Obliczyć granicę funkcji $\lim\limits_{x\to\ 0}\frac{29 \cdot x}{tan(32 \cdot x)}$.
\zadStop
\rozwStart{Patryk Wirkus}{}
$$\lim\limits_{x\to\ 0}\frac{29 \cdot x}{tan(32 \cdot x)}=\lim\limits_{x\to\ 0}\frac{29 \cdot x \cdot cos(32 \cdot x)}{sin(32 \cdot x)}=\lim\limits_{x\to\ 0}\frac{29 \cdot cos(32 \cdot x)}{\frac{sin(32 \cdot x)}{x}}=\lim\limits_{x\to\ 0}\frac{29 \cdot cos(32 \cdot x)}{32 \cdot \frac{sin(32 \cdot x)}{32 \cdot x}} = \frac{29}{32}$$
\rozwStop
\odpStart
$\frac{29}{32}$
\odpStop
\testStart
A.$\frac{29}{32}$
B.$\infty$
C.$-\infty$
D.$0$
E.$-\frac{29}{32}$
F.$\frac{32}{29}$
G.$-\frac{32}{29}$
H.$32$
I.$29$
\testStop
\kluczStart
A
\kluczStop



\zadStart{Przykład z Wikieł P 4.3a moja wersja nr 606}


Obliczyć granicę funkcji $\lim\limits_{x\to\ 0}\frac{29 \cdot x}{tan(33 \cdot x)}$.
\zadStop
\rozwStart{Patryk Wirkus}{}
$$\lim\limits_{x\to\ 0}\frac{29 \cdot x}{tan(33 \cdot x)}=\lim\limits_{x\to\ 0}\frac{29 \cdot x \cdot cos(33 \cdot x)}{sin(33 \cdot x)}=\lim\limits_{x\to\ 0}\frac{29 \cdot cos(33 \cdot x)}{\frac{sin(33 \cdot x)}{x}}=\lim\limits_{x\to\ 0}\frac{29 \cdot cos(33 \cdot x)}{33 \cdot \frac{sin(33 \cdot x)}{33 \cdot x}} = \frac{29}{33}$$
\rozwStop
\odpStart
$\frac{29}{33}$
\odpStop
\testStart
A.$\frac{29}{33}$
B.$\infty$
C.$-\infty$
D.$0$
E.$-\frac{29}{33}$
F.$\frac{33}{29}$
G.$-\frac{33}{29}$
H.$33$
I.$29$
\testStop
\kluczStart
A
\kluczStop



\zadStart{Przykład z Wikieł P 4.3a moja wersja nr 607}


Obliczyć granicę funkcji $\lim\limits_{x\to\ 0}\frac{29 \cdot x}{tan(34 \cdot x)}$.
\zadStop
\rozwStart{Patryk Wirkus}{}
$$\lim\limits_{x\to\ 0}\frac{29 \cdot x}{tan(34 \cdot x)}=\lim\limits_{x\to\ 0}\frac{29 \cdot x \cdot cos(34 \cdot x)}{sin(34 \cdot x)}=\lim\limits_{x\to\ 0}\frac{29 \cdot cos(34 \cdot x)}{\frac{sin(34 \cdot x)}{x}}=\lim\limits_{x\to\ 0}\frac{29 \cdot cos(34 \cdot x)}{34 \cdot \frac{sin(34 \cdot x)}{34 \cdot x}} = \frac{29}{34}$$
\rozwStop
\odpStart
$\frac{29}{34}$
\odpStop
\testStart
A.$\frac{29}{34}$
B.$\infty$
C.$-\infty$
D.$0$
E.$-\frac{29}{34}$
F.$\frac{34}{29}$
G.$-\frac{34}{29}$
H.$34$
I.$29$
\testStop
\kluczStart
A
\kluczStop



\zadStart{Przykład z Wikieł P 4.3a moja wersja nr 608}


Obliczyć granicę funkcji $\lim\limits_{x\to\ 0}\frac{29 \cdot x}{tan(35 \cdot x)}$.
\zadStop
\rozwStart{Patryk Wirkus}{}
$$\lim\limits_{x\to\ 0}\frac{29 \cdot x}{tan(35 \cdot x)}=\lim\limits_{x\to\ 0}\frac{29 \cdot x \cdot cos(35 \cdot x)}{sin(35 \cdot x)}=\lim\limits_{x\to\ 0}\frac{29 \cdot cos(35 \cdot x)}{\frac{sin(35 \cdot x)}{x}}=\lim\limits_{x\to\ 0}\frac{29 \cdot cos(35 \cdot x)}{35 \cdot \frac{sin(35 \cdot x)}{35 \cdot x}} = \frac{29}{35}$$
\rozwStop
\odpStart
$\frac{29}{35}$
\odpStop
\testStart
A.$\frac{29}{35}$
B.$\infty$
C.$-\infty$
D.$0$
E.$-\frac{29}{35}$
F.$\frac{35}{29}$
G.$-\frac{35}{29}$
H.$35$
I.$29$
\testStop
\kluczStart
A
\kluczStop



\zadStart{Przykład z Wikieł P 4.3a moja wersja nr 609}


Obliczyć granicę funkcji $\lim\limits_{x\to\ 0}\frac{29 \cdot x}{tan(36 \cdot x)}$.
\zadStop
\rozwStart{Patryk Wirkus}{}
$$\lim\limits_{x\to\ 0}\frac{29 \cdot x}{tan(36 \cdot x)}=\lim\limits_{x\to\ 0}\frac{29 \cdot x \cdot cos(36 \cdot x)}{sin(36 \cdot x)}=\lim\limits_{x\to\ 0}\frac{29 \cdot cos(36 \cdot x)}{\frac{sin(36 \cdot x)}{x}}=\lim\limits_{x\to\ 0}\frac{29 \cdot cos(36 \cdot x)}{36 \cdot \frac{sin(36 \cdot x)}{36 \cdot x}} = \frac{29}{36}$$
\rozwStop
\odpStart
$\frac{29}{36}$
\odpStop
\testStart
A.$\frac{29}{36}$
B.$\infty$
C.$-\infty$
D.$0$
E.$-\frac{29}{36}$
F.$\frac{36}{29}$
G.$-\frac{36}{29}$
H.$36$
I.$29$
\testStop
\kluczStart
A
\kluczStop



\zadStart{Przykład z Wikieł P 4.3a moja wersja nr 610}


Obliczyć granicę funkcji $\lim\limits_{x\to\ 0}\frac{29 \cdot x}{tan(37 \cdot x)}$.
\zadStop
\rozwStart{Patryk Wirkus}{}
$$\lim\limits_{x\to\ 0}\frac{29 \cdot x}{tan(37 \cdot x)}=\lim\limits_{x\to\ 0}\frac{29 \cdot x \cdot cos(37 \cdot x)}{sin(37 \cdot x)}=\lim\limits_{x\to\ 0}\frac{29 \cdot cos(37 \cdot x)}{\frac{sin(37 \cdot x)}{x}}=\lim\limits_{x\to\ 0}\frac{29 \cdot cos(37 \cdot x)}{37 \cdot \frac{sin(37 \cdot x)}{37 \cdot x}} = \frac{29}{37}$$
\rozwStop
\odpStart
$\frac{29}{37}$
\odpStop
\testStart
A.$\frac{29}{37}$
B.$\infty$
C.$-\infty$
D.$0$
E.$-\frac{29}{37}$
F.$\frac{37}{29}$
G.$-\frac{37}{29}$
H.$37$
I.$29$
\testStop
\kluczStart
A
\kluczStop



\zadStart{Przykład z Wikieł P 4.3a moja wersja nr 611}


Obliczyć granicę funkcji $\lim\limits_{x\to\ 0}\frac{29 \cdot x}{tan(38 \cdot x)}$.
\zadStop
\rozwStart{Patryk Wirkus}{}
$$\lim\limits_{x\to\ 0}\frac{29 \cdot x}{tan(38 \cdot x)}=\lim\limits_{x\to\ 0}\frac{29 \cdot x \cdot cos(38 \cdot x)}{sin(38 \cdot x)}=\lim\limits_{x\to\ 0}\frac{29 \cdot cos(38 \cdot x)}{\frac{sin(38 \cdot x)}{x}}=\lim\limits_{x\to\ 0}\frac{29 \cdot cos(38 \cdot x)}{38 \cdot \frac{sin(38 \cdot x)}{38 \cdot x}} = \frac{29}{38}$$
\rozwStop
\odpStart
$\frac{29}{38}$
\odpStop
\testStart
A.$\frac{29}{38}$
B.$\infty$
C.$-\infty$
D.$0$
E.$-\frac{29}{38}$
F.$\frac{38}{29}$
G.$-\frac{38}{29}$
H.$38$
I.$29$
\testStop
\kluczStart
A
\kluczStop



\zadStart{Przykład z Wikieł P 4.3a moja wersja nr 612}


Obliczyć granicę funkcji $\lim\limits_{x\to\ 0}\frac{29 \cdot x}{tan(39 \cdot x)}$.
\zadStop
\rozwStart{Patryk Wirkus}{}
$$\lim\limits_{x\to\ 0}\frac{29 \cdot x}{tan(39 \cdot x)}=\lim\limits_{x\to\ 0}\frac{29 \cdot x \cdot cos(39 \cdot x)}{sin(39 \cdot x)}=\lim\limits_{x\to\ 0}\frac{29 \cdot cos(39 \cdot x)}{\frac{sin(39 \cdot x)}{x}}=\lim\limits_{x\to\ 0}\frac{29 \cdot cos(39 \cdot x)}{39 \cdot \frac{sin(39 \cdot x)}{39 \cdot x}} = \frac{29}{39}$$
\rozwStop
\odpStart
$\frac{29}{39}$
\odpStop
\testStart
A.$\frac{29}{39}$
B.$\infty$
C.$-\infty$
D.$0$
E.$-\frac{29}{39}$
F.$\frac{39}{29}$
G.$-\frac{39}{29}$
H.$39$
I.$29$
\testStop
\kluczStart
A
\kluczStop



\zadStart{Przykład z Wikieł P 4.3a moja wersja nr 613}


Obliczyć granicę funkcji $\lim\limits_{x\to\ 0}\frac{29 \cdot x}{tan(40 \cdot x)}$.
\zadStop
\rozwStart{Patryk Wirkus}{}
$$\lim\limits_{x\to\ 0}\frac{29 \cdot x}{tan(40 \cdot x)}=\lim\limits_{x\to\ 0}\frac{29 \cdot x \cdot cos(40 \cdot x)}{sin(40 \cdot x)}=\lim\limits_{x\to\ 0}\frac{29 \cdot cos(40 \cdot x)}{\frac{sin(40 \cdot x)}{x}}=\lim\limits_{x\to\ 0}\frac{29 \cdot cos(40 \cdot x)}{40 \cdot \frac{sin(40 \cdot x)}{40 \cdot x}} = \frac{29}{40}$$
\rozwStop
\odpStart
$\frac{29}{40}$
\odpStop
\testStart
A.$\frac{29}{40}$
B.$\infty$
C.$-\infty$
D.$0$
E.$-\frac{29}{40}$
F.$\frac{40}{29}$
G.$-\frac{40}{29}$
H.$40$
I.$29$
\testStop
\kluczStart
A
\kluczStop



\zadStart{Przykład z Wikieł P 4.3a moja wersja nr 614}


Obliczyć granicę funkcji $\lim\limits_{x\to\ 0}\frac{30 \cdot x}{tan(7 \cdot x)}$.
\zadStop
\rozwStart{Patryk Wirkus}{}
$$\lim\limits_{x\to\ 0}\frac{30 \cdot x}{tan(7 \cdot x)}=\lim\limits_{x\to\ 0}\frac{30 \cdot x \cdot cos(7 \cdot x)}{sin(7 \cdot x)}=\lim\limits_{x\to\ 0}\frac{30 \cdot cos(7 \cdot x)}{\frac{sin(7 \cdot x)}{x}}=\lim\limits_{x\to\ 0}\frac{30 \cdot cos(7 \cdot x)}{7 \cdot \frac{sin(7 \cdot x)}{7 \cdot x}} = \frac{30}{7}$$
\rozwStop
\odpStart
$\frac{30}{7}$
\odpStop
\testStart
A.$\frac{30}{7}$
B.$\infty$
C.$-\infty$
D.$0$
E.$-\frac{30}{7}$
F.$\frac{7}{30}$
G.$-\frac{7}{30}$
H.$7$
I.$30$
\testStop
\kluczStart
A
\kluczStop



\zadStart{Przykład z Wikieł P 4.3a moja wersja nr 615}


Obliczyć granicę funkcji $\lim\limits_{x\to\ 0}\frac{30 \cdot x}{tan(11 \cdot x)}$.
\zadStop
\rozwStart{Patryk Wirkus}{}
$$\lim\limits_{x\to\ 0}\frac{30 \cdot x}{tan(11 \cdot x)}=\lim\limits_{x\to\ 0}\frac{30 \cdot x \cdot cos(11 \cdot x)}{sin(11 \cdot x)}=\lim\limits_{x\to\ 0}\frac{30 \cdot cos(11 \cdot x)}{\frac{sin(11 \cdot x)}{x}}=\lim\limits_{x\to\ 0}\frac{30 \cdot cos(11 \cdot x)}{11 \cdot \frac{sin(11 \cdot x)}{11 \cdot x}} = \frac{30}{11}$$
\rozwStop
\odpStart
$\frac{30}{11}$
\odpStop
\testStart
A.$\frac{30}{11}$
B.$\infty$
C.$-\infty$
D.$0$
E.$-\frac{30}{11}$
F.$\frac{11}{30}$
G.$-\frac{11}{30}$
H.$11$
I.$30$
\testStop
\kluczStart
A
\kluczStop



\zadStart{Przykład z Wikieł P 4.3a moja wersja nr 616}


Obliczyć granicę funkcji $\lim\limits_{x\to\ 0}\frac{30 \cdot x}{tan(13 \cdot x)}$.
\zadStop
\rozwStart{Patryk Wirkus}{}
$$\lim\limits_{x\to\ 0}\frac{30 \cdot x}{tan(13 \cdot x)}=\lim\limits_{x\to\ 0}\frac{30 \cdot x \cdot cos(13 \cdot x)}{sin(13 \cdot x)}=\lim\limits_{x\to\ 0}\frac{30 \cdot cos(13 \cdot x)}{\frac{sin(13 \cdot x)}{x}}=\lim\limits_{x\to\ 0}\frac{30 \cdot cos(13 \cdot x)}{13 \cdot \frac{sin(13 \cdot x)}{13 \cdot x}} = \frac{30}{13}$$
\rozwStop
\odpStart
$\frac{30}{13}$
\odpStop
\testStart
A.$\frac{30}{13}$
B.$\infty$
C.$-\infty$
D.$0$
E.$-\frac{30}{13}$
F.$\frac{13}{30}$
G.$-\frac{13}{30}$
H.$13$
I.$30$
\testStop
\kluczStart
A
\kluczStop



\zadStart{Przykład z Wikieł P 4.3a moja wersja nr 617}


Obliczyć granicę funkcji $\lim\limits_{x\to\ 0}\frac{30 \cdot x}{tan(17 \cdot x)}$.
\zadStop
\rozwStart{Patryk Wirkus}{}
$$\lim\limits_{x\to\ 0}\frac{30 \cdot x}{tan(17 \cdot x)}=\lim\limits_{x\to\ 0}\frac{30 \cdot x \cdot cos(17 \cdot x)}{sin(17 \cdot x)}=\lim\limits_{x\to\ 0}\frac{30 \cdot cos(17 \cdot x)}{\frac{sin(17 \cdot x)}{x}}=\lim\limits_{x\to\ 0}\frac{30 \cdot cos(17 \cdot x)}{17 \cdot \frac{sin(17 \cdot x)}{17 \cdot x}} = \frac{30}{17}$$
\rozwStop
\odpStart
$\frac{30}{17}$
\odpStop
\testStart
A.$\frac{30}{17}$
B.$\infty$
C.$-\infty$
D.$0$
E.$-\frac{30}{17}$
F.$\frac{17}{30}$
G.$-\frac{17}{30}$
H.$17$
I.$30$
\testStop
\kluczStart
A
\kluczStop



\zadStart{Przykład z Wikieł P 4.3a moja wersja nr 618}


Obliczyć granicę funkcji $\lim\limits_{x\to\ 0}\frac{30 \cdot x}{tan(19 \cdot x)}$.
\zadStop
\rozwStart{Patryk Wirkus}{}
$$\lim\limits_{x\to\ 0}\frac{30 \cdot x}{tan(19 \cdot x)}=\lim\limits_{x\to\ 0}\frac{30 \cdot x \cdot cos(19 \cdot x)}{sin(19 \cdot x)}=\lim\limits_{x\to\ 0}\frac{30 \cdot cos(19 \cdot x)}{\frac{sin(19 \cdot x)}{x}}=\lim\limits_{x\to\ 0}\frac{30 \cdot cos(19 \cdot x)}{19 \cdot \frac{sin(19 \cdot x)}{19 \cdot x}} = \frac{30}{19}$$
\rozwStop
\odpStart
$\frac{30}{19}$
\odpStop
\testStart
A.$\frac{30}{19}$
B.$\infty$
C.$-\infty$
D.$0$
E.$-\frac{30}{19}$
F.$\frac{19}{30}$
G.$-\frac{19}{30}$
H.$19$
I.$30$
\testStop
\kluczStart
A
\kluczStop



\zadStart{Przykład z Wikieł P 4.3a moja wersja nr 619}


Obliczyć granicę funkcji $\lim\limits_{x\to\ 0}\frac{30 \cdot x}{tan(23 \cdot x)}$.
\zadStop
\rozwStart{Patryk Wirkus}{}
$$\lim\limits_{x\to\ 0}\frac{30 \cdot x}{tan(23 \cdot x)}=\lim\limits_{x\to\ 0}\frac{30 \cdot x \cdot cos(23 \cdot x)}{sin(23 \cdot x)}=\lim\limits_{x\to\ 0}\frac{30 \cdot cos(23 \cdot x)}{\frac{sin(23 \cdot x)}{x}}=\lim\limits_{x\to\ 0}\frac{30 \cdot cos(23 \cdot x)}{23 \cdot \frac{sin(23 \cdot x)}{23 \cdot x}} = \frac{30}{23}$$
\rozwStop
\odpStart
$\frac{30}{23}$
\odpStop
\testStart
A.$\frac{30}{23}$
B.$\infty$
C.$-\infty$
D.$0$
E.$-\frac{30}{23}$
F.$\frac{23}{30}$
G.$-\frac{23}{30}$
H.$23$
I.$30$
\testStop
\kluczStart
A
\kluczStop



\zadStart{Przykład z Wikieł P 4.3a moja wersja nr 620}


Obliczyć granicę funkcji $\lim\limits_{x\to\ 0}\frac{30 \cdot x}{tan(29 \cdot x)}$.
\zadStop
\rozwStart{Patryk Wirkus}{}
$$\lim\limits_{x\to\ 0}\frac{30 \cdot x}{tan(29 \cdot x)}=\lim\limits_{x\to\ 0}\frac{30 \cdot x \cdot cos(29 \cdot x)}{sin(29 \cdot x)}=\lim\limits_{x\to\ 0}\frac{30 \cdot cos(29 \cdot x)}{\frac{sin(29 \cdot x)}{x}}=\lim\limits_{x\to\ 0}\frac{30 \cdot cos(29 \cdot x)}{29 \cdot \frac{sin(29 \cdot x)}{29 \cdot x}} = \frac{30}{29}$$
\rozwStop
\odpStart
$\frac{30}{29}$
\odpStop
\testStart
A.$\frac{30}{29}$
B.$\infty$
C.$-\infty$
D.$0$
E.$-\frac{30}{29}$
F.$\frac{29}{30}$
G.$-\frac{29}{30}$
H.$29$
I.$30$
\testStop
\kluczStart
A
\kluczStop



\zadStart{Przykład z Wikieł P 4.3a moja wersja nr 621}


Obliczyć granicę funkcji $\lim\limits_{x\to\ 0}\frac{30 \cdot x}{tan(31 \cdot x)}$.
\zadStop
\rozwStart{Patryk Wirkus}{}
$$\lim\limits_{x\to\ 0}\frac{30 \cdot x}{tan(31 \cdot x)}=\lim\limits_{x\to\ 0}\frac{30 \cdot x \cdot cos(31 \cdot x)}{sin(31 \cdot x)}=\lim\limits_{x\to\ 0}\frac{30 \cdot cos(31 \cdot x)}{\frac{sin(31 \cdot x)}{x}}=\lim\limits_{x\to\ 0}\frac{30 \cdot cos(31 \cdot x)}{31 \cdot \frac{sin(31 \cdot x)}{31 \cdot x}} = \frac{30}{31}$$
\rozwStop
\odpStart
$\frac{30}{31}$
\odpStop
\testStart
A.$\frac{30}{31}$
B.$\infty$
C.$-\infty$
D.$0$
E.$-\frac{30}{31}$
F.$\frac{31}{30}$
G.$-\frac{31}{30}$
H.$31$
I.$30$
\testStop
\kluczStart
A
\kluczStop



\zadStart{Przykład z Wikieł P 4.3a moja wersja nr 622}


Obliczyć granicę funkcji $\lim\limits_{x\to\ 0}\frac{30 \cdot x}{tan(37 \cdot x)}$.
\zadStop
\rozwStart{Patryk Wirkus}{}
$$\lim\limits_{x\to\ 0}\frac{30 \cdot x}{tan(37 \cdot x)}=\lim\limits_{x\to\ 0}\frac{30 \cdot x \cdot cos(37 \cdot x)}{sin(37 \cdot x)}=\lim\limits_{x\to\ 0}\frac{30 \cdot cos(37 \cdot x)}{\frac{sin(37 \cdot x)}{x}}=\lim\limits_{x\to\ 0}\frac{30 \cdot cos(37 \cdot x)}{37 \cdot \frac{sin(37 \cdot x)}{37 \cdot x}} = \frac{30}{37}$$
\rozwStop
\odpStart
$\frac{30}{37}$
\odpStop
\testStart
A.$\frac{30}{37}$
B.$\infty$
C.$-\infty$
D.$0$
E.$-\frac{30}{37}$
F.$\frac{37}{30}$
G.$-\frac{37}{30}$
H.$37$
I.$30$
\testStop
\kluczStart
A
\kluczStop



\zadStart{Przykład z Wikieł P 4.3a moja wersja nr 623}


Obliczyć granicę funkcji $\lim\limits_{x\to\ 0}\frac{31 \cdot x}{tan(2 \cdot x)}$.
\zadStop
\rozwStart{Patryk Wirkus}{}
$$\lim\limits_{x\to\ 0}\frac{31 \cdot x}{tan(2 \cdot x)}=\lim\limits_{x\to\ 0}\frac{31 \cdot x \cdot cos(2 \cdot x)}{sin(2 \cdot x)}=\lim\limits_{x\to\ 0}\frac{31 \cdot cos(2 \cdot x)}{\frac{sin(2 \cdot x)}{x}}=\lim\limits_{x\to\ 0}\frac{31 \cdot cos(2 \cdot x)}{2 \cdot \frac{sin(2 \cdot x)}{2 \cdot x}} = \frac{31}{2}$$
\rozwStop
\odpStart
$\frac{31}{2}$
\odpStop
\testStart
A.$\frac{31}{2}$
B.$\infty$
C.$-\infty$
D.$0$
E.$-\frac{31}{2}$
F.$\frac{2}{31}$
G.$-\frac{2}{31}$
H.$2$
I.$31$
\testStop
\kluczStart
A
\kluczStop



\zadStart{Przykład z Wikieł P 4.3a moja wersja nr 624}


Obliczyć granicę funkcji $\lim\limits_{x\to\ 0}\frac{31 \cdot x}{tan(3 \cdot x)}$.
\zadStop
\rozwStart{Patryk Wirkus}{}
$$\lim\limits_{x\to\ 0}\frac{31 \cdot x}{tan(3 \cdot x)}=\lim\limits_{x\to\ 0}\frac{31 \cdot x \cdot cos(3 \cdot x)}{sin(3 \cdot x)}=\lim\limits_{x\to\ 0}\frac{31 \cdot cos(3 \cdot x)}{\frac{sin(3 \cdot x)}{x}}=\lim\limits_{x\to\ 0}\frac{31 \cdot cos(3 \cdot x)}{3 \cdot \frac{sin(3 \cdot x)}{3 \cdot x}} = \frac{31}{3}$$
\rozwStop
\odpStart
$\frac{31}{3}$
\odpStop
\testStart
A.$\frac{31}{3}$
B.$\infty$
C.$-\infty$
D.$0$
E.$-\frac{31}{3}$
F.$\frac{3}{31}$
G.$-\frac{3}{31}$
H.$3$
I.$31$
\testStop
\kluczStart
A
\kluczStop



\zadStart{Przykład z Wikieł P 4.3a moja wersja nr 625}


Obliczyć granicę funkcji $\lim\limits_{x\to\ 0}\frac{31 \cdot x}{tan(4 \cdot x)}$.
\zadStop
\rozwStart{Patryk Wirkus}{}
$$\lim\limits_{x\to\ 0}\frac{31 \cdot x}{tan(4 \cdot x)}=\lim\limits_{x\to\ 0}\frac{31 \cdot x \cdot cos(4 \cdot x)}{sin(4 \cdot x)}=\lim\limits_{x\to\ 0}\frac{31 \cdot cos(4 \cdot x)}{\frac{sin(4 \cdot x)}{x}}=\lim\limits_{x\to\ 0}\frac{31 \cdot cos(4 \cdot x)}{4 \cdot \frac{sin(4 \cdot x)}{4 \cdot x}} = \frac{31}{4}$$
\rozwStop
\odpStart
$\frac{31}{4}$
\odpStop
\testStart
A.$\frac{31}{4}$
B.$\infty$
C.$-\infty$
D.$0$
E.$-\frac{31}{4}$
F.$\frac{4}{31}$
G.$-\frac{4}{31}$
H.$4$
I.$31$
\testStop
\kluczStart
A
\kluczStop



\zadStart{Przykład z Wikieł P 4.3a moja wersja nr 626}


Obliczyć granicę funkcji $\lim\limits_{x\to\ 0}\frac{31 \cdot x}{tan(5 \cdot x)}$.
\zadStop
\rozwStart{Patryk Wirkus}{}
$$\lim\limits_{x\to\ 0}\frac{31 \cdot x}{tan(5 \cdot x)}=\lim\limits_{x\to\ 0}\frac{31 \cdot x \cdot cos(5 \cdot x)}{sin(5 \cdot x)}=\lim\limits_{x\to\ 0}\frac{31 \cdot cos(5 \cdot x)}{\frac{sin(5 \cdot x)}{x}}=\lim\limits_{x\to\ 0}\frac{31 \cdot cos(5 \cdot x)}{5 \cdot \frac{sin(5 \cdot x)}{5 \cdot x}} = \frac{31}{5}$$
\rozwStop
\odpStart
$\frac{31}{5}$
\odpStop
\testStart
A.$\frac{31}{5}$
B.$\infty$
C.$-\infty$
D.$0$
E.$-\frac{31}{5}$
F.$\frac{5}{31}$
G.$-\frac{5}{31}$
H.$5$
I.$31$
\testStop
\kluczStart
A
\kluczStop



\zadStart{Przykład z Wikieł P 4.3a moja wersja nr 627}


Obliczyć granicę funkcji $\lim\limits_{x\to\ 0}\frac{31 \cdot x}{tan(6 \cdot x)}$.
\zadStop
\rozwStart{Patryk Wirkus}{}
$$\lim\limits_{x\to\ 0}\frac{31 \cdot x}{tan(6 \cdot x)}=\lim\limits_{x\to\ 0}\frac{31 \cdot x \cdot cos(6 \cdot x)}{sin(6 \cdot x)}=\lim\limits_{x\to\ 0}\frac{31 \cdot cos(6 \cdot x)}{\frac{sin(6 \cdot x)}{x}}=\lim\limits_{x\to\ 0}\frac{31 \cdot cos(6 \cdot x)}{6 \cdot \frac{sin(6 \cdot x)}{6 \cdot x}} = \frac{31}{6}$$
\rozwStop
\odpStart
$\frac{31}{6}$
\odpStop
\testStart
A.$\frac{31}{6}$
B.$\infty$
C.$-\infty$
D.$0$
E.$-\frac{31}{6}$
F.$\frac{6}{31}$
G.$-\frac{6}{31}$
H.$6$
I.$31$
\testStop
\kluczStart
A
\kluczStop



\zadStart{Przykład z Wikieł P 4.3a moja wersja nr 628}


Obliczyć granicę funkcji $\lim\limits_{x\to\ 0}\frac{31 \cdot x}{tan(7 \cdot x)}$.
\zadStop
\rozwStart{Patryk Wirkus}{}
$$\lim\limits_{x\to\ 0}\frac{31 \cdot x}{tan(7 \cdot x)}=\lim\limits_{x\to\ 0}\frac{31 \cdot x \cdot cos(7 \cdot x)}{sin(7 \cdot x)}=\lim\limits_{x\to\ 0}\frac{31 \cdot cos(7 \cdot x)}{\frac{sin(7 \cdot x)}{x}}=\lim\limits_{x\to\ 0}\frac{31 \cdot cos(7 \cdot x)}{7 \cdot \frac{sin(7 \cdot x)}{7 \cdot x}} = \frac{31}{7}$$
\rozwStop
\odpStart
$\frac{31}{7}$
\odpStop
\testStart
A.$\frac{31}{7}$
B.$\infty$
C.$-\infty$
D.$0$
E.$-\frac{31}{7}$
F.$\frac{7}{31}$
G.$-\frac{7}{31}$
H.$7$
I.$31$
\testStop
\kluczStart
A
\kluczStop



\zadStart{Przykład z Wikieł P 4.3a moja wersja nr 629}


Obliczyć granicę funkcji $\lim\limits_{x\to\ 0}\frac{31 \cdot x}{tan(8 \cdot x)}$.
\zadStop
\rozwStart{Patryk Wirkus}{}
$$\lim\limits_{x\to\ 0}\frac{31 \cdot x}{tan(8 \cdot x)}=\lim\limits_{x\to\ 0}\frac{31 \cdot x \cdot cos(8 \cdot x)}{sin(8 \cdot x)}=\lim\limits_{x\to\ 0}\frac{31 \cdot cos(8 \cdot x)}{\frac{sin(8 \cdot x)}{x}}=\lim\limits_{x\to\ 0}\frac{31 \cdot cos(8 \cdot x)}{8 \cdot \frac{sin(8 \cdot x)}{8 \cdot x}} = \frac{31}{8}$$
\rozwStop
\odpStart
$\frac{31}{8}$
\odpStop
\testStart
A.$\frac{31}{8}$
B.$\infty$
C.$-\infty$
D.$0$
E.$-\frac{31}{8}$
F.$\frac{8}{31}$
G.$-\frac{8}{31}$
H.$8$
I.$31$
\testStop
\kluczStart
A
\kluczStop



\zadStart{Przykład z Wikieł P 4.3a moja wersja nr 630}


Obliczyć granicę funkcji $\lim\limits_{x\to\ 0}\frac{31 \cdot x}{tan(9 \cdot x)}$.
\zadStop
\rozwStart{Patryk Wirkus}{}
$$\lim\limits_{x\to\ 0}\frac{31 \cdot x}{tan(9 \cdot x)}=\lim\limits_{x\to\ 0}\frac{31 \cdot x \cdot cos(9 \cdot x)}{sin(9 \cdot x)}=\lim\limits_{x\to\ 0}\frac{31 \cdot cos(9 \cdot x)}{\frac{sin(9 \cdot x)}{x}}=\lim\limits_{x\to\ 0}\frac{31 \cdot cos(9 \cdot x)}{9 \cdot \frac{sin(9 \cdot x)}{9 \cdot x}} = \frac{31}{9}$$
\rozwStop
\odpStart
$\frac{31}{9}$
\odpStop
\testStart
A.$\frac{31}{9}$
B.$\infty$
C.$-\infty$
D.$0$
E.$-\frac{31}{9}$
F.$\frac{9}{31}$
G.$-\frac{9}{31}$
H.$9$
I.$31$
\testStop
\kluczStart
A
\kluczStop



\zadStart{Przykład z Wikieł P 4.3a moja wersja nr 631}


Obliczyć granicę funkcji $\lim\limits_{x\to\ 0}\frac{31 \cdot x}{tan(10 \cdot x)}$.
\zadStop
\rozwStart{Patryk Wirkus}{}
$$\lim\limits_{x\to\ 0}\frac{31 \cdot x}{tan(10 \cdot x)}=\lim\limits_{x\to\ 0}\frac{31 \cdot x \cdot cos(10 \cdot x)}{sin(10 \cdot x)}=\lim\limits_{x\to\ 0}\frac{31 \cdot cos(10 \cdot x)}{\frac{sin(10 \cdot x)}{x}}=\lim\limits_{x\to\ 0}\frac{31 \cdot cos(10 \cdot x)}{10 \cdot \frac{sin(10 \cdot x)}{10 \cdot x}} = \frac{31}{10}$$
\rozwStop
\odpStart
$\frac{31}{10}$
\odpStop
\testStart
A.$\frac{31}{10}$
B.$\infty$
C.$-\infty$
D.$0$
E.$-\frac{31}{10}$
F.$\frac{10}{31}$
G.$-\frac{10}{31}$
H.$10$
I.$31$
\testStop
\kluczStart
A
\kluczStop



\zadStart{Przykład z Wikieł P 4.3a moja wersja nr 632}


Obliczyć granicę funkcji $\lim\limits_{x\to\ 0}\frac{31 \cdot x}{tan(11 \cdot x)}$.
\zadStop
\rozwStart{Patryk Wirkus}{}
$$\lim\limits_{x\to\ 0}\frac{31 \cdot x}{tan(11 \cdot x)}=\lim\limits_{x\to\ 0}\frac{31 \cdot x \cdot cos(11 \cdot x)}{sin(11 \cdot x)}=\lim\limits_{x\to\ 0}\frac{31 \cdot cos(11 \cdot x)}{\frac{sin(11 \cdot x)}{x}}=\lim\limits_{x\to\ 0}\frac{31 \cdot cos(11 \cdot x)}{11 \cdot \frac{sin(11 \cdot x)}{11 \cdot x}} = \frac{31}{11}$$
\rozwStop
\odpStart
$\frac{31}{11}$
\odpStop
\testStart
A.$\frac{31}{11}$
B.$\infty$
C.$-\infty$
D.$0$
E.$-\frac{31}{11}$
F.$\frac{11}{31}$
G.$-\frac{11}{31}$
H.$11$
I.$31$
\testStop
\kluczStart
A
\kluczStop



\zadStart{Przykład z Wikieł P 4.3a moja wersja nr 633}


Obliczyć granicę funkcji $\lim\limits_{x\to\ 0}\frac{31 \cdot x}{tan(12 \cdot x)}$.
\zadStop
\rozwStart{Patryk Wirkus}{}
$$\lim\limits_{x\to\ 0}\frac{31 \cdot x}{tan(12 \cdot x)}=\lim\limits_{x\to\ 0}\frac{31 \cdot x \cdot cos(12 \cdot x)}{sin(12 \cdot x)}=\lim\limits_{x\to\ 0}\frac{31 \cdot cos(12 \cdot x)}{\frac{sin(12 \cdot x)}{x}}=\lim\limits_{x\to\ 0}\frac{31 \cdot cos(12 \cdot x)}{12 \cdot \frac{sin(12 \cdot x)}{12 \cdot x}} = \frac{31}{12}$$
\rozwStop
\odpStart
$\frac{31}{12}$
\odpStop
\testStart
A.$\frac{31}{12}$
B.$\infty$
C.$-\infty$
D.$0$
E.$-\frac{31}{12}$
F.$\frac{12}{31}$
G.$-\frac{12}{31}$
H.$12$
I.$31$
\testStop
\kluczStart
A
\kluczStop



\zadStart{Przykład z Wikieł P 4.3a moja wersja nr 634}


Obliczyć granicę funkcji $\lim\limits_{x\to\ 0}\frac{31 \cdot x}{tan(13 \cdot x)}$.
\zadStop
\rozwStart{Patryk Wirkus}{}
$$\lim\limits_{x\to\ 0}\frac{31 \cdot x}{tan(13 \cdot x)}=\lim\limits_{x\to\ 0}\frac{31 \cdot x \cdot cos(13 \cdot x)}{sin(13 \cdot x)}=\lim\limits_{x\to\ 0}\frac{31 \cdot cos(13 \cdot x)}{\frac{sin(13 \cdot x)}{x}}=\lim\limits_{x\to\ 0}\frac{31 \cdot cos(13 \cdot x)}{13 \cdot \frac{sin(13 \cdot x)}{13 \cdot x}} = \frac{31}{13}$$
\rozwStop
\odpStart
$\frac{31}{13}$
\odpStop
\testStart
A.$\frac{31}{13}$
B.$\infty$
C.$-\infty$
D.$0$
E.$-\frac{31}{13}$
F.$\frac{13}{31}$
G.$-\frac{13}{31}$
H.$13$
I.$31$
\testStop
\kluczStart
A
\kluczStop



\zadStart{Przykład z Wikieł P 4.3a moja wersja nr 635}


Obliczyć granicę funkcji $\lim\limits_{x\to\ 0}\frac{31 \cdot x}{tan(14 \cdot x)}$.
\zadStop
\rozwStart{Patryk Wirkus}{}
$$\lim\limits_{x\to\ 0}\frac{31 \cdot x}{tan(14 \cdot x)}=\lim\limits_{x\to\ 0}\frac{31 \cdot x \cdot cos(14 \cdot x)}{sin(14 \cdot x)}=\lim\limits_{x\to\ 0}\frac{31 \cdot cos(14 \cdot x)}{\frac{sin(14 \cdot x)}{x}}=\lim\limits_{x\to\ 0}\frac{31 \cdot cos(14 \cdot x)}{14 \cdot \frac{sin(14 \cdot x)}{14 \cdot x}} = \frac{31}{14}$$
\rozwStop
\odpStart
$\frac{31}{14}$
\odpStop
\testStart
A.$\frac{31}{14}$
B.$\infty$
C.$-\infty$
D.$0$
E.$-\frac{31}{14}$
F.$\frac{14}{31}$
G.$-\frac{14}{31}$
H.$14$
I.$31$
\testStop
\kluczStart
A
\kluczStop



\zadStart{Przykład z Wikieł P 4.3a moja wersja nr 636}


Obliczyć granicę funkcji $\lim\limits_{x\to\ 0}\frac{31 \cdot x}{tan(15 \cdot x)}$.
\zadStop
\rozwStart{Patryk Wirkus}{}
$$\lim\limits_{x\to\ 0}\frac{31 \cdot x}{tan(15 \cdot x)}=\lim\limits_{x\to\ 0}\frac{31 \cdot x \cdot cos(15 \cdot x)}{sin(15 \cdot x)}=\lim\limits_{x\to\ 0}\frac{31 \cdot cos(15 \cdot x)}{\frac{sin(15 \cdot x)}{x}}=\lim\limits_{x\to\ 0}\frac{31 \cdot cos(15 \cdot x)}{15 \cdot \frac{sin(15 \cdot x)}{15 \cdot x}} = \frac{31}{15}$$
\rozwStop
\odpStart
$\frac{31}{15}$
\odpStop
\testStart
A.$\frac{31}{15}$
B.$\infty$
C.$-\infty$
D.$0$
E.$-\frac{31}{15}$
F.$\frac{15}{31}$
G.$-\frac{15}{31}$
H.$15$
I.$31$
\testStop
\kluczStart
A
\kluczStop



\zadStart{Przykład z Wikieł P 4.3a moja wersja nr 637}


Obliczyć granicę funkcji $\lim\limits_{x\to\ 0}\frac{31 \cdot x}{tan(16 \cdot x)}$.
\zadStop
\rozwStart{Patryk Wirkus}{}
$$\lim\limits_{x\to\ 0}\frac{31 \cdot x}{tan(16 \cdot x)}=\lim\limits_{x\to\ 0}\frac{31 \cdot x \cdot cos(16 \cdot x)}{sin(16 \cdot x)}=\lim\limits_{x\to\ 0}\frac{31 \cdot cos(16 \cdot x)}{\frac{sin(16 \cdot x)}{x}}=\lim\limits_{x\to\ 0}\frac{31 \cdot cos(16 \cdot x)}{16 \cdot \frac{sin(16 \cdot x)}{16 \cdot x}} = \frac{31}{16}$$
\rozwStop
\odpStart
$\frac{31}{16}$
\odpStop
\testStart
A.$\frac{31}{16}$
B.$\infty$
C.$-\infty$
D.$0$
E.$-\frac{31}{16}$
F.$\frac{16}{31}$
G.$-\frac{16}{31}$
H.$16$
I.$31$
\testStop
\kluczStart
A
\kluczStop



\zadStart{Przykład z Wikieł P 4.3a moja wersja nr 638}


Obliczyć granicę funkcji $\lim\limits_{x\to\ 0}\frac{31 \cdot x}{tan(17 \cdot x)}$.
\zadStop
\rozwStart{Patryk Wirkus}{}
$$\lim\limits_{x\to\ 0}\frac{31 \cdot x}{tan(17 \cdot x)}=\lim\limits_{x\to\ 0}\frac{31 \cdot x \cdot cos(17 \cdot x)}{sin(17 \cdot x)}=\lim\limits_{x\to\ 0}\frac{31 \cdot cos(17 \cdot x)}{\frac{sin(17 \cdot x)}{x}}=\lim\limits_{x\to\ 0}\frac{31 \cdot cos(17 \cdot x)}{17 \cdot \frac{sin(17 \cdot x)}{17 \cdot x}} = \frac{31}{17}$$
\rozwStop
\odpStart
$\frac{31}{17}$
\odpStop
\testStart
A.$\frac{31}{17}$
B.$\infty$
C.$-\infty$
D.$0$
E.$-\frac{31}{17}$
F.$\frac{17}{31}$
G.$-\frac{17}{31}$
H.$17$
I.$31$
\testStop
\kluczStart
A
\kluczStop



\zadStart{Przykład z Wikieł P 4.3a moja wersja nr 639}


Obliczyć granicę funkcji $\lim\limits_{x\to\ 0}\frac{31 \cdot x}{tan(18 \cdot x)}$.
\zadStop
\rozwStart{Patryk Wirkus}{}
$$\lim\limits_{x\to\ 0}\frac{31 \cdot x}{tan(18 \cdot x)}=\lim\limits_{x\to\ 0}\frac{31 \cdot x \cdot cos(18 \cdot x)}{sin(18 \cdot x)}=\lim\limits_{x\to\ 0}\frac{31 \cdot cos(18 \cdot x)}{\frac{sin(18 \cdot x)}{x}}=\lim\limits_{x\to\ 0}\frac{31 \cdot cos(18 \cdot x)}{18 \cdot \frac{sin(18 \cdot x)}{18 \cdot x}} = \frac{31}{18}$$
\rozwStop
\odpStart
$\frac{31}{18}$
\odpStop
\testStart
A.$\frac{31}{18}$
B.$\infty$
C.$-\infty$
D.$0$
E.$-\frac{31}{18}$
F.$\frac{18}{31}$
G.$-\frac{18}{31}$
H.$18$
I.$31$
\testStop
\kluczStart
A
\kluczStop



\zadStart{Przykład z Wikieł P 4.3a moja wersja nr 640}


Obliczyć granicę funkcji $\lim\limits_{x\to\ 0}\frac{31 \cdot x}{tan(19 \cdot x)}$.
\zadStop
\rozwStart{Patryk Wirkus}{}
$$\lim\limits_{x\to\ 0}\frac{31 \cdot x}{tan(19 \cdot x)}=\lim\limits_{x\to\ 0}\frac{31 \cdot x \cdot cos(19 \cdot x)}{sin(19 \cdot x)}=\lim\limits_{x\to\ 0}\frac{31 \cdot cos(19 \cdot x)}{\frac{sin(19 \cdot x)}{x}}=\lim\limits_{x\to\ 0}\frac{31 \cdot cos(19 \cdot x)}{19 \cdot \frac{sin(19 \cdot x)}{19 \cdot x}} = \frac{31}{19}$$
\rozwStop
\odpStart
$\frac{31}{19}$
\odpStop
\testStart
A.$\frac{31}{19}$
B.$\infty$
C.$-\infty$
D.$0$
E.$-\frac{31}{19}$
F.$\frac{19}{31}$
G.$-\frac{19}{31}$
H.$19$
I.$31$
\testStop
\kluczStart
A
\kluczStop



\zadStart{Przykład z Wikieł P 4.3a moja wersja nr 641}


Obliczyć granicę funkcji $\lim\limits_{x\to\ 0}\frac{31 \cdot x}{tan(20 \cdot x)}$.
\zadStop
\rozwStart{Patryk Wirkus}{}
$$\lim\limits_{x\to\ 0}\frac{31 \cdot x}{tan(20 \cdot x)}=\lim\limits_{x\to\ 0}\frac{31 \cdot x \cdot cos(20 \cdot x)}{sin(20 \cdot x)}=\lim\limits_{x\to\ 0}\frac{31 \cdot cos(20 \cdot x)}{\frac{sin(20 \cdot x)}{x}}=\lim\limits_{x\to\ 0}\frac{31 \cdot cos(20 \cdot x)}{20 \cdot \frac{sin(20 \cdot x)}{20 \cdot x}} = \frac{31}{20}$$
\rozwStop
\odpStart
$\frac{31}{20}$
\odpStop
\testStart
A.$\frac{31}{20}$
B.$\infty$
C.$-\infty$
D.$0$
E.$-\frac{31}{20}$
F.$\frac{20}{31}$
G.$-\frac{20}{31}$
H.$20$
I.$31$
\testStop
\kluczStart
A
\kluczStop



\zadStart{Przykład z Wikieł P 4.3a moja wersja nr 642}


Obliczyć granicę funkcji $\lim\limits_{x\to\ 0}\frac{31 \cdot x}{tan(21 \cdot x)}$.
\zadStop
\rozwStart{Patryk Wirkus}{}
$$\lim\limits_{x\to\ 0}\frac{31 \cdot x}{tan(21 \cdot x)}=\lim\limits_{x\to\ 0}\frac{31 \cdot x \cdot cos(21 \cdot x)}{sin(21 \cdot x)}=\lim\limits_{x\to\ 0}\frac{31 \cdot cos(21 \cdot x)}{\frac{sin(21 \cdot x)}{x}}=\lim\limits_{x\to\ 0}\frac{31 \cdot cos(21 \cdot x)}{21 \cdot \frac{sin(21 \cdot x)}{21 \cdot x}} = \frac{31}{21}$$
\rozwStop
\odpStart
$\frac{31}{21}$
\odpStop
\testStart
A.$\frac{31}{21}$
B.$\infty$
C.$-\infty$
D.$0$
E.$-\frac{31}{21}$
F.$\frac{21}{31}$
G.$-\frac{21}{31}$
H.$21$
I.$31$
\testStop
\kluczStart
A
\kluczStop



\zadStart{Przykład z Wikieł P 4.3a moja wersja nr 643}


Obliczyć granicę funkcji $\lim\limits_{x\to\ 0}\frac{31 \cdot x}{tan(22 \cdot x)}$.
\zadStop
\rozwStart{Patryk Wirkus}{}
$$\lim\limits_{x\to\ 0}\frac{31 \cdot x}{tan(22 \cdot x)}=\lim\limits_{x\to\ 0}\frac{31 \cdot x \cdot cos(22 \cdot x)}{sin(22 \cdot x)}=\lim\limits_{x\to\ 0}\frac{31 \cdot cos(22 \cdot x)}{\frac{sin(22 \cdot x)}{x}}=\lim\limits_{x\to\ 0}\frac{31 \cdot cos(22 \cdot x)}{22 \cdot \frac{sin(22 \cdot x)}{22 \cdot x}} = \frac{31}{22}$$
\rozwStop
\odpStart
$\frac{31}{22}$
\odpStop
\testStart
A.$\frac{31}{22}$
B.$\infty$
C.$-\infty$
D.$0$
E.$-\frac{31}{22}$
F.$\frac{22}{31}$
G.$-\frac{22}{31}$
H.$22$
I.$31$
\testStop
\kluczStart
A
\kluczStop



\zadStart{Przykład z Wikieł P 4.3a moja wersja nr 644}


Obliczyć granicę funkcji $\lim\limits_{x\to\ 0}\frac{31 \cdot x}{tan(23 \cdot x)}$.
\zadStop
\rozwStart{Patryk Wirkus}{}
$$\lim\limits_{x\to\ 0}\frac{31 \cdot x}{tan(23 \cdot x)}=\lim\limits_{x\to\ 0}\frac{31 \cdot x \cdot cos(23 \cdot x)}{sin(23 \cdot x)}=\lim\limits_{x\to\ 0}\frac{31 \cdot cos(23 \cdot x)}{\frac{sin(23 \cdot x)}{x}}=\lim\limits_{x\to\ 0}\frac{31 \cdot cos(23 \cdot x)}{23 \cdot \frac{sin(23 \cdot x)}{23 \cdot x}} = \frac{31}{23}$$
\rozwStop
\odpStart
$\frac{31}{23}$
\odpStop
\testStart
A.$\frac{31}{23}$
B.$\infty$
C.$-\infty$
D.$0$
E.$-\frac{31}{23}$
F.$\frac{23}{31}$
G.$-\frac{23}{31}$
H.$23$
I.$31$
\testStop
\kluczStart
A
\kluczStop



\zadStart{Przykład z Wikieł P 4.3a moja wersja nr 645}


Obliczyć granicę funkcji $\lim\limits_{x\to\ 0}\frac{31 \cdot x}{tan(24 \cdot x)}$.
\zadStop
\rozwStart{Patryk Wirkus}{}
$$\lim\limits_{x\to\ 0}\frac{31 \cdot x}{tan(24 \cdot x)}=\lim\limits_{x\to\ 0}\frac{31 \cdot x \cdot cos(24 \cdot x)}{sin(24 \cdot x)}=\lim\limits_{x\to\ 0}\frac{31 \cdot cos(24 \cdot x)}{\frac{sin(24 \cdot x)}{x}}=\lim\limits_{x\to\ 0}\frac{31 \cdot cos(24 \cdot x)}{24 \cdot \frac{sin(24 \cdot x)}{24 \cdot x}} = \frac{31}{24}$$
\rozwStop
\odpStart
$\frac{31}{24}$
\odpStop
\testStart
A.$\frac{31}{24}$
B.$\infty$
C.$-\infty$
D.$0$
E.$-\frac{31}{24}$
F.$\frac{24}{31}$
G.$-\frac{24}{31}$
H.$24$
I.$31$
\testStop
\kluczStart
A
\kluczStop



\zadStart{Przykład z Wikieł P 4.3a moja wersja nr 646}


Obliczyć granicę funkcji $\lim\limits_{x\to\ 0}\frac{31 \cdot x}{tan(25 \cdot x)}$.
\zadStop
\rozwStart{Patryk Wirkus}{}
$$\lim\limits_{x\to\ 0}\frac{31 \cdot x}{tan(25 \cdot x)}=\lim\limits_{x\to\ 0}\frac{31 \cdot x \cdot cos(25 \cdot x)}{sin(25 \cdot x)}=\lim\limits_{x\to\ 0}\frac{31 \cdot cos(25 \cdot x)}{\frac{sin(25 \cdot x)}{x}}=\lim\limits_{x\to\ 0}\frac{31 \cdot cos(25 \cdot x)}{25 \cdot \frac{sin(25 \cdot x)}{25 \cdot x}} = \frac{31}{25}$$
\rozwStop
\odpStart
$\frac{31}{25}$
\odpStop
\testStart
A.$\frac{31}{25}$
B.$\infty$
C.$-\infty$
D.$0$
E.$-\frac{31}{25}$
F.$\frac{25}{31}$
G.$-\frac{25}{31}$
H.$25$
I.$31$
\testStop
\kluczStart
A
\kluczStop



\zadStart{Przykład z Wikieł P 4.3a moja wersja nr 647}


Obliczyć granicę funkcji $\lim\limits_{x\to\ 0}\frac{31 \cdot x}{tan(26 \cdot x)}$.
\zadStop
\rozwStart{Patryk Wirkus}{}
$$\lim\limits_{x\to\ 0}\frac{31 \cdot x}{tan(26 \cdot x)}=\lim\limits_{x\to\ 0}\frac{31 \cdot x \cdot cos(26 \cdot x)}{sin(26 \cdot x)}=\lim\limits_{x\to\ 0}\frac{31 \cdot cos(26 \cdot x)}{\frac{sin(26 \cdot x)}{x}}=\lim\limits_{x\to\ 0}\frac{31 \cdot cos(26 \cdot x)}{26 \cdot \frac{sin(26 \cdot x)}{26 \cdot x}} = \frac{31}{26}$$
\rozwStop
\odpStart
$\frac{31}{26}$
\odpStop
\testStart
A.$\frac{31}{26}$
B.$\infty$
C.$-\infty$
D.$0$
E.$-\frac{31}{26}$
F.$\frac{26}{31}$
G.$-\frac{26}{31}$
H.$26$
I.$31$
\testStop
\kluczStart
A
\kluczStop



\zadStart{Przykład z Wikieł P 4.3a moja wersja nr 648}


Obliczyć granicę funkcji $\lim\limits_{x\to\ 0}\frac{31 \cdot x}{tan(27 \cdot x)}$.
\zadStop
\rozwStart{Patryk Wirkus}{}
$$\lim\limits_{x\to\ 0}\frac{31 \cdot x}{tan(27 \cdot x)}=\lim\limits_{x\to\ 0}\frac{31 \cdot x \cdot cos(27 \cdot x)}{sin(27 \cdot x)}=\lim\limits_{x\to\ 0}\frac{31 \cdot cos(27 \cdot x)}{\frac{sin(27 \cdot x)}{x}}=\lim\limits_{x\to\ 0}\frac{31 \cdot cos(27 \cdot x)}{27 \cdot \frac{sin(27 \cdot x)}{27 \cdot x}} = \frac{31}{27}$$
\rozwStop
\odpStart
$\frac{31}{27}$
\odpStop
\testStart
A.$\frac{31}{27}$
B.$\infty$
C.$-\infty$
D.$0$
E.$-\frac{31}{27}$
F.$\frac{27}{31}$
G.$-\frac{27}{31}$
H.$27$
I.$31$
\testStop
\kluczStart
A
\kluczStop



\zadStart{Przykład z Wikieł P 4.3a moja wersja nr 649}


Obliczyć granicę funkcji $\lim\limits_{x\to\ 0}\frac{31 \cdot x}{tan(28 \cdot x)}$.
\zadStop
\rozwStart{Patryk Wirkus}{}
$$\lim\limits_{x\to\ 0}\frac{31 \cdot x}{tan(28 \cdot x)}=\lim\limits_{x\to\ 0}\frac{31 \cdot x \cdot cos(28 \cdot x)}{sin(28 \cdot x)}=\lim\limits_{x\to\ 0}\frac{31 \cdot cos(28 \cdot x)}{\frac{sin(28 \cdot x)}{x}}=\lim\limits_{x\to\ 0}\frac{31 \cdot cos(28 \cdot x)}{28 \cdot \frac{sin(28 \cdot x)}{28 \cdot x}} = \frac{31}{28}$$
\rozwStop
\odpStart
$\frac{31}{28}$
\odpStop
\testStart
A.$\frac{31}{28}$
B.$\infty$
C.$-\infty$
D.$0$
E.$-\frac{31}{28}$
F.$\frac{28}{31}$
G.$-\frac{28}{31}$
H.$28$
I.$31$
\testStop
\kluczStart
A
\kluczStop



\zadStart{Przykład z Wikieł P 4.3a moja wersja nr 650}


Obliczyć granicę funkcji $\lim\limits_{x\to\ 0}\frac{31 \cdot x}{tan(29 \cdot x)}$.
\zadStop
\rozwStart{Patryk Wirkus}{}
$$\lim\limits_{x\to\ 0}\frac{31 \cdot x}{tan(29 \cdot x)}=\lim\limits_{x\to\ 0}\frac{31 \cdot x \cdot cos(29 \cdot x)}{sin(29 \cdot x)}=\lim\limits_{x\to\ 0}\frac{31 \cdot cos(29 \cdot x)}{\frac{sin(29 \cdot x)}{x}}=\lim\limits_{x\to\ 0}\frac{31 \cdot cos(29 \cdot x)}{29 \cdot \frac{sin(29 \cdot x)}{29 \cdot x}} = \frac{31}{29}$$
\rozwStop
\odpStart
$\frac{31}{29}$
\odpStop
\testStart
A.$\frac{31}{29}$
B.$\infty$
C.$-\infty$
D.$0$
E.$-\frac{31}{29}$
F.$\frac{29}{31}$
G.$-\frac{29}{31}$
H.$29$
I.$31$
\testStop
\kluczStart
A
\kluczStop



\zadStart{Przykład z Wikieł P 4.3a moja wersja nr 651}


Obliczyć granicę funkcji $\lim\limits_{x\to\ 0}\frac{31 \cdot x}{tan(30 \cdot x)}$.
\zadStop
\rozwStart{Patryk Wirkus}{}
$$\lim\limits_{x\to\ 0}\frac{31 \cdot x}{tan(30 \cdot x)}=\lim\limits_{x\to\ 0}\frac{31 \cdot x \cdot cos(30 \cdot x)}{sin(30 \cdot x)}=\lim\limits_{x\to\ 0}\frac{31 \cdot cos(30 \cdot x)}{\frac{sin(30 \cdot x)}{x}}=\lim\limits_{x\to\ 0}\frac{31 \cdot cos(30 \cdot x)}{30 \cdot \frac{sin(30 \cdot x)}{30 \cdot x}} = \frac{31}{30}$$
\rozwStop
\odpStart
$\frac{31}{30}$
\odpStop
\testStart
A.$\frac{31}{30}$
B.$\infty$
C.$-\infty$
D.$0$
E.$-\frac{31}{30}$
F.$\frac{30}{31}$
G.$-\frac{30}{31}$
H.$30$
I.$31$
\testStop
\kluczStart
A
\kluczStop



\zadStart{Przykład z Wikieł P 4.3a moja wersja nr 652}


Obliczyć granicę funkcji $\lim\limits_{x\to\ 0}\frac{31 \cdot x}{tan(32 \cdot x)}$.
\zadStop
\rozwStart{Patryk Wirkus}{}
$$\lim\limits_{x\to\ 0}\frac{31 \cdot x}{tan(32 \cdot x)}=\lim\limits_{x\to\ 0}\frac{31 \cdot x \cdot cos(32 \cdot x)}{sin(32 \cdot x)}=\lim\limits_{x\to\ 0}\frac{31 \cdot cos(32 \cdot x)}{\frac{sin(32 \cdot x)}{x}}=\lim\limits_{x\to\ 0}\frac{31 \cdot cos(32 \cdot x)}{32 \cdot \frac{sin(32 \cdot x)}{32 \cdot x}} = \frac{31}{32}$$
\rozwStop
\odpStart
$\frac{31}{32}$
\odpStop
\testStart
A.$\frac{31}{32}$
B.$\infty$
C.$-\infty$
D.$0$
E.$-\frac{31}{32}$
F.$\frac{32}{31}$
G.$-\frac{32}{31}$
H.$32$
I.$31$
\testStop
\kluczStart
A
\kluczStop



\zadStart{Przykład z Wikieł P 4.3a moja wersja nr 653}


Obliczyć granicę funkcji $\lim\limits_{x\to\ 0}\frac{31 \cdot x}{tan(33 \cdot x)}$.
\zadStop
\rozwStart{Patryk Wirkus}{}
$$\lim\limits_{x\to\ 0}\frac{31 \cdot x}{tan(33 \cdot x)}=\lim\limits_{x\to\ 0}\frac{31 \cdot x \cdot cos(33 \cdot x)}{sin(33 \cdot x)}=\lim\limits_{x\to\ 0}\frac{31 \cdot cos(33 \cdot x)}{\frac{sin(33 \cdot x)}{x}}=\lim\limits_{x\to\ 0}\frac{31 \cdot cos(33 \cdot x)}{33 \cdot \frac{sin(33 \cdot x)}{33 \cdot x}} = \frac{31}{33}$$
\rozwStop
\odpStart
$\frac{31}{33}$
\odpStop
\testStart
A.$\frac{31}{33}$
B.$\infty$
C.$-\infty$
D.$0$
E.$-\frac{31}{33}$
F.$\frac{33}{31}$
G.$-\frac{33}{31}$
H.$33$
I.$31$
\testStop
\kluczStart
A
\kluczStop



\zadStart{Przykład z Wikieł P 4.3a moja wersja nr 654}


Obliczyć granicę funkcji $\lim\limits_{x\to\ 0}\frac{31 \cdot x}{tan(34 \cdot x)}$.
\zadStop
\rozwStart{Patryk Wirkus}{}
$$\lim\limits_{x\to\ 0}\frac{31 \cdot x}{tan(34 \cdot x)}=\lim\limits_{x\to\ 0}\frac{31 \cdot x \cdot cos(34 \cdot x)}{sin(34 \cdot x)}=\lim\limits_{x\to\ 0}\frac{31 \cdot cos(34 \cdot x)}{\frac{sin(34 \cdot x)}{x}}=\lim\limits_{x\to\ 0}\frac{31 \cdot cos(34 \cdot x)}{34 \cdot \frac{sin(34 \cdot x)}{34 \cdot x}} = \frac{31}{34}$$
\rozwStop
\odpStart
$\frac{31}{34}$
\odpStop
\testStart
A.$\frac{31}{34}$
B.$\infty$
C.$-\infty$
D.$0$
E.$-\frac{31}{34}$
F.$\frac{34}{31}$
G.$-\frac{34}{31}$
H.$34$
I.$31$
\testStop
\kluczStart
A
\kluczStop



\zadStart{Przykład z Wikieł P 4.3a moja wersja nr 655}


Obliczyć granicę funkcji $\lim\limits_{x\to\ 0}\frac{31 \cdot x}{tan(35 \cdot x)}$.
\zadStop
\rozwStart{Patryk Wirkus}{}
$$\lim\limits_{x\to\ 0}\frac{31 \cdot x}{tan(35 \cdot x)}=\lim\limits_{x\to\ 0}\frac{31 \cdot x \cdot cos(35 \cdot x)}{sin(35 \cdot x)}=\lim\limits_{x\to\ 0}\frac{31 \cdot cos(35 \cdot x)}{\frac{sin(35 \cdot x)}{x}}=\lim\limits_{x\to\ 0}\frac{31 \cdot cos(35 \cdot x)}{35 \cdot \frac{sin(35 \cdot x)}{35 \cdot x}} = \frac{31}{35}$$
\rozwStop
\odpStart
$\frac{31}{35}$
\odpStop
\testStart
A.$\frac{31}{35}$
B.$\infty$
C.$-\infty$
D.$0$
E.$-\frac{31}{35}$
F.$\frac{35}{31}$
G.$-\frac{35}{31}$
H.$35$
I.$31$
\testStop
\kluczStart
A
\kluczStop



\zadStart{Przykład z Wikieł P 4.3a moja wersja nr 656}


Obliczyć granicę funkcji $\lim\limits_{x\to\ 0}\frac{31 \cdot x}{tan(36 \cdot x)}$.
\zadStop
\rozwStart{Patryk Wirkus}{}
$$\lim\limits_{x\to\ 0}\frac{31 \cdot x}{tan(36 \cdot x)}=\lim\limits_{x\to\ 0}\frac{31 \cdot x \cdot cos(36 \cdot x)}{sin(36 \cdot x)}=\lim\limits_{x\to\ 0}\frac{31 \cdot cos(36 \cdot x)}{\frac{sin(36 \cdot x)}{x}}=\lim\limits_{x\to\ 0}\frac{31 \cdot cos(36 \cdot x)}{36 \cdot \frac{sin(36 \cdot x)}{36 \cdot x}} = \frac{31}{36}$$
\rozwStop
\odpStart
$\frac{31}{36}$
\odpStop
\testStart
A.$\frac{31}{36}$
B.$\infty$
C.$-\infty$
D.$0$
E.$-\frac{31}{36}$
F.$\frac{36}{31}$
G.$-\frac{36}{31}$
H.$36$
I.$31$
\testStop
\kluczStart
A
\kluczStop



\zadStart{Przykład z Wikieł P 4.3a moja wersja nr 657}


Obliczyć granicę funkcji $\lim\limits_{x\to\ 0}\frac{31 \cdot x}{tan(37 \cdot x)}$.
\zadStop
\rozwStart{Patryk Wirkus}{}
$$\lim\limits_{x\to\ 0}\frac{31 \cdot x}{tan(37 \cdot x)}=\lim\limits_{x\to\ 0}\frac{31 \cdot x \cdot cos(37 \cdot x)}{sin(37 \cdot x)}=\lim\limits_{x\to\ 0}\frac{31 \cdot cos(37 \cdot x)}{\frac{sin(37 \cdot x)}{x}}=\lim\limits_{x\to\ 0}\frac{31 \cdot cos(37 \cdot x)}{37 \cdot \frac{sin(37 \cdot x)}{37 \cdot x}} = \frac{31}{37}$$
\rozwStop
\odpStart
$\frac{31}{37}$
\odpStop
\testStart
A.$\frac{31}{37}$
B.$\infty$
C.$-\infty$
D.$0$
E.$-\frac{31}{37}$
F.$\frac{37}{31}$
G.$-\frac{37}{31}$
H.$37$
I.$31$
\testStop
\kluczStart
A
\kluczStop



\zadStart{Przykład z Wikieł P 4.3a moja wersja nr 658}


Obliczyć granicę funkcji $\lim\limits_{x\to\ 0}\frac{31 \cdot x}{tan(38 \cdot x)}$.
\zadStop
\rozwStart{Patryk Wirkus}{}
$$\lim\limits_{x\to\ 0}\frac{31 \cdot x}{tan(38 \cdot x)}=\lim\limits_{x\to\ 0}\frac{31 \cdot x \cdot cos(38 \cdot x)}{sin(38 \cdot x)}=\lim\limits_{x\to\ 0}\frac{31 \cdot cos(38 \cdot x)}{\frac{sin(38 \cdot x)}{x}}=\lim\limits_{x\to\ 0}\frac{31 \cdot cos(38 \cdot x)}{38 \cdot \frac{sin(38 \cdot x)}{38 \cdot x}} = \frac{31}{38}$$
\rozwStop
\odpStart
$\frac{31}{38}$
\odpStop
\testStart
A.$\frac{31}{38}$
B.$\infty$
C.$-\infty$
D.$0$
E.$-\frac{31}{38}$
F.$\frac{38}{31}$
G.$-\frac{38}{31}$
H.$38$
I.$31$
\testStop
\kluczStart
A
\kluczStop



\zadStart{Przykład z Wikieł P 4.3a moja wersja nr 659}


Obliczyć granicę funkcji $\lim\limits_{x\to\ 0}\frac{31 \cdot x}{tan(39 \cdot x)}$.
\zadStop
\rozwStart{Patryk Wirkus}{}
$$\lim\limits_{x\to\ 0}\frac{31 \cdot x}{tan(39 \cdot x)}=\lim\limits_{x\to\ 0}\frac{31 \cdot x \cdot cos(39 \cdot x)}{sin(39 \cdot x)}=\lim\limits_{x\to\ 0}\frac{31 \cdot cos(39 \cdot x)}{\frac{sin(39 \cdot x)}{x}}=\lim\limits_{x\to\ 0}\frac{31 \cdot cos(39 \cdot x)}{39 \cdot \frac{sin(39 \cdot x)}{39 \cdot x}} = \frac{31}{39}$$
\rozwStop
\odpStart
$\frac{31}{39}$
\odpStop
\testStart
A.$\frac{31}{39}$
B.$\infty$
C.$-\infty$
D.$0$
E.$-\frac{31}{39}$
F.$\frac{39}{31}$
G.$-\frac{39}{31}$
H.$39$
I.$31$
\testStop
\kluczStart
A
\kluczStop



\zadStart{Przykład z Wikieł P 4.3a moja wersja nr 660}


Obliczyć granicę funkcji $\lim\limits_{x\to\ 0}\frac{31 \cdot x}{tan(40 \cdot x)}$.
\zadStop
\rozwStart{Patryk Wirkus}{}
$$\lim\limits_{x\to\ 0}\frac{31 \cdot x}{tan(40 \cdot x)}=\lim\limits_{x\to\ 0}\frac{31 \cdot x \cdot cos(40 \cdot x)}{sin(40 \cdot x)}=\lim\limits_{x\to\ 0}\frac{31 \cdot cos(40 \cdot x)}{\frac{sin(40 \cdot x)}{x}}=\lim\limits_{x\to\ 0}\frac{31 \cdot cos(40 \cdot x)}{40 \cdot \frac{sin(40 \cdot x)}{40 \cdot x}} = \frac{31}{40}$$
\rozwStop
\odpStart
$\frac{31}{40}$
\odpStop
\testStart
A.$\frac{31}{40}$
B.$\infty$
C.$-\infty$
D.$0$
E.$-\frac{31}{40}$
F.$\frac{40}{31}$
G.$-\frac{40}{31}$
H.$40$
I.$31$
\testStop
\kluczStart
A
\kluczStop



\zadStart{Przykład z Wikieł P 4.3a moja wersja nr 661}


Obliczyć granicę funkcji $\lim\limits_{x\to\ 0}\frac{32 \cdot x}{tan(3 \cdot x)}$.
\zadStop
\rozwStart{Patryk Wirkus}{}
$$\lim\limits_{x\to\ 0}\frac{32 \cdot x}{tan(3 \cdot x)}=\lim\limits_{x\to\ 0}\frac{32 \cdot x \cdot cos(3 \cdot x)}{sin(3 \cdot x)}=\lim\limits_{x\to\ 0}\frac{32 \cdot cos(3 \cdot x)}{\frac{sin(3 \cdot x)}{x}}=\lim\limits_{x\to\ 0}\frac{32 \cdot cos(3 \cdot x)}{3 \cdot \frac{sin(3 \cdot x)}{3 \cdot x}} = \frac{32}{3}$$
\rozwStop
\odpStart
$\frac{32}{3}$
\odpStop
\testStart
A.$\frac{32}{3}$
B.$\infty$
C.$-\infty$
D.$0$
E.$-\frac{32}{3}$
F.$\frac{3}{32}$
G.$-\frac{3}{32}$
H.$3$
I.$32$
\testStop
\kluczStart
A
\kluczStop



\zadStart{Przykład z Wikieł P 4.3a moja wersja nr 662}


Obliczyć granicę funkcji $\lim\limits_{x\to\ 0}\frac{32 \cdot x}{tan(5 \cdot x)}$.
\zadStop
\rozwStart{Patryk Wirkus}{}
$$\lim\limits_{x\to\ 0}\frac{32 \cdot x}{tan(5 \cdot x)}=\lim\limits_{x\to\ 0}\frac{32 \cdot x \cdot cos(5 \cdot x)}{sin(5 \cdot x)}=\lim\limits_{x\to\ 0}\frac{32 \cdot cos(5 \cdot x)}{\frac{sin(5 \cdot x)}{x}}=\lim\limits_{x\to\ 0}\frac{32 \cdot cos(5 \cdot x)}{5 \cdot \frac{sin(5 \cdot x)}{5 \cdot x}} = \frac{32}{5}$$
\rozwStop
\odpStart
$\frac{32}{5}$
\odpStop
\testStart
A.$\frac{32}{5}$
B.$\infty$
C.$-\infty$
D.$0$
E.$-\frac{32}{5}$
F.$\frac{5}{32}$
G.$-\frac{5}{32}$
H.$5$
I.$32$
\testStop
\kluczStart
A
\kluczStop



\zadStart{Przykład z Wikieł P 4.3a moja wersja nr 663}


Obliczyć granicę funkcji $\lim\limits_{x\to\ 0}\frac{32 \cdot x}{tan(7 \cdot x)}$.
\zadStop
\rozwStart{Patryk Wirkus}{}
$$\lim\limits_{x\to\ 0}\frac{32 \cdot x}{tan(7 \cdot x)}=\lim\limits_{x\to\ 0}\frac{32 \cdot x \cdot cos(7 \cdot x)}{sin(7 \cdot x)}=\lim\limits_{x\to\ 0}\frac{32 \cdot cos(7 \cdot x)}{\frac{sin(7 \cdot x)}{x}}=\lim\limits_{x\to\ 0}\frac{32 \cdot cos(7 \cdot x)}{7 \cdot \frac{sin(7 \cdot x)}{7 \cdot x}} = \frac{32}{7}$$
\rozwStop
\odpStart
$\frac{32}{7}$
\odpStop
\testStart
A.$\frac{32}{7}$
B.$\infty$
C.$-\infty$
D.$0$
E.$-\frac{32}{7}$
F.$\frac{7}{32}$
G.$-\frac{7}{32}$
H.$7$
I.$32$
\testStop
\kluczStart
A
\kluczStop



\zadStart{Przykład z Wikieł P 4.3a moja wersja nr 664}


Obliczyć granicę funkcji $\lim\limits_{x\to\ 0}\frac{32 \cdot x}{tan(9 \cdot x)}$.
\zadStop
\rozwStart{Patryk Wirkus}{}
$$\lim\limits_{x\to\ 0}\frac{32 \cdot x}{tan(9 \cdot x)}=\lim\limits_{x\to\ 0}\frac{32 \cdot x \cdot cos(9 \cdot x)}{sin(9 \cdot x)}=\lim\limits_{x\to\ 0}\frac{32 \cdot cos(9 \cdot x)}{\frac{sin(9 \cdot x)}{x}}=\lim\limits_{x\to\ 0}\frac{32 \cdot cos(9 \cdot x)}{9 \cdot \frac{sin(9 \cdot x)}{9 \cdot x}} = \frac{32}{9}$$
\rozwStop
\odpStart
$\frac{32}{9}$
\odpStop
\testStart
A.$\frac{32}{9}$
B.$\infty$
C.$-\infty$
D.$0$
E.$-\frac{32}{9}$
F.$\frac{9}{32}$
G.$-\frac{9}{32}$
H.$9$
I.$32$
\testStop
\kluczStart
A
\kluczStop



\zadStart{Przykład z Wikieł P 4.3a moja wersja nr 665}


Obliczyć granicę funkcji $\lim\limits_{x\to\ 0}\frac{32 \cdot x}{tan(11 \cdot x)}$.
\zadStop
\rozwStart{Patryk Wirkus}{}
$$\lim\limits_{x\to\ 0}\frac{32 \cdot x}{tan(11 \cdot x)}=\lim\limits_{x\to\ 0}\frac{32 \cdot x \cdot cos(11 \cdot x)}{sin(11 \cdot x)}=\lim\limits_{x\to\ 0}\frac{32 \cdot cos(11 \cdot x)}{\frac{sin(11 \cdot x)}{x}}=\lim\limits_{x\to\ 0}\frac{32 \cdot cos(11 \cdot x)}{11 \cdot \frac{sin(11 \cdot x)}{11 \cdot x}} = \frac{32}{11}$$
\rozwStop
\odpStart
$\frac{32}{11}$
\odpStop
\testStart
A.$\frac{32}{11}$
B.$\infty$
C.$-\infty$
D.$0$
E.$-\frac{32}{11}$
F.$\frac{11}{32}$
G.$-\frac{11}{32}$
H.$11$
I.$32$
\testStop
\kluczStart
A
\kluczStop



\zadStart{Przykład z Wikieł P 4.3a moja wersja nr 666}


Obliczyć granicę funkcji $\lim\limits_{x\to\ 0}\frac{32 \cdot x}{tan(13 \cdot x)}$.
\zadStop
\rozwStart{Patryk Wirkus}{}
$$\lim\limits_{x\to\ 0}\frac{32 \cdot x}{tan(13 \cdot x)}=\lim\limits_{x\to\ 0}\frac{32 \cdot x \cdot cos(13 \cdot x)}{sin(13 \cdot x)}=\lim\limits_{x\to\ 0}\frac{32 \cdot cos(13 \cdot x)}{\frac{sin(13 \cdot x)}{x}}=\lim\limits_{x\to\ 0}\frac{32 \cdot cos(13 \cdot x)}{13 \cdot \frac{sin(13 \cdot x)}{13 \cdot x}} = \frac{32}{13}$$
\rozwStop
\odpStart
$\frac{32}{13}$
\odpStop
\testStart
A.$\frac{32}{13}$
B.$\infty$
C.$-\infty$
D.$0$
E.$-\frac{32}{13}$
F.$\frac{13}{32}$
G.$-\frac{13}{32}$
H.$13$
I.$32$
\testStop
\kluczStart
A
\kluczStop



\zadStart{Przykład z Wikieł P 4.3a moja wersja nr 667}


Obliczyć granicę funkcji $\lim\limits_{x\to\ 0}\frac{32 \cdot x}{tan(15 \cdot x)}$.
\zadStop
\rozwStart{Patryk Wirkus}{}
$$\lim\limits_{x\to\ 0}\frac{32 \cdot x}{tan(15 \cdot x)}=\lim\limits_{x\to\ 0}\frac{32 \cdot x \cdot cos(15 \cdot x)}{sin(15 \cdot x)}=\lim\limits_{x\to\ 0}\frac{32 \cdot cos(15 \cdot x)}{\frac{sin(15 \cdot x)}{x}}=\lim\limits_{x\to\ 0}\frac{32 \cdot cos(15 \cdot x)}{15 \cdot \frac{sin(15 \cdot x)}{15 \cdot x}} = \frac{32}{15}$$
\rozwStop
\odpStart
$\frac{32}{15}$
\odpStop
\testStart
A.$\frac{32}{15}$
B.$\infty$
C.$-\infty$
D.$0$
E.$-\frac{32}{15}$
F.$\frac{15}{32}$
G.$-\frac{15}{32}$
H.$15$
I.$32$
\testStop
\kluczStart
A
\kluczStop



\zadStart{Przykład z Wikieł P 4.3a moja wersja nr 668}


Obliczyć granicę funkcji $\lim\limits_{x\to\ 0}\frac{32 \cdot x}{tan(17 \cdot x)}$.
\zadStop
\rozwStart{Patryk Wirkus}{}
$$\lim\limits_{x\to\ 0}\frac{32 \cdot x}{tan(17 \cdot x)}=\lim\limits_{x\to\ 0}\frac{32 \cdot x \cdot cos(17 \cdot x)}{sin(17 \cdot x)}=\lim\limits_{x\to\ 0}\frac{32 \cdot cos(17 \cdot x)}{\frac{sin(17 \cdot x)}{x}}=\lim\limits_{x\to\ 0}\frac{32 \cdot cos(17 \cdot x)}{17 \cdot \frac{sin(17 \cdot x)}{17 \cdot x}} = \frac{32}{17}$$
\rozwStop
\odpStart
$\frac{32}{17}$
\odpStop
\testStart
A.$\frac{32}{17}$
B.$\infty$
C.$-\infty$
D.$0$
E.$-\frac{32}{17}$
F.$\frac{17}{32}$
G.$-\frac{17}{32}$
H.$17$
I.$32$
\testStop
\kluczStart
A
\kluczStop



\zadStart{Przykład z Wikieł P 4.3a moja wersja nr 669}


Obliczyć granicę funkcji $\lim\limits_{x\to\ 0}\frac{32 \cdot x}{tan(19 \cdot x)}$.
\zadStop
\rozwStart{Patryk Wirkus}{}
$$\lim\limits_{x\to\ 0}\frac{32 \cdot x}{tan(19 \cdot x)}=\lim\limits_{x\to\ 0}\frac{32 \cdot x \cdot cos(19 \cdot x)}{sin(19 \cdot x)}=\lim\limits_{x\to\ 0}\frac{32 \cdot cos(19 \cdot x)}{\frac{sin(19 \cdot x)}{x}}=\lim\limits_{x\to\ 0}\frac{32 \cdot cos(19 \cdot x)}{19 \cdot \frac{sin(19 \cdot x)}{19 \cdot x}} = \frac{32}{19}$$
\rozwStop
\odpStart
$\frac{32}{19}$
\odpStop
\testStart
A.$\frac{32}{19}$
B.$\infty$
C.$-\infty$
D.$0$
E.$-\frac{32}{19}$
F.$\frac{19}{32}$
G.$-\frac{19}{32}$
H.$19$
I.$32$
\testStop
\kluczStart
A
\kluczStop



\zadStart{Przykład z Wikieł P 4.3a moja wersja nr 670}


Obliczyć granicę funkcji $\lim\limits_{x\to\ 0}\frac{32 \cdot x}{tan(21 \cdot x)}$.
\zadStop
\rozwStart{Patryk Wirkus}{}
$$\lim\limits_{x\to\ 0}\frac{32 \cdot x}{tan(21 \cdot x)}=\lim\limits_{x\to\ 0}\frac{32 \cdot x \cdot cos(21 \cdot x)}{sin(21 \cdot x)}=\lim\limits_{x\to\ 0}\frac{32 \cdot cos(21 \cdot x)}{\frac{sin(21 \cdot x)}{x}}=\lim\limits_{x\to\ 0}\frac{32 \cdot cos(21 \cdot x)}{21 \cdot \frac{sin(21 \cdot x)}{21 \cdot x}} = \frac{32}{21}$$
\rozwStop
\odpStart
$\frac{32}{21}$
\odpStop
\testStart
A.$\frac{32}{21}$
B.$\infty$
C.$-\infty$
D.$0$
E.$-\frac{32}{21}$
F.$\frac{21}{32}$
G.$-\frac{21}{32}$
H.$21$
I.$32$
\testStop
\kluczStart
A
\kluczStop



\zadStart{Przykład z Wikieł P 4.3a moja wersja nr 671}


Obliczyć granicę funkcji $\lim\limits_{x\to\ 0}\frac{32 \cdot x}{tan(23 \cdot x)}$.
\zadStop
\rozwStart{Patryk Wirkus}{}
$$\lim\limits_{x\to\ 0}\frac{32 \cdot x}{tan(23 \cdot x)}=\lim\limits_{x\to\ 0}\frac{32 \cdot x \cdot cos(23 \cdot x)}{sin(23 \cdot x)}=\lim\limits_{x\to\ 0}\frac{32 \cdot cos(23 \cdot x)}{\frac{sin(23 \cdot x)}{x}}=\lim\limits_{x\to\ 0}\frac{32 \cdot cos(23 \cdot x)}{23 \cdot \frac{sin(23 \cdot x)}{23 \cdot x}} = \frac{32}{23}$$
\rozwStop
\odpStart
$\frac{32}{23}$
\odpStop
\testStart
A.$\frac{32}{23}$
B.$\infty$
C.$-\infty$
D.$0$
E.$-\frac{32}{23}$
F.$\frac{23}{32}$
G.$-\frac{23}{32}$
H.$23$
I.$32$
\testStop
\kluczStart
A
\kluczStop



\zadStart{Przykład z Wikieł P 4.3a moja wersja nr 672}


Obliczyć granicę funkcji $\lim\limits_{x\to\ 0}\frac{32 \cdot x}{tan(25 \cdot x)}$.
\zadStop
\rozwStart{Patryk Wirkus}{}
$$\lim\limits_{x\to\ 0}\frac{32 \cdot x}{tan(25 \cdot x)}=\lim\limits_{x\to\ 0}\frac{32 \cdot x \cdot cos(25 \cdot x)}{sin(25 \cdot x)}=\lim\limits_{x\to\ 0}\frac{32 \cdot cos(25 \cdot x)}{\frac{sin(25 \cdot x)}{x}}=\lim\limits_{x\to\ 0}\frac{32 \cdot cos(25 \cdot x)}{25 \cdot \frac{sin(25 \cdot x)}{25 \cdot x}} = \frac{32}{25}$$
\rozwStop
\odpStart
$\frac{32}{25}$
\odpStop
\testStart
A.$\frac{32}{25}$
B.$\infty$
C.$-\infty$
D.$0$
E.$-\frac{32}{25}$
F.$\frac{25}{32}$
G.$-\frac{25}{32}$
H.$25$
I.$32$
\testStop
\kluczStart
A
\kluczStop



\zadStart{Przykład z Wikieł P 4.3a moja wersja nr 673}


Obliczyć granicę funkcji $\lim\limits_{x\to\ 0}\frac{32 \cdot x}{tan(27 \cdot x)}$.
\zadStop
\rozwStart{Patryk Wirkus}{}
$$\lim\limits_{x\to\ 0}\frac{32 \cdot x}{tan(27 \cdot x)}=\lim\limits_{x\to\ 0}\frac{32 \cdot x \cdot cos(27 \cdot x)}{sin(27 \cdot x)}=\lim\limits_{x\to\ 0}\frac{32 \cdot cos(27 \cdot x)}{\frac{sin(27 \cdot x)}{x}}=\lim\limits_{x\to\ 0}\frac{32 \cdot cos(27 \cdot x)}{27 \cdot \frac{sin(27 \cdot x)}{27 \cdot x}} = \frac{32}{27}$$
\rozwStop
\odpStart
$\frac{32}{27}$
\odpStop
\testStart
A.$\frac{32}{27}$
B.$\infty$
C.$-\infty$
D.$0$
E.$-\frac{32}{27}$
F.$\frac{27}{32}$
G.$-\frac{27}{32}$
H.$27$
I.$32$
\testStop
\kluczStart
A
\kluczStop



\zadStart{Przykład z Wikieł P 4.3a moja wersja nr 674}


Obliczyć granicę funkcji $\lim\limits_{x\to\ 0}\frac{32 \cdot x}{tan(29 \cdot x)}$.
\zadStop
\rozwStart{Patryk Wirkus}{}
$$\lim\limits_{x\to\ 0}\frac{32 \cdot x}{tan(29 \cdot x)}=\lim\limits_{x\to\ 0}\frac{32 \cdot x \cdot cos(29 \cdot x)}{sin(29 \cdot x)}=\lim\limits_{x\to\ 0}\frac{32 \cdot cos(29 \cdot x)}{\frac{sin(29 \cdot x)}{x}}=\lim\limits_{x\to\ 0}\frac{32 \cdot cos(29 \cdot x)}{29 \cdot \frac{sin(29 \cdot x)}{29 \cdot x}} = \frac{32}{29}$$
\rozwStop
\odpStart
$\frac{32}{29}$
\odpStop
\testStart
A.$\frac{32}{29}$
B.$\infty$
C.$-\infty$
D.$0$
E.$-\frac{32}{29}$
F.$\frac{29}{32}$
G.$-\frac{29}{32}$
H.$29$
I.$32$
\testStop
\kluczStart
A
\kluczStop



\zadStart{Przykład z Wikieł P 4.3a moja wersja nr 675}


Obliczyć granicę funkcji $\lim\limits_{x\to\ 0}\frac{32 \cdot x}{tan(31 \cdot x)}$.
\zadStop
\rozwStart{Patryk Wirkus}{}
$$\lim\limits_{x\to\ 0}\frac{32 \cdot x}{tan(31 \cdot x)}=\lim\limits_{x\to\ 0}\frac{32 \cdot x \cdot cos(31 \cdot x)}{sin(31 \cdot x)}=\lim\limits_{x\to\ 0}\frac{32 \cdot cos(31 \cdot x)}{\frac{sin(31 \cdot x)}{x}}=\lim\limits_{x\to\ 0}\frac{32 \cdot cos(31 \cdot x)}{31 \cdot \frac{sin(31 \cdot x)}{31 \cdot x}} = \frac{32}{31}$$
\rozwStop
\odpStart
$\frac{32}{31}$
\odpStop
\testStart
A.$\frac{32}{31}$
B.$\infty$
C.$-\infty$
D.$0$
E.$-\frac{32}{31}$
F.$\frac{31}{32}$
G.$-\frac{31}{32}$
H.$31$
I.$32$
\testStop
\kluczStart
A
\kluczStop



\zadStart{Przykład z Wikieł P 4.3a moja wersja nr 676}


Obliczyć granicę funkcji $\lim\limits_{x\to\ 0}\frac{32 \cdot x}{tan(33 \cdot x)}$.
\zadStop
\rozwStart{Patryk Wirkus}{}
$$\lim\limits_{x\to\ 0}\frac{32 \cdot x}{tan(33 \cdot x)}=\lim\limits_{x\to\ 0}\frac{32 \cdot x \cdot cos(33 \cdot x)}{sin(33 \cdot x)}=\lim\limits_{x\to\ 0}\frac{32 \cdot cos(33 \cdot x)}{\frac{sin(33 \cdot x)}{x}}=\lim\limits_{x\to\ 0}\frac{32 \cdot cos(33 \cdot x)}{33 \cdot \frac{sin(33 \cdot x)}{33 \cdot x}} = \frac{32}{33}$$
\rozwStop
\odpStart
$\frac{32}{33}$
\odpStop
\testStart
A.$\frac{32}{33}$
B.$\infty$
C.$-\infty$
D.$0$
E.$-\frac{32}{33}$
F.$\frac{33}{32}$
G.$-\frac{33}{32}$
H.$33$
I.$32$
\testStop
\kluczStart
A
\kluczStop



\zadStart{Przykład z Wikieł P 4.3a moja wersja nr 677}


Obliczyć granicę funkcji $\lim\limits_{x\to\ 0}\frac{32 \cdot x}{tan(35 \cdot x)}$.
\zadStop
\rozwStart{Patryk Wirkus}{}
$$\lim\limits_{x\to\ 0}\frac{32 \cdot x}{tan(35 \cdot x)}=\lim\limits_{x\to\ 0}\frac{32 \cdot x \cdot cos(35 \cdot x)}{sin(35 \cdot x)}=\lim\limits_{x\to\ 0}\frac{32 \cdot cos(35 \cdot x)}{\frac{sin(35 \cdot x)}{x}}=\lim\limits_{x\to\ 0}\frac{32 \cdot cos(35 \cdot x)}{35 \cdot \frac{sin(35 \cdot x)}{35 \cdot x}} = \frac{32}{35}$$
\rozwStop
\odpStart
$\frac{32}{35}$
\odpStop
\testStart
A.$\frac{32}{35}$
B.$\infty$
C.$-\infty$
D.$0$
E.$-\frac{32}{35}$
F.$\frac{35}{32}$
G.$-\frac{35}{32}$
H.$35$
I.$32$
\testStop
\kluczStart
A
\kluczStop



\zadStart{Przykład z Wikieł P 4.3a moja wersja nr 678}


Obliczyć granicę funkcji $\lim\limits_{x\to\ 0}\frac{32 \cdot x}{tan(37 \cdot x)}$.
\zadStop
\rozwStart{Patryk Wirkus}{}
$$\lim\limits_{x\to\ 0}\frac{32 \cdot x}{tan(37 \cdot x)}=\lim\limits_{x\to\ 0}\frac{32 \cdot x \cdot cos(37 \cdot x)}{sin(37 \cdot x)}=\lim\limits_{x\to\ 0}\frac{32 \cdot cos(37 \cdot x)}{\frac{sin(37 \cdot x)}{x}}=\lim\limits_{x\to\ 0}\frac{32 \cdot cos(37 \cdot x)}{37 \cdot \frac{sin(37 \cdot x)}{37 \cdot x}} = \frac{32}{37}$$
\rozwStop
\odpStart
$\frac{32}{37}$
\odpStop
\testStart
A.$\frac{32}{37}$
B.$\infty$
C.$-\infty$
D.$0$
E.$-\frac{32}{37}$
F.$\frac{37}{32}$
G.$-\frac{37}{32}$
H.$37$
I.$32$
\testStop
\kluczStart
A
\kluczStop



\zadStart{Przykład z Wikieł P 4.3a moja wersja nr 679}


Obliczyć granicę funkcji $\lim\limits_{x\to\ 0}\frac{32 \cdot x}{tan(39 \cdot x)}$.
\zadStop
\rozwStart{Patryk Wirkus}{}
$$\lim\limits_{x\to\ 0}\frac{32 \cdot x}{tan(39 \cdot x)}=\lim\limits_{x\to\ 0}\frac{32 \cdot x \cdot cos(39 \cdot x)}{sin(39 \cdot x)}=\lim\limits_{x\to\ 0}\frac{32 \cdot cos(39 \cdot x)}{\frac{sin(39 \cdot x)}{x}}=\lim\limits_{x\to\ 0}\frac{32 \cdot cos(39 \cdot x)}{39 \cdot \frac{sin(39 \cdot x)}{39 \cdot x}} = \frac{32}{39}$$
\rozwStop
\odpStart
$\frac{32}{39}$
\odpStop
\testStart
A.$\frac{32}{39}$
B.$\infty$
C.$-\infty$
D.$0$
E.$-\frac{32}{39}$
F.$\frac{39}{32}$
G.$-\frac{39}{32}$
H.$39$
I.$32$
\testStop
\kluczStart
A
\kluczStop



\zadStart{Przykład z Wikieł P 4.3a moja wersja nr 680}


Obliczyć granicę funkcji $\lim\limits_{x\to\ 0}\frac{33 \cdot x}{tan(2 \cdot x)}$.
\zadStop
\rozwStart{Patryk Wirkus}{}
$$\lim\limits_{x\to\ 0}\frac{33 \cdot x}{tan(2 \cdot x)}=\lim\limits_{x\to\ 0}\frac{33 \cdot x \cdot cos(2 \cdot x)}{sin(2 \cdot x)}=\lim\limits_{x\to\ 0}\frac{33 \cdot cos(2 \cdot x)}{\frac{sin(2 \cdot x)}{x}}=\lim\limits_{x\to\ 0}\frac{33 \cdot cos(2 \cdot x)}{2 \cdot \frac{sin(2 \cdot x)}{2 \cdot x}} = \frac{33}{2}$$
\rozwStop
\odpStart
$\frac{33}{2}$
\odpStop
\testStart
A.$\frac{33}{2}$
B.$\infty$
C.$-\infty$
D.$0$
E.$-\frac{33}{2}$
F.$\frac{2}{33}$
G.$-\frac{2}{33}$
H.$2$
I.$33$
\testStop
\kluczStart
A
\kluczStop



\zadStart{Przykład z Wikieł P 4.3a moja wersja nr 681}


Obliczyć granicę funkcji $\lim\limits_{x\to\ 0}\frac{33 \cdot x}{tan(4 \cdot x)}$.
\zadStop
\rozwStart{Patryk Wirkus}{}
$$\lim\limits_{x\to\ 0}\frac{33 \cdot x}{tan(4 \cdot x)}=\lim\limits_{x\to\ 0}\frac{33 \cdot x \cdot cos(4 \cdot x)}{sin(4 \cdot x)}=\lim\limits_{x\to\ 0}\frac{33 \cdot cos(4 \cdot x)}{\frac{sin(4 \cdot x)}{x}}=\lim\limits_{x\to\ 0}\frac{33 \cdot cos(4 \cdot x)}{4 \cdot \frac{sin(4 \cdot x)}{4 \cdot x}} = \frac{33}{4}$$
\rozwStop
\odpStart
$\frac{33}{4}$
\odpStop
\testStart
A.$\frac{33}{4}$
B.$\infty$
C.$-\infty$
D.$0$
E.$-\frac{33}{4}$
F.$\frac{4}{33}$
G.$-\frac{4}{33}$
H.$4$
I.$33$
\testStop
\kluczStart
A
\kluczStop



\zadStart{Przykład z Wikieł P 4.3a moja wersja nr 682}


Obliczyć granicę funkcji $\lim\limits_{x\to\ 0}\frac{33 \cdot x}{tan(5 \cdot x)}$.
\zadStop
\rozwStart{Patryk Wirkus}{}
$$\lim\limits_{x\to\ 0}\frac{33 \cdot x}{tan(5 \cdot x)}=\lim\limits_{x\to\ 0}\frac{33 \cdot x \cdot cos(5 \cdot x)}{sin(5 \cdot x)}=\lim\limits_{x\to\ 0}\frac{33 \cdot cos(5 \cdot x)}{\frac{sin(5 \cdot x)}{x}}=\lim\limits_{x\to\ 0}\frac{33 \cdot cos(5 \cdot x)}{5 \cdot \frac{sin(5 \cdot x)}{5 \cdot x}} = \frac{33}{5}$$
\rozwStop
\odpStart
$\frac{33}{5}$
\odpStop
\testStart
A.$\frac{33}{5}$
B.$\infty$
C.$-\infty$
D.$0$
E.$-\frac{33}{5}$
F.$\frac{5}{33}$
G.$-\frac{5}{33}$
H.$5$
I.$33$
\testStop
\kluczStart
A
\kluczStop



\zadStart{Przykład z Wikieł P 4.3a moja wersja nr 683}


Obliczyć granicę funkcji $\lim\limits_{x\to\ 0}\frac{33 \cdot x}{tan(7 \cdot x)}$.
\zadStop
\rozwStart{Patryk Wirkus}{}
$$\lim\limits_{x\to\ 0}\frac{33 \cdot x}{tan(7 \cdot x)}=\lim\limits_{x\to\ 0}\frac{33 \cdot x \cdot cos(7 \cdot x)}{sin(7 \cdot x)}=\lim\limits_{x\to\ 0}\frac{33 \cdot cos(7 \cdot x)}{\frac{sin(7 \cdot x)}{x}}=\lim\limits_{x\to\ 0}\frac{33 \cdot cos(7 \cdot x)}{7 \cdot \frac{sin(7 \cdot x)}{7 \cdot x}} = \frac{33}{7}$$
\rozwStop
\odpStart
$\frac{33}{7}$
\odpStop
\testStart
A.$\frac{33}{7}$
B.$\infty$
C.$-\infty$
D.$0$
E.$-\frac{33}{7}$
F.$\frac{7}{33}$
G.$-\frac{7}{33}$
H.$7$
I.$33$
\testStop
\kluczStart
A
\kluczStop



\zadStart{Przykład z Wikieł P 4.3a moja wersja nr 684}


Obliczyć granicę funkcji $\lim\limits_{x\to\ 0}\frac{33 \cdot x}{tan(8 \cdot x)}$.
\zadStop
\rozwStart{Patryk Wirkus}{}
$$\lim\limits_{x\to\ 0}\frac{33 \cdot x}{tan(8 \cdot x)}=\lim\limits_{x\to\ 0}\frac{33 \cdot x \cdot cos(8 \cdot x)}{sin(8 \cdot x)}=\lim\limits_{x\to\ 0}\frac{33 \cdot cos(8 \cdot x)}{\frac{sin(8 \cdot x)}{x}}=\lim\limits_{x\to\ 0}\frac{33 \cdot cos(8 \cdot x)}{8 \cdot \frac{sin(8 \cdot x)}{8 \cdot x}} = \frac{33}{8}$$
\rozwStop
\odpStart
$\frac{33}{8}$
\odpStop
\testStart
A.$\frac{33}{8}$
B.$\infty$
C.$-\infty$
D.$0$
E.$-\frac{33}{8}$
F.$\frac{8}{33}$
G.$-\frac{8}{33}$
H.$8$
I.$33$
\testStop
\kluczStart
A
\kluczStop



\zadStart{Przykład z Wikieł P 4.3a moja wersja nr 685}


Obliczyć granicę funkcji $\lim\limits_{x\to\ 0}\frac{33 \cdot x}{tan(10 \cdot x)}$.
\zadStop
\rozwStart{Patryk Wirkus}{}
$$\lim\limits_{x\to\ 0}\frac{33 \cdot x}{tan(10 \cdot x)}=\lim\limits_{x\to\ 0}\frac{33 \cdot x \cdot cos(10 \cdot x)}{sin(10 \cdot x)}=\lim\limits_{x\to\ 0}\frac{33 \cdot cos(10 \cdot x)}{\frac{sin(10 \cdot x)}{x}}=\lim\limits_{x\to\ 0}\frac{33 \cdot cos(10 \cdot x)}{10 \cdot \frac{sin(10 \cdot x)}{10 \cdot x}} = \frac{33}{10}$$
\rozwStop
\odpStart
$\frac{33}{10}$
\odpStop
\testStart
A.$\frac{33}{10}$
B.$\infty$
C.$-\infty$
D.$0$
E.$-\frac{33}{10}$
F.$\frac{10}{33}$
G.$-\frac{10}{33}$
H.$10$
I.$33$
\testStop
\kluczStart
A
\kluczStop



\zadStart{Przykład z Wikieł P 4.3a moja wersja nr 686}


Obliczyć granicę funkcji $\lim\limits_{x\to\ 0}\frac{33 \cdot x}{tan(13 \cdot x)}$.
\zadStop
\rozwStart{Patryk Wirkus}{}
$$\lim\limits_{x\to\ 0}\frac{33 \cdot x}{tan(13 \cdot x)}=\lim\limits_{x\to\ 0}\frac{33 \cdot x \cdot cos(13 \cdot x)}{sin(13 \cdot x)}=\lim\limits_{x\to\ 0}\frac{33 \cdot cos(13 \cdot x)}{\frac{sin(13 \cdot x)}{x}}=\lim\limits_{x\to\ 0}\frac{33 \cdot cos(13 \cdot x)}{13 \cdot \frac{sin(13 \cdot x)}{13 \cdot x}} = \frac{33}{13}$$
\rozwStop
\odpStart
$\frac{33}{13}$
\odpStop
\testStart
A.$\frac{33}{13}$
B.$\infty$
C.$-\infty$
D.$0$
E.$-\frac{33}{13}$
F.$\frac{13}{33}$
G.$-\frac{13}{33}$
H.$13$
I.$33$
\testStop
\kluczStart
A
\kluczStop



\zadStart{Przykład z Wikieł P 4.3a moja wersja nr 687}


Obliczyć granicę funkcji $\lim\limits_{x\to\ 0}\frac{33 \cdot x}{tan(14 \cdot x)}$.
\zadStop
\rozwStart{Patryk Wirkus}{}
$$\lim\limits_{x\to\ 0}\frac{33 \cdot x}{tan(14 \cdot x)}=\lim\limits_{x\to\ 0}\frac{33 \cdot x \cdot cos(14 \cdot x)}{sin(14 \cdot x)}=\lim\limits_{x\to\ 0}\frac{33 \cdot cos(14 \cdot x)}{\frac{sin(14 \cdot x)}{x}}=\lim\limits_{x\to\ 0}\frac{33 \cdot cos(14 \cdot x)}{14 \cdot \frac{sin(14 \cdot x)}{14 \cdot x}} = \frac{33}{14}$$
\rozwStop
\odpStart
$\frac{33}{14}$
\odpStop
\testStart
A.$\frac{33}{14}$
B.$\infty$
C.$-\infty$
D.$0$
E.$-\frac{33}{14}$
F.$\frac{14}{33}$
G.$-\frac{14}{33}$
H.$14$
I.$33$
\testStop
\kluczStart
A
\kluczStop



\zadStart{Przykład z Wikieł P 4.3a moja wersja nr 688}


Obliczyć granicę funkcji $\lim\limits_{x\to\ 0}\frac{33 \cdot x}{tan(16 \cdot x)}$.
\zadStop
\rozwStart{Patryk Wirkus}{}
$$\lim\limits_{x\to\ 0}\frac{33 \cdot x}{tan(16 \cdot x)}=\lim\limits_{x\to\ 0}\frac{33 \cdot x \cdot cos(16 \cdot x)}{sin(16 \cdot x)}=\lim\limits_{x\to\ 0}\frac{33 \cdot cos(16 \cdot x)}{\frac{sin(16 \cdot x)}{x}}=\lim\limits_{x\to\ 0}\frac{33 \cdot cos(16 \cdot x)}{16 \cdot \frac{sin(16 \cdot x)}{16 \cdot x}} = \frac{33}{16}$$
\rozwStop
\odpStart
$\frac{33}{16}$
\odpStop
\testStart
A.$\frac{33}{16}$
B.$\infty$
C.$-\infty$
D.$0$
E.$-\frac{33}{16}$
F.$\frac{16}{33}$
G.$-\frac{16}{33}$
H.$16$
I.$33$
\testStop
\kluczStart
A
\kluczStop



\zadStart{Przykład z Wikieł P 4.3a moja wersja nr 689}


Obliczyć granicę funkcji $\lim\limits_{x\to\ 0}\frac{33 \cdot x}{tan(17 \cdot x)}$.
\zadStop
\rozwStart{Patryk Wirkus}{}
$$\lim\limits_{x\to\ 0}\frac{33 \cdot x}{tan(17 \cdot x)}=\lim\limits_{x\to\ 0}\frac{33 \cdot x \cdot cos(17 \cdot x)}{sin(17 \cdot x)}=\lim\limits_{x\to\ 0}\frac{33 \cdot cos(17 \cdot x)}{\frac{sin(17 \cdot x)}{x}}=\lim\limits_{x\to\ 0}\frac{33 \cdot cos(17 \cdot x)}{17 \cdot \frac{sin(17 \cdot x)}{17 \cdot x}} = \frac{33}{17}$$
\rozwStop
\odpStart
$\frac{33}{17}$
\odpStop
\testStart
A.$\frac{33}{17}$
B.$\infty$
C.$-\infty$
D.$0$
E.$-\frac{33}{17}$
F.$\frac{17}{33}$
G.$-\frac{17}{33}$
H.$17$
I.$33$
\testStop
\kluczStart
A
\kluczStop



\zadStart{Przykład z Wikieł P 4.3a moja wersja nr 690}


Obliczyć granicę funkcji $\lim\limits_{x\to\ 0}\frac{33 \cdot x}{tan(19 \cdot x)}$.
\zadStop
\rozwStart{Patryk Wirkus}{}
$$\lim\limits_{x\to\ 0}\frac{33 \cdot x}{tan(19 \cdot x)}=\lim\limits_{x\to\ 0}\frac{33 \cdot x \cdot cos(19 \cdot x)}{sin(19 \cdot x)}=\lim\limits_{x\to\ 0}\frac{33 \cdot cos(19 \cdot x)}{\frac{sin(19 \cdot x)}{x}}=\lim\limits_{x\to\ 0}\frac{33 \cdot cos(19 \cdot x)}{19 \cdot \frac{sin(19 \cdot x)}{19 \cdot x}} = \frac{33}{19}$$
\rozwStop
\odpStart
$\frac{33}{19}$
\odpStop
\testStart
A.$\frac{33}{19}$
B.$\infty$
C.$-\infty$
D.$0$
E.$-\frac{33}{19}$
F.$\frac{19}{33}$
G.$-\frac{19}{33}$
H.$19$
I.$33$
\testStop
\kluczStart
A
\kluczStop



\zadStart{Przykład z Wikieł P 4.3a moja wersja nr 691}


Obliczyć granicę funkcji $\lim\limits_{x\to\ 0}\frac{33 \cdot x}{tan(20 \cdot x)}$.
\zadStop
\rozwStart{Patryk Wirkus}{}
$$\lim\limits_{x\to\ 0}\frac{33 \cdot x}{tan(20 \cdot x)}=\lim\limits_{x\to\ 0}\frac{33 \cdot x \cdot cos(20 \cdot x)}{sin(20 \cdot x)}=\lim\limits_{x\to\ 0}\frac{33 \cdot cos(20 \cdot x)}{\frac{sin(20 \cdot x)}{x}}=\lim\limits_{x\to\ 0}\frac{33 \cdot cos(20 \cdot x)}{20 \cdot \frac{sin(20 \cdot x)}{20 \cdot x}} = \frac{33}{20}$$
\rozwStop
\odpStart
$\frac{33}{20}$
\odpStop
\testStart
A.$\frac{33}{20}$
B.$\infty$
C.$-\infty$
D.$0$
E.$-\frac{33}{20}$
F.$\frac{20}{33}$
G.$-\frac{20}{33}$
H.$20$
I.$33$
\testStop
\kluczStart
A
\kluczStop



\zadStart{Przykład z Wikieł P 4.3a moja wersja nr 692}


Obliczyć granicę funkcji $\lim\limits_{x\to\ 0}\frac{33 \cdot x}{tan(23 \cdot x)}$.
\zadStop
\rozwStart{Patryk Wirkus}{}
$$\lim\limits_{x\to\ 0}\frac{33 \cdot x}{tan(23 \cdot x)}=\lim\limits_{x\to\ 0}\frac{33 \cdot x \cdot cos(23 \cdot x)}{sin(23 \cdot x)}=\lim\limits_{x\to\ 0}\frac{33 \cdot cos(23 \cdot x)}{\frac{sin(23 \cdot x)}{x}}=\lim\limits_{x\to\ 0}\frac{33 \cdot cos(23 \cdot x)}{23 \cdot \frac{sin(23 \cdot x)}{23 \cdot x}} = \frac{33}{23}$$
\rozwStop
\odpStart
$\frac{33}{23}$
\odpStop
\testStart
A.$\frac{33}{23}$
B.$\infty$
C.$-\infty$
D.$0$
E.$-\frac{33}{23}$
F.$\frac{23}{33}$
G.$-\frac{23}{33}$
H.$23$
I.$33$
\testStop
\kluczStart
A
\kluczStop



\zadStart{Przykład z Wikieł P 4.3a moja wersja nr 693}


Obliczyć granicę funkcji $\lim\limits_{x\to\ 0}\frac{33 \cdot x}{tan(25 \cdot x)}$.
\zadStop
\rozwStart{Patryk Wirkus}{}
$$\lim\limits_{x\to\ 0}\frac{33 \cdot x}{tan(25 \cdot x)}=\lim\limits_{x\to\ 0}\frac{33 \cdot x \cdot cos(25 \cdot x)}{sin(25 \cdot x)}=\lim\limits_{x\to\ 0}\frac{33 \cdot cos(25 \cdot x)}{\frac{sin(25 \cdot x)}{x}}=\lim\limits_{x\to\ 0}\frac{33 \cdot cos(25 \cdot x)}{25 \cdot \frac{sin(25 \cdot x)}{25 \cdot x}} = \frac{33}{25}$$
\rozwStop
\odpStart
$\frac{33}{25}$
\odpStop
\testStart
A.$\frac{33}{25}$
B.$\infty$
C.$-\infty$
D.$0$
E.$-\frac{33}{25}$
F.$\frac{25}{33}$
G.$-\frac{25}{33}$
H.$25$
I.$33$
\testStop
\kluczStart
A
\kluczStop



\zadStart{Przykład z Wikieł P 4.3a moja wersja nr 694}


Obliczyć granicę funkcji $\lim\limits_{x\to\ 0}\frac{33 \cdot x}{tan(26 \cdot x)}$.
\zadStop
\rozwStart{Patryk Wirkus}{}
$$\lim\limits_{x\to\ 0}\frac{33 \cdot x}{tan(26 \cdot x)}=\lim\limits_{x\to\ 0}\frac{33 \cdot x \cdot cos(26 \cdot x)}{sin(26 \cdot x)}=\lim\limits_{x\to\ 0}\frac{33 \cdot cos(26 \cdot x)}{\frac{sin(26 \cdot x)}{x}}=\lim\limits_{x\to\ 0}\frac{33 \cdot cos(26 \cdot x)}{26 \cdot \frac{sin(26 \cdot x)}{26 \cdot x}} = \frac{33}{26}$$
\rozwStop
\odpStart
$\frac{33}{26}$
\odpStop
\testStart
A.$\frac{33}{26}$
B.$\infty$
C.$-\infty$
D.$0$
E.$-\frac{33}{26}$
F.$\frac{26}{33}$
G.$-\frac{26}{33}$
H.$26$
I.$33$
\testStop
\kluczStart
A
\kluczStop



\zadStart{Przykład z Wikieł P 4.3a moja wersja nr 695}


Obliczyć granicę funkcji $\lim\limits_{x\to\ 0}\frac{33 \cdot x}{tan(28 \cdot x)}$.
\zadStop
\rozwStart{Patryk Wirkus}{}
$$\lim\limits_{x\to\ 0}\frac{33 \cdot x}{tan(28 \cdot x)}=\lim\limits_{x\to\ 0}\frac{33 \cdot x \cdot cos(28 \cdot x)}{sin(28 \cdot x)}=\lim\limits_{x\to\ 0}\frac{33 \cdot cos(28 \cdot x)}{\frac{sin(28 \cdot x)}{x}}=\lim\limits_{x\to\ 0}\frac{33 \cdot cos(28 \cdot x)}{28 \cdot \frac{sin(28 \cdot x)}{28 \cdot x}} = \frac{33}{28}$$
\rozwStop
\odpStart
$\frac{33}{28}$
\odpStop
\testStart
A.$\frac{33}{28}$
B.$\infty$
C.$-\infty$
D.$0$
E.$-\frac{33}{28}$
F.$\frac{28}{33}$
G.$-\frac{28}{33}$
H.$28$
I.$33$
\testStop
\kluczStart
A
\kluczStop



\zadStart{Przykład z Wikieł P 4.3a moja wersja nr 696}


Obliczyć granicę funkcji $\lim\limits_{x\to\ 0}\frac{33 \cdot x}{tan(29 \cdot x)}$.
\zadStop
\rozwStart{Patryk Wirkus}{}
$$\lim\limits_{x\to\ 0}\frac{33 \cdot x}{tan(29 \cdot x)}=\lim\limits_{x\to\ 0}\frac{33 \cdot x \cdot cos(29 \cdot x)}{sin(29 \cdot x)}=\lim\limits_{x\to\ 0}\frac{33 \cdot cos(29 \cdot x)}{\frac{sin(29 \cdot x)}{x}}=\lim\limits_{x\to\ 0}\frac{33 \cdot cos(29 \cdot x)}{29 \cdot \frac{sin(29 \cdot x)}{29 \cdot x}} = \frac{33}{29}$$
\rozwStop
\odpStart
$\frac{33}{29}$
\odpStop
\testStart
A.$\frac{33}{29}$
B.$\infty$
C.$-\infty$
D.$0$
E.$-\frac{33}{29}$
F.$\frac{29}{33}$
G.$-\frac{29}{33}$
H.$29$
I.$33$
\testStop
\kluczStart
A
\kluczStop



\zadStart{Przykład z Wikieł P 4.3a moja wersja nr 697}


Obliczyć granicę funkcji $\lim\limits_{x\to\ 0}\frac{33 \cdot x}{tan(31 \cdot x)}$.
\zadStop
\rozwStart{Patryk Wirkus}{}
$$\lim\limits_{x\to\ 0}\frac{33 \cdot x}{tan(31 \cdot x)}=\lim\limits_{x\to\ 0}\frac{33 \cdot x \cdot cos(31 \cdot x)}{sin(31 \cdot x)}=\lim\limits_{x\to\ 0}\frac{33 \cdot cos(31 \cdot x)}{\frac{sin(31 \cdot x)}{x}}=\lim\limits_{x\to\ 0}\frac{33 \cdot cos(31 \cdot x)}{31 \cdot \frac{sin(31 \cdot x)}{31 \cdot x}} = \frac{33}{31}$$
\rozwStop
\odpStart
$\frac{33}{31}$
\odpStop
\testStart
A.$\frac{33}{31}$
B.$\infty$
C.$-\infty$
D.$0$
E.$-\frac{33}{31}$
F.$\frac{31}{33}$
G.$-\frac{31}{33}$
H.$31$
I.$33$
\testStop
\kluczStart
A
\kluczStop



\zadStart{Przykład z Wikieł P 4.3a moja wersja nr 698}


Obliczyć granicę funkcji $\lim\limits_{x\to\ 0}\frac{33 \cdot x}{tan(32 \cdot x)}$.
\zadStop
\rozwStart{Patryk Wirkus}{}
$$\lim\limits_{x\to\ 0}\frac{33 \cdot x}{tan(32 \cdot x)}=\lim\limits_{x\to\ 0}\frac{33 \cdot x \cdot cos(32 \cdot x)}{sin(32 \cdot x)}=\lim\limits_{x\to\ 0}\frac{33 \cdot cos(32 \cdot x)}{\frac{sin(32 \cdot x)}{x}}=\lim\limits_{x\to\ 0}\frac{33 \cdot cos(32 \cdot x)}{32 \cdot \frac{sin(32 \cdot x)}{32 \cdot x}} = \frac{33}{32}$$
\rozwStop
\odpStart
$\frac{33}{32}$
\odpStop
\testStart
A.$\frac{33}{32}$
B.$\infty$
C.$-\infty$
D.$0$
E.$-\frac{33}{32}$
F.$\frac{32}{33}$
G.$-\frac{32}{33}$
H.$32$
I.$33$
\testStop
\kluczStart
A
\kluczStop



\zadStart{Przykład z Wikieł P 4.3a moja wersja nr 699}


Obliczyć granicę funkcji $\lim\limits_{x\to\ 0}\frac{33 \cdot x}{tan(34 \cdot x)}$.
\zadStop
\rozwStart{Patryk Wirkus}{}
$$\lim\limits_{x\to\ 0}\frac{33 \cdot x}{tan(34 \cdot x)}=\lim\limits_{x\to\ 0}\frac{33 \cdot x \cdot cos(34 \cdot x)}{sin(34 \cdot x)}=\lim\limits_{x\to\ 0}\frac{33 \cdot cos(34 \cdot x)}{\frac{sin(34 \cdot x)}{x}}=\lim\limits_{x\to\ 0}\frac{33 \cdot cos(34 \cdot x)}{34 \cdot \frac{sin(34 \cdot x)}{34 \cdot x}} = \frac{33}{34}$$
\rozwStop
\odpStart
$\frac{33}{34}$
\odpStop
\testStart
A.$\frac{33}{34}$
B.$\infty$
C.$-\infty$
D.$0$
E.$-\frac{33}{34}$
F.$\frac{34}{33}$
G.$-\frac{34}{33}$
H.$34$
I.$33$
\testStop
\kluczStart
A
\kluczStop



\zadStart{Przykład z Wikieł P 4.3a moja wersja nr 700}


Obliczyć granicę funkcji $\lim\limits_{x\to\ 0}\frac{33 \cdot x}{tan(35 \cdot x)}$.
\zadStop
\rozwStart{Patryk Wirkus}{}
$$\lim\limits_{x\to\ 0}\frac{33 \cdot x}{tan(35 \cdot x)}=\lim\limits_{x\to\ 0}\frac{33 \cdot x \cdot cos(35 \cdot x)}{sin(35 \cdot x)}=\lim\limits_{x\to\ 0}\frac{33 \cdot cos(35 \cdot x)}{\frac{sin(35 \cdot x)}{x}}=\lim\limits_{x\to\ 0}\frac{33 \cdot cos(35 \cdot x)}{35 \cdot \frac{sin(35 \cdot x)}{35 \cdot x}} = \frac{33}{35}$$
\rozwStop
\odpStart
$\frac{33}{35}$
\odpStop
\testStart
A.$\frac{33}{35}$
B.$\infty$
C.$-\infty$
D.$0$
E.$-\frac{33}{35}$
F.$\frac{35}{33}$
G.$-\frac{35}{33}$
H.$35$
I.$33$
\testStop
\kluczStart
A
\kluczStop



\zadStart{Przykład z Wikieł P 4.3a moja wersja nr 701}


Obliczyć granicę funkcji $\lim\limits_{x\to\ 0}\frac{33 \cdot x}{tan(37 \cdot x)}$.
\zadStop
\rozwStart{Patryk Wirkus}{}
$$\lim\limits_{x\to\ 0}\frac{33 \cdot x}{tan(37 \cdot x)}=\lim\limits_{x\to\ 0}\frac{33 \cdot x \cdot cos(37 \cdot x)}{sin(37 \cdot x)}=\lim\limits_{x\to\ 0}\frac{33 \cdot cos(37 \cdot x)}{\frac{sin(37 \cdot x)}{x}}=\lim\limits_{x\to\ 0}\frac{33 \cdot cos(37 \cdot x)}{37 \cdot \frac{sin(37 \cdot x)}{37 \cdot x}} = \frac{33}{37}$$
\rozwStop
\odpStart
$\frac{33}{37}$
\odpStop
\testStart
A.$\frac{33}{37}$
B.$\infty$
C.$-\infty$
D.$0$
E.$-\frac{33}{37}$
F.$\frac{37}{33}$
G.$-\frac{37}{33}$
H.$37$
I.$33$
\testStop
\kluczStart
A
\kluczStop



\zadStart{Przykład z Wikieł P 4.3a moja wersja nr 702}


Obliczyć granicę funkcji $\lim\limits_{x\to\ 0}\frac{33 \cdot x}{tan(38 \cdot x)}$.
\zadStop
\rozwStart{Patryk Wirkus}{}
$$\lim\limits_{x\to\ 0}\frac{33 \cdot x}{tan(38 \cdot x)}=\lim\limits_{x\to\ 0}\frac{33 \cdot x \cdot cos(38 \cdot x)}{sin(38 \cdot x)}=\lim\limits_{x\to\ 0}\frac{33 \cdot cos(38 \cdot x)}{\frac{sin(38 \cdot x)}{x}}=\lim\limits_{x\to\ 0}\frac{33 \cdot cos(38 \cdot x)}{38 \cdot \frac{sin(38 \cdot x)}{38 \cdot x}} = \frac{33}{38}$$
\rozwStop
\odpStart
$\frac{33}{38}$
\odpStop
\testStart
A.$\frac{33}{38}$
B.$\infty$
C.$-\infty$
D.$0$
E.$-\frac{33}{38}$
F.$\frac{38}{33}$
G.$-\frac{38}{33}$
H.$38$
I.$33$
\testStop
\kluczStart
A
\kluczStop



\zadStart{Przykład z Wikieł P 4.3a moja wersja nr 703}


Obliczyć granicę funkcji $\lim\limits_{x\to\ 0}\frac{33 \cdot x}{tan(40 \cdot x)}$.
\zadStop
\rozwStart{Patryk Wirkus}{}
$$\lim\limits_{x\to\ 0}\frac{33 \cdot x}{tan(40 \cdot x)}=\lim\limits_{x\to\ 0}\frac{33 \cdot x \cdot cos(40 \cdot x)}{sin(40 \cdot x)}=\lim\limits_{x\to\ 0}\frac{33 \cdot cos(40 \cdot x)}{\frac{sin(40 \cdot x)}{x}}=\lim\limits_{x\to\ 0}\frac{33 \cdot cos(40 \cdot x)}{40 \cdot \frac{sin(40 \cdot x)}{40 \cdot x}} = \frac{33}{40}$$
\rozwStop
\odpStart
$\frac{33}{40}$
\odpStop
\testStart
A.$\frac{33}{40}$
B.$\infty$
C.$-\infty$
D.$0$
E.$-\frac{33}{40}$
F.$\frac{40}{33}$
G.$-\frac{40}{33}$
H.$40$
I.$33$
\testStop
\kluczStart
A
\kluczStop



\zadStart{Przykład z Wikieł P 4.3a moja wersja nr 704}


Obliczyć granicę funkcji $\lim\limits_{x\to\ 0}\frac{34 \cdot x}{tan(3 \cdot x)}$.
\zadStop
\rozwStart{Patryk Wirkus}{}
$$\lim\limits_{x\to\ 0}\frac{34 \cdot x}{tan(3 \cdot x)}=\lim\limits_{x\to\ 0}\frac{34 \cdot x \cdot cos(3 \cdot x)}{sin(3 \cdot x)}=\lim\limits_{x\to\ 0}\frac{34 \cdot cos(3 \cdot x)}{\frac{sin(3 \cdot x)}{x}}=\lim\limits_{x\to\ 0}\frac{34 \cdot cos(3 \cdot x)}{3 \cdot \frac{sin(3 \cdot x)}{3 \cdot x}} = \frac{34}{3}$$
\rozwStop
\odpStart
$\frac{34}{3}$
\odpStop
\testStart
A.$\frac{34}{3}$
B.$\infty$
C.$-\infty$
D.$0$
E.$-\frac{34}{3}$
F.$\frac{3}{34}$
G.$-\frac{3}{34}$
H.$3$
I.$34$
\testStop
\kluczStart
A
\kluczStop



\zadStart{Przykład z Wikieł P 4.3a moja wersja nr 705}


Obliczyć granicę funkcji $\lim\limits_{x\to\ 0}\frac{34 \cdot x}{tan(5 \cdot x)}$.
\zadStop
\rozwStart{Patryk Wirkus}{}
$$\lim\limits_{x\to\ 0}\frac{34 \cdot x}{tan(5 \cdot x)}=\lim\limits_{x\to\ 0}\frac{34 \cdot x \cdot cos(5 \cdot x)}{sin(5 \cdot x)}=\lim\limits_{x\to\ 0}\frac{34 \cdot cos(5 \cdot x)}{\frac{sin(5 \cdot x)}{x}}=\lim\limits_{x\to\ 0}\frac{34 \cdot cos(5 \cdot x)}{5 \cdot \frac{sin(5 \cdot x)}{5 \cdot x}} = \frac{34}{5}$$
\rozwStop
\odpStart
$\frac{34}{5}$
\odpStop
\testStart
A.$\frac{34}{5}$
B.$\infty$
C.$-\infty$
D.$0$
E.$-\frac{34}{5}$
F.$\frac{5}{34}$
G.$-\frac{5}{34}$
H.$5$
I.$34$
\testStop
\kluczStart
A
\kluczStop



\zadStart{Przykład z Wikieł P 4.3a moja wersja nr 706}


Obliczyć granicę funkcji $\lim\limits_{x\to\ 0}\frac{34 \cdot x}{tan(7 \cdot x)}$.
\zadStop
\rozwStart{Patryk Wirkus}{}
$$\lim\limits_{x\to\ 0}\frac{34 \cdot x}{tan(7 \cdot x)}=\lim\limits_{x\to\ 0}\frac{34 \cdot x \cdot cos(7 \cdot x)}{sin(7 \cdot x)}=\lim\limits_{x\to\ 0}\frac{34 \cdot cos(7 \cdot x)}{\frac{sin(7 \cdot x)}{x}}=\lim\limits_{x\to\ 0}\frac{34 \cdot cos(7 \cdot x)}{7 \cdot \frac{sin(7 \cdot x)}{7 \cdot x}} = \frac{34}{7}$$
\rozwStop
\odpStart
$\frac{34}{7}$
\odpStop
\testStart
A.$\frac{34}{7}$
B.$\infty$
C.$-\infty$
D.$0$
E.$-\frac{34}{7}$
F.$\frac{7}{34}$
G.$-\frac{7}{34}$
H.$7$
I.$34$
\testStop
\kluczStart
A
\kluczStop



\zadStart{Przykład z Wikieł P 4.3a moja wersja nr 707}


Obliczyć granicę funkcji $\lim\limits_{x\to\ 0}\frac{34 \cdot x}{tan(9 \cdot x)}$.
\zadStop
\rozwStart{Patryk Wirkus}{}
$$\lim\limits_{x\to\ 0}\frac{34 \cdot x}{tan(9 \cdot x)}=\lim\limits_{x\to\ 0}\frac{34 \cdot x \cdot cos(9 \cdot x)}{sin(9 \cdot x)}=\lim\limits_{x\to\ 0}\frac{34 \cdot cos(9 \cdot x)}{\frac{sin(9 \cdot x)}{x}}=\lim\limits_{x\to\ 0}\frac{34 \cdot cos(9 \cdot x)}{9 \cdot \frac{sin(9 \cdot x)}{9 \cdot x}} = \frac{34}{9}$$
\rozwStop
\odpStart
$\frac{34}{9}$
\odpStop
\testStart
A.$\frac{34}{9}$
B.$\infty$
C.$-\infty$
D.$0$
E.$-\frac{34}{9}$
F.$\frac{9}{34}$
G.$-\frac{9}{34}$
H.$9$
I.$34$
\testStop
\kluczStart
A
\kluczStop



\zadStart{Przykład z Wikieł P 4.3a moja wersja nr 708}


Obliczyć granicę funkcji $\lim\limits_{x\to\ 0}\frac{34 \cdot x}{tan(11 \cdot x)}$.
\zadStop
\rozwStart{Patryk Wirkus}{}
$$\lim\limits_{x\to\ 0}\frac{34 \cdot x}{tan(11 \cdot x)}=\lim\limits_{x\to\ 0}\frac{34 \cdot x \cdot cos(11 \cdot x)}{sin(11 \cdot x)}=\lim\limits_{x\to\ 0}\frac{34 \cdot cos(11 \cdot x)}{\frac{sin(11 \cdot x)}{x}}=\lim\limits_{x\to\ 0}\frac{34 \cdot cos(11 \cdot x)}{11 \cdot \frac{sin(11 \cdot x)}{11 \cdot x}} = \frac{34}{11}$$
\rozwStop
\odpStart
$\frac{34}{11}$
\odpStop
\testStart
A.$\frac{34}{11}$
B.$\infty$
C.$-\infty$
D.$0$
E.$-\frac{34}{11}$
F.$\frac{11}{34}$
G.$-\frac{11}{34}$
H.$11$
I.$34$
\testStop
\kluczStart
A
\kluczStop



\zadStart{Przykład z Wikieł P 4.3a moja wersja nr 709}


Obliczyć granicę funkcji $\lim\limits_{x\to\ 0}\frac{34 \cdot x}{tan(13 \cdot x)}$.
\zadStop
\rozwStart{Patryk Wirkus}{}
$$\lim\limits_{x\to\ 0}\frac{34 \cdot x}{tan(13 \cdot x)}=\lim\limits_{x\to\ 0}\frac{34 \cdot x \cdot cos(13 \cdot x)}{sin(13 \cdot x)}=\lim\limits_{x\to\ 0}\frac{34 \cdot cos(13 \cdot x)}{\frac{sin(13 \cdot x)}{x}}=\lim\limits_{x\to\ 0}\frac{34 \cdot cos(13 \cdot x)}{13 \cdot \frac{sin(13 \cdot x)}{13 \cdot x}} = \frac{34}{13}$$
\rozwStop
\odpStart
$\frac{34}{13}$
\odpStop
\testStart
A.$\frac{34}{13}$
B.$\infty$
C.$-\infty$
D.$0$
E.$-\frac{34}{13}$
F.$\frac{13}{34}$
G.$-\frac{13}{34}$
H.$13$
I.$34$
\testStop
\kluczStart
A
\kluczStop



\zadStart{Przykład z Wikieł P 4.3a moja wersja nr 710}


Obliczyć granicę funkcji $\lim\limits_{x\to\ 0}\frac{34 \cdot x}{tan(15 \cdot x)}$.
\zadStop
\rozwStart{Patryk Wirkus}{}
$$\lim\limits_{x\to\ 0}\frac{34 \cdot x}{tan(15 \cdot x)}=\lim\limits_{x\to\ 0}\frac{34 \cdot x \cdot cos(15 \cdot x)}{sin(15 \cdot x)}=\lim\limits_{x\to\ 0}\frac{34 \cdot cos(15 \cdot x)}{\frac{sin(15 \cdot x)}{x}}=\lim\limits_{x\to\ 0}\frac{34 \cdot cos(15 \cdot x)}{15 \cdot \frac{sin(15 \cdot x)}{15 \cdot x}} = \frac{34}{15}$$
\rozwStop
\odpStart
$\frac{34}{15}$
\odpStop
\testStart
A.$\frac{34}{15}$
B.$\infty$
C.$-\infty$
D.$0$
E.$-\frac{34}{15}$
F.$\frac{15}{34}$
G.$-\frac{15}{34}$
H.$15$
I.$34$
\testStop
\kluczStart
A
\kluczStop



\zadStart{Przykład z Wikieł P 4.3a moja wersja nr 711}


Obliczyć granicę funkcji $\lim\limits_{x\to\ 0}\frac{34 \cdot x}{tan(19 \cdot x)}$.
\zadStop
\rozwStart{Patryk Wirkus}{}
$$\lim\limits_{x\to\ 0}\frac{34 \cdot x}{tan(19 \cdot x)}=\lim\limits_{x\to\ 0}\frac{34 \cdot x \cdot cos(19 \cdot x)}{sin(19 \cdot x)}=\lim\limits_{x\to\ 0}\frac{34 \cdot cos(19 \cdot x)}{\frac{sin(19 \cdot x)}{x}}=\lim\limits_{x\to\ 0}\frac{34 \cdot cos(19 \cdot x)}{19 \cdot \frac{sin(19 \cdot x)}{19 \cdot x}} = \frac{34}{19}$$
\rozwStop
\odpStart
$\frac{34}{19}$
\odpStop
\testStart
A.$\frac{34}{19}$
B.$\infty$
C.$-\infty$
D.$0$
E.$-\frac{34}{19}$
F.$\frac{19}{34}$
G.$-\frac{19}{34}$
H.$19$
I.$34$
\testStop
\kluczStart
A
\kluczStop



\zadStart{Przykład z Wikieł P 4.3a moja wersja nr 712}


Obliczyć granicę funkcji $\lim\limits_{x\to\ 0}\frac{34 \cdot x}{tan(21 \cdot x)}$.
\zadStop
\rozwStart{Patryk Wirkus}{}
$$\lim\limits_{x\to\ 0}\frac{34 \cdot x}{tan(21 \cdot x)}=\lim\limits_{x\to\ 0}\frac{34 \cdot x \cdot cos(21 \cdot x)}{sin(21 \cdot x)}=\lim\limits_{x\to\ 0}\frac{34 \cdot cos(21 \cdot x)}{\frac{sin(21 \cdot x)}{x}}=\lim\limits_{x\to\ 0}\frac{34 \cdot cos(21 \cdot x)}{21 \cdot \frac{sin(21 \cdot x)}{21 \cdot x}} = \frac{34}{21}$$
\rozwStop
\odpStart
$\frac{34}{21}$
\odpStop
\testStart
A.$\frac{34}{21}$
B.$\infty$
C.$-\infty$
D.$0$
E.$-\frac{34}{21}$
F.$\frac{21}{34}$
G.$-\frac{21}{34}$
H.$21$
I.$34$
\testStop
\kluczStart
A
\kluczStop



\zadStart{Przykład z Wikieł P 4.3a moja wersja nr 713}


Obliczyć granicę funkcji $\lim\limits_{x\to\ 0}\frac{34 \cdot x}{tan(23 \cdot x)}$.
\zadStop
\rozwStart{Patryk Wirkus}{}
$$\lim\limits_{x\to\ 0}\frac{34 \cdot x}{tan(23 \cdot x)}=\lim\limits_{x\to\ 0}\frac{34 \cdot x \cdot cos(23 \cdot x)}{sin(23 \cdot x)}=\lim\limits_{x\to\ 0}\frac{34 \cdot cos(23 \cdot x)}{\frac{sin(23 \cdot x)}{x}}=\lim\limits_{x\to\ 0}\frac{34 \cdot cos(23 \cdot x)}{23 \cdot \frac{sin(23 \cdot x)}{23 \cdot x}} = \frac{34}{23}$$
\rozwStop
\odpStart
$\frac{34}{23}$
\odpStop
\testStart
A.$\frac{34}{23}$
B.$\infty$
C.$-\infty$
D.$0$
E.$-\frac{34}{23}$
F.$\frac{23}{34}$
G.$-\frac{23}{34}$
H.$23$
I.$34$
\testStop
\kluczStart
A
\kluczStop



\zadStart{Przykład z Wikieł P 4.3a moja wersja nr 714}


Obliczyć granicę funkcji $\lim\limits_{x\to\ 0}\frac{34 \cdot x}{tan(25 \cdot x)}$.
\zadStop
\rozwStart{Patryk Wirkus}{}
$$\lim\limits_{x\to\ 0}\frac{34 \cdot x}{tan(25 \cdot x)}=\lim\limits_{x\to\ 0}\frac{34 \cdot x \cdot cos(25 \cdot x)}{sin(25 \cdot x)}=\lim\limits_{x\to\ 0}\frac{34 \cdot cos(25 \cdot x)}{\frac{sin(25 \cdot x)}{x}}=\lim\limits_{x\to\ 0}\frac{34 \cdot cos(25 \cdot x)}{25 \cdot \frac{sin(25 \cdot x)}{25 \cdot x}} = \frac{34}{25}$$
\rozwStop
\odpStart
$\frac{34}{25}$
\odpStop
\testStart
A.$\frac{34}{25}$
B.$\infty$
C.$-\infty$
D.$0$
E.$-\frac{34}{25}$
F.$\frac{25}{34}$
G.$-\frac{25}{34}$
H.$25$
I.$34$
\testStop
\kluczStart
A
\kluczStop



\zadStart{Przykład z Wikieł P 4.3a moja wersja nr 715}


Obliczyć granicę funkcji $\lim\limits_{x\to\ 0}\frac{34 \cdot x}{tan(27 \cdot x)}$.
\zadStop
\rozwStart{Patryk Wirkus}{}
$$\lim\limits_{x\to\ 0}\frac{34 \cdot x}{tan(27 \cdot x)}=\lim\limits_{x\to\ 0}\frac{34 \cdot x \cdot cos(27 \cdot x)}{sin(27 \cdot x)}=\lim\limits_{x\to\ 0}\frac{34 \cdot cos(27 \cdot x)}{\frac{sin(27 \cdot x)}{x}}=\lim\limits_{x\to\ 0}\frac{34 \cdot cos(27 \cdot x)}{27 \cdot \frac{sin(27 \cdot x)}{27 \cdot x}} = \frac{34}{27}$$
\rozwStop
\odpStart
$\frac{34}{27}$
\odpStop
\testStart
A.$\frac{34}{27}$
B.$\infty$
C.$-\infty$
D.$0$
E.$-\frac{34}{27}$
F.$\frac{27}{34}$
G.$-\frac{27}{34}$
H.$27$
I.$34$
\testStop
\kluczStart
A
\kluczStop



\zadStart{Przykład z Wikieł P 4.3a moja wersja nr 716}


Obliczyć granicę funkcji $\lim\limits_{x\to\ 0}\frac{34 \cdot x}{tan(29 \cdot x)}$.
\zadStop
\rozwStart{Patryk Wirkus}{}
$$\lim\limits_{x\to\ 0}\frac{34 \cdot x}{tan(29 \cdot x)}=\lim\limits_{x\to\ 0}\frac{34 \cdot x \cdot cos(29 \cdot x)}{sin(29 \cdot x)}=\lim\limits_{x\to\ 0}\frac{34 \cdot cos(29 \cdot x)}{\frac{sin(29 \cdot x)}{x}}=\lim\limits_{x\to\ 0}\frac{34 \cdot cos(29 \cdot x)}{29 \cdot \frac{sin(29 \cdot x)}{29 \cdot x}} = \frac{34}{29}$$
\rozwStop
\odpStart
$\frac{34}{29}$
\odpStop
\testStart
A.$\frac{34}{29}$
B.$\infty$
C.$-\infty$
D.$0$
E.$-\frac{34}{29}$
F.$\frac{29}{34}$
G.$-\frac{29}{34}$
H.$29$
I.$34$
\testStop
\kluczStart
A
\kluczStop



\zadStart{Przykład z Wikieł P 4.3a moja wersja nr 717}


Obliczyć granicę funkcji $\lim\limits_{x\to\ 0}\frac{34 \cdot x}{tan(31 \cdot x)}$.
\zadStop
\rozwStart{Patryk Wirkus}{}
$$\lim\limits_{x\to\ 0}\frac{34 \cdot x}{tan(31 \cdot x)}=\lim\limits_{x\to\ 0}\frac{34 \cdot x \cdot cos(31 \cdot x)}{sin(31 \cdot x)}=\lim\limits_{x\to\ 0}\frac{34 \cdot cos(31 \cdot x)}{\frac{sin(31 \cdot x)}{x}}=\lim\limits_{x\to\ 0}\frac{34 \cdot cos(31 \cdot x)}{31 \cdot \frac{sin(31 \cdot x)}{31 \cdot x}} = \frac{34}{31}$$
\rozwStop
\odpStart
$\frac{34}{31}$
\odpStop
\testStart
A.$\frac{34}{31}$
B.$\infty$
C.$-\infty$
D.$0$
E.$-\frac{34}{31}$
F.$\frac{31}{34}$
G.$-\frac{31}{34}$
H.$31$
I.$34$
\testStop
\kluczStart
A
\kluczStop



\zadStart{Przykład z Wikieł P 4.3a moja wersja nr 718}


Obliczyć granicę funkcji $\lim\limits_{x\to\ 0}\frac{34 \cdot x}{tan(33 \cdot x)}$.
\zadStop
\rozwStart{Patryk Wirkus}{}
$$\lim\limits_{x\to\ 0}\frac{34 \cdot x}{tan(33 \cdot x)}=\lim\limits_{x\to\ 0}\frac{34 \cdot x \cdot cos(33 \cdot x)}{sin(33 \cdot x)}=\lim\limits_{x\to\ 0}\frac{34 \cdot cos(33 \cdot x)}{\frac{sin(33 \cdot x)}{x}}=\lim\limits_{x\to\ 0}\frac{34 \cdot cos(33 \cdot x)}{33 \cdot \frac{sin(33 \cdot x)}{33 \cdot x}} = \frac{34}{33}$$
\rozwStop
\odpStart
$\frac{34}{33}$
\odpStop
\testStart
A.$\frac{34}{33}$
B.$\infty$
C.$-\infty$
D.$0$
E.$-\frac{34}{33}$
F.$\frac{33}{34}$
G.$-\frac{33}{34}$
H.$33$
I.$34$
\testStop
\kluczStart
A
\kluczStop



\zadStart{Przykład z Wikieł P 4.3a moja wersja nr 719}


Obliczyć granicę funkcji $\lim\limits_{x\to\ 0}\frac{34 \cdot x}{tan(35 \cdot x)}$.
\zadStop
\rozwStart{Patryk Wirkus}{}
$$\lim\limits_{x\to\ 0}\frac{34 \cdot x}{tan(35 \cdot x)}=\lim\limits_{x\to\ 0}\frac{34 \cdot x \cdot cos(35 \cdot x)}{sin(35 \cdot x)}=\lim\limits_{x\to\ 0}\frac{34 \cdot cos(35 \cdot x)}{\frac{sin(35 \cdot x)}{x}}=\lim\limits_{x\to\ 0}\frac{34 \cdot cos(35 \cdot x)}{35 \cdot \frac{sin(35 \cdot x)}{35 \cdot x}} = \frac{34}{35}$$
\rozwStop
\odpStart
$\frac{34}{35}$
\odpStop
\testStart
A.$\frac{34}{35}$
B.$\infty$
C.$-\infty$
D.$0$
E.$-\frac{34}{35}$
F.$\frac{35}{34}$
G.$-\frac{35}{34}$
H.$35$
I.$34$
\testStop
\kluczStart
A
\kluczStop



\zadStart{Przykład z Wikieł P 4.3a moja wersja nr 720}


Obliczyć granicę funkcji $\lim\limits_{x\to\ 0}\frac{34 \cdot x}{tan(37 \cdot x)}$.
\zadStop
\rozwStart{Patryk Wirkus}{}
$$\lim\limits_{x\to\ 0}\frac{34 \cdot x}{tan(37 \cdot x)}=\lim\limits_{x\to\ 0}\frac{34 \cdot x \cdot cos(37 \cdot x)}{sin(37 \cdot x)}=\lim\limits_{x\to\ 0}\frac{34 \cdot cos(37 \cdot x)}{\frac{sin(37 \cdot x)}{x}}=\lim\limits_{x\to\ 0}\frac{34 \cdot cos(37 \cdot x)}{37 \cdot \frac{sin(37 \cdot x)}{37 \cdot x}} = \frac{34}{37}$$
\rozwStop
\odpStart
$\frac{34}{37}$
\odpStop
\testStart
A.$\frac{34}{37}$
B.$\infty$
C.$-\infty$
D.$0$
E.$-\frac{34}{37}$
F.$\frac{37}{34}$
G.$-\frac{37}{34}$
H.$37$
I.$34$
\testStop
\kluczStart
A
\kluczStop



\zadStart{Przykład z Wikieł P 4.3a moja wersja nr 721}


Obliczyć granicę funkcji $\lim\limits_{x\to\ 0}\frac{34 \cdot x}{tan(39 \cdot x)}$.
\zadStop
\rozwStart{Patryk Wirkus}{}
$$\lim\limits_{x\to\ 0}\frac{34 \cdot x}{tan(39 \cdot x)}=\lim\limits_{x\to\ 0}\frac{34 \cdot x \cdot cos(39 \cdot x)}{sin(39 \cdot x)}=\lim\limits_{x\to\ 0}\frac{34 \cdot cos(39 \cdot x)}{\frac{sin(39 \cdot x)}{x}}=\lim\limits_{x\to\ 0}\frac{34 \cdot cos(39 \cdot x)}{39 \cdot \frac{sin(39 \cdot x)}{39 \cdot x}} = \frac{34}{39}$$
\rozwStop
\odpStart
$\frac{34}{39}$
\odpStop
\testStart
A.$\frac{34}{39}$
B.$\infty$
C.$-\infty$
D.$0$
E.$-\frac{34}{39}$
F.$\frac{39}{34}$
G.$-\frac{39}{34}$
H.$39$
I.$34$
\testStop
\kluczStart
A
\kluczStop



\zadStart{Przykład z Wikieł P 4.3a moja wersja nr 722}


Obliczyć granicę funkcji $\lim\limits_{x\to\ 0}\frac{35 \cdot x}{tan(2 \cdot x)}$.
\zadStop
\rozwStart{Patryk Wirkus}{}
$$\lim\limits_{x\to\ 0}\frac{35 \cdot x}{tan(2 \cdot x)}=\lim\limits_{x\to\ 0}\frac{35 \cdot x \cdot cos(2 \cdot x)}{sin(2 \cdot x)}=\lim\limits_{x\to\ 0}\frac{35 \cdot cos(2 \cdot x)}{\frac{sin(2 \cdot x)}{x}}=\lim\limits_{x\to\ 0}\frac{35 \cdot cos(2 \cdot x)}{2 \cdot \frac{sin(2 \cdot x)}{2 \cdot x}} = \frac{35}{2}$$
\rozwStop
\odpStart
$\frac{35}{2}$
\odpStop
\testStart
A.$\frac{35}{2}$
B.$\infty$
C.$-\infty$
D.$0$
E.$-\frac{35}{2}$
F.$\frac{2}{35}$
G.$-\frac{2}{35}$
H.$2$
I.$35$
\testStop
\kluczStart
A
\kluczStop



\zadStart{Przykład z Wikieł P 4.3a moja wersja nr 723}


Obliczyć granicę funkcji $\lim\limits_{x\to\ 0}\frac{35 \cdot x}{tan(3 \cdot x)}$.
\zadStop
\rozwStart{Patryk Wirkus}{}
$$\lim\limits_{x\to\ 0}\frac{35 \cdot x}{tan(3 \cdot x)}=\lim\limits_{x\to\ 0}\frac{35 \cdot x \cdot cos(3 \cdot x)}{sin(3 \cdot x)}=\lim\limits_{x\to\ 0}\frac{35 \cdot cos(3 \cdot x)}{\frac{sin(3 \cdot x)}{x}}=\lim\limits_{x\to\ 0}\frac{35 \cdot cos(3 \cdot x)}{3 \cdot \frac{sin(3 \cdot x)}{3 \cdot x}} = \frac{35}{3}$$
\rozwStop
\odpStart
$\frac{35}{3}$
\odpStop
\testStart
A.$\frac{35}{3}$
B.$\infty$
C.$-\infty$
D.$0$
E.$-\frac{35}{3}$
F.$\frac{3}{35}$
G.$-\frac{3}{35}$
H.$3$
I.$35$
\testStop
\kluczStart
A
\kluczStop



\zadStart{Przykład z Wikieł P 4.3a moja wersja nr 724}


Obliczyć granicę funkcji $\lim\limits_{x\to\ 0}\frac{35 \cdot x}{tan(4 \cdot x)}$.
\zadStop
\rozwStart{Patryk Wirkus}{}
$$\lim\limits_{x\to\ 0}\frac{35 \cdot x}{tan(4 \cdot x)}=\lim\limits_{x\to\ 0}\frac{35 \cdot x \cdot cos(4 \cdot x)}{sin(4 \cdot x)}=\lim\limits_{x\to\ 0}\frac{35 \cdot cos(4 \cdot x)}{\frac{sin(4 \cdot x)}{x}}=\lim\limits_{x\to\ 0}\frac{35 \cdot cos(4 \cdot x)}{4 \cdot \frac{sin(4 \cdot x)}{4 \cdot x}} = \frac{35}{4}$$
\rozwStop
\odpStart
$\frac{35}{4}$
\odpStop
\testStart
A.$\frac{35}{4}$
B.$\infty$
C.$-\infty$
D.$0$
E.$-\frac{35}{4}$
F.$\frac{4}{35}$
G.$-\frac{4}{35}$
H.$4$
I.$35$
\testStop
\kluczStart
A
\kluczStop



\zadStart{Przykład z Wikieł P 4.3a moja wersja nr 725}


Obliczyć granicę funkcji $\lim\limits_{x\to\ 0}\frac{35 \cdot x}{tan(6 \cdot x)}$.
\zadStop
\rozwStart{Patryk Wirkus}{}
$$\lim\limits_{x\to\ 0}\frac{35 \cdot x}{tan(6 \cdot x)}=\lim\limits_{x\to\ 0}\frac{35 \cdot x \cdot cos(6 \cdot x)}{sin(6 \cdot x)}=\lim\limits_{x\to\ 0}\frac{35 \cdot cos(6 \cdot x)}{\frac{sin(6 \cdot x)}{x}}=\lim\limits_{x\to\ 0}\frac{35 \cdot cos(6 \cdot x)}{6 \cdot \frac{sin(6 \cdot x)}{6 \cdot x}} = \frac{35}{6}$$
\rozwStop
\odpStart
$\frac{35}{6}$
\odpStop
\testStart
A.$\frac{35}{6}$
B.$\infty$
C.$-\infty$
D.$0$
E.$-\frac{35}{6}$
F.$\frac{6}{35}$
G.$-\frac{6}{35}$
H.$6$
I.$35$
\testStop
\kluczStart
A
\kluczStop



\zadStart{Przykład z Wikieł P 4.3a moja wersja nr 726}


Obliczyć granicę funkcji $\lim\limits_{x\to\ 0}\frac{35 \cdot x}{tan(8 \cdot x)}$.
\zadStop
\rozwStart{Patryk Wirkus}{}
$$\lim\limits_{x\to\ 0}\frac{35 \cdot x}{tan(8 \cdot x)}=\lim\limits_{x\to\ 0}\frac{35 \cdot x \cdot cos(8 \cdot x)}{sin(8 \cdot x)}=\lim\limits_{x\to\ 0}\frac{35 \cdot cos(8 \cdot x)}{\frac{sin(8 \cdot x)}{x}}=\lim\limits_{x\to\ 0}\frac{35 \cdot cos(8 \cdot x)}{8 \cdot \frac{sin(8 \cdot x)}{8 \cdot x}} = \frac{35}{8}$$
\rozwStop
\odpStart
$\frac{35}{8}$
\odpStop
\testStart
A.$\frac{35}{8}$
B.$\infty$
C.$-\infty$
D.$0$
E.$-\frac{35}{8}$
F.$\frac{8}{35}$
G.$-\frac{8}{35}$
H.$8$
I.$35$
\testStop
\kluczStart
A
\kluczStop



\zadStart{Przykład z Wikieł P 4.3a moja wersja nr 727}


Obliczyć granicę funkcji $\lim\limits_{x\to\ 0}\frac{35 \cdot x}{tan(9 \cdot x)}$.
\zadStop
\rozwStart{Patryk Wirkus}{}
$$\lim\limits_{x\to\ 0}\frac{35 \cdot x}{tan(9 \cdot x)}=\lim\limits_{x\to\ 0}\frac{35 \cdot x \cdot cos(9 \cdot x)}{sin(9 \cdot x)}=\lim\limits_{x\to\ 0}\frac{35 \cdot cos(9 \cdot x)}{\frac{sin(9 \cdot x)}{x}}=\lim\limits_{x\to\ 0}\frac{35 \cdot cos(9 \cdot x)}{9 \cdot \frac{sin(9 \cdot x)}{9 \cdot x}} = \frac{35}{9}$$
\rozwStop
\odpStart
$\frac{35}{9}$
\odpStop
\testStart
A.$\frac{35}{9}$
B.$\infty$
C.$-\infty$
D.$0$
E.$-\frac{35}{9}$
F.$\frac{9}{35}$
G.$-\frac{9}{35}$
H.$9$
I.$35$
\testStop
\kluczStart
A
\kluczStop



\zadStart{Przykład z Wikieł P 4.3a moja wersja nr 728}


Obliczyć granicę funkcji $\lim\limits_{x\to\ 0}\frac{35 \cdot x}{tan(11 \cdot x)}$.
\zadStop
\rozwStart{Patryk Wirkus}{}
$$\lim\limits_{x\to\ 0}\frac{35 \cdot x}{tan(11 \cdot x)}=\lim\limits_{x\to\ 0}\frac{35 \cdot x \cdot cos(11 \cdot x)}{sin(11 \cdot x)}=\lim\limits_{x\to\ 0}\frac{35 \cdot cos(11 \cdot x)}{\frac{sin(11 \cdot x)}{x}}=\lim\limits_{x\to\ 0}\frac{35 \cdot cos(11 \cdot x)}{11 \cdot \frac{sin(11 \cdot x)}{11 \cdot x}} = \frac{35}{11}$$
\rozwStop
\odpStart
$\frac{35}{11}$
\odpStop
\testStart
A.$\frac{35}{11}$
B.$\infty$
C.$-\infty$
D.$0$
E.$-\frac{35}{11}$
F.$\frac{11}{35}$
G.$-\frac{11}{35}$
H.$11$
I.$35$
\testStop
\kluczStart
A
\kluczStop



\zadStart{Przykład z Wikieł P 4.3a moja wersja nr 729}


Obliczyć granicę funkcji $\lim\limits_{x\to\ 0}\frac{35 \cdot x}{tan(12 \cdot x)}$.
\zadStop
\rozwStart{Patryk Wirkus}{}
$$\lim\limits_{x\to\ 0}\frac{35 \cdot x}{tan(12 \cdot x)}=\lim\limits_{x\to\ 0}\frac{35 \cdot x \cdot cos(12 \cdot x)}{sin(12 \cdot x)}=\lim\limits_{x\to\ 0}\frac{35 \cdot cos(12 \cdot x)}{\frac{sin(12 \cdot x)}{x}}=\lim\limits_{x\to\ 0}\frac{35 \cdot cos(12 \cdot x)}{12 \cdot \frac{sin(12 \cdot x)}{12 \cdot x}} = \frac{35}{12}$$
\rozwStop
\odpStart
$\frac{35}{12}$
\odpStop
\testStart
A.$\frac{35}{12}$
B.$\infty$
C.$-\infty$
D.$0$
E.$-\frac{35}{12}$
F.$\frac{12}{35}$
G.$-\frac{12}{35}$
H.$12$
I.$35$
\testStop
\kluczStart
A
\kluczStop



\zadStart{Przykład z Wikieł P 4.3a moja wersja nr 730}


Obliczyć granicę funkcji $\lim\limits_{x\to\ 0}\frac{35 \cdot x}{tan(13 \cdot x)}$.
\zadStop
\rozwStart{Patryk Wirkus}{}
$$\lim\limits_{x\to\ 0}\frac{35 \cdot x}{tan(13 \cdot x)}=\lim\limits_{x\to\ 0}\frac{35 \cdot x \cdot cos(13 \cdot x)}{sin(13 \cdot x)}=\lim\limits_{x\to\ 0}\frac{35 \cdot cos(13 \cdot x)}{\frac{sin(13 \cdot x)}{x}}=\lim\limits_{x\to\ 0}\frac{35 \cdot cos(13 \cdot x)}{13 \cdot \frac{sin(13 \cdot x)}{13 \cdot x}} = \frac{35}{13}$$
\rozwStop
\odpStart
$\frac{35}{13}$
\odpStop
\testStart
A.$\frac{35}{13}$
B.$\infty$
C.$-\infty$
D.$0$
E.$-\frac{35}{13}$
F.$\frac{13}{35}$
G.$-\frac{13}{35}$
H.$13$
I.$35$
\testStop
\kluczStart
A
\kluczStop



\zadStart{Przykład z Wikieł P 4.3a moja wersja nr 731}


Obliczyć granicę funkcji $\lim\limits_{x\to\ 0}\frac{35 \cdot x}{tan(16 \cdot x)}$.
\zadStop
\rozwStart{Patryk Wirkus}{}
$$\lim\limits_{x\to\ 0}\frac{35 \cdot x}{tan(16 \cdot x)}=\lim\limits_{x\to\ 0}\frac{35 \cdot x \cdot cos(16 \cdot x)}{sin(16 \cdot x)}=\lim\limits_{x\to\ 0}\frac{35 \cdot cos(16 \cdot x)}{\frac{sin(16 \cdot x)}{x}}=\lim\limits_{x\to\ 0}\frac{35 \cdot cos(16 \cdot x)}{16 \cdot \frac{sin(16 \cdot x)}{16 \cdot x}} = \frac{35}{16}$$
\rozwStop
\odpStart
$\frac{35}{16}$
\odpStop
\testStart
A.$\frac{35}{16}$
B.$\infty$
C.$-\infty$
D.$0$
E.$-\frac{35}{16}$
F.$\frac{16}{35}$
G.$-\frac{16}{35}$
H.$16$
I.$35$
\testStop
\kluczStart
A
\kluczStop



\zadStart{Przykład z Wikieł P 4.3a moja wersja nr 732}


Obliczyć granicę funkcji $\lim\limits_{x\to\ 0}\frac{35 \cdot x}{tan(17 \cdot x)}$.
\zadStop
\rozwStart{Patryk Wirkus}{}
$$\lim\limits_{x\to\ 0}\frac{35 \cdot x}{tan(17 \cdot x)}=\lim\limits_{x\to\ 0}\frac{35 \cdot x \cdot cos(17 \cdot x)}{sin(17 \cdot x)}=\lim\limits_{x\to\ 0}\frac{35 \cdot cos(17 \cdot x)}{\frac{sin(17 \cdot x)}{x}}=\lim\limits_{x\to\ 0}\frac{35 \cdot cos(17 \cdot x)}{17 \cdot \frac{sin(17 \cdot x)}{17 \cdot x}} = \frac{35}{17}$$
\rozwStop
\odpStart
$\frac{35}{17}$
\odpStop
\testStart
A.$\frac{35}{17}$
B.$\infty$
C.$-\infty$
D.$0$
E.$-\frac{35}{17}$
F.$\frac{17}{35}$
G.$-\frac{17}{35}$
H.$17$
I.$35$
\testStop
\kluczStart
A
\kluczStop



\zadStart{Przykład z Wikieł P 4.3a moja wersja nr 733}


Obliczyć granicę funkcji $\lim\limits_{x\to\ 0}\frac{35 \cdot x}{tan(18 \cdot x)}$.
\zadStop
\rozwStart{Patryk Wirkus}{}
$$\lim\limits_{x\to\ 0}\frac{35 \cdot x}{tan(18 \cdot x)}=\lim\limits_{x\to\ 0}\frac{35 \cdot x \cdot cos(18 \cdot x)}{sin(18 \cdot x)}=\lim\limits_{x\to\ 0}\frac{35 \cdot cos(18 \cdot x)}{\frac{sin(18 \cdot x)}{x}}=\lim\limits_{x\to\ 0}\frac{35 \cdot cos(18 \cdot x)}{18 \cdot \frac{sin(18 \cdot x)}{18 \cdot x}} = \frac{35}{18}$$
\rozwStop
\odpStart
$\frac{35}{18}$
\odpStop
\testStart
A.$\frac{35}{18}$
B.$\infty$
C.$-\infty$
D.$0$
E.$-\frac{35}{18}$
F.$\frac{18}{35}$
G.$-\frac{18}{35}$
H.$18$
I.$35$
\testStop
\kluczStart
A
\kluczStop



\zadStart{Przykład z Wikieł P 4.3a moja wersja nr 734}


Obliczyć granicę funkcji $\lim\limits_{x\to\ 0}\frac{35 \cdot x}{tan(19 \cdot x)}$.
\zadStop
\rozwStart{Patryk Wirkus}{}
$$\lim\limits_{x\to\ 0}\frac{35 \cdot x}{tan(19 \cdot x)}=\lim\limits_{x\to\ 0}\frac{35 \cdot x \cdot cos(19 \cdot x)}{sin(19 \cdot x)}=\lim\limits_{x\to\ 0}\frac{35 \cdot cos(19 \cdot x)}{\frac{sin(19 \cdot x)}{x}}=\lim\limits_{x\to\ 0}\frac{35 \cdot cos(19 \cdot x)}{19 \cdot \frac{sin(19 \cdot x)}{19 \cdot x}} = \frac{35}{19}$$
\rozwStop
\odpStart
$\frac{35}{19}$
\odpStop
\testStart
A.$\frac{35}{19}$
B.$\infty$
C.$-\infty$
D.$0$
E.$-\frac{35}{19}$
F.$\frac{19}{35}$
G.$-\frac{19}{35}$
H.$19$
I.$35$
\testStop
\kluczStart
A
\kluczStop



\zadStart{Przykład z Wikieł P 4.3a moja wersja nr 735}


Obliczyć granicę funkcji $\lim\limits_{x\to\ 0}\frac{35 \cdot x}{tan(22 \cdot x)}$.
\zadStop
\rozwStart{Patryk Wirkus}{}
$$\lim\limits_{x\to\ 0}\frac{35 \cdot x}{tan(22 \cdot x)}=\lim\limits_{x\to\ 0}\frac{35 \cdot x \cdot cos(22 \cdot x)}{sin(22 \cdot x)}=\lim\limits_{x\to\ 0}\frac{35 \cdot cos(22 \cdot x)}{\frac{sin(22 \cdot x)}{x}}=\lim\limits_{x\to\ 0}\frac{35 \cdot cos(22 \cdot x)}{22 \cdot \frac{sin(22 \cdot x)}{22 \cdot x}} = \frac{35}{22}$$
\rozwStop
\odpStart
$\frac{35}{22}$
\odpStop
\testStart
A.$\frac{35}{22}$
B.$\infty$
C.$-\infty$
D.$0$
E.$-\frac{35}{22}$
F.$\frac{22}{35}$
G.$-\frac{22}{35}$
H.$22$
I.$35$
\testStop
\kluczStart
A
\kluczStop



\zadStart{Przykład z Wikieł P 4.3a moja wersja nr 736}


Obliczyć granicę funkcji $\lim\limits_{x\to\ 0}\frac{35 \cdot x}{tan(23 \cdot x)}$.
\zadStop
\rozwStart{Patryk Wirkus}{}
$$\lim\limits_{x\to\ 0}\frac{35 \cdot x}{tan(23 \cdot x)}=\lim\limits_{x\to\ 0}\frac{35 \cdot x \cdot cos(23 \cdot x)}{sin(23 \cdot x)}=\lim\limits_{x\to\ 0}\frac{35 \cdot cos(23 \cdot x)}{\frac{sin(23 \cdot x)}{x}}=\lim\limits_{x\to\ 0}\frac{35 \cdot cos(23 \cdot x)}{23 \cdot \frac{sin(23 \cdot x)}{23 \cdot x}} = \frac{35}{23}$$
\rozwStop
\odpStart
$\frac{35}{23}$
\odpStop
\testStart
A.$\frac{35}{23}$
B.$\infty$
C.$-\infty$
D.$0$
E.$-\frac{35}{23}$
F.$\frac{23}{35}$
G.$-\frac{23}{35}$
H.$23$
I.$35$
\testStop
\kluczStart
A
\kluczStop



\zadStart{Przykład z Wikieł P 4.3a moja wersja nr 737}


Obliczyć granicę funkcji $\lim\limits_{x\to\ 0}\frac{35 \cdot x}{tan(24 \cdot x)}$.
\zadStop
\rozwStart{Patryk Wirkus}{}
$$\lim\limits_{x\to\ 0}\frac{35 \cdot x}{tan(24 \cdot x)}=\lim\limits_{x\to\ 0}\frac{35 \cdot x \cdot cos(24 \cdot x)}{sin(24 \cdot x)}=\lim\limits_{x\to\ 0}\frac{35 \cdot cos(24 \cdot x)}{\frac{sin(24 \cdot x)}{x}}=\lim\limits_{x\to\ 0}\frac{35 \cdot cos(24 \cdot x)}{24 \cdot \frac{sin(24 \cdot x)}{24 \cdot x}} = \frac{35}{24}$$
\rozwStop
\odpStart
$\frac{35}{24}$
\odpStop
\testStart
A.$\frac{35}{24}$
B.$\infty$
C.$-\infty$
D.$0$
E.$-\frac{35}{24}$
F.$\frac{24}{35}$
G.$-\frac{24}{35}$
H.$24$
I.$35$
\testStop
\kluczStart
A
\kluczStop



\zadStart{Przykład z Wikieł P 4.3a moja wersja nr 738}


Obliczyć granicę funkcji $\lim\limits_{x\to\ 0}\frac{35 \cdot x}{tan(26 \cdot x)}$.
\zadStop
\rozwStart{Patryk Wirkus}{}
$$\lim\limits_{x\to\ 0}\frac{35 \cdot x}{tan(26 \cdot x)}=\lim\limits_{x\to\ 0}\frac{35 \cdot x \cdot cos(26 \cdot x)}{sin(26 \cdot x)}=\lim\limits_{x\to\ 0}\frac{35 \cdot cos(26 \cdot x)}{\frac{sin(26 \cdot x)}{x}}=\lim\limits_{x\to\ 0}\frac{35 \cdot cos(26 \cdot x)}{26 \cdot \frac{sin(26 \cdot x)}{26 \cdot x}} = \frac{35}{26}$$
\rozwStop
\odpStart
$\frac{35}{26}$
\odpStop
\testStart
A.$\frac{35}{26}$
B.$\infty$
C.$-\infty$
D.$0$
E.$-\frac{35}{26}$
F.$\frac{26}{35}$
G.$-\frac{26}{35}$
H.$26$
I.$35$
\testStop
\kluczStart
A
\kluczStop



\zadStart{Przykład z Wikieł P 4.3a moja wersja nr 739}


Obliczyć granicę funkcji $\lim\limits_{x\to\ 0}\frac{35 \cdot x}{tan(27 \cdot x)}$.
\zadStop
\rozwStart{Patryk Wirkus}{}
$$\lim\limits_{x\to\ 0}\frac{35 \cdot x}{tan(27 \cdot x)}=\lim\limits_{x\to\ 0}\frac{35 \cdot x \cdot cos(27 \cdot x)}{sin(27 \cdot x)}=\lim\limits_{x\to\ 0}\frac{35 \cdot cos(27 \cdot x)}{\frac{sin(27 \cdot x)}{x}}=\lim\limits_{x\to\ 0}\frac{35 \cdot cos(27 \cdot x)}{27 \cdot \frac{sin(27 \cdot x)}{27 \cdot x}} = \frac{35}{27}$$
\rozwStop
\odpStart
$\frac{35}{27}$
\odpStop
\testStart
A.$\frac{35}{27}$
B.$\infty$
C.$-\infty$
D.$0$
E.$-\frac{35}{27}$
F.$\frac{27}{35}$
G.$-\frac{27}{35}$
H.$27$
I.$35$
\testStop
\kluczStart
A
\kluczStop



\zadStart{Przykład z Wikieł P 4.3a moja wersja nr 740}


Obliczyć granicę funkcji $\lim\limits_{x\to\ 0}\frac{35 \cdot x}{tan(29 \cdot x)}$.
\zadStop
\rozwStart{Patryk Wirkus}{}
$$\lim\limits_{x\to\ 0}\frac{35 \cdot x}{tan(29 \cdot x)}=\lim\limits_{x\to\ 0}\frac{35 \cdot x \cdot cos(29 \cdot x)}{sin(29 \cdot x)}=\lim\limits_{x\to\ 0}\frac{35 \cdot cos(29 \cdot x)}{\frac{sin(29 \cdot x)}{x}}=\lim\limits_{x\to\ 0}\frac{35 \cdot cos(29 \cdot x)}{29 \cdot \frac{sin(29 \cdot x)}{29 \cdot x}} = \frac{35}{29}$$
\rozwStop
\odpStart
$\frac{35}{29}$
\odpStop
\testStart
A.$\frac{35}{29}$
B.$\infty$
C.$-\infty$
D.$0$
E.$-\frac{35}{29}$
F.$\frac{29}{35}$
G.$-\frac{29}{35}$
H.$29$
I.$35$
\testStop
\kluczStart
A
\kluczStop



\zadStart{Przykład z Wikieł P 4.3a moja wersja nr 741}


Obliczyć granicę funkcji $\lim\limits_{x\to\ 0}\frac{35 \cdot x}{tan(31 \cdot x)}$.
\zadStop
\rozwStart{Patryk Wirkus}{}
$$\lim\limits_{x\to\ 0}\frac{35 \cdot x}{tan(31 \cdot x)}=\lim\limits_{x\to\ 0}\frac{35 \cdot x \cdot cos(31 \cdot x)}{sin(31 \cdot x)}=\lim\limits_{x\to\ 0}\frac{35 \cdot cos(31 \cdot x)}{\frac{sin(31 \cdot x)}{x}}=\lim\limits_{x\to\ 0}\frac{35 \cdot cos(31 \cdot x)}{31 \cdot \frac{sin(31 \cdot x)}{31 \cdot x}} = \frac{35}{31}$$
\rozwStop
\odpStart
$\frac{35}{31}$
\odpStop
\testStart
A.$\frac{35}{31}$
B.$\infty$
C.$-\infty$
D.$0$
E.$-\frac{35}{31}$
F.$\frac{31}{35}$
G.$-\frac{31}{35}$
H.$31$
I.$35$
\testStop
\kluczStart
A
\kluczStop



\zadStart{Przykład z Wikieł P 4.3a moja wersja nr 742}


Obliczyć granicę funkcji $\lim\limits_{x\to\ 0}\frac{35 \cdot x}{tan(32 \cdot x)}$.
\zadStop
\rozwStart{Patryk Wirkus}{}
$$\lim\limits_{x\to\ 0}\frac{35 \cdot x}{tan(32 \cdot x)}=\lim\limits_{x\to\ 0}\frac{35 \cdot x \cdot cos(32 \cdot x)}{sin(32 \cdot x)}=\lim\limits_{x\to\ 0}\frac{35 \cdot cos(32 \cdot x)}{\frac{sin(32 \cdot x)}{x}}=\lim\limits_{x\to\ 0}\frac{35 \cdot cos(32 \cdot x)}{32 \cdot \frac{sin(32 \cdot x)}{32 \cdot x}} = \frac{35}{32}$$
\rozwStop
\odpStart
$\frac{35}{32}$
\odpStop
\testStart
A.$\frac{35}{32}$
B.$\infty$
C.$-\infty$
D.$0$
E.$-\frac{35}{32}$
F.$\frac{32}{35}$
G.$-\frac{32}{35}$
H.$32$
I.$35$
\testStop
\kluczStart
A
\kluczStop



\zadStart{Przykład z Wikieł P 4.3a moja wersja nr 743}


Obliczyć granicę funkcji $\lim\limits_{x\to\ 0}\frac{35 \cdot x}{tan(33 \cdot x)}$.
\zadStop
\rozwStart{Patryk Wirkus}{}
$$\lim\limits_{x\to\ 0}\frac{35 \cdot x}{tan(33 \cdot x)}=\lim\limits_{x\to\ 0}\frac{35 \cdot x \cdot cos(33 \cdot x)}{sin(33 \cdot x)}=\lim\limits_{x\to\ 0}\frac{35 \cdot cos(33 \cdot x)}{\frac{sin(33 \cdot x)}{x}}=\lim\limits_{x\to\ 0}\frac{35 \cdot cos(33 \cdot x)}{33 \cdot \frac{sin(33 \cdot x)}{33 \cdot x}} = \frac{35}{33}$$
\rozwStop
\odpStart
$\frac{35}{33}$
\odpStop
\testStart
A.$\frac{35}{33}$
B.$\infty$
C.$-\infty$
D.$0$
E.$-\frac{35}{33}$
F.$\frac{33}{35}$
G.$-\frac{33}{35}$
H.$33$
I.$35$
\testStop
\kluczStart
A
\kluczStop



\zadStart{Przykład z Wikieł P 4.3a moja wersja nr 744}


Obliczyć granicę funkcji $\lim\limits_{x\to\ 0}\frac{35 \cdot x}{tan(34 \cdot x)}$.
\zadStop
\rozwStart{Patryk Wirkus}{}
$$\lim\limits_{x\to\ 0}\frac{35 \cdot x}{tan(34 \cdot x)}=\lim\limits_{x\to\ 0}\frac{35 \cdot x \cdot cos(34 \cdot x)}{sin(34 \cdot x)}=\lim\limits_{x\to\ 0}\frac{35 \cdot cos(34 \cdot x)}{\frac{sin(34 \cdot x)}{x}}=\lim\limits_{x\to\ 0}\frac{35 \cdot cos(34 \cdot x)}{34 \cdot \frac{sin(34 \cdot x)}{34 \cdot x}} = \frac{35}{34}$$
\rozwStop
\odpStart
$\frac{35}{34}$
\odpStop
\testStart
A.$\frac{35}{34}$
B.$\infty$
C.$-\infty$
D.$0$
E.$-\frac{35}{34}$
F.$\frac{34}{35}$
G.$-\frac{34}{35}$
H.$34$
I.$35$
\testStop
\kluczStart
A
\kluczStop



\zadStart{Przykład z Wikieł P 4.3a moja wersja nr 745}


Obliczyć granicę funkcji $\lim\limits_{x\to\ 0}\frac{35 \cdot x}{tan(36 \cdot x)}$.
\zadStop
\rozwStart{Patryk Wirkus}{}
$$\lim\limits_{x\to\ 0}\frac{35 \cdot x}{tan(36 \cdot x)}=\lim\limits_{x\to\ 0}\frac{35 \cdot x \cdot cos(36 \cdot x)}{sin(36 \cdot x)}=\lim\limits_{x\to\ 0}\frac{35 \cdot cos(36 \cdot x)}{\frac{sin(36 \cdot x)}{x}}=\lim\limits_{x\to\ 0}\frac{35 \cdot cos(36 \cdot x)}{36 \cdot \frac{sin(36 \cdot x)}{36 \cdot x}} = \frac{35}{36}$$
\rozwStop
\odpStart
$\frac{35}{36}$
\odpStop
\testStart
A.$\frac{35}{36}$
B.$\infty$
C.$-\infty$
D.$0$
E.$-\frac{35}{36}$
F.$\frac{36}{35}$
G.$-\frac{36}{35}$
H.$36$
I.$35$
\testStop
\kluczStart
A
\kluczStop



\zadStart{Przykład z Wikieł P 4.3a moja wersja nr 746}


Obliczyć granicę funkcji $\lim\limits_{x\to\ 0}\frac{35 \cdot x}{tan(37 \cdot x)}$.
\zadStop
\rozwStart{Patryk Wirkus}{}
$$\lim\limits_{x\to\ 0}\frac{35 \cdot x}{tan(37 \cdot x)}=\lim\limits_{x\to\ 0}\frac{35 \cdot x \cdot cos(37 \cdot x)}{sin(37 \cdot x)}=\lim\limits_{x\to\ 0}\frac{35 \cdot cos(37 \cdot x)}{\frac{sin(37 \cdot x)}{x}}=\lim\limits_{x\to\ 0}\frac{35 \cdot cos(37 \cdot x)}{37 \cdot \frac{sin(37 \cdot x)}{37 \cdot x}} = \frac{35}{37}$$
\rozwStop
\odpStart
$\frac{35}{37}$
\odpStop
\testStart
A.$\frac{35}{37}$
B.$\infty$
C.$-\infty$
D.$0$
E.$-\frac{35}{37}$
F.$\frac{37}{35}$
G.$-\frac{37}{35}$
H.$37$
I.$35$
\testStop
\kluczStart
A
\kluczStop



\zadStart{Przykład z Wikieł P 4.3a moja wersja nr 747}


Obliczyć granicę funkcji $\lim\limits_{x\to\ 0}\frac{35 \cdot x}{tan(38 \cdot x)}$.
\zadStop
\rozwStart{Patryk Wirkus}{}
$$\lim\limits_{x\to\ 0}\frac{35 \cdot x}{tan(38 \cdot x)}=\lim\limits_{x\to\ 0}\frac{35 \cdot x \cdot cos(38 \cdot x)}{sin(38 \cdot x)}=\lim\limits_{x\to\ 0}\frac{35 \cdot cos(38 \cdot x)}{\frac{sin(38 \cdot x)}{x}}=\lim\limits_{x\to\ 0}\frac{35 \cdot cos(38 \cdot x)}{38 \cdot \frac{sin(38 \cdot x)}{38 \cdot x}} = \frac{35}{38}$$
\rozwStop
\odpStart
$\frac{35}{38}$
\odpStop
\testStart
A.$\frac{35}{38}$
B.$\infty$
C.$-\infty$
D.$0$
E.$-\frac{35}{38}$
F.$\frac{38}{35}$
G.$-\frac{38}{35}$
H.$38$
I.$35$
\testStop
\kluczStart
A
\kluczStop



\zadStart{Przykład z Wikieł P 4.3a moja wersja nr 748}


Obliczyć granicę funkcji $\lim\limits_{x\to\ 0}\frac{35 \cdot x}{tan(39 \cdot x)}$.
\zadStop
\rozwStart{Patryk Wirkus}{}
$$\lim\limits_{x\to\ 0}\frac{35 \cdot x}{tan(39 \cdot x)}=\lim\limits_{x\to\ 0}\frac{35 \cdot x \cdot cos(39 \cdot x)}{sin(39 \cdot x)}=\lim\limits_{x\to\ 0}\frac{35 \cdot cos(39 \cdot x)}{\frac{sin(39 \cdot x)}{x}}=\lim\limits_{x\to\ 0}\frac{35 \cdot cos(39 \cdot x)}{39 \cdot \frac{sin(39 \cdot x)}{39 \cdot x}} = \frac{35}{39}$$
\rozwStop
\odpStart
$\frac{35}{39}$
\odpStop
\testStart
A.$\frac{35}{39}$
B.$\infty$
C.$-\infty$
D.$0$
E.$-\frac{35}{39}$
F.$\frac{39}{35}$
G.$-\frac{39}{35}$
H.$39$
I.$35$
\testStop
\kluczStart
A
\kluczStop



\zadStart{Przykład z Wikieł P 4.3a moja wersja nr 749}


Obliczyć granicę funkcji $\lim\limits_{x\to\ 0}\frac{36 \cdot x}{tan(5 \cdot x)}$.
\zadStop
\rozwStart{Patryk Wirkus}{}
$$\lim\limits_{x\to\ 0}\frac{36 \cdot x}{tan(5 \cdot x)}=\lim\limits_{x\to\ 0}\frac{36 \cdot x \cdot cos(5 \cdot x)}{sin(5 \cdot x)}=\lim\limits_{x\to\ 0}\frac{36 \cdot cos(5 \cdot x)}{\frac{sin(5 \cdot x)}{x}}=\lim\limits_{x\to\ 0}\frac{36 \cdot cos(5 \cdot x)}{5 \cdot \frac{sin(5 \cdot x)}{5 \cdot x}} = \frac{36}{5}$$
\rozwStop
\odpStart
$\frac{36}{5}$
\odpStop
\testStart
A.$\frac{36}{5}$
B.$\infty$
C.$-\infty$
D.$0$
E.$-\frac{36}{5}$
F.$\frac{5}{36}$
G.$-\frac{5}{36}$
H.$5$
I.$36$
\testStop
\kluczStart
A
\kluczStop



\zadStart{Przykład z Wikieł P 4.3a moja wersja nr 750}


Obliczyć granicę funkcji $\lim\limits_{x\to\ 0}\frac{36 \cdot x}{tan(7 \cdot x)}$.
\zadStop
\rozwStart{Patryk Wirkus}{}
$$\lim\limits_{x\to\ 0}\frac{36 \cdot x}{tan(7 \cdot x)}=\lim\limits_{x\to\ 0}\frac{36 \cdot x \cdot cos(7 \cdot x)}{sin(7 \cdot x)}=\lim\limits_{x\to\ 0}\frac{36 \cdot cos(7 \cdot x)}{\frac{sin(7 \cdot x)}{x}}=\lim\limits_{x\to\ 0}\frac{36 \cdot cos(7 \cdot x)}{7 \cdot \frac{sin(7 \cdot x)}{7 \cdot x}} = \frac{36}{7}$$
\rozwStop
\odpStart
$\frac{36}{7}$
\odpStop
\testStart
A.$\frac{36}{7}$
B.$\infty$
C.$-\infty$
D.$0$
E.$-\frac{36}{7}$
F.$\frac{7}{36}$
G.$-\frac{7}{36}$
H.$7$
I.$36$
\testStop
\kluczStart
A
\kluczStop



\zadStart{Przykład z Wikieł P 4.3a moja wersja nr 751}


Obliczyć granicę funkcji $\lim\limits_{x\to\ 0}\frac{36 \cdot x}{tan(11 \cdot x)}$.
\zadStop
\rozwStart{Patryk Wirkus}{}
$$\lim\limits_{x\to\ 0}\frac{36 \cdot x}{tan(11 \cdot x)}=\lim\limits_{x\to\ 0}\frac{36 \cdot x \cdot cos(11 \cdot x)}{sin(11 \cdot x)}=\lim\limits_{x\to\ 0}\frac{36 \cdot cos(11 \cdot x)}{\frac{sin(11 \cdot x)}{x}}=\lim\limits_{x\to\ 0}\frac{36 \cdot cos(11 \cdot x)}{11 \cdot \frac{sin(11 \cdot x)}{11 \cdot x}} = \frac{36}{11}$$
\rozwStop
\odpStart
$\frac{36}{11}$
\odpStop
\testStart
A.$\frac{36}{11}$
B.$\infty$
C.$-\infty$
D.$0$
E.$-\frac{36}{11}$
F.$\frac{11}{36}$
G.$-\frac{11}{36}$
H.$11$
I.$36$
\testStop
\kluczStart
A
\kluczStop



\zadStart{Przykład z Wikieł P 4.3a moja wersja nr 752}


Obliczyć granicę funkcji $\lim\limits_{x\to\ 0}\frac{36 \cdot x}{tan(13 \cdot x)}$.
\zadStop
\rozwStart{Patryk Wirkus}{}
$$\lim\limits_{x\to\ 0}\frac{36 \cdot x}{tan(13 \cdot x)}=\lim\limits_{x\to\ 0}\frac{36 \cdot x \cdot cos(13 \cdot x)}{sin(13 \cdot x)}=\lim\limits_{x\to\ 0}\frac{36 \cdot cos(13 \cdot x)}{\frac{sin(13 \cdot x)}{x}}=\lim\limits_{x\to\ 0}\frac{36 \cdot cos(13 \cdot x)}{13 \cdot \frac{sin(13 \cdot x)}{13 \cdot x}} = \frac{36}{13}$$
\rozwStop
\odpStart
$\frac{36}{13}$
\odpStop
\testStart
A.$\frac{36}{13}$
B.$\infty$
C.$-\infty$
D.$0$
E.$-\frac{36}{13}$
F.$\frac{13}{36}$
G.$-\frac{13}{36}$
H.$13$
I.$36$
\testStop
\kluczStart
A
\kluczStop



\zadStart{Przykład z Wikieł P 4.3a moja wersja nr 753}


Obliczyć granicę funkcji $\lim\limits_{x\to\ 0}\frac{36 \cdot x}{tan(17 \cdot x)}$.
\zadStop
\rozwStart{Patryk Wirkus}{}
$$\lim\limits_{x\to\ 0}\frac{36 \cdot x}{tan(17 \cdot x)}=\lim\limits_{x\to\ 0}\frac{36 \cdot x \cdot cos(17 \cdot x)}{sin(17 \cdot x)}=\lim\limits_{x\to\ 0}\frac{36 \cdot cos(17 \cdot x)}{\frac{sin(17 \cdot x)}{x}}=\lim\limits_{x\to\ 0}\frac{36 \cdot cos(17 \cdot x)}{17 \cdot \frac{sin(17 \cdot x)}{17 \cdot x}} = \frac{36}{17}$$
\rozwStop
\odpStart
$\frac{36}{17}$
\odpStop
\testStart
A.$\frac{36}{17}$
B.$\infty$
C.$-\infty$
D.$0$
E.$-\frac{36}{17}$
F.$\frac{17}{36}$
G.$-\frac{17}{36}$
H.$17$
I.$36$
\testStop
\kluczStart
A
\kluczStop



\zadStart{Przykład z Wikieł P 4.3a moja wersja nr 754}


Obliczyć granicę funkcji $\lim\limits_{x\to\ 0}\frac{36 \cdot x}{tan(19 \cdot x)}$.
\zadStop
\rozwStart{Patryk Wirkus}{}
$$\lim\limits_{x\to\ 0}\frac{36 \cdot x}{tan(19 \cdot x)}=\lim\limits_{x\to\ 0}\frac{36 \cdot x \cdot cos(19 \cdot x)}{sin(19 \cdot x)}=\lim\limits_{x\to\ 0}\frac{36 \cdot cos(19 \cdot x)}{\frac{sin(19 \cdot x)}{x}}=\lim\limits_{x\to\ 0}\frac{36 \cdot cos(19 \cdot x)}{19 \cdot \frac{sin(19 \cdot x)}{19 \cdot x}} = \frac{36}{19}$$
\rozwStop
\odpStart
$\frac{36}{19}$
\odpStop
\testStart
A.$\frac{36}{19}$
B.$\infty$
C.$-\infty$
D.$0$
E.$-\frac{36}{19}$
F.$\frac{19}{36}$
G.$-\frac{19}{36}$
H.$19$
I.$36$
\testStop
\kluczStart
A
\kluczStop



\zadStart{Przykład z Wikieł P 4.3a moja wersja nr 755}


Obliczyć granicę funkcji $\lim\limits_{x\to\ 0}\frac{36 \cdot x}{tan(23 \cdot x)}$.
\zadStop
\rozwStart{Patryk Wirkus}{}
$$\lim\limits_{x\to\ 0}\frac{36 \cdot x}{tan(23 \cdot x)}=\lim\limits_{x\to\ 0}\frac{36 \cdot x \cdot cos(23 \cdot x)}{sin(23 \cdot x)}=\lim\limits_{x\to\ 0}\frac{36 \cdot cos(23 \cdot x)}{\frac{sin(23 \cdot x)}{x}}=\lim\limits_{x\to\ 0}\frac{36 \cdot cos(23 \cdot x)}{23 \cdot \frac{sin(23 \cdot x)}{23 \cdot x}} = \frac{36}{23}$$
\rozwStop
\odpStart
$\frac{36}{23}$
\odpStop
\testStart
A.$\frac{36}{23}$
B.$\infty$
C.$-\infty$
D.$0$
E.$-\frac{36}{23}$
F.$\frac{23}{36}$
G.$-\frac{23}{36}$
H.$23$
I.$36$
\testStop
\kluczStart
A
\kluczStop



\zadStart{Przykład z Wikieł P 4.3a moja wersja nr 756}


Obliczyć granicę funkcji $\lim\limits_{x\to\ 0}\frac{36 \cdot x}{tan(25 \cdot x)}$.
\zadStop
\rozwStart{Patryk Wirkus}{}
$$\lim\limits_{x\to\ 0}\frac{36 \cdot x}{tan(25 \cdot x)}=\lim\limits_{x\to\ 0}\frac{36 \cdot x \cdot cos(25 \cdot x)}{sin(25 \cdot x)}=\lim\limits_{x\to\ 0}\frac{36 \cdot cos(25 \cdot x)}{\frac{sin(25 \cdot x)}{x}}=\lim\limits_{x\to\ 0}\frac{36 \cdot cos(25 \cdot x)}{25 \cdot \frac{sin(25 \cdot x)}{25 \cdot x}} = \frac{36}{25}$$
\rozwStop
\odpStart
$\frac{36}{25}$
\odpStop
\testStart
A.$\frac{36}{25}$
B.$\infty$
C.$-\infty$
D.$0$
E.$-\frac{36}{25}$
F.$\frac{25}{36}$
G.$-\frac{25}{36}$
H.$25$
I.$36$
\testStop
\kluczStart
A
\kluczStop



\zadStart{Przykład z Wikieł P 4.3a moja wersja nr 757}


Obliczyć granicę funkcji $\lim\limits_{x\to\ 0}\frac{36 \cdot x}{tan(29 \cdot x)}$.
\zadStop
\rozwStart{Patryk Wirkus}{}
$$\lim\limits_{x\to\ 0}\frac{36 \cdot x}{tan(29 \cdot x)}=\lim\limits_{x\to\ 0}\frac{36 \cdot x \cdot cos(29 \cdot x)}{sin(29 \cdot x)}=\lim\limits_{x\to\ 0}\frac{36 \cdot cos(29 \cdot x)}{\frac{sin(29 \cdot x)}{x}}=\lim\limits_{x\to\ 0}\frac{36 \cdot cos(29 \cdot x)}{29 \cdot \frac{sin(29 \cdot x)}{29 \cdot x}} = \frac{36}{29}$$
\rozwStop
\odpStart
$\frac{36}{29}$
\odpStop
\testStart
A.$\frac{36}{29}$
B.$\infty$
C.$-\infty$
D.$0$
E.$-\frac{36}{29}$
F.$\frac{29}{36}$
G.$-\frac{29}{36}$
H.$29$
I.$36$
\testStop
\kluczStart
A
\kluczStop



\zadStart{Przykład z Wikieł P 4.3a moja wersja nr 758}


Obliczyć granicę funkcji $\lim\limits_{x\to\ 0}\frac{36 \cdot x}{tan(31 \cdot x)}$.
\zadStop
\rozwStart{Patryk Wirkus}{}
$$\lim\limits_{x\to\ 0}\frac{36 \cdot x}{tan(31 \cdot x)}=\lim\limits_{x\to\ 0}\frac{36 \cdot x \cdot cos(31 \cdot x)}{sin(31 \cdot x)}=\lim\limits_{x\to\ 0}\frac{36 \cdot cos(31 \cdot x)}{\frac{sin(31 \cdot x)}{x}}=\lim\limits_{x\to\ 0}\frac{36 \cdot cos(31 \cdot x)}{31 \cdot \frac{sin(31 \cdot x)}{31 \cdot x}} = \frac{36}{31}$$
\rozwStop
\odpStart
$\frac{36}{31}$
\odpStop
\testStart
A.$\frac{36}{31}$
B.$\infty$
C.$-\infty$
D.$0$
E.$-\frac{36}{31}$
F.$\frac{31}{36}$
G.$-\frac{31}{36}$
H.$31$
I.$36$
\testStop
\kluczStart
A
\kluczStop



\zadStart{Przykład z Wikieł P 4.3a moja wersja nr 759}


Obliczyć granicę funkcji $\lim\limits_{x\to\ 0}\frac{36 \cdot x}{tan(35 \cdot x)}$.
\zadStop
\rozwStart{Patryk Wirkus}{}
$$\lim\limits_{x\to\ 0}\frac{36 \cdot x}{tan(35 \cdot x)}=\lim\limits_{x\to\ 0}\frac{36 \cdot x \cdot cos(35 \cdot x)}{sin(35 \cdot x)}=\lim\limits_{x\to\ 0}\frac{36 \cdot cos(35 \cdot x)}{\frac{sin(35 \cdot x)}{x}}=\lim\limits_{x\to\ 0}\frac{36 \cdot cos(35 \cdot x)}{35 \cdot \frac{sin(35 \cdot x)}{35 \cdot x}} = \frac{36}{35}$$
\rozwStop
\odpStart
$\frac{36}{35}$
\odpStop
\testStart
A.$\frac{36}{35}$
B.$\infty$
C.$-\infty$
D.$0$
E.$-\frac{36}{35}$
F.$\frac{35}{36}$
G.$-\frac{35}{36}$
H.$35$
I.$36$
\testStop
\kluczStart
A
\kluczStop



\zadStart{Przykład z Wikieł P 4.3a moja wersja nr 760}


Obliczyć granicę funkcji $\lim\limits_{x\to\ 0}\frac{36 \cdot x}{tan(37 \cdot x)}$.
\zadStop
\rozwStart{Patryk Wirkus}{}
$$\lim\limits_{x\to\ 0}\frac{36 \cdot x}{tan(37 \cdot x)}=\lim\limits_{x\to\ 0}\frac{36 \cdot x \cdot cos(37 \cdot x)}{sin(37 \cdot x)}=\lim\limits_{x\to\ 0}\frac{36 \cdot cos(37 \cdot x)}{\frac{sin(37 \cdot x)}{x}}=\lim\limits_{x\to\ 0}\frac{36 \cdot cos(37 \cdot x)}{37 \cdot \frac{sin(37 \cdot x)}{37 \cdot x}} = \frac{36}{37}$$
\rozwStop
\odpStart
$\frac{36}{37}$
\odpStop
\testStart
A.$\frac{36}{37}$
B.$\infty$
C.$-\infty$
D.$0$
E.$-\frac{36}{37}$
F.$\frac{37}{36}$
G.$-\frac{37}{36}$
H.$37$
I.$36$
\testStop
\kluczStart
A
\kluczStop



\zadStart{Przykład z Wikieł P 4.3a moja wersja nr 761}


Obliczyć granicę funkcji $\lim\limits_{x\to\ 0}\frac{37 \cdot x}{tan(2 \cdot x)}$.
\zadStop
\rozwStart{Patryk Wirkus}{}
$$\lim\limits_{x\to\ 0}\frac{37 \cdot x}{tan(2 \cdot x)}=\lim\limits_{x\to\ 0}\frac{37 \cdot x \cdot cos(2 \cdot x)}{sin(2 \cdot x)}=\lim\limits_{x\to\ 0}\frac{37 \cdot cos(2 \cdot x)}{\frac{sin(2 \cdot x)}{x}}=\lim\limits_{x\to\ 0}\frac{37 \cdot cos(2 \cdot x)}{2 \cdot \frac{sin(2 \cdot x)}{2 \cdot x}} = \frac{37}{2}$$
\rozwStop
\odpStart
$\frac{37}{2}$
\odpStop
\testStart
A.$\frac{37}{2}$
B.$\infty$
C.$-\infty$
D.$0$
E.$-\frac{37}{2}$
F.$\frac{2}{37}$
G.$-\frac{2}{37}$
H.$2$
I.$37$
\testStop
\kluczStart
A
\kluczStop



\zadStart{Przykład z Wikieł P 4.3a moja wersja nr 762}


Obliczyć granicę funkcji $\lim\limits_{x\to\ 0}\frac{37 \cdot x}{tan(3 \cdot x)}$.
\zadStop
\rozwStart{Patryk Wirkus}{}
$$\lim\limits_{x\to\ 0}\frac{37 \cdot x}{tan(3 \cdot x)}=\lim\limits_{x\to\ 0}\frac{37 \cdot x \cdot cos(3 \cdot x)}{sin(3 \cdot x)}=\lim\limits_{x\to\ 0}\frac{37 \cdot cos(3 \cdot x)}{\frac{sin(3 \cdot x)}{x}}=\lim\limits_{x\to\ 0}\frac{37 \cdot cos(3 \cdot x)}{3 \cdot \frac{sin(3 \cdot x)}{3 \cdot x}} = \frac{37}{3}$$
\rozwStop
\odpStart
$\frac{37}{3}$
\odpStop
\testStart
A.$\frac{37}{3}$
B.$\infty$
C.$-\infty$
D.$0$
E.$-\frac{37}{3}$
F.$\frac{3}{37}$
G.$-\frac{3}{37}$
H.$3$
I.$37$
\testStop
\kluczStart
A
\kluczStop



\zadStart{Przykład z Wikieł P 4.3a moja wersja nr 763}


Obliczyć granicę funkcji $\lim\limits_{x\to\ 0}\frac{37 \cdot x}{tan(4 \cdot x)}$.
\zadStop
\rozwStart{Patryk Wirkus}{}
$$\lim\limits_{x\to\ 0}\frac{37 \cdot x}{tan(4 \cdot x)}=\lim\limits_{x\to\ 0}\frac{37 \cdot x \cdot cos(4 \cdot x)}{sin(4 \cdot x)}=\lim\limits_{x\to\ 0}\frac{37 \cdot cos(4 \cdot x)}{\frac{sin(4 \cdot x)}{x}}=\lim\limits_{x\to\ 0}\frac{37 \cdot cos(4 \cdot x)}{4 \cdot \frac{sin(4 \cdot x)}{4 \cdot x}} = \frac{37}{4}$$
\rozwStop
\odpStart
$\frac{37}{4}$
\odpStop
\testStart
A.$\frac{37}{4}$
B.$\infty$
C.$-\infty$
D.$0$
E.$-\frac{37}{4}$
F.$\frac{4}{37}$
G.$-\frac{4}{37}$
H.$4$
I.$37$
\testStop
\kluczStart
A
\kluczStop



\zadStart{Przykład z Wikieł P 4.3a moja wersja nr 764}


Obliczyć granicę funkcji $\lim\limits_{x\to\ 0}\frac{37 \cdot x}{tan(5 \cdot x)}$.
\zadStop
\rozwStart{Patryk Wirkus}{}
$$\lim\limits_{x\to\ 0}\frac{37 \cdot x}{tan(5 \cdot x)}=\lim\limits_{x\to\ 0}\frac{37 \cdot x \cdot cos(5 \cdot x)}{sin(5 \cdot x)}=\lim\limits_{x\to\ 0}\frac{37 \cdot cos(5 \cdot x)}{\frac{sin(5 \cdot x)}{x}}=\lim\limits_{x\to\ 0}\frac{37 \cdot cos(5 \cdot x)}{5 \cdot \frac{sin(5 \cdot x)}{5 \cdot x}} = \frac{37}{5}$$
\rozwStop
\odpStart
$\frac{37}{5}$
\odpStop
\testStart
A.$\frac{37}{5}$
B.$\infty$
C.$-\infty$
D.$0$
E.$-\frac{37}{5}$
F.$\frac{5}{37}$
G.$-\frac{5}{37}$
H.$5$
I.$37$
\testStop
\kluczStart
A
\kluczStop



\zadStart{Przykład z Wikieł P 4.3a moja wersja nr 765}


Obliczyć granicę funkcji $\lim\limits_{x\to\ 0}\frac{37 \cdot x}{tan(6 \cdot x)}$.
\zadStop
\rozwStart{Patryk Wirkus}{}
$$\lim\limits_{x\to\ 0}\frac{37 \cdot x}{tan(6 \cdot x)}=\lim\limits_{x\to\ 0}\frac{37 \cdot x \cdot cos(6 \cdot x)}{sin(6 \cdot x)}=\lim\limits_{x\to\ 0}\frac{37 \cdot cos(6 \cdot x)}{\frac{sin(6 \cdot x)}{x}}=\lim\limits_{x\to\ 0}\frac{37 \cdot cos(6 \cdot x)}{6 \cdot \frac{sin(6 \cdot x)}{6 \cdot x}} = \frac{37}{6}$$
\rozwStop
\odpStart
$\frac{37}{6}$
\odpStop
\testStart
A.$\frac{37}{6}$
B.$\infty$
C.$-\infty$
D.$0$
E.$-\frac{37}{6}$
F.$\frac{6}{37}$
G.$-\frac{6}{37}$
H.$6$
I.$37$
\testStop
\kluczStart
A
\kluczStop



\zadStart{Przykład z Wikieł P 4.3a moja wersja nr 766}


Obliczyć granicę funkcji $\lim\limits_{x\to\ 0}\frac{37 \cdot x}{tan(7 \cdot x)}$.
\zadStop
\rozwStart{Patryk Wirkus}{}
$$\lim\limits_{x\to\ 0}\frac{37 \cdot x}{tan(7 \cdot x)}=\lim\limits_{x\to\ 0}\frac{37 \cdot x \cdot cos(7 \cdot x)}{sin(7 \cdot x)}=\lim\limits_{x\to\ 0}\frac{37 \cdot cos(7 \cdot x)}{\frac{sin(7 \cdot x)}{x}}=\lim\limits_{x\to\ 0}\frac{37 \cdot cos(7 \cdot x)}{7 \cdot \frac{sin(7 \cdot x)}{7 \cdot x}} = \frac{37}{7}$$
\rozwStop
\odpStart
$\frac{37}{7}$
\odpStop
\testStart
A.$\frac{37}{7}$
B.$\infty$
C.$-\infty$
D.$0$
E.$-\frac{37}{7}$
F.$\frac{7}{37}$
G.$-\frac{7}{37}$
H.$7$
I.$37$
\testStop
\kluczStart
A
\kluczStop



\zadStart{Przykład z Wikieł P 4.3a moja wersja nr 767}


Obliczyć granicę funkcji $\lim\limits_{x\to\ 0}\frac{37 \cdot x}{tan(8 \cdot x)}$.
\zadStop
\rozwStart{Patryk Wirkus}{}
$$\lim\limits_{x\to\ 0}\frac{37 \cdot x}{tan(8 \cdot x)}=\lim\limits_{x\to\ 0}\frac{37 \cdot x \cdot cos(8 \cdot x)}{sin(8 \cdot x)}=\lim\limits_{x\to\ 0}\frac{37 \cdot cos(8 \cdot x)}{\frac{sin(8 \cdot x)}{x}}=\lim\limits_{x\to\ 0}\frac{37 \cdot cos(8 \cdot x)}{8 \cdot \frac{sin(8 \cdot x)}{8 \cdot x}} = \frac{37}{8}$$
\rozwStop
\odpStart
$\frac{37}{8}$
\odpStop
\testStart
A.$\frac{37}{8}$
B.$\infty$
C.$-\infty$
D.$0$
E.$-\frac{37}{8}$
F.$\frac{8}{37}$
G.$-\frac{8}{37}$
H.$8$
I.$37$
\testStop
\kluczStart
A
\kluczStop



\zadStart{Przykład z Wikieł P 4.3a moja wersja nr 768}


Obliczyć granicę funkcji $\lim\limits_{x\to\ 0}\frac{37 \cdot x}{tan(9 \cdot x)}$.
\zadStop
\rozwStart{Patryk Wirkus}{}
$$\lim\limits_{x\to\ 0}\frac{37 \cdot x}{tan(9 \cdot x)}=\lim\limits_{x\to\ 0}\frac{37 \cdot x \cdot cos(9 \cdot x)}{sin(9 \cdot x)}=\lim\limits_{x\to\ 0}\frac{37 \cdot cos(9 \cdot x)}{\frac{sin(9 \cdot x)}{x}}=\lim\limits_{x\to\ 0}\frac{37 \cdot cos(9 \cdot x)}{9 \cdot \frac{sin(9 \cdot x)}{9 \cdot x}} = \frac{37}{9}$$
\rozwStop
\odpStart
$\frac{37}{9}$
\odpStop
\testStart
A.$\frac{37}{9}$
B.$\infty$
C.$-\infty$
D.$0$
E.$-\frac{37}{9}$
F.$\frac{9}{37}$
G.$-\frac{9}{37}$
H.$9$
I.$37$
\testStop
\kluczStart
A
\kluczStop



\zadStart{Przykład z Wikieł P 4.3a moja wersja nr 769}


Obliczyć granicę funkcji $\lim\limits_{x\to\ 0}\frac{37 \cdot x}{tan(10 \cdot x)}$.
\zadStop
\rozwStart{Patryk Wirkus}{}
$$\lim\limits_{x\to\ 0}\frac{37 \cdot x}{tan(10 \cdot x)}=\lim\limits_{x\to\ 0}\frac{37 \cdot x \cdot cos(10 \cdot x)}{sin(10 \cdot x)}=\lim\limits_{x\to\ 0}\frac{37 \cdot cos(10 \cdot x)}{\frac{sin(10 \cdot x)}{x}}=\lim\limits_{x\to\ 0}\frac{37 \cdot cos(10 \cdot x)}{10 \cdot \frac{sin(10 \cdot x)}{10 \cdot x}} = \frac{37}{10}$$
\rozwStop
\odpStart
$\frac{37}{10}$
\odpStop
\testStart
A.$\frac{37}{10}$
B.$\infty$
C.$-\infty$
D.$0$
E.$-\frac{37}{10}$
F.$\frac{10}{37}$
G.$-\frac{10}{37}$
H.$10$
I.$37$
\testStop
\kluczStart
A
\kluczStop



\zadStart{Przykład z Wikieł P 4.3a moja wersja nr 770}


Obliczyć granicę funkcji $\lim\limits_{x\to\ 0}\frac{37 \cdot x}{tan(11 \cdot x)}$.
\zadStop
\rozwStart{Patryk Wirkus}{}
$$\lim\limits_{x\to\ 0}\frac{37 \cdot x}{tan(11 \cdot x)}=\lim\limits_{x\to\ 0}\frac{37 \cdot x \cdot cos(11 \cdot x)}{sin(11 \cdot x)}=\lim\limits_{x\to\ 0}\frac{37 \cdot cos(11 \cdot x)}{\frac{sin(11 \cdot x)}{x}}=\lim\limits_{x\to\ 0}\frac{37 \cdot cos(11 \cdot x)}{11 \cdot \frac{sin(11 \cdot x)}{11 \cdot x}} = \frac{37}{11}$$
\rozwStop
\odpStart
$\frac{37}{11}$
\odpStop
\testStart
A.$\frac{37}{11}$
B.$\infty$
C.$-\infty$
D.$0$
E.$-\frac{37}{11}$
F.$\frac{11}{37}$
G.$-\frac{11}{37}$
H.$11$
I.$37$
\testStop
\kluczStart
A
\kluczStop



\zadStart{Przykład z Wikieł P 4.3a moja wersja nr 771}


Obliczyć granicę funkcji $\lim\limits_{x\to\ 0}\frac{37 \cdot x}{tan(12 \cdot x)}$.
\zadStop
\rozwStart{Patryk Wirkus}{}
$$\lim\limits_{x\to\ 0}\frac{37 \cdot x}{tan(12 \cdot x)}=\lim\limits_{x\to\ 0}\frac{37 \cdot x \cdot cos(12 \cdot x)}{sin(12 \cdot x)}=\lim\limits_{x\to\ 0}\frac{37 \cdot cos(12 \cdot x)}{\frac{sin(12 \cdot x)}{x}}=\lim\limits_{x\to\ 0}\frac{37 \cdot cos(12 \cdot x)}{12 \cdot \frac{sin(12 \cdot x)}{12 \cdot x}} = \frac{37}{12}$$
\rozwStop
\odpStart
$\frac{37}{12}$
\odpStop
\testStart
A.$\frac{37}{12}$
B.$\infty$
C.$-\infty$
D.$0$
E.$-\frac{37}{12}$
F.$\frac{12}{37}$
G.$-\frac{12}{37}$
H.$12$
I.$37$
\testStop
\kluczStart
A
\kluczStop



\zadStart{Przykład z Wikieł P 4.3a moja wersja nr 772}


Obliczyć granicę funkcji $\lim\limits_{x\to\ 0}\frac{37 \cdot x}{tan(13 \cdot x)}$.
\zadStop
\rozwStart{Patryk Wirkus}{}
$$\lim\limits_{x\to\ 0}\frac{37 \cdot x}{tan(13 \cdot x)}=\lim\limits_{x\to\ 0}\frac{37 \cdot x \cdot cos(13 \cdot x)}{sin(13 \cdot x)}=\lim\limits_{x\to\ 0}\frac{37 \cdot cos(13 \cdot x)}{\frac{sin(13 \cdot x)}{x}}=\lim\limits_{x\to\ 0}\frac{37 \cdot cos(13 \cdot x)}{13 \cdot \frac{sin(13 \cdot x)}{13 \cdot x}} = \frac{37}{13}$$
\rozwStop
\odpStart
$\frac{37}{13}$
\odpStop
\testStart
A.$\frac{37}{13}$
B.$\infty$
C.$-\infty$
D.$0$
E.$-\frac{37}{13}$
F.$\frac{13}{37}$
G.$-\frac{13}{37}$
H.$13$
I.$37$
\testStop
\kluczStart
A
\kluczStop



\zadStart{Przykład z Wikieł P 4.3a moja wersja nr 773}


Obliczyć granicę funkcji $\lim\limits_{x\to\ 0}\frac{37 \cdot x}{tan(14 \cdot x)}$.
\zadStop
\rozwStart{Patryk Wirkus}{}
$$\lim\limits_{x\to\ 0}\frac{37 \cdot x}{tan(14 \cdot x)}=\lim\limits_{x\to\ 0}\frac{37 \cdot x \cdot cos(14 \cdot x)}{sin(14 \cdot x)}=\lim\limits_{x\to\ 0}\frac{37 \cdot cos(14 \cdot x)}{\frac{sin(14 \cdot x)}{x}}=\lim\limits_{x\to\ 0}\frac{37 \cdot cos(14 \cdot x)}{14 \cdot \frac{sin(14 \cdot x)}{14 \cdot x}} = \frac{37}{14}$$
\rozwStop
\odpStart
$\frac{37}{14}$
\odpStop
\testStart
A.$\frac{37}{14}$
B.$\infty$
C.$-\infty$
D.$0$
E.$-\frac{37}{14}$
F.$\frac{14}{37}$
G.$-\frac{14}{37}$
H.$14$
I.$37$
\testStop
\kluczStart
A
\kluczStop



\zadStart{Przykład z Wikieł P 4.3a moja wersja nr 774}


Obliczyć granicę funkcji $\lim\limits_{x\to\ 0}\frac{37 \cdot x}{tan(15 \cdot x)}$.
\zadStop
\rozwStart{Patryk Wirkus}{}
$$\lim\limits_{x\to\ 0}\frac{37 \cdot x}{tan(15 \cdot x)}=\lim\limits_{x\to\ 0}\frac{37 \cdot x \cdot cos(15 \cdot x)}{sin(15 \cdot x)}=\lim\limits_{x\to\ 0}\frac{37 \cdot cos(15 \cdot x)}{\frac{sin(15 \cdot x)}{x}}=\lim\limits_{x\to\ 0}\frac{37 \cdot cos(15 \cdot x)}{15 \cdot \frac{sin(15 \cdot x)}{15 \cdot x}} = \frac{37}{15}$$
\rozwStop
\odpStart
$\frac{37}{15}$
\odpStop
\testStart
A.$\frac{37}{15}$
B.$\infty$
C.$-\infty$
D.$0$
E.$-\frac{37}{15}$
F.$\frac{15}{37}$
G.$-\frac{15}{37}$
H.$15$
I.$37$
\testStop
\kluczStart
A
\kluczStop



\zadStart{Przykład z Wikieł P 4.3a moja wersja nr 775}


Obliczyć granicę funkcji $\lim\limits_{x\to\ 0}\frac{37 \cdot x}{tan(16 \cdot x)}$.
\zadStop
\rozwStart{Patryk Wirkus}{}
$$\lim\limits_{x\to\ 0}\frac{37 \cdot x}{tan(16 \cdot x)}=\lim\limits_{x\to\ 0}\frac{37 \cdot x \cdot cos(16 \cdot x)}{sin(16 \cdot x)}=\lim\limits_{x\to\ 0}\frac{37 \cdot cos(16 \cdot x)}{\frac{sin(16 \cdot x)}{x}}=\lim\limits_{x\to\ 0}\frac{37 \cdot cos(16 \cdot x)}{16 \cdot \frac{sin(16 \cdot x)}{16 \cdot x}} = \frac{37}{16}$$
\rozwStop
\odpStart
$\frac{37}{16}$
\odpStop
\testStart
A.$\frac{37}{16}$
B.$\infty$
C.$-\infty$
D.$0$
E.$-\frac{37}{16}$
F.$\frac{16}{37}$
G.$-\frac{16}{37}$
H.$16$
I.$37$
\testStop
\kluczStart
A
\kluczStop



\zadStart{Przykład z Wikieł P 4.3a moja wersja nr 776}


Obliczyć granicę funkcji $\lim\limits_{x\to\ 0}\frac{37 \cdot x}{tan(17 \cdot x)}$.
\zadStop
\rozwStart{Patryk Wirkus}{}
$$\lim\limits_{x\to\ 0}\frac{37 \cdot x}{tan(17 \cdot x)}=\lim\limits_{x\to\ 0}\frac{37 \cdot x \cdot cos(17 \cdot x)}{sin(17 \cdot x)}=\lim\limits_{x\to\ 0}\frac{37 \cdot cos(17 \cdot x)}{\frac{sin(17 \cdot x)}{x}}=\lim\limits_{x\to\ 0}\frac{37 \cdot cos(17 \cdot x)}{17 \cdot \frac{sin(17 \cdot x)}{17 \cdot x}} = \frac{37}{17}$$
\rozwStop
\odpStart
$\frac{37}{17}$
\odpStop
\testStart
A.$\frac{37}{17}$
B.$\infty$
C.$-\infty$
D.$0$
E.$-\frac{37}{17}$
F.$\frac{17}{37}$
G.$-\frac{17}{37}$
H.$17$
I.$37$
\testStop
\kluczStart
A
\kluczStop



\zadStart{Przykład z Wikieł P 4.3a moja wersja nr 777}


Obliczyć granicę funkcji $\lim\limits_{x\to\ 0}\frac{37 \cdot x}{tan(18 \cdot x)}$.
\zadStop
\rozwStart{Patryk Wirkus}{}
$$\lim\limits_{x\to\ 0}\frac{37 \cdot x}{tan(18 \cdot x)}=\lim\limits_{x\to\ 0}\frac{37 \cdot x \cdot cos(18 \cdot x)}{sin(18 \cdot x)}=\lim\limits_{x\to\ 0}\frac{37 \cdot cos(18 \cdot x)}{\frac{sin(18 \cdot x)}{x}}=\lim\limits_{x\to\ 0}\frac{37 \cdot cos(18 \cdot x)}{18 \cdot \frac{sin(18 \cdot x)}{18 \cdot x}} = \frac{37}{18}$$
\rozwStop
\odpStart
$\frac{37}{18}$
\odpStop
\testStart
A.$\frac{37}{18}$
B.$\infty$
C.$-\infty$
D.$0$
E.$-\frac{37}{18}$
F.$\frac{18}{37}$
G.$-\frac{18}{37}$
H.$18$
I.$37$
\testStop
\kluczStart
A
\kluczStop



\zadStart{Przykład z Wikieł P 4.3a moja wersja nr 778}


Obliczyć granicę funkcji $\lim\limits_{x\to\ 0}\frac{37 \cdot x}{tan(19 \cdot x)}$.
\zadStop
\rozwStart{Patryk Wirkus}{}
$$\lim\limits_{x\to\ 0}\frac{37 \cdot x}{tan(19 \cdot x)}=\lim\limits_{x\to\ 0}\frac{37 \cdot x \cdot cos(19 \cdot x)}{sin(19 \cdot x)}=\lim\limits_{x\to\ 0}\frac{37 \cdot cos(19 \cdot x)}{\frac{sin(19 \cdot x)}{x}}=\lim\limits_{x\to\ 0}\frac{37 \cdot cos(19 \cdot x)}{19 \cdot \frac{sin(19 \cdot x)}{19 \cdot x}} = \frac{37}{19}$$
\rozwStop
\odpStart
$\frac{37}{19}$
\odpStop
\testStart
A.$\frac{37}{19}$
B.$\infty$
C.$-\infty$
D.$0$
E.$-\frac{37}{19}$
F.$\frac{19}{37}$
G.$-\frac{19}{37}$
H.$19$
I.$37$
\testStop
\kluczStart
A
\kluczStop



\zadStart{Przykład z Wikieł P 4.3a moja wersja nr 779}


Obliczyć granicę funkcji $\lim\limits_{x\to\ 0}\frac{37 \cdot x}{tan(20 \cdot x)}$.
\zadStop
\rozwStart{Patryk Wirkus}{}
$$\lim\limits_{x\to\ 0}\frac{37 \cdot x}{tan(20 \cdot x)}=\lim\limits_{x\to\ 0}\frac{37 \cdot x \cdot cos(20 \cdot x)}{sin(20 \cdot x)}=\lim\limits_{x\to\ 0}\frac{37 \cdot cos(20 \cdot x)}{\frac{sin(20 \cdot x)}{x}}=\lim\limits_{x\to\ 0}\frac{37 \cdot cos(20 \cdot x)}{20 \cdot \frac{sin(20 \cdot x)}{20 \cdot x}} = \frac{37}{20}$$
\rozwStop
\odpStart
$\frac{37}{20}$
\odpStop
\testStart
A.$\frac{37}{20}$
B.$\infty$
C.$-\infty$
D.$0$
E.$-\frac{37}{20}$
F.$\frac{20}{37}$
G.$-\frac{20}{37}$
H.$20$
I.$37$
\testStop
\kluczStart
A
\kluczStop



\zadStart{Przykład z Wikieł P 4.3a moja wersja nr 780}


Obliczyć granicę funkcji $\lim\limits_{x\to\ 0}\frac{37 \cdot x}{tan(21 \cdot x)}$.
\zadStop
\rozwStart{Patryk Wirkus}{}
$$\lim\limits_{x\to\ 0}\frac{37 \cdot x}{tan(21 \cdot x)}=\lim\limits_{x\to\ 0}\frac{37 \cdot x \cdot cos(21 \cdot x)}{sin(21 \cdot x)}=\lim\limits_{x\to\ 0}\frac{37 \cdot cos(21 \cdot x)}{\frac{sin(21 \cdot x)}{x}}=\lim\limits_{x\to\ 0}\frac{37 \cdot cos(21 \cdot x)}{21 \cdot \frac{sin(21 \cdot x)}{21 \cdot x}} = \frac{37}{21}$$
\rozwStop
\odpStart
$\frac{37}{21}$
\odpStop
\testStart
A.$\frac{37}{21}$
B.$\infty$
C.$-\infty$
D.$0$
E.$-\frac{37}{21}$
F.$\frac{21}{37}$
G.$-\frac{21}{37}$
H.$21$
I.$37$
\testStop
\kluczStart
A
\kluczStop



\zadStart{Przykład z Wikieł P 4.3a moja wersja nr 781}


Obliczyć granicę funkcji $\lim\limits_{x\to\ 0}\frac{37 \cdot x}{tan(22 \cdot x)}$.
\zadStop
\rozwStart{Patryk Wirkus}{}
$$\lim\limits_{x\to\ 0}\frac{37 \cdot x}{tan(22 \cdot x)}=\lim\limits_{x\to\ 0}\frac{37 \cdot x \cdot cos(22 \cdot x)}{sin(22 \cdot x)}=\lim\limits_{x\to\ 0}\frac{37 \cdot cos(22 \cdot x)}{\frac{sin(22 \cdot x)}{x}}=\lim\limits_{x\to\ 0}\frac{37 \cdot cos(22 \cdot x)}{22 \cdot \frac{sin(22 \cdot x)}{22 \cdot x}} = \frac{37}{22}$$
\rozwStop
\odpStart
$\frac{37}{22}$
\odpStop
\testStart
A.$\frac{37}{22}$
B.$\infty$
C.$-\infty$
D.$0$
E.$-\frac{37}{22}$
F.$\frac{22}{37}$
G.$-\frac{22}{37}$
H.$22$
I.$37$
\testStop
\kluczStart
A
\kluczStop



\zadStart{Przykład z Wikieł P 4.3a moja wersja nr 782}


Obliczyć granicę funkcji $\lim\limits_{x\to\ 0}\frac{37 \cdot x}{tan(23 \cdot x)}$.
\zadStop
\rozwStart{Patryk Wirkus}{}
$$\lim\limits_{x\to\ 0}\frac{37 \cdot x}{tan(23 \cdot x)}=\lim\limits_{x\to\ 0}\frac{37 \cdot x \cdot cos(23 \cdot x)}{sin(23 \cdot x)}=\lim\limits_{x\to\ 0}\frac{37 \cdot cos(23 \cdot x)}{\frac{sin(23 \cdot x)}{x}}=\lim\limits_{x\to\ 0}\frac{37 \cdot cos(23 \cdot x)}{23 \cdot \frac{sin(23 \cdot x)}{23 \cdot x}} = \frac{37}{23}$$
\rozwStop
\odpStart
$\frac{37}{23}$
\odpStop
\testStart
A.$\frac{37}{23}$
B.$\infty$
C.$-\infty$
D.$0$
E.$-\frac{37}{23}$
F.$\frac{23}{37}$
G.$-\frac{23}{37}$
H.$23$
I.$37$
\testStop
\kluczStart
A
\kluczStop



\zadStart{Przykład z Wikieł P 4.3a moja wersja nr 783}


Obliczyć granicę funkcji $\lim\limits_{x\to\ 0}\frac{37 \cdot x}{tan(24 \cdot x)}$.
\zadStop
\rozwStart{Patryk Wirkus}{}
$$\lim\limits_{x\to\ 0}\frac{37 \cdot x}{tan(24 \cdot x)}=\lim\limits_{x\to\ 0}\frac{37 \cdot x \cdot cos(24 \cdot x)}{sin(24 \cdot x)}=\lim\limits_{x\to\ 0}\frac{37 \cdot cos(24 \cdot x)}{\frac{sin(24 \cdot x)}{x}}=\lim\limits_{x\to\ 0}\frac{37 \cdot cos(24 \cdot x)}{24 \cdot \frac{sin(24 \cdot x)}{24 \cdot x}} = \frac{37}{24}$$
\rozwStop
\odpStart
$\frac{37}{24}$
\odpStop
\testStart
A.$\frac{37}{24}$
B.$\infty$
C.$-\infty$
D.$0$
E.$-\frac{37}{24}$
F.$\frac{24}{37}$
G.$-\frac{24}{37}$
H.$24$
I.$37$
\testStop
\kluczStart
A
\kluczStop



\zadStart{Przykład z Wikieł P 4.3a moja wersja nr 784}


Obliczyć granicę funkcji $\lim\limits_{x\to\ 0}\frac{37 \cdot x}{tan(25 \cdot x)}$.
\zadStop
\rozwStart{Patryk Wirkus}{}
$$\lim\limits_{x\to\ 0}\frac{37 \cdot x}{tan(25 \cdot x)}=\lim\limits_{x\to\ 0}\frac{37 \cdot x \cdot cos(25 \cdot x)}{sin(25 \cdot x)}=\lim\limits_{x\to\ 0}\frac{37 \cdot cos(25 \cdot x)}{\frac{sin(25 \cdot x)}{x}}=\lim\limits_{x\to\ 0}\frac{37 \cdot cos(25 \cdot x)}{25 \cdot \frac{sin(25 \cdot x)}{25 \cdot x}} = \frac{37}{25}$$
\rozwStop
\odpStart
$\frac{37}{25}$
\odpStop
\testStart
A.$\frac{37}{25}$
B.$\infty$
C.$-\infty$
D.$0$
E.$-\frac{37}{25}$
F.$\frac{25}{37}$
G.$-\frac{25}{37}$
H.$25$
I.$37$
\testStop
\kluczStart
A
\kluczStop



\zadStart{Przykład z Wikieł P 4.3a moja wersja nr 785}


Obliczyć granicę funkcji $\lim\limits_{x\to\ 0}\frac{37 \cdot x}{tan(26 \cdot x)}$.
\zadStop
\rozwStart{Patryk Wirkus}{}
$$\lim\limits_{x\to\ 0}\frac{37 \cdot x}{tan(26 \cdot x)}=\lim\limits_{x\to\ 0}\frac{37 \cdot x \cdot cos(26 \cdot x)}{sin(26 \cdot x)}=\lim\limits_{x\to\ 0}\frac{37 \cdot cos(26 \cdot x)}{\frac{sin(26 \cdot x)}{x}}=\lim\limits_{x\to\ 0}\frac{37 \cdot cos(26 \cdot x)}{26 \cdot \frac{sin(26 \cdot x)}{26 \cdot x}} = \frac{37}{26}$$
\rozwStop
\odpStart
$\frac{37}{26}$
\odpStop
\testStart
A.$\frac{37}{26}$
B.$\infty$
C.$-\infty$
D.$0$
E.$-\frac{37}{26}$
F.$\frac{26}{37}$
G.$-\frac{26}{37}$
H.$26$
I.$37$
\testStop
\kluczStart
A
\kluczStop



\zadStart{Przykład z Wikieł P 4.3a moja wersja nr 786}


Obliczyć granicę funkcji $\lim\limits_{x\to\ 0}\frac{37 \cdot x}{tan(27 \cdot x)}$.
\zadStop
\rozwStart{Patryk Wirkus}{}
$$\lim\limits_{x\to\ 0}\frac{37 \cdot x}{tan(27 \cdot x)}=\lim\limits_{x\to\ 0}\frac{37 \cdot x \cdot cos(27 \cdot x)}{sin(27 \cdot x)}=\lim\limits_{x\to\ 0}\frac{37 \cdot cos(27 \cdot x)}{\frac{sin(27 \cdot x)}{x}}=\lim\limits_{x\to\ 0}\frac{37 \cdot cos(27 \cdot x)}{27 \cdot \frac{sin(27 \cdot x)}{27 \cdot x}} = \frac{37}{27}$$
\rozwStop
\odpStart
$\frac{37}{27}$
\odpStop
\testStart
A.$\frac{37}{27}$
B.$\infty$
C.$-\infty$
D.$0$
E.$-\frac{37}{27}$
F.$\frac{27}{37}$
G.$-\frac{27}{37}$
H.$27$
I.$37$
\testStop
\kluczStart
A
\kluczStop



\zadStart{Przykład z Wikieł P 4.3a moja wersja nr 787}


Obliczyć granicę funkcji $\lim\limits_{x\to\ 0}\frac{37 \cdot x}{tan(28 \cdot x)}$.
\zadStop
\rozwStart{Patryk Wirkus}{}
$$\lim\limits_{x\to\ 0}\frac{37 \cdot x}{tan(28 \cdot x)}=\lim\limits_{x\to\ 0}\frac{37 \cdot x \cdot cos(28 \cdot x)}{sin(28 \cdot x)}=\lim\limits_{x\to\ 0}\frac{37 \cdot cos(28 \cdot x)}{\frac{sin(28 \cdot x)}{x}}=\lim\limits_{x\to\ 0}\frac{37 \cdot cos(28 \cdot x)}{28 \cdot \frac{sin(28 \cdot x)}{28 \cdot x}} = \frac{37}{28}$$
\rozwStop
\odpStart
$\frac{37}{28}$
\odpStop
\testStart
A.$\frac{37}{28}$
B.$\infty$
C.$-\infty$
D.$0$
E.$-\frac{37}{28}$
F.$\frac{28}{37}$
G.$-\frac{28}{37}$
H.$28$
I.$37$
\testStop
\kluczStart
A
\kluczStop



\zadStart{Przykład z Wikieł P 4.3a moja wersja nr 788}


Obliczyć granicę funkcji $\lim\limits_{x\to\ 0}\frac{37 \cdot x}{tan(29 \cdot x)}$.
\zadStop
\rozwStart{Patryk Wirkus}{}
$$\lim\limits_{x\to\ 0}\frac{37 \cdot x}{tan(29 \cdot x)}=\lim\limits_{x\to\ 0}\frac{37 \cdot x \cdot cos(29 \cdot x)}{sin(29 \cdot x)}=\lim\limits_{x\to\ 0}\frac{37 \cdot cos(29 \cdot x)}{\frac{sin(29 \cdot x)}{x}}=\lim\limits_{x\to\ 0}\frac{37 \cdot cos(29 \cdot x)}{29 \cdot \frac{sin(29 \cdot x)}{29 \cdot x}} = \frac{37}{29}$$
\rozwStop
\odpStart
$\frac{37}{29}$
\odpStop
\testStart
A.$\frac{37}{29}$
B.$\infty$
C.$-\infty$
D.$0$
E.$-\frac{37}{29}$
F.$\frac{29}{37}$
G.$-\frac{29}{37}$
H.$29$
I.$37$
\testStop
\kluczStart
A
\kluczStop



\zadStart{Przykład z Wikieł P 4.3a moja wersja nr 789}


Obliczyć granicę funkcji $\lim\limits_{x\to\ 0}\frac{37 \cdot x}{tan(30 \cdot x)}$.
\zadStop
\rozwStart{Patryk Wirkus}{}
$$\lim\limits_{x\to\ 0}\frac{37 \cdot x}{tan(30 \cdot x)}=\lim\limits_{x\to\ 0}\frac{37 \cdot x \cdot cos(30 \cdot x)}{sin(30 \cdot x)}=\lim\limits_{x\to\ 0}\frac{37 \cdot cos(30 \cdot x)}{\frac{sin(30 \cdot x)}{x}}=\lim\limits_{x\to\ 0}\frac{37 \cdot cos(30 \cdot x)}{30 \cdot \frac{sin(30 \cdot x)}{30 \cdot x}} = \frac{37}{30}$$
\rozwStop
\odpStart
$\frac{37}{30}$
\odpStop
\testStart
A.$\frac{37}{30}$
B.$\infty$
C.$-\infty$
D.$0$
E.$-\frac{37}{30}$
F.$\frac{30}{37}$
G.$-\frac{30}{37}$
H.$30$
I.$37$
\testStop
\kluczStart
A
\kluczStop



\zadStart{Przykład z Wikieł P 4.3a moja wersja nr 790}


Obliczyć granicę funkcji $\lim\limits_{x\to\ 0}\frac{37 \cdot x}{tan(31 \cdot x)}$.
\zadStop
\rozwStart{Patryk Wirkus}{}
$$\lim\limits_{x\to\ 0}\frac{37 \cdot x}{tan(31 \cdot x)}=\lim\limits_{x\to\ 0}\frac{37 \cdot x \cdot cos(31 \cdot x)}{sin(31 \cdot x)}=\lim\limits_{x\to\ 0}\frac{37 \cdot cos(31 \cdot x)}{\frac{sin(31 \cdot x)}{x}}=\lim\limits_{x\to\ 0}\frac{37 \cdot cos(31 \cdot x)}{31 \cdot \frac{sin(31 \cdot x)}{31 \cdot x}} = \frac{37}{31}$$
\rozwStop
\odpStart
$\frac{37}{31}$
\odpStop
\testStart
A.$\frac{37}{31}$
B.$\infty$
C.$-\infty$
D.$0$
E.$-\frac{37}{31}$
F.$\frac{31}{37}$
G.$-\frac{31}{37}$
H.$31$
I.$37$
\testStop
\kluczStart
A
\kluczStop



\zadStart{Przykład z Wikieł P 4.3a moja wersja nr 791}


Obliczyć granicę funkcji $\lim\limits_{x\to\ 0}\frac{37 \cdot x}{tan(32 \cdot x)}$.
\zadStop
\rozwStart{Patryk Wirkus}{}
$$\lim\limits_{x\to\ 0}\frac{37 \cdot x}{tan(32 \cdot x)}=\lim\limits_{x\to\ 0}\frac{37 \cdot x \cdot cos(32 \cdot x)}{sin(32 \cdot x)}=\lim\limits_{x\to\ 0}\frac{37 \cdot cos(32 \cdot x)}{\frac{sin(32 \cdot x)}{x}}=\lim\limits_{x\to\ 0}\frac{37 \cdot cos(32 \cdot x)}{32 \cdot \frac{sin(32 \cdot x)}{32 \cdot x}} = \frac{37}{32}$$
\rozwStop
\odpStart
$\frac{37}{32}$
\odpStop
\testStart
A.$\frac{37}{32}$
B.$\infty$
C.$-\infty$
D.$0$
E.$-\frac{37}{32}$
F.$\frac{32}{37}$
G.$-\frac{32}{37}$
H.$32$
I.$37$
\testStop
\kluczStart
A
\kluczStop



\zadStart{Przykład z Wikieł P 4.3a moja wersja nr 792}


Obliczyć granicę funkcji $\lim\limits_{x\to\ 0}\frac{37 \cdot x}{tan(33 \cdot x)}$.
\zadStop
\rozwStart{Patryk Wirkus}{}
$$\lim\limits_{x\to\ 0}\frac{37 \cdot x}{tan(33 \cdot x)}=\lim\limits_{x\to\ 0}\frac{37 \cdot x \cdot cos(33 \cdot x)}{sin(33 \cdot x)}=\lim\limits_{x\to\ 0}\frac{37 \cdot cos(33 \cdot x)}{\frac{sin(33 \cdot x)}{x}}=\lim\limits_{x\to\ 0}\frac{37 \cdot cos(33 \cdot x)}{33 \cdot \frac{sin(33 \cdot x)}{33 \cdot x}} = \frac{37}{33}$$
\rozwStop
\odpStart
$\frac{37}{33}$
\odpStop
\testStart
A.$\frac{37}{33}$
B.$\infty$
C.$-\infty$
D.$0$
E.$-\frac{37}{33}$
F.$\frac{33}{37}$
G.$-\frac{33}{37}$
H.$33$
I.$37$
\testStop
\kluczStart
A
\kluczStop



\zadStart{Przykład z Wikieł P 4.3a moja wersja nr 793}


Obliczyć granicę funkcji $\lim\limits_{x\to\ 0}\frac{37 \cdot x}{tan(34 \cdot x)}$.
\zadStop
\rozwStart{Patryk Wirkus}{}
$$\lim\limits_{x\to\ 0}\frac{37 \cdot x}{tan(34 \cdot x)}=\lim\limits_{x\to\ 0}\frac{37 \cdot x \cdot cos(34 \cdot x)}{sin(34 \cdot x)}=\lim\limits_{x\to\ 0}\frac{37 \cdot cos(34 \cdot x)}{\frac{sin(34 \cdot x)}{x}}=\lim\limits_{x\to\ 0}\frac{37 \cdot cos(34 \cdot x)}{34 \cdot \frac{sin(34 \cdot x)}{34 \cdot x}} = \frac{37}{34}$$
\rozwStop
\odpStart
$\frac{37}{34}$
\odpStop
\testStart
A.$\frac{37}{34}$
B.$\infty$
C.$-\infty$
D.$0$
E.$-\frac{37}{34}$
F.$\frac{34}{37}$
G.$-\frac{34}{37}$
H.$34$
I.$37$
\testStop
\kluczStart
A
\kluczStop



\zadStart{Przykład z Wikieł P 4.3a moja wersja nr 794}


Obliczyć granicę funkcji $\lim\limits_{x\to\ 0}\frac{37 \cdot x}{tan(35 \cdot x)}$.
\zadStop
\rozwStart{Patryk Wirkus}{}
$$\lim\limits_{x\to\ 0}\frac{37 \cdot x}{tan(35 \cdot x)}=\lim\limits_{x\to\ 0}\frac{37 \cdot x \cdot cos(35 \cdot x)}{sin(35 \cdot x)}=\lim\limits_{x\to\ 0}\frac{37 \cdot cos(35 \cdot x)}{\frac{sin(35 \cdot x)}{x}}=\lim\limits_{x\to\ 0}\frac{37 \cdot cos(35 \cdot x)}{35 \cdot \frac{sin(35 \cdot x)}{35 \cdot x}} = \frac{37}{35}$$
\rozwStop
\odpStart
$\frac{37}{35}$
\odpStop
\testStart
A.$\frac{37}{35}$
B.$\infty$
C.$-\infty$
D.$0$
E.$-\frac{37}{35}$
F.$\frac{35}{37}$
G.$-\frac{35}{37}$
H.$35$
I.$37$
\testStop
\kluczStart
A
\kluczStop



\zadStart{Przykład z Wikieł P 4.3a moja wersja nr 795}


Obliczyć granicę funkcji $\lim\limits_{x\to\ 0}\frac{37 \cdot x}{tan(36 \cdot x)}$.
\zadStop
\rozwStart{Patryk Wirkus}{}
$$\lim\limits_{x\to\ 0}\frac{37 \cdot x}{tan(36 \cdot x)}=\lim\limits_{x\to\ 0}\frac{37 \cdot x \cdot cos(36 \cdot x)}{sin(36 \cdot x)}=\lim\limits_{x\to\ 0}\frac{37 \cdot cos(36 \cdot x)}{\frac{sin(36 \cdot x)}{x}}=\lim\limits_{x\to\ 0}\frac{37 \cdot cos(36 \cdot x)}{36 \cdot \frac{sin(36 \cdot x)}{36 \cdot x}} = \frac{37}{36}$$
\rozwStop
\odpStart
$\frac{37}{36}$
\odpStop
\testStart
A.$\frac{37}{36}$
B.$\infty$
C.$-\infty$
D.$0$
E.$-\frac{37}{36}$
F.$\frac{36}{37}$
G.$-\frac{36}{37}$
H.$36$
I.$37$
\testStop
\kluczStart
A
\kluczStop



\zadStart{Przykład z Wikieł P 4.3a moja wersja nr 796}


Obliczyć granicę funkcji $\lim\limits_{x\to\ 0}\frac{37 \cdot x}{tan(38 \cdot x)}$.
\zadStop
\rozwStart{Patryk Wirkus}{}
$$\lim\limits_{x\to\ 0}\frac{37 \cdot x}{tan(38 \cdot x)}=\lim\limits_{x\to\ 0}\frac{37 \cdot x \cdot cos(38 \cdot x)}{sin(38 \cdot x)}=\lim\limits_{x\to\ 0}\frac{37 \cdot cos(38 \cdot x)}{\frac{sin(38 \cdot x)}{x}}=\lim\limits_{x\to\ 0}\frac{37 \cdot cos(38 \cdot x)}{38 \cdot \frac{sin(38 \cdot x)}{38 \cdot x}} = \frac{37}{38}$$
\rozwStop
\odpStart
$\frac{37}{38}$
\odpStop
\testStart
A.$\frac{37}{38}$
B.$\infty$
C.$-\infty$
D.$0$
E.$-\frac{37}{38}$
F.$\frac{38}{37}$
G.$-\frac{38}{37}$
H.$38$
I.$37$
\testStop
\kluczStart
A
\kluczStop



\zadStart{Przykład z Wikieł P 4.3a moja wersja nr 797}


Obliczyć granicę funkcji $\lim\limits_{x\to\ 0}\frac{37 \cdot x}{tan(39 \cdot x)}$.
\zadStop
\rozwStart{Patryk Wirkus}{}
$$\lim\limits_{x\to\ 0}\frac{37 \cdot x}{tan(39 \cdot x)}=\lim\limits_{x\to\ 0}\frac{37 \cdot x \cdot cos(39 \cdot x)}{sin(39 \cdot x)}=\lim\limits_{x\to\ 0}\frac{37 \cdot cos(39 \cdot x)}{\frac{sin(39 \cdot x)}{x}}=\lim\limits_{x\to\ 0}\frac{37 \cdot cos(39 \cdot x)}{39 \cdot \frac{sin(39 \cdot x)}{39 \cdot x}} = \frac{37}{39}$$
\rozwStop
\odpStart
$\frac{37}{39}$
\odpStop
\testStart
A.$\frac{37}{39}$
B.$\infty$
C.$-\infty$
D.$0$
E.$-\frac{37}{39}$
F.$\frac{39}{37}$
G.$-\frac{39}{37}$
H.$39$
I.$37$
\testStop
\kluczStart
A
\kluczStop



\zadStart{Przykład z Wikieł P 4.3a moja wersja nr 798}


Obliczyć granicę funkcji $\lim\limits_{x\to\ 0}\frac{37 \cdot x}{tan(40 \cdot x)}$.
\zadStop
\rozwStart{Patryk Wirkus}{}
$$\lim\limits_{x\to\ 0}\frac{37 \cdot x}{tan(40 \cdot x)}=\lim\limits_{x\to\ 0}\frac{37 \cdot x \cdot cos(40 \cdot x)}{sin(40 \cdot x)}=\lim\limits_{x\to\ 0}\frac{37 \cdot cos(40 \cdot x)}{\frac{sin(40 \cdot x)}{x}}=\lim\limits_{x\to\ 0}\frac{37 \cdot cos(40 \cdot x)}{40 \cdot \frac{sin(40 \cdot x)}{40 \cdot x}} = \frac{37}{40}$$
\rozwStop
\odpStart
$\frac{37}{40}$
\odpStop
\testStart
A.$\frac{37}{40}$
B.$\infty$
C.$-\infty$
D.$0$
E.$-\frac{37}{40}$
F.$\frac{40}{37}$
G.$-\frac{40}{37}$
H.$40$
I.$37$
\testStop
\kluczStart
A
\kluczStop



\zadStart{Przykład z Wikieł P 4.3a moja wersja nr 799}


Obliczyć granicę funkcji $\lim\limits_{x\to\ 0}\frac{38 \cdot x}{tan(3 \cdot x)}$.
\zadStop
\rozwStart{Patryk Wirkus}{}
$$\lim\limits_{x\to\ 0}\frac{38 \cdot x}{tan(3 \cdot x)}=\lim\limits_{x\to\ 0}\frac{38 \cdot x \cdot cos(3 \cdot x)}{sin(3 \cdot x)}=\lim\limits_{x\to\ 0}\frac{38 \cdot cos(3 \cdot x)}{\frac{sin(3 \cdot x)}{x}}=\lim\limits_{x\to\ 0}\frac{38 \cdot cos(3 \cdot x)}{3 \cdot \frac{sin(3 \cdot x)}{3 \cdot x}} = \frac{38}{3}$$
\rozwStop
\odpStart
$\frac{38}{3}$
\odpStop
\testStart
A.$\frac{38}{3}$
B.$\infty$
C.$-\infty$
D.$0$
E.$-\frac{38}{3}$
F.$\frac{3}{38}$
G.$-\frac{3}{38}$
H.$3$
I.$38$
\testStop
\kluczStart
A
\kluczStop



\zadStart{Przykład z Wikieł P 4.3a moja wersja nr 800}


Obliczyć granicę funkcji $\lim\limits_{x\to\ 0}\frac{38 \cdot x}{tan(5 \cdot x)}$.
\zadStop
\rozwStart{Patryk Wirkus}{}
$$\lim\limits_{x\to\ 0}\frac{38 \cdot x}{tan(5 \cdot x)}=\lim\limits_{x\to\ 0}\frac{38 \cdot x \cdot cos(5 \cdot x)}{sin(5 \cdot x)}=\lim\limits_{x\to\ 0}\frac{38 \cdot cos(5 \cdot x)}{\frac{sin(5 \cdot x)}{x}}=\lim\limits_{x\to\ 0}\frac{38 \cdot cos(5 \cdot x)}{5 \cdot \frac{sin(5 \cdot x)}{5 \cdot x}} = \frac{38}{5}$$
\rozwStop
\odpStart
$\frac{38}{5}$
\odpStop
\testStart
A.$\frac{38}{5}$
B.$\infty$
C.$-\infty$
D.$0$
E.$-\frac{38}{5}$
F.$\frac{5}{38}$
G.$-\frac{5}{38}$
H.$5$
I.$38$
\testStop
\kluczStart
A
\kluczStop



\zadStart{Przykład z Wikieł P 4.3a moja wersja nr 801}


Obliczyć granicę funkcji $\lim\limits_{x\to\ 0}\frac{38 \cdot x}{tan(7 \cdot x)}$.
\zadStop
\rozwStart{Patryk Wirkus}{}
$$\lim\limits_{x\to\ 0}\frac{38 \cdot x}{tan(7 \cdot x)}=\lim\limits_{x\to\ 0}\frac{38 \cdot x \cdot cos(7 \cdot x)}{sin(7 \cdot x)}=\lim\limits_{x\to\ 0}\frac{38 \cdot cos(7 \cdot x)}{\frac{sin(7 \cdot x)}{x}}=\lim\limits_{x\to\ 0}\frac{38 \cdot cos(7 \cdot x)}{7 \cdot \frac{sin(7 \cdot x)}{7 \cdot x}} = \frac{38}{7}$$
\rozwStop
\odpStart
$\frac{38}{7}$
\odpStop
\testStart
A.$\frac{38}{7}$
B.$\infty$
C.$-\infty$
D.$0$
E.$-\frac{38}{7}$
F.$\frac{7}{38}$
G.$-\frac{7}{38}$
H.$7$
I.$38$
\testStop
\kluczStart
A
\kluczStop



\zadStart{Przykład z Wikieł P 4.3a moja wersja nr 802}


Obliczyć granicę funkcji $\lim\limits_{x\to\ 0}\frac{38 \cdot x}{tan(9 \cdot x)}$.
\zadStop
\rozwStart{Patryk Wirkus}{}
$$\lim\limits_{x\to\ 0}\frac{38 \cdot x}{tan(9 \cdot x)}=\lim\limits_{x\to\ 0}\frac{38 \cdot x \cdot cos(9 \cdot x)}{sin(9 \cdot x)}=\lim\limits_{x\to\ 0}\frac{38 \cdot cos(9 \cdot x)}{\frac{sin(9 \cdot x)}{x}}=\lim\limits_{x\to\ 0}\frac{38 \cdot cos(9 \cdot x)}{9 \cdot \frac{sin(9 \cdot x)}{9 \cdot x}} = \frac{38}{9}$$
\rozwStop
\odpStart
$\frac{38}{9}$
\odpStop
\testStart
A.$\frac{38}{9}$
B.$\infty$
C.$-\infty$
D.$0$
E.$-\frac{38}{9}$
F.$\frac{9}{38}$
G.$-\frac{9}{38}$
H.$9$
I.$38$
\testStop
\kluczStart
A
\kluczStop



\zadStart{Przykład z Wikieł P 4.3a moja wersja nr 803}


Obliczyć granicę funkcji $\lim\limits_{x\to\ 0}\frac{38 \cdot x}{tan(11 \cdot x)}$.
\zadStop
\rozwStart{Patryk Wirkus}{}
$$\lim\limits_{x\to\ 0}\frac{38 \cdot x}{tan(11 \cdot x)}=\lim\limits_{x\to\ 0}\frac{38 \cdot x \cdot cos(11 \cdot x)}{sin(11 \cdot x)}=\lim\limits_{x\to\ 0}\frac{38 \cdot cos(11 \cdot x)}{\frac{sin(11 \cdot x)}{x}}=\lim\limits_{x\to\ 0}\frac{38 \cdot cos(11 \cdot x)}{11 \cdot \frac{sin(11 \cdot x)}{11 \cdot x}} = \frac{38}{11}$$
\rozwStop
\odpStart
$\frac{38}{11}$
\odpStop
\testStart
A.$\frac{38}{11}$
B.$\infty$
C.$-\infty$
D.$0$
E.$-\frac{38}{11}$
F.$\frac{11}{38}$
G.$-\frac{11}{38}$
H.$11$
I.$38$
\testStop
\kluczStart
A
\kluczStop



\zadStart{Przykład z Wikieł P 4.3a moja wersja nr 804}


Obliczyć granicę funkcji $\lim\limits_{x\to\ 0}\frac{38 \cdot x}{tan(13 \cdot x)}$.
\zadStop
\rozwStart{Patryk Wirkus}{}
$$\lim\limits_{x\to\ 0}\frac{38 \cdot x}{tan(13 \cdot x)}=\lim\limits_{x\to\ 0}\frac{38 \cdot x \cdot cos(13 \cdot x)}{sin(13 \cdot x)}=\lim\limits_{x\to\ 0}\frac{38 \cdot cos(13 \cdot x)}{\frac{sin(13 \cdot x)}{x}}=\lim\limits_{x\to\ 0}\frac{38 \cdot cos(13 \cdot x)}{13 \cdot \frac{sin(13 \cdot x)}{13 \cdot x}} = \frac{38}{13}$$
\rozwStop
\odpStart
$\frac{38}{13}$
\odpStop
\testStart
A.$\frac{38}{13}$
B.$\infty$
C.$-\infty$
D.$0$
E.$-\frac{38}{13}$
F.$\frac{13}{38}$
G.$-\frac{13}{38}$
H.$13$
I.$38$
\testStop
\kluczStart
A
\kluczStop



\zadStart{Przykład z Wikieł P 4.3a moja wersja nr 805}


Obliczyć granicę funkcji $\lim\limits_{x\to\ 0}\frac{38 \cdot x}{tan(15 \cdot x)}$.
\zadStop
\rozwStart{Patryk Wirkus}{}
$$\lim\limits_{x\to\ 0}\frac{38 \cdot x}{tan(15 \cdot x)}=\lim\limits_{x\to\ 0}\frac{38 \cdot x \cdot cos(15 \cdot x)}{sin(15 \cdot x)}=\lim\limits_{x\to\ 0}\frac{38 \cdot cos(15 \cdot x)}{\frac{sin(15 \cdot x)}{x}}=\lim\limits_{x\to\ 0}\frac{38 \cdot cos(15 \cdot x)}{15 \cdot \frac{sin(15 \cdot x)}{15 \cdot x}} = \frac{38}{15}$$
\rozwStop
\odpStart
$\frac{38}{15}$
\odpStop
\testStart
A.$\frac{38}{15}$
B.$\infty$
C.$-\infty$
D.$0$
E.$-\frac{38}{15}$
F.$\frac{15}{38}$
G.$-\frac{15}{38}$
H.$15$
I.$38$
\testStop
\kluczStart
A
\kluczStop



\zadStart{Przykład z Wikieł P 4.3a moja wersja nr 806}


Obliczyć granicę funkcji $\lim\limits_{x\to\ 0}\frac{38 \cdot x}{tan(17 \cdot x)}$.
\zadStop
\rozwStart{Patryk Wirkus}{}
$$\lim\limits_{x\to\ 0}\frac{38 \cdot x}{tan(17 \cdot x)}=\lim\limits_{x\to\ 0}\frac{38 \cdot x \cdot cos(17 \cdot x)}{sin(17 \cdot x)}=\lim\limits_{x\to\ 0}\frac{38 \cdot cos(17 \cdot x)}{\frac{sin(17 \cdot x)}{x}}=\lim\limits_{x\to\ 0}\frac{38 \cdot cos(17 \cdot x)}{17 \cdot \frac{sin(17 \cdot x)}{17 \cdot x}} = \frac{38}{17}$$
\rozwStop
\odpStart
$\frac{38}{17}$
\odpStop
\testStart
A.$\frac{38}{17}$
B.$\infty$
C.$-\infty$
D.$0$
E.$-\frac{38}{17}$
F.$\frac{17}{38}$
G.$-\frac{17}{38}$
H.$17$
I.$38$
\testStop
\kluczStart
A
\kluczStop



\zadStart{Przykład z Wikieł P 4.3a moja wersja nr 807}


Obliczyć granicę funkcji $\lim\limits_{x\to\ 0}\frac{38 \cdot x}{tan(21 \cdot x)}$.
\zadStop
\rozwStart{Patryk Wirkus}{}
$$\lim\limits_{x\to\ 0}\frac{38 \cdot x}{tan(21 \cdot x)}=\lim\limits_{x\to\ 0}\frac{38 \cdot x \cdot cos(21 \cdot x)}{sin(21 \cdot x)}=\lim\limits_{x\to\ 0}\frac{38 \cdot cos(21 \cdot x)}{\frac{sin(21 \cdot x)}{x}}=\lim\limits_{x\to\ 0}\frac{38 \cdot cos(21 \cdot x)}{21 \cdot \frac{sin(21 \cdot x)}{21 \cdot x}} = \frac{38}{21}$$
\rozwStop
\odpStart
$\frac{38}{21}$
\odpStop
\testStart
A.$\frac{38}{21}$
B.$\infty$
C.$-\infty$
D.$0$
E.$-\frac{38}{21}$
F.$\frac{21}{38}$
G.$-\frac{21}{38}$
H.$21$
I.$38$
\testStop
\kluczStart
A
\kluczStop



\zadStart{Przykład z Wikieł P 4.3a moja wersja nr 808}


Obliczyć granicę funkcji $\lim\limits_{x\to\ 0}\frac{38 \cdot x}{tan(23 \cdot x)}$.
\zadStop
\rozwStart{Patryk Wirkus}{}
$$\lim\limits_{x\to\ 0}\frac{38 \cdot x}{tan(23 \cdot x)}=\lim\limits_{x\to\ 0}\frac{38 \cdot x \cdot cos(23 \cdot x)}{sin(23 \cdot x)}=\lim\limits_{x\to\ 0}\frac{38 \cdot cos(23 \cdot x)}{\frac{sin(23 \cdot x)}{x}}=\lim\limits_{x\to\ 0}\frac{38 \cdot cos(23 \cdot x)}{23 \cdot \frac{sin(23 \cdot x)}{23 \cdot x}} = \frac{38}{23}$$
\rozwStop
\odpStart
$\frac{38}{23}$
\odpStop
\testStart
A.$\frac{38}{23}$
B.$\infty$
C.$-\infty$
D.$0$
E.$-\frac{38}{23}$
F.$\frac{23}{38}$
G.$-\frac{23}{38}$
H.$23$
I.$38$
\testStop
\kluczStart
A
\kluczStop



\zadStart{Przykład z Wikieł P 4.3a moja wersja nr 809}


Obliczyć granicę funkcji $\lim\limits_{x\to\ 0}\frac{38 \cdot x}{tan(25 \cdot x)}$.
\zadStop
\rozwStart{Patryk Wirkus}{}
$$\lim\limits_{x\to\ 0}\frac{38 \cdot x}{tan(25 \cdot x)}=\lim\limits_{x\to\ 0}\frac{38 \cdot x \cdot cos(25 \cdot x)}{sin(25 \cdot x)}=\lim\limits_{x\to\ 0}\frac{38 \cdot cos(25 \cdot x)}{\frac{sin(25 \cdot x)}{x}}=\lim\limits_{x\to\ 0}\frac{38 \cdot cos(25 \cdot x)}{25 \cdot \frac{sin(25 \cdot x)}{25 \cdot x}} = \frac{38}{25}$$
\rozwStop
\odpStart
$\frac{38}{25}$
\odpStop
\testStart
A.$\frac{38}{25}$
B.$\infty$
C.$-\infty$
D.$0$
E.$-\frac{38}{25}$
F.$\frac{25}{38}$
G.$-\frac{25}{38}$
H.$25$
I.$38$
\testStop
\kluczStart
A
\kluczStop



\zadStart{Przykład z Wikieł P 4.3a moja wersja nr 810}


Obliczyć granicę funkcji $\lim\limits_{x\to\ 0}\frac{38 \cdot x}{tan(27 \cdot x)}$.
\zadStop
\rozwStart{Patryk Wirkus}{}
$$\lim\limits_{x\to\ 0}\frac{38 \cdot x}{tan(27 \cdot x)}=\lim\limits_{x\to\ 0}\frac{38 \cdot x \cdot cos(27 \cdot x)}{sin(27 \cdot x)}=\lim\limits_{x\to\ 0}\frac{38 \cdot cos(27 \cdot x)}{\frac{sin(27 \cdot x)}{x}}=\lim\limits_{x\to\ 0}\frac{38 \cdot cos(27 \cdot x)}{27 \cdot \frac{sin(27 \cdot x)}{27 \cdot x}} = \frac{38}{27}$$
\rozwStop
\odpStart
$\frac{38}{27}$
\odpStop
\testStart
A.$\frac{38}{27}$
B.$\infty$
C.$-\infty$
D.$0$
E.$-\frac{38}{27}$
F.$\frac{27}{38}$
G.$-\frac{27}{38}$
H.$27$
I.$38$
\testStop
\kluczStart
A
\kluczStop



\zadStart{Przykład z Wikieł P 4.3a moja wersja nr 811}


Obliczyć granicę funkcji $\lim\limits_{x\to\ 0}\frac{38 \cdot x}{tan(29 \cdot x)}$.
\zadStop
\rozwStart{Patryk Wirkus}{}
$$\lim\limits_{x\to\ 0}\frac{38 \cdot x}{tan(29 \cdot x)}=\lim\limits_{x\to\ 0}\frac{38 \cdot x \cdot cos(29 \cdot x)}{sin(29 \cdot x)}=\lim\limits_{x\to\ 0}\frac{38 \cdot cos(29 \cdot x)}{\frac{sin(29 \cdot x)}{x}}=\lim\limits_{x\to\ 0}\frac{38 \cdot cos(29 \cdot x)}{29 \cdot \frac{sin(29 \cdot x)}{29 \cdot x}} = \frac{38}{29}$$
\rozwStop
\odpStart
$\frac{38}{29}$
\odpStop
\testStart
A.$\frac{38}{29}$
B.$\infty$
C.$-\infty$
D.$0$
E.$-\frac{38}{29}$
F.$\frac{29}{38}$
G.$-\frac{29}{38}$
H.$29$
I.$38$
\testStop
\kluczStart
A
\kluczStop



\zadStart{Przykład z Wikieł P 4.3a moja wersja nr 812}


Obliczyć granicę funkcji $\lim\limits_{x\to\ 0}\frac{38 \cdot x}{tan(31 \cdot x)}$.
\zadStop
\rozwStart{Patryk Wirkus}{}
$$\lim\limits_{x\to\ 0}\frac{38 \cdot x}{tan(31 \cdot x)}=\lim\limits_{x\to\ 0}\frac{38 \cdot x \cdot cos(31 \cdot x)}{sin(31 \cdot x)}=\lim\limits_{x\to\ 0}\frac{38 \cdot cos(31 \cdot x)}{\frac{sin(31 \cdot x)}{x}}=\lim\limits_{x\to\ 0}\frac{38 \cdot cos(31 \cdot x)}{31 \cdot \frac{sin(31 \cdot x)}{31 \cdot x}} = \frac{38}{31}$$
\rozwStop
\odpStart
$\frac{38}{31}$
\odpStop
\testStart
A.$\frac{38}{31}$
B.$\infty$
C.$-\infty$
D.$0$
E.$-\frac{38}{31}$
F.$\frac{31}{38}$
G.$-\frac{31}{38}$
H.$31$
I.$38$
\testStop
\kluczStart
A
\kluczStop



\zadStart{Przykład z Wikieł P 4.3a moja wersja nr 813}


Obliczyć granicę funkcji $\lim\limits_{x\to\ 0}\frac{38 \cdot x}{tan(33 \cdot x)}$.
\zadStop
\rozwStart{Patryk Wirkus}{}
$$\lim\limits_{x\to\ 0}\frac{38 \cdot x}{tan(33 \cdot x)}=\lim\limits_{x\to\ 0}\frac{38 \cdot x \cdot cos(33 \cdot x)}{sin(33 \cdot x)}=\lim\limits_{x\to\ 0}\frac{38 \cdot cos(33 \cdot x)}{\frac{sin(33 \cdot x)}{x}}=\lim\limits_{x\to\ 0}\frac{38 \cdot cos(33 \cdot x)}{33 \cdot \frac{sin(33 \cdot x)}{33 \cdot x}} = \frac{38}{33}$$
\rozwStop
\odpStart
$\frac{38}{33}$
\odpStop
\testStart
A.$\frac{38}{33}$
B.$\infty$
C.$-\infty$
D.$0$
E.$-\frac{38}{33}$
F.$\frac{33}{38}$
G.$-\frac{33}{38}$
H.$33$
I.$38$
\testStop
\kluczStart
A
\kluczStop



\zadStart{Przykład z Wikieł P 4.3a moja wersja nr 814}


Obliczyć granicę funkcji $\lim\limits_{x\to\ 0}\frac{38 \cdot x}{tan(35 \cdot x)}$.
\zadStop
\rozwStart{Patryk Wirkus}{}
$$\lim\limits_{x\to\ 0}\frac{38 \cdot x}{tan(35 \cdot x)}=\lim\limits_{x\to\ 0}\frac{38 \cdot x \cdot cos(35 \cdot x)}{sin(35 \cdot x)}=\lim\limits_{x\to\ 0}\frac{38 \cdot cos(35 \cdot x)}{\frac{sin(35 \cdot x)}{x}}=\lim\limits_{x\to\ 0}\frac{38 \cdot cos(35 \cdot x)}{35 \cdot \frac{sin(35 \cdot x)}{35 \cdot x}} = \frac{38}{35}$$
\rozwStop
\odpStart
$\frac{38}{35}$
\odpStop
\testStart
A.$\frac{38}{35}$
B.$\infty$
C.$-\infty$
D.$0$
E.$-\frac{38}{35}$
F.$\frac{35}{38}$
G.$-\frac{35}{38}$
H.$35$
I.$38$
\testStop
\kluczStart
A
\kluczStop



\zadStart{Przykład z Wikieł P 4.3a moja wersja nr 815}


Obliczyć granicę funkcji $\lim\limits_{x\to\ 0}\frac{38 \cdot x}{tan(37 \cdot x)}$.
\zadStop
\rozwStart{Patryk Wirkus}{}
$$\lim\limits_{x\to\ 0}\frac{38 \cdot x}{tan(37 \cdot x)}=\lim\limits_{x\to\ 0}\frac{38 \cdot x \cdot cos(37 \cdot x)}{sin(37 \cdot x)}=\lim\limits_{x\to\ 0}\frac{38 \cdot cos(37 \cdot x)}{\frac{sin(37 \cdot x)}{x}}=\lim\limits_{x\to\ 0}\frac{38 \cdot cos(37 \cdot x)}{37 \cdot \frac{sin(37 \cdot x)}{37 \cdot x}} = \frac{38}{37}$$
\rozwStop
\odpStart
$\frac{38}{37}$
\odpStop
\testStart
A.$\frac{38}{37}$
B.$\infty$
C.$-\infty$
D.$0$
E.$-\frac{38}{37}$
F.$\frac{37}{38}$
G.$-\frac{37}{38}$
H.$37$
I.$38$
\testStop
\kluczStart
A
\kluczStop



\zadStart{Przykład z Wikieł P 4.3a moja wersja nr 816}


Obliczyć granicę funkcji $\lim\limits_{x\to\ 0}\frac{38 \cdot x}{tan(39 \cdot x)}$.
\zadStop
\rozwStart{Patryk Wirkus}{}
$$\lim\limits_{x\to\ 0}\frac{38 \cdot x}{tan(39 \cdot x)}=\lim\limits_{x\to\ 0}\frac{38 \cdot x \cdot cos(39 \cdot x)}{sin(39 \cdot x)}=\lim\limits_{x\to\ 0}\frac{38 \cdot cos(39 \cdot x)}{\frac{sin(39 \cdot x)}{x}}=\lim\limits_{x\to\ 0}\frac{38 \cdot cos(39 \cdot x)}{39 \cdot \frac{sin(39 \cdot x)}{39 \cdot x}} = \frac{38}{39}$$
\rozwStop
\odpStart
$\frac{38}{39}$
\odpStop
\testStart
A.$\frac{38}{39}$
B.$\infty$
C.$-\infty$
D.$0$
E.$-\frac{38}{39}$
F.$\frac{39}{38}$
G.$-\frac{39}{38}$
H.$39$
I.$38$
\testStop
\kluczStart
A
\kluczStop



\zadStart{Przykład z Wikieł P 4.3a moja wersja nr 817}


Obliczyć granicę funkcji $\lim\limits_{x\to\ 0}\frac{39 \cdot x}{tan(2 \cdot x)}$.
\zadStop
\rozwStart{Patryk Wirkus}{}
$$\lim\limits_{x\to\ 0}\frac{39 \cdot x}{tan(2 \cdot x)}=\lim\limits_{x\to\ 0}\frac{39 \cdot x \cdot cos(2 \cdot x)}{sin(2 \cdot x)}=\lim\limits_{x\to\ 0}\frac{39 \cdot cos(2 \cdot x)}{\frac{sin(2 \cdot x)}{x}}=\lim\limits_{x\to\ 0}\frac{39 \cdot cos(2 \cdot x)}{2 \cdot \frac{sin(2 \cdot x)}{2 \cdot x}} = \frac{39}{2}$$
\rozwStop
\odpStart
$\frac{39}{2}$
\odpStop
\testStart
A.$\frac{39}{2}$
B.$\infty$
C.$-\infty$
D.$0$
E.$-\frac{39}{2}$
F.$\frac{2}{39}$
G.$-\frac{2}{39}$
H.$2$
I.$39$
\testStop
\kluczStart
A
\kluczStop



\zadStart{Przykład z Wikieł P 4.3a moja wersja nr 818}


Obliczyć granicę funkcji $\lim\limits_{x\to\ 0}\frac{39 \cdot x}{tan(4 \cdot x)}$.
\zadStop
\rozwStart{Patryk Wirkus}{}
$$\lim\limits_{x\to\ 0}\frac{39 \cdot x}{tan(4 \cdot x)}=\lim\limits_{x\to\ 0}\frac{39 \cdot x \cdot cos(4 \cdot x)}{sin(4 \cdot x)}=\lim\limits_{x\to\ 0}\frac{39 \cdot cos(4 \cdot x)}{\frac{sin(4 \cdot x)}{x}}=\lim\limits_{x\to\ 0}\frac{39 \cdot cos(4 \cdot x)}{4 \cdot \frac{sin(4 \cdot x)}{4 \cdot x}} = \frac{39}{4}$$
\rozwStop
\odpStart
$\frac{39}{4}$
\odpStop
\testStart
A.$\frac{39}{4}$
B.$\infty$
C.$-\infty$
D.$0$
E.$-\frac{39}{4}$
F.$\frac{4}{39}$
G.$-\frac{4}{39}$
H.$4$
I.$39$
\testStop
\kluczStart
A
\kluczStop



\zadStart{Przykład z Wikieł P 4.3a moja wersja nr 819}


Obliczyć granicę funkcji $\lim\limits_{x\to\ 0}\frac{39 \cdot x}{tan(5 \cdot x)}$.
\zadStop
\rozwStart{Patryk Wirkus}{}
$$\lim\limits_{x\to\ 0}\frac{39 \cdot x}{tan(5 \cdot x)}=\lim\limits_{x\to\ 0}\frac{39 \cdot x \cdot cos(5 \cdot x)}{sin(5 \cdot x)}=\lim\limits_{x\to\ 0}\frac{39 \cdot cos(5 \cdot x)}{\frac{sin(5 \cdot x)}{x}}=\lim\limits_{x\to\ 0}\frac{39 \cdot cos(5 \cdot x)}{5 \cdot \frac{sin(5 \cdot x)}{5 \cdot x}} = \frac{39}{5}$$
\rozwStop
\odpStart
$\frac{39}{5}$
\odpStop
\testStart
A.$\frac{39}{5}$
B.$\infty$
C.$-\infty$
D.$0$
E.$-\frac{39}{5}$
F.$\frac{5}{39}$
G.$-\frac{5}{39}$
H.$5$
I.$39$
\testStop
\kluczStart
A
\kluczStop



\zadStart{Przykład z Wikieł P 4.3a moja wersja nr 820}


Obliczyć granicę funkcji $\lim\limits_{x\to\ 0}\frac{39 \cdot x}{tan(7 \cdot x)}$.
\zadStop
\rozwStart{Patryk Wirkus}{}
$$\lim\limits_{x\to\ 0}\frac{39 \cdot x}{tan(7 \cdot x)}=\lim\limits_{x\to\ 0}\frac{39 \cdot x \cdot cos(7 \cdot x)}{sin(7 \cdot x)}=\lim\limits_{x\to\ 0}\frac{39 \cdot cos(7 \cdot x)}{\frac{sin(7 \cdot x)}{x}}=\lim\limits_{x\to\ 0}\frac{39 \cdot cos(7 \cdot x)}{7 \cdot \frac{sin(7 \cdot x)}{7 \cdot x}} = \frac{39}{7}$$
\rozwStop
\odpStart
$\frac{39}{7}$
\odpStop
\testStart
A.$\frac{39}{7}$
B.$\infty$
C.$-\infty$
D.$0$
E.$-\frac{39}{7}$
F.$\frac{7}{39}$
G.$-\frac{7}{39}$
H.$7$
I.$39$
\testStop
\kluczStart
A
\kluczStop



\zadStart{Przykład z Wikieł P 4.3a moja wersja nr 821}


Obliczyć granicę funkcji $\lim\limits_{x\to\ 0}\frac{39 \cdot x}{tan(8 \cdot x)}$.
\zadStop
\rozwStart{Patryk Wirkus}{}
$$\lim\limits_{x\to\ 0}\frac{39 \cdot x}{tan(8 \cdot x)}=\lim\limits_{x\to\ 0}\frac{39 \cdot x \cdot cos(8 \cdot x)}{sin(8 \cdot x)}=\lim\limits_{x\to\ 0}\frac{39 \cdot cos(8 \cdot x)}{\frac{sin(8 \cdot x)}{x}}=\lim\limits_{x\to\ 0}\frac{39 \cdot cos(8 \cdot x)}{8 \cdot \frac{sin(8 \cdot x)}{8 \cdot x}} = \frac{39}{8}$$
\rozwStop
\odpStart
$\frac{39}{8}$
\odpStop
\testStart
A.$\frac{39}{8}$
B.$\infty$
C.$-\infty$
D.$0$
E.$-\frac{39}{8}$
F.$\frac{8}{39}$
G.$-\frac{8}{39}$
H.$8$
I.$39$
\testStop
\kluczStart
A
\kluczStop



\zadStart{Przykład z Wikieł P 4.3a moja wersja nr 822}


Obliczyć granicę funkcji $\lim\limits_{x\to\ 0}\frac{39 \cdot x}{tan(10 \cdot x)}$.
\zadStop
\rozwStart{Patryk Wirkus}{}
$$\lim\limits_{x\to\ 0}\frac{39 \cdot x}{tan(10 \cdot x)}=\lim\limits_{x\to\ 0}\frac{39 \cdot x \cdot cos(10 \cdot x)}{sin(10 \cdot x)}=\lim\limits_{x\to\ 0}\frac{39 \cdot cos(10 \cdot x)}{\frac{sin(10 \cdot x)}{x}}=\lim\limits_{x\to\ 0}\frac{39 \cdot cos(10 \cdot x)}{10 \cdot \frac{sin(10 \cdot x)}{10 \cdot x}} = \frac{39}{10}$$
\rozwStop
\odpStart
$\frac{39}{10}$
\odpStop
\testStart
A.$\frac{39}{10}$
B.$\infty$
C.$-\infty$
D.$0$
E.$-\frac{39}{10}$
F.$\frac{10}{39}$
G.$-\frac{10}{39}$
H.$10$
I.$39$
\testStop
\kluczStart
A
\kluczStop



\zadStart{Przykład z Wikieł P 4.3a moja wersja nr 823}


Obliczyć granicę funkcji $\lim\limits_{x\to\ 0}\frac{39 \cdot x}{tan(11 \cdot x)}$.
\zadStop
\rozwStart{Patryk Wirkus}{}
$$\lim\limits_{x\to\ 0}\frac{39 \cdot x}{tan(11 \cdot x)}=\lim\limits_{x\to\ 0}\frac{39 \cdot x \cdot cos(11 \cdot x)}{sin(11 \cdot x)}=\lim\limits_{x\to\ 0}\frac{39 \cdot cos(11 \cdot x)}{\frac{sin(11 \cdot x)}{x}}=\lim\limits_{x\to\ 0}\frac{39 \cdot cos(11 \cdot x)}{11 \cdot \frac{sin(11 \cdot x)}{11 \cdot x}} = \frac{39}{11}$$
\rozwStop
\odpStart
$\frac{39}{11}$
\odpStop
\testStart
A.$\frac{39}{11}$
B.$\infty$
C.$-\infty$
D.$0$
E.$-\frac{39}{11}$
F.$\frac{11}{39}$
G.$-\frac{11}{39}$
H.$11$
I.$39$
\testStop
\kluczStart
A
\kluczStop



\zadStart{Przykład z Wikieł P 4.3a moja wersja nr 824}


Obliczyć granicę funkcji $\lim\limits_{x\to\ 0}\frac{39 \cdot x}{tan(14 \cdot x)}$.
\zadStop
\rozwStart{Patryk Wirkus}{}
$$\lim\limits_{x\to\ 0}\frac{39 \cdot x}{tan(14 \cdot x)}=\lim\limits_{x\to\ 0}\frac{39 \cdot x \cdot cos(14 \cdot x)}{sin(14 \cdot x)}=\lim\limits_{x\to\ 0}\frac{39 \cdot cos(14 \cdot x)}{\frac{sin(14 \cdot x)}{x}}=\lim\limits_{x\to\ 0}\frac{39 \cdot cos(14 \cdot x)}{14 \cdot \frac{sin(14 \cdot x)}{14 \cdot x}} = \frac{39}{14}$$
\rozwStop
\odpStart
$\frac{39}{14}$
\odpStop
\testStart
A.$\frac{39}{14}$
B.$\infty$
C.$-\infty$
D.$0$
E.$-\frac{39}{14}$
F.$\frac{14}{39}$
G.$-\frac{14}{39}$
H.$14$
I.$39$
\testStop
\kluczStart
A
\kluczStop



\zadStart{Przykład z Wikieł P 4.3a moja wersja nr 825}


Obliczyć granicę funkcji $\lim\limits_{x\to\ 0}\frac{39 \cdot x}{tan(16 \cdot x)}$.
\zadStop
\rozwStart{Patryk Wirkus}{}
$$\lim\limits_{x\to\ 0}\frac{39 \cdot x}{tan(16 \cdot x)}=\lim\limits_{x\to\ 0}\frac{39 \cdot x \cdot cos(16 \cdot x)}{sin(16 \cdot x)}=\lim\limits_{x\to\ 0}\frac{39 \cdot cos(16 \cdot x)}{\frac{sin(16 \cdot x)}{x}}=\lim\limits_{x\to\ 0}\frac{39 \cdot cos(16 \cdot x)}{16 \cdot \frac{sin(16 \cdot x)}{16 \cdot x}} = \frac{39}{16}$$
\rozwStop
\odpStart
$\frac{39}{16}$
\odpStop
\testStart
A.$\frac{39}{16}$
B.$\infty$
C.$-\infty$
D.$0$
E.$-\frac{39}{16}$
F.$\frac{16}{39}$
G.$-\frac{16}{39}$
H.$16$
I.$39$
\testStop
\kluczStart
A
\kluczStop



\zadStart{Przykład z Wikieł P 4.3a moja wersja nr 826}


Obliczyć granicę funkcji $\lim\limits_{x\to\ 0}\frac{39 \cdot x}{tan(17 \cdot x)}$.
\zadStop
\rozwStart{Patryk Wirkus}{}
$$\lim\limits_{x\to\ 0}\frac{39 \cdot x}{tan(17 \cdot x)}=\lim\limits_{x\to\ 0}\frac{39 \cdot x \cdot cos(17 \cdot x)}{sin(17 \cdot x)}=\lim\limits_{x\to\ 0}\frac{39 \cdot cos(17 \cdot x)}{\frac{sin(17 \cdot x)}{x}}=\lim\limits_{x\to\ 0}\frac{39 \cdot cos(17 \cdot x)}{17 \cdot \frac{sin(17 \cdot x)}{17 \cdot x}} = \frac{39}{17}$$
\rozwStop
\odpStart
$\frac{39}{17}$
\odpStop
\testStart
A.$\frac{39}{17}$
B.$\infty$
C.$-\infty$
D.$0$
E.$-\frac{39}{17}$
F.$\frac{17}{39}$
G.$-\frac{17}{39}$
H.$17$
I.$39$
\testStop
\kluczStart
A
\kluczStop



\zadStart{Przykład z Wikieł P 4.3a moja wersja nr 827}


Obliczyć granicę funkcji $\lim\limits_{x\to\ 0}\frac{39 \cdot x}{tan(19 \cdot x)}$.
\zadStop
\rozwStart{Patryk Wirkus}{}
$$\lim\limits_{x\to\ 0}\frac{39 \cdot x}{tan(19 \cdot x)}=\lim\limits_{x\to\ 0}\frac{39 \cdot x \cdot cos(19 \cdot x)}{sin(19 \cdot x)}=\lim\limits_{x\to\ 0}\frac{39 \cdot cos(19 \cdot x)}{\frac{sin(19 \cdot x)}{x}}=\lim\limits_{x\to\ 0}\frac{39 \cdot cos(19 \cdot x)}{19 \cdot \frac{sin(19 \cdot x)}{19 \cdot x}} = \frac{39}{19}$$
\rozwStop
\odpStart
$\frac{39}{19}$
\odpStop
\testStart
A.$\frac{39}{19}$
B.$\infty$
C.$-\infty$
D.$0$
E.$-\frac{39}{19}$
F.$\frac{19}{39}$
G.$-\frac{19}{39}$
H.$19$
I.$39$
\testStop
\kluczStart
A
\kluczStop



\zadStart{Przykład z Wikieł P 4.3a moja wersja nr 828}


Obliczyć granicę funkcji $\lim\limits_{x\to\ 0}\frac{39 \cdot x}{tan(20 \cdot x)}$.
\zadStop
\rozwStart{Patryk Wirkus}{}
$$\lim\limits_{x\to\ 0}\frac{39 \cdot x}{tan(20 \cdot x)}=\lim\limits_{x\to\ 0}\frac{39 \cdot x \cdot cos(20 \cdot x)}{sin(20 \cdot x)}=\lim\limits_{x\to\ 0}\frac{39 \cdot cos(20 \cdot x)}{\frac{sin(20 \cdot x)}{x}}=\lim\limits_{x\to\ 0}\frac{39 \cdot cos(20 \cdot x)}{20 \cdot \frac{sin(20 \cdot x)}{20 \cdot x}} = \frac{39}{20}$$
\rozwStop
\odpStart
$\frac{39}{20}$
\odpStop
\testStart
A.$\frac{39}{20}$
B.$\infty$
C.$-\infty$
D.$0$
E.$-\frac{39}{20}$
F.$\frac{20}{39}$
G.$-\frac{20}{39}$
H.$20$
I.$39$
\testStop
\kluczStart
A
\kluczStop



\zadStart{Przykład z Wikieł P 4.3a moja wersja nr 829}


Obliczyć granicę funkcji $\lim\limits_{x\to\ 0}\frac{39 \cdot x}{tan(22 \cdot x)}$.
\zadStop
\rozwStart{Patryk Wirkus}{}
$$\lim\limits_{x\to\ 0}\frac{39 \cdot x}{tan(22 \cdot x)}=\lim\limits_{x\to\ 0}\frac{39 \cdot x \cdot cos(22 \cdot x)}{sin(22 \cdot x)}=\lim\limits_{x\to\ 0}\frac{39 \cdot cos(22 \cdot x)}{\frac{sin(22 \cdot x)}{x}}=\lim\limits_{x\to\ 0}\frac{39 \cdot cos(22 \cdot x)}{22 \cdot \frac{sin(22 \cdot x)}{22 \cdot x}} = \frac{39}{22}$$
\rozwStop
\odpStart
$\frac{39}{22}$
\odpStop
\testStart
A.$\frac{39}{22}$
B.$\infty$
C.$-\infty$
D.$0$
E.$-\frac{39}{22}$
F.$\frac{22}{39}$
G.$-\frac{22}{39}$
H.$22$
I.$39$
\testStop
\kluczStart
A
\kluczStop



\zadStart{Przykład z Wikieł P 4.3a moja wersja nr 830}


Obliczyć granicę funkcji $\lim\limits_{x\to\ 0}\frac{39 \cdot x}{tan(23 \cdot x)}$.
\zadStop
\rozwStart{Patryk Wirkus}{}
$$\lim\limits_{x\to\ 0}\frac{39 \cdot x}{tan(23 \cdot x)}=\lim\limits_{x\to\ 0}\frac{39 \cdot x \cdot cos(23 \cdot x)}{sin(23 \cdot x)}=\lim\limits_{x\to\ 0}\frac{39 \cdot cos(23 \cdot x)}{\frac{sin(23 \cdot x)}{x}}=\lim\limits_{x\to\ 0}\frac{39 \cdot cos(23 \cdot x)}{23 \cdot \frac{sin(23 \cdot x)}{23 \cdot x}} = \frac{39}{23}$$
\rozwStop
\odpStart
$\frac{39}{23}$
\odpStop
\testStart
A.$\frac{39}{23}$
B.$\infty$
C.$-\infty$
D.$0$
E.$-\frac{39}{23}$
F.$\frac{23}{39}$
G.$-\frac{23}{39}$
H.$23$
I.$39$
\testStop
\kluczStart
A
\kluczStop



\zadStart{Przykład z Wikieł P 4.3a moja wersja nr 831}


Obliczyć granicę funkcji $\lim\limits_{x\to\ 0}\frac{39 \cdot x}{tan(25 \cdot x)}$.
\zadStop
\rozwStart{Patryk Wirkus}{}
$$\lim\limits_{x\to\ 0}\frac{39 \cdot x}{tan(25 \cdot x)}=\lim\limits_{x\to\ 0}\frac{39 \cdot x \cdot cos(25 \cdot x)}{sin(25 \cdot x)}=\lim\limits_{x\to\ 0}\frac{39 \cdot cos(25 \cdot x)}{\frac{sin(25 \cdot x)}{x}}=\lim\limits_{x\to\ 0}\frac{39 \cdot cos(25 \cdot x)}{25 \cdot \frac{sin(25 \cdot x)}{25 \cdot x}} = \frac{39}{25}$$
\rozwStop
\odpStart
$\frac{39}{25}$
\odpStop
\testStart
A.$\frac{39}{25}$
B.$\infty$
C.$-\infty$
D.$0$
E.$-\frac{39}{25}$
F.$\frac{25}{39}$
G.$-\frac{25}{39}$
H.$25$
I.$39$
\testStop
\kluczStart
A
\kluczStop



\zadStart{Przykład z Wikieł P 4.3a moja wersja nr 832}


Obliczyć granicę funkcji $\lim\limits_{x\to\ 0}\frac{39 \cdot x}{tan(28 \cdot x)}$.
\zadStop
\rozwStart{Patryk Wirkus}{}
$$\lim\limits_{x\to\ 0}\frac{39 \cdot x}{tan(28 \cdot x)}=\lim\limits_{x\to\ 0}\frac{39 \cdot x \cdot cos(28 \cdot x)}{sin(28 \cdot x)}=\lim\limits_{x\to\ 0}\frac{39 \cdot cos(28 \cdot x)}{\frac{sin(28 \cdot x)}{x}}=\lim\limits_{x\to\ 0}\frac{39 \cdot cos(28 \cdot x)}{28 \cdot \frac{sin(28 \cdot x)}{28 \cdot x}} = \frac{39}{28}$$
\rozwStop
\odpStart
$\frac{39}{28}$
\odpStop
\testStart
A.$\frac{39}{28}$
B.$\infty$
C.$-\infty$
D.$0$
E.$-\frac{39}{28}$
F.$\frac{28}{39}$
G.$-\frac{28}{39}$
H.$28$
I.$39$
\testStop
\kluczStart
A
\kluczStop



\zadStart{Przykład z Wikieł P 4.3a moja wersja nr 833}


Obliczyć granicę funkcji $\lim\limits_{x\to\ 0}\frac{39 \cdot x}{tan(29 \cdot x)}$.
\zadStop
\rozwStart{Patryk Wirkus}{}
$$\lim\limits_{x\to\ 0}\frac{39 \cdot x}{tan(29 \cdot x)}=\lim\limits_{x\to\ 0}\frac{39 \cdot x \cdot cos(29 \cdot x)}{sin(29 \cdot x)}=\lim\limits_{x\to\ 0}\frac{39 \cdot cos(29 \cdot x)}{\frac{sin(29 \cdot x)}{x}}=\lim\limits_{x\to\ 0}\frac{39 \cdot cos(29 \cdot x)}{29 \cdot \frac{sin(29 \cdot x)}{29 \cdot x}} = \frac{39}{29}$$
\rozwStop
\odpStart
$\frac{39}{29}$
\odpStop
\testStart
A.$\frac{39}{29}$
B.$\infty$
C.$-\infty$
D.$0$
E.$-\frac{39}{29}$
F.$\frac{29}{39}$
G.$-\frac{29}{39}$
H.$29$
I.$39$
\testStop
\kluczStart
A
\kluczStop



\zadStart{Przykład z Wikieł P 4.3a moja wersja nr 834}


Obliczyć granicę funkcji $\lim\limits_{x\to\ 0}\frac{39 \cdot x}{tan(31 \cdot x)}$.
\zadStop
\rozwStart{Patryk Wirkus}{}
$$\lim\limits_{x\to\ 0}\frac{39 \cdot x}{tan(31 \cdot x)}=\lim\limits_{x\to\ 0}\frac{39 \cdot x \cdot cos(31 \cdot x)}{sin(31 \cdot x)}=\lim\limits_{x\to\ 0}\frac{39 \cdot cos(31 \cdot x)}{\frac{sin(31 \cdot x)}{x}}=\lim\limits_{x\to\ 0}\frac{39 \cdot cos(31 \cdot x)}{31 \cdot \frac{sin(31 \cdot x)}{31 \cdot x}} = \frac{39}{31}$$
\rozwStop
\odpStart
$\frac{39}{31}$
\odpStop
\testStart
A.$\frac{39}{31}$
B.$\infty$
C.$-\infty$
D.$0$
E.$-\frac{39}{31}$
F.$\frac{31}{39}$
G.$-\frac{31}{39}$
H.$31$
I.$39$
\testStop
\kluczStart
A
\kluczStop



\zadStart{Przykład z Wikieł P 4.3a moja wersja nr 835}


Obliczyć granicę funkcji $\lim\limits_{x\to\ 0}\frac{39 \cdot x}{tan(32 \cdot x)}$.
\zadStop
\rozwStart{Patryk Wirkus}{}
$$\lim\limits_{x\to\ 0}\frac{39 \cdot x}{tan(32 \cdot x)}=\lim\limits_{x\to\ 0}\frac{39 \cdot x \cdot cos(32 \cdot x)}{sin(32 \cdot x)}=\lim\limits_{x\to\ 0}\frac{39 \cdot cos(32 \cdot x)}{\frac{sin(32 \cdot x)}{x}}=\lim\limits_{x\to\ 0}\frac{39 \cdot cos(32 \cdot x)}{32 \cdot \frac{sin(32 \cdot x)}{32 \cdot x}} = \frac{39}{32}$$
\rozwStop
\odpStart
$\frac{39}{32}$
\odpStop
\testStart
A.$\frac{39}{32}$
B.$\infty$
C.$-\infty$
D.$0$
E.$-\frac{39}{32}$
F.$\frac{32}{39}$
G.$-\frac{32}{39}$
H.$32$
I.$39$
\testStop
\kluczStart
A
\kluczStop



\zadStart{Przykład z Wikieł P 4.3a moja wersja nr 836}


Obliczyć granicę funkcji $\lim\limits_{x\to\ 0}\frac{39 \cdot x}{tan(34 \cdot x)}$.
\zadStop
\rozwStart{Patryk Wirkus}{}
$$\lim\limits_{x\to\ 0}\frac{39 \cdot x}{tan(34 \cdot x)}=\lim\limits_{x\to\ 0}\frac{39 \cdot x \cdot cos(34 \cdot x)}{sin(34 \cdot x)}=\lim\limits_{x\to\ 0}\frac{39 \cdot cos(34 \cdot x)}{\frac{sin(34 \cdot x)}{x}}=\lim\limits_{x\to\ 0}\frac{39 \cdot cos(34 \cdot x)}{34 \cdot \frac{sin(34 \cdot x)}{34 \cdot x}} = \frac{39}{34}$$
\rozwStop
\odpStart
$\frac{39}{34}$
\odpStop
\testStart
A.$\frac{39}{34}$
B.$\infty$
C.$-\infty$
D.$0$
E.$-\frac{39}{34}$
F.$\frac{34}{39}$
G.$-\frac{34}{39}$
H.$34$
I.$39$
\testStop
\kluczStart
A
\kluczStop



\zadStart{Przykład z Wikieł P 4.3a moja wersja nr 837}


Obliczyć granicę funkcji $\lim\limits_{x\to\ 0}\frac{39 \cdot x}{tan(35 \cdot x)}$.
\zadStop
\rozwStart{Patryk Wirkus}{}
$$\lim\limits_{x\to\ 0}\frac{39 \cdot x}{tan(35 \cdot x)}=\lim\limits_{x\to\ 0}\frac{39 \cdot x \cdot cos(35 \cdot x)}{sin(35 \cdot x)}=\lim\limits_{x\to\ 0}\frac{39 \cdot cos(35 \cdot x)}{\frac{sin(35 \cdot x)}{x}}=\lim\limits_{x\to\ 0}\frac{39 \cdot cos(35 \cdot x)}{35 \cdot \frac{sin(35 \cdot x)}{35 \cdot x}} = \frac{39}{35}$$
\rozwStop
\odpStart
$\frac{39}{35}$
\odpStop
\testStart
A.$\frac{39}{35}$
B.$\infty$
C.$-\infty$
D.$0$
E.$-\frac{39}{35}$
F.$\frac{35}{39}$
G.$-\frac{35}{39}$
H.$35$
I.$39$
\testStop
\kluczStart
A
\kluczStop



\zadStart{Przykład z Wikieł P 4.3a moja wersja nr 838}


Obliczyć granicę funkcji $\lim\limits_{x\to\ 0}\frac{39 \cdot x}{tan(37 \cdot x)}$.
\zadStop
\rozwStart{Patryk Wirkus}{}
$$\lim\limits_{x\to\ 0}\frac{39 \cdot x}{tan(37 \cdot x)}=\lim\limits_{x\to\ 0}\frac{39 \cdot x \cdot cos(37 \cdot x)}{sin(37 \cdot x)}=\lim\limits_{x\to\ 0}\frac{39 \cdot cos(37 \cdot x)}{\frac{sin(37 \cdot x)}{x}}=\lim\limits_{x\to\ 0}\frac{39 \cdot cos(37 \cdot x)}{37 \cdot \frac{sin(37 \cdot x)}{37 \cdot x}} = \frac{39}{37}$$
\rozwStop
\odpStart
$\frac{39}{37}$
\odpStop
\testStart
A.$\frac{39}{37}$
B.$\infty$
C.$-\infty$
D.$0$
E.$-\frac{39}{37}$
F.$\frac{37}{39}$
G.$-\frac{37}{39}$
H.$37$
I.$39$
\testStop
\kluczStart
A
\kluczStop



\zadStart{Przykład z Wikieł P 4.3a moja wersja nr 839}


Obliczyć granicę funkcji $\lim\limits_{x\to\ 0}\frac{39 \cdot x}{tan(38 \cdot x)}$.
\zadStop
\rozwStart{Patryk Wirkus}{}
$$\lim\limits_{x\to\ 0}\frac{39 \cdot x}{tan(38 \cdot x)}=\lim\limits_{x\to\ 0}\frac{39 \cdot x \cdot cos(38 \cdot x)}{sin(38 \cdot x)}=\lim\limits_{x\to\ 0}\frac{39 \cdot cos(38 \cdot x)}{\frac{sin(38 \cdot x)}{x}}=\lim\limits_{x\to\ 0}\frac{39 \cdot cos(38 \cdot x)}{38 \cdot \frac{sin(38 \cdot x)}{38 \cdot x}} = \frac{39}{38}$$
\rozwStop
\odpStart
$\frac{39}{38}$
\odpStop
\testStart
A.$\frac{39}{38}$
B.$\infty$
C.$-\infty$
D.$0$
E.$-\frac{39}{38}$
F.$\frac{38}{39}$
G.$-\frac{38}{39}$
H.$38$
I.$39$
\testStop
\kluczStart
A
\kluczStop



\zadStart{Przykład z Wikieł P 4.3a moja wersja nr 840}


Obliczyć granicę funkcji $\lim\limits_{x\to\ 0}\frac{39 \cdot x}{tan(40 \cdot x)}$.
\zadStop
\rozwStart{Patryk Wirkus}{}
$$\lim\limits_{x\to\ 0}\frac{39 \cdot x}{tan(40 \cdot x)}=\lim\limits_{x\to\ 0}\frac{39 \cdot x \cdot cos(40 \cdot x)}{sin(40 \cdot x)}=\lim\limits_{x\to\ 0}\frac{39 \cdot cos(40 \cdot x)}{\frac{sin(40 \cdot x)}{x}}=\lim\limits_{x\to\ 0}\frac{39 \cdot cos(40 \cdot x)}{40 \cdot \frac{sin(40 \cdot x)}{40 \cdot x}} = \frac{39}{40}$$
\rozwStop
\odpStart
$\frac{39}{40}$
\odpStop
\testStart
A.$\frac{39}{40}$
B.$\infty$
C.$-\infty$
D.$0$
E.$-\frac{39}{40}$
F.$\frac{40}{39}$
G.$-\frac{40}{39}$
H.$40$
I.$39$
\testStop
\kluczStart
A
\kluczStop



\zadStart{Przykład z Wikieł P 4.3a moja wersja nr 841}


Obliczyć granicę funkcji $\lim\limits_{x\to\ 0}\frac{40 \cdot x}{tan(3 \cdot x)}$.
\zadStop
\rozwStart{Patryk Wirkus}{}
$$\lim\limits_{x\to\ 0}\frac{40 \cdot x}{tan(3 \cdot x)}=\lim\limits_{x\to\ 0}\frac{40 \cdot x \cdot cos(3 \cdot x)}{sin(3 \cdot x)}=\lim\limits_{x\to\ 0}\frac{40 \cdot cos(3 \cdot x)}{\frac{sin(3 \cdot x)}{x}}=\lim\limits_{x\to\ 0}\frac{40 \cdot cos(3 \cdot x)}{3 \cdot \frac{sin(3 \cdot x)}{3 \cdot x}} = \frac{40}{3}$$
\rozwStop
\odpStart
$\frac{40}{3}$
\odpStop
\testStart
A.$\frac{40}{3}$
B.$\infty$
C.$-\infty$
D.$0$
E.$-\frac{40}{3}$
F.$\frac{3}{40}$
G.$-\frac{3}{40}$
H.$3$
I.$40$
\testStop
\kluczStart
A
\kluczStop



\zadStart{Przykład z Wikieł P 4.3a moja wersja nr 842}


Obliczyć granicę funkcji $\lim\limits_{x\to\ 0}\frac{40 \cdot x}{tan(7 \cdot x)}$.
\zadStop
\rozwStart{Patryk Wirkus}{}
$$\lim\limits_{x\to\ 0}\frac{40 \cdot x}{tan(7 \cdot x)}=\lim\limits_{x\to\ 0}\frac{40 \cdot x \cdot cos(7 \cdot x)}{sin(7 \cdot x)}=\lim\limits_{x\to\ 0}\frac{40 \cdot cos(7 \cdot x)}{\frac{sin(7 \cdot x)}{x}}=\lim\limits_{x\to\ 0}\frac{40 \cdot cos(7 \cdot x)}{7 \cdot \frac{sin(7 \cdot x)}{7 \cdot x}} = \frac{40}{7}$$
\rozwStop
\odpStart
$\frac{40}{7}$
\odpStop
\testStart
A.$\frac{40}{7}$
B.$\infty$
C.$-\infty$
D.$0$
E.$-\frac{40}{7}$
F.$\frac{7}{40}$
G.$-\frac{7}{40}$
H.$7$
I.$40$
\testStop
\kluczStart
A
\kluczStop



\zadStart{Przykład z Wikieł P 4.3a moja wersja nr 843}


Obliczyć granicę funkcji $\lim\limits_{x\to\ 0}\frac{40 \cdot x}{tan(9 \cdot x)}$.
\zadStop
\rozwStart{Patryk Wirkus}{}
$$\lim\limits_{x\to\ 0}\frac{40 \cdot x}{tan(9 \cdot x)}=\lim\limits_{x\to\ 0}\frac{40 \cdot x \cdot cos(9 \cdot x)}{sin(9 \cdot x)}=\lim\limits_{x\to\ 0}\frac{40 \cdot cos(9 \cdot x)}{\frac{sin(9 \cdot x)}{x}}=\lim\limits_{x\to\ 0}\frac{40 \cdot cos(9 \cdot x)}{9 \cdot \frac{sin(9 \cdot x)}{9 \cdot x}} = \frac{40}{9}$$
\rozwStop
\odpStart
$\frac{40}{9}$
\odpStop
\testStart
A.$\frac{40}{9}$
B.$\infty$
C.$-\infty$
D.$0$
E.$-\frac{40}{9}$
F.$\frac{9}{40}$
G.$-\frac{9}{40}$
H.$9$
I.$40$
\testStop
\kluczStart
A
\kluczStop



\zadStart{Przykład z Wikieł P 4.3a moja wersja nr 844}


Obliczyć granicę funkcji $\lim\limits_{x\to\ 0}\frac{40 \cdot x}{tan(11 \cdot x)}$.
\zadStop
\rozwStart{Patryk Wirkus}{}
$$\lim\limits_{x\to\ 0}\frac{40 \cdot x}{tan(11 \cdot x)}=\lim\limits_{x\to\ 0}\frac{40 \cdot x \cdot cos(11 \cdot x)}{sin(11 \cdot x)}=\lim\limits_{x\to\ 0}\frac{40 \cdot cos(11 \cdot x)}{\frac{sin(11 \cdot x)}{x}}=\lim\limits_{x\to\ 0}\frac{40 \cdot cos(11 \cdot x)}{11 \cdot \frac{sin(11 \cdot x)}{11 \cdot x}} = \frac{40}{11}$$
\rozwStop
\odpStart
$\frac{40}{11}$
\odpStop
\testStart
A.$\frac{40}{11}$
B.$\infty$
C.$-\infty$
D.$0$
E.$-\frac{40}{11}$
F.$\frac{11}{40}$
G.$-\frac{11}{40}$
H.$11$
I.$40$
\testStop
\kluczStart
A
\kluczStop



\zadStart{Przykład z Wikieł P 4.3a moja wersja nr 845}


Obliczyć granicę funkcji $\lim\limits_{x\to\ 0}\frac{40 \cdot x}{tan(13 \cdot x)}$.
\zadStop
\rozwStart{Patryk Wirkus}{}
$$\lim\limits_{x\to\ 0}\frac{40 \cdot x}{tan(13 \cdot x)}=\lim\limits_{x\to\ 0}\frac{40 \cdot x \cdot cos(13 \cdot x)}{sin(13 \cdot x)}=\lim\limits_{x\to\ 0}\frac{40 \cdot cos(13 \cdot x)}{\frac{sin(13 \cdot x)}{x}}=\lim\limits_{x\to\ 0}\frac{40 \cdot cos(13 \cdot x)}{13 \cdot \frac{sin(13 \cdot x)}{13 \cdot x}} = \frac{40}{13}$$
\rozwStop
\odpStart
$\frac{40}{13}$
\odpStop
\testStart
A.$\frac{40}{13}$
B.$\infty$
C.$-\infty$
D.$0$
E.$-\frac{40}{13}$
F.$\frac{13}{40}$
G.$-\frac{13}{40}$
H.$13$
I.$40$
\testStop
\kluczStart
A
\kluczStop



\zadStart{Przykład z Wikieł P 4.3a moja wersja nr 846}


Obliczyć granicę funkcji $\lim\limits_{x\to\ 0}\frac{40 \cdot x}{tan(17 \cdot x)}$.
\zadStop
\rozwStart{Patryk Wirkus}{}
$$\lim\limits_{x\to\ 0}\frac{40 \cdot x}{tan(17 \cdot x)}=\lim\limits_{x\to\ 0}\frac{40 \cdot x \cdot cos(17 \cdot x)}{sin(17 \cdot x)}=\lim\limits_{x\to\ 0}\frac{40 \cdot cos(17 \cdot x)}{\frac{sin(17 \cdot x)}{x}}=\lim\limits_{x\to\ 0}\frac{40 \cdot cos(17 \cdot x)}{17 \cdot \frac{sin(17 \cdot x)}{17 \cdot x}} = \frac{40}{17}$$
\rozwStop
\odpStart
$\frac{40}{17}$
\odpStop
\testStart
A.$\frac{40}{17}$
B.$\infty$
C.$-\infty$
D.$0$
E.$-\frac{40}{17}$
F.$\frac{17}{40}$
G.$-\frac{17}{40}$
H.$17$
I.$40$
\testStop
\kluczStart
A
\kluczStop



\zadStart{Przykład z Wikieł P 4.3a moja wersja nr 847}


Obliczyć granicę funkcji $\lim\limits_{x\to\ 0}\frac{40 \cdot x}{tan(19 \cdot x)}$.
\zadStop
\rozwStart{Patryk Wirkus}{}
$$\lim\limits_{x\to\ 0}\frac{40 \cdot x}{tan(19 \cdot x)}=\lim\limits_{x\to\ 0}\frac{40 \cdot x \cdot cos(19 \cdot x)}{sin(19 \cdot x)}=\lim\limits_{x\to\ 0}\frac{40 \cdot cos(19 \cdot x)}{\frac{sin(19 \cdot x)}{x}}=\lim\limits_{x\to\ 0}\frac{40 \cdot cos(19 \cdot x)}{19 \cdot \frac{sin(19 \cdot x)}{19 \cdot x}} = \frac{40}{19}$$
\rozwStop
\odpStart
$\frac{40}{19}$
\odpStop
\testStart
A.$\frac{40}{19}$
B.$\infty$
C.$-\infty$
D.$0$
E.$-\frac{40}{19}$
F.$\frac{19}{40}$
G.$-\frac{19}{40}$
H.$19$
I.$40$
\testStop
\kluczStart
A
\kluczStop



\zadStart{Przykład z Wikieł P 4.3a moja wersja nr 848}


Obliczyć granicę funkcji $\lim\limits_{x\to\ 0}\frac{40 \cdot x}{tan(21 \cdot x)}$.
\zadStop
\rozwStart{Patryk Wirkus}{}
$$\lim\limits_{x\to\ 0}\frac{40 \cdot x}{tan(21 \cdot x)}=\lim\limits_{x\to\ 0}\frac{40 \cdot x \cdot cos(21 \cdot x)}{sin(21 \cdot x)}=\lim\limits_{x\to\ 0}\frac{40 \cdot cos(21 \cdot x)}{\frac{sin(21 \cdot x)}{x}}=\lim\limits_{x\to\ 0}\frac{40 \cdot cos(21 \cdot x)}{21 \cdot \frac{sin(21 \cdot x)}{21 \cdot x}} = \frac{40}{21}$$
\rozwStop
\odpStart
$\frac{40}{21}$
\odpStop
\testStart
A.$\frac{40}{21}$
B.$\infty$
C.$-\infty$
D.$0$
E.$-\frac{40}{21}$
F.$\frac{21}{40}$
G.$-\frac{21}{40}$
H.$21$
I.$40$
\testStop
\kluczStart
A
\kluczStop



\zadStart{Przykład z Wikieł P 4.3a moja wersja nr 849}


Obliczyć granicę funkcji $\lim\limits_{x\to\ 0}\frac{40 \cdot x}{tan(23 \cdot x)}$.
\zadStop
\rozwStart{Patryk Wirkus}{}
$$\lim\limits_{x\to\ 0}\frac{40 \cdot x}{tan(23 \cdot x)}=\lim\limits_{x\to\ 0}\frac{40 \cdot x \cdot cos(23 \cdot x)}{sin(23 \cdot x)}=\lim\limits_{x\to\ 0}\frac{40 \cdot cos(23 \cdot x)}{\frac{sin(23 \cdot x)}{x}}=\lim\limits_{x\to\ 0}\frac{40 \cdot cos(23 \cdot x)}{23 \cdot \frac{sin(23 \cdot x)}{23 \cdot x}} = \frac{40}{23}$$
\rozwStop
\odpStart
$\frac{40}{23}$
\odpStop
\testStart
A.$\frac{40}{23}$
B.$\infty$
C.$-\infty$
D.$0$
E.$-\frac{40}{23}$
F.$\frac{23}{40}$
G.$-\frac{23}{40}$
H.$23$
I.$40$
\testStop
\kluczStart
A
\kluczStop



\zadStart{Przykład z Wikieł P 4.3a moja wersja nr 850}


Obliczyć granicę funkcji $\lim\limits_{x\to\ 0}\frac{40 \cdot x}{tan(27 \cdot x)}$.
\zadStop
\rozwStart{Patryk Wirkus}{}
$$\lim\limits_{x\to\ 0}\frac{40 \cdot x}{tan(27 \cdot x)}=\lim\limits_{x\to\ 0}\frac{40 \cdot x \cdot cos(27 \cdot x)}{sin(27 \cdot x)}=\lim\limits_{x\to\ 0}\frac{40 \cdot cos(27 \cdot x)}{\frac{sin(27 \cdot x)}{x}}=\lim\limits_{x\to\ 0}\frac{40 \cdot cos(27 \cdot x)}{27 \cdot \frac{sin(27 \cdot x)}{27 \cdot x}} = \frac{40}{27}$$
\rozwStop
\odpStart
$\frac{40}{27}$
\odpStop
\testStart
A.$\frac{40}{27}$
B.$\infty$
C.$-\infty$
D.$0$
E.$-\frac{40}{27}$
F.$\frac{27}{40}$
G.$-\frac{27}{40}$
H.$27$
I.$40$
\testStop
\kluczStart
A
\kluczStop



\zadStart{Przykład z Wikieł P 4.3a moja wersja nr 851}


Obliczyć granicę funkcji $\lim\limits_{x\to\ 0}\frac{40 \cdot x}{tan(29 \cdot x)}$.
\zadStop
\rozwStart{Patryk Wirkus}{}
$$\lim\limits_{x\to\ 0}\frac{40 \cdot x}{tan(29 \cdot x)}=\lim\limits_{x\to\ 0}\frac{40 \cdot x \cdot cos(29 \cdot x)}{sin(29 \cdot x)}=\lim\limits_{x\to\ 0}\frac{40 \cdot cos(29 \cdot x)}{\frac{sin(29 \cdot x)}{x}}=\lim\limits_{x\to\ 0}\frac{40 \cdot cos(29 \cdot x)}{29 \cdot \frac{sin(29 \cdot x)}{29 \cdot x}} = \frac{40}{29}$$
\rozwStop
\odpStart
$\frac{40}{29}$
\odpStop
\testStart
A.$\frac{40}{29}$
B.$\infty$
C.$-\infty$
D.$0$
E.$-\frac{40}{29}$
F.$\frac{29}{40}$
G.$-\frac{29}{40}$
H.$29$
I.$40$
\testStop
\kluczStart
A
\kluczStop



\zadStart{Przykład z Wikieł P 4.3a moja wersja nr 852}


Obliczyć granicę funkcji $\lim\limits_{x\to\ 0}\frac{40 \cdot x}{tan(31 \cdot x)}$.
\zadStop
\rozwStart{Patryk Wirkus}{}
$$\lim\limits_{x\to\ 0}\frac{40 \cdot x}{tan(31 \cdot x)}=\lim\limits_{x\to\ 0}\frac{40 \cdot x \cdot cos(31 \cdot x)}{sin(31 \cdot x)}=\lim\limits_{x\to\ 0}\frac{40 \cdot cos(31 \cdot x)}{\frac{sin(31 \cdot x)}{x}}=\lim\limits_{x\to\ 0}\frac{40 \cdot cos(31 \cdot x)}{31 \cdot \frac{sin(31 \cdot x)}{31 \cdot x}} = \frac{40}{31}$$
\rozwStop
\odpStart
$\frac{40}{31}$
\odpStop
\testStart
A.$\frac{40}{31}$
B.$\infty$
C.$-\infty$
D.$0$
E.$-\frac{40}{31}$
F.$\frac{31}{40}$
G.$-\frac{31}{40}$
H.$31$
I.$40$
\testStop
\kluczStart
A
\kluczStop



\zadStart{Przykład z Wikieł P 4.3a moja wersja nr 853}


Obliczyć granicę funkcji $\lim\limits_{x\to\ 0}\frac{40 \cdot x}{tan(33 \cdot x)}$.
\zadStop
\rozwStart{Patryk Wirkus}{}
$$\lim\limits_{x\to\ 0}\frac{40 \cdot x}{tan(33 \cdot x)}=\lim\limits_{x\to\ 0}\frac{40 \cdot x \cdot cos(33 \cdot x)}{sin(33 \cdot x)}=\lim\limits_{x\to\ 0}\frac{40 \cdot cos(33 \cdot x)}{\frac{sin(33 \cdot x)}{x}}=\lim\limits_{x\to\ 0}\frac{40 \cdot cos(33 \cdot x)}{33 \cdot \frac{sin(33 \cdot x)}{33 \cdot x}} = \frac{40}{33}$$
\rozwStop
\odpStart
$\frac{40}{33}$
\odpStop
\testStart
A.$\frac{40}{33}$
B.$\infty$
C.$-\infty$
D.$0$
E.$-\frac{40}{33}$
F.$\frac{33}{40}$
G.$-\frac{33}{40}$
H.$33$
I.$40$
\testStop
\kluczStart
A
\kluczStop



\zadStart{Przykład z Wikieł P 4.3a moja wersja nr 854}


Obliczyć granicę funkcji $\lim\limits_{x\to\ 0}\frac{40 \cdot x}{tan(37 \cdot x)}$.
\zadStop
\rozwStart{Patryk Wirkus}{}
$$\lim\limits_{x\to\ 0}\frac{40 \cdot x}{tan(37 \cdot x)}=\lim\limits_{x\to\ 0}\frac{40 \cdot x \cdot cos(37 \cdot x)}{sin(37 \cdot x)}=\lim\limits_{x\to\ 0}\frac{40 \cdot cos(37 \cdot x)}{\frac{sin(37 \cdot x)}{x}}=\lim\limits_{x\to\ 0}\frac{40 \cdot cos(37 \cdot x)}{37 \cdot \frac{sin(37 \cdot x)}{37 \cdot x}} = \frac{40}{37}$$
\rozwStop
\odpStart
$\frac{40}{37}$
\odpStop
\testStart
A.$\frac{40}{37}$
B.$\infty$
C.$-\infty$
D.$0$
E.$-\frac{40}{37}$
F.$\frac{37}{40}$
G.$-\frac{37}{40}$
H.$37$
I.$40$
\testStop
\kluczStart
A
\kluczStop



\zadStart{Przykład z Wikieł P 4.3a moja wersja nr 855}


Obliczyć granicę funkcji $\lim\limits_{x\to\ 0}\frac{40 \cdot x}{tan(39 \cdot x)}$.
\zadStop
\rozwStart{Patryk Wirkus}{}
$$\lim\limits_{x\to\ 0}\frac{40 \cdot x}{tan(39 \cdot x)}=\lim\limits_{x\to\ 0}\frac{40 \cdot x \cdot cos(39 \cdot x)}{sin(39 \cdot x)}=\lim\limits_{x\to\ 0}\frac{40 \cdot cos(39 \cdot x)}{\frac{sin(39 \cdot x)}{x}}=\lim\limits_{x\to\ 0}\frac{40 \cdot cos(39 \cdot x)}{39 \cdot \frac{sin(39 \cdot x)}{39 \cdot x}} = \frac{40}{39}$$
\rozwStop
\odpStart
$\frac{40}{39}$
\odpStop
\testStart
A.$\frac{40}{39}$
B.$\infty$
C.$-\infty$
D.$0$
E.$-\frac{40}{39}$
F.$\frac{39}{40}$
G.$-\frac{39}{40}$
H.$39$
I.$40$
\testStop
\kluczStart
A
\kluczStop





\end{document}
