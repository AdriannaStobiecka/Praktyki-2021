\documentclass[12pt, a4paper]{article}
\usepackage[utf8]{inputenc}
\usepackage{polski}

\usepackage{amsthm}  %pakiet do tworzenia twierdzeń itp.
\usepackage{amsmath} %pakiet do niektórych symboli matematycznych
\usepackage{amssymb} %pakiet do symboli mat., np. \nsubseteq
\usepackage{amsfonts}
\usepackage{graphicx} %obsługa plików graficznych z rozszerzeniem png, jpg
\theoremstyle{definition} %styl dla definicji
\newtheorem{zad}{} 
\title{Multizestaw zadań}
\author{Robert Fidytek}
%\date{\today}
\date{}
\newcounter{liczniksekcji}
\newcommand{\kategoria}[1]{\section{#1}} %olreślamy nazwę kateforii zadań
\newcommand{\zadStart}[1]{\begin{zad}#1\newline} %oznaczenie początku zadania
\newcommand{\zadStop}{\end{zad}}   %oznaczenie końca zadania
%Makra opcjonarne (nie muszą występować):
\newcommand{\rozwStart}[2]{\noindent \textbf{Rozwiązanie (autor #1 , recenzent #2): }\newline} %oznaczenie początku rozwiązania, opcjonarnie można wprowadzić informację o autorze rozwiązania zadania i recenzencie poprawności wykonania rozwiązania zadania
\newcommand{\rozwStop}{\newline}                                            %oznaczenie końca rozwiązania
\newcommand{\odpStart}{\noindent \textbf{Odpowiedź:}\newline}    %oznaczenie początku odpowiedzi końcowej (wypisanie wyniku)
\newcommand{\odpStop}{\newline}                                             %oznaczenie końca odpowiedzi końcowej (wypisanie wyniku)
\newcommand{\testStart}{\noindent \textbf{Test:}\newline} %ewentualne możliwe opcje odpowiedzi testowej: A. ? B. ? C. ? D. ? itd.
\newcommand{\testStop}{\newline} %koniec wprowadzania odpowiedzi testowych
\newcommand{\kluczStart}{\noindent \textbf{Test poprawna odpowiedź:}\newline} %klucz, poprawna odpowiedź pytania testowego (jedna literka): A lub B lub C lub D itd.
\newcommand{\kluczStop}{\newline} %koniec poprawnej odpowiedzi pytania testowego 
\newcommand{\wstawGrafike}[2]{\begin{figure}[h] \includegraphics[scale=#2] {#1} \end{figure}} %gdyby była potrzeba wstawienia obrazka, parametry: nazwa pliku, skala (jak nie wiesz co wpisać, to wpisz 1)

\begin{document}
\maketitle


\kategoria{Wikieł/Z1.106b}
\zadStart{Zadanie z Wikieł Z 1.106 b) moja wersja nr [nrWersji]}
%[y]:[2,49,64,81,100,121,144,1,2,3,4,5,6,7]
%[x]:[4,9,16,25,36,49,64,81,100,121,144]
%[z]:[4,9,16,25,36,49,64,81,100,121,144]
%[a]=random.randint(2,20)
%[b]=random.randint(2,20)
%[c]=random.randint(2,20)
%[d]=random.randint(2,20)
%[e]=[a]-[d]
%[f]=[b]*[c]
%[g]=[c]+[b]
%[h]=[a]+[d]
%[k]=[c]-[b]
%[m]=[f]*[e]
%[n]=[f]*[h]
%math.gcd(2,[h])==1 and  math.gcd(2,[e])==1 and [k]>1 and [e]>1 and math.gcd([n],[k])==1 and  math.gcd([m],[g])==1
Rozwiązać równania.\\
 $cos\left([a]x+\frac{\pi}{[b]}\right)-cos\left(\frac{\pi}{[c]}-[d]x\right)=0$
\zadStop
\rozwStart{Katarzyna Filipowicz}{}
$$
cos\left([a]x+\frac{\pi}{[b]}\right)-cos\left(\frac{\pi}{[c]}-[d]x\right)=0
$$
$$
-2sin\left(\frac{[a]x+\frac{\pi}{[b]}+\frac{\pi}{[c]}-[d]x}{2}\right)sin\left(\frac{[a]x+\frac{\pi}{[b]}-\frac{\pi}{[c]}+[d]x}{2}\right)=0
$$ $$
sin\left(\frac{[e]x+\frac{[g]\pi}{[f]}}{2}\right)sin\left(\frac{[h]x+\frac{[k]\pi}{[f]}}{2}\right)=0
$$
1. 
$$sin\left(\frac{[e]x+\frac{[g]\pi}{[f]}}{2}\right)=0$$
 $$
\frac{[e]x+\frac{[g]\pi}{[f]}}{2}=0+k\pi 
 $$ $$
 [e]x=-\frac{[g]\pi}{[f]}+2k\pi 
 $$ $$
x=-\frac{[g]\pi}{[m]}+\frac{2k\pi}{[e]} 
$$ 
2. 
$$sin\left(\frac{[h]x+\frac{[k]\pi}{[f]}}{2}\right)=0$$
$$
\frac{[h]x+\frac{[k]\pi}{[f]}}{2}=0+k\pi 
 $$ $$
 [h]x=-\frac{[k]\pi}{[f]}+2k
 $$ $$
x=-\frac{[k]\pi}{[n]}+\frac{2k\pi}{[h]} 
$$
\rozwStop
\odpStart
$x=-\frac{[g]\pi}{[m]}+\frac{2k\pi}{[e]}   \vee x=-\frac{[k]\pi}{[n]}+\frac{2k\pi}{[h]}$
\odpStop
\testStart
A.$x=-\frac{[g]\pi}{[m]}+\frac{2k\pi}{[e]}   \vee x=-\frac{[k]\pi}{[n]}+\frac{2k\pi}{[h]}$\\
B.$x=-\frac{[g]\pi}{[n]}+\frac{2k\pi}{[e]}   \vee x=-\frac{[k]\pi}{[m]}+\frac{2k\pi}{[h]}$\\
C.$x=-\frac{[g]\pi}{[m]}+\frac{2k\pi}{[e]}$\\
D.$x=-\frac{[k]\pi}{[n]}+\frac{2k\pi}{[h]}$\\
E. $x=0+2k\pi$\\
F.$x=\frac{\pi}{2}+\frac{2k\pi}{[e]}$\\
G.$x=-\frac{[g]\pi}{[m]}+\frac{2k\pi}{[e]}   \vee x=-\frac{[k]\pi}{[n]}+\frac{4k\pi}{[h]}$\\
H.$x=-\frac{[g]\pi}{[m]}+\frac{2k\pi}{[e]}   \vee x=-\frac{[k]\pi}{[n]}+2k\pi$\\
I.$x=-\frac{[g]\pi}{[m]}+k\pi  \vee x=-\frac{[k]\pi}{[n]}+\frac{2k\pi}{[h]}$
\testStop
\kluczStart
A
\kluczStop



\end{document}