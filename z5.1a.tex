\documentclass[12pt, a4paper]{article}
\usepackage[utf8]{inputenc}
\usepackage{polski}

\usepackage{amsthm}  %pakiet do tworzenia twierdzeń itp.
\usepackage{amsmath} %pakiet do niektórych symboli matematycznych
\usepackage{amssymb} %pakiet do symboli mat., np. \nsubseteq
\usepackage{amsfonts}
\usepackage{graphicx} %obsługa plików graficznych z rozszerzeniem png, jpg
\theoremstyle{definition} %styl dla definicji
\newtheorem{zad}{} 
\title{Multizestaw zadań}
\author{Robert Fidytek}
%\date{\today}
\date{}
\newcounter{liczniksekcji}
\newcommand{\kategoria}[1]{\section{#1}} %olreślamy nazwę kateforii zadań
\newcommand{\zadStart}[1]{\begin{zad}#1\newline} %oznaczenie początku zadania
\newcommand{\zadStop}{\end{zad}}   %oznaczenie końca zadania
%Makra opcjonarne (nie muszą występować):
\newcommand{\rozwStart}[2]{\noindent \textbf{Rozwiązanie (autor #1 , recenzent #2): }\newline} %oznaczenie początku rozwiązania, opcjonarnie można wprowadzić informację o autorze rozwiązania zadania i recenzencie poprawności wykonania rozwiązania zadania
\newcommand{\rozwStop}{\newline}                                            %oznaczenie końca rozwiązania
\newcommand{\odpStart}{\noindent \textbf{Odpowiedź:}\newline}    %oznaczenie początku odpowiedzi końcowej (wypisanie wyniku)
\newcommand{\odpStop}{\newline}                                             %oznaczenie końca odpowiedzi końcowej (wypisanie wyniku)
\newcommand{\testStart}{\noindent \textbf{Test:}\newline} %ewentualne możliwe opcje odpowiedzi testowej: A. ? B. ? C. ? D. ? itd.
\newcommand{\testStop}{\newline} %koniec wprowadzania odpowiedzi testowych
\newcommand{\kluczStart}{\noindent \textbf{Test poprawna odpowiedź:}\newline} %klucz, poprawna odpowiedź pytania testowego (jedna literka): A lub B lub C lub D itd.
\newcommand{\kluczStop}{\newline} %koniec poprawnej odpowiedzi pytania testowego 
\newcommand{\wstawGrafike}[2]{\begin{figure}[h] \includegraphics[scale=#2] {#1} \end{figure}} %gdyby była potrzeba wstawienia obrazka, parametry: nazwa pliku, skala (jak nie wiesz co wpisać, to wpisz 1)

\begin{document}
\maketitle


\kategoria{Wikieł/Z5.1a}
\zadStart{Zadanie z Wikieł Z 5.1a) moja wersja nr [nrWersji]}
%[x]:[2,3,4,5,6,7,8,9,10,11,12,15,17]
%[y]:[2,3,4,5,6,7,8,9,10,11,12,15,17]
%[a]=random.randint(2,100)
%[b]=random.randint(2,100)
%[c]=random.randint(2,100)
%[m]=2*[a]
Na podstawie definicji obliczyć pochodne poniższych funkcji.\\
 $f(x)=[a]x^2+[b]x-[c]$
\zadStop
\rozwStart{Katarzyna Filipowicz}{}
Niech $h=\Delta x$
$$
f'(x)=lim_{h\rightarrow 0} \frac{([a](x+h)^2+[b](x+h)-[c])-([a]x^2+[b]x-[c])}{h}=
$$ $$
=lim_{h\rightarrow 0}\frac{[a]x^2+[m]xh+[a]h^2+[b]x+[b]h-[c]-[a]x^2-[b]x+[c]}{h}=
$$ $$
=lim_{h\rightarrow 0}\frac{[m]xh+[a]h^2+[b]h}{h}=lim_{h\rightarrow 0}[m]x+[a]h+[b]=[m]x+[b]
$$
\rozwStop
\odpStart
$f'(x)=[m]x+[b]$
\odpStop
\testStart
A.$f'(x)=[m]x+[b]$
B.$f'(x)=0$
C.$f'(x)=[m]x$
D.$f'(x)=[b]x+[x]$
E.$f'(x)=[a]x^2+[b]$
F.$f'(x)=2x+[c]$
G.$f'(x)=[a]x+[b]$
H.$f'(x)=[c]$
I.$f'(x)=[x]$
\testStop
\kluczStart
A
\kluczStop



\end{document}