\documentclass[12pt, a4paper]{article}
\usepackage[utf8]{inputenc}
\usepackage{polski}

\usepackage{amsthm}  %pakiet do tworzenia twierdzeń itp.
\usepackage{amsmath} %pakiet do niektórych symboli matematycznych
\usepackage{amssymb} %pakiet do symboli mat., np. \nsubseteq
\usepackage{amsfonts}
\usepackage{graphicx} %obsługa plików graficznych z rozszerzeniem png, jpg
\theoremstyle{definition} %styl dla definicji
\newtheorem{zad}{} 
\title{Multizestaw zadań}
\author{Robert Fidytek}
%\date{\today}
\date{}
\newcounter{liczniksekcji}
\newcommand{\kategoria}[1]{\section{#1}} %olreślamy nazwę kateforii zadań
\newcommand{\zadStart}[1]{\begin{zad}#1\newline} %oznaczenie początku zadania
\newcommand{\zadStop}{\end{zad}}   %oznaczenie końca zadania
%Makra opcjonarne (nie muszą występować):
\newcommand{\rozwStart}[2]{\noindent \textbf{Rozwiązanie (autor #1 , recenzent #2): }\newline} %oznaczenie początku rozwiązania, opcjonarnie można wprowadzić informację o autorze rozwiązania zadania i recenzencie poprawności wykonania rozwiązania zadania
\newcommand{\rozwStop}{\newline}                                            %oznaczenie końca rozwiązania
\newcommand{\odpStart}{\noindent \textbf{Odpowiedź:}\newline}    %oznaczenie początku odpowiedzi końcowej (wypisanie wyniku)
\newcommand{\odpStop}{\newline}                                             %oznaczenie końca odpowiedzi końcowej (wypisanie wyniku)
\newcommand{\testStart}{\noindent \textbf{Test:}\newline} %ewentualne możliwe opcje odpowiedzi testowej: A. ? B. ? C. ? D. ? itd.
\newcommand{\testStop}{\newline} %koniec wprowadzania odpowiedzi testowych
\newcommand{\kluczStart}{\noindent \textbf{Test poprawna odpowiedź:}\newline} %klucz, poprawna odpowiedź pytania testowego (jedna literka): A lub B lub C lub D itd.
\newcommand{\kluczStop}{\newline} %koniec poprawnej odpowiedzi pytania testowego 
\newcommand{\wstawGrafike}[2]{\begin{figure}[h] \includegraphics[scale=#2] {#1} \end{figure}} %gdyby była potrzeba wstawienia obrazka, parametry: nazwa pliku, skala (jak nie wiesz co wpisać, to wpisz 1)

\begin{document}
\maketitle


\kategoria{Wikieł/Z1.70c}
\zadStart{Zadanie z Wikieł Z 1.70 c) moja wersja nr [nrWersji]}
%[a]:[2,3,4,5]
%[b]:[2,3,4,5]
%[c]:[2,3,4,5]
%[d]:[2,3,4,5]
%[e]:[2,3,4,5]
%[f]:[2,3,4,5]
%[a]=random.randint(1,6)
%[b]=random.randint(2,6)
%[c]=random.randint(2,6)
%[d]=random.randint(1,6)
%[e]=random.randint(2,6)
%[f]=random.randint(1,6)
%[cd]=int(-[d]/[c])
%[ef]=int([f]/[e])
%[cdp]=int([d]/[c])
%[ab]=int([a]/[b])
%math.gcd([d],[c])==[c] and math.gcd([e],[f])==[e] and math.gcd([a],[b])==[b] and [ab]<[ef]
Rozwiązać nierówność: $\frac{[a]-[b]x}{([c]x+[d])([e]x-[f])}\leq0$
\zadStop
\rozwStart{Pascal Nawrocki}{}
Zaczynamy od wyznaczenia dziedziny: $x\in\mathbb{R}\symbol{92}\{[cd],[ef]\}$
Mnożymy obustronnie przez mianownik podniesiony do kwadratu:
$$([a]-[b]x)([c]x+[d])([e]x-[f])\leq0$$
$$([ab]-x)(x+[cdp])(x-[ef])\leq0$$
Korzystamy z metody rysowania wykresu wielomianu i na jej podstawie wypisujemy przedziały pamiętając o wyznaczonej wcześniej dziedzinie:
$$x\in(-[cdp],[ab]]\cup([ef],\infty)$$
\odpStart
$x\in(-[cdp],[ab]]\cup([ef],\infty)$
\odpStop
\testStart
A.$x\in(-[cdp],[ab]]\cup([ef],\infty)$
B.$x\in([b],\infty)$
C.$x\in\emptyset$
D.$x\in(-\infty,-[a])$
\testStop
\kluczStart
A
\kluczStop
\end{document}