\documentclass[12pt, a4paper]{article}
\usepackage[utf8]{inputenc}
\usepackage{polski}

\usepackage{amsthm}  %pakiet do tworzenia twierdzeń itp.
\usepackage{amsmath} %pakiet do niektórych symboli matematycznych
\usepackage{amssymb} %pakiet do symboli mat., np. \nsubseteq
\usepackage{amsfonts}
\usepackage{graphicx} %obsługa plików graficznych z rozszerzeniem png, jpg
\theoremstyle{definition} %styl dla definicji
\newtheorem{zad}{} 
\title{Multizestaw zadań}
\author{Robert Fidytek}
%\date{\today}
\date{}
\newcounter{liczniksekcji}
\newcommand{\kategoria}[1]{\section{#1}} %olreślamy nazwę kateforii zadań
\newcommand{\zadStart}[1]{\begin{zad}#1\newline} %oznaczenie początku zadania
\newcommand{\zadStop}{\end{zad}}   %oznaczenie końca zadania
%Makra opcjonarne (nie muszą występować):
\newcommand{\rozwStart}[2]{\noindent \textbf{Rozwiązanie (autor #1 , recenzent #2): }\newline} %oznaczenie początku rozwiązania, opcjonarnie można wprowadzić informację o autorze rozwiązania zadania i recenzencie poprawności wykonania rozwiązania zadania
\newcommand{\rozwStop}{\newline}                                            %oznaczenie końca rozwiązania
\newcommand{\odpStart}{\noindent \textbf{Odpowiedź:}\newline}    %oznaczenie początku odpowiedzi końcowej (wypisanie wyniku)
\newcommand{\odpStop}{\newline}                                             %oznaczenie końca odpowiedzi końcowej (wypisanie wyniku)
\newcommand{\testStart}{\noindent \textbf{Test:}\newline} %ewentualne możliwe opcje odpowiedzi testowej: A. ? B. ? C. ? D. ? itd.
\newcommand{\testStop}{\newline} %koniec wprowadzania odpowiedzi testowych
\newcommand{\kluczStart}{\noindent \textbf{Test poprawna odpowiedź:}\newline} %klucz, poprawna odpowiedź pytania testowego (jedna literka): A lub B lub C lub D itd.
\newcommand{\kluczStop}{\newline} %koniec poprawnej odpowiedzi pytania testowego 
\newcommand{\wstawGrafike}[2]{\begin{figure}[h] \includegraphics[scale=#2] {#1} \end{figure}} %gdyby była potrzeba wstawienia obrazka, parametry: nazwa pliku, skala (jak nie wiesz co wpisać, to wpisz 1)

\begin{document}
\maketitle


\kategoria{Wikieł/Z1.76c}
\zadStart{Zadanie z Wikieł Z 1.76 c) moja wersja nr [nrWersji]}
%[a]:[2,3,4,5,6,7,8,9]
%[b]:[2,4,6,8]
%[c]:[2,3,4,5,6,7,8,9]
%[pd]:[2,4,6,8]
%[e]:[3,5,7,9]
%[d]=[pd]**2
%[a]<[pd] and -[a]<[pd] and math.gcd([e],[pd])==1
Wyznaczyć dziedzinę naturalną funkcji określonej podanym wzorem.
$$f(x)=\sqrt{x(x+[a])^{[b]}}+(x^2+[c])([d]-x^2)^{\frac{[e]}{[pd]}}$$
\zadStop
\rozwStart{Adrianna Stobiecka}{}
Funkcja $f_1(x)=x(x+[a])^{[b]}$ znajdująca się pod pierwiastkiem musi spełniać nierówność $f_1(x)\geq0$.
Funkcja potęgowa $f_2(x)=([d]-x^2)^{\frac{[e]}{[pd]}}$ jest określona na zbiorze $\mathbb{R}_+\cup\{0\}$. Zatem dziedziną funkcji $f$ jest zbiór
$$D_{f}=\{x\in\mathbb{R}:x(x+[a])^{[b]}\geq0 \land [d]-x^2\geq0\}.$$
Rozwiążmy najpierw pierwszą nierówność. Mamy:
$$x(x+[a])^{[b]}\geq0$$
Zauważamy, że pierwiastkami funkcji po lewej stronie nierówności są $x_1=0$ oraz $x_2=-[a]$, gdzie $x_2$ jest pierwiastkiem stopnia parzystego. Zatem nierówność jest spełniona dla $x\in\{-[a]\}\cup[0,\infty)$.
\\Przejdziemy teraz do rozwiązania drugiej nierówności. Mamy:
$$[d]-x^2\geq0\qquad\Leftrightarrow\qquad x^2\leq[d]\Leftrightarrow\qquad x\leq[pd]~~\land~~ x\geq-[pd]$$
Nierówność ta jest spełniona dla $x\in[-[pd],[pd]]$.
Ostatecznie otrzymujemy, że
$$D_{f}=\{-[a]\}\cup[0,[pd]].$$
\rozwStop
\odpStart
$D_{f}=\{-[a]\}\cup[0,[pd]]$
\odpStop
\testStart
A.$D_{f}=\{-[a]\}\cup[0,\infty)$
B.$D_{f}=[-[pd],[pd]]$
C.$D_{f}=\mathbb{R}\setminus\{0\}$
D.$D_{f}=[0,\infty)$
E.$D_{f}=[0,[pd]]$
F.$D_{f}=\mathbb{R}$
G.$D_{f}=(-\infty,0]$
H.$D_{f}=\{-[a]\}\cup[0,[pd]]$
I.$D_{f}=[-[pd],0]$
\testStop
\kluczStart
H
\kluczStop



\end{document}
