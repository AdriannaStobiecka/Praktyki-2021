\documentclass[12pt, a4paper]{article}
\usepackage[utf8]{inputenc}
\usepackage{polski}

\usepackage{amsthm}  %pakiet do tworzenia twierdzeń itp.
\usepackage{amsmath} %pakiet do niektórych symboli matematycznych
\usepackage{amssymb} %pakiet do symboli mat., np. \nsubseteq
\usepackage{amsfonts}
\usepackage{graphicx} %obsługa plików graficznych z rozszerzeniem png, jpg
\theoremstyle{definition} %styl dla definicji
\newtheorem{zad}{} 
\title{Multizestaw zadań}
\author{Robert Fidytek}
%\date{\today}
\date{}
\newcounter{liczniksekcji}
\newcommand{\kategoria}[1]{\section{#1}} %olreślamy nazwę kateforii zadań
\newcommand{\zadStart}[1]{\begin{zad}#1\newline} %oznaczenie początku zadania
\newcommand{\zadStop}{\end{zad}}   %oznaczenie końca zadania
%Makra opcjonarne (nie muszą występować):
\newcommand{\rozwStart}[2]{\noindent \textbf{Rozwiązanie (autor #1 , recenzent #2): }\newline} %oznaczenie początku rozwiązania, opcjonarnie można wprowadzić informację o autorze rozwiązania zadania i recenzencie poprawności wykonania rozwiązania zadania
\newcommand{\rozwStop}{\newline}                                            %oznaczenie końca rozwiązania
\newcommand{\odpStart}{\noindent \textbf{Odpowiedź:}\newline}    %oznaczenie początku odpowiedzi końcowej (wypisanie wyniku)
\newcommand{\odpStop}{\newline}                                             %oznaczenie końca odpowiedzi końcowej (wypisanie wyniku)
\newcommand{\testStart}{\noindent \textbf{Test:}\newline} %ewentualne możliwe opcje odpowiedzi testowej: A. ? B. ? C. ? D. ? itd.
\newcommand{\testStop}{\newline} %koniec wprowadzania odpowiedzi testowych
\newcommand{\kluczStart}{\noindent \textbf{Test poprawna odpowiedź:}\newline} %klucz, poprawna odpowiedź pytania testowego (jedna literka): A lub B lub C lub D itd.
\newcommand{\kluczStop}{\newline} %koniec poprawnej odpowiedzi pytania testowego 
\newcommand{\wstawGrafike}[2]{\begin{figure}[h] \includegraphics[scale=#2] {#1} \end{figure}} %gdyby była potrzeba wstawienia obrazka, parametry: nazwa pliku, skala (jak nie wiesz co wpisać, to wpisz 1)

\begin{document}
\maketitle


\kategoria{Wikieł/Z4.15e}
\zadStart{Zadanie z Wikieł Z 4.15 e) moja wersja nr [nrWersji]}
%[a]:[6,7,8,9,10,11,12,13,14,15,16,17,18,19,20,21,22,23,24,25,26,27,28,29,30,31,32,33,34,35,36,37,38,39,40]
%[x]:[1,2,3,4,5,6,7,8,9,10,11,12,13,14,15,16,17,18,19,20,21,22,23,24,25,26,27,28,29,30,31,32,33,34,35,36,37,38,39,40,41,42]
%[aa]=[a]*[a]
%[b]=[a]+[aa]
%[xa]=[a]+[x]
%[baa]=[b]-[aa]
%[baa]=[a]
%[bb]=[b]-[x]
%[bx]=math.sqrt([bb]) 
%[bx].is_integer()==True 
%[bxa]=[bx]+[a]
%[q]=[xa]*[bxa]
%[cbxa]=int([bxa])
%[cq]=int([q])
Zbadać, czy istnieją poniższe granice. Jeśli tak, to obliczyć je $$\lim_{x\rightarrow [x]}\frac{|x^2-[aa]|}{\sqrt{[b]-x}-[a]}.$$
\zadStop
\rozwStart{Aleksandra Pasińska}{}
$$\lim_{x\rightarrow [x]^-}\frac{|x^2-[aa]|}{\sqrt{[b]-x}-[a]}\cdot \frac{\sqrt{[b]-x}+[a]}{\sqrt{[b]-x}+[a]}=\lim_{x\rightarrow [x]^-}\frac{|x^2-[aa]|(\sqrt{[b]-x}+[a])}{(\sqrt{[b]-x})^2-[aa]}=$$
$$=\lim_{x\rightarrow [x]^-}\frac{|(x-[a])(x+[a])|(\sqrt{[b]-x}+[a])}{(\sqrt{[b]-x})^2-[aa]}=\lim_{x\rightarrow [x]^-}\frac{-(x-[a])(x+[a])(\sqrt{[b]-x}+[a])}{-(x-[baa])}=$$ 
$$=\lim_{x\rightarrow [x]^-}(x+[a])(\sqrt{[b]-x}+[a])=[xa]\cdot [cbxa]=[cq]$$
$$\lim_{x\rightarrow [x]^+}\frac{|x^2-[aa]|}{\sqrt{[b]-x}-[a]}\cdot \frac{\sqrt{[b]-x}+[a]}{\sqrt{[b]-x}+[a]}=\lim_{x\rightarrow [x]^+}\frac{|x^2-[aa]|(\sqrt{[b]-x}+[a])}{(\sqrt{[b]-x})^2-[aa]}=$$
$$=\lim_{x\rightarrow [x]^+}\frac{|(x-[a])(x+[a])|(\sqrt{[b]-x}+[a])}{(\sqrt{[b]-x})^2-[aa]}=\lim_{x\rightarrow [x]^+}\frac{(x-[a])(x+[a])(\sqrt{[b]-x}+[a])}{-(x-[baa])}=$$ 
$$=\lim_{x\rightarrow [x]^+}-(x+[a])(\sqrt{[b]-x}+[a])=-[xa]\cdot [cbxa]=-[cq]$$
\rozwStop
\odpStart
brak granicy
\odpStop
\testStart
A.brak granicy
B.$0$
C.$\infty$
D.$\frac{3}{2}$
E.$-2$
F.$\frac{1}{4}$
G.$9$
H.$2$
I.$-\infty$
\testStop
\kluczStart
A
\kluczStop



\end{document}