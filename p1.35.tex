\documentclass[12pt, a4paper]{article}
\usepackage[utf8]{inputenc}
\usepackage{polski}

\usepackage{amsthm}  %pakiet do tworzenia twierdzeń itp.
\usepackage{amsmath} %pakiet do niektórych symboli matematycznych
\usepackage{amssymb} %pakiet do symboli mat., np. \nsubseteq
\usepackage{amsfonts}
\usepackage{graphicx} %obsługa plików graficznych z rozszerzeniem png, jpg
\theoremstyle{definition} %styl dla definicji
\newtheorem{zad}{} 
\title{Multizestaw zadań}
\author{Robert Fidytek}
%\date{\today}
\date{}
\newcounter{liczniksekcji}
\newcommand{\kategoria}[1]{\section{#1}} %olreślamy nazwę kateforii zadań
\newcommand{\zadStart}[1]{\begin{zad}#1\newline} %oznaczenie początku zadania
\newcommand{\zadStop}{\end{zad}}   %oznaczenie końca zadania
%Makra opcjonarne (nie muszą występować):
\newcommand{\rozwStart}[2]{\noindent \textbf{Rozwiązanie (autor #1 , recenzent #2): }\newline} %oznaczenie początku rozwiązania, opcjonarnie można wprowadzić informację o autorze rozwiązania zadania i recenzencie poprawności wykonania rozwiązania zadania
\newcommand{\rozwStop}{\newline}                                            %oznaczenie końca rozwiązania
\newcommand{\odpStart}{\noindent \textbf{Odpowiedź:}\newline}    %oznaczenie początku odpowiedzi końcowej (wypisanie wyniku)
\newcommand{\odpStop}{\newline}                                             %oznaczenie końca odpowiedzi końcowej (wypisanie wyniku)
\newcommand{\testStart}{\noindent \textbf{Test:}\newline} %ewentualne możliwe opcje odpowiedzi testowej: A. ? B. ? C. ? D. ? itd.
\newcommand{\testStop}{\newline} %koniec wprowadzania odpowiedzi testowych
\newcommand{\kluczStart}{\noindent \textbf{Test poprawna odpowiedź:}\newline} %klucz, poprawna odpowiedź pytania testowego (jedna literka): A lub B lub C lub D itd.
\newcommand{\kluczStop}{\newline} %koniec poprawnej odpowiedzi pytania testowego 
\newcommand{\wstawGrafike}[2]{\begin{figure}[h] \includegraphics[scale=#2] {#1} \end{figure}} %gdyby była potrzeba wstawienia obrazka, parametry: nazwa pliku, skala (jak nie wiesz co wpisać, to wpisz 1)

\begin{document}
\maketitle


\kategoria{Wikieł/P1.35}
\zadStart{Zadanie z Wikieł P 1.35) moja wersja nr [nrWersji]}
%[p1]:[1,4,9,16,25,36,49,64,91,100]
%[p2]:[2,3,4,5,6,7,8,9,10,11,12]
%[p3]:[2,3,4,5,6,7,8,9,10,11,12]
%[a]=random.randint(1,20)
%[e]=random.randint(2,10)
%[c]=random.randint(1,20)
%[d]=random.randint(1,20)
%[b]=random.randint(1,20)
%[f]=random.randint(1,10)
%[p1p2m]=[p1]-[p2]
%[p1p3m]=[p1]-[p3]
%[ab]=[a]*[b]
%[cd]=[c]*[d]
%[ac]=[a]+[c]
%[abcd]=-[ab]+[cd]
%[4abcd]=4*[abcd]
%[ackw]=[ac]*[ac]
%[delta]=[ackw]-[4abcd]
%[pierw]=(pow([delta],1/2))
%[pierw1]=[pierw].real
%[pierw2]=int([pierw1])
%[t1]=([ac]+[pierw2])/2
%[t2]=([ac]-[pierw2])/2
%[t11]=int([t1])
%[t22]=int([t2])
%[a]<[b] and [ab]<[cd] and [ac]>0 and [abcd]>0 and [ab]>0 and [ackw]>[4abcd] and [pierw].is_integer()==True
Rozwiązać równanie $x^4-[a](x^2-[b])=[c](x^2-[d])$.
\zadStop
\rozwStart{Pascal Nawrocki}{}
Równanie postaci $ax^4+bx^2+c=0, a\neq0$ nazywamy równaniem dwukwadratowym. Równanie to sprowadza się do równania kwadratowego za pomocą podstawienia $t=x^2$. Zatem stosując to podstawienie do naszego równania, otrzymamy: $$ t^2-[a](t^2-[b])=[c](t-[d])\Leftrightarrow t^2-[ac]t+[abcd]=0$$
Wówczas: $\bigtriangleup=[ackw]-[4abcd]\Rightarrow\sqrt{[delta]}=[pierw2]$, stąd mamy, że:
$$t_1=\frac{[ac]+[pierw2]}{2}=[t11] \text{ oraz } t_2=\frac{[ac]-[pierw2]}{2}=[t22]$$
Czyli:
$$x^2={[t11]}\Leftrightarrow x_1=\sqrt{[t11]} \vee x_2=-\sqrt{[t11]} \text{ oraz } x^2=[t22]\Leftrightarrow x_3=\sqrt{[t22]} \vee x_4=-\sqrt{[t22]}$$
\rozwStop
\odpStart
$x\in\{\sqrt{[t11]},-\sqrt{[t11]},\sqrt{[t22]},-\sqrt{[t22]}\}$
\odpStop
\testStart
A.$x\in\{\sqrt{[t11]},-\sqrt{[t11]},\sqrt{[t22]},-\sqrt{[t22]}\}$
B.$x=3$
C.$x=\sqrt{[t11]}$
D.$x\in \emptyset$
\testStop
\kluczStart
A
\kluczStop
\end{document}