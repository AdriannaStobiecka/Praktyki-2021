\documentclass[12pt, a4paper]{article}
\usepackage[utf8]{inputenc}
\usepackage{polski}
\usepackage{amsthm}  %pakiet do tworzenia twierdzeń itp.
\usepackage{amsmath} %pakiet do niektórych symboli matematycznych
\usepackage{amssymb} %pakiet do symboli mat., np. \nsubseteq
\usepackage{amsfonts}
\usepackage{graphicx} %obsługa plików graficznych z rozszerzeniem png, jpg
\theoremstyle{definition} %styl dla definicji
\newtheorem{zad}{} 
\title{Multizestaw zadań}
\author{Patryk Wirkus}
%\date{\today}
\date{}
\newcommand{\kategoria}[1]{\section{#1}}
\newcommand{\zadStart}[1]{\begin{zad}#1\newline}
\newcommand{\zadStop}{\end{zad}}
\newcommand{\rozwStart}[2]{\noindent \textbf{Rozwiązanie (autor #1 , recenzent #2): }\newline}
\newcommand{\rozwStop}{\newline}                                           
\newcommand{\odpStart}{\noindent \textbf{Odpowiedź:}\newline}
\newcommand{\odpStop}{\newline}
\newcommand{\testStart}{\noindent \textbf{Test:}\newline}
\newcommand{\testStop}{\newline}
\newcommand{\kluczStart}{\noindent \textbf{Test poprawna odpowiedź:}\newline}
\newcommand{\kluczStop}{\newline}
\newcommand{\wstawGrafike}[2]{\begin{figure}[h] \includegraphics[scale=#2] {#1} \end{figure}}

\begin{document}
\maketitle

\kategoria{Wikieł/Z4.3h}


\zadStart{Zadanie z Wikieł Z 4.3 h) moja wersja nr 1}
Obliczyć granicę funkcji $f(x)=\frac{x^{2} - 1^{2}}{\sqrt{x-(1-1)}-1}$.
\zadStop
\rozwStart{Patryk Wirkus}{Szymon Tokarski}
$$\frac{x^{2} - 1^{2}}{\sqrt{x-(1-1)}-1}=\frac{(x-1)(x+1)(\sqrt{x-(1-1)}+1)}{x-(1-1)-1}=(x+1)(\sqrt{x-(1-1)}+1)$$
\\
$$\lim\limits_{x\to 1}\frac{x^{2} - 1^{2}}{\sqrt{x-(1-1)}-1}=[\frac{0}{0}]=
\lim\limits_{x\to 1}(x+1)(\sqrt{x-(1-1)}+1) = 2\cdot1 \cdot 2 = 4$$
\rozwStop
\odpStart
$4$
\odpStop
\testStart
A.$4$
B.$-4$
C.$0$
D.$1$
E.$\infty$
F.$-\infty$
G.$-1$
H.$0.25$
I.$-0.25$
\testStop
\kluczStart
A
\kluczStop



\zadStart{Zadanie z Wikieł Z 4.3 h) moja wersja nr 2}
Obliczyć granicę funkcji $f(x)=\frac{x^{2} - 4^{2}}{\sqrt{x-(4-1)}-1}$.
\zadStop
\rozwStart{Patryk Wirkus}{Szymon Tokarski}
$$\frac{x^{2} - 4^{2}}{\sqrt{x-(4-1)}-1}=\frac{(x-4)(x+4)(\sqrt{x-(4-1)}+1)}{x-(4-1)-1}=(x+4)(\sqrt{x-(4-1)}+1)$$
\\
$$\lim\limits_{x\to 4}\frac{x^{2} - 4^{2}}{\sqrt{x-(4-1)}-1}=[\frac{0}{0}]=
\lim\limits_{x\to 4}(x+4)(\sqrt{x-(4-1)}+1) = 2\cdot4 \cdot 2 = 16$$
\rozwStop
\odpStart
$16$
\odpStop
\testStart
A.$16$
B.$-16$
C.$0$
D.$4$
E.$\infty$
F.$-\infty$
G.$-4$
H.$1.0$
I.$-1.0$
\testStop
\kluczStart
A
\kluczStop



\zadStart{Zadanie z Wikieł Z 4.3 h) moja wersja nr 3}
Obliczyć granicę funkcji $f(x)=\frac{x^{2} - 6^{2}}{\sqrt{x-(6-1)}-1}$.
\zadStop
\rozwStart{Patryk Wirkus}{Szymon Tokarski}
$$\frac{x^{2} - 6^{2}}{\sqrt{x-(6-1)}-1}=\frac{(x-6)(x+6)(\sqrt{x-(6-1)}+1)}{x-(6-1)-1}=(x+6)(\sqrt{x-(6-1)}+1)$$
\\
$$\lim\limits_{x\to 6}\frac{x^{2} - 6^{2}}{\sqrt{x-(6-1)}-1}=[\frac{0}{0}]=
\lim\limits_{x\to 6}(x+6)(\sqrt{x-(6-1)}+1) = 2\cdot6 \cdot 2 = 24$$
\rozwStop
\odpStart
$24$
\odpStop
\testStart
A.$24$
B.$-24$
C.$0$
D.$6$
E.$\infty$
F.$-\infty$
G.$-6$
H.$1.5$
I.$-1.5$
\testStop
\kluczStart
A
\kluczStop



\zadStart{Zadanie z Wikieł Z 4.3 h) moja wersja nr 4}
Obliczyć granicę funkcji $f(x)=\frac{x^{2} - 9^{2}}{\sqrt{x-(9-1)}-1}$.
\zadStop
\rozwStart{Patryk Wirkus}{Szymon Tokarski}
$$\frac{x^{2} - 9^{2}}{\sqrt{x-(9-1)}-1}=\frac{(x-9)(x+9)(\sqrt{x-(9-1)}+1)}{x-(9-1)-1}=(x+9)(\sqrt{x-(9-1)}+1)$$
\\
$$\lim\limits_{x\to 9}\frac{x^{2} - 9^{2}}{\sqrt{x-(9-1)}-1}=[\frac{0}{0}]=
\lim\limits_{x\to 9}(x+9)(\sqrt{x-(9-1)}+1) = 2\cdot9 \cdot 2 = 36$$
\rozwStop
\odpStart
$36$
\odpStop
\testStart
A.$36$
B.$-36$
C.$0$
D.$9$
E.$\infty$
F.$-\infty$
G.$-9$
H.$2.25$
I.$-2.25$
\testStop
\kluczStart
A
\kluczStop



\zadStart{Zadanie z Wikieł Z 4.3 h) moja wersja nr 5}
Obliczyć granicę funkcji $f(x)=\frac{x^{2} - 16^{2}}{\sqrt{x-(16-1)}-1}$.
\zadStop
\rozwStart{Patryk Wirkus}{Szymon Tokarski}
$$\frac{x^{2} - 16^{2}}{\sqrt{x-(16-1)}-1}=\frac{(x-16)(x+16)(\sqrt{x-(16-1)}+1)}{x-(16-1)-1}=(x+16)(\sqrt{x-(16-1)}+1)$$
\\
$$\lim\limits_{x\to 16}\frac{x^{2} - 16^{2}}{\sqrt{x-(16-1)}-1}=[\frac{0}{0}]=
\lim\limits_{x\to 16}(x+16)(\sqrt{x-(16-1)}+1) = 2\cdot16 \cdot 2 = 64$$
\rozwStop
\odpStart
$64$
\odpStop
\testStart
A.$64$
B.$-64$
C.$0$
D.$16$
E.$\infty$
F.$-\infty$
G.$-16$
H.$4.0$
I.$-4.0$
\testStop
\kluczStart
A
\kluczStop



\zadStart{Zadanie z Wikieł Z 4.3 h) moja wersja nr 6}
Obliczyć granicę funkcji $f(x)=\frac{x^{2} - 25^{2}}{\sqrt{x-(25-1)}-1}$.
\zadStop
\rozwStart{Patryk Wirkus}{Szymon Tokarski}
$$\frac{x^{2} - 25^{2}}{\sqrt{x-(25-1)}-1}=\frac{(x-25)(x+25)(\sqrt{x-(25-1)}+1)}{x-(25-1)-1}=(x+25)(\sqrt{x-(25-1)}+1)$$
\\
$$\lim\limits_{x\to 25}\frac{x^{2} - 25^{2}}{\sqrt{x-(25-1)}-1}=[\frac{0}{0}]=
\lim\limits_{x\to 25}(x+25)(\sqrt{x-(25-1)}+1) = 2\cdot25 \cdot 2 = 100$$
\rozwStop
\odpStart
$100$
\odpStop
\testStart
A.$100$
B.$-100$
C.$0$
D.$25$
E.$\infty$
F.$-\infty$
G.$-25$
H.$6.25$
I.$-6.25$
\testStop
\kluczStart
A
\kluczStop



\zadStart{Zadanie z Wikieł Z 4.3 h) moja wersja nr 7}
Obliczyć granicę funkcji $f(x)=\frac{x^{2} - 36^{2}}{\sqrt{x-(36-1)}-1}$.
\zadStop
\rozwStart{Patryk Wirkus}{Szymon Tokarski}
$$\frac{x^{2} - 36^{2}}{\sqrt{x-(36-1)}-1}=\frac{(x-36)(x+36)(\sqrt{x-(36-1)}+1)}{x-(36-1)-1}=(x+36)(\sqrt{x-(36-1)}+1)$$
\\
$$\lim\limits_{x\to 36}\frac{x^{2} - 36^{2}}{\sqrt{x-(36-1)}-1}=[\frac{0}{0}]=
\lim\limits_{x\to 36}(x+36)(\sqrt{x-(36-1)}+1) = 2\cdot36 \cdot 2 = 144$$
\rozwStop
\odpStart
$144$
\odpStop
\testStart
A.$144$
B.$-144$
C.$0$
D.$36$
E.$\infty$
F.$-\infty$
G.$-36$
H.$9.0$
I.$-9.0$
\testStop
\kluczStart
A
\kluczStop



\zadStart{Zadanie z Wikieł Z 4.3 h) moja wersja nr 8}
Obliczyć granicę funkcji $f(x)=\frac{x^{2} - 49^{2}}{\sqrt{x-(49-1)}-1}$.
\zadStop
\rozwStart{Patryk Wirkus}{Szymon Tokarski}
$$\frac{x^{2} - 49^{2}}{\sqrt{x-(49-1)}-1}=\frac{(x-49)(x+49)(\sqrt{x-(49-1)}+1)}{x-(49-1)-1}=(x+49)(\sqrt{x-(49-1)}+1)$$
\\
$$\lim\limits_{x\to 49}\frac{x^{2} - 49^{2}}{\sqrt{x-(49-1)}-1}=[\frac{0}{0}]=
\lim\limits_{x\to 49}(x+49)(\sqrt{x-(49-1)}+1) = 2\cdot49 \cdot 2 = 196$$
\rozwStop
\odpStart
$196$
\odpStop
\testStart
A.$196$
B.$-196$
C.$0$
D.$49$
E.$\infty$
F.$-\infty$
G.$-49$
H.$12.25$
I.$-12.25$
\testStop
\kluczStart
A
\kluczStop



\zadStart{Zadanie z Wikieł Z 4.3 h) moja wersja nr 9}
Obliczyć granicę funkcji $f(x)=\frac{x^{2} - 64^{2}}{\sqrt{x-(64-1)}-1}$.
\zadStop
\rozwStart{Patryk Wirkus}{Szymon Tokarski}
$$\frac{x^{2} - 64^{2}}{\sqrt{x-(64-1)}-1}=\frac{(x-64)(x+64)(\sqrt{x-(64-1)}+1)}{x-(64-1)-1}=(x+64)(\sqrt{x-(64-1)}+1)$$
\\
$$\lim\limits_{x\to 64}\frac{x^{2} - 64^{2}}{\sqrt{x-(64-1)}-1}=[\frac{0}{0}]=
\lim\limits_{x\to 64}(x+64)(\sqrt{x-(64-1)}+1) = 2\cdot64 \cdot 2 = 256$$
\rozwStop
\odpStart
$256$
\odpStop
\testStart
A.$256$
B.$-256$
C.$0$
D.$64$
E.$\infty$
F.$-\infty$
G.$-64$
H.$16.0$
I.$-16.0$
\testStop
\kluczStart
A
\kluczStop



\zadStart{Zadanie z Wikieł Z 4.3 h) moja wersja nr 10}
Obliczyć granicę funkcji $f(x)=\frac{x^{2} - 81^{2}}{\sqrt{x-(81-1)}-1}$.
\zadStop
\rozwStart{Patryk Wirkus}{Szymon Tokarski}
$$\frac{x^{2} - 81^{2}}{\sqrt{x-(81-1)}-1}=\frac{(x-81)(x+81)(\sqrt{x-(81-1)}+1)}{x-(81-1)-1}=(x+81)(\sqrt{x-(81-1)}+1)$$
\\
$$\lim\limits_{x\to 81}\frac{x^{2} - 81^{2}}{\sqrt{x-(81-1)}-1}=[\frac{0}{0}]=
\lim\limits_{x\to 81}(x+81)(\sqrt{x-(81-1)}+1) = 2\cdot81 \cdot 2 = 324$$
\rozwStop
\odpStart
$324$
\odpStop
\testStart
A.$324$
B.$-324$
C.$0$
D.$81$
E.$\infty$
F.$-\infty$
G.$-81$
H.$20.25$
I.$-20.25$
\testStop
\kluczStart
A
\kluczStop



\zadStart{Zadanie z Wikieł Z 4.3 h) moja wersja nr 11}
Obliczyć granicę funkcji $f(x)=\frac{x^{2} - 100^{2}}{\sqrt{x-(100-1)}-1}$.
\zadStop
\rozwStart{Patryk Wirkus}{Szymon Tokarski}
$$\frac{x^{2} - 100^{2}}{\sqrt{x-(100-1)}-1}=\frac{(x-100)(x+100)(\sqrt{x-(100-1)}+1)}{x-(100-1)-1}=(x+100)(\sqrt{x-(100-1)}+1)$$
\\
$$\lim\limits_{x\to 100}\frac{x^{2} - 100^{2}}{\sqrt{x-(100-1)}-1}=[\frac{0}{0}]=
\lim\limits_{x\to 100}(x+100)(\sqrt{x-(100-1)}+1) = 2\cdot100 \cdot 2 = 400$$
\rozwStop
\odpStart
$400$
\odpStop
\testStart
A.$400$
B.$-400$
C.$0$
D.$100$
E.$\infty$
F.$-\infty$
G.$-100$
H.$25.0$
I.$-25.0$
\testStop
\kluczStart
A
\kluczStop



\zadStart{Zadanie z Wikieł Z 4.3 h) moja wersja nr 12}
Obliczyć granicę funkcji $f(x)=\frac{x^{2} - 121^{2}}{\sqrt{x-(121-1)}-1}$.
\zadStop
\rozwStart{Patryk Wirkus}{Szymon Tokarski}
$$\frac{x^{2} - 121^{2}}{\sqrt{x-(121-1)}-1}=\frac{(x-121)(x+121)(\sqrt{x-(121-1)}+1)}{x-(121-1)-1}=(x+121)(\sqrt{x-(121-1)}+1)$$
\\
$$\lim\limits_{x\to 121}\frac{x^{2} - 121^{2}}{\sqrt{x-(121-1)}-1}=[\frac{0}{0}]=
\lim\limits_{x\to 121}(x+121)(\sqrt{x-(121-1)}+1) = 2\cdot121 \cdot 2 = 484$$
\rozwStop
\odpStart
$484$
\odpStop
\testStart
A.$484$
B.$-484$
C.$0$
D.$121$
E.$\infty$
F.$-\infty$
G.$-121$
H.$30.25$
I.$-30.25$
\testStop
\kluczStart
A
\kluczStop



\zadStart{Zadanie z Wikieł Z 4.3 h) moja wersja nr 13}
Obliczyć granicę funkcji $f(x)=\frac{x^{2} - 144^{2}}{\sqrt{x-(144-1)}-1}$.
\zadStop
\rozwStart{Patryk Wirkus}{Szymon Tokarski}
$$\frac{x^{2} - 144^{2}}{\sqrt{x-(144-1)}-1}=\frac{(x-144)(x+144)(\sqrt{x-(144-1)}+1)}{x-(144-1)-1}=(x+144)(\sqrt{x-(144-1)}+1)$$
\\
$$\lim\limits_{x\to 144}\frac{x^{2} - 144^{2}}{\sqrt{x-(144-1)}-1}=[\frac{0}{0}]=
\lim\limits_{x\to 144}(x+144)(\sqrt{x-(144-1)}+1) = 2\cdot144 \cdot 2 = 576$$
\rozwStop
\odpStart
$576$
\odpStop
\testStart
A.$576$
B.$-576$
C.$0$
D.$144$
E.$\infty$
F.$-\infty$
G.$-144$
H.$36.0$
I.$-36.0$
\testStop
\kluczStart
A
\kluczStop



\zadStart{Zadanie z Wikieł Z 4.3 h) moja wersja nr 14}
Obliczyć granicę funkcji $f(x)=\frac{x^{2} - 169^{2}}{\sqrt{x-(169-1)}-1}$.
\zadStop
\rozwStart{Patryk Wirkus}{Szymon Tokarski}
$$\frac{x^{2} - 169^{2}}{\sqrt{x-(169-1)}-1}=\frac{(x-169)(x+169)(\sqrt{x-(169-1)}+1)}{x-(169-1)-1}=(x+169)(\sqrt{x-(169-1)}+1)$$
\\
$$\lim\limits_{x\to 169}\frac{x^{2} - 169^{2}}{\sqrt{x-(169-1)}-1}=[\frac{0}{0}]=
\lim\limits_{x\to 169}(x+169)(\sqrt{x-(169-1)}+1) = 2\cdot169 \cdot 2 = 676$$
\rozwStop
\odpStart
$676$
\odpStop
\testStart
A.$676$
B.$-676$
C.$0$
D.$169$
E.$\infty$
F.$-\infty$
G.$-169$
H.$42.25$
I.$-42.25$
\testStop
\kluczStart
A
\kluczStop



\zadStart{Zadanie z Wikieł Z 4.3 h) moja wersja nr 15}
Obliczyć granicę funkcji $f(x)=\frac{x^{2} - 196^{2}}{\sqrt{x-(196-1)}-1}$.
\zadStop
\rozwStart{Patryk Wirkus}{Szymon Tokarski}
$$\frac{x^{2} - 196^{2}}{\sqrt{x-(196-1)}-1}=\frac{(x-196)(x+196)(\sqrt{x-(196-1)}+1)}{x-(196-1)-1}=(x+196)(\sqrt{x-(196-1)}+1)$$
\\
$$\lim\limits_{x\to 196}\frac{x^{2} - 196^{2}}{\sqrt{x-(196-1)}-1}=[\frac{0}{0}]=
\lim\limits_{x\to 196}(x+196)(\sqrt{x-(196-1)}+1) = 2\cdot196 \cdot 2 = 784$$
\rozwStop
\odpStart
$784$
\odpStop
\testStart
A.$784$
B.$-784$
C.$0$
D.$196$
E.$\infty$
F.$-\infty$
G.$-196$
H.$49.0$
I.$-49.0$
\testStop
\kluczStart
A
\kluczStop



\zadStart{Zadanie z Wikieł Z 4.3 h) moja wersja nr 16}
Obliczyć granicę funkcji $f(x)=\frac{x^{2} - 225^{2}}{\sqrt{x-(225-1)}-1}$.
\zadStop
\rozwStart{Patryk Wirkus}{Szymon Tokarski}
$$\frac{x^{2} - 225^{2}}{\sqrt{x-(225-1)}-1}=\frac{(x-225)(x+225)(\sqrt{x-(225-1)}+1)}{x-(225-1)-1}=(x+225)(\sqrt{x-(225-1)}+1)$$
\\
$$\lim\limits_{x\to 225}\frac{x^{2} - 225^{2}}{\sqrt{x-(225-1)}-1}=[\frac{0}{0}]=
\lim\limits_{x\to 225}(x+225)(\sqrt{x-(225-1)}+1) = 2\cdot225 \cdot 2 = 900$$
\rozwStop
\odpStart
$900$
\odpStop
\testStart
A.$900$
B.$-900$
C.$0$
D.$225$
E.$\infty$
F.$-\infty$
G.$-225$
H.$56.25$
I.$-56.25$
\testStop
\kluczStart
A
\kluczStop



\zadStart{Zadanie z Wikieł Z 4.3 h) moja wersja nr 17}
Obliczyć granicę funkcji $f(x)=\frac{x^{2} - 256^{2}}{\sqrt{x-(256-1)}-1}$.
\zadStop
\rozwStart{Patryk Wirkus}{Szymon Tokarski}
$$\frac{x^{2} - 256^{2}}{\sqrt{x-(256-1)}-1}=\frac{(x-256)(x+256)(\sqrt{x-(256-1)}+1)}{x-(256-1)-1}=(x+256)(\sqrt{x-(256-1)}+1)$$
\\
$$\lim\limits_{x\to 256}\frac{x^{2} - 256^{2}}{\sqrt{x-(256-1)}-1}=[\frac{0}{0}]=
\lim\limits_{x\to 256}(x+256)(\sqrt{x-(256-1)}+1) = 2\cdot256 \cdot 2 = 1024$$
\rozwStop
\odpStart
$1024$
\odpStop
\testStart
A.$1024$
B.$-1024$
C.$0$
D.$256$
E.$\infty$
F.$-\infty$
G.$-256$
H.$64.0$
I.$-64.0$
\testStop
\kluczStart
A
\kluczStop



\zadStart{Zadanie z Wikieł Z 4.3 h) moja wersja nr 18}
Obliczyć granicę funkcji $f(x)=\frac{x^{2} - 289^{2}}{\sqrt{x-(289-1)}-1}$.
\zadStop
\rozwStart{Patryk Wirkus}{Szymon Tokarski}
$$\frac{x^{2} - 289^{2}}{\sqrt{x-(289-1)}-1}=\frac{(x-289)(x+289)(\sqrt{x-(289-1)}+1)}{x-(289-1)-1}=(x+289)(\sqrt{x-(289-1)}+1)$$
\\
$$\lim\limits_{x\to 289}\frac{x^{2} - 289^{2}}{\sqrt{x-(289-1)}-1}=[\frac{0}{0}]=
\lim\limits_{x\to 289}(x+289)(\sqrt{x-(289-1)}+1) = 2\cdot289 \cdot 2 = 1156$$
\rozwStop
\odpStart
$1156$
\odpStop
\testStart
A.$1156$
B.$-1156$
C.$0$
D.$289$
E.$\infty$
F.$-\infty$
G.$-289$
H.$72.25$
I.$-72.25$
\testStop
\kluczStart
A
\kluczStop



\zadStart{Zadanie z Wikieł Z 4.3 h) moja wersja nr 19}
Obliczyć granicę funkcji $f(x)=\frac{x^{2} - 324^{2}}{\sqrt{x-(324-1)}-1}$.
\zadStop
\rozwStart{Patryk Wirkus}{Szymon Tokarski}
$$\frac{x^{2} - 324^{2}}{\sqrt{x-(324-1)}-1}=\frac{(x-324)(x+324)(\sqrt{x-(324-1)}+1)}{x-(324-1)-1}=(x+324)(\sqrt{x-(324-1)}+1)$$
\\
$$\lim\limits_{x\to 324}\frac{x^{2} - 324^{2}}{\sqrt{x-(324-1)}-1}=[\frac{0}{0}]=
\lim\limits_{x\to 324}(x+324)(\sqrt{x-(324-1)}+1) = 2\cdot324 \cdot 2 = 1296$$
\rozwStop
\odpStart
$1296$
\odpStop
\testStart
A.$1296$
B.$-1296$
C.$0$
D.$324$
E.$\infty$
F.$-\infty$
G.$-324$
H.$81.0$
I.$-81.0$
\testStop
\kluczStart
A
\kluczStop



\zadStart{Zadanie z Wikieł Z 4.3 h) moja wersja nr 20}
Obliczyć granicę funkcji $f(x)=\frac{x^{2} - 361^{2}}{\sqrt{x-(361-1)}-1}$.
\zadStop
\rozwStart{Patryk Wirkus}{Szymon Tokarski}
$$\frac{x^{2} - 361^{2}}{\sqrt{x-(361-1)}-1}=\frac{(x-361)(x+361)(\sqrt{x-(361-1)}+1)}{x-(361-1)-1}=(x+361)(\sqrt{x-(361-1)}+1)$$
\\
$$\lim\limits_{x\to 361}\frac{x^{2} - 361^{2}}{\sqrt{x-(361-1)}-1}=[\frac{0}{0}]=
\lim\limits_{x\to 361}(x+361)(\sqrt{x-(361-1)}+1) = 2\cdot361 \cdot 2 = 1444$$
\rozwStop
\odpStart
$1444$
\odpStop
\testStart
A.$1444$
B.$-1444$
C.$0$
D.$361$
E.$\infty$
F.$-\infty$
G.$-361$
H.$90.25$
I.$-90.25$
\testStop
\kluczStart
A
\kluczStop



\zadStart{Zadanie z Wikieł Z 4.3 h) moja wersja nr 21}
Obliczyć granicę funkcji $f(x)=\frac{x^{2} - 400^{2}}{\sqrt{x-(400-1)}-1}$.
\zadStop
\rozwStart{Patryk Wirkus}{Szymon Tokarski}
$$\frac{x^{2} - 400^{2}}{\sqrt{x-(400-1)}-1}=\frac{(x-400)(x+400)(\sqrt{x-(400-1)}+1)}{x-(400-1)-1}=(x+400)(\sqrt{x-(400-1)}+1)$$
\\
$$\lim\limits_{x\to 400}\frac{x^{2} - 400^{2}}{\sqrt{x-(400-1)}-1}=[\frac{0}{0}]=
\lim\limits_{x\to 400}(x+400)(\sqrt{x-(400-1)}+1) = 2\cdot400 \cdot 2 = 1600$$
\rozwStop
\odpStart
$1600$
\odpStop
\testStart
A.$1600$
B.$-1600$
C.$0$
D.$400$
E.$\infty$
F.$-\infty$
G.$-400$
H.$100.0$
I.$-100.0$
\testStop
\kluczStart
A
\kluczStop



\zadStart{Zadanie z Wikieł Z 4.3 h) moja wersja nr 22}
Obliczyć granicę funkcji $f(x)=\frac{x^{2} - 625^{2}}{\sqrt{x-(625-1)}-1}$.
\zadStop
\rozwStart{Patryk Wirkus}{Szymon Tokarski}
$$\frac{x^{2} - 625^{2}}{\sqrt{x-(625-1)}-1}=\frac{(x-625)(x+625)(\sqrt{x-(625-1)}+1)}{x-(625-1)-1}=(x+625)(\sqrt{x-(625-1)}+1)$$
\\
$$\lim\limits_{x\to 625}\frac{x^{2} - 625^{2}}{\sqrt{x-(625-1)}-1}=[\frac{0}{0}]=
\lim\limits_{x\to 625}(x+625)(\sqrt{x-(625-1)}+1) = 2\cdot625 \cdot 2 = 2500$$
\rozwStop
\odpStart
$2500$
\odpStop
\testStart
A.$2500$
B.$-2500$
C.$0$
D.$625$
E.$\infty$
F.$-\infty$
G.$-625$
H.$156.25$
I.$-156.25$
\testStop
\kluczStart
A
\kluczStop



\zadStart{Zadanie z Wikieł Z 4.3 h) moja wersja nr 23}
Obliczyć granicę funkcji $f(x)=\frac{x^{2} - 900^{2}}{\sqrt{x-(900-1)}-1}$.
\zadStop
\rozwStart{Patryk Wirkus}{Szymon Tokarski}
$$\frac{x^{2} - 900^{2}}{\sqrt{x-(900-1)}-1}=\frac{(x-900)(x+900)(\sqrt{x-(900-1)}+1)}{x-(900-1)-1}=(x+900)(\sqrt{x-(900-1)}+1)$$
\\
$$\lim\limits_{x\to 900}\frac{x^{2} - 900^{2}}{\sqrt{x-(900-1)}-1}=[\frac{0}{0}]=
\lim\limits_{x\to 900}(x+900)(\sqrt{x-(900-1)}+1) = 2\cdot900 \cdot 2 = 3600$$
\rozwStop
\odpStart
$3600$
\odpStop
\testStart
A.$3600$
B.$-3600$
C.$0$
D.$900$
E.$\infty$
F.$-\infty$
G.$-900$
H.$225.0$
I.$-225.0$
\testStop
\kluczStart
A
\kluczStop



\zadStart{Zadanie z Wikieł Z 4.3 h) moja wersja nr 24}
Obliczyć granicę funkcji $f(x)=\frac{x^{2} - 1600^{2}}{\sqrt{x-(1600-1)}-1}$.
\zadStop
\rozwStart{Patryk Wirkus}{Szymon Tokarski}
$$\frac{x^{2} - 1600^{2}}{\sqrt{x-(1600-1)}-1}=\frac{(x-1600)(x+1600)(\sqrt{x-(1600-1)}+1)}{x-(1600-1)-1}=(x+1600)(\sqrt{x-(1600-1)}+1)$$
\\
$$\lim\limits_{x\to 1600}\frac{x^{2} - 1600^{2}}{\sqrt{x-(1600-1)}-1}=[\frac{0}{0}]=
\lim\limits_{x\to 1600}(x+1600)(\sqrt{x-(1600-1)}+1) = 2\cdot1600 \cdot 2 = 6400$$
\rozwStop
\odpStart
$6400$
\odpStop
\testStart
A.$6400$
B.$-6400$
C.$0$
D.$1600$
E.$\infty$
F.$-\infty$
G.$-1600$
H.$400.0$
I.$-400.0$
\testStop
\kluczStart
A
\kluczStop



\zadStart{Zadanie z Wikieł Z 4.3 h) moja wersja nr 25}
Obliczyć granicę funkcji $f(x)=\frac{x^{2} - 2500^{2}}{\sqrt{x-(2500-1)}-1}$.
\zadStop
\rozwStart{Patryk Wirkus}{Szymon Tokarski}
$$\frac{x^{2} - 2500^{2}}{\sqrt{x-(2500-1)}-1}=\frac{(x-2500)(x+2500)(\sqrt{x-(2500-1)}+1)}{x-(2500-1)-1}=(x+2500)(\sqrt{x-(2500-1)}+1)$$
\\
$$\lim\limits_{x\to 2500}\frac{x^{2} - 2500^{2}}{\sqrt{x-(2500-1)}-1}=[\frac{0}{0}]=
\lim\limits_{x\to 2500}(x+2500)(\sqrt{x-(2500-1)}+1) = 2\cdot2500 \cdot 2 = 10000$$
\rozwStop
\odpStart
$10000$
\odpStop
\testStart
A.$10000$
B.$-10000$
C.$0$
D.$2500$
E.$\infty$
F.$-\infty$
G.$-2500$
H.$625.0$
I.$-625.0$
\testStop
\kluczStart
A
\kluczStop



\zadStart{Zadanie z Wikieł Z 4.3 h) moja wersja nr 26}
Obliczyć granicę funkcji $f(x)=\frac{x^{2} - 3600^{2}}{\sqrt{x-(3600-1)}-1}$.
\zadStop
\rozwStart{Patryk Wirkus}{Szymon Tokarski}
$$\frac{x^{2} - 3600^{2}}{\sqrt{x-(3600-1)}-1}=\frac{(x-3600)(x+3600)(\sqrt{x-(3600-1)}+1)}{x-(3600-1)-1}=(x+3600)(\sqrt{x-(3600-1)}+1)$$
\\
$$\lim\limits_{x\to 3600}\frac{x^{2} - 3600^{2}}{\sqrt{x-(3600-1)}-1}=[\frac{0}{0}]=
\lim\limits_{x\to 3600}(x+3600)(\sqrt{x-(3600-1)}+1) = 2\cdot3600 \cdot 2 = 14400$$
\rozwStop
\odpStart
$14400$
\odpStop
\testStart
A.$14400$
B.$-14400$
C.$0$
D.$3600$
E.$\infty$
F.$-\infty$
G.$-3600$
H.$900.0$
I.$-900.0$
\testStop
\kluczStart
A
\kluczStop



\zadStart{Zadanie z Wikieł Z 4.3 h) moja wersja nr 27}
Obliczyć granicę funkcji $f(x)=\frac{x^{2} - 4900^{2}}{\sqrt{x-(4900-1)}-1}$.
\zadStop
\rozwStart{Patryk Wirkus}{Szymon Tokarski}
$$\frac{x^{2} - 4900^{2}}{\sqrt{x-(4900-1)}-1}=\frac{(x-4900)(x+4900)(\sqrt{x-(4900-1)}+1)}{x-(4900-1)-1}=(x+4900)(\sqrt{x-(4900-1)}+1)$$
\\
$$\lim\limits_{x\to 4900}\frac{x^{2} - 4900^{2}}{\sqrt{x-(4900-1)}-1}=[\frac{0}{0}]=
\lim\limits_{x\to 4900}(x+4900)(\sqrt{x-(4900-1)}+1) = 2\cdot4900 \cdot 2 = 19600$$
\rozwStop
\odpStart
$19600$
\odpStop
\testStart
A.$19600$
B.$-19600$
C.$0$
D.$4900$
E.$\infty$
F.$-\infty$
G.$-4900$
H.$1225.0$
I.$-1225.0$
\testStop
\kluczStart
A
\kluczStop



\zadStart{Zadanie z Wikieł Z 4.3 h) moja wersja nr 28}
Obliczyć granicę funkcji $f(x)=\frac{x^{2} - 6400^{2}}{\sqrt{x-(6400-1)}-1}$.
\zadStop
\rozwStart{Patryk Wirkus}{Szymon Tokarski}
$$\frac{x^{2} - 6400^{2}}{\sqrt{x-(6400-1)}-1}=\frac{(x-6400)(x+6400)(\sqrt{x-(6400-1)}+1)}{x-(6400-1)-1}=(x+6400)(\sqrt{x-(6400-1)}+1)$$
\\
$$\lim\limits_{x\to 6400}\frac{x^{2} - 6400^{2}}{\sqrt{x-(6400-1)}-1}=[\frac{0}{0}]=
\lim\limits_{x\to 6400}(x+6400)(\sqrt{x-(6400-1)}+1) = 2\cdot6400 \cdot 2 = 25600$$
\rozwStop
\odpStart
$25600$
\odpStop
\testStart
A.$25600$
B.$-25600$
C.$0$
D.$6400$
E.$\infty$
F.$-\infty$
G.$-6400$
H.$1600.0$
I.$-1600.0$
\testStop
\kluczStart
A
\kluczStop



\zadStart{Zadanie z Wikieł Z 4.3 h) moja wersja nr 29}
Obliczyć granicę funkcji $f(x)=\frac{x^{2} - 8100^{2}}{\sqrt{x-(8100-1)}-1}$.
\zadStop
\rozwStart{Patryk Wirkus}{Szymon Tokarski}
$$\frac{x^{2} - 8100^{2}}{\sqrt{x-(8100-1)}-1}=\frac{(x-8100)(x+8100)(\sqrt{x-(8100-1)}+1)}{x-(8100-1)-1}=(x+8100)(\sqrt{x-(8100-1)}+1)$$
\\
$$\lim\limits_{x\to 8100}\frac{x^{2} - 8100^{2}}{\sqrt{x-(8100-1)}-1}=[\frac{0}{0}]=
\lim\limits_{x\to 8100}(x+8100)(\sqrt{x-(8100-1)}+1) = 2\cdot8100 \cdot 2 = 32400$$
\rozwStop
\odpStart
$32400$
\odpStop
\testStart
A.$32400$
B.$-32400$
C.$0$
D.$8100$
E.$\infty$
F.$-\infty$
G.$-8100$
H.$2025.0$
I.$-2025.0$
\testStop
\kluczStart
A
\kluczStop



\zadStart{Zadanie z Wikieł Z 4.3 h) moja wersja nr 30}
Obliczyć granicę funkcji $f(x)=\frac{x^{2} - 10000^{2}}{\sqrt{x-(10000-1)}-1}$.
\zadStop
\rozwStart{Patryk Wirkus}{Szymon Tokarski}
$$\frac{x^{2} - 10000^{2}}{\sqrt{x-(10000-1)}-1}=\frac{(x-10000)(x+10000)(\sqrt{x-(10000-1)}+1)}{x-(10000-1)-1}=(x+10000)(\sqrt{x-(10000-1)}+1)$$
\\
$$\lim\limits_{x\to 10000}\frac{x^{2} - 10000^{2}}{\sqrt{x-(10000-1)}-1}=[\frac{0}{0}]=
\lim\limits_{x\to 10000}(x+10000)(\sqrt{x-(10000-1)}+1) = 2\cdot10000 \cdot 2 = 40000$$
\rozwStop
\odpStart
$40000$
\odpStop
\testStart
A.$40000$
B.$-40000$
C.$0$
D.$10000$
E.$\infty$
F.$-\infty$
G.$-10000$
H.$2500.0$
I.$-2500.0$
\testStop
\kluczStart
A
\kluczStop



\zadStart{Zadanie z Wikieł Z 4.3 h) moja wersja nr 31}
Obliczyć granicę funkcji $f(x)=\frac{x^{2} - 2^{2}}{\sqrt{x-(2-1)}-1}$.
\zadStop
\rozwStart{Patryk Wirkus}{Szymon Tokarski}
$$\frac{x^{2} - 2^{2}}{\sqrt{x-(2-1)}-1}=\frac{(x-2)(x+2)(\sqrt{x-(2-1)}+1)}{x-(2-1)-1}=(x+2)(\sqrt{x-(2-1)}+1)$$
\\
$$\lim\limits_{x\to 2}\frac{x^{2} - 2^{2}}{\sqrt{x-(2-1)}-1}=[\frac{0}{0}]=
\lim\limits_{x\to 2}(x+2)(\sqrt{x-(2-1)}+1) = 2\cdot2 \cdot 2 = 8$$
\rozwStop
\odpStart
$8$
\odpStop
\testStart
A.$8$
B.$-8$
C.$0$
D.$2$
E.$\infty$
F.$-\infty$
G.$-2$
H.$0.5$
I.$-0.5$
\testStop
\kluczStart
A
\kluczStop



\zadStart{Zadanie z Wikieł Z 4.3 h) moja wersja nr 32}
Obliczyć granicę funkcji $f(x)=\frac{x^{2} - 3^{2}}{\sqrt{x-(3-1)}-1}$.
\zadStop
\rozwStart{Patryk Wirkus}{Szymon Tokarski}
$$\frac{x^{2} - 3^{2}}{\sqrt{x-(3-1)}-1}=\frac{(x-3)(x+3)(\sqrt{x-(3-1)}+1)}{x-(3-1)-1}=(x+3)(\sqrt{x-(3-1)}+1)$$
\\
$$\lim\limits_{x\to 3}\frac{x^{2} - 3^{2}}{\sqrt{x-(3-1)}-1}=[\frac{0}{0}]=
\lim\limits_{x\to 3}(x+3)(\sqrt{x-(3-1)}+1) = 2\cdot3 \cdot 2 = 12$$
\rozwStop
\odpStart
$12$
\odpStop
\testStart
A.$12$
B.$-12$
C.$0$
D.$3$
E.$\infty$
F.$-\infty$
G.$-3$
H.$0.75$
I.$-0.75$
\testStop
\kluczStart
A
\kluczStop



\zadStart{Zadanie z Wikieł Z 4.3 h) moja wersja nr 33}
Obliczyć granicę funkcji $f(x)=\frac{x^{2} - 5^{2}}{\sqrt{x-(5-1)}-1}$.
\zadStop
\rozwStart{Patryk Wirkus}{Szymon Tokarski}
$$\frac{x^{2} - 5^{2}}{\sqrt{x-(5-1)}-1}=\frac{(x-5)(x+5)(\sqrt{x-(5-1)}+1)}{x-(5-1)-1}=(x+5)(\sqrt{x-(5-1)}+1)$$
\\
$$\lim\limits_{x\to 5}\frac{x^{2} - 5^{2}}{\sqrt{x-(5-1)}-1}=[\frac{0}{0}]=
\lim\limits_{x\to 5}(x+5)(\sqrt{x-(5-1)}+1) = 2\cdot5 \cdot 2 = 20$$
\rozwStop
\odpStart
$20$
\odpStop
\testStart
A.$20$
B.$-20$
C.$0$
D.$5$
E.$\infty$
F.$-\infty$
G.$-5$
H.$1.25$
I.$-1.25$
\testStop
\kluczStart
A
\kluczStop



\zadStart{Zadanie z Wikieł Z 4.3 h) moja wersja nr 34}
Obliczyć granicę funkcji $f(x)=\frac{x^{2} - 7^{2}}{\sqrt{x-(7-1)}-1}$.
\zadStop
\rozwStart{Patryk Wirkus}{Szymon Tokarski}
$$\frac{x^{2} - 7^{2}}{\sqrt{x-(7-1)}-1}=\frac{(x-7)(x+7)(\sqrt{x-(7-1)}+1)}{x-(7-1)-1}=(x+7)(\sqrt{x-(7-1)}+1)$$
\\
$$\lim\limits_{x\to 7}\frac{x^{2} - 7^{2}}{\sqrt{x-(7-1)}-1}=[\frac{0}{0}]=
\lim\limits_{x\to 7}(x+7)(\sqrt{x-(7-1)}+1) = 2\cdot7 \cdot 2 = 28$$
\rozwStop
\odpStart
$28$
\odpStop
\testStart
A.$28$
B.$-28$
C.$0$
D.$7$
E.$\infty$
F.$-\infty$
G.$-7$
H.$1.75$
I.$-1.75$
\testStop
\kluczStart
A
\kluczStop



\zadStart{Zadanie z Wikieł Z 4.3 h) moja wersja nr 35}
Obliczyć granicę funkcji $f(x)=\frac{x^{2} - 11^{2}}{\sqrt{x-(11-1)}-1}$.
\zadStop
\rozwStart{Patryk Wirkus}{Szymon Tokarski}
$$\frac{x^{2} - 11^{2}}{\sqrt{x-(11-1)}-1}=\frac{(x-11)(x+11)(\sqrt{x-(11-1)}+1)}{x-(11-1)-1}=(x+11)(\sqrt{x-(11-1)}+1)$$
\\
$$\lim\limits_{x\to 11}\frac{x^{2} - 11^{2}}{\sqrt{x-(11-1)}-1}=[\frac{0}{0}]=
\lim\limits_{x\to 11}(x+11)(\sqrt{x-(11-1)}+1) = 2\cdot11 \cdot 2 = 44$$
\rozwStop
\odpStart
$44$
\odpStop
\testStart
A.$44$
B.$-44$
C.$0$
D.$11$
E.$\infty$
F.$-\infty$
G.$-11$
H.$2.75$
I.$-2.75$
\testStop
\kluczStart
A
\kluczStop



\zadStart{Zadanie z Wikieł Z 4.3 h) moja wersja nr 36}
Obliczyć granicę funkcji $f(x)=\frac{x^{2} - 13^{2}}{\sqrt{x-(13-1)}-1}$.
\zadStop
\rozwStart{Patryk Wirkus}{Szymon Tokarski}
$$\frac{x^{2} - 13^{2}}{\sqrt{x-(13-1)}-1}=\frac{(x-13)(x+13)(\sqrt{x-(13-1)}+1)}{x-(13-1)-1}=(x+13)(\sqrt{x-(13-1)}+1)$$
\\
$$\lim\limits_{x\to 13}\frac{x^{2} - 13^{2}}{\sqrt{x-(13-1)}-1}=[\frac{0}{0}]=
\lim\limits_{x\to 13}(x+13)(\sqrt{x-(13-1)}+1) = 2\cdot13 \cdot 2 = 52$$
\rozwStop
\odpStart
$52$
\odpStop
\testStart
A.$52$
B.$-52$
C.$0$
D.$13$
E.$\infty$
F.$-\infty$
G.$-13$
H.$3.25$
I.$-3.25$
\testStop
\kluczStart
A
\kluczStop



\zadStart{Zadanie z Wikieł Z 4.3 h) moja wersja nr 37}
Obliczyć granicę funkcji $f(x)=\frac{x^{2} - 17^{2}}{\sqrt{x-(17-1)}-1}$.
\zadStop
\rozwStart{Patryk Wirkus}{Szymon Tokarski}
$$\frac{x^{2} - 17^{2}}{\sqrt{x-(17-1)}-1}=\frac{(x-17)(x+17)(\sqrt{x-(17-1)}+1)}{x-(17-1)-1}=(x+17)(\sqrt{x-(17-1)}+1)$$
\\
$$\lim\limits_{x\to 17}\frac{x^{2} - 17^{2}}{\sqrt{x-(17-1)}-1}=[\frac{0}{0}]=
\lim\limits_{x\to 17}(x+17)(\sqrt{x-(17-1)}+1) = 2\cdot17 \cdot 2 = 68$$
\rozwStop
\odpStart
$68$
\odpStop
\testStart
A.$68$
B.$-68$
C.$0$
D.$17$
E.$\infty$
F.$-\infty$
G.$-17$
H.$4.25$
I.$-4.25$
\testStop
\kluczStart
A
\kluczStop



\zadStart{Zadanie z Wikieł Z 4.3 h) moja wersja nr 38}
Obliczyć granicę funkcji $f(x)=\frac{x^{2} - 19^{2}}{\sqrt{x-(19-1)}-1}$.
\zadStop
\rozwStart{Patryk Wirkus}{Szymon Tokarski}
$$\frac{x^{2} - 19^{2}}{\sqrt{x-(19-1)}-1}=\frac{(x-19)(x+19)(\sqrt{x-(19-1)}+1)}{x-(19-1)-1}=(x+19)(\sqrt{x-(19-1)}+1)$$
\\
$$\lim\limits_{x\to 19}\frac{x^{2} - 19^{2}}{\sqrt{x-(19-1)}-1}=[\frac{0}{0}]=
\lim\limits_{x\to 19}(x+19)(\sqrt{x-(19-1)}+1) = 2\cdot19 \cdot 2 = 76$$
\rozwStop
\odpStart
$76$
\odpStop
\testStart
A.$76$
B.$-76$
C.$0$
D.$19$
E.$\infty$
F.$-\infty$
G.$-19$
H.$4.75$
I.$-4.75$
\testStop
\kluczStart
A
\kluczStop



\zadStart{Zadanie z Wikieł Z 4.3 h) moja wersja nr 39}
Obliczyć granicę funkcji $f(x)=\frac{x^{2} - 23^{2}}{\sqrt{x-(23-1)}-1}$.
\zadStop
\rozwStart{Patryk Wirkus}{Szymon Tokarski}
$$\frac{x^{2} - 23^{2}}{\sqrt{x-(23-1)}-1}=\frac{(x-23)(x+23)(\sqrt{x-(23-1)}+1)}{x-(23-1)-1}=(x+23)(\sqrt{x-(23-1)}+1)$$
\\
$$\lim\limits_{x\to 23}\frac{x^{2} - 23^{2}}{\sqrt{x-(23-1)}-1}=[\frac{0}{0}]=
\lim\limits_{x\to 23}(x+23)(\sqrt{x-(23-1)}+1) = 2\cdot23 \cdot 2 = 92$$
\rozwStop
\odpStart
$92$
\odpStop
\testStart
A.$92$
B.$-92$
C.$0$
D.$23$
E.$\infty$
F.$-\infty$
G.$-23$
H.$5.75$
I.$-5.75$
\testStop
\kluczStart
A
\kluczStop



\zadStart{Zadanie z Wikieł Z 4.3 h) moja wersja nr 40}
Obliczyć granicę funkcji $f(x)=\frac{x^{2} - 29^{2}}{\sqrt{x-(29-1)}-1}$.
\zadStop
\rozwStart{Patryk Wirkus}{Szymon Tokarski}
$$\frac{x^{2} - 29^{2}}{\sqrt{x-(29-1)}-1}=\frac{(x-29)(x+29)(\sqrt{x-(29-1)}+1)}{x-(29-1)-1}=(x+29)(\sqrt{x-(29-1)}+1)$$
\\
$$\lim\limits_{x\to 29}\frac{x^{2} - 29^{2}}{\sqrt{x-(29-1)}-1}=[\frac{0}{0}]=
\lim\limits_{x\to 29}(x+29)(\sqrt{x-(29-1)}+1) = 2\cdot29 \cdot 2 = 116$$
\rozwStop
\odpStart
$116$
\odpStop
\testStart
A.$116$
B.$-116$
C.$0$
D.$29$
E.$\infty$
F.$-\infty$
G.$-29$
H.$7.25$
I.$-7.25$
\testStop
\kluczStart
A
\kluczStop



\zadStart{Zadanie z Wikieł Z 4.3 h) moja wersja nr 41}
Obliczyć granicę funkcji $f(x)=\frac{x^{2} - 31^{2}}{\sqrt{x-(31-1)}-1}$.
\zadStop
\rozwStart{Patryk Wirkus}{Szymon Tokarski}
$$\frac{x^{2} - 31^{2}}{\sqrt{x-(31-1)}-1}=\frac{(x-31)(x+31)(\sqrt{x-(31-1)}+1)}{x-(31-1)-1}=(x+31)(\sqrt{x-(31-1)}+1)$$
\\
$$\lim\limits_{x\to 31}\frac{x^{2} - 31^{2}}{\sqrt{x-(31-1)}-1}=[\frac{0}{0}]=
\lim\limits_{x\to 31}(x+31)(\sqrt{x-(31-1)}+1) = 2\cdot31 \cdot 2 = 124$$
\rozwStop
\odpStart
$124$
\odpStop
\testStart
A.$124$
B.$-124$
C.$0$
D.$31$
E.$\infty$
F.$-\infty$
G.$-31$
H.$7.75$
I.$-7.75$
\testStop
\kluczStart
A
\kluczStop



\zadStart{Zadanie z Wikieł Z 4.3 h) moja wersja nr 42}
Obliczyć granicę funkcji $f(x)=\frac{x^{2} - 37^{2}}{\sqrt{x-(37-1)}-1}$.
\zadStop
\rozwStart{Patryk Wirkus}{Szymon Tokarski}
$$\frac{x^{2} - 37^{2}}{\sqrt{x-(37-1)}-1}=\frac{(x-37)(x+37)(\sqrt{x-(37-1)}+1)}{x-(37-1)-1}=(x+37)(\sqrt{x-(37-1)}+1)$$
\\
$$\lim\limits_{x\to 37}\frac{x^{2} - 37^{2}}{\sqrt{x-(37-1)}-1}=[\frac{0}{0}]=
\lim\limits_{x\to 37}(x+37)(\sqrt{x-(37-1)}+1) = 2\cdot37 \cdot 2 = 148$$
\rozwStop
\odpStart
$148$
\odpStop
\testStart
A.$148$
B.$-148$
C.$0$
D.$37$
E.$\infty$
F.$-\infty$
G.$-37$
H.$9.25$
I.$-9.25$
\testStop
\kluczStart
A
\kluczStop



\zadStart{Zadanie z Wikieł Z 4.3 h) moja wersja nr 43}
Obliczyć granicę funkcji $f(x)=\frac{x^{2} - 41^{2}}{\sqrt{x-(41-1)}-1}$.
\zadStop
\rozwStart{Patryk Wirkus}{Szymon Tokarski}
$$\frac{x^{2} - 41^{2}}{\sqrt{x-(41-1)}-1}=\frac{(x-41)(x+41)(\sqrt{x-(41-1)}+1)}{x-(41-1)-1}=(x+41)(\sqrt{x-(41-1)}+1)$$
\\
$$\lim\limits_{x\to 41}\frac{x^{2} - 41^{2}}{\sqrt{x-(41-1)}-1}=[\frac{0}{0}]=
\lim\limits_{x\to 41}(x+41)(\sqrt{x-(41-1)}+1) = 2\cdot41 \cdot 2 = 164$$
\rozwStop
\odpStart
$164$
\odpStop
\testStart
A.$164$
B.$-164$
C.$0$
D.$41$
E.$\infty$
F.$-\infty$
G.$-41$
H.$10.25$
I.$-10.25$
\testStop
\kluczStart
A
\kluczStop



\zadStart{Zadanie z Wikieł Z 4.3 h) moja wersja nr 44}
Obliczyć granicę funkcji $f(x)=\frac{x^{2} - 43^{2}}{\sqrt{x-(43-1)}-1}$.
\zadStop
\rozwStart{Patryk Wirkus}{Szymon Tokarski}
$$\frac{x^{2} - 43^{2}}{\sqrt{x-(43-1)}-1}=\frac{(x-43)(x+43)(\sqrt{x-(43-1)}+1)}{x-(43-1)-1}=(x+43)(\sqrt{x-(43-1)}+1)$$
\\
$$\lim\limits_{x\to 43}\frac{x^{2} - 43^{2}}{\sqrt{x-(43-1)}-1}=[\frac{0}{0}]=
\lim\limits_{x\to 43}(x+43)(\sqrt{x-(43-1)}+1) = 2\cdot43 \cdot 2 = 172$$
\rozwStop
\odpStart
$172$
\odpStop
\testStart
A.$172$
B.$-172$
C.$0$
D.$43$
E.$\infty$
F.$-\infty$
G.$-43$
H.$10.75$
I.$-10.75$
\testStop
\kluczStart
A
\kluczStop



\zadStart{Zadanie z Wikieł Z 4.3 h) moja wersja nr 45}
Obliczyć granicę funkcji $f(x)=\frac{x^{2} - 47^{2}}{\sqrt{x-(47-1)}-1}$.
\zadStop
\rozwStart{Patryk Wirkus}{Szymon Tokarski}
$$\frac{x^{2} - 47^{2}}{\sqrt{x-(47-1)}-1}=\frac{(x-47)(x+47)(\sqrt{x-(47-1)}+1)}{x-(47-1)-1}=(x+47)(\sqrt{x-(47-1)}+1)$$
\\
$$\lim\limits_{x\to 47}\frac{x^{2} - 47^{2}}{\sqrt{x-(47-1)}-1}=[\frac{0}{0}]=
\lim\limits_{x\to 47}(x+47)(\sqrt{x-(47-1)}+1) = 2\cdot47 \cdot 2 = 188$$
\rozwStop
\odpStart
$188$
\odpStop
\testStart
A.$188$
B.$-188$
C.$0$
D.$47$
E.$\infty$
F.$-\infty$
G.$-47$
H.$11.75$
I.$-11.75$
\testStop
\kluczStart
A
\kluczStop



\zadStart{Zadanie z Wikieł Z 4.3 h) moja wersja nr 46}
Obliczyć granicę funkcji $f(x)=\frac{x^{2} - 53^{2}}{\sqrt{x-(53-1)}-1}$.
\zadStop
\rozwStart{Patryk Wirkus}{Szymon Tokarski}
$$\frac{x^{2} - 53^{2}}{\sqrt{x-(53-1)}-1}=\frac{(x-53)(x+53)(\sqrt{x-(53-1)}+1)}{x-(53-1)-1}=(x+53)(\sqrt{x-(53-1)}+1)$$
\\
$$\lim\limits_{x\to 53}\frac{x^{2} - 53^{2}}{\sqrt{x-(53-1)}-1}=[\frac{0}{0}]=
\lim\limits_{x\to 53}(x+53)(\sqrt{x-(53-1)}+1) = 2\cdot53 \cdot 2 = 212$$
\rozwStop
\odpStart
$212$
\odpStop
\testStart
A.$212$
B.$-212$
C.$0$
D.$53$
E.$\infty$
F.$-\infty$
G.$-53$
H.$13.25$
I.$-13.25$
\testStop
\kluczStart
A
\kluczStop



\zadStart{Zadanie z Wikieł Z 4.3 h) moja wersja nr 47}
Obliczyć granicę funkcji $f(x)=\frac{x^{2} - 59^{2}}{\sqrt{x-(59-1)}-1}$.
\zadStop
\rozwStart{Patryk Wirkus}{Szymon Tokarski}
$$\frac{x^{2} - 59^{2}}{\sqrt{x-(59-1)}-1}=\frac{(x-59)(x+59)(\sqrt{x-(59-1)}+1)}{x-(59-1)-1}=(x+59)(\sqrt{x-(59-1)}+1)$$
\\
$$\lim\limits_{x\to 59}\frac{x^{2} - 59^{2}}{\sqrt{x-(59-1)}-1}=[\frac{0}{0}]=
\lim\limits_{x\to 59}(x+59)(\sqrt{x-(59-1)}+1) = 2\cdot59 \cdot 2 = 236$$
\rozwStop
\odpStart
$236$
\odpStop
\testStart
A.$236$
B.$-236$
C.$0$
D.$59$
E.$\infty$
F.$-\infty$
G.$-59$
H.$14.75$
I.$-14.75$
\testStop
\kluczStart
A
\kluczStop



\zadStart{Zadanie z Wikieł Z 4.3 h) moja wersja nr 48}
Obliczyć granicę funkcji $f(x)=\frac{x^{2} - 61^{2}}{\sqrt{x-(61-1)}-1}$.
\zadStop
\rozwStart{Patryk Wirkus}{Szymon Tokarski}
$$\frac{x^{2} - 61^{2}}{\sqrt{x-(61-1)}-1}=\frac{(x-61)(x+61)(\sqrt{x-(61-1)}+1)}{x-(61-1)-1}=(x+61)(\sqrt{x-(61-1)}+1)$$
\\
$$\lim\limits_{x\to 61}\frac{x^{2} - 61^{2}}{\sqrt{x-(61-1)}-1}=[\frac{0}{0}]=
\lim\limits_{x\to 61}(x+61)(\sqrt{x-(61-1)}+1) = 2\cdot61 \cdot 2 = 244$$
\rozwStop
\odpStart
$244$
\odpStop
\testStart
A.$244$
B.$-244$
C.$0$
D.$61$
E.$\infty$
F.$-\infty$
G.$-61$
H.$15.25$
I.$-15.25$
\testStop
\kluczStart
A
\kluczStop



\zadStart{Zadanie z Wikieł Z 4.3 h) moja wersja nr 49}
Obliczyć granicę funkcji $f(x)=\frac{x^{2} - 67^{2}}{\sqrt{x-(67-1)}-1}$.
\zadStop
\rozwStart{Patryk Wirkus}{Szymon Tokarski}
$$\frac{x^{2} - 67^{2}}{\sqrt{x-(67-1)}-1}=\frac{(x-67)(x+67)(\sqrt{x-(67-1)}+1)}{x-(67-1)-1}=(x+67)(\sqrt{x-(67-1)}+1)$$
\\
$$\lim\limits_{x\to 67}\frac{x^{2} - 67^{2}}{\sqrt{x-(67-1)}-1}=[\frac{0}{0}]=
\lim\limits_{x\to 67}(x+67)(\sqrt{x-(67-1)}+1) = 2\cdot67 \cdot 2 = 268$$
\rozwStop
\odpStart
$268$
\odpStop
\testStart
A.$268$
B.$-268$
C.$0$
D.$67$
E.$\infty$
F.$-\infty$
G.$-67$
H.$16.75$
I.$-16.75$
\testStop
\kluczStart
A
\kluczStop



\zadStart{Zadanie z Wikieł Z 4.3 h) moja wersja nr 50}
Obliczyć granicę funkcji $f(x)=\frac{x^{2} - 71^{2}}{\sqrt{x-(71-1)}-1}$.
\zadStop
\rozwStart{Patryk Wirkus}{Szymon Tokarski}
$$\frac{x^{2} - 71^{2}}{\sqrt{x-(71-1)}-1}=\frac{(x-71)(x+71)(\sqrt{x-(71-1)}+1)}{x-(71-1)-1}=(x+71)(\sqrt{x-(71-1)}+1)$$
\\
$$\lim\limits_{x\to 71}\frac{x^{2} - 71^{2}}{\sqrt{x-(71-1)}-1}=[\frac{0}{0}]=
\lim\limits_{x\to 71}(x+71)(\sqrt{x-(71-1)}+1) = 2\cdot71 \cdot 2 = 284$$
\rozwStop
\odpStart
$284$
\odpStop
\testStart
A.$284$
B.$-284$
C.$0$
D.$71$
E.$\infty$
F.$-\infty$
G.$-71$
H.$17.75$
I.$-17.75$
\testStop
\kluczStart
A
\kluczStop



\zadStart{Zadanie z Wikieł Z 4.3 h) moja wersja nr 51}
Obliczyć granicę funkcji $f(x)=\frac{x^{2} - 73^{2}}{\sqrt{x-(73-1)}-1}$.
\zadStop
\rozwStart{Patryk Wirkus}{Szymon Tokarski}
$$\frac{x^{2} - 73^{2}}{\sqrt{x-(73-1)}-1}=\frac{(x-73)(x+73)(\sqrt{x-(73-1)}+1)}{x-(73-1)-1}=(x+73)(\sqrt{x-(73-1)}+1)$$
\\
$$\lim\limits_{x\to 73}\frac{x^{2} - 73^{2}}{\sqrt{x-(73-1)}-1}=[\frac{0}{0}]=
\lim\limits_{x\to 73}(x+73)(\sqrt{x-(73-1)}+1) = 2\cdot73 \cdot 2 = 292$$
\rozwStop
\odpStart
$292$
\odpStop
\testStart
A.$292$
B.$-292$
C.$0$
D.$73$
E.$\infty$
F.$-\infty$
G.$-73$
H.$18.25$
I.$-18.25$
\testStop
\kluczStart
A
\kluczStop



\zadStart{Zadanie z Wikieł Z 4.3 h) moja wersja nr 52}
Obliczyć granicę funkcji $f(x)=\frac{x^{2} - 79^{2}}{\sqrt{x-(79-1)}-1}$.
\zadStop
\rozwStart{Patryk Wirkus}{Szymon Tokarski}
$$\frac{x^{2} - 79^{2}}{\sqrt{x-(79-1)}-1}=\frac{(x-79)(x+79)(\sqrt{x-(79-1)}+1)}{x-(79-1)-1}=(x+79)(\sqrt{x-(79-1)}+1)$$
\\
$$\lim\limits_{x\to 79}\frac{x^{2} - 79^{2}}{\sqrt{x-(79-1)}-1}=[\frac{0}{0}]=
\lim\limits_{x\to 79}(x+79)(\sqrt{x-(79-1)}+1) = 2\cdot79 \cdot 2 = 316$$
\rozwStop
\odpStart
$316$
\odpStop
\testStart
A.$316$
B.$-316$
C.$0$
D.$79$
E.$\infty$
F.$-\infty$
G.$-79$
H.$19.75$
I.$-19.75$
\testStop
\kluczStart
A
\kluczStop



\zadStart{Zadanie z Wikieł Z 4.3 h) moja wersja nr 53}
Obliczyć granicę funkcji $f(x)=\frac{x^{2} - 83^{2}}{\sqrt{x-(83-1)}-1}$.
\zadStop
\rozwStart{Patryk Wirkus}{Szymon Tokarski}
$$\frac{x^{2} - 83^{2}}{\sqrt{x-(83-1)}-1}=\frac{(x-83)(x+83)(\sqrt{x-(83-1)}+1)}{x-(83-1)-1}=(x+83)(\sqrt{x-(83-1)}+1)$$
\\
$$\lim\limits_{x\to 83}\frac{x^{2} - 83^{2}}{\sqrt{x-(83-1)}-1}=[\frac{0}{0}]=
\lim\limits_{x\to 83}(x+83)(\sqrt{x-(83-1)}+1) = 2\cdot83 \cdot 2 = 332$$
\rozwStop
\odpStart
$332$
\odpStop
\testStart
A.$332$
B.$-332$
C.$0$
D.$83$
E.$\infty$
F.$-\infty$
G.$-83$
H.$20.75$
I.$-20.75$
\testStop
\kluczStart
A
\kluczStop



\zadStart{Zadanie z Wikieł Z 4.3 h) moja wersja nr 54}
Obliczyć granicę funkcji $f(x)=\frac{x^{2} - 89^{2}}{\sqrt{x-(89-1)}-1}$.
\zadStop
\rozwStart{Patryk Wirkus}{Szymon Tokarski}
$$\frac{x^{2} - 89^{2}}{\sqrt{x-(89-1)}-1}=\frac{(x-89)(x+89)(\sqrt{x-(89-1)}+1)}{x-(89-1)-1}=(x+89)(\sqrt{x-(89-1)}+1)$$
\\
$$\lim\limits_{x\to 89}\frac{x^{2} - 89^{2}}{\sqrt{x-(89-1)}-1}=[\frac{0}{0}]=
\lim\limits_{x\to 89}(x+89)(\sqrt{x-(89-1)}+1) = 2\cdot89 \cdot 2 = 356$$
\rozwStop
\odpStart
$356$
\odpStop
\testStart
A.$356$
B.$-356$
C.$0$
D.$89$
E.$\infty$
F.$-\infty$
G.$-89$
H.$22.25$
I.$-22.25$
\testStop
\kluczStart
A
\kluczStop



\zadStart{Zadanie z Wikieł Z 4.3 h) moja wersja nr 55}
Obliczyć granicę funkcji $f(x)=\frac{x^{2} - 97^{2}}{\sqrt{x-(97-1)}-1}$.
\zadStop
\rozwStart{Patryk Wirkus}{Szymon Tokarski}
$$\frac{x^{2} - 97^{2}}{\sqrt{x-(97-1)}-1}=\frac{(x-97)(x+97)(\sqrt{x-(97-1)}+1)}{x-(97-1)-1}=(x+97)(\sqrt{x-(97-1)}+1)$$
\\
$$\lim\limits_{x\to 97}\frac{x^{2} - 97^{2}}{\sqrt{x-(97-1)}-1}=[\frac{0}{0}]=
\lim\limits_{x\to 97}(x+97)(\sqrt{x-(97-1)}+1) = 2\cdot97 \cdot 2 = 388$$
\rozwStop
\odpStart
$388$
\odpStop
\testStart
A.$388$
B.$-388$
C.$0$
D.$97$
E.$\infty$
F.$-\infty$
G.$-97$
H.$24.25$
I.$-24.25$
\testStop
\kluczStart
A
\kluczStop



\zadStart{Zadanie z Wikieł Z 4.3 h) moja wersja nr 56}
Obliczyć granicę funkcji $f(x)=\frac{x^{2} - 101^{2}}{\sqrt{x-(101-1)}-1}$.
\zadStop
\rozwStart{Patryk Wirkus}{Szymon Tokarski}
$$\frac{x^{2} - 101^{2}}{\sqrt{x-(101-1)}-1}=\frac{(x-101)(x+101)(\sqrt{x-(101-1)}+1)}{x-(101-1)-1}=(x+101)(\sqrt{x-(101-1)}+1)$$
\\
$$\lim\limits_{x\to 101}\frac{x^{2} - 101^{2}}{\sqrt{x-(101-1)}-1}=[\frac{0}{0}]=
\lim\limits_{x\to 101}(x+101)(\sqrt{x-(101-1)}+1) = 2\cdot101 \cdot 2 = 404$$
\rozwStop
\odpStart
$404$
\odpStop
\testStart
A.$404$
B.$-404$
C.$0$
D.$101$
E.$\infty$
F.$-\infty$
G.$-101$
H.$25.25$
I.$-25.25$
\testStop
\kluczStart
A
\kluczStop



\zadStart{Zadanie z Wikieł Z 4.3 h) moja wersja nr 57}
Obliczyć granicę funkcji $f(x)=\frac{x^{2} - 103^{2}}{\sqrt{x-(103-1)}-1}$.
\zadStop
\rozwStart{Patryk Wirkus}{Szymon Tokarski}
$$\frac{x^{2} - 103^{2}}{\sqrt{x-(103-1)}-1}=\frac{(x-103)(x+103)(\sqrt{x-(103-1)}+1)}{x-(103-1)-1}=(x+103)(\sqrt{x-(103-1)}+1)$$
\\
$$\lim\limits_{x\to 103}\frac{x^{2} - 103^{2}}{\sqrt{x-(103-1)}-1}=[\frac{0}{0}]=
\lim\limits_{x\to 103}(x+103)(\sqrt{x-(103-1)}+1) = 2\cdot103 \cdot 2 = 412$$
\rozwStop
\odpStart
$412$
\odpStop
\testStart
A.$412$
B.$-412$
C.$0$
D.$103$
E.$\infty$
F.$-\infty$
G.$-103$
H.$25.75$
I.$-25.75$
\testStop
\kluczStart
A
\kluczStop



\zadStart{Zadanie z Wikieł Z 4.3 h) moja wersja nr 58}
Obliczyć granicę funkcji $f(x)=\frac{x^{2} - 107^{2}}{\sqrt{x-(107-1)}-1}$.
\zadStop
\rozwStart{Patryk Wirkus}{Szymon Tokarski}
$$\frac{x^{2} - 107^{2}}{\sqrt{x-(107-1)}-1}=\frac{(x-107)(x+107)(\sqrt{x-(107-1)}+1)}{x-(107-1)-1}=(x+107)(\sqrt{x-(107-1)}+1)$$
\\
$$\lim\limits_{x\to 107}\frac{x^{2} - 107^{2}}{\sqrt{x-(107-1)}-1}=[\frac{0}{0}]=
\lim\limits_{x\to 107}(x+107)(\sqrt{x-(107-1)}+1) = 2\cdot107 \cdot 2 = 428$$
\rozwStop
\odpStart
$428$
\odpStop
\testStart
A.$428$
B.$-428$
C.$0$
D.$107$
E.$\infty$
F.$-\infty$
G.$-107$
H.$26.75$
I.$-26.75$
\testStop
\kluczStart
A
\kluczStop



\zadStart{Zadanie z Wikieł Z 4.3 h) moja wersja nr 59}
Obliczyć granicę funkcji $f(x)=\frac{x^{2} - 109^{2}}{\sqrt{x-(109-1)}-1}$.
\zadStop
\rozwStart{Patryk Wirkus}{Szymon Tokarski}
$$\frac{x^{2} - 109^{2}}{\sqrt{x-(109-1)}-1}=\frac{(x-109)(x+109)(\sqrt{x-(109-1)}+1)}{x-(109-1)-1}=(x+109)(\sqrt{x-(109-1)}+1)$$
\\
$$\lim\limits_{x\to 109}\frac{x^{2} - 109^{2}}{\sqrt{x-(109-1)}-1}=[\frac{0}{0}]=
\lim\limits_{x\to 109}(x+109)(\sqrt{x-(109-1)}+1) = 2\cdot109 \cdot 2 = 436$$
\rozwStop
\odpStart
$436$
\odpStop
\testStart
A.$436$
B.$-436$
C.$0$
D.$109$
E.$\infty$
F.$-\infty$
G.$-109$
H.$27.25$
I.$-27.25$
\testStop
\kluczStart
A
\kluczStop



\zadStart{Zadanie z Wikieł Z 4.3 h) moja wersja nr 60}
Obliczyć granicę funkcji $f(x)=\frac{x^{2} - 113^{2}}{\sqrt{x-(113-1)}-1}$.
\zadStop
\rozwStart{Patryk Wirkus}{Szymon Tokarski}
$$\frac{x^{2} - 113^{2}}{\sqrt{x-(113-1)}-1}=\frac{(x-113)(x+113)(\sqrt{x-(113-1)}+1)}{x-(113-1)-1}=(x+113)(\sqrt{x-(113-1)}+1)$$
\\
$$\lim\limits_{x\to 113}\frac{x^{2} - 113^{2}}{\sqrt{x-(113-1)}-1}=[\frac{0}{0}]=
\lim\limits_{x\to 113}(x+113)(\sqrt{x-(113-1)}+1) = 2\cdot113 \cdot 2 = 452$$
\rozwStop
\odpStart
$452$
\odpStop
\testStart
A.$452$
B.$-452$
C.$0$
D.$113$
E.$\infty$
F.$-\infty$
G.$-113$
H.$28.25$
I.$-28.25$
\testStop
\kluczStart
A
\kluczStop



\zadStart{Zadanie z Wikieł Z 4.3 h) moja wersja nr 61}
Obliczyć granicę funkcji $f(x)=\frac{x^{2} - 127^{2}}{\sqrt{x-(127-1)}-1}$.
\zadStop
\rozwStart{Patryk Wirkus}{Szymon Tokarski}
$$\frac{x^{2} - 127^{2}}{\sqrt{x-(127-1)}-1}=\frac{(x-127)(x+127)(\sqrt{x-(127-1)}+1)}{x-(127-1)-1}=(x+127)(\sqrt{x-(127-1)}+1)$$
\\
$$\lim\limits_{x\to 127}\frac{x^{2} - 127^{2}}{\sqrt{x-(127-1)}-1}=[\frac{0}{0}]=
\lim\limits_{x\to 127}(x+127)(\sqrt{x-(127-1)}+1) = 2\cdot127 \cdot 2 = 508$$
\rozwStop
\odpStart
$508$
\odpStop
\testStart
A.$508$
B.$-508$
C.$0$
D.$127$
E.$\infty$
F.$-\infty$
G.$-127$
H.$31.75$
I.$-31.75$
\testStop
\kluczStart
A
\kluczStop



\zadStart{Zadanie z Wikieł Z 4.3 h) moja wersja nr 62}
Obliczyć granicę funkcji $f(x)=\frac{x^{2} - 131^{2}}{\sqrt{x-(131-1)}-1}$.
\zadStop
\rozwStart{Patryk Wirkus}{Szymon Tokarski}
$$\frac{x^{2} - 131^{2}}{\sqrt{x-(131-1)}-1}=\frac{(x-131)(x+131)(\sqrt{x-(131-1)}+1)}{x-(131-1)-1}=(x+131)(\sqrt{x-(131-1)}+1)$$
\\
$$\lim\limits_{x\to 131}\frac{x^{2} - 131^{2}}{\sqrt{x-(131-1)}-1}=[\frac{0}{0}]=
\lim\limits_{x\to 131}(x+131)(\sqrt{x-(131-1)}+1) = 2\cdot131 \cdot 2 = 524$$
\rozwStop
\odpStart
$524$
\odpStop
\testStart
A.$524$
B.$-524$
C.$0$
D.$131$
E.$\infty$
F.$-\infty$
G.$-131$
H.$32.75$
I.$-32.75$
\testStop
\kluczStart
A
\kluczStop



\zadStart{Zadanie z Wikieł Z 4.3 h) moja wersja nr 63}
Obliczyć granicę funkcji $f(x)=\frac{x^{2} - 137^{2}}{\sqrt{x-(137-1)}-1}$.
\zadStop
\rozwStart{Patryk Wirkus}{Szymon Tokarski}
$$\frac{x^{2} - 137^{2}}{\sqrt{x-(137-1)}-1}=\frac{(x-137)(x+137)(\sqrt{x-(137-1)}+1)}{x-(137-1)-1}=(x+137)(\sqrt{x-(137-1)}+1)$$
\\
$$\lim\limits_{x\to 137}\frac{x^{2} - 137^{2}}{\sqrt{x-(137-1)}-1}=[\frac{0}{0}]=
\lim\limits_{x\to 137}(x+137)(\sqrt{x-(137-1)}+1) = 2\cdot137 \cdot 2 = 548$$
\rozwStop
\odpStart
$548$
\odpStop
\testStart
A.$548$
B.$-548$
C.$0$
D.$137$
E.$\infty$
F.$-\infty$
G.$-137$
H.$34.25$
I.$-34.25$
\testStop
\kluczStart
A
\kluczStop



\zadStart{Zadanie z Wikieł Z 4.3 h) moja wersja nr 64}
Obliczyć granicę funkcji $f(x)=\frac{x^{2} - 139^{2}}{\sqrt{x-(139-1)}-1}$.
\zadStop
\rozwStart{Patryk Wirkus}{Szymon Tokarski}
$$\frac{x^{2} - 139^{2}}{\sqrt{x-(139-1)}-1}=\frac{(x-139)(x+139)(\sqrt{x-(139-1)}+1)}{x-(139-1)-1}=(x+139)(\sqrt{x-(139-1)}+1)$$
\\
$$\lim\limits_{x\to 139}\frac{x^{2} - 139^{2}}{\sqrt{x-(139-1)}-1}=[\frac{0}{0}]=
\lim\limits_{x\to 139}(x+139)(\sqrt{x-(139-1)}+1) = 2\cdot139 \cdot 2 = 556$$
\rozwStop
\odpStart
$556$
\odpStop
\testStart
A.$556$
B.$-556$
C.$0$
D.$139$
E.$\infty$
F.$-\infty$
G.$-139$
H.$34.75$
I.$-34.75$
\testStop
\kluczStart
A
\kluczStop



\zadStart{Zadanie z Wikieł Z 4.3 h) moja wersja nr 65}
Obliczyć granicę funkcji $f(x)=\frac{x^{2} - 149^{2}}{\sqrt{x-(149-1)}-1}$.
\zadStop
\rozwStart{Patryk Wirkus}{Szymon Tokarski}
$$\frac{x^{2} - 149^{2}}{\sqrt{x-(149-1)}-1}=\frac{(x-149)(x+149)(\sqrt{x-(149-1)}+1)}{x-(149-1)-1}=(x+149)(\sqrt{x-(149-1)}+1)$$
\\
$$\lim\limits_{x\to 149}\frac{x^{2} - 149^{2}}{\sqrt{x-(149-1)}-1}=[\frac{0}{0}]=
\lim\limits_{x\to 149}(x+149)(\sqrt{x-(149-1)}+1) = 2\cdot149 \cdot 2 = 596$$
\rozwStop
\odpStart
$596$
\odpStop
\testStart
A.$596$
B.$-596$
C.$0$
D.$149$
E.$\infty$
F.$-\infty$
G.$-149$
H.$37.25$
I.$-37.25$
\testStop
\kluczStart
A
\kluczStop



\zadStart{Zadanie z Wikieł Z 4.3 h) moja wersja nr 66}
Obliczyć granicę funkcji $f(x)=\frac{x^{2} - 151^{2}}{\sqrt{x-(151-1)}-1}$.
\zadStop
\rozwStart{Patryk Wirkus}{Szymon Tokarski}
$$\frac{x^{2} - 151^{2}}{\sqrt{x-(151-1)}-1}=\frac{(x-151)(x+151)(\sqrt{x-(151-1)}+1)}{x-(151-1)-1}=(x+151)(\sqrt{x-(151-1)}+1)$$
\\
$$\lim\limits_{x\to 151}\frac{x^{2} - 151^{2}}{\sqrt{x-(151-1)}-1}=[\frac{0}{0}]=
\lim\limits_{x\to 151}(x+151)(\sqrt{x-(151-1)}+1) = 2\cdot151 \cdot 2 = 604$$
\rozwStop
\odpStart
$604$
\odpStop
\testStart
A.$604$
B.$-604$
C.$0$
D.$151$
E.$\infty$
F.$-\infty$
G.$-151$
H.$37.75$
I.$-37.75$
\testStop
\kluczStart
A
\kluczStop



\zadStart{Zadanie z Wikieł Z 4.3 h) moja wersja nr 67}
Obliczyć granicę funkcji $f(x)=\frac{x^{2} - 157^{2}}{\sqrt{x-(157-1)}-1}$.
\zadStop
\rozwStart{Patryk Wirkus}{Szymon Tokarski}
$$\frac{x^{2} - 157^{2}}{\sqrt{x-(157-1)}-1}=\frac{(x-157)(x+157)(\sqrt{x-(157-1)}+1)}{x-(157-1)-1}=(x+157)(\sqrt{x-(157-1)}+1)$$
\\
$$\lim\limits_{x\to 157}\frac{x^{2} - 157^{2}}{\sqrt{x-(157-1)}-1}=[\frac{0}{0}]=
\lim\limits_{x\to 157}(x+157)(\sqrt{x-(157-1)}+1) = 2\cdot157 \cdot 2 = 628$$
\rozwStop
\odpStart
$628$
\odpStop
\testStart
A.$628$
B.$-628$
C.$0$
D.$157$
E.$\infty$
F.$-\infty$
G.$-157$
H.$39.25$
I.$-39.25$
\testStop
\kluczStart
A
\kluczStop



\zadStart{Zadanie z Wikieł Z 4.3 h) moja wersja nr 68}
Obliczyć granicę funkcji $f(x)=\frac{x^{2} - 163^{2}}{\sqrt{x-(163-1)}-1}$.
\zadStop
\rozwStart{Patryk Wirkus}{Szymon Tokarski}
$$\frac{x^{2} - 163^{2}}{\sqrt{x-(163-1)}-1}=\frac{(x-163)(x+163)(\sqrt{x-(163-1)}+1)}{x-(163-1)-1}=(x+163)(\sqrt{x-(163-1)}+1)$$
\\
$$\lim\limits_{x\to 163}\frac{x^{2} - 163^{2}}{\sqrt{x-(163-1)}-1}=[\frac{0}{0}]=
\lim\limits_{x\to 163}(x+163)(\sqrt{x-(163-1)}+1) = 2\cdot163 \cdot 2 = 652$$
\rozwStop
\odpStart
$652$
\odpStop
\testStart
A.$652$
B.$-652$
C.$0$
D.$163$
E.$\infty$
F.$-\infty$
G.$-163$
H.$40.75$
I.$-40.75$
\testStop
\kluczStart
A
\kluczStop



\zadStart{Zadanie z Wikieł Z 4.3 h) moja wersja nr 69}
Obliczyć granicę funkcji $f(x)=\frac{x^{2} - 167^{2}}{\sqrt{x-(167-1)}-1}$.
\zadStop
\rozwStart{Patryk Wirkus}{Szymon Tokarski}
$$\frac{x^{2} - 167^{2}}{\sqrt{x-(167-1)}-1}=\frac{(x-167)(x+167)(\sqrt{x-(167-1)}+1)}{x-(167-1)-1}=(x+167)(\sqrt{x-(167-1)}+1)$$
\\
$$\lim\limits_{x\to 167}\frac{x^{2} - 167^{2}}{\sqrt{x-(167-1)}-1}=[\frac{0}{0}]=
\lim\limits_{x\to 167}(x+167)(\sqrt{x-(167-1)}+1) = 2\cdot167 \cdot 2 = 668$$
\rozwStop
\odpStart
$668$
\odpStop
\testStart
A.$668$
B.$-668$
C.$0$
D.$167$
E.$\infty$
F.$-\infty$
G.$-167$
H.$41.75$
I.$-41.75$
\testStop
\kluczStart
A
\kluczStop



\zadStart{Zadanie z Wikieł Z 4.3 h) moja wersja nr 70}
Obliczyć granicę funkcji $f(x)=\frac{x^{2} - 173^{2}}{\sqrt{x-(173-1)}-1}$.
\zadStop
\rozwStart{Patryk Wirkus}{Szymon Tokarski}
$$\frac{x^{2} - 173^{2}}{\sqrt{x-(173-1)}-1}=\frac{(x-173)(x+173)(\sqrt{x-(173-1)}+1)}{x-(173-1)-1}=(x+173)(\sqrt{x-(173-1)}+1)$$
\\
$$\lim\limits_{x\to 173}\frac{x^{2} - 173^{2}}{\sqrt{x-(173-1)}-1}=[\frac{0}{0}]=
\lim\limits_{x\to 173}(x+173)(\sqrt{x-(173-1)}+1) = 2\cdot173 \cdot 2 = 692$$
\rozwStop
\odpStart
$692$
\odpStop
\testStart
A.$692$
B.$-692$
C.$0$
D.$173$
E.$\infty$
F.$-\infty$
G.$-173$
H.$43.25$
I.$-43.25$
\testStop
\kluczStart
A
\kluczStop



\zadStart{Zadanie z Wikieł Z 4.3 h) moja wersja nr 71}
Obliczyć granicę funkcji $f(x)=\frac{x^{2} - 179^{2}}{\sqrt{x-(179-1)}-1}$.
\zadStop
\rozwStart{Patryk Wirkus}{Szymon Tokarski}
$$\frac{x^{2} - 179^{2}}{\sqrt{x-(179-1)}-1}=\frac{(x-179)(x+179)(\sqrt{x-(179-1)}+1)}{x-(179-1)-1}=(x+179)(\sqrt{x-(179-1)}+1)$$
\\
$$\lim\limits_{x\to 179}\frac{x^{2} - 179^{2}}{\sqrt{x-(179-1)}-1}=[\frac{0}{0}]=
\lim\limits_{x\to 179}(x+179)(\sqrt{x-(179-1)}+1) = 2\cdot179 \cdot 2 = 716$$
\rozwStop
\odpStart
$716$
\odpStop
\testStart
A.$716$
B.$-716$
C.$0$
D.$179$
E.$\infty$
F.$-\infty$
G.$-179$
H.$44.75$
I.$-44.75$
\testStop
\kluczStart
A
\kluczStop



\zadStart{Zadanie z Wikieł Z 4.3 h) moja wersja nr 72}
Obliczyć granicę funkcji $f(x)=\frac{x^{2} - 181^{2}}{\sqrt{x-(181-1)}-1}$.
\zadStop
\rozwStart{Patryk Wirkus}{Szymon Tokarski}
$$\frac{x^{2} - 181^{2}}{\sqrt{x-(181-1)}-1}=\frac{(x-181)(x+181)(\sqrt{x-(181-1)}+1)}{x-(181-1)-1}=(x+181)(\sqrt{x-(181-1)}+1)$$
\\
$$\lim\limits_{x\to 181}\frac{x^{2} - 181^{2}}{\sqrt{x-(181-1)}-1}=[\frac{0}{0}]=
\lim\limits_{x\to 181}(x+181)(\sqrt{x-(181-1)}+1) = 2\cdot181 \cdot 2 = 724$$
\rozwStop
\odpStart
$724$
\odpStop
\testStart
A.$724$
B.$-724$
C.$0$
D.$181$
E.$\infty$
F.$-\infty$
G.$-181$
H.$45.25$
I.$-45.25$
\testStop
\kluczStart
A
\kluczStop



\zadStart{Zadanie z Wikieł Z 4.3 h) moja wersja nr 73}
Obliczyć granicę funkcji $f(x)=\frac{x^{2} - 191^{2}}{\sqrt{x-(191-1)}-1}$.
\zadStop
\rozwStart{Patryk Wirkus}{Szymon Tokarski}
$$\frac{x^{2} - 191^{2}}{\sqrt{x-(191-1)}-1}=\frac{(x-191)(x+191)(\sqrt{x-(191-1)}+1)}{x-(191-1)-1}=(x+191)(\sqrt{x-(191-1)}+1)$$
\\
$$\lim\limits_{x\to 191}\frac{x^{2} - 191^{2}}{\sqrt{x-(191-1)}-1}=[\frac{0}{0}]=
\lim\limits_{x\to 191}(x+191)(\sqrt{x-(191-1)}+1) = 2\cdot191 \cdot 2 = 764$$
\rozwStop
\odpStart
$764$
\odpStop
\testStart
A.$764$
B.$-764$
C.$0$
D.$191$
E.$\infty$
F.$-\infty$
G.$-191$
H.$47.75$
I.$-47.75$
\testStop
\kluczStart
A
\kluczStop



\zadStart{Zadanie z Wikieł Z 4.3 h) moja wersja nr 74}
Obliczyć granicę funkcji $f(x)=\frac{x^{2} - 193^{2}}{\sqrt{x-(193-1)}-1}$.
\zadStop
\rozwStart{Patryk Wirkus}{Szymon Tokarski}
$$\frac{x^{2} - 193^{2}}{\sqrt{x-(193-1)}-1}=\frac{(x-193)(x+193)(\sqrt{x-(193-1)}+1)}{x-(193-1)-1}=(x+193)(\sqrt{x-(193-1)}+1)$$
\\
$$\lim\limits_{x\to 193}\frac{x^{2} - 193^{2}}{\sqrt{x-(193-1)}-1}=[\frac{0}{0}]=
\lim\limits_{x\to 193}(x+193)(\sqrt{x-(193-1)}+1) = 2\cdot193 \cdot 2 = 772$$
\rozwStop
\odpStart
$772$
\odpStop
\testStart
A.$772$
B.$-772$
C.$0$
D.$193$
E.$\infty$
F.$-\infty$
G.$-193$
H.$48.25$
I.$-48.25$
\testStop
\kluczStart
A
\kluczStop



\zadStart{Zadanie z Wikieł Z 4.3 h) moja wersja nr 75}
Obliczyć granicę funkcji $f(x)=\frac{x^{2} - 197^{2}}{\sqrt{x-(197-1)}-1}$.
\zadStop
\rozwStart{Patryk Wirkus}{Szymon Tokarski}
$$\frac{x^{2} - 197^{2}}{\sqrt{x-(197-1)}-1}=\frac{(x-197)(x+197)(\sqrt{x-(197-1)}+1)}{x-(197-1)-1}=(x+197)(\sqrt{x-(197-1)}+1)$$
\\
$$\lim\limits_{x\to 197}\frac{x^{2} - 197^{2}}{\sqrt{x-(197-1)}-1}=[\frac{0}{0}]=
\lim\limits_{x\to 197}(x+197)(\sqrt{x-(197-1)}+1) = 2\cdot197 \cdot 2 = 788$$
\rozwStop
\odpStart
$788$
\odpStop
\testStart
A.$788$
B.$-788$
C.$0$
D.$197$
E.$\infty$
F.$-\infty$
G.$-197$
H.$49.25$
I.$-49.25$
\testStop
\kluczStart
A
\kluczStop



\zadStart{Zadanie z Wikieł Z 4.3 h) moja wersja nr 76}
Obliczyć granicę funkcji $f(x)=\frac{x^{2} - 199^{2}}{\sqrt{x-(199-1)}-1}$.
\zadStop
\rozwStart{Patryk Wirkus}{Szymon Tokarski}
$$\frac{x^{2} - 199^{2}}{\sqrt{x-(199-1)}-1}=\frac{(x-199)(x+199)(\sqrt{x-(199-1)}+1)}{x-(199-1)-1}=(x+199)(\sqrt{x-(199-1)}+1)$$
\\
$$\lim\limits_{x\to 199}\frac{x^{2} - 199^{2}}{\sqrt{x-(199-1)}-1}=[\frac{0}{0}]=
\lim\limits_{x\to 199}(x+199)(\sqrt{x-(199-1)}+1) = 2\cdot199 \cdot 2 = 796$$
\rozwStop
\odpStart
$796$
\odpStop
\testStart
A.$796$
B.$-796$
C.$0$
D.$199$
E.$\infty$
F.$-\infty$
G.$-199$
H.$49.75$
I.$-49.75$
\testStop
\kluczStart
A
\kluczStop



\zadStart{Zadanie z Wikieł Z 4.3 h) moja wersja nr 77}
Obliczyć granicę funkcji $f(x)=\frac{x^{2} - 211^{2}}{\sqrt{x-(211-1)}-1}$.
\zadStop
\rozwStart{Patryk Wirkus}{Szymon Tokarski}
$$\frac{x^{2} - 211^{2}}{\sqrt{x-(211-1)}-1}=\frac{(x-211)(x+211)(\sqrt{x-(211-1)}+1)}{x-(211-1)-1}=(x+211)(\sqrt{x-(211-1)}+1)$$
\\
$$\lim\limits_{x\to 211}\frac{x^{2} - 211^{2}}{\sqrt{x-(211-1)}-1}=[\frac{0}{0}]=
\lim\limits_{x\to 211}(x+211)(\sqrt{x-(211-1)}+1) = 2\cdot211 \cdot 2 = 844$$
\rozwStop
\odpStart
$844$
\odpStop
\testStart
A.$844$
B.$-844$
C.$0$
D.$211$
E.$\infty$
F.$-\infty$
G.$-211$
H.$52.75$
I.$-52.75$
\testStop
\kluczStart
A
\kluczStop



\zadStart{Zadanie z Wikieł Z 4.3 h) moja wersja nr 78}
Obliczyć granicę funkcji $f(x)=\frac{x^{2} - 223^{2}}{\sqrt{x-(223-1)}-1}$.
\zadStop
\rozwStart{Patryk Wirkus}{Szymon Tokarski}
$$\frac{x^{2} - 223^{2}}{\sqrt{x-(223-1)}-1}=\frac{(x-223)(x+223)(\sqrt{x-(223-1)}+1)}{x-(223-1)-1}=(x+223)(\sqrt{x-(223-1)}+1)$$
\\
$$\lim\limits_{x\to 223}\frac{x^{2} - 223^{2}}{\sqrt{x-(223-1)}-1}=[\frac{0}{0}]=
\lim\limits_{x\to 223}(x+223)(\sqrt{x-(223-1)}+1) = 2\cdot223 \cdot 2 = 892$$
\rozwStop
\odpStart
$892$
\odpStop
\testStart
A.$892$
B.$-892$
C.$0$
D.$223$
E.$\infty$
F.$-\infty$
G.$-223$
H.$55.75$
I.$-55.75$
\testStop
\kluczStart
A
\kluczStop



\zadStart{Zadanie z Wikieł Z 4.3 h) moja wersja nr 79}
Obliczyć granicę funkcji $f(x)=\frac{x^{2} - 227^{2}}{\sqrt{x-(227-1)}-1}$.
\zadStop
\rozwStart{Patryk Wirkus}{Szymon Tokarski}
$$\frac{x^{2} - 227^{2}}{\sqrt{x-(227-1)}-1}=\frac{(x-227)(x+227)(\sqrt{x-(227-1)}+1)}{x-(227-1)-1}=(x+227)(\sqrt{x-(227-1)}+1)$$
\\
$$\lim\limits_{x\to 227}\frac{x^{2} - 227^{2}}{\sqrt{x-(227-1)}-1}=[\frac{0}{0}]=
\lim\limits_{x\to 227}(x+227)(\sqrt{x-(227-1)}+1) = 2\cdot227 \cdot 2 = 908$$
\rozwStop
\odpStart
$908$
\odpStop
\testStart
A.$908$
B.$-908$
C.$0$
D.$227$
E.$\infty$
F.$-\infty$
G.$-227$
H.$56.75$
I.$-56.75$
\testStop
\kluczStart
A
\kluczStop



\zadStart{Zadanie z Wikieł Z 4.3 h) moja wersja nr 80}
Obliczyć granicę funkcji $f(x)=\frac{x^{2} - 229^{2}}{\sqrt{x-(229-1)}-1}$.
\zadStop
\rozwStart{Patryk Wirkus}{Szymon Tokarski}
$$\frac{x^{2} - 229^{2}}{\sqrt{x-(229-1)}-1}=\frac{(x-229)(x+229)(\sqrt{x-(229-1)}+1)}{x-(229-1)-1}=(x+229)(\sqrt{x-(229-1)}+1)$$
\\
$$\lim\limits_{x\to 229}\frac{x^{2} - 229^{2}}{\sqrt{x-(229-1)}-1}=[\frac{0}{0}]=
\lim\limits_{x\to 229}(x+229)(\sqrt{x-(229-1)}+1) = 2\cdot229 \cdot 2 = 916$$
\rozwStop
\odpStart
$916$
\odpStop
\testStart
A.$916$
B.$-916$
C.$0$
D.$229$
E.$\infty$
F.$-\infty$
G.$-229$
H.$57.25$
I.$-57.25$
\testStop
\kluczStart
A
\kluczStop



\zadStart{Zadanie z Wikieł Z 4.3 h) moja wersja nr 81}
Obliczyć granicę funkcji $f(x)=\frac{x^{2} - 233^{2}}{\sqrt{x-(233-1)}-1}$.
\zadStop
\rozwStart{Patryk Wirkus}{Szymon Tokarski}
$$\frac{x^{2} - 233^{2}}{\sqrt{x-(233-1)}-1}=\frac{(x-233)(x+233)(\sqrt{x-(233-1)}+1)}{x-(233-1)-1}=(x+233)(\sqrt{x-(233-1)}+1)$$
\\
$$\lim\limits_{x\to 233}\frac{x^{2} - 233^{2}}{\sqrt{x-(233-1)}-1}=[\frac{0}{0}]=
\lim\limits_{x\to 233}(x+233)(\sqrt{x-(233-1)}+1) = 2\cdot233 \cdot 2 = 932$$
\rozwStop
\odpStart
$932$
\odpStop
\testStart
A.$932$
B.$-932$
C.$0$
D.$233$
E.$\infty$
F.$-\infty$
G.$-233$
H.$58.25$
I.$-58.25$
\testStop
\kluczStart
A
\kluczStop



\zadStart{Zadanie z Wikieł Z 4.3 h) moja wersja nr 82}
Obliczyć granicę funkcji $f(x)=\frac{x^{2} - 239^{2}}{\sqrt{x-(239-1)}-1}$.
\zadStop
\rozwStart{Patryk Wirkus}{Szymon Tokarski}
$$\frac{x^{2} - 239^{2}}{\sqrt{x-(239-1)}-1}=\frac{(x-239)(x+239)(\sqrt{x-(239-1)}+1)}{x-(239-1)-1}=(x+239)(\sqrt{x-(239-1)}+1)$$
\\
$$\lim\limits_{x\to 239}\frac{x^{2} - 239^{2}}{\sqrt{x-(239-1)}-1}=[\frac{0}{0}]=
\lim\limits_{x\to 239}(x+239)(\sqrt{x-(239-1)}+1) = 2\cdot239 \cdot 2 = 956$$
\rozwStop
\odpStart
$956$
\odpStop
\testStart
A.$956$
B.$-956$
C.$0$
D.$239$
E.$\infty$
F.$-\infty$
G.$-239$
H.$59.75$
I.$-59.75$
\testStop
\kluczStart
A
\kluczStop



\zadStart{Zadanie z Wikieł Z 4.3 h) moja wersja nr 83}
Obliczyć granicę funkcji $f(x)=\frac{x^{2} - 241^{2}}{\sqrt{x-(241-1)}-1}$.
\zadStop
\rozwStart{Patryk Wirkus}{Szymon Tokarski}
$$\frac{x^{2} - 241^{2}}{\sqrt{x-(241-1)}-1}=\frac{(x-241)(x+241)(\sqrt{x-(241-1)}+1)}{x-(241-1)-1}=(x+241)(\sqrt{x-(241-1)}+1)$$
\\
$$\lim\limits_{x\to 241}\frac{x^{2} - 241^{2}}{\sqrt{x-(241-1)}-1}=[\frac{0}{0}]=
\lim\limits_{x\to 241}(x+241)(\sqrt{x-(241-1)}+1) = 2\cdot241 \cdot 2 = 964$$
\rozwStop
\odpStart
$964$
\odpStop
\testStart
A.$964$
B.$-964$
C.$0$
D.$241$
E.$\infty$
F.$-\infty$
G.$-241$
H.$60.25$
I.$-60.25$
\testStop
\kluczStart
A
\kluczStop



\zadStart{Zadanie z Wikieł Z 4.3 h) moja wersja nr 84}
Obliczyć granicę funkcji $f(x)=\frac{x^{2} - 251^{2}}{\sqrt{x-(251-1)}-1}$.
\zadStop
\rozwStart{Patryk Wirkus}{Szymon Tokarski}
$$\frac{x^{2} - 251^{2}}{\sqrt{x-(251-1)}-1}=\frac{(x-251)(x+251)(\sqrt{x-(251-1)}+1)}{x-(251-1)-1}=(x+251)(\sqrt{x-(251-1)}+1)$$
\\
$$\lim\limits_{x\to 251}\frac{x^{2} - 251^{2}}{\sqrt{x-(251-1)}-1}=[\frac{0}{0}]=
\lim\limits_{x\to 251}(x+251)(\sqrt{x-(251-1)}+1) = 2\cdot251 \cdot 2 = 1004$$
\rozwStop
\odpStart
$1004$
\odpStop
\testStart
A.$1004$
B.$-1004$
C.$0$
D.$251$
E.$\infty$
F.$-\infty$
G.$-251$
H.$62.75$
I.$-62.75$
\testStop
\kluczStart
A
\kluczStop



\zadStart{Zadanie z Wikieł Z 4.3 h) moja wersja nr 85}
Obliczyć granicę funkcji $f(x)=\frac{x^{2} - 257^{2}}{\sqrt{x-(257-1)}-1}$.
\zadStop
\rozwStart{Patryk Wirkus}{Szymon Tokarski}
$$\frac{x^{2} - 257^{2}}{\sqrt{x-(257-1)}-1}=\frac{(x-257)(x+257)(\sqrt{x-(257-1)}+1)}{x-(257-1)-1}=(x+257)(\sqrt{x-(257-1)}+1)$$
\\
$$\lim\limits_{x\to 257}\frac{x^{2} - 257^{2}}{\sqrt{x-(257-1)}-1}=[\frac{0}{0}]=
\lim\limits_{x\to 257}(x+257)(\sqrt{x-(257-1)}+1) = 2\cdot257 \cdot 2 = 1028$$
\rozwStop
\odpStart
$1028$
\odpStop
\testStart
A.$1028$
B.$-1028$
C.$0$
D.$257$
E.$\infty$
F.$-\infty$
G.$-257$
H.$64.25$
I.$-64.25$
\testStop
\kluczStart
A
\kluczStop



\zadStart{Zadanie z Wikieł Z 4.3 h) moja wersja nr 86}
Obliczyć granicę funkcji $f(x)=\frac{x^{2} - 263^{2}}{\sqrt{x-(263-1)}-1}$.
\zadStop
\rozwStart{Patryk Wirkus}{Szymon Tokarski}
$$\frac{x^{2} - 263^{2}}{\sqrt{x-(263-1)}-1}=\frac{(x-263)(x+263)(\sqrt{x-(263-1)}+1)}{x-(263-1)-1}=(x+263)(\sqrt{x-(263-1)}+1)$$
\\
$$\lim\limits_{x\to 263}\frac{x^{2} - 263^{2}}{\sqrt{x-(263-1)}-1}=[\frac{0}{0}]=
\lim\limits_{x\to 263}(x+263)(\sqrt{x-(263-1)}+1) = 2\cdot263 \cdot 2 = 1052$$
\rozwStop
\odpStart
$1052$
\odpStop
\testStart
A.$1052$
B.$-1052$
C.$0$
D.$263$
E.$\infty$
F.$-\infty$
G.$-263$
H.$65.75$
I.$-65.75$
\testStop
\kluczStart
A
\kluczStop



\zadStart{Zadanie z Wikieł Z 4.3 h) moja wersja nr 87}
Obliczyć granicę funkcji $f(x)=\frac{x^{2} - 269^{2}}{\sqrt{x-(269-1)}-1}$.
\zadStop
\rozwStart{Patryk Wirkus}{Szymon Tokarski}
$$\frac{x^{2} - 269^{2}}{\sqrt{x-(269-1)}-1}=\frac{(x-269)(x+269)(\sqrt{x-(269-1)}+1)}{x-(269-1)-1}=(x+269)(\sqrt{x-(269-1)}+1)$$
\\
$$\lim\limits_{x\to 269}\frac{x^{2} - 269^{2}}{\sqrt{x-(269-1)}-1}=[\frac{0}{0}]=
\lim\limits_{x\to 269}(x+269)(\sqrt{x-(269-1)}+1) = 2\cdot269 \cdot 2 = 1076$$
\rozwStop
\odpStart
$1076$
\odpStop
\testStart
A.$1076$
B.$-1076$
C.$0$
D.$269$
E.$\infty$
F.$-\infty$
G.$-269$
H.$67.25$
I.$-67.25$
\testStop
\kluczStart
A
\kluczStop



\zadStart{Zadanie z Wikieł Z 4.3 h) moja wersja nr 88}
Obliczyć granicę funkcji $f(x)=\frac{x^{2} - 271^{2}}{\sqrt{x-(271-1)}-1}$.
\zadStop
\rozwStart{Patryk Wirkus}{Szymon Tokarski}
$$\frac{x^{2} - 271^{2}}{\sqrt{x-(271-1)}-1}=\frac{(x-271)(x+271)(\sqrt{x-(271-1)}+1)}{x-(271-1)-1}=(x+271)(\sqrt{x-(271-1)}+1)$$
\\
$$\lim\limits_{x\to 271}\frac{x^{2} - 271^{2}}{\sqrt{x-(271-1)}-1}=[\frac{0}{0}]=
\lim\limits_{x\to 271}(x+271)(\sqrt{x-(271-1)}+1) = 2\cdot271 \cdot 2 = 1084$$
\rozwStop
\odpStart
$1084$
\odpStop
\testStart
A.$1084$
B.$-1084$
C.$0$
D.$271$
E.$\infty$
F.$-\infty$
G.$-271$
H.$67.75$
I.$-67.75$
\testStop
\kluczStart
A
\kluczStop





\end{document}
