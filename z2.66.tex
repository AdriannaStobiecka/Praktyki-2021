\documentclass[12pt, a4paper]{article}
\usepackage[utf8]{inputenc}
\usepackage{polski}

\usepackage{amsthm}  %pakiet do tworzenia twierdzeń itp.
\usepackage{amsmath} %pakiet do niektórych symboli matematycznych
\usepackage{amssymb} %pakiet do symboli mat., np. \nsubseteq
\usepackage{amsfonts}
\usepackage{graphicx} %obsługa plików graficznych z rozszerzeniem png, jpg
\theoremstyle{definition} %styl dla definicji
\newtheorem{zad}{} 
\title{Multizestaw zadań}
\author{Robert Fidytek}
%\date{\today}
\date{}
\newcounter{liczniksekcji}
\newcommand{\kategoria}[1]{\section{#1}} %olreślamy nazwę kateforii zadań
\newcommand{\zadStart}[1]{\begin{zad}#1\newline} %oznaczenie początku zadania
\newcommand{\zadStop}{\end{zad}}   %oznaczenie końca zadania
%Makra opcjonarne (nie muszą występować):
\newcommand{\rozwStart}[2]{\noindent \textbf{Rozwiązanie (autor #1 , recenzent #2): }\newline} %oznaczenie początku rozwiązania, opcjonarnie można wprowadzić informację o autorze rozwiązania zadania i recenzencie poprawności wykonania rozwiązania zadania
\newcommand{\rozwStop}{\newline}                                            %oznaczenie końca rozwiązania
\newcommand{\odpStart}{\noindent \textbf{Odpowiedź:}\newline}    %oznaczenie początku odpowiedzi końcowej (wypisanie wyniku)
\newcommand{\odpStop}{\newline}                                             %oznaczenie końca odpowiedzi końcowej (wypisanie wyniku)
\newcommand{\testStart}{\noindent \textbf{Test:}\newline} %ewentualne możliwe opcje odpowiedzi testowej: A. ? B. ? C. ? D. ? itd.
\newcommand{\testStop}{\newline} %koniec wprowadzania odpowiedzi testowych
\newcommand{\kluczStart}{\noindent \textbf{Test poprawna odpowiedź:}\newline} %klucz, poprawna odpowiedź pytania testowego (jedna literka): A lub B lub C lub D itd.
\newcommand{\kluczStop}{\newline} %koniec poprawnej odpowiedzi pytania testowego 
\newcommand{\wstawGrafike}[2]{\begin{figure}[h] \includegraphics[scale=#2] {#1} \end{figure}} %gdyby była potrzeba wstawienia obrazka, parametry: nazwa pliku, skala (jak nie wiesz co wpisać, to wpisz 1)

\begin{document}
\maketitle


\kategoria{Wikieł/Z2.66}
\zadStart{Zadanie z Wikieł Z 2.66 moja wersja nr [nrWersji]}
%[a]:[2,3,4,5,6,7,8,9,10,11,12,13,14,15,16,17,18,19,20,21,22,23,24,25,26,27,28,29,30]
%[c]:[2,3,5,6,7,10,11,13,14,15,17,19,21,22,23,26,29,30,31,33,34,35,37,38,39,41]
%[b]:[2,3,4,5,6,7,8,9,10,11,12,13,14,15,16,17,18,19,20,21,22,23,24,25,26,27,28,29,30]
%[d]:[2,3,4,5,6,7,8,9,10]
%[cc]=[c]*[c]
%[2c]=2*[c]
%[acc]=[a]*[cc]
%[2ca]=[2c]*[a]
%[1a]=[acc]-1
%[k2ba1]=[2ca]*[2ca]
%[41aa]=4*[1a]*[a]
%[41ab]=4*[1a]*[b]
%[k]=[k2ba1]-[41aa]
%[ka]=[41ab]/[k]
%[cka]=int([ka])
%[bb1]=math.sqrt([cka])
%[cbb1]=int([bb1])
%[ka].is_integer()==True and [bb1].is_integer()==True and [k]>0
Napisać równania stycznych do hiperboli $x^2-[a]y^2=[b]$ równoległych do prostej $y=[c]x-[d]$.
\zadStop
\rozwStart{Aleksandra Pasińska}{}
$$y=ax+b$$
$$a_1=a_2=[c]$$
$$y=[c]x+b$$
$$x^2-[a]([c]x+b)^2=[b]$$
$$x^2-[a]([cc]x^2+[2c]bx+b^2)-[b]=0$$
$$x^2-[acc]x^2-[2ca]bx-[a]b^2-[b]=0$$
$$-[1a]x^2-[2ca]bx-[a]b^2-[b]=0$$
$$[1a]x^2+[2ca]bx+[a]b^2+[b]=0$$
$$\Delta=[k2ba1]b^2-4\cdot[1a]([a]b^2+[b])$$
$$[k2ba1]b^2-[41aa]b^2-[41ab]=0$$
$$[k]b^2=[41ab]$$
$$b^2=[cka]$$
$$b=\pm \sqrt{[cka]}$$
$$b_1=-[cbb1],b_2=[cbb1]$$
$$ y=[c]x-[cbb1], y=[c]x+[cbb1]$$
\rozwStop
\odpStart
$ y=[c]x-[cbb1], y=[c]x+[cbb1]$\\
\odpStop
\testStart
A.$ y=[c]x-[cbb1], y=[c]x+[cbb1]$
B.$ y=-x-[cbb1], y=x-[cbb1]$
C.$ y=-[cbb1], y=[cbb1]$
D.$ y=x-[cbb1], y=0$
E.$ y=0, y=-x+[cbb1]$
F.$ y=-[cbb1], y=-x+[cbb1]$
G.$ y=x, y=[cbb1]$
H.$ y=x-[cbb1], y=-x$
I.$ y=x, y=x+[cbb1]$
\testStop
\kluczStart
A
\kluczStop



\end{document}