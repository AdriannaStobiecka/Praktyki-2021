\documentclass[12pt, a4paper]{article}
\usepackage[utf8]{inputenc}
\usepackage{polski}

\usepackage{amsthm}  %pakiet do tworzenia twierdzeń itp.
\usepackage{amsmath} %pakiet do niektórych symboli matematycznych
\usepackage{amssymb} %pakiet do symboli mat., np. \nsubseteq
\usepackage{amsfonts}
\usepackage{graphicx} %obsługa plików graficznych z rozszerzeniem png, jpg
\theoremstyle{definition} %styl dla definicji
\newtheorem{zad}{} 
\title{Multizestaw zadań}
\author{Robert Fidytek}
%\date{\today}
\date{}
\newcounter{liczniksekcji}
\newcommand{\kategoria}[1]{\section{#1}} %olreślamy nazwę kateforii zadań
\newcommand{\zadStart}[1]{\begin{zad}#1\newline} %oznaczenie początku zadania
\newcommand{\zadStop}{\end{zad}}   %oznaczenie końca zadania
%Makra opcjonarne (nie muszą występować):
\newcommand{\rozwStart}[2]{\noindent \textbf{Rozwiązanie (autor #1 , recenzent #2): }\newline} %oznaczenie początku rozwiązania, opcjonarnie można wprowadzić informację o autorze rozwiązania zadania i recenzencie poprawności wykonania rozwiązania zadania
\newcommand{\rozwStop}{\newline}                                            %oznaczenie końca rozwiązania
\newcommand{\odpStart}{\noindent \textbf{Odpowiedź:}\newline}    %oznaczenie początku odpowiedzi końcowej (wypisanie wyniku)
\newcommand{\odpStop}{\newline}                                             %oznaczenie końca odpowiedzi końcowej (wypisanie wyniku)
\newcommand{\testStart}{\noindent \textbf{Test:}\newline} %ewentualne możliwe opcje odpowiedzi testowej: A. ? B. ? C. ? D. ? itd.
\newcommand{\testStop}{\newline} %koniec wprowadzania odpowiedzi testowych
\newcommand{\kluczStart}{\noindent \textbf{Test poprawna odpowiedź:}\newline} %klucz, poprawna odpowiedź pytania testowego (jedna literka): A lub B lub C lub D itd.
\newcommand{\kluczStop}{\newline} %koniec poprawnej odpowiedzi pytania testowego 
\newcommand{\wstawGrafike}[2]{\begin{figure}[h] \includegraphics[scale=#2] {#1} \end{figure}} %gdyby była potrzeba wstawienia obrazka, parametry: nazwa pliku, skala (jak nie wiesz co wpisać, to wpisz 1)

\begin{document}
\maketitle


\kategoria{Wikieł/P3.2a}
\zadStart{Zadanie z Wikieł P 3.2 a) moja wersja nr [nrWersji]}
%[b]:[2,3,4,5,6,7,8,9,10,11,12]
%[a2]=[b]*[b]
%[a3]=[b]*[b]*[b]
%[a4]=[b]*[b]*[b]*[b]
Zbadać monotoniczność i ograniczoność ciągu $a_{n}=\left(-\frac{1}{[b]}\right)^n$.
\zadStop
\rozwStart{Aleksandra Pasińska}{}
$$a_{1}=-\frac{1}{[b]}, a_{2}=\frac{1}{[a2]},a_{3}=-\frac{1}{[a3]},a_{4}=\frac{1}{[a4]}$$ 
a więc ciąg $(a_{n})$ nie jest monotoniczny.
Z zapisanych powyżej kolejnych wyrazów ciągu $(a_{n})$ otrzymujemy również jego ograniczoność. Zauważmy, że dla każdego $n \in \mathbb{N}$ zachodzi 
$$-\frac{1}{[b]}\leq a_{n} \leq \frac{1}{[a2]}$$
czyli
$$m=-\frac{1}{[b]},M=\frac{1}{[a2]}$$
\rozwStop
\odpStart
$m=-\frac{1}{[b]},M=\frac{1}{[a2]}$
\odpStop
\testStart
A.$m=-\frac{1}{[b]},M=\frac{1}{[a2]}$
B.$m=-\frac{2}{[b]},M=\frac{8}{[a2]}$
C.$m=-\frac{1}{[b]},M=\frac{5}{[a2]}$
D.$m=-\frac{3}{[b]},M=\frac{1}{[a2]}$
E.$m=-\frac{7}{[b]},M=\frac{1}{[a2]}$
F.$m=-\frac{1}{[b]},M=\frac{2}{[a2]}$
G.$m=-\frac{2}{[b]},M=\frac{1}{[a2]}$
H.$m=-\frac{4}{[b]},M=\frac{1}{[a2]}$
I.$m=-\frac{9}{[b]},M=\frac{9}{[a2]}$
\testStop
\kluczStart
A
\kluczStop



\end{document}