\documentclass[12pt, a4paper]{article}
\usepackage[utf8]{inputenc}
\usepackage{polski}

\usepackage{amsthm}  %pakiet do tworzenia twierdzeń itp.
\usepackage{amsmath} %pakiet do niektórych symboli matematycznych
\usepackage{amssymb} %pakiet do symboli mat., np. \nsubseteq
\usepackage{amsfonts}
\usepackage{graphicx} %obsługa plików graficznych z rozszerzeniem png, jpg
\theoremstyle{definition} %styl dla definicji
\newtheorem{zad}{} 
\title{Multizestaw zadań}
\author{Robert Fidytek}
%\date{\today}
\date{}
\newcounter{liczniksekcji}
\newcommand{\kategoria}[1]{\section{#1}} %olreślamy nazwę kateforii zadań
\newcommand{\zadStart}[1]{\begin{zad}#1\newline} %oznaczenie początku zadania
\newcommand{\zadStop}{\end{zad}}   %oznaczenie końca zadania
%Makra opcjonarne (nie muszą występować):
\newcommand{\rozwStart}[2]{\noindent \textbf{Rozwiązanie (autor #1 , recenzent #2): }\newline} %oznaczenie początku rozwiązania, opcjonarnie można wprowadzić informację o autorze rozwiązania zadania i recenzencie poprawności wykonania rozwiązania zadania
\newcommand{\rozwStop}{\newline}                                            %oznaczenie końca rozwiązania
\newcommand{\odpStart}{\noindent \textbf{Odpowiedź:}\newline}    %oznaczenie początku odpowiedzi końcowej (wypisanie wyniku)
\newcommand{\odpStop}{\newline}                                             %oznaczenie końca odpowiedzi końcowej (wypisanie wyniku)
\newcommand{\testStart}{\noindent \textbf{Test:}\newline} %ewentualne możliwe opcje odpowiedzi testowej: A. ? B. ? C. ? D. ? itd.
\newcommand{\testStop}{\newline} %koniec wprowadzania odpowiedzi testowych
\newcommand{\kluczStart}{\noindent \textbf{Test poprawna odpowiedź:}\newline} %klucz, poprawna odpowiedź pytania testowego (jedna literka): A lub B lub C lub D itd.
\newcommand{\kluczStop}{\newline} %koniec poprawnej odpowiedzi pytania testowego 
\newcommand{\wstawGrafike}[2]{\begin{figure}[h] \includegraphics[scale=#2] {#1} \end{figure}} %gdyby była potrzeba wstawienia obrazka, parametry: nazwa pliku, skala (jak nie wiesz co wpisać, to wpisz 1)

\begin{document}
\maketitle


\kategoria{Wikieł/Z2.3}
\zadStart{Zadanie z Wikieł Z 2.3) moja wersja nr [nrWersji]}
%[a]:[2,3,4,5,6,7]
%[b]:[-6,-5,-4,-3,-2,-1]
%[c]:[2,3,4,5,6,7]
%[d]=[a]-3
%[e]=[b]+6
%[f]=[c]+6
%[dam]=[d]-[a]
%[ebm]=[e]-[b]
%[fcm]=[f]-[c]
%[damd]=int([dam]/3)
%[ebmd]=int([ebm]/3)
%[fcmd]=int([fcm]/3)
%[x]=[a]+[damd]
%[y]=[b]+[ebmd]
%[z]=[c]+[fcmd]
%[x]!=[y] and [y]!=[z] and [x]!=[z]
Na odcinku $\overline{AB},$ $A([a],[b],[c]),$ $B([d],[e],[f])$ znaleźć taki punkt $C$, że $|\overline{AC}|:|\overline{CB}|=1:2$.
\zadStop
\rozwStart{Justyna Chojecka}{}
Punkt $C$ otrzymamy przesuwając punkt $A$ o $\frac{1}{3}$ wektora $\overrightarrow{AB}$. Obliczamy najpierw wektor $\overrightarrow{AB}$.
$$\overrightarrow{AB}=([d],[e],[f])-([a],[b],[c])=\left[[d]-[a],[e]-([b]),[f]-[c]\right]=\left[[dam],[ebm],[fcm]\right].$$
Następnie wyznaczamy współrzędne punktu $C$.
$$C=A+\frac{1}{3}\overrightarrow{AB}=([a],[b],[c])+\frac{1}{3}\left[[dam],[ebm],[fcm]\right]$$$$=([a],[b],[c])+\left[[damd],[ebmd],[fcmd]\right]=([x],[y],[z])$$
\rozwStop
\odpStart
$C([x],[y],[z])$
\odpStop
\testStart
A.$C([x],[y],[z])$
B.$C([y],[x],[z])$
C.$C([y],[z],[x])$
D.$C([x],[x],[y])$
E.$C([y],[y],[z])$
F.$C([y],[x],[x])$
G.$C([x],[z],[z])$
H.$C([x],[z],[y])$
I.$C([z],[x],[y])$
\testStop
\kluczStart
A
\kluczStop



\end{document}