\documentclass[12pt, a4paper]{article}
\usepackage[utf8]{inputenc}
\usepackage{polski}

\usepackage{amsthm}  %pakiet do tworzenia twierdzeń itp.
\usepackage{amsmath} %pakiet do niektórych symboli matematycznych
\usepackage{amssymb} %pakiet do symboli mat., np. \nsubseteq
\usepackage{amsfonts}
\usepackage{graphicx} %obsługa plików graficznych z rozszerzeniem png, jpg
\theoremstyle{definition} %styl dla definicji
\newtheorem{zad}{} 
\title{Multizestaw zadań}
\author{Robert Fidytek}
%\date{\today}
\date{}
\newcounter{liczniksekcji}
\newcommand{\kategoria}[1]{\section{#1}} %olreślamy nazwę kateforii zadań
\newcommand{\zadStart}[1]{\begin{zad}#1\newline} %oznaczenie początku zadania
\newcommand{\zadStop}{\end{zad}}   %oznaczenie końca zadania
%Makra opcjonarne (nie muszą występować):
\newcommand{\rozwStart}[2]{\noindent \textbf{Rozwiązanie (autor #1 , recenzent #2): }\newline} %oznaczenie początku rozwiązania, opcjonarnie można wprowadzić informację o autorze rozwiązania zadania i recenzencie poprawności wykonania rozwiązania zadania
\newcommand{\rozwStop}{\newline}                                            %oznaczenie końca rozwiązania
\newcommand{\odpStart}{\noindent \textbf{Odpowiedź:}\newline}    %oznaczenie początku odpowiedzi końcowej (wypisanie wyniku)
\newcommand{\odpStop}{\newline}                                             %oznaczenie końca odpowiedzi końcowej (wypisanie wyniku)
\newcommand{\testStart}{\noindent \textbf{Test:}\newline} %ewentualne możliwe opcje odpowiedzi testowej: A. ? B. ? C. ? D. ? itd.
\newcommand{\testStop}{\newline} %koniec wprowadzania odpowiedzi testowych
\newcommand{\kluczStart}{\noindent \textbf{Test poprawna odpowiedź:}\newline} %klucz, poprawna odpowiedź pytania testowego (jedna literka): A lub B lub C lub D itd.
\newcommand{\kluczStop}{\newline} %koniec poprawnej odpowiedzi pytania testowego 
\newcommand{\wstawGrafike}[2]{\begin{figure}[h] \includegraphics[scale=#2] {#1} \end{figure}} %gdyby była potrzeba wstawienia obrazka, parametry: nazwa pliku, skala (jak nie wiesz co wpisać, to wpisz 1)

\begin{document}
\maketitle


\kategoria{Wikieł/Z1.97j}
\zadStart{Zadanie z Wikieł Z 1.97 j) moja wersja nr [nrWersji]}
%[b]:[2,3,4,5,6,7,8,9,10]
%[c]:[1,2,3,4,5,6,7,8,9,10,11,12,13,14,15,16,17,18,19,20,21,22,23,24,25,26,27,28,29,30]
%[d]=[b]**2-4*[c]
%[pr2]=(pow([d],(1/2)))
%[pr1]=[pr2].real
%[pr]=int([pr1])
%[zz1]=([b]-[pr])/2
%[z1]=int([zz1])
%[zz2]=([b]+[pr])/2
%[z2]=int([zz2])
%[d]>0 and [pr2].is_integer()==True and [zz1].is_integer()==True and [zz2].is_integer()==True
Rozwiązać nierówno\'sć $\log_{x^2}{([b]x-[c])} \ge 1$
\zadStop
\rozwStart{Małgorzata Ugowska}{}
$$\log_{x^2}{([b]x-[c])} \ge 1 \quad \Longleftrightarrow \quad \log_{x^2}{([b]x-[c])} \ge \log_{x^2}{x^2} $$
$$\Longleftrightarrow \quad [b]x-[c] \ge x^2 \quad \Longleftrightarrow \quad x^2 -[b]x+[c] \le 0$$
Szukamy miejsc zerowych funkcji $x^2-[b]x+[c]$:
$$ \bigtriangleup = [b]^2-4 \cdot 1 \cdot [c] = [d] \quad  \Longrightarrow \quad x_1=\frac{[b]-\sqrt{\bigtriangleup}}{2} = [z1], \quad x_2=\frac{[b]+\sqrt{\bigtriangleup}}{2} = [z2]$$
$$x^2-[b]x-[c] \le 0 \quad \Longleftrightarrow \quad (x-[z1])(x-[z2])>0 \quad \Longleftrightarrow \quad x \in [[z1],[z2]]$$
\rozwStop
\odpStart
$x \in [[z1],[z2]]$
\odpStop
\testStart
A. $x \in [[z1],[z2]]$\\
B. $x \in (-\infty, [z1]] \cup [[z2],\infty)$\\
C. $x \in (-\infty, -4] \cup [3,\infty)$\\
D. $x \in [-4,3]$\\
E. $x \in \emptyset$
\testStop
\kluczStart
A
\kluczStop



\end{document}