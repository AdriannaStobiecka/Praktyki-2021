\documentclass[12pt, a4paper]{article}
\usepackage[utf8]{inputenc}
\usepackage{polski}

\usepackage{amsthm}  %pakiet do tworzenia twierdzeń itp.
\usepackage{amsmath} %pakiet do niektórych symboli matematycznych
\usepackage{amssymb} %pakiet do symboli mat., np. \nsubseteq
\usepackage{amsfonts}
\usepackage{graphicx} %obsługa plików graficznych z rozszerzeniem png, jpg
\theoremstyle{definition} %styl dla definicji
\newtheorem{zad}{} 
\title{Multizestaw zadań}
\author{Robert Fidytek}
%\date{\today}
\date{}
\newcounter{liczniksekcji}
\newcommand{\kategoria}[1]{\section{#1}} %olreślamy nazwę kateforii zadań
\newcommand{\zadStart}[1]{\begin{zad}#1\newline} %oznaczenie początku zadania
\newcommand{\zadStop}{\end{zad}}   %oznaczenie końca zadania
%Makra opcjonarne (nie muszą występować):
\newcommand{\rozwStart}[2]{\noindent \textbf{Rozwiązanie (autor #1 , recenzent #2): }\newline} %oznaczenie początku rozwiązania, opcjonarnie można wprowadzić informację o autorze rozwiązania zadania i recenzencie poprawności wykonania rozwiązania zadania
\newcommand{\rozwStop}{\newline}                                            %oznaczenie końca rozwiązania
\newcommand{\odpStart}{\noindent \textbf{Odpowiedź:}\newline}    %oznaczenie początku odpowiedzi końcowej (wypisanie wyniku)
\newcommand{\odpStop}{\newline}                                             %oznaczenie końca odpowiedzi końcowej (wypisanie wyniku)
\newcommand{\testStart}{\noindent \textbf{Test:}\newline} %ewentualne możliwe opcje odpowiedzi testowej: A. ? B. ? C. ? D. ? itd.
\newcommand{\testStop}{\newline} %koniec wprowadzania odpowiedzi testowych
\newcommand{\kluczStart}{\noindent \textbf{Test poprawna odpowiedź:}\newline} %klucz, poprawna odpowiedź pytania testowego (jedna literka): A lub B lub C lub D itd.
\newcommand{\kluczStop}{\newline} %koniec poprawnej odpowiedzi pytania testowego 
\newcommand{\wstawGrafike}[2]{\begin{figure}[h] \includegraphics[scale=#2] {#1} \end{figure}} %gdyby była potrzeba wstawienia obrazka, parametry: nazwa pliku, skala (jak nie wiesz co wpisać, to wpisz 1)

\begin{document}
\maketitle


\kategoria{Wikieł/Z1.78e}
\zadStart{Zadanie z Wikieł Z 1.78 e) moja wersja nr [nrWersji]}
%[a]:[2,3,4,5,6,7,8,9,10,11,12,13,14,15,16,17,18,19,20]
%[b]:[2,3,4,5,6,7,8,9,10,11,12,13,14,15,16,17,18,19,20]
%[c]:[2,3,4,5,6,7,8,9,10,11,12,13,14,15,16,17,18,19,20]
%[c2]=[c]**2
%[2c]=2*[c]
%[2ca]=[2c]+[a]
%[d]=[c2]-[b]
%[delta1]=[2ca]**2
%[delta2]=4*[d]
%[delta]=[delta1]-[delta2]
%[pdelta]=int(math.sqrt(abs([delta])))
%[lx1]=[2ca]-[pdelta]
%[lx2]=[2ca]+[pdelta]
%[x1]=int([lx1]/2)
%[x2]=int([lx2]/2)
%[delta]>0 and int([pdelta])**2==[delta] and math.gcd([lx1],2)==2 and math.gcd([lx2],2)==2 and [x2]>[c] and [x1]>=-[b]/[a] and [x1]<=[c] and math.gcd([b],[a])==1 and [x1]>0 
Rozwiązać równanie
$$x+\sqrt{[a]x+[b]}=[c].$$
\zadStop
\rozwStart{Adrianna Stobiecka}{}
Zakładamy, że $[a]x+[b]\geq0$.
$$[a]x+[b]\geq0\Leftrightarrow [a]x\geq-[b]\Leftrightarrow x\geq-\frac{[b]}{[a]}$$
Więc $x\in\big[-\frac{[b]}{[a]},\infty\big)$. Powyższe równanie możemy zapisać w postaci
$$\sqrt{[a]x+[b]}=[c]-x.$$
Lewa strona rozważanego równiania jest nieujemna, żeby zatem równanie nie było sprzeczne, musimy dodatkowo założyć $[c]-x\geq0$. Otrzymujemy więc założenie: $x\in\big[-\frac{[b]}{[a]},[c]\big]$.
\\Wiemy, że dla $a\geq0$, $b\geq0$ oraz $n\in\mathbb{N}$ zachodzi własność
$$a=b\Leftrightarrow a^n=b^n.$$ 
Korzystając z tej własności otrzymujemy:
$$\sqrt{[a]x+[b]}=[c]-x\Leftrightarrow\big(\sqrt{[a]x+[b]}\big)^2=([c]-x)^2\Leftrightarrow [a]x+[b]=[c2]-[2c]x+x^2$$
$$\Leftrightarrow x^2-[2c]x-[a]x+[c2]-[b]=0\Leftrightarrow x^2-[2ca]x+[d]=0$$
Dla równania $ax^2+bx+c=0$, deltę oblicza się ze wzoru $\Delta=b^2-4\cdot a\cdot c$. Obliczamy deltę dla naszego równania:
$$\Delta=(-[2ca])^2-4\cdot[d]=[delta1]-[delta2]=[delta]$$
Otrzymujemy stąd, że $\sqrt{\Delta}=[pdelta]$. Pierwiastki funkcji kwadratowej obliczamy ze wzorów
$$x_1=\frac{-b-\sqrt{\Delta}}{2a},\qquad x_2=\frac{-b+\sqrt{\Delta}}{2a}.$$ 
Mamy zatem
$$x_1=\frac{[2ca]-[pdelta]}{2}=\frac{[lx1]}{2}=[x1],\qquad x_2=\frac{[2ca]+[pdelta]}{2}=\frac{[lx2]}{2}=[x2].$$
Uwzględniając założenie $x\in\big[-\frac{[b]}{[a]},[c]\big]$, otrzymujemy, że rozwiązaniem równania jest $x=[x1].$
\rozwStop
\odpStart
$x=[x1]$
\odpStop
\testStart
A.$x=-[x2]$
B.$x=-[x1]$
C.$x=0$
D.$x=[x2]$
E.$x\in\emptyset$
F.$x=[x1]$
G.$-\frac{[b]}{[a]}$
H.$x=-1$
I.$-\frac{[a]}{[b]}$
\testStop
\kluczStart
F
\kluczStop



\end{document}
