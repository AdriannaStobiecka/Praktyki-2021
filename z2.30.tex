\documentclass[12pt, a4paper]{article}
\usepackage[utf8]{inputenc}
\usepackage{polski}

\usepackage{amsthm}  %pakiet do tworzenia twierdzeń itp.
\usepackage{amsmath} %pakiet do niektórych symboli matematycznych
\usepackage{amssymb} %pakiet do symboli mat., np. \nsubseteq
\usepackage{amsfonts}
\usepackage{graphicx} %obsługa plików graficznych z rozszerzeniem png, jpg
\theoremstyle{definition} %styl dla definicji
\newtheorem{zad}{} 
\title{Multizestaw zadań}
\author{Robert Fidytek}
%\date{\today}
\date{}
\newcounter{liczniksekcji}
\newcommand{\kategoria}[1]{\section{#1}} %olreślamy nazwę kateforii zadań
\newcommand{\zadStart}[1]{\begin{zad}#1\newline} %oznaczenie początku zadania
\newcommand{\zadStop}{\end{zad}}   %oznaczenie końca zadania
%Makra opcjonarne (nie muszą występować):
\newcommand{\rozwStart}[2]{\noindent \textbf{Rozwiązanie (autor #1 , recenzent #2): }\newline} %oznaczenie początku rozwiązania, opcjonarnie można wprowadzić informację o autorze rozwiązania zadania i recenzencie poprawności wykonania rozwiązania zadania
\newcommand{\rozwStop}{\newline}                                            %oznaczenie końca rozwiązania
\newcommand{\odpStart}{\noindent \textbf{Odpowiedź:}\newline}    %oznaczenie początku odpowiedzi końcowej (wypisanie wyniku)
\newcommand{\odpStop}{\newline}                                             %oznaczenie końca odpowiedzi końcowej (wypisanie wyniku)
\newcommand{\testStart}{\noindent \textbf{Test:}\newline} %ewentualne możliwe opcje odpowiedzi testowej: A. ? B. ? C. ? D. ? itd.
\newcommand{\testStop}{\newline} %koniec wprowadzania odpowiedzi testowych
\newcommand{\kluczStart}{\noindent \textbf{Test poprawna odpowiedź:}\newline} %klucz, poprawna odpowiedź pytania testowego (jedna literka): A lub B lub C lub D itd.
\newcommand{\kluczStop}{\newline} %koniec poprawnej odpowiedzi pytania testowego 
\newcommand{\wstawGrafike}[2]{\begin{figure}[h] \includegraphics[scale=#2] {#1} \end{figure}} %gdyby była potrzeba wstawienia obrazka, parametry: nazwa pliku, skala (jak nie wiesz co wpisać, to wpisz 1)

\begin{document}
\maketitle


\kategoria{Wikieł/Z2.30}
\zadStart{Zadanie z Wikieł Z 2.30 moja wersja nr [nrWersji]}
%[a1]:[2,3,4,5,6,7,8,9,10,11,12,13,14,15,16,17,18,19,20]
%[a2]:[1,2,3,4,5,6,7,8,9,10,11,12,13,14,15,16,17,18,19,20]
%[b1]=[a1]
%[b2]:[1,2,3,4,5,6,7,8,9,10,11,12,13,14,15,16,17,18,19,20]
%[h1]=[a1]+1
%[h2]=[b2]+3
%[ah1]=[h1]-[a1]
%[aah]=[h2]-[a2]
%[bb1]=[a1]*[aah]
%[bah]=[a2]-[bb1]
%[bh1]=[h1]-[b1]
%[abh]=[h2]-[b2]
%[bb2]=[b1]*[abh]
%[bbh]=[b2]-[bb2]
%[y1]=[abh]*[a2]
%[bb]=[y1]+[a1]
%[y2]=[aah]*[b2]
%[ba]=[y2]+[b1]
%[c]=[abh]-[aah]
%[d]=[bb]*[aah]
%[e]=[ba]*[abh]
%[cc]=[d]-[e]
%[f]=[cc]+[bb]
%[c2]=[f]/3
%[cc2]=int([c2])
%[aah]>1 and [bah]>0 and [abh]>1 and [bbh]<0 and [cc]>0 and math.gcd([cc],[c])==1
W trójkącie ABC dane są A([a1],[a2]), B([b1],[b2]) oraz punkt przecięcia wysokości H([h1],[h2]). Wyznaczyć współrzędne wierzchołka C.
\zadStop
\rozwStart{Aleksandra Pasińska}{}
1) Prosta AH:
$$\left\{ \begin{array}{ll}
[a2]=[a1]a+b\\ 
.[h2]=[h1]a+b 
\end{array} \right.$$
$$b=[a2]-[a1]a$$
$$[h2]=[h1]a+[a2]-[a1]a$$
$$a=[h2]-[a2]$$
$$a=[aah], b=[bah]$$
$$y=[aah]x+[bah]$$
2) Prosta BH:
$$\left\{ \begin{array}{ll}
[b2]=[b1]a+b\\ 
.[h2]=[h1]a+b 
\end{array} \right.$$
$$b=[b2]-[b1]a$$
$$[h2]=[h1]a+[b2]-[b1]a$$
$$a=[h2]-[b2]$$
$$a=[abh], b=[bbh]$$
$$y=[abh]x[bbh]$$
3)AC$\perp$BH
$$y=[abh]x[bbh],a\cdot [abh]=-1\Rightarrow a=-\frac{1}{[abh]}$$
$$y=-\frac{1}{[abh]}x+b, A([a1],[a2])$$
$$[a2]=-\frac{[a1]}{[abh]}+b$$
$$b=\frac{[bb]}{[abh]},y=-\frac{1}{[abh]}x+\frac{[bb]}{[abh]}$$
4)BC$\perp$AH
$$y=[aah]x+[bah],a\cdot [aah]=-1\Rightarrow a=-\frac{1}{[aah]}$$
$$y=-\frac{1}{[aah]}x+b, B([b1],[b2])$$
$$[b2]=-\frac{[b1]}{[aah]}+b$$
$$b=\frac{[ba]}{[aah]},y=-\frac{1}{[aah]}x+\frac{[ba]}{[aah]}$$
5) $$\left\{ \begin{array}{ll}
y=-\frac{1}{[abh]}x+\frac{[bb]}{[abh]}\\ 
y=-\frac{1}{[aah]}x+\frac{[ba]}{[aah]}
\end{array} \right.$$
$$-\frac{1}{[abh]}x+\frac{[bb]}{[abh]}=-\frac{1}{[aah]}x+\frac{[ba]}{[aah]}$$
$$x=-[cc], y=[cc2], C(-[cc],[cc2])$$
\rozwStop
\odpStart
$C(-[cc],[cc2])$\\
\odpStop
\testStart
A.$ C(-[cc],[cc2])$
B.$ C(0,[cc2])$
C.$ C(9,[cc2])$
D.$ C(6,[cc2])$
E.$ C(1,[cc2])$
F.$ C(99,[cc2])$
G.$ C(44,[cc2])$
H.$ C(36,[cc2])$
I.$ C(70,[cc2])$
\testStop
\kluczStart
A
\kluczStop



\end{document}