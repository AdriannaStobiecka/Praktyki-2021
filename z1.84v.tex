\documentclass[12pt, a4paper]{article}
\usepackage[utf8]{inputenc}
\usepackage{polski}

\usepackage{amsthm}  %pakiet do tworzenia twierdzeń itp.
\usepackage{amsmath} %pakiet do niektórych symboli matematycznych
\usepackage{amssymb} %pakiet do symboli mat., np. \nsubseteq
\usepackage{amsfonts}
\usepackage{graphicx} %obsługa plików graficznych z rozszerzeniem png, jpg
\theoremstyle{definition} %styl dla definicji
\newtheorem{zad}{} 
\title{Multizestaw zadań}
\author{Robert Fidytek}
%\date{\today}
\date{}
\newcounter{liczniksekcji}
\newcommand{\kategoria}[1]{\section{#1}} %olreślamy nazwę kateforii zadań
\newcommand{\zadStart}[1]{\begin{zad}#1\newline} %oznaczenie początku zadania
\newcommand{\zadStop}{\end{zad}}   %oznaczenie końca zadania
%Makra opcjonarne (nie muszą występować):
\newcommand{\rozwStart}[2]{\noindent \textbf{Rozwiązanie (autor #1 , recenzent #2): }\newline} %oznaczenie początku rozwiązania, opcjonarnie można wprowadzić informację o autorze rozwiązania zadania i recenzencie poprawności wykonania rozwiązania zadania
\newcommand{\rozwStop}{\newline}                                            %oznaczenie końca rozwiązania
\newcommand{\odpStart}{\noindent \textbf{Odpowiedź:}\newline}    %oznaczenie początku odpowiedzi końcowej (wypisanie wyniku)
\newcommand{\odpStop}{\newline}                                             %oznaczenie końca odpowiedzi końcowej (wypisanie wyniku)
\newcommand{\testStart}{\noindent \textbf{Test:}\newline} %ewentualne możliwe opcje odpowiedzi testowej: A. ? B. ? C. ? D. ? itd.
\newcommand{\testStop}{\newline} %koniec wprowadzania odpowiedzi testowych
\newcommand{\kluczStart}{\noindent \textbf{Test poprawna odpowiedź:}\newline} %klucz, poprawna odpowiedź pytania testowego (jedna literka): A lub B lub C lub D itd.
\newcommand{\kluczStop}{\newline} %koniec poprawnej odpowiedzi pytania testowego 
\newcommand{\wstawGrafike}[2]{\begin{figure}[h] \includegraphics[scale=#2] {#1} \end{figure}} %gdyby była potrzeba wstawienia obrazka, parametry: nazwa pliku, skala (jak nie wiesz co wpisać, to wpisz 1)

\begin{document}
\maketitle


\kategoria{Wikieł/Z1.84v}
\zadStart{Zadanie z Wikieł Z 1.84 v) moja wersja nr [nrWersji]}
%[a]:[2,3,4,5,6,7,8,9]
%[b]:[2,3,4,5,6,7,8,9]
%[c]=[b]*[b]
%[d]=[a]*[b]
%[e]=[a]*[a]
%[a]!=[b] and [a]!=[c] and [d]!=[e] 
Rozwiązać równanie $[a]^{2x} \cdot [c]^x -2 \cdot [d]^{3x-1} + [e]^{2x-1} \cdot [b]^{4x-2}=0$.
\zadStop
\rozwStart{Barbara Bączek}{}
$$[a]^{2x} \cdot [c]^x -2 \cdot [d]^{3x-1} + [e]^{2x-1} \cdot [b]^{4x-2}=0$$
$$[a]^{2x} \cdot [b]^{2x} -2 \cdot [d]^{3x-1} + [a]^{4x-2} \cdot [b]^{4x-2}=0$$
$${([a] \cdot [b])}^{2x} -2 \cdot [d]^{3x-1} + {([a] \cdot [b])}^{4x-2}=0$$
$$[d]^{2x} -2 \cdot [d]^{3x-1} + [d]^{4x-2}=0$$
$$1 -2 \cdot [d]^{x-1} + [d]^{2x-2}=0$$
$${([d]^{x-1})}^2 -2 \cdot [d]^{x-1} +1=0$$
$${([d]^{x-1}-1)}^2=0$$
$$[d]^{x-1}=1 \Rightarrow x-1=0 \Rightarrow x=1$$.
\rozwStop
\odpStart
$1$
\odpStop
\testStart
A.$[a]$
B.$-[e]$
C.$1$
D.$0$
E.$[b]$
G.$[d]$
H.$-1$
\testStop
\kluczStart
C
\kluczStop



\end{document}