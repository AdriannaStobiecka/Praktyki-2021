\documentclass[12pt, a4paper]{article}
\usepackage[utf8]{inputenc}
\usepackage{polski}

\usepackage{amsthm}  %pakiet do tworzenia twierdzeń itp.
\usepackage{amsmath} %pakiet do niektórych symboli matematycznych
\usepackage{amssymb} %pakiet do symboli mat., np. \nsubseteq
\usepackage{amsfonts}
\usepackage{graphicx} %obsługa plików graficznych z rozszerzeniem png, jpg
\theoremstyle{definition} %styl dla definicji
\newtheorem{zad}{} 
\title{Multizestaw zadań}
\author{Robert Fidytek}
%\date{\today}
\date{}
\newcounter{liczniksekcji}
\newcommand{\kategoria}[1]{\section{#1}} %olreślamy nazwę kateforii zadań
\newcommand{\zadStart}[1]{\begin{zad}#1\newline} %oznaczenie początku zadania
\newcommand{\zadStop}{\end{zad}}   %oznaczenie końca zadania
%Makra opcjonarne (nie muszą występować):
\newcommand{\rozwStart}[2]{\noindent \textbf{Rozwiązanie (autor #1 , recenzent #2): }\newline} %oznaczenie początku rozwiązania, opcjonarnie można wprowadzić informację o autorze rozwiązania zadania i recenzencie poprawności wykonania rozwiązania zadania
\newcommand{\rozwStop}{\newline}                                            %oznaczenie końca rozwiązania
\newcommand{\odpStart}{\noindent \textbf{Odpowiedź:}\newline}    %oznaczenie początku odpowiedzi końcowej (wypisanie wyniku)
\newcommand{\odpStop}{\newline}                                             %oznaczenie końca odpowiedzi końcowej (wypisanie wyniku)
\newcommand{\testStart}{\noindent \textbf{Test:}\newline} %ewentualne możliwe opcje odpowiedzi testowej: A. ? B. ? C. ? D. ? itd.
\newcommand{\testStop}{\newline} %koniec wprowadzania odpowiedzi testowych
\newcommand{\kluczStart}{\noindent \textbf{Test poprawna odpowiedź:}\newline} %klucz, poprawna odpowiedź pytania testowego (jedna literka): A lub B lub C lub D itd.
\newcommand{\kluczStop}{\newline} %koniec poprawnej odpowiedzi pytania testowego 
\newcommand{\wstawGrafike}[2]{\begin{figure}[h] \includegraphics[scale=#2] {#1} \end{figure}} %gdyby była potrzeba wstawienia obrazka, parametry: nazwa pliku, skala (jak nie wiesz co wpisać, to wpisz 1)

\begin{document}
\maketitle


\kategoria{Wikieł/Z1.93t}
\zadStart{Zadanie z Wikieł Z 1.93 t) moja wersja nr [nrWersji]}
%[b]:[2,3,4,5,6,7,8,9]
%[a]:[1,2,3,4,5,6,7,8,9,10,11,12]
%[c]:[1,2,3,4,5,6,7,8,9,10,11,12]
%[e]:[1,2,3,4,5,6,7,8,9,10]
%[db]=1*[b]
%[f]=[a]-[c]
%[aa]=[a]**2
%[ce]=[c]*[e]
%[fe]=[e]-[f]
%[delta]=[fe]**2-4*[ce]
%[pr2]=(pow([delta],(1/2)))
%[pr1]=[pr2].real
%[pr]=int([pr1])
%[zz1]=([fe]-[pr])/(2)
%[zz2]=([fe]+[pr])/(2)
%[z1]=int([zz1])
%[z2]=int([zz2])
%[x1]=pow([b],[z1])
%[x2]=pow([b],[z2])
%[kk]=pow([b],[c])
%[f]>0 and [fe]>0 and [delta]>0 and [pr2].is_integer()==True and [zz1].is_integer()==True and [zz2].is_integer()==True and [z1]!=[c] and [z2]!=[c] and [z1]!=1 and [z2]!=1
Rozwiązać równanie $\frac{[a]}{\log_{[b]}{x}-[c]} + 1 = [e] \log_{x}{[b]}$
\zadStop
\rozwStart{Małgorzata Ugowska}{}
Szukamy dziedziny:
$$ x>0 \quad \land \quad \log_{[b]}{x}-[c] \ne 0 \quad \land \quad x \ne 1 \quad \Longrightarrow \quad  x>0 \quad \land \quad x \ne [kk] \quad \land \quad x \ne 1$$
Następnie rozwiązujemy równanie:
$$\frac{[a]}{\log_{[b]}{x}-[c]} + 1 = [e] \log_{x}{[b]} \quad \Longleftrightarrow \quad \frac{[a]+1(\log_{[b]}{x}-[c])}{\log_{[b]}{x}-[c]} = \frac{[e]}{\log_{[b]}{x}} $$
$$\Longleftrightarrow \quad \frac{[f]+\log_{[b]}{x}}{\log_{[b]}{x}-[c]} = \frac{[e]}{\log_{[b]}{x}} \quad \Longleftrightarrow \quad \log_{[b]}{x} ([f]+\log_{[b]}{x}) = [e](\log_{[b]}{x}-[c]) $$
Podstawiamy $y=\log_{[b]}{x}$:
$$ [f]y +y^2 = [e]y- [ce] \quad \Longleftrightarrow \quad y^2 -[fe]y +[ce]$$
$$ \bigtriangleup = [fe]^2 - 4 \cdot [ce] = [delta] \quad  \Longrightarrow \quad \sqrt{\bigtriangleup} = [pr]$$
$$y_1=\frac{[fe]-\sqrt{\bigtriangleup}}{2} =[z1]  \quad \land \quad y_2=\frac{[fe]+\sqrt{\bigtriangleup}}{2} =[z2]$$
dla $y=[z1]$:
$$\log_{[b]}{x} = [z1] \quad  \Longrightarrow \quad x = [b]^{[z1]} = [x1]$$
dla $y=[z2]$:
$$\log_{[b]}{x} = [z2] \quad  \Longrightarrow \quad x = [b]^{[z2]} = [x2]$$
\rozwStop
\odpStart
$x \in \{[x1],[x2]\}$
\odpStop
\testStart
A. $x \in \{[delta], [c]\}$\\
B. $x \in \{-1, 1\}$\\
C. $x \in \{[x1], [x2]\}$\\
D. $x \in \{\frac{1}{2}, 2\}$\\
E. $x \in \{3, 5\}$
\testStop
\kluczStart
C
\kluczStop



\end{document}