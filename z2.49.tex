\documentclass[12pt, a4paper]{article}
\usepackage[utf8]{inputenc}
\usepackage{polski}

\usepackage{amsthm}  %pakiet do tworzenia twierdzeń itp.
\usepackage{amsmath} %pakiet do niektórych symboli matematycznych
\usepackage{amssymb} %pakiet do symboli mat., np. \nsubseteq
\usepackage{amsfonts}
\usepackage{graphicx} %obsługa plików graficznych z rozszerzeniem png, jpg
\theoremstyle{definition} %styl dla definicji
\newtheorem{zad}{} 
\title{Multizestaw zadań}
\author{Robert Fidytek}
%\date{\today}
\date{}\documentclass[12pt, a4paper]{article}
\usepackage[utf8]{inputenc}
\usepackage{polski}

\usepackage{amsthm}  %pakiet do tworzenia twierdzeń itp.
\usepackage{amsmath} %pakiet do niektórych symboli matematycznych
\usepackage{amssymb} %pakiet do symboli mat., np. \nsubseteq
\usepackage{amsfonts}
\usepackage{graphicx} %obsługa plików graficznych z rozszerzeniem png, jpg
\theoremstyle{definition} %styl dla definicji
\newtheorem{zad}{} 
\title{Multizestaw zadań}
\author{Robert Fidytek}
%\date{\today}
\date{}
\newcounter{liczniksekcji}
\newcommand{\kategoria}[1]{\section{#1}} %olreślamy nazwę kateforii zadań
\newcommand{\zadStart}[1]{\begin{zad}#1\newline} %oznaczenie początku zadania
\newcommand{\zadStop}{\end{zad}}   %oznaczenie końca zadania
%Makra opcjonarne (nie muszą występować):
\newcommand{\rozwStart}[2]{\noindent \textbf{Rozwiązanie (autor #1 , recenzent #2): }\newline} %oznaczenie początku rozwiązania, opcjonarnie można wprowadzić informację o autorze rozwiązania zadania i recenzencie poprawności wykonania rozwiązania zadania
\newcommand{\rozwStop}{\newline}                                            %oznaczenie końca rozwiązania
\newcommand{\odpStart}{\noindent \textbf{Odpowiedź:}\newline}    %oznaczenie początku odpowiedzi końcowej (wypisanie wyniku)
\newcommand{\odpStop}{\newline}                                             %oznaczenie końca odpowiedzi końcowej (wypisanie wyniku)
\newcommand{\testStart}{\noindent \textbf{Test:}\newline} %ewentualne możliwe opcje odpowiedzi testowej: A. ? B. ? C. ? D. ? itd.
\newcommand{\testStop}{\newline} %koniec wprowadzania odpowiedzi testowych
\newcommand{\kluczStart}{\noindent \textbf{Test poprawna odpowiedź:}\newline} %klucz, poprawna odpowiedź pytania testowego (jedna literka): A lub B lub C lub D itd.
\newcommand{\kluczStop}{\newline} %koniec poprawnej odpowiedzi pytania testowego 
\newcommand{\wstawGrafike}[2]{\begin{figure}[h] \includegraphics[scale=#2] {#1} \end{figure}} %gdyby była potrzeba wstawienia obrazka, parametry: nazwa pliku, skala (jak nie wiesz co wpisać, to wpisz 1)

\begin{document}
\maketitle


\kategoria{Wikieł/Z2.49}
\zadStart{Zadanie z Wikieł Z 2.49 moja wersja nr [nrWersji]}
%[p1]:[2,3,4,5,6,7,8]
%[p2]:[2,3,4,5,6,7,8]
%[p3]=random.randint(2,10)
%[p4]=random.randint(2,10)
%[p5]:[2,3,4,5,6,7,8]
%[p6]=random.randint(2,10)
%[p12]=[p1]/2
%[p22]=[p2]/2
%[p12k]=[p12]*[p12]
%[p22k]=[p22]*[p22]
%[r]=[p12k]+[p22k]-[p3]
%[dl]=abs([p4]*[p12]-[p5]*[p22]+[p6])
%[dm]=round(math.sqrt([p4]*[p4]+[p5]*[p5]),2)
%[d]=round([dl]/[dm],2)
%[dk]=round([d]*[d],2)
%[dkr]=[dk]-[r]
%[dkr]>0


Podać, dla jakich wartości parametru $m$ równanie
$$x^{2}+y^{2}-[p1]x+[p2]y-m^{2}+[p3]=0$$
przedstawia okrąg, który nie ma punktów wspólnych z prostą $k: [p4]x+[p5]y+[p6]=0.$

\zadStop

\rozwStart{Maja Szabłowska}{}
$$r^{2}=\left(\frac{[p1]}{2}\right)^{2}+\left(-\frac{[p2]}{2}\right)^{2}-([p3]-m^{2})$$
$$r^{2}=[p12k]+[p22k]-[p3]+m^{2}=[r]+m^{2}$$
$$r=\sqrt{[r]+m^{2}}$$
$$S(a,b)=([p12],-[p22])$$

Obliczamy odległość środka okręgu od prostej:
$$d=\frac{|[p4]\cdot[p12]+[p5]\cdot(-[p22])+[p6]|}{\sqrt{[p4]^{2}+[p5]^{2}}}=\frac{[dl]}{[dm]}\approx [d]$$

Aby prosta nie miała punktów wspólnych z okręgiem musi być spełniony warunek $d(S,k)>r.$
$$[d]>\sqrt{[r]+m^{2}}$$
$$[dk]>[r]+m^{2}$$
$$m^{2}<[dkr]$$
$$m\in(-\sqrt{[dkr]}, \sqrt{[dkr]})$$
\rozwStop


\odpStart
$m\in(-\sqrt{[dkr]}, \sqrt{[dkr]})$
\odpStop
\testStart
A.$m\in(-\sqrt{[dkr]}, \sqrt{[dkr]})$
B.$m=\sqrt{[dkr]}$
C.$m\in(-\sqrt{[dk]}, \sqrt{[dk]})$
D.$m\in[-\sqrt{[p12]}, \sqrt{[dkr]}]$
E.$m\in(-[p4], [p5])$
F.$m=[p22]$
G.$m\in\emptyset$

\testStop
\kluczStart
A
\kluczStop



\end{document}
