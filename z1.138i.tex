\documentclass[12pt, a4paper]{article}
\usepackage[utf8]{inputenc}
\usepackage{polski}

\usepackage{amsthm}  %pakiet do tworzenia twierdzeń itp.
\usepackage{amsmath} %pakiet do niektórych symboli matematycznych
\usepackage{amssymb} %pakiet do symboli mat., np. \nsubseteq
\usepackage{amsfonts}
\usepackage{graphicx} %obsługa plików graficznych z rozszerzeniem png, jpg
\theoremstyle{definition} %styl dla definicji
\newtheorem{zad}{} 
\title{Multizestaw zadań}
\author{Robert Fidytek}
%\date{\today}
\date{}
\newcounter{liczniksekcji}
\newcommand{\kategoria}[1]{\section{#1}} %olreślamy nazwę kateforii zadań
\newcommand{\zadStart}[1]{\begin{zad}#1\newline} %oznaczenie początku zadania
\newcommand{\zadStop}{\end{zad}}   %oznaczenie końca zadania
%Makra opcjonarne (nie muszą występować):
\newcommand{\rozwStart}[2]{\noindent \textbf{Rozwiązanie (autor #1 , recenzent #2): }\newline} %oznaczenie początku rozwiązania, opcjonarnie można wprowadzić informację o autorze rozwiązania zadania i recenzencie poprawności wykonania rozwiązania zadania
\newcommand{\rozwStop}{\newline}                                            %oznaczenie końca rozwiązania
\newcommand{\odpStart}{\noindent \textbf{Odpowiedź:}\newline}    %oznaczenie początku odpowiedzi końcowej (wypisanie wyniku)
\newcommand{\odpStop}{\newline}                                             %oznaczenie końca odpowiedzi końcowej (wypisanie wyniku)
\newcommand{\testStart}{\noindent \textbf{Test:}\newline} %ewentualne możliwe opcje odpowiedzi testowej: A. ? B. ? C. ? D. ? itd.
\newcommand{\testStop}{\newline} %koniec wprowadzania odpowiedzi testowych
\newcommand{\kluczStart}{\noindent \textbf{Test poprawna odpowiedź:}\newline} %klucz, poprawna odpowiedź pytania testowego (jedna literka): A lub B lub C lub D itd.
\newcommand{\kluczStop}{\newline} %koniec poprawnej odpowiedzi pytania testowego 
\newcommand{\wstawGrafike}[2]{\begin{figure}[h] \includegraphics[scale=#2] {#1} \end{figure}} %gdyby była potrzeba wstawienia obrazka, parametry: nazwa pliku, skala (jak nie wiesz co wpisać, to wpisz 1)

\begin{document}
\maketitle


\kategoria{Wikieł/Z1.138i}
\zadStart{Zadanie z Wikieł Z 1.138 i) moja wersja nr [nrWersji]}
%[a]:[2,3,4,5,6,7,8,9,10]
Wyznaczyć funckję odwrotną do danej funkcji $f$ określonej na zbiorze $\mathcal{D}_{f}$.\\
i) $f(x)=arctg(\frac{x}{[a]+x})\hspace{5mm}\mathcal{D}_{f}=\mathbb{R}\backslash\{-[a]\}$
\zadStop
\rozwStart{Wojciech Przybylski}{}
$$f(x)=arctg(\frac{x}{[a]+x})\hspace{5mm} \mathcal{D}_{f}=\mathbb{R}\backslash\{-[a]\}$$
$$y=arctg(\frac{x}{[a]+x})\Rightarrow tg(y)=\frac{x}{[a]+x} \Rightarrow x=\frac{[a]tg(y)}{1-tg(y)}$$
$$1-tg(y)\neq0 \Rightarrow y\neq\frac{\pi}{4}+2k\pi\hspace{3mm}k\in\mathbb{Z}$$
$$y=f^{-1}(x)=\frac{[a]tg(x)}{1-tg(x)} \mbox{ dla } x\in \mathbb{R}\backslash\{\frac{\pi}{4}+k\pi,\frac{\pi}{2}+k\pi\}\hspace{3mm}k\in\mathbb{Z}$$
\rozwStop
\odpStart
$f^{-1}(x)=\frac{[a]tg(x)}{1-tg(x)}  \mbox{ dla }x\in \mathbb{R}\backslash\{\frac{\pi}{4}+k\pi,\frac{\pi}{2}+k\pi\}\hspace{3mm}k\in\mathbb{Z}$
\odpStop
\testStart
A. $f^{-1}(x)=\frac{[a]tg(x)}{1-tg(x)}   \mbox{ dla }x\in \mathbb{R}\backslash\{\frac{\pi}{4}+k\pi,\frac{\pi}{2}+k\pi\}\hspace{3mm}k\in\mathbb{Z}$\\
B. $f^{-1}(x)=\frac{ctg(x)}{1+ctg(x)}  \mbox{ dla }x\in \mathbb{R}\backslash\{\frac{\pi}{3}+k\pi,\frac{\pi}{2}+k\pi\}\hspace{3mm}k\in\mathbb{Z}$\\
C. $f^{-1}(x)=\frac{[a]}{1+tg(x)} \mbox{ dla }x\in \mathbb{R}\backslash\{\frac{\pi}{3}+k\pi,\frac{\pi}{2}+k\pi\}\hspace{3mm}k\in\mathbb{Z}$\\
D. $f^{-1}(x)=\frac{[a]}{1-tg(x)} \mbox{ dla }x\in \mathbb{R}\backslash\{\frac{\pi}{4}+k\pi,\frac{\pi}{2}+k\pi\}\hspace{3mm}k\in\mathbb{Z}$\\
E. $f^{-1}(x)=\frac{[a]tg(x)}{1+tg(x)} \mbox{ dla }x\in \mathbb{R}\backslash\{\frac{\pi}{4},\frac{\pi}{2}\}\hspace{3mm}k\in\mathbb{Z}$\\
F. $f^{-1}(x)=\frac{[a]ctg(x)}{1-ctg(x)} \mbox{ dla }x\in \mathbb{R}$
\testStop
\kluczStart
A
\kluczStop



\end{document}