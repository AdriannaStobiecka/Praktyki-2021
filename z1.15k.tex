\documentclass[12pt, a4paper]{article}
\usepackage[utf8]{inputenc}
\usepackage{polski}

\usepackage{amsthm}  %pakiet do tworzenia twierdzeń itp.
\usepackage{amsmath} %pakiet do niektórych symboli matematycznych
\usepackage{amssymb} %pakiet do symboli mat., np. \nsubseteq
\usepackage{amsfonts}
\usepackage{graphicx} %obsługa plików graficznych z rozszerzeniem png, jpg
\theoremstyle{definition} %styl dla definicji
\newtheorem{zad}{} 
\title{Multizestaw zadań}
\author{Robert Fidytek}
%\date{\today}
\date{}
\newcounter{liczniksekcji}
\newcommand{\kategoria}[1]{\section{#1}} %olreślamy nazwę kateforii zadań
\newcommand{\zadStart}[1]{\begin{zad}#1\newline} %oznaczenie początku zadania
\newcommand{\zadStop}{\end{zad}}   %oznaczenie końca zadania
%Makra opcjonarne (nie muszą występować):
\newcommand{\rozwStart}[2]{\noindent \textbf{Rozwiązanie (autor #1 , recenzent #2): }\newline} %oznaczenie początku rozwiązania, opcjonarnie można wprowadzić informację o autorze rozwiązania zadania i recenzencie poprawności wykonania rozwiązania zadania
\newcommand{\rozwStop}{\newline}                                            %oznaczenie końca rozwiązania
\newcommand{\odpStart}{\noindent \textbf{Odpowiedź:}\newline}    %oznaczenie początku odpowiedzi końcowej (wypisanie wyniku)
\newcommand{\odpStop}{\newline}                                             %oznaczenie końca odpowiedzi końcowej (wypisanie wyniku)
\newcommand{\testStart}{\noindent \textbf{Test:}\newline} %ewentualne możliwe opcje odpowiedzi testowej: A. ? B. ? C. ? D. ? itd.
\newcommand{\testStop}{\newline} %koniec wprowadzania odpowiedzi testowych
\newcommand{\kluczStart}{\noindent \textbf{Test poprawna odpowiedź:}\newline} %klucz, poprawna odpowiedź pytania testowego (jedna literka): A lub B lub C lub D itd.
\newcommand{\kluczStop}{\newline} %koniec poprawnej odpowiedzi pytania testowego 
\newcommand{\wstawGrafike}[2]{\begin{figure}[h] \includegraphics[scale=#2] {#1} \end{figure}} %gdyby była potrzeba wstawienia obrazka, parametry: nazwa pliku, skala (jak nie wiesz co wpisać, to wpisz 1)

\begin{document}
\maketitle



\kategoria{Wikieł/Z1.15k}
\zadStart{Zadanie z Wikieł Z 1.15 k) moja wersja nr [nrWersji]}
%[a]:[1,2,3,4,5,6,7,8]
%[b]:[1,2,3,4,5,6,7,8]
%[a]=random.randint(1,8)
%[b]=random.randint(1,8)
%[bpa]=[b]-[a]
%[bpa2]=round(([bpa]/2),2)
%[b]<[a] and [bpa2]<[a]
Rozwiązać nierówność $|x+[a]|-|x|>[b]$
\zadStop
\rozwStart{Pascal Nawrocki}{Jakub Ulrych}
Rozpatrujemy 3 przypadki i bierzemy sumę rozwiązań:
\begin{enumerate}
\item (obie wartości bezwględne z przeciwnym znakiem) $$x\in(-\infty,-[a])$$ 
$$|x+[a]|-|x|>[b]$$
$$-x-[a]-(-x)>[b]$$
$$-[a]>[b]$$
$$x\in\emptyset$$
\item (pierwsza bez zmiany druga z przeciwnym)$$x\in[-[a],0)$$ 
$$|x+[a]|-|x|>[b]$$
$$x+[a]+x>[b]$$
$$2x>[b]-[a]$$
$$x>[bpa2]$$
Biorąc pod uwagę przedział w jakim się znajdujemy mamy:
$$x\in[[bpa2],0)$$
\item (obie bez zmiany) $$x\in[0,\infty)$$ 
$$|x+[a]|-|x|>[b]$$
$$x+[a]-x>[b]$$
$$[a]>[b]$$
Jako, że jest to zawsze prawda, rozwiązaniem będzie cały przedział , w którym się znajdujemy:
$$x\in[0,\infty)$$
\end{enumerate}
Podsumowując, rozwiązaniem jest suma tych 3 przypadków, zatem:
$$x\in[[bpa2],\infty)$$
\odpStop
\testStart
A.$x\in[[bpa2],\infty)$
B.$x\in\mathbb{R}$
C.$x\in(-\infty,[bpa2]]$
D.$x\in\emptyset$
\testStop
\kluczStart
A
\kluczStop


\end{document}