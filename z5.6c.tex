\documentclass[12pt, a4paper]{article}
\usepackage[utf8]{inputenc}
\usepackage{polski}

\usepackage{amsthm}  %pakiet do tworzenia twierdzeń itp.
\usepackage{amsmath} %pakiet do niektórych symboli matematycznych
\usepackage{amssymb} %pakiet do symboli mat., np. \nsubseteq
\usepackage{amsfonts}
\usepackage{graphicx} %obsługa plików graficznych z rozszerzeniem png, jpg
\theoremstyle{definition} %styl dla definicji
\newtheorem{zad}{} 
\title{Multizestaw zadań}
\author{Robert Fidytek}
%\date{\today}
\date{}
\newcounter{liczniksekcji}
\newcommand{\kategoria}[1]{\section{#1}} %olreślamy nazwę kateforii zadań
\newcommand{\zadStart}[1]{\begin{zad}#1\newline} %oznaczenie początku zadania
\newcommand{\zadStop}{\end{zad}}   %oznaczenie końca zadania
%Makra opcjonarne (nie muszą występować):
\newcommand{\rozwStart}[2]{\noindent \textbf{Rozwiązanie (autor #1 , recenzent #2): }\newline} %oznaczenie początku rozwiązania, opcjonarnie można wprowadzić informację o autorze rozwiązania zadania i recenzencie poprawności wykonania rozwiązania zadania
\newcommand{\rozwStop}{\newline}                                            %oznaczenie końca rozwiązania
\newcommand{\odpStart}{\noindent \textbf{Odpowiedź:}\newline}    %oznaczenie początku odpowiedzi końcowej (wypisanie wyniku)
\newcommand{\odpStop}{\newline}                                             %oznaczenie końca odpowiedzi końcowej (wypisanie wyniku)
\newcommand{\testStart}{\noindent \textbf{Test:}\newline} %ewentualne możliwe opcje odpowiedzi testowej: A. ? B. ? C. ? D. ? itd.
\newcommand{\testStop}{\newline} %koniec wprowadzania odpowiedzi testowych
\newcommand{\kluczStart}{\noindent \textbf{Test poprawna odpowiedź:}\newline} %klucz, poprawna odpowiedź pytania testowego (jedna literka): A lub B lub C lub D itd.
\newcommand{\kluczStop}{\newline} %koniec poprawnej odpowiedzi pytania testowego 
\newcommand{\wstawGrafike}[2]{\begin{figure}[h] \includegraphics[scale=#2] {#1} \end{figure}} %gdyby była potrzeba wstawienia obrazka, parametry: nazwa pliku, skala (jak nie wiesz co wpisać, to wpisz 1)

\begin{document}
\maketitle


\kategoria{Wikieł/Z5.6c}
\zadStart{Zadanie z Wikieł Z 5.6c) moja wersja nr [nrWersji]}
%[x]:[3,5,7,9,11,13,15,17,19,20,21,22]
%[y]:[2,3,4,5,6,7,8,9,10,11,12,15,17]
%[a]=random.randint(3,50)
%[m]=random.randint(1,50)
%[w]=[m]*[a]
%[k]=[a]-1
Obliczyć pochodną funkcji $f$ oraz określić dziedzinę funkcji $f$ i funkcji pochodnej $f'$.\\
$f(x)=[m]\sqrt[[a]]{-x}sin(x)$
\zadStop
\rozwStart{Katarzyna Filipowicz}{}
Dziedzina $D_f: -x \geq 0 \Rightarrow x\leq 0 \Rightarrow x \in (-\infty,0]$
$$
f'(x)=[m]\left(\frac{-sin(x)}{[a](-x)^{\frac{[k]}{[a]}}}+\sqrt[[a]]{-x}cos(x)\right)=[m]\left(\frac{[a](-x)^{\frac{[k]}{[a]}}\cdot \sqrt[[a]]{-x} \cdot cos(x)-sin(x)}{[a](-x)^{\frac{[k]}{[a]}}}\right)=
$$  $$
=\frac{-[w] x\cdot cos(x)-[m]sin(x)}{[a](-x)^{\frac{[k]}{[a]}}}
$$
Dziedzina $D_{f'}: ([a]\sqrt[[a]]{-x}) \neq 0 \wedge -x\geq 0 \Rightarrow x < 0 \Rightarrow   x \in (-\infty,0)$
\rozwStop
\odpStart
$f'(x)=\frac{-[w] x\cdot cos(x)-[m]sin(x)}{[a](-x)^{\frac{[k]}{[a]}}}, D_f:x \in (-\infty,0], D_{f'}:x \in (-\infty,0)$
\odpStop
\testStart
A.$f'(x)=\frac{-[w] x\cdot cos(x)-[m]sin(x)}{[a](-x)^{\frac{[k]}{[a]}}}, D_f:x \in (-\infty,0], D_{f'}:x \in (-\infty,0)$
B.$f'(x)=\frac{[w] x\cdot cos(x)-[m]sin(x)}{[a]\sqrt[[a]]{-x}}, D_f:x \in (-\infty,0], D_{f'}:x \in (-\infty,0)$
C.$f'(x)=\frac{-[w]\cdot cos(x)-[m]sin(x)}{[a]\sqrt[[a]]{-x}}, D_f:x \in (-\infty,0], D_{f'}:x \in (-\infty,0)$
D.$f'(x)=\frac{-[w] x\cdot sin(x)-[m]sin(x)}{[a]\sqrt[[a]]{-x}}, D_f:x \in (-\infty,0], D_{f'}:x \in (-\infty,0)$
E.$f'(x)=\frac{-[w] x\cdot cos(x)}{[a]\sqrt[[a]]{-x}}, D_f:x \in (-\infty,0], D_{f'}:x \in (-\infty,0)$
F.$f'(x)=\frac{-[w] x\cdot cos(x)+sin(x)}{[a]\sqrt[[a]]{-x}}, D_f:x \in (-\infty,0], D_{f'}:x \in (-\infty,0)$
G.$f'(x)=\frac{-[w] x\cdot cos(x)-[m]sin(x)}{\sqrt[[a]]{-x}}, D_f:x \in (-\infty,0], D_{f'}:x \in (-\infty,0)$
H$f'(x)=\frac{[w] x\cdot cos(x)+[m]sin(x)}{[a]\sqrt[[a]]{x}}, D_f:x \in (-\infty,0], D_{f'}:x \in (-\infty,0)$
I.$f'(x)=\frac{-[w] x\cdot cos(x)-[m]sin(x)}{[a]\sqrt[[a]]{-x}}, D_f:x \in [0,\infty), D_{f'}:x \in (0,\infty]$
\testStop
\kluczStart
A
\kluczStop



\end{document}