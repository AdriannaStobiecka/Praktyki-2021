\documentclass[12pt, a4paper]{article}
\usepackage[utf8]{inputenc}
\usepackage{polski}

\usepackage{amsthm}  %pakiet do tworzenia twierdzeń itp.
\usepackage{amsmath} %pakiet do niektórych symboli matematycznych
\usepackage{amssymb} %pakiet do symboli mat., np. \nsubseteq
\usepackage{amsfonts}
\usepackage{graphicx} %obsługa plików graficznych z rozszerzeniem png, jpg
\theoremstyle{definition} %styl dla definicji
\newtheorem{zad}{} 
\title{Multizestaw zadań}
\author{Robert Fidytek}
%\date{\today}
\date{}
\newcounter{liczniksekcji}
\newcommand{\kategoria}[1]{\section{#1}} %olreślamy nazwę kateforii zadań
\newcommand{\zadStart}[1]{\begin{zad}#1\newline} %oznaczenie początku zadania
\newcommand{\zadStop}{\end{zad}}   %oznaczenie końca zadania
%Makra opcjonarne (nie muszą występować):
\newcommand{\rozwStart}[2]{\noindent \textbf{Rozwiązanie (autor #1 , recenzent #2): }\newline} %oznaczenie początku rozwiązania, opcjonarnie można wprowadzić informację o autorze rozwiązania zadania i recenzencie poprawności wykonania rozwiązania zadania
\newcommand{\rozwStop}{\newline}                                            %oznaczenie końca rozwiązania
\newcommand{\odpStart}{\noindent \textbf{Odpowiedź:}\newline}    %oznaczenie początku odpowiedzi końcowej (wypisanie wyniku)
\newcommand{\odpStop}{\newline}                                             %oznaczenie końca odpowiedzi końcowej (wypisanie wyniku)
\newcommand{\testStart}{\noindent \textbf{Test:}\newline} %ewentualne możliwe opcje odpowiedzi testowej: A. ? B. ? C. ? D. ? itd.
\newcommand{\testStop}{\newline} %koniec wprowadzania odpowiedzi testowych
\newcommand{\kluczStart}{\noindent \textbf{Test poprawna odpowiedź:}\newline} %klucz, poprawna odpowiedź pytania testowego (jedna literka): A lub B lub C lub D itd.
\newcommand{\kluczStop}{\newline} %koniec poprawnej odpowiedzi pytania testowego 
\newcommand{\wstawGrafike}[2]{\begin{figure}[h] \includegraphics[scale=#2] {#1} \end{figure}} %gdyby była potrzeba wstawienia obrazka, parametry: nazwa pliku, skala (jak nie wiesz co wpisać, to wpisz 1)

\begin{document}
\maketitle


\kategoria{Wikieł/Z5.18 g}
\zadStart{Zadanie z Wikieł Z 5.18 g) moja wersja nr [nrWersji]}
%[a]:[2,3,4,5,6,7,8,9]
%[b]:[3,5,7,9,11,13]
%[c]=random.randint(2,10)
%math.gcd([a],[b])==1
Oblicz granicę $\lim_{x \rightarrow 0^+} \left( \frac{[a]}{[b]x}\right)^{\tg([c]x)}$.
\zadStop
\rozwStart{Joanna Świerzbin}{}
$$ \lim_{x \rightarrow 0^+} \left( \frac{[a]}{[b]x}\right)^{\tg([c]x)} = \lim_{x \rightarrow 0^+} e^{\ln \left( \frac{[a]}{[b]x}\right)^{\tg([c]x)}}
= \lim_{x \rightarrow 0^+} e^{{\tg([c]x)} \ln \left( \frac{[a]}{[b]x}\right)}= e^{ \lim_{x \rightarrow 0^+}{\tg([c]x)} \ln \left( \frac{[a]}{[b]x}\right)}$$
Policzmy ${ \lim_{x \rightarrow 0^+}{\tg([c]x)} \ln \left( \frac{[a]}{[b]x}\right)}$.
$${ \lim_{x \rightarrow 0^+}{\tg([c]x)} \ln \left( \frac{[a]}{[b]x}\right)} = { \lim_{x \rightarrow 0^+}\frac{\sin([c]x) \ln \left( \frac{[a]}{[b]x}\right)}{\cos([c]x)}} = 0$$
Podstawmy do początkowego przykładu.
$$e^{ \lim_{x \rightarrow 0^+}{\tg([c]x)} \ln \left( \frac{[a]}{[b]x}\right)} = e^0 = 1$$
\rozwStop
\odpStart
$\lim_{x \rightarrow 0^+} \left( \frac{[a]}{[b]x}\right)^{\tg([c]x)} = 1$
\odpStop
\testStart
A. $\lim_{x \rightarrow 0^+} \left( \frac{[a]}{[b]x}\right)^{\tg([c]x)} = 1$\\
B. $\lim_{x \rightarrow 0^+} \left( \frac{[a]}{[b]x}\right)^{\tg([c]x)} = e$\\
C. $\lim_{x \rightarrow 0^+} \left( \frac{[a]}{[b]x}\right)^{\tg([c]x)} = 0$\\
D. $\lim_{x \rightarrow 0^+} \left( \frac{[a]}{[b]x}\right)^{\tg([c]x)} = \infty$\\
E. $\lim_{x \rightarrow 0^+} \left( \frac{[a]}{[b]x}\right)^{\tg([c]x)} = -\infty$\\
F. $\lim_{x \rightarrow 0^+} \left( \frac{[a]}{[b]x}\right)^{\tg([c]x)} = -2$
\testStop
\kluczStart
A
\kluczStop



\end{document}