\documentclass[12pt, a4paper]{article}
\usepackage[utf8]{inputenc}
\usepackage{polski}

\usepackage{amsthm}  %pakiet do tworzenia twierdzeń itp.
\usepackage{amsmath} %pakiet do niektórych symboli matematycznych
\usepackage{amssymb} %pakiet do symboli mat., np. \nsubseteq
\usepackage{amsfonts}
\usepackage{graphicx} %obsługa plików graficznych z rozszerzeniem png, jpg
\theoremstyle{definition} %styl dla definicji
\newtheorem{zad}{} 
\title{Multizestaw zadań}
\author{Robert Fidytek}
%\date{\today}
\date{}
\newcounter{liczniksekcji}
\newcommand{\kategoria}[1]{\section{#1}} %olreślamy nazwę kateforii zadań
\newcommand{\zadStart}[1]{\begin{zad}#1\newline} %oznaczenie początku zadania
\newcommand{\zadStop}{\end{zad}}   %oznaczenie końca zadania
%Makra opcjonarne (nie muszą występować):
\newcommand{\rozwStart}[2]{\noindent \textbf{Rozwiązanie (autor #1 , recenzent #2): }\newline} %oznaczenie początku rozwiązania, opcjonarnie można wprowadzić informację o autorze rozwiązania zadania i recenzencie poprawności wykonania rozwiązania zadania
\newcommand{\rozwStop}{\newline}                                            %oznaczenie końca rozwiązania
\newcommand{\odpStart}{\noindent \textbf{Odpowiedź:}\newline}    %oznaczenie początku odpowiedzi końcowej (wypisanie wyniku)
\newcommand{\odpStop}{\newline}                                             %oznaczenie końca odpowiedzi końcowej (wypisanie wyniku)
\newcommand{\testStart}{\noindent \textbf{Test:}\newline} %ewentualne możliwe opcje odpowiedzi testowej: A. ? B. ? C. ? D. ? itd.
\newcommand{\testStop}{\newline} %koniec wprowadzania odpowiedzi testowych
\newcommand{\kluczStart}{\noindent \textbf{Test poprawna odpowiedź:}\newline} %klucz, poprawna odpowiedź pytania testowego (jedna literka): A lub B lub C lub D itd.
\newcommand{\kluczStop}{\newline} %koniec poprawnej odpowiedzi pytania testowego 
\newcommand{\wstawGrafike}[2]{\begin{figure}[h] \includegraphics[scale=#2] {#1} \end{figure}} %gdyby była potrzeba wstawienia obrazka, parametry: nazwa pliku, skala (jak nie wiesz co wpisać, to wpisz 1)

\begin{document}
\maketitle


\kategoria{Wikieł/Z1.36g}
\zadStart{Zadanie z Wikieł Z 1.36 g) moja wersja nr [nrWersji]}
%[a]:[2, 3, 4, 5, 6, 7, 8, 9, 10, 11, 12, 13, 14, 15, 16, 17, 18, 19, 20]
%[b]:[2, 3, 4, 5, 6, 7, 8, 9, 10, 11, 12, 13, 14, 15, 16, 17, 18, 19, 20]
%[c]:[2, 3, 4, 5, 6, 7, 8, 9, 10, 11, 12, 13, 14, 15, 16, 17, 18, 19, 20]
%[1x]=(-2)*[a]+[b]
%[2x]=2*[a]+[b]
%[ba]=[b]*[a]
%[1reszta]=([a]*[a]-[ba]-[c])
%[2reszta]=([a]*[a]+[ba]-[c])
%[delta1]=[b]*[b]+4*[c]
%[sdelta1]=pow([delta1],1/2)
%[x1]=((-1)*[1x]-[sdelta1])/2
%[x2]=((-1)*[1x]+[sdelta1])/2
%[x3]=([2x]-[sdelta1])/2
%[x4]=([2x]+[sdelta1])/2
%[sdelta11]=int([sdelta1].real)
%[x11]=int([x1].real)
%[x21]=int([x2].real)
%[x31]=int([x3].real)
%[x41]=int([x4].real)
%[delta1]>0 and [sdelta1].is_integer()==True and [x1].is_integer()==True and [x11]!=0 and [x41]!=0
Rozwiązać równanie. $|x-[a]|^{2}-[b]|x-[a]|-[c]=0$
\zadStop
\rozwStart{Jakub Ulrych}{}
$$|x-[a]|^{2}-[b]|x-[a]|-[c]=0$$
Liczymy miejsca zerowe obu wartości bezwzględnych
$$(x-[a])^{2}=0,(x-[a])=0$$
$$x=[a]\text{ 2-krotny},x=[a]\text{ 1-krotny}$$
Robimy przedziały
$$\textbf{ 1) }x\in(-\infty,[a])\vee\textbf{2) }x\in[[a],\infty)$$
W przedziale 1) pierwsza wartość bezwzględna przyjmuje warości dodatnie, a druga ujemne więc
$$(x-[a])^{2}-[b](-x+[a])-[c]=0$$
$$x^{2}+([1x])x+[1reszta]=0$$
$$\Delta_{1}=[delta1]\Rightarrow\sqrt{\Delta_{1}}=[sdelta11]$$
$$x_{1}=[x11]\in1),x_{2}=[x21]\notin1)$$
W przedziale 2) pierwsza i druga wartość bezwzględna przyjmują wartości dodatnie więc
$$(x-[a])^{2}-[b](x-[a])-[c]=0$$
$$x^{2}-[2x]x+[2reszta]=0$$
$$\Delta_{2}=[delta1]\Rightarrow\sqrt{\Delta_{2}}=[sdelta11]$$
$$x_{3}=[x31]\notin2),x_{4}=[x41]\in2)$$
$$\textbf{1) }\vee\textbf{ 2)}\Rightarrow x\in\{[x11],[x41]\}$$
\rozwStop
\odpStart
$$x\in\{[x11],[x41]\}$$
\odpStop
\testStart
A.$x\in\{[x11],[x41]\}$
B.$x\in\{[x11]\}$
C.$x\in\{-[x41],[x41]\}$
D.$x\in\{-[x11],[x41]\}$
\testStop
\kluczStart
A
\kluczStop



\end{document}