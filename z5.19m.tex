\documentclass[12pt, a4paper]{article}
\usepackage[utf8]{inputenc}
\usepackage{polski}

\usepackage{amsthm}  %pakiet do tworzenia twierdzeń itp.
\usepackage{amsmath} %pakiet do niektórych symboli matematycznych
\usepackage{amssymb} %pakiet do symboli mat., np. \nsubseteq
\usepackage{amsfonts}
\usepackage{graphicx} %obsługa plików graficznych z rozszerzeniem png, jpg
\theoremstyle{definition} %styl dla definicji
\newtheorem{zad}{} 
\title{Multizestaw zadań}
\author{Robert Fidytek}
%\date{\today}
\date{}
\newcounter{liczniksekcji}
\newcommand{\kategoria}[1]{\section{#1}} %olreślamy nazwę kateforii zadań
\newcommand{\zadStart}[1]{\begin{zad}#1\newline} %oznaczenie początku zadania
\newcommand{\zadStop}{\end{zad}}   %oznaczenie końca zadania
%Makra opcjonarne (nie muszą występować):
\newcommand{\rozwStart}[2]{\noindent \textbf{Rozwiązanie (autor #1 , recenzent #2): }\newline} %oznaczenie początku rozwiązania, opcjonarnie można wprowadzić informację o autorze rozwiązania zadania i recenzencie poprawności wykonania rozwiązania zadania
\newcommand{\rozwStop}{\newline}                                            %oznaczenie końca rozwiązania
\newcommand{\odpStart}{\noindent \textbf{Odpowiedź:}\newline}    %oznaczenie początku odpowiedzi końcowej (wypisanie wyniku)
\newcommand{\odpStop}{\newline}                                             %oznaczenie końca odpowiedzi końcowej (wypisanie wyniku)
\newcommand{\testStart}{\noindent \textbf{Test:}\newline} %ewentualne możliwe opcje odpowiedzi testowej: A. ? B. ? C. ? D. ? itd.
\newcommand{\testStop}{\newline} %koniec wprowadzania odpowiedzi testowych
\newcommand{\kluczStart}{\noindent \textbf{Test poprawna odpowiedź:}\newline} %klucz, poprawna odpowiedź pytania testowego (jedna literka): A lub B lub C lub D itd.
\newcommand{\kluczStop}{\newline} %koniec poprawnej odpowiedzi pytania testowego 
\newcommand{\wstawGrafike}[2]{\begin{figure}[h] \includegraphics[scale=#2] {#1} \end{figure}} %gdyby była potrzeba wstawienia obrazka, parametry: nazwa pliku, skala (jak nie wiesz co wpisać, to wpisz 1)

\begin{document}
\maketitle


\kategoria{Wikieł/Z5.19 m}
\zadStart{Zadanie z Wikieł Z 5.19 m) moja wersja nr [nrWersji]}
%[a]:[2,3,4,5,6,7,8,9]
%[c]:[2,3,4,5,6,7,8,9]
%[b]=random.randint(2,10)
%[a]!=0
Oblicz granicę $\lim_{x \rightarrow \infty} \left( [a]+\frac{[b]}{x^2}\right)^{[c]x}$.
\zadStop
\rozwStart{Joanna Świerzbin}{}
$$\lim_{x \rightarrow \infty} \left( [a]+\frac{[b]}{x^2}\right)^{[c]x} = \lim_{x \rightarrow \infty} \left(\frac{[a]x^2+[b]}{x^2}\right)^{[c]x} 
= \lim_{x \rightarrow \infty} \frac{([a]x^2+[b])^{[c]x}}{x^{2\cdot[c]x}} = $$
$$ = \lim_{x \rightarrow \infty} {([a]x^2+[b])^{[c]x}}{x^{-2\cdot[c]x}} = \lim_{x \rightarrow \infty} e^{\ln({([a]x^2+[b])^{[c]x}}{x^{-2\cdot[c]x}})} = $$
$$= \lim_{x \rightarrow \infty} e^{{[c]x}\ln([a]x^2+[b]){-2\cdot[c]x}\ln({x})} =  e^{\lim_{x \rightarrow \infty}({[c]x}(\ln([a]x^2+[b]){-2}\ln({x})))} =$$
$$ =  e^{\lim_{x \rightarrow \infty} \frac{\ln([a]x^2+[b])-2\ln({x})}{\frac{1}{[c]x}} } $$
Otrzymujemy w potędze $ \left[ \frac{0}{0} \right] $ więc możemy skorzystać z twierdzenia de l'Hospitala.
$${\lim_{x\rightarrow \infty}\frac{\ln([a]x^2+[b])-2\ln({x})}{\frac{1}{[c]x}}} = \lim_{x \rightarrow \infty} \frac{\frac{-2}{x}+\frac{2\cdot[a]x}{[a]x^2+[b]}}{-\frac{[c]}{([c]x)^2}} = \lim_{x \rightarrow \infty} \frac{\frac{-2\cdot([a]x^2+[b])+2\cdot[a]x^2}{x([a]x^2+[b])}}{-\frac{1}{[c]x^2}}=$$
$$= \lim_{x \rightarrow \infty} \frac{-2\cdot[a]x^2-2\cdot[b]+2\cdot[a]x^2}{[a]x^3+[b]x} \left(-[c]x^2 \right)= \lim_{x \rightarrow \infty} \frac{(-[c]x)(-2\cdot[a]x^2-2\cdot[b]+2\cdot[a]x^2)}{[a]x^2+[b]}=$$
$$= \lim_{x \rightarrow \infty} \frac{2\cdot[b]\cdot[c]x}{[a]x^2+[b]}= \lim_{x \rightarrow \infty} \frac{\frac{2\cdot[b]\cdot[c]}{x}}{[a]+\frac{[b]}{x^2}} =0$$
Podstawmy do początkowego przykładu.
$$e^{\lim_{x \rightarrow \infty} \frac{\ln([a]x^2+[b])-2\ln({x})}{\frac{1}{[c]x}}} = e^0 = 1$$
\rozwStop
\odpStart
$\lim_{x \rightarrow \infty} \left( [a]+\frac{[b]}{x^2}\right)^{[c]x}=1 $
\odpStop
\testStart
A. $\lim_{x \rightarrow \infty} \left( [a]+\frac{[b]}{x^2}\right)^{[c]x}=1 $\\
B. $\lim_{x \rightarrow \infty} \left( [a]+\frac{[b]}{x^2}\right)^{[c]x}=0 $\\
C. $\lim_{x \rightarrow \infty} \left( [a]+\frac{[b]}{x^2}\right)^{[c]x}=[a] $\\
D. $\lim_{x \rightarrow \infty} \left( [a]+\frac{[b]}{x^2}\right)^{[c]x}=\infty $\\
E. $\lim_{x \rightarrow \infty} \left( [a]+\frac{[b]}{x^2}\right)^{[c]x}=-\infty $\\
F. $\lim_{x \rightarrow \infty} \left( [a]+\frac{[b]}{x^2}\right)^{[c]x}=\frac{1}{[a]} $
\testStop
\kluczStart
A
\kluczStop



\end{document}