\documentclass[12pt, a4paper]{article}
\usepackage[utf8]{inputenc}
\usepackage{polski}

\usepackage{amsthm}  %pakiet do tworzenia twierdzeń itp.
\usepackage{amsmath} %pakiet do niektórych symboli matematycznych
\usepackage{amssymb} %pakiet do symboli mat., np. \nsubseteq
\usepackage{amsfonts}
\usepackage{graphicx} %obsługa plików graficznych z rozszerzeniem png, jpg
\theoremstyle{definition} %styl dla definicji
\newtheorem{zad}{} 
\title{Multizestaw zadań}
\author{Robert Fidytek}
%\date{\today}
\date{}
\newcounter{liczniksekcji}
\newcommand{\kategoria}[1]{\section{#1}} %olreślamy nazwę kateforii zadań
\newcommand{\zadStart}[1]{\begin{zad}#1\newline} %oznaczenie początku zadania
\newcommand{\zadStop}{\end{zad}}   %oznaczenie końca zadania
%Makra opcjonarne (nie muszą występować):
\newcommand{\rozwStart}[2]{\noindent \textbf{Rozwiązanie (autor #1 , recenzent #2): }\newline} %oznaczenie początku rozwiązania, opcjonarnie można wprowadzić informację o autorze rozwiązania zadania i recenzencie poprawności wykonania rozwiązania zadania
\newcommand{\rozwStop}{\newline}                                            %oznaczenie końca rozwiązania
\newcommand{\odpStart}{\noindent \textbf{Odpowiedź:}\newline}    %oznaczenie początku odpowiedzi końcowej (wypisanie wyniku)
\newcommand{\odpStop}{\newline}                                             %oznaczenie końca odpowiedzi końcowej (wypisanie wyniku)
\newcommand{\testStart}{\noindent \textbf{Test:}\newline} %ewentualne możliwe opcje odpowiedzi testowej: A. ? B. ? C. ? D. ? itd.
\newcommand{\testStop}{\newline} %koniec wprowadzania odpowiedzi testowych
\newcommand{\kluczStart}{\noindent \textbf{Test poprawna odpowiedź:}\newline} %klucz, poprawna odpowiedź pytania testowego (jedna literka): A lub B lub C lub D itd.
\newcommand{\kluczStop}{\newline} %koniec poprawnej odpowiedzi pytania testowego 
\newcommand{\wstawGrafike}[2]{\begin{figure}[h] \includegraphics[scale=#2] {#1} \end{figure}} %gdyby była potrzeba wstawienia obrazka, parametry: nazwa pliku, skala (jak nie wiesz co wpisać, to wpisz 1)

\begin{document}
\maketitle


\kategoria{Wikieł/Z2.8}
\kategoria{Matematyka z Wikieł P 1.8)}
\zadStart{Zadanie z Wikieł Z 2.8 moja wersja nr [nrWersji]}
%[v1]:[1,2,3,4]
%[v2]:[1,2,3,4]
%[v3]:[1,2,3,4]
%[u1]:[1,2,3,4]
%[u2]:[1,2,3,4]
%[u3]:[1,2,3,4]
%[ku1]=[u1]*[u1]
%[ku2]=[u2]*[u2]
%[ku3]=[u3]*[u3]
%[kv1]=[v1]*[v1]
%[kv2]=[v2]*[v2]
%[kv3]=[v3]*[v3]
%[U]=[ku1]+[ku2]+[ku3]
%[V]=[kv1]+[kv2]+[kv3]
%[a]=[u1]*[v1]
%[b]=[u2]*[v2]
%[c]=[u3]*[v3]
%[abc]=[a]+[b]+[c]
%[9V]=9*[V]
%[12abc]=12*[abc]
%[4U]=4*[U]
%[x]=[12abc]+[4U]
%[w]=[9V]-[12abc]+[4U]
%[9V]>[x]
Obliczyć długość wektora $\vec a=3\vec v-2\vec u$, gdzie $\vec u=[[u1],[u2],[u3]],\\ \vec v=[[v1],[v2],[v3]]$.
\zadStop
\rozwStart{Aleksandra Pasińska}{}
$$|\vec u|=\sqrt{[u1]^2+[u2]^2+[u3]^2}=\sqrt{[U]}$$ 
$$|\vec v|=\sqrt{[v1]^2+[v2]^2+[v3]^2}=\sqrt{[V]}$$ 
$$\vec u\circ \vec v=[u1]\cdot [v1]+[u2]\cdot [v2]+[u3]\cdot [v3]=[a]+[b]+[c]=[abc]$$
$$(\vec a)^2=(3\vec v-2\vec u)^2=9|\vec v|^2-12(\vec v \circ \vec u)+4|\vec u|^2=9\cdot [V]-12\cdot [abc]+4\cdot [U]=[w]$$
$$|\vec a|=\sqrt{[w]}$$
\rozwStop
\odpStart
$|\vec a|=\sqrt{[w]}$
\odpStop
\testStart
A.$|\vec a|=\sqrt{[w]}$
B.$|\vec a|=-\infty$
C.$|\vec a|=\infty$
D.$|\vec a|=e^{[v1]}$
E.$|\vec a|=e^{-2}$
F.$|\vec a|=0$
G.$|\vec a|=1$
\testStop
\kluczStart
A
\kluczStop

\end{document}