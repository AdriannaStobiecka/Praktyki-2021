\documentclass[12pt, a4paper]{article}
\usepackage[utf8]{inputenc}
\usepackage{polski}

\usepackage{amsthm}  %pakiet do tworzenia twierdzeń itp.
\usepackage{amsmath} %pakiet do niektórych symboli matematycznych
\usepackage{amssymb} %pakiet do symboli mat., np. \nsubseteq
\usepackage{amsfonts}
\usepackage{graphicx} %obsługa plików graficznych z rozszerzeniem png, jpg
\theoremstyle{definition} %styl dla definicji
\newtheorem{zad}{} 
\title{Multizestaw zadań}
\author{Robert Fidytek}
%\date{\today}
\date{}
\newcounter{liczniksekcji}
\newcommand{\kategoria}[1]{\section{#1}} %olreślamy nazwę kateforii zadań
\newcommand{\zadStart}[1]{\begin{zad}#1\newline} %oznaczenie początku zadania
\newcommand{\zadStop}{\end{zad}}   %oznaczenie końca zadania
%Makra opcjonarne (nie muszą występować):
\newcommand{\rozwStart}[2]{\noindent \textbf{Rozwiązanie (autor #1 , recenzent #2): }\newline} %oznaczenie początku rozwiązania, opcjonarnie można wprowadzić informację o autorze rozwiązania zadania i recenzencie poprawności wykonania rozwiązania zadania
\newcommand{\rozwStop}{\newline}                                            %oznaczenie końca rozwiązania
\newcommand{\odpStart}{\noindent \textbf{Odpowiedź:}\newline}    %oznaczenie początku odpowiedzi końcowej (wypisanie wyniku)
\newcommand{\odpStop}{\newline}                                             %oznaczenie końca odpowiedzi końcowej (wypisanie wyniku)
\newcommand{\testStart}{\noindent \textbf{Test:}\newline} %ewentualne możliwe opcje odpowiedzi testowej: A. ? B. ? C. ? D. ? itd.
\newcommand{\testStop}{\newline} %koniec wprowadzania odpowiedzi testowych
\newcommand{\kluczStart}{\noindent \textbf{Test poprawna odpowiedź:}\newline} %klucz, poprawna odpowiedź pytania testowego (jedna literka): A lub B lub C lub D itd.
\newcommand{\kluczStop}{\newline} %koniec poprawnej odpowiedzi pytania testowego 
\newcommand{\wstawGrafike}[2]{\begin{figure}[h] \includegraphics[scale=#2] {#1} \end{figure}} %gdyby była potrzeba wstawienia obrazka, parametry: nazwa pliku, skala (jak nie wiesz co wpisać, to wpisz 1)

\begin{document}
\maketitle


\kategoria{Wikieł/Z4.9b}
\zadStart{Zadanie z Wikieł Z 4.9 b) moja wersja nr 1}
%[a]:[2,3,4,5,6,7,8]
%[b]:[2,3,4,5,6,7,8]
%[c]:[2,3,4,5,6,7,8]
%[d]:[2,3,4,5,6,7,8]
%[a]=random.randint(2,15)
%[b]=random.randint(2,15)
%[c]=random.randint(2,15)
%[d]=random.randint(2,15)
%[ca]=([c]/[a])
%[ca1]=int([ca])
%[c]>[a] and [c]>[b] and [c]>[d] and [c]!=1 and [ca]>2 and [ca].is_integer()==True and math.gcd([b],[c])==1 and math.gcd([d],[c])==1
Obliczyć granicę funkcji $\lim_{x \to \infty}\bigg(\frac{[a]^{x+2}+[b]^{x}}{[c]^{x}-[d]^{x}}\bigg)$.
\zadStop
\rozwStart{Jakub Ulrych}{}
$$\lim_{x \to \infty}\bigg(\frac{[a]^{x+2}+[b]^{x}}{[c]^{x}-[d]^{x}}\bigg)$$
$$\lim_{x \to \infty}\bigg(\frac{[c]^{x}\big(\frac{[a]^{2}[a]^{x}}{[c]^{x}}+\frac{[b]^{x}}{[c]^{x}}\big)}{[c]^{x}\big(1-\frac{[d]^{x}}{[c]^{x}}\big)}\bigg)$$
$$\lim_{x \to \infty}\bigg(\frac{\frac{[a]^{2}}{[ca1]^{x}}+(\frac{[b]}{[c]})^{x}}{1-(\frac{[d]}{[c]})^{x}}\bigg)$$
$$\frac{\lim_{x \to \infty}\big(\frac{[a]^{2}}{[ca1]^{x}}+(\frac{[b]}{[c]})^{x}\big)}{\lim_{x \to \infty}\big(1-(\frac{[d]}{[c]})^{x}\big)}$$
$$\frac{0+0}{1-0}=0$$
\rozwStop
\odpStart
$$0$$
\odpStop
\testStart
A.$0$
B.$[a]$
C.$-[a]$
D.$[b]$
\testStop
\kluczStart
A
\kluczStop



\end{document}