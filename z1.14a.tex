\documentclass[12pt, a4paper]{article}
\usepackage[utf8]{inputenc}
\usepackage{polski}

\usepackage{amsthm}  %pakiet do tworzenia twierdzeń itp.
\usepackage{amsmath} %pakiet do niektórych symboli matematycznych
\usepackage{amssymb} %pakiet do symboli mat., np. \nsubseteq
\usepackage{amsfonts}
\usepackage{graphicx} %obsługa plików graficznych z rozszerzeniem png, jpg
\theoremstyle{definition} %styl dla definicji
\newtheorem{zad}{} 
\title{Multizestaw zadań}
\author{Robert Fidytek}
%\date{\today}
\date{}
\newcounter{liczniksekcji}
\newcommand{\kategoria}[1]{\section{#1}} %olreślamy nazwę kateforii zadań
\newcommand{\zadStart}[1]{\begin{zad}#1\newline} %oznaczenie początku zadania
\newcommand{\zadStop}{\end{zad}}   %oznaczenie końca zadania
%Makra opcjonarne (nie muszą występować):
\newcommand{\rozwStart}[2]{\noindent \textbf{Rozwiązanie (autor #1 , recenzent #2): }\newline} %oznaczenie początku rozwiązania, opcjonarnie można wprowadzić informację o autorze rozwiązania zadania i recenzencie poprawności wykonania rozwiązania zadania
\newcommand{\rozwStop}{\newline}                                            %oznaczenie końca rozwiązania
\newcommand{\odpStart}{\noindent \textbf{Odpowiedź:}\newline}    %oznaczenie początku odpowiedzi końcowej (wypisanie wyniku)
\newcommand{\odpStop}{\newline}                                             %oznaczenie końca odpowiedzi końcowej (wypisanie wyniku)
\newcommand{\testStart}{\noindent \textbf{Test:}\newline} %ewentualne możliwe opcje odpowiedzi testowej: A. ? B. ? C. ? D. ? itd.
\newcommand{\testStop}{\newline} %koniec wprowadzania odpowiedzi testowych
\newcommand{\kluczStart}{\noindent \textbf{Test poprawna odpowiedź:}\newline} %klucz, poprawna odpowiedź pytania testowego (jedna literka): A lub B lub C lub D itd.
\newcommand{\kluczStop}{\newline} %koniec poprawnej odpowiedzi pytania testowego 
\newcommand{\wstawGrafike}[2]{\begin{figure}[h] \includegraphics[scale=#2] {#1} \end{figure}} %gdyby była potrzeba wstawienia obrazka, parametry: nazwa pliku, skala (jak nie wiesz co wpisać, to wpisz 1)

\begin{document}
\maketitle



\kategoria{Wikieł/Z1.14a}
\zadStart{Zadanie z Wikieł Z 1.14 a) moja wersja nr [nrWersji]}
%[a]:[1,2,3]
%[b]:[1,2,3]
%[c]:[1,2,3]
%[d]:[1,2,3]
%[e]:[1,2,3]
%[f]:[1,2,3]
%[a]=random.randint(2,10)
%[e]=random.randint(2,10)
%[c]=random.randint(1,10)
%[d]=random.randint(2,10)
%[b]=random.randint(2,10)
%[f]=random.randint(1,10)
%[g]=2*[a]
%math.gcd([a],[d])==1
Rozwiązać rówanie $|x+[a]|=x+[a]$
\zadStop
\rozwStart{Pascal Nawrocki}{Jakub Ulrych}
Zadanie będziemy rozpatrywać w dwóch przedziałach:
\begin{enumerate}
\item $x\in (-\infty,-[a])$
\item $x\in[-[a],+\infty)$
\end{enumerate}
Zacznijmy rozwiązywać:
\begin{enumerate}
\item dla $x\in (-\infty,-[a])$ mamy:
$$-x-[a]=x+[a]$$
$$2x=-[g]$$
$$x=-[a]$$
Jako, że nasz przedział z prawej strony nie jest ostro-domknięty to znaczy, że rozwiązanie $x=-[a]$ nie jest w tym przedziale rozwiązaniem tej równości. Zatem w tym przypadku piszemy, że $x\in \emptyset$
\item dla $x\in[-[a],+\infty)$ mamy:
$$x+[a]=x+[a]$$
$$0=0$$
Równanie jest prawdziwe na całym przedziale, zatem odpowiedzią tutaj będzie nasz wyjściowy przedział $x\in[-[a],+\infty)$.
\end{enumerate}
Podsumowując, rozwiązaniem jest przedział $x\in[-[a],+\infty)$.
\rozwStop
\odpStart
$x\in[-[a],+\infty)$
\odpStop
\testStart
A.$x\in[-[a],+\infty)$
B.$x\in \emptyset$
C.$\infty$
D. $x\in (-\infty,-[a])$
E. $x\in (-\infty,-[a])\cup x\in[-[a],+\infty)$
F.$\frac{1}{[d]}$
G.$[d]$
H.$-[a]$
I.$-\infty$
\testStop
\kluczStart
A
\kluczStop


\end{document}