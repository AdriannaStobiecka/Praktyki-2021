\documentclass[12pt, a4paper]{article}
\usepackage[utf8]{inputenc}
\usepackage{polski}

\usepackage{amsthm}  %pakiet do tworzenia twierdzeń itp.
\usepackage{amsmath} %pakiet do niektórych symboli matematycznych
\usepackage{amssymb} %pakiet do symboli mat., np. \nsubseteq
\usepackage{amsfonts}
\usepackage{graphicx} %obsługa plików graficznych z rozszerzeniem png, jpg
\theoremstyle{definition} %styl dla definicji
\newtheorem{zad}{} 
\title{Multizestaw zadań}
\author{Robert Fidytek}
%\date{\today}
\date{}
\newcounter{liczniksekcji}
\newcommand{\kategoria}[1]{\section{#1}} %olreślamy nazwę kateforii zadań
\newcommand{\zadStart}[1]{\begin{zad}#1\newline} %oznaczenie początku zadania
\newcommand{\zadStop}{\end{zad}}   %oznaczenie końca zadania
%Makra opcjonarne (nie muszą występować):
\newcommand{\rozwStart}[2]{\noindent \textbf{Rozwiązanie (autor #1 , recenzent #2): }\newline} %oznaczenie początku rozwiązania, opcjonarnie można wprowadzić informację o autorze rozwiązania zadania i recenzencie poprawności wykonania rozwiązania zadania
\newcommand{\rozwStop}{\newline}                                            %oznaczenie końca rozwiązania
\newcommand{\odpStart}{\noindent \textbf{Odpowiedź:}\newline}    %oznaczenie początku odpowiedzi końcowej (wypisanie wyniku)
\newcommand{\odpStop}{\newline}                                             %oznaczenie końca odpowiedzi końcowej (wypisanie wyniku)
\newcommand{\testStart}{\noindent \textbf{Test:}\newline} %ewentualne możliwe opcje odpowiedzi testowej: A. ? B. ? C. ? D. ? itd.
\newcommand{\testStop}{\newline} %koniec wprowadzania odpowiedzi testowych
\newcommand{\kluczStart}{\noindent \textbf{Test poprawna odpowiedź:}\newline} %klucz, poprawna odpowiedź pytania testowego (jedna literka): A lub B lub C lub D itd.
\newcommand{\kluczStop}{\newline} %koniec poprawnej odpowiedzi pytania testowego 
\newcommand{\wstawGrafike}[2]{\begin{figure}[h] \includegraphics[scale=#2] {#1} \end{figure}} %gdyby była potrzeba wstawienia obrazka, parametry: nazwa pliku, skala (jak nie wiesz co wpisać, to wpisz 1)

\begin{document}
\maketitle


\kategoria{Wikieł/Z1.151b}
\zadStart{Zadanie z Wikieł Z 1.151 b) moja wersja nr [nrWersji]}
%[a]:[2,3,5,6,7,8,10]
%[d]:[2,3,5,6,7,8,10]
%[c]:[2,3,5,6,7,8,10]
%[e]=random.randint(1,14)
%[dee]=[d]+[e]*[e]
%[2e]=2*[e]
%[c2]=[c]*[c]
%[4a]=4*[a]
%[t1]=int(([c]-1)/2)
%[t2]=int(([c]+1)/2)
%[t22]=[t2]*[t2]
%([c2]-4*[a])==1 and ([c]%2)==1 and [t1]==1
Rozwiązać równanie: $\bigg(\sqrt{[dee]-[2e]\sqrt{[d]}}\bigg)^{x}+[a]=[c]\cdot\bigg(\sqrt{\sqrt{[d]}-[e]}\bigg)^{x}$
\zadStop
\rozwStart{Wojciech Przybylski}{}
$$\bigg(\sqrt{[dee]-[2e]\sqrt{[d]}}\bigg)^{x}+[a]=[c]\cdot\bigg(\sqrt{\sqrt{[d]}-[e]}\bigg)^{x}$$
$$\bigg([dee]-[2e]\sqrt{[d]}\bigg)^{\frac{x}{2}}+[a]=[c]\cdot\bigg(\sqrt{[d]}-[e]\bigg)^{\frac{x}{2}}$$
$$\bigg((\sqrt{[d]}-[e])^{2}\bigg)^{\frac{x}{2}}+[a]=[c]\cdot\bigg(\sqrt{[d]}-[e]\bigg)^{\frac{x}{2}}$$
$$\bigg(\sqrt{[d]}-[e]\bigg)^{x}+[a]=[c]\cdot\bigg(\sqrt{[d]}-[e]\bigg)^{\frac{x}{2}}\hspace{5mm}, t=(\sqrt{[d]}-[e])^{\frac{x}{2}}$$
$$t^{2}-[c]\cdot t+[a]=0 \Rightarrow \Delta=[c2]-[4a]=1$$
$$t_{1}=\frac{[c]-1}{2}=[t1], \hspace{6mm} t_{2}=\frac{[c]-1}{2}=[t2]$$
$$(\sqrt{[d]}-[e])^{\frac{x}{2}}=[t1]  \hspace{3mm} \vee  \hspace{3mm}(\sqrt{[d]}-[e])^{\frac{x}{2}}=[t2]$$
$$(\sqrt{[d]}-[e])^{\frac{x}{2}}=(\sqrt{[d]}-[e])^{0}  \hspace{3mm} \vee  \hspace{3mm} log((\sqrt{[d]}-[e])^{\frac{x}{2}})=log([t2])$$
$$\frac{x}{2}=0  \hspace{3mm} \vee  \hspace{3mm} \frac{x}{2}log\big(\sqrt{[d]}-[e]\big)=log([t2])$$
$$x=0  \hspace{3mm} \vee  \hspace{3mm} x=\frac{log([t22])}{log\big(\sqrt{[d]}-[e]\big)}$$
\rozwStop
\odpStart
$x=0  \hspace{3mm} \vee  \hspace{3mm} x=\frac{log[t22]}{log\big(\sqrt{[d]}-[e]\big)}$
\odpStop
\testStart
A. $x=0  \hspace{3mm} \vee  \hspace{3mm} x=\frac{log[t22]}{log\big(\sqrt{[d]}-[e]\big)}$\\
B. $x=\frac{log[t2]}{log\big((\sqrt{[d]}-[e])\big)}$\\
C. $x=0$\\
D. $x=-1  \hspace{3mm} \vee  \hspace{3mm} x=\frac{log[t2]}{log\big(\sqrt{[d]}-[e]\big)}$\\
E. $x=1  \hspace{3mm} \vee  \hspace{3mm} x=\frac{log[t22]}{log\big(\sqrt{[d]}-[e]\big)}$\\
F. $x=1$\\
\testStop
\kluczStart
A
\kluczStop



\end{document}