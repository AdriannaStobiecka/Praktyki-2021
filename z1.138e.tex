\documentclass[12pt, a4paper]{article}
\usepackage[utf8]{inputenc}
\usepackage{polski}

\usepackage{amsthm}  %pakiet do tworzenia twierdzeń itp.
\usepackage{amsmath} %pakiet do niektórych symboli matematycznych
\usepackage{amssymb} %pakiet do symboli mat., np. \nsubseteq
\usepackage{amsfonts}
\usepackage{graphicx} %obsługa plików graficznych z rozszerzeniem png, jpg
\theoremstyle{definition} %styl dla definicji
\newtheorem{zad}{} 
\title{Multizestaw zadań}
\author{Robert Fidytek}
%\date{\today}
\date{}
\newcounter{liczniksekcji}
\newcommand{\kategoria}[1]{\section{#1}} %olreślamy nazwę kateforii zadań
\newcommand{\zadStart}[1]{\begin{zad}#1\newline} %oznaczenie początku zadania
\newcommand{\zadStop}{\end{zad}}   %oznaczenie końca zadania
%Makra opcjonarne (nie muszą występować):
\newcommand{\rozwStart}[2]{\noindent \textbf{Rozwiązanie (autor #1 , recenzent #2): }\newline} %oznaczenie początku rozwiązania, opcjonarnie można wprowadzić informację o autorze rozwiązania zadania i recenzencie poprawności wykonania rozwiązania zadania
\newcommand{\rozwStop}{\newline}                                            %oznaczenie końca rozwiązania
\newcommand{\odpStart}{\noindent \textbf{Odpowiedź:}\newline}    %oznaczenie początku odpowiedzi końcowej (wypisanie wyniku)
\newcommand{\odpStop}{\newline}                                             %oznaczenie końca odpowiedzi końcowej (wypisanie wyniku)
\newcommand{\testStart}{\noindent \textbf{Test:}\newline} %ewentualne możliwe opcje odpowiedzi testowej: A. ? B. ? C. ? D. ? itd.
\newcommand{\testStop}{\newline} %koniec wprowadzania odpowiedzi testowych
\newcommand{\kluczStart}{\noindent \textbf{Test poprawna odpowiedź:}\newline} %klucz, poprawna odpowiedź pytania testowego (jedna literka): A lub B lub C lub D itd.
\newcommand{\kluczStop}{\newline} %koniec poprawnej odpowiedzi pytania testowego 
\newcommand{\wstawGrafike}[2]{\begin{figure}[h] \includegraphics[scale=#2] {#1} \end{figure}} %gdyby była potrzeba wstawienia obrazka, parametry: nazwa pliku, skala (jak nie wiesz co wpisać, to wpisz 1)

\begin{document}
\maketitle


\kategoria{Wikieł/Z1.138e}
\zadStart{Zadanie z Wikieł Z 1.138 e) moja wersja nr [nrWersji]}
%[a]:[4,5,9,10,12]
%[b]:[2,3,6,7,8]
%[c]:[2,3,4,5,6,7,8]
%[bc]=[b]*[c]
Wyznaczyć funckję odwrotną do danej funkcji $f$ określonej na zbiorze $\mathcal{D}_{f}$.\\
e) $f(x)=log_{[a]}(\frac{1}{[b]}x+[c])\hspace{5mm}\mathcal{D}_{f}=\mathbb{R}$
\zadStop
\rozwStart{Wojciech Przybylski}{}
$$f(x)=log_{[a]}(\frac{1}{[b]}x+[c])\hspace{5mm} \mathcal{D}_{f}=(-[bc],\infty), \quad f(\mathcal{D}_{f})=\mathbb{R}$$
$$y=log_{[a]}\Big(\frac{1}{[b]}x+[c]\Big) \quad \Rightarrow \quad \frac{ln(\frac{1}{[b]}x+[c])}{ln([a])}=y \quad \Rightarrow $$
$$\Rightarrow \quad ln\Big(\frac{1}{[b]}x+[c]\Big)=ln([a]^{y}) \quad \Rightarrow \quad x=[b]\cdot([a]^{y}-[c])$$
$$y=f^{-1}(x)=[b]\cdot([a]^{x}-[c]) \mbox{ dla } x\in \mathbb{R}$$
\rozwStop
\odpStart
Funkcja odwrotna jest postaci $y=[b]\cdot([a]^{x}-[c]) \mbox{ dla }x\in \mathbb{R}$
\odpStop
\testStart
A. Funkcja odwrotna jest postaci $y=[b]\cdot([a]^{x}-[c])  \mbox{ dla }x\in\mathbb{R}$\\
B. Funkcja odwrotna jest postaci $y=[b]\cdot[a]^{x} \mbox{ dla }x\in\mathbb{R}$\\
C. Funkcja odwrotna jest postaci $y=[b]\cdot([a]^{x}+[c])  \mbox{ dla }x\in(-\infty,[a])$\\
D. Funkcja odwrotna jest postaci $y=([a]^{x}-[c]) \mbox{ dla }x\in[0,\infty)$\\
E. Funkcja odwrotna jest postaci $y=([a]^{x}+[c]) \mbox{ dla }x\in \mathbb{R}$\\
F. Funkcja odwrotna jest postaci $y=[b]\cdot([a]^{2x}+[c])\mbox{ dla }x\in[[a],\infty)$
\testStop
\kluczStart
A
\kluczStop



\end{document}