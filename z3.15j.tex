\documentclass[12pt, a4paper]{article}
\usepackage[utf8]{inputenc}
\usepackage{polski}

\usepackage{amsthm}  %pakiet do tworzenia twierdzeń itp.
\usepackage{amsmath} %pakiet do niektórych symboli matematycznych
\usepackage{amssymb} %pakiet do symboli mat., np. \nsubseteq
\usepackage{amsfonts}
\usepackage{graphicx} %obsługa plików graficznych z rozszerzeniem png, jpg
\theoremstyle{definition} %styl dla definicji
\newtheorem{zad}{} 
\title{Multizestaw zadań}
\author{Laura Mieczkowska}
%\date{\today}
\date{}
\newcounter{liczniksekcji}
\newcommand{\kategoria}[1]{\section{#1}} %olreślamy nazwę kateforii zadań
\newcommand{\zadStart}[1]{\begin{zad}#1\newline} %oznaczenie początku zadania
\newcommand{\zadStop}{\end{zad}}   %oznaczenie końca zadania
%Makra opcjonarne (nie muszą występować):
\newcommand{\rozwStart}[2]{\noindent \textbf{Rozwiązanie (autor #1 , recenzent #2): }\newline} %oznaczenie początku rozwiązania, opcjonarnie można wprowadzić informację o autorze rozwiązania zadania i recenzencie poprawności wykonania rozwiązania zadania
\newcommand{\rozwStop}{\newline}                                            %oznaczenie końca rozwiązania
\newcommand{\odpStart}{\noindent \textbf{Odpowiedź:}\newline}    %oznaczenie początku odpowiedzi końcowej (wypisanie wyniku)
\newcommand{\odpStop}{\newline}                                             %oznaczenie końca odpowiedzi końcowej (wypisanie wyniku)
\newcommand{\testStart}{\noindent \textbf{Test:}\newline} %ewentualne możliwe opcje odpowiedzi testowej: A. ? B. ? C. ? D. ? itd.
\newcommand{\testStop}{\newline} %koniec wprowadzania odpowiedzi testowych
\newcommand{\kluczStart}{\noindent \textbf{Test poprawna odpowiedź:}\newline} %klucz, poprawna odpowiedź pytania testowego (jedna literka): A lub B lub C lub D itd.
\newcommand{\kluczStop}{\newline} %koniec poprawnej odpowiedzi pytania testowego 
\newcommand{\wstawGrafike}[2]{\begin{figure}[h] \includegraphics[scale=#2] {#1} \end{figure}} %gdyby była potrzeba wstawienia obrazka, parametry: nazwa pliku, skala (jak nie wiesz co wpisać, to wpisz 1)

\begin{document}
\maketitle


\kategoria{Wikieł/Z3.15j}
\zadStart{Zadanie z Wikieł Z 3.15 j) moja wersja nr [nrWersji]}
%[b]:[2,3,4,5,6,7,8,9,10]
%[a]:[2,3,5,6,7]
Obliczyć granicę ciągu $a_n=\big(1-\frac{[a]}{n^2}\big)^{2-[b]n}$
\zadStop
\rozwStart{Laura Mieczkowska}{}
$$\lim_{n\to\infty} \bigg(1-\frac{[a]}{n^2}\bigg)^{2-[b]n}=
\lim_{n\to\infty} \bigg[\bigg(1-\frac{\sqrt{[a]}}{n}\bigg)\bigg(1+\frac{\sqrt{[a]}}{n}\bigg)\bigg]^{2-[b]n}=$$
$$=\lim_{n\to\infty}\bigg[\bigg(1-\frac{\sqrt{[a]}}{n}\bigg)\bigg(1+\frac{\sqrt{[a]}}{n}\bigg)\bigg]^2 \cdot \bigg[\bigg(1-\frac{\sqrt{[a]}}{n}\bigg)\bigg(1+\frac{\sqrt{[a]}}{n}\bigg)\bigg]^{-[b]n}=$$
$$=\lim_{n\to\infty}\bigg(1-\frac{\sqrt{[a]}}{n}\bigg)^2\cdot\bigg(1+\frac{\sqrt{[a]}}{n}\bigg)^2 \cdot \bigg(\bigg(1-\frac{\sqrt{[a]}}{n}\bigg)^{\frac{n}{\sqrt{[a]}}}\bigg)^{-[b]\sqrt{[a]}}\cdot\bigg(\bigg(1+\frac{\sqrt{[a]}}{n}\bigg)^{\frac{n}{\sqrt{[a]}}}\bigg)^{-[b]\sqrt{[a]}}=$$
$$=1^2\cdot 1^2\cdot e^{[b]\sqrt{[a]}} \cdot e^{-[b]\sqrt{[a]}}=1 $$
\odpStart
$1$
\odpStop
\testStart
A. $1$ \\
B. $e^{-[a]}$ \\
C. $e$ \\
D. $e^{[b]}$ 
\testStop
\kluczStart
A
\kluczStop



\end{document}