\documentclass[12pt, a4paper]{article}
\usepackage[utf8]{inputenc}
\usepackage{polski}

\usepackage{amsthm}  %pakiet do tworzenia twierdzeń itp.
\usepackage{amsmath} %pakiet do niektórych symboli matematycznych
\usepackage{amssymb} %pakiet do symboli mat., np. \nsubseteq
\usepackage{amsfonts}
\usepackage{graphicx} %obsługa plików graficznych z rozszerzeniem png, jpg
\theoremstyle{definition} %styl dla definicji
\newtheorem{zad}{} 
\title{Multizestaw zadań}
\author{Robert Fidytek}
%\date{\today}
\date{}
\newcounter{liczniksekcji}
\newcommand{\kategoria}[1]{\section{#1}} %olreślamy nazwę kateforii zadań
\newcommand{\zadStart}[1]{\begin{zad}#1\newline} %oznaczenie początku zadania
\newcommand{\zadStop}{\end{zad}}   %oznaczenie końca zadania
%Makra opcjonarne (nie muszą występować):
\newcommand{\rozwStart}[2]{\noindent \textbf{Rozwiązanie (autor #1 , recenzent #2): }\newline} %oznaczenie początku rozwiązania, opcjonarnie można wprowadzić informację o autorze rozwiązania zadania i recenzencie poprawności wykonania rozwiązania zadania
\newcommand{\rozwStop}{\newline}                                            %oznaczenie końca rozwiązania
\newcommand{\odpStart}{\noindent \textbf{Odpowiedź:}\newline}    %oznaczenie początku odpowiedzi końcowej (wypisanie wyniku)
\newcommand{\odpStop}{\newline}                                             %oznaczenie końca odpowiedzi końcowej (wypisanie wyniku)
\newcommand{\testStart}{\noindent \textbf{Test:}\newline} %ewentualne możliwe opcje odpowiedzi testowej: A. ? B. ? C. ? D. ? itd.
\newcommand{\testStop}{\newline} %koniec wprowadzania odpowiedzi testowych
\newcommand{\kluczStart}{\noindent \textbf{Test poprawna odpowiedź:}\newline} %klucz, poprawna odpowiedź pytania testowego (jedna literka): A lub B lub C lub D itd.
\newcommand{\kluczStop}{\newline} %koniec poprawnej odpowiedzi pytania testowego 
\newcommand{\wstawGrafike}[2]{\begin{figure}[h] \includegraphics[scale=#2] {#1} \end{figure}} %gdyby była potrzeba wstawienia obrazka, parametry: nazwa pliku, skala (jak nie wiesz co wpisać, to wpisz 1)

\begin{document}
\maketitle


\kategoria{Wikieł/Z3.3b}
\zadStart{Zadanie z Wikieł Z 3.3 b) moja wersja nr [nrWersji]}
%[p1]:[2,3,4,5,6,7,8]
%[p2]:[4,5,6,7,10]
%[p3]:[2,3,4,5,6,7]
%[p4]:[4,5,8,9,14]
%[a]=random.randint(-10,-1)
%[b]=random.randint(-10,-1)
%[p1m]=[p1]-1
%[p2m]=[p2]-1
%[p3m]=[p3]-1
%[p4m]=[p4]-1
%[ap2]=[a]*[p2m]
%[a1]=1-[a]
%[ap2p1]=[p1m]-[ap2]
%[c]=-round([ap2p1]/[a1],2)
%[cp3]=round([c]+[p3m],2)
%[cp4]=round([c]+[p4m],2)
%[cp3p4]=round([cp4]*[cp3],2)
%[d]=round([b]/([cp3p4]+0.00001),2)
%[c1]=round([c]*math.sqrt(abs([d])),2)
%[c2]=round(-[c]*math.sqrt(abs([d])),2)
%((abs([c])>[p3m] and abs([c])<[p4m] ) or (abs([c])>[p4m] and abs([c])<[p3m])) and abs([d])>1
Znaleźć wyraz pierwszy $a_{1}$ oraz różnicę $r$ ciągu arytmetycznego ($a_{n}$), w którym \\
b) $a_{[p1]} : a_{[p2]}=[a]$ oraz $a_{[p3]}\cdot a_{[p4]}=[b]$
\zadStop
\rozwStart{Wojciech Przybylski}{}
$$\frac{a_{[p1]}}{a_{[p2]}}=[a]$$
$$\frac{a_{1}+[p1m]r}{a_{1}+[p2m]r}=[a]$$
$$a_{1}+[p1m]r= [a]a_{1}+([ap2]r)$$
$$[a1]a_{1}+[ap2p1]r=0$$
$$a_{1}=[c]r$$
$$a_{[p3]}\cdot a_{[p4]}=[b]$$
$$(a_{1}+[p3m]r)(a_{1}+[p4m]r)=[b]$$
$$([c]r+[p3m]r)([c]r+[p4m]r)=[b]$$
$$[cp3]r\cdot [cp4]r=[b]$$
$$[cp3p4]r^{2}=[b]$$
$$r^{2}=[d]$$
$$r_{1}=\sqrt{[d]} \vee r_{2}=-\sqrt{[d]}$$
$$a_{1_{1}}=[c]\cdot\sqrt{[d]}=[c1] \vee a_{1_{2}}=[c]\cdot(-\sqrt{[d]})=[c2]$$
\rozwStop
\odpStart
1
\odpStop
\testStart
A.1
B.$1$
C.$\infty$
D.1
E.1
F.1
G.$[d]$
H.$[a]$
I.$-\infty$
\testStop
\kluczStart
A
\kluczStop



\end{document}