\documentclass[12pt, a4paper]{article}
\usepackage[utf8]{inputenc}
\usepackage{polski}

\usepackage{amsthm}  %pakiet do tworzenia twierdzeń itp.
\usepackage{amsmath} %pakiet do niektórych symboli matematycznych
\usepackage{amssymb} %pakiet do symboli mat., np. \nsubseteq
\usepackage{amsfonts}
\usepackage{graphicx} %obsługa plików graficznych z rozszerzeniem png, jpg
\theoremstyle{definition} %styl dla definicji
\newtheorem{zad}{} 
\title{Multizestaw zadań}
\author{Laura Mieczkowska}
%\date{\today}
\date{}
\newcounter{liczniksekcji}
\newcommand{\kategoria}[1]{\section{#1}} %olreślamy nazwę kateforii zadań
\newcommand{\zadStart}[1]{\begin{zad}#1\newline} %oznaczenie początku zadania
\newcommand{\zadStop}{\end{zad}}   %oznaczenie końca zadania
%Makra opcjonarne (nie muszą występować):
\newcommand{\rozwStart}[2]{\noindent \textbf{Rozwiązanie (autor #1 , recenzent #2): }\newline} %oznaczenie początku rozwiązania, opcjonarnie można wprowadzić informację o autorze rozwiązania zadania i recenzencie poprawności wykonania rozwiązania zadania
\newcommand{\rozwStop}{\newline}                                            %oznaczenie końca rozwiązania
\newcommand{\odpStart}{\noindent \textbf{Odpowiedź:}\newline}    %oznaczenie początku odpowiedzi końcowej (wypisanie wyniku)
\newcommand{\odpStop}{\newline}                                             %oznaczenie końca odpowiedzi końcowej (wypisanie wyniku)
\newcommand{\testStart}{\noindent \textbf{Test:}\newline} %ewentualne możliwe opcje odpowiedzi testowej: A. ? B. ? C. ? D. ? itd.
\newcommand{\testStop}{\newline} %koniec wprowadzania odpowiedzi testowych
\newcommand{\kluczStart}{\noindent \textbf{Test poprawna odpowiedź:}\newline} %klucz, poprawna odpowiedź pytania testowego (jedna literka): A lub B lub C lub D itd.
\newcommand{\kluczStop}{\newline} %koniec poprawnej odpowiedzi pytania testowego 
\newcommand{\wstawGrafike}[2]{\begin{figure}[h] \includegraphics[scale=#2] {#1} \end{figure}} %gdyby była potrzeba wstawienia obrazka, parametry: nazwa pliku, skala (jak nie wiesz co wpisać, to wpisz 1)

\begin{document}
\maketitle


\kategoria{Wikieł/Z4.21c}
\zadStart{Zadanie z Wikieł Z 4.21 c) moja wersja nr [nrWersji]}
%[b]:[2,3,4,5,6,7]
%[a1]=pow(2,[b])
%[a]=int([a1])
%[c]=[b]*[a]
%[d]=[b]-1
%[e1]=pow(2,[d])
%[e]=int([e1])
%[licz]=2*[e]
%[nwd]=math.gcd([licz],[c])
%[licznik1]=[licz]/[nwd]
%[licznik]=int([licznik1])
%[mian1]=[c]/[nwd]
%[mian]=int([mian1])
Obliczyć granicę funkcji $\lim_{x \to \frac{1}{2}} \frac{arcsin(1-2x)}{[a]x^{[b]}-1}$.
\zadStop
\rozwStart{Laura Mieczkowska}{}
$$\lim_{x \to \frac{1}{2}} \frac{arcsin(1-2x)}{[a]x^{[b]}-1}=\bigg[\frac{0}{0}\bigg]$$
Korzystając z reguły de l'Hospitala:
$$\lim_{x \to \frac{1}{2}} \frac{arcsin(1-2x)}{[a]x^{[b]}-1} \stackrel{H}{=}
\lim_{x\to\frac{1}{2}} \frac{(1-2x)'}{\sqrt{1-(1-2x)^2}}\cdot \frac{1}{([a]x^{[b]}-1)'}=$$
$$=\lim_{x\to\frac{1}{2}} \frac{-1}{\sqrt{x(1-x)}}\cdot \frac{1}{[c]x^{[d]}}=\bigg(-\frac{1}{\sqrt{\frac{1}{2}(1-\frac{1}{2})}}\bigg)\cdot \frac{1}{[c]\cdot(\frac{1}{2})^{[d]}}=$$
$$=-2\cdot\frac{[e]}{[c]}=-\frac{[licznik]}{[mian]}$$
\odpStart
$-\frac{[licznik]}{[mian]}$
\odpStop
\testStart
A. $\infty$ \\
B. $-\frac{[licznik]}{[mian]}$ \\
C. $0$ \\
D. $\frac{1}{[licz]}$ 
\testStop
\kluczStart
B
\kluczStop



\end{document}