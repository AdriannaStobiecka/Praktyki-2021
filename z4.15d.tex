\documentclass[12pt, a4paper]{article}
\usepackage[utf8]{inputenc}
\usepackage{polski}

\usepackage{amsthm}  %pakiet do tworzenia twierdzeń itp.
\usepackage{amsmath} %pakiet do niektórych symboli matematycznych
\usepackage{amssymb} %pakiet do symboli mat., np. \nsubseteq
\usepackage{amsfonts}
\usepackage{graphicx} %obsługa plików graficznych z rozszerzeniem png, jpg
\theoremstyle{definition} %styl dla definicji
\newtheorem{zad}{} 
\title{Multizestaw zadań}
\author{Robert Fidytek}
%\date{\today}
\date{}
\newcounter{liczniksekcji}
\newcommand{\kategoria}[1]{\section{#1}} %olreślamy nazwę kateforii zadań
\newcommand{\zadStart}[1]{\begin{zad}#1\newline} %oznaczenie początku zadania
\newcommand{\zadStop}{\end{zad}}   %oznaczenie końca zadania
%Makra opcjonarne (nie muszą występować):
\newcommand{\rozwStart}[2]{\noindent \textbf{Rozwiązanie (autor #1 , recenzent #2): }\newline} %oznaczenie początku rozwiązania, opcjonarnie można wprowadzić informację o autorze rozwiązania zadania i recenzencie poprawności wykonania rozwiązania zadania
\newcommand{\rozwStop}{\newline}                                            %oznaczenie końca rozwiązania
\newcommand{\odpStart}{\noindent \textbf{Odpowiedź:}\newline}    %oznaczenie początku odpowiedzi końcowej (wypisanie wyniku)
\newcommand{\odpStop}{\newline}                                             %oznaczenie końca odpowiedzi końcowej (wypisanie wyniku)
\newcommand{\testStart}{\noindent \textbf{Test:}\newline} %ewentualne możliwe opcje odpowiedzi testowej: A. ? B. ? C. ? D. ? itd.
\newcommand{\testStop}{\newline} %koniec wprowadzania odpowiedzi testowych
\newcommand{\kluczStart}{\noindent \textbf{Test poprawna odpowiedź:}\newline} %klucz, poprawna odpowiedź pytania testowego (jedna literka): A lub B lub C lub D itd.
\newcommand{\kluczStop}{\newline} %koniec poprawnej odpowiedzi pytania testowego 
\newcommand{\wstawGrafike}[2]{\begin{figure}[h] \includegraphics[scale=#2] {#1} \end{figure}} %gdyby była potrzeba wstawienia obrazka, parametry: nazwa pliku, skala (jak nie wiesz co wpisać, to wpisz 1)

\begin{document}
\maketitle


\kategoria{Wikieł/Z4.15d}
\zadStart{Zadanie z Wikieł Z 4.15 d) moja wersja nr [nrWersji]}
%[a]:[2,3,4,5,6,7,8,9,10,11,12,13,14,15,16,17,18,19,20,21,22,23,24,25,26,27,28,29,30,31,32,33,34,35,36,37,38,39,40,51,52,53,54,55,56,57,58,59,60,61,62,63,64,65,66,67,68,69,70,71]
%[xo]=[a]
%[ad]=2*[a]
%[at]=3*[a]
%[ac]=4*[a]
%[axom]=[a]-[xo]
Zbadać, czy istnieje następująca granica $\lim\limits_{x\to [xo]}\frac{([a]-x)^{3}}{|x-[a]|}$. Jeśli tak, to obliczyć ją.
\zadStop
\rozwStart{Justyna Chojecka}{}
Obliczamy granice jednostronne w punkcie $x_{0}=[xo]$.
$$\lim\limits_{x\to [xo]^{-}}\frac{([a]-x)^{3}}{|x-[a]|}=\lim\limits_{x\to [xo]^{-}}\frac{([a]-x)^{3}}{[a]-x}=\lim\limits_{x\to [xo]^{-}}([a]-x)^{2}=([a]-[xo])^{2}=[axom]^{2}=[axom]$$
$$\lim\limits_{x\to [xo]^{+}}\frac{([a]-x)^{3}}{|x-[a]|}=\lim\limits_{x\to [xo]^{+}}\frac{([a]-x)^{3}}{-([a]-x)}=\lim\limits_{x\to [xo]^{+}}-([a]-x)^{2}=-([a]-[xo])^{2}=-[axom]^{2}=[axom]$$
Skoro 
$$\lim\limits_{x\to [xo]^{-}}\frac{([a]-x)^{3}}{|x-[a]|}=\lim\limits_{x\to [xo]^{+}}\frac{([a]-x)^{3}}{|x-[a]|}=[axom],$$
to granica $\lim\limits_{x\to [xo]}\frac{([a]-x)^{3}}{|x-[a]|}$ istnieje i jest równa $[axom]$.
\rozwStop
\odpStart
$[axom]$
\odpStop
\testStart
A.$[axom]$
B.nie istnieje
C.$[a]$
D.$-[a]$
E.$[ad]$
F.$[at]$
G.$-[ad]$
H.$-[ac]$
I.$-[at]$
\testStop
\kluczStart
A
\kluczStop



\end{document}
