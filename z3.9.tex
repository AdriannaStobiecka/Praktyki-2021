\documentclass[12pt, a4paper]{article}
\usepackage[utf8]{inputenc}
\usepackage{polski}

\usepackage{amsthm}  %pakiet do tworzenia twierdzeń itp.
\usepackage{amsmath} %pakiet do niektórych symboli matematycznych
\usepackage{amssymb} %pakiet do symboli mat., np. \nsubseteq
\usepackage{amsfonts}
\usepackage{graphicx} %obsługa plików graficznych z rozszerzeniem png, jpg
\theoremstyle{definition} %styl dla definicji
\newtheorem{zad}{} 
\title{Multizestaw zadań}
\author{Robert Fidytek}
%\date{\today}
\date{}
\newcounter{liczniksekcji}
\newcommand{\kategoria}[1]{\section{#1}} %olreślamy nazwę kateforii zadań
\newcommand{\zadStart}[1]{\begin{zad}#1\newline} %oznaczenie początku zadania
\newcommand{\zadStop}{\end{zad}}   %oznaczenie końca zadania
%Makra opcjonarne (nie muszą występować):
\newcommand{\rozwStart}[2]{\noindent \textbf{Rozwiązanie (autor #1 , recenzent #2): }\newline} %oznaczenie początku rozwiązania, opcjonarnie można wprowadzić informację o autorze rozwiązania zadania i recenzencie poprawności wykonania rozwiązania zadania
\newcommand{\rozwStop}{\newline}                                            %oznaczenie końca rozwiązania
\newcommand{\odpStart}{\noindent \textbf{Odpowiedź:}\newline}    %oznaczenie początku odpowiedzi końcowej (wypisanie wyniku)
\newcommand{\odpStop}{\newline}                                             %oznaczenie końca odpowiedzi końcowej (wypisanie wyniku)
\newcommand{\testStart}{\noindent \textbf{Test:}\newline} %ewentualne możliwe opcje odpowiedzi testowej: A. ? B. ? C. ? D. ? itd.
\newcommand{\testStop}{\newline} %koniec wprowadzania odpowiedzi testowych
\newcommand{\kluczStart}{\noindent \textbf{Test poprawna odpowiedź:}\newline} %klucz, poprawna odpowiedź pytania testowego (jedna literka): A lub B lub C lub D itd.
\newcommand{\kluczStop}{\newline} %koniec poprawnej odpowiedzi pytania testowego 
\newcommand{\wstawGrafike}[2]{\begin{figure}[h] \includegraphics[scale=#2] {#1} \end{figure}} %gdyby była potrzeba wstawienia obrazka, parametry: nazwa pliku, skala (jak nie wiesz co wpisać, to wpisz 1)

\begin{document}
\maketitle


\kategoria{Wikieł/Z3.9}
\zadStart{Zadanie z Wikieł Z 3.9 ) moja wersja nr [nrWersji]}
%[p1]:[1,2,3,4,6,7,8,10,11,12,14]
%[a]=random.randint(1,10)
%[ap2]=[p1]+[a]
%[ap3]=[ap2]+[a]
%[ap4]=[ap3]+[a]
Ciąg $(a_{n})$ określony jest rekurencyjnie w następujący sposób:
$$
 \left\{ \begin{array}{ll}
a_{1}= [p1] & \\
a_{n+1}=a_{n}+[a] & \mbox{dla }n\geq1
\end{array} \right.
$$
Wyznaczyć wzór na $n$-ty wyraz tego ciągu oraz zbadać jego monotoniczność. 
\zadStop
\rozwStart{Wojciech Przybylski}{}
$$\mbox{Zaczynamy od wypisania pierwszych wyrazów naszego ciągu: }$$
$$a_{1}=[p1], a_{2}=[ap2], a_{3}=[ap3], a_{4}=[ap4],\cdots $$
$$\mbox{Zauważamy wzór na n-ty wyraz z wypisanych wyrazów: }$$
$$a_{n}=[p1]+[a]\cdot(n-1)$$
$$\mbox{Przechodzimy do badania monotoniczności}$$
$$\mbox{Dowód indukcyjny:}$$
$$\mbox{I Sprawdzenie } [ap2]=a_{2}>a_{1}=[p1]$$
$$\mbox{II Założenie niech }n\geq1 \mbox{ takie, że } a_{n+1}>a_{n}$$
$$\mbox{III Krok indukcyjny }\mbox{Udowodnimy, że } a_{n+2}>a_{n+1}$$
$$a_{n+2}=[p1]+[a]\cdot(n+2-1)=[p1]+[a]\cdot(n+1)=$$
$$=[ap2]+[a]\cdot n>[p1]+[a]\cdot n=a_{n+1}$$
$$a_{n+2}>a_{n+1}$$
$$\mbox{Dowód indukcyjny jest spełniony ciąg rekurencyjny jest rosnący}$$
\rozwStop
\odpStart
Ciąg rekurencyjny jest ciągiem rosnącym, \\
wzór na n-ty wyraz: $a_{n}=[p1]+[a]\cdot(n-1)$.
\odpStop
\testStart
A. Ciąg rekurencyjny jest ciągiem rosnącym,\\
    wzór na n-ty wyraz: $a_{n}=[p1]+[a]\cdot(n-1)$.\\
B. Ciąg rekurencyjny jest ciągiem malejącym,\\
    wzór na n-ty wyraz: $a_{n}=[p1]+[a]\cdot(n-1)$.\\
C. Ciąg rekurencyjny jest ciągiem rosnącym,\\
    wzór na n-ty wyraz: $a_{n}=[p1]+[a]\cdot n$.\\
D. Ciąg rekurencyjny jest ciągiem stałym,\\
    wzór na n-ty wyraz: $a_{n}=[p1]+[a]\cdot n$.\\
E. Ciąg rekurencyjny jest ciągiem malejącym,
    wzór na n-ty wyraz: $a_{n}=[p1]+[a]\cdot n$.\\
\testStop
\kluczStart
A
\kluczStop



\end{document}