\documentclass[12pt, a4paper]{article}
\usepackage[utf8]{inputenc}
\usepackage{polski}

\usepackage{amsthm}  %pakiet do tworzenia twierdzeń itp.
\usepackage{amsmath} %pakiet do niektórych symboli matematycznych
\usepackage{amssymb} %pakiet do symboli mat., np. \nsubseteq
\usepackage{amsfonts}
\usepackage{graphicx} %obsługa plików graficznych z rozszerzeniem png, jpg
\theoremstyle{definition} %styl dla definicji
\newtheorem{zad}{} 
\title{Multizestaw zadań}
\author{Robert Fidytek}
%\date{\today}
\date{}
\newcounter{liczniksekcji}
\newcommand{\kategoria}[1]{\section{#1}} %olreślamy nazwę kateforii zadań
\newcommand{\zadStart}[1]{\begin{zad}#1\newline} %oznaczenie początku zadania
\newcommand{\zadStop}{\end{zad}}   %oznaczenie końca zadania
%Makra opcjonarne (nie muszą występować):
\newcommand{\rozwStart}[2]{\noindent \textbf{Rozwiązanie (autor #1 , recenzent #2): }\newline} %oznaczenie początku rozwiązania, opcjonarnie można wprowadzić informację o autorze rozwiązania zadania i recenzencie poprawności wykonania rozwiązania zadania
\newcommand{\rozwStop}{\newline}                                            %oznaczenie końca rozwiązania
\newcommand{\odpStart}{\noindent \textbf{Odpowiedź:}\newline}    %oznaczenie początku odpowiedzi końcowej (wypisanie wyniku)
\newcommand{\odpStop}{\newline}                                             %oznaczenie końca odpowiedzi końcowej (wypisanie wyniku)
\newcommand{\testStart}{\noindent \textbf{Test:}\newline} %ewentualne możliwe opcje odpowiedzi testowej: A. ? B. ? C. ? D. ? itd.
\newcommand{\testStop}{\newline} %koniec wprowadzania odpowiedzi testowych
\newcommand{\kluczStart}{\noindent \textbf{Test poprawna odpowiedź:}\newline} %klucz, poprawna odpowiedź pytania testowego (jedna literka): A lub B lub C lub D itd.
\newcommand{\kluczStop}{\newline} %koniec poprawnej odpowiedzi pytania testowego 
\newcommand{\wstawGrafike}[2]{\begin{figure}[h] \includegraphics[scale=#2] {#1} \end{figure}} %gdyby była potrzeba wstawienia obrazka, parametry: nazwa pliku, skala (jak nie wiesz co wpisać, to wpisz 1)

\begin{document}
\maketitle


\kategoria{Wikieł/P1.2}
\zadStart{Zadanie z Wikieł P 1.2 moja wersja nr [nrWersji]}
%[c]:[11,27,44,72]
%[d]:[11,27,44,72]
%[a]=round(random.uniform(0,1),2)
%[k]=int([a]*100)
%[b]=round(random.uniform(0,1),2)
%[k2]=int([b]*100)
%[a]!= [b]
%[c] != [d]
%[a]!=0
%[a]!=1
%[b]!=0
%[b]!=1
%[w]=int(100*[a])
%[x]=[w]/99
%[z]=[a]+[b]
%[z2]= round([z], 2)
%[x2]= [z2] - [a]
%[x3]=int(100*[x2])

Zamienić ułamki dziesiętnie okresowe $0.0([k])$ i $0.([k2])$ na ułamki zwykłe.
\zadStop
\rozwStart{Martyna Czarnobaj}{}
Przyjmijmy $a = [a]$.\\
$ 100a = [w]+ a $\\
$ 100a - a = [w] $\\
$ 99a = [w] $\\
$ a = \frac{[w]}{99} $\\
Więc $ a = \frac{[w]}{99} $.\\
\\
Zauważmy, że $ [a] + [b] = [z]  = [z2] $\\
Przyjmijmy $ b = [b] $\\
Zatem $ b = [z2] - [a] =[x2]= $ $\frac{[x3]}{100} $ .\\
Koniec rozwiązania.\\
\rozwStop
\odpStart
$  \frac{[w]}{99} $ i $ \frac{[x3]}{100} $
\odpStop
\testStart
A.$\frac{[w]}{99} $ i $ \frac{[x3]}{100} $
B.$\frac{[w]}{2} $ i $ \frac{[x3]}{3} $
C.$\frac{[w]}{5} $ i $ \frac{[x3]}{10} $
\testStop
\kluczStart
A
\kluczStop



\end{document}