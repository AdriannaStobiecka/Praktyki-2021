\documentclass[12pt, a4paper]{article}
\usepackage[utf8]{inputenc}
\usepackage{polski}

\usepackage{amsthm}  %pakiet do tworzenia twierdzeń itp.
\usepackage{amsmath} %pakiet do niektórych symboli matematycznych
\usepackage{amssymb} %pakiet do symboli mat., np. \nsubseteq
\usepackage{amsfonts}
\usepackage{graphicx} %obsługa plików graficznych z rozszerzeniem png, jpg
\theoremstyle{definition} %styl dla definicji
\newtheorem{zad}{} 
\title{Multizestaw zadań}
\author{Robert Fidytek}
%\date{\today}
\date{}
\newcounter{liczniksekcji}
\newcommand{\kategoria}[1]{\section{#1}} %olreślamy nazwę kateforii zadań
\newcommand{\zadStart}[1]{\begin{zad}#1\newline} %oznaczenie początku zadania
\newcommand{\zadStop}{\end{zad}}   %oznaczenie końca zadania
%Makra opcjonarne (nie muszą występować):
\newcommand{\rozwStart}[2]{\noindent \textbf{Rozwiązanie (autor #1 , recenzent #2): }\newline} %oznaczenie początku rozwiązania, opcjonarnie można wprowadzić informację o autorze rozwiązania zadania i recenzencie poprawności wykonania rozwiązania zadania
\newcommand{\rozwStop}{\newline}                                            %oznaczenie końca rozwiązania
\newcommand{\odpStart}{\noindent \textbf{Odpowiedź:}\newline}    %oznaczenie początku odpowiedzi końcowej (wypisanie wyniku)
\newcommand{\odpStop}{\newline}                                             %oznaczenie końca odpowiedzi końcowej (wypisanie wyniku)
\newcommand{\testStart}{\noindent \textbf{Test:}\newline} %ewentualne możliwe opcje odpowiedzi testowej: A. ? B. ? C. ? D. ? itd.
\newcommand{\testStop}{\newline} %koniec wprowadzania odpowiedzi testowych
\newcommand{\kluczStart}{\noindent \textbf{Test poprawna odpowiedź:}\newline} %klucz, poprawna odpowiedź pytania testowego (jedna literka): A lub B lub C lub D itd.
\newcommand{\kluczStop}{\newline} %koniec poprawnej odpowiedzi pytania testowego 
\newcommand{\wstawGrafike}[2]{\begin{figure}[h] \includegraphics[scale=#2] {#1} \end{figure}} %gdyby była potrzeba wstawienia obrazka, parametry: nazwa pliku, skala (jak nie wiesz co wpisać, to wpisz 1)

\begin{document}
\maketitle


\kategoria{Wikieł/Z1.87a}
\zadStart{Zadanie z Wikieł Z 1.87 a) moja wersja nr 1}
%[a]:[2,3,4,5,6,7,8,9,10]
%[b]:[2,3,4,5,6,7,8,9,10]
%[c]:[2,3,4,5,6,7,8,9,10]
%[a]=random.randint(2,30)
%[b]=random.randint(2,30)
%[c]=random.randint(2,30)
%[2b]=[b]*[b]
%[2a]=2*[a]
%[-c]=(-1)*[c]
%[ac]=[a]*[c]
%[delta]=pow([b],2)-(4*(-[ac]))
%[sdelta]=(pow([delta],1/2))
%[sdelta1]=int([sdelta])
%[x1]=([b]-[sdelta])/([2a])
%[x2]=([b]+[sdelta])/([2a])
%[x11]=round([x1],2)
%[x21]=int([x2])
%[sdelta].is_integer()==True and [x2].is_integer()==True and [x1].is_integer()==False
Wyznaczyć wartości zmiennej x, dla których funkcja przyjmuje wartości\\ dodatnie.\\ $f(x)=\big(\frac{3}{5}\big)^{[a]x^{2}-[b]x-[c]}-1$
\zadStop
\rozwStart{Jakub Ulrych}{}
$$f(x)=\big(\frac{3}{5}\big)^{[a]x^{2}-[b]x-[c]}-1>0$$
$$\big(\frac{3}{5}\big)^{[a]x^{2}-[b]x-[c]}>1\Leftrightarrow [a]x^{2}-[b]x-[c]<0$$
$$\Delta=[b]^{2}-4\cdot[a]\cdot-[c]=[delta]\Rightarrow \sqrt{\Delta}=[sdelta1]$$
$$x_{1}=[x11],x_{2}=[x21]$$
$$[a](x-([x11]))(x-[x21])<0$$
$$(x-([x11]))(x-[x21])<0$$
$$x\in([x11],[x21])$$
\rozwStop
\odpStart
$$x\in([x11],[x21])$$
\odpStop
\testStart
A.$x\in([x11],[x21])$
B.$x\in([x11],+\infty)$
C.$x\in(-\infty,0)$
D.$x\in(-\infty,[x21])$
\testStop
\kluczStart
A
\kluczStop



\end{document}