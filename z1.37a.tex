\documentclass[12pt, a4paper]{article}
\usepackage[utf8]{inputenc}
\usepackage{polski}

\usepackage{amsthm}  %pakiet do tworzenia twierdzeń itp.
\usepackage{amsmath} %pakiet do niektórych symboli matematycznych
\usepackage{amssymb} %pakiet do symboli mat., np. \nsubseteq
\usepackage{amsfonts}
\usepackage{graphicx} %obsługa plików graficznych z rozszerzeniem png, jpg
\theoremstyle{definition} %styl dla definicji
\newtheorem{zad}{} 
\title{Multizestaw zadań}
\author{Robert Fidytek}
%\date{\today}
\date{}
\newcounter{liczniksekcji}
\newcommand{\kategoria}[1]{\section{#1}} %olreślamy nazwę kateforii zadań
\newcommand{\zadStart}[1]{\begin{zad}#1\newline} %oznaczenie początku zadania
\newcommand{\zadStop}{\end{zad}}   %oznaczenie końca zadania
%Makra opcjonarne (nie muszą występować):
\newcommand{\rozwStart}[2]{\noindent \textbf{Rozwiązanie (autor #1 , recenzent #2): }\newline} %oznaczenie początku rozwiązania, opcjonarnie można wprowadzić informację o autorze rozwiązania zadania i recenzencie poprawności wykonania rozwiązania zadania
\newcommand{\rozwStop}{\newline}                                            %oznaczenie końca rozwiązania
\newcommand{\odpStart}{\noindent \textbf{Odpowiedź:}\newline}    %oznaczenie początku odpowiedzi końcowej (wypisanie wyniku)
\newcommand{\odpStop}{\newline}                                             %oznaczenie końca odpowiedzi końcowej (wypisanie wyniku)
\newcommand{\testStart}{\noindent \textbf{Test:}\newline} %ewentualne możliwe opcje odpowiedzi testowej: A. ? B. ? C. ? D. ? itd.
\newcommand{\testStop}{\newline} %koniec wprowadzania odpowiedzi testowych
\newcommand{\kluczStart}{\noindent \textbf{Test poprawna odpowiedź:}\newline} %klucz, poprawna odpowiedź pytania testowego (jedna literka): A lub B lub C lub D itd.
\newcommand{\kluczStop}{\newline} %koniec poprawnej odpowiedzi pytania testowego 
\newcommand{\wstawGrafike}[2]{\begin{figure}[h] \includegraphics[scale=#2] {#1} \end{figure}} %gdyby była potrzeba wstawienia obrazka, parametry: nazwa pliku, skala (jak nie wiesz co wpisać, to wpisz 1)

\begin{document}
\maketitle


\kategoria{Wikieł/Z1.37a}
\zadStart{Zadanie z Wikieł Z 1.37 a) moja wersja nr [nrWersji]}
%[a]:[70,80,90,100,110,120,180,220]
%[b]:[20,25,30,35,40,45,50,55,60]
%[ab]=[a]-[b]
%[abm]=[a]*[b]
%[4abm]=[abm]*4
%[delta]=[ab]*[ab]+[4abm]
%[pierw]=int(math.sqrt([delta]))
%[x1]=int(([ab]-[pierw])/2)
%[x2]=int(([ab]+[pierw])/2)
Rozwiązać nierówność: $x^{2}-[ab]x-[abm]<0$.
\zadStop
\rozwStart{Wojciech Przybylski}{}
$$x^{2}-[ab]x-[abm]<0$$
$$\Delta=[ab]^{2}+[4abm]=[delta] \Rightarrow \sqrt{\Delta}=[pierw]$$
$$x_{1}=\frac{[ab]-[pierw]}{2}=[x1] \hspace{3mm} x_{2}=\frac{[ab]+[pierw]}{2}=[x2]$$
Parabola ma ramiona skierowane ku górzę, więc $x \in ([x1],[x2])$.
\rozwStop
\odpStart
$x \in ([x1],[x2])$
\odpStop
\testStart
A. $x \in ([x1],[x2])$.\\
B. $x \in ([a],[x2])$.\\
C. $x \in ([x1],[b])$.\\
D.$x \in ([a],[b])$.\\
E. $x \in ([ab],[ab])$.\\
F. Nie ma takiego x, który by spełniał tą nierówność
\testStop
\kluczStart
A
\kluczStop



\end{document}