\documentclass[12pt, a4paper]{article}
\usepackage[utf8]{inputenc}
\usepackage{polski}

\usepackage{amsthm}  %pakiet do tworzenia twierdzeń itp.
\usepackage{amsmath} %pakiet do niektórych symboli matematycznych
\usepackage{amssymb} %pakiet do symboli mat., np. \nsubseteq
\usepackage{amsfonts}
\usepackage{graphicx} %obsługa plików graficznych z rozszerzeniem png, jpg
\theoremstyle{definition} %styl dla definicji
\newtheorem{zad}{} 
\title{Multizestaw zadań}
\author{Robert Fidytek}
%\date{\today}
\date{}
\newcounter{liczniksekcji}
\newcommand{\kategoria}[1]{\section{#1}} %olreślamy nazwę kateforii zadań
\newcommand{\zadStart}[1]{\begin{zad}#1\newline} %oznaczenie początku zadania
\newcommand{\zadStop}{\end{zad}}   %oznaczenie końca zadania
%Makra opcjonarne (nie muszą występować):
\newcommand{\rozwStart}[2]{\noindent \textbf{Rozwiązanie (autor #1 , recenzent #2): }\newline} %oznaczenie początku rozwiązania, opcjonarnie można wprowadzić informację o autorze rozwiązania zadania i recenzencie poprawności wykonania rozwiązania zadania
\newcommand{\rozwStop}{\newline}                                            %oznaczenie końca rozwiązania
\newcommand{\odpStart}{\noindent \textbf{Odpowiedź:}\newline}    %oznaczenie początku odpowiedzi końcowej (wypisanie wyniku)
\newcommand{\odpStop}{\newline}                                             %oznaczenie końca odpowiedzi końcowej (wypisanie wyniku)
\newcommand{\testStart}{\noindent \textbf{Test:}\newline} %ewentualne możliwe opcje odpowiedzi testowej: A. ? B. ? C. ? D. ? itd.
\newcommand{\testStop}{\newline} %koniec wprowadzania odpowiedzi testowych
\newcommand{\kluczStart}{\noindent \textbf{Test poprawna odpowiedź:}\newline} %klucz, poprawna odpowiedź pytania testowego (jedna literka): A lub B lub C lub D itd.
\newcommand{\kluczStop}{\newline} %koniec poprawnej odpowiedzi pytania testowego 
\newcommand{\wstawGrafike}[2]{\begin{figure}[h] \includegraphics[scale=#2] {#1} \end{figure}} %gdyby była potrzeba wstawienia obrazka, parametry: nazwa pliku, skala (jak nie wiesz co wpisać, to wpisz 1)

\begin{document}
\maketitle


\kategoria{Wikieł/Z4.7g}
\zadStart{Zadanie z Wikieł Z 4.7 g) moja wersja nr [nrWersji]}
%[b]:[2,3]
%[c]:[2,3]
%[d]:[2,3]
%[e]:[2,3]
%[b]=random.randint(2,5)
%[c]=random.randint(2,5)
%[d]=random.randint(2,5)
%[e]=random.randint(2,5)
%[bd]=[b]+[d]
%[ce]=[c]-[e]
%[c]>[e]
Obliczyć granice funkcji $\displaystyle{\lim_{x \to \infty}}(\sqrt[3]{x^3-[b]x^2+[c]}-\sqrt[3]{x^3+[d]x^2+[e]})$
\zadStop
\rozwStart{Pascal Nawrocki}{}
Zadanie może wydawać się dość nieprzyjemne (i takie jest), ale my się nie boimy i przywołamy do pomocy nasze ukochane wzory skróconego męczenia do 3 potęgi. Przywołując pamięcią, w zadaniach ze kwadratowymi pierwiastkami korzystaliśmy z tak zwanego "sprzężenia", które polegało na przemnożeniu danego wyrażenia przez to samo w liczniku i mianowniku tylko z przeciwnym znakiem. Zapewne wiesz o co chodzi, został tam po prostu zastosowany wzór skróconego mnożenia $a^2-b^2=(a-b)(a+b)$. To samo postaramy się zrobić tutaj. Tylko zapis będzie inny, ponieważ:$a^3-b^3=(a-b)(a^2+ab+b^2)$. Pierwsze wyrażenie sobie oznaczamy poprzez a, drugie poprzez b i teraz zauważmy, że wystarczy przemnożyć nasze wyrażenie w nawiasie przez "jedynkę", która wygląda w ten sposób:
$$\frac{(\sqrt[3]{x^3-[b]x^2+[c]})^2+\sqrt[3]{x^3-[b]x^2+[c]}\cdot\sqrt[3]{x^3+[d]x^2+[e]}+(\sqrt[3]{x^3+[d]x^2+[e]})^2}{(\sqrt[3]{x^3-[b]x^2+[c]})^2+\sqrt[3]{x^3-[b]x^2+[c]}\cdot\sqrt[3]{x^3+[d]x^2+[e]}+(\sqrt[3]{x^3+[d]x^2+[e]})^2}$$
(To jest ta część $(a^2+ab+b^2)$.
Dzięki temu nasza wyjściowa granica sprowadza się do:
$$\displaystyle{\lim_{x \to \infty}}\bigg(\frac{x^3-[b]x^2+[c]-x^3-[d]x^2-[e]}{(\sqrt[3]{x^3-[b]x^2+[c]})^2+\sqrt[3]{x^3-[b]x^2+[c]}\cdot\sqrt[3]{x^3+[d]x^2+[e]}+(\sqrt[3]{x^3+[d]x^2+[e]})^2}\bigg)$$
Czyli:
$$\displaystyle{\lim_{x \to \infty}}\bigg(\frac{-[bd]x^2+[ce]}{(\sqrt[3]{x^3-[b]x^2+[c]})^2+\sqrt[3]{x^3-[b]x^2+[c]}\cdot\sqrt[3]{x^3+[d]x^2+[e]}+(\sqrt[3]{x^3+[d]x^2+[e]})^2}\bigg)$$
Żeby uniknąć zbyt wiele pisania, zauważmy teraz, że w mianowniku praktycznie każdy składnik posiada wyraz $x^3$ i jeśli wszystko wymnożymy (kwadraty oraz ab) to w każdym pierwiastku będzie $x^6$, jako że jest to najwyższy wyraz to wyłączymy go przed każdy z tych nawiasów, a wyłączając x z $\sqrt[3]{x^6}$ otrzymamy $x^2$. Teraz możemy zauważyć, że skraca się on z licznikiem i w sumie otrzymamy wyrażenie:
$$-\frac{[bd]}{3}$$ Trójka jest z sumy w mianowniku, ponieważ z każdego pierwiastka tam otrzymamy 1. Stąd $1+1+1=3$.
\rozwStop
\odpStart
$-\frac{[bd]}{3}$
\odpStop
\testStart
A.$-\frac{[bd]}{3}$
B.$\infty$
C.$-\infty$
D.$0$
\testStop
\kluczStart
A
\kluczStop
\end{document}