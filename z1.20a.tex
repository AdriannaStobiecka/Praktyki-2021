\documentclass[12pt, a4paper]{article}
\usepackage[utf8]{inputenc}
\usepackage{polski}

\usepackage{amsthm}  %pakiet do tworzenia twierdzeń itp.
\usepackage{amsmath} %pakiet do niektórych symboli matematycznych
\usepackage{amssymb} %pakiet do symboli mat., np. \nsubseteq
\usepackage{amsfonts}
\usepackage{graphicx} %obsługa plików graficznych z rozszerzeniem png, jpg
\theoremstyle{definition} %styl dla definicji
\newtheorem{zad}{} 
\title{Multizestaw zadań}
\author{Robert Fidytek}
%\date{\today}
\date{}
\newcounter{liczniksekcji}
\newcommand{\kategoria}[1]{\section{#1}} %olreślamy nazwę kateforii zadań
\newcommand{\zadStart}[1]{\begin{zad}#1\newline} %oznaczenie początku zadania
\newcommand{\zadStop}{\end{zad}}   %oznaczenie końca zadania
%Makra opcjonarne (nie muszą występować):
\newcommand{\rozwStart}[2]{\noindent \textbf{Rozwiązanie (autor #1 , recenzent #2): }\newline} %oznaczenie początku rozwiązania, opcjonarnie można wprowadzić informację o autorze rozwiązania zadania i recenzencie poprawności wykonania rozwiązania zadania
\newcommand{\rozwStop}{\newline}                                            %oznaczenie końca rozwiązania
\newcommand{\odpStart}{\noindent \textbf{Odpowiedź:}\newline}    %oznaczenie początku odpowiedzi końcowej (wypisanie wyniku)
\newcommand{\odpStop}{\newline}                                             %oznaczenie końca odpowiedzi końcowej (wypisanie wyniku)
\newcommand{\testStart}{\noindent \textbf{Test:}\newline} %ewentualne możliwe opcje odpowiedzi testowej: A. ? B. ? C. ? D. ? itd.
\newcommand{\testStop}{\newline} %koniec wprowadzania odpowiedzi testowych
\newcommand{\kluczStart}{\noindent \textbf{Test poprawna odpowiedź:}\newline} %klucz, poprawna odpowiedź pytania testowego (jedna literka): A lub B lub C lub D itd.
\newcommand{\kluczStop}{\newline} %koniec poprawnej odpowiedzi pytania testowego 
\newcommand{\wstawGrafike}[2]{\begin{figure}[h] \includegraphics[scale=#2] {#1} \end{figure}} %gdyby była potrzeba wstawienia obrazka, parametry: nazwa pliku, skala (jak nie wiesz co wpisać, to wpisz 1)

\begin{document}
\maketitle



\kategoria{Wikieł/Z1.20 a}
\zadStart{Zadanie z Wikieł Z 1.20 a) moja wersja nr [nrWersji]}
%[a]:[2,3,4]
%[b]:[2,3,4]
%[a]=random.randint(2,4)
%[b]=random.randint(1,4)
%[a4]=[a]*[a]*[a]*[a]
%[a3]=[a]*[a]*[a]
%[a2]=[a]*[a]
%[b4]=(-[b])*(-[b])*(-[b])*(-[b])
%[b3]=(-[b])*(-[b])*(-[b])
%[b2]=(-[b])*(-[b])
%[a3b]=-[a3]*[b]*4
%[a2b2]=[a2]*[b2]*6
%[ab3]=[a]*[b3]*4
Rozwinąć według wzoru Newtona: $([a]x-[b])^4$
\zadStop
\rozwStart{Pascal Nawrocki}{Jakub Ulrych}
Na początku przypomnijmy wzór Newtona:$$(a+b)^n={n\choose 0}a^{n}b^{0}+{n\choose 1}a^{n-1}b^{1}+{n\choose 2}a^{n-2}b^{2}+\dots+{n\choose n}a^{0}b^{n}=\sum_{k=0}^{n} {n\choose k}a^{n-k}b^{k}$$
Zatem:
$$([a]x-[b])^4=$$
$$={4\choose 0}([a]x)^{4}(-[b])^{0}+{4\choose 1}([a]x)^{3}(-[b])^{1}+{4\choose 2}([a]x)^{2}(-[b])^{2}+{4\choose 3}([a]x)^{1}(-[b])^{3}+{4\choose 4}([a]x)^{0}(-[b])^{4}=$$
$$=1\cdot[a4]x^4\cdot1+4\cdot[a3]x^3\cdot(-[b])+6\cdot[a2]x^2\cdot([b2])+4\cdot[a]x\cdot([b3])+1\cdot1\cdot[b4]=$$
$$=[a4]x^4[a3b]x^3+[a2b2]x^2[ab3]x+[b4]$$
\odpStop
\testStart
A.$[a4]x^4[a3b]x^3+[a2b2]x^2[ab3]x+[b4]$
B.$[a4]x^4[a3b]x^3[ab3]x+[b4]$
C.$[a4]x^4[a3b]x^3+[a2b2]x^2[ab3]x$
D.$[a4]x^4+[a2b2]x^2[ab3]x+[b4]$
\testStop
\kluczStart
A
\kluczStop


\end{document}