\documentclass[12pt, a4paper]{article}
\usepackage[utf8]{inputenc}
\usepackage{polski}

\usepackage{amsthm}  %pakiet do tworzenia twierdzeń itp.
\usepackage{amsmath} %pakiet do niektórych symboli matematycznych
\usepackage{amssymb} %pakiet do symboli mat., np. \nsubseteq
\usepackage{amsfonts}
\usepackage{graphicx} %obsługa plików graficznych z rozszerzeniem png, jpg
\theoremstyle{definition} %styl dla definicji
\newtheorem{zad}{} 
\title{Multizestaw zadań}
\author{Robert Fidytek}
%\date{\today}
\date{}
\newcounter{liczniksekcji}
\newcommand{\kategoria}[1]{\section{#1}} %olreślamy nazwę kateforii zadań
\newcommand{\zadStart}[1]{\begin{zad}#1\newline} %oznaczenie początku zadania
\newcommand{\zadStop}{\end{zad}}   %oznaczenie końca zadania
%Makra opcjonarne (nie muszą występować):
\newcommand{\rozwStart}[2]{\noindent \textbf{Rozwiązanie (autor #1 , recenzent #2): }\newline} %oznaczenie początku rozwiązania, opcjonarnie można wprowadzić informację o autorze rozwiązania zadania i recenzencie poprawności wykonania rozwiązania zadania
\newcommand{\rozwStop}{\newline}                                            %oznaczenie końca rozwiązania
\newcommand{\odpStart}{\noindent \textbf{Odpowiedź:}\newline}    %oznaczenie początku odpowiedzi końcowej (wypisanie wyniku)
\newcommand{\odpStop}{\newline}                                             %oznaczenie końca odpowiedzi końcowej (wypisanie wyniku)
\newcommand{\testStart}{\noindent \textbf{Test:}\newline} %ewentualne możliwe opcje odpowiedzi testowej: A. ? B. ? C. ? D. ? itd.
\newcommand{\testStop}{\newline} %koniec wprowadzania odpowiedzi testowych
\newcommand{\kluczStart}{\noindent \textbf{Test poprawna odpowiedź:}\newline} %klucz, poprawna odpowiedź pytania testowego (jedna literka): A lub B lub C lub D itd.
\newcommand{\kluczStop}{\newline} %koniec poprawnej odpowiedzi pytania testowego 
\newcommand{\wstawGrafike}[2]{\begin{figure}[h] \includegraphics[scale=#2] {#1} \end{figure}} %gdyby była potrzeba wstawienia obrazka, parametry: nazwa pliku, skala (jak nie wiesz co wpisać, to wpisz 1)

\begin{document}
\maketitle


\kategoria{Wikieł/Z5.55i}
\zadStart{Zadanie z Wikieł Z 5.55i) moja wersja nr [nrWersji]}
%[a]=random.randint(0,5)
%[b]:[1,2,3,4,5]
%[c]:[2,3,4,5]
%[e]:[1,2,3,4,5]
%[f]:[1,2,3,4,5]
%[g]:[1,2,3,4,5]
%[d]=[b]+[c]*[a]
%[w]=[g]*[e]*[c]
%[ge]=[g]*[e]
%[ce]=[c]*[e]
%[cg]=[c]*[g]
%[w]!=[ge] and [w]!=[ce] and [w]!=[cg] and [ce]!=[ge] and [ce]!=[cg] and [cg]!=[ge] and [a]!=[w] and [a]!=[ge] and [a]!=[ce] and [a]!=[cg]
Na podstawie podanych wartości $f'([f])=[g],$ $g'([d])=[e],$ $g([d])=[f]$ obliczyć wartość następującej pochodnej $\frac{d}{dx}\left[f(g([b]+[c]x))\right]\big |_{x=[a]}$.
\zadStop
\rozwStart{Justyna Chojecka}{}
Zauważmy, że 
$$\left[f(g([b]+[c]x))\right]'=f'(g([b]+[c]x))\cdot \left(g([b]+[c]x)\right)'$$$$=f'(g([b]+[c]x))\cdot g'([b]+[c]x)\cdot ([b]+[c]x)'$$$$=f'(g([b]+[c]x))\cdot g'([b]+[c]x)\cdot [c].$$
Obliczamy wartość pochodnej $(\left[f(g([b]+[c]x))\right]'$ dla $x=[a]$.
$$\left[f(g([b]+[c]\cdot [a]))\right]'=f'(g([b]+[c]\cdot [a]))\cdot g'([b]+[c]\cdot [a])\cdot [c]$$$$=f'(g([d]))\cdot g'([d])\cdot [c]=f'([f])\cdot [e]\cdot [c]=[g]\cdot [e] \cdot [c]=[w]$$
\rozwStop
\odpStart
$[w]$
\odpStop
\testStart
A.$[w]$
B.$-[ge]$
C.$-[ce]$
D.$-[w]$
E.$[ce]$
F.$[cg]$
G.$[a]$
H.$[ge]$
I.$-[cg]$
\testStop
\kluczStart
A
\kluczStop



\end{document}