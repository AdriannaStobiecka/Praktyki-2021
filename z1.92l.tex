\documentclass[12pt, a4paper]{article}
\usepackage[utf8]{inputenc}
\usepackage{polski}

\usepackage{amsthm}  %pakiet do tworzenia twierdzeń itp.
\usepackage{amsmath} %pakiet do niektórych symboli matematycznych
\usepackage{amssymb} %pakiet do symboli mat., np. \nsubseteq
\usepackage{amsfonts}
\usepackage{graphicx} %obsługa plików graficznych z rozszerzeniem png, jpg
\theoremstyle{definition} %styl dla definicji
\newtheorem{zad}{} 
\title{Multizestaw zadań}
\author{Robert Fidytek}
%\date{\today}
\date{}
\newcounter{liczniksekcji}
\newcommand{\kategoria}[1]{\section{#1}} %olreślamy nazwę kateforii zadań
\newcommand{\zadStart}[1]{\begin{zad}#1\newline} %oznaczenie początku zadania
\newcommand{\zadStop}{\end{zad}}   %oznaczenie końca zadania
%Makra opcjonarne (nie muszą występować):
\newcommand{\rozwStart}[2]{\noindent \textbf{Rozwiązanie (autor #1 , recenzent #2): }\newline} %oznaczenie początku rozwiązania, opcjonarnie można wprowadzić informację o autorze rozwiązania zadania i recenzencie poprawności wykonania rozwiązania zadania
\newcommand{\rozwStop}{\newline}                                            %oznaczenie końca rozwiązania
\newcommand{\odpStart}{\noindent \textbf{Odpowiedź:}\newline}    %oznaczenie początku odpowiedzi końcowej (wypisanie wyniku)
\newcommand{\odpStop}{\newline}                                             %oznaczenie końca odpowiedzi końcowej (wypisanie wyniku)
\newcommand{\testStart}{\noindent \textbf{Test:}\newline} %ewentualne możliwe opcje odpowiedzi testowej: A. ? B. ? C. ? D. ? itd.
\newcommand{\testStop}{\newline} %koniec wprowadzania odpowiedzi testowych
\newcommand{\kluczStart}{\noindent \textbf{Test poprawna odpowiedź:}\newline} %klucz, poprawna odpowiedź pytania testowego (jedna literka): A lub B lub C lub D itd.
\newcommand{\kluczStop}{\newline} %koniec poprawnej odpowiedzi pytania testowego 
\newcommand{\wstawGrafike}[2]{\begin{figure}[h] \includegraphics[scale=#2] {#1} \end{figure}} %gdyby była potrzeba wstawienia obrazka, parametry: nazwa pliku, skala (jak nie wiesz co wpisać, to wpisz 1)

\begin{document}
\maketitle


\kategoria{Wikieł/Z1.92l}
\zadStart{Zadanie z Wikieł Z 1.92 l) moja wersja nr [nrWersji]}
%[a]:[2,3,4,5,6,7,8,9]
%[b]:[2,3,4,5,6,7,8,9,10]
%[c]:[2,3,4,5]
%[d]=random.randint(1,9)
%[e]=random.randint(2,5)
%[f]=random.randint(1,9)
%[g]=[a]-[b]
%[t1]=[g]*[e]
%[t2]=[g]*[f]
%[k1]=[a]*[c]
%[k2]=[a]*[d]
%[p1]=[k1]-[t1]
%[p2]=[k2]-[t2]
%[m]=math.gcd([p1],[p2])
%[x1]=int([p1]/[m])
%[x2]=int([p2]/[m])
%[a]!=[b] and [g]>0 and math.gcd([a],[b])==1 and [p1]!=0 and ([p2]*[c])!=([d]*[p1]) and ([p2]*[e])!=([f]*[p1]) and ([e]*([a]*[d]+[b]*[f]-[a]*[f])-([f]*([b]*[e]+[a]*[c]-[a]*[e])))>0 and ([c]*([a]*[d]+[b]*[f]-[a]*[f])-([d]*([b]*[e]+[a]*[c]-[a]*[e])))<([e]*([a]*[d]+[b]*[f]-[a]*[f])-([f]*([b]*[e]+[a]*[c]-[a]*[e]))) and ([e]*([a]*[d]+[b]*[f]-[a]*[f])-([f]*([b]*[e]+[a]*[c]-[a]*[e])))>0 and [x1]!=1
Rozwiązać równanie $\log_{\frac{[a]}{[b]}}{\Big(1-\frac{[c]x-[d]}{[e]x-[f]}\Big)}=-1$
\zadStop
\rozwStart{Małgorzata Ugowska}{}
$$\log_{\frac{[a]}{[b]}}{\Big(1-\frac{[c]x-[d]}{[e]x-[f]}\Big)}=-1 $$
$$ \Big(\frac{[a]}{[b]}\Big)^{-1}= 1-\frac{[c]x-[d]}{[e]x-[f]} $$
$$ \frac{[b]}{[a]} = 1-\frac{[c]x-[d]}{[e]x-[f]} $$
$$ \frac{[c]x-[d]}{[e]x-[f]} =1-\frac{[b]}{[a]} $$
$$ \frac{[c]x-[d]}{[e]x-[f]} =\frac{[g]}{[a]} $$
$$ [a]([c]x-[d])=[g]([e]x-[f])$$
$$ [k1]x-[k2]=[t1]x-[t2] $$
$$ [p1]x=[p2] $$
$$ x=\frac{[x2]}{[x1]}$$
\rozwStop
\odpStart
$x=\frac{[x2]}{[x1]}$
\odpStop
\testStart
A. $[k1]$
B. $[t2]$
C. $\frac{[x2]}{[x1]}$
D. $[p1]$
E. $[k2]$
\testStop
\kluczStart
C
\kluczStop



\end{document}