\documentclass[12pt, a4paper]{article}
\usepackage[utf8]{inputenc}
\usepackage{polski}

\usepackage{amsthm}  %pakiet do tworzenia twierdzeń itp.
\usepackage{amsmath} %pakiet do niektórych symboli matematycznych
\usepackage{amssymb} %pakiet do symboli mat., np. \nsubseteq
\usepackage{amsfonts}
\usepackage{graphicx} %obsługa plików graficznych z rozszerzeniem png, jpg
\theoremstyle{definition} %styl dla definicji
\newtheorem{zad}{} 
\title{Multizestaw zadań}
\author{Robert Fidytek}
%\date{\today}
\date{}
\newcounter{liczniksekcji}
\newcommand{\kategoria}[1]{\section{#1}} %olreślamy nazwę kateforii zadań
\newcommand{\zadStart}[1]{\begin{zad}#1\newline} %oznaczenie początku zadania
\newcommand{\zadStop}{\end{zad}}   %oznaczenie końca zadania
%Makra opcjonarne (nie muszą występować):
\newcommand{\rozwStart}[2]{\noindent \textbf{Rozwiązanie (autor #1 , recenzent #2): }\newline} %oznaczenie początku rozwiązania, opcjonarnie można wprowadzić informację o autorze rozwiązania zadania i recenzencie poprawności wykonania rozwiązania zadania
\newcommand{\rozwStop}{\newline}                                            %oznaczenie końca rozwiązania
\newcommand{\odpStart}{\noindent \textbf{Odpowiedź:}\newline}    %oznaczenie początku odpowiedzi końcowej (wypisanie wyniku)
\newcommand{\odpStop}{\newline}                                             %oznaczenie końca odpowiedzi końcowej (wypisanie wyniku)
\newcommand{\testStart}{\noindent \textbf{Test:}\newline} %ewentualne możliwe opcje odpowiedzi testowej: A. ? B. ? C. ? D. ? itd.
\newcommand{\testStop}{\newline} %koniec wprowadzania odpowiedzi testowych
\newcommand{\kluczStart}{\noindent \textbf{Test poprawna odpowiedź:}\newline} %klucz, poprawna odpowiedź pytania testowego (jedna literka): A lub B lub C lub D itd.
\newcommand{\kluczStop}{\newline} %koniec poprawnej odpowiedzi pytania testowego 
\newcommand{\wstawGrafike}[2]{\begin{figure}[h] \includegraphics[scale=#2] {#1} \end{figure}} %gdyby była potrzeba wstawienia obrazka, parametry: nazwa pliku, skala (jak nie wiesz co wpisać, to wpisz 1)

\begin{document}
\maketitle


\kategoria{Wikieł/Z3.21a}
\zadStart{Zadanie z Wikieł Z 3.21 a) moja wersja nr [nrWersji]}
%[a]:[4,9,16]
%[b]:[15,28,35,40,45,63,70,88,99,143,195
%[2b]=2*[b]
%[delta1]=[a]**2
%[delta2]=4*[b]*[a]
%[delta]=[delta1]+[delta2]
%[pdelta]=int(math.sqrt(abs([delta])))
%[lx1]=-[a]-[pdelta]
%[lx2]=-[a]+[pdelta]
%[u1]=math.gcd([lx1],[2b])
%[lx11]=int([lx1]/[u1])
%[2b1]=int([2b]/[u1])
%[u2]=math.gcd([lx2],[2b])
%[lx22]=int([lx2]/[u2])
%[2b2]=int([2b]/[u2])
%[lx11m]=-[lx11]
%math.gcd([a],[b])==1 and [lx1]/[2b]>-1 and [lx1]/[2b]<1 and [pdelta]**2==[delta] and ([lx1]/[2b])<1 and ([lx1]/[2b])>-1 and ([lx2]/[2b])<1 and ([lx2]/[2b])>-1
Rozwiązać równanie, którego lewa strona jest sumą nieskończonego ciągu geometrycznego.
$$x^2+x^3+x^4+\cdots=\frac{[a]}{[b]}$$
\zadStop
\rozwStart{Adrianna Stobiecka}{}
Lewa strona równania jest sumą nieskończonego ciągu geometrycznego o pierwszym wyrazie $a_1=x^2$ oraz ilorazie $q=x$. Mamy:
$$|q|<1\qquad\Leftrightarrow\qquad|x|<1\qquad\Leftrightarrow\qquad x<1~~\land~~x>-1\qquad\Leftrightarrow\qquad x\in(-1,1)$$
Obliczamy sumę znajdującą się po lewej stronie równania.
$$S=\frac{a_1}{1-q}=\frac{x^2}{1-x}$$
Obliczoną sumę wstawiamy do równania.
$$\frac{x^2}{1-x}=\frac{[a]}{[b]}\Leftrightarrow[a](1-x)=[b]x^2\Leftrightarrow[a]-[a]x=[b]x^2\Leftrightarrow[b]x^2+[a]x-[a]=0$$
Otrzymaliśmy równanie kwadratowe. Znajdziemy teraz jego pierwiastki.
$$\Delta=[a]^2-4\cdot[b]\cdot(-[a])=[delta1]+[delta2]=[delta]\qquad\Rightarrow\qquad\sqrt{\Delta}=[pdelta]$$
$$x_1=\frac{-[a]-[pdelta]}{2\cdot[b]}=-\frac{[lx11m]}{[2b1]},\qquad x_2=\frac{-[a]+[pdelta]}{2\cdot[b]}=\frac{[lx22]}{[2b2]}$$
$x_1$ oraz $x_2$ należą do przedziału $(-1,1)$, zatem rozwiązaniem naszego równania jest $x\in\{-\frac{[lx11m]}{[2b1]},\frac{[lx22]}{[2b2]}\}$.
\rozwStop
\odpStart
$x\in\{-\frac{[lx11m]}{[2b1]},\frac{[lx22]}{[2b2]}\}$
\odpStop
\testStart
A.$x=-2$
B.$x\in\{-\frac{[lx22]}{[2b2]},\frac{[lx11m]}{[2b1]}\}$
C.$x\in\{-\frac{[lx11m]}{[2b1]},\frac{[lx22]}{[2b2]}\}$
D.$x=-\frac{[lx22]}{[2b2]}$
E.$x=\frac{[lx11m]}{[2b1]}$
F.$x=-\frac{[lx11m]}{[2b1]}$
G.$x\in\emptyset$
H.$x=\frac{[lx22]}{[2b2]}$
I.$x=3$
\testStop
\kluczStart
C
\kluczStop



\end{document}
