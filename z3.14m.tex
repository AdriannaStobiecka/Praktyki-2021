\documentclass[12pt, a4paper]{article}
\usepackage[utf8]{inputenc}
\usepackage{polski}

\usepackage{amsthm}  %pakiet do tworzenia twierdzeń itp.
\usepackage{amsmath} %pakiet do niektórych symboli matematycznych
\usepackage{amssymb} %pakiet do symboli mat., np. \nsubseteq
\usepackage{amsfonts}
\usepackage{graphicx} %obsługa plików graficznych z rozszerzeniem png, jpg
\theoremstyle{definition} %styl dla definicji
\newtheorem{zad}{} 
\title{Multizestaw zadań}
\author{Robert Fidytek}
%\date{\today}
\date{}
\newcounter{liczniksekcji}
\newcommand{\kategoria}[1]{\section{#1}} %olreślamy nazwę kateforii zadań
\newcommand{\zadStart}[1]{\begin{zad}#1\newline} %oznaczenie początku zadania
\newcommand{\zadStop}{\end{zad}}   %oznaczenie końca zadania
%Makra opcjonarne (nie muszą występować):
\newcommand{\rozwStart}[2]{\noindent \textbf{Rozwiązanie (autor #1 , recenzent #2): }\newline} %oznaczenie początku rozwiązania, opcjonarnie można wprowadzić informację o autorze rozwiązania zadania i recenzencie poprawności wykonania rozwiązania zadania
\newcommand{\rozwStop}{\newline}                                            %oznaczenie końca rozwiązania
\newcommand{\odpStart}{\noindent \textbf{Odpowiedź:}\newline}    %oznaczenie początku odpowiedzi końcowej (wypisanie wyniku)
\newcommand{\odpStop}{\newline}                                             %oznaczenie końca odpowiedzi końcowej (wypisanie wyniku)
\newcommand{\testStart}{\noindent \textbf{Test:}\newline} %ewentualne możliwe opcje odpowiedzi testowej: A. ? B. ? C. ? D. ? itd.
\newcommand{\testStop}{\newline} %koniec wprowadzania odpowiedzi testowych
\newcommand{\kluczStart}{\noindent \textbf{Test poprawna odpowiedź:}\newline} %klucz, poprawna odpowiedź pytania testowego (jedna literka): A lub B lub C lub D itd.
\newcommand{\kluczStop}{\newline} %koniec poprawnej odpowiedzi pytania testowego 
\newcommand{\wstawGrafike}[2]{\begin{figure}[h] \includegraphics[scale=#2] {#1} \end{figure}} %gdyby była potrzeba wstawienia obrazka, parametry: nazwa pliku, skala (jak nie wiesz co wpisać, to wpisz 1)

\begin{document}
\maketitle


\kategoria{Wikieł/Z3.14m}
\zadStart{Zadanie z Wikieł Z 3.14 m) moja wersja nr [nrWersji]}
%[z]:[1,2,3,4,5,6,7,8,9,10,11,12,13,14,15,16,17,18,19,20,21,22,23,24,25,26,27,28,29,30,31,32,33,34,35,36,37,38,39,40,41,42,43,44,45,46,47,48,49,50]
%[z1]=[z]+1
%[z2]=[z]+2
Obliczyć granicę ciągu $a_n= \sqrt[n]{[z]+\frac{[z]}{2} + \frac{[z]}{3}+ \ldots + \frac{[z]}{n}}$.
\zadStop
\rozwStart{Barbara Bączek}{}
$$\lim_{n \rightarrow \infty} a_n= \lim_{n \rightarrow \infty} \sqrt[n]{[z]+\frac{[z]}{2} + \frac{[z]}{3}+ \ldots + \frac{[z]}{n}}$$
Obustronnie ograniczamy $a_n$. 
$$ \sqrt[n]{\frac{[z]}{n}+ \frac{[z]}{n} +  \frac{[z]}{n}+ \ldots + \frac{[z]}{n}}  \leq \sqrt[n]{[z]+\frac{[z]}{2} + \frac{[z]}{3}+ \ldots + \frac{[z]}{n}} \leq \sqrt[n]{[z]+ [z] + [z]+ \ldots + [z]} $$
$$ \sqrt[n]{n \cdot \frac{[z]}{n}}  \leq \sqrt[n]{[z]+\frac{[z]}{2} + \frac{[z]}{3}+ \ldots + \frac{[z]}{n}} \leq \sqrt[n]{n \cdot [z]}$$
$$\sqrt[n]{[z]}  \leq \sqrt[n]{[z]+\frac{[z]}{2} + \frac{[z]}{3}+ \ldots + \frac{[z]}{n}} \leq \sqrt[n]{n} \cdot \sqrt[n]{[z]}$$
Skorzystamy z dwóch zależności.
\begin{itemize}
\item Jeżeli $a>0$, to $$\lim_{n \rightarrow \infty} \sqrt[n]{a}=1.$$
\item Jeżeli $n \in \mathbb{N}$, to $$\lim_{n \rightarrow \infty} \sqrt[n]{n}=1.$$
\end{itemize}
Zatem, gdy $n \rightarrow \infty$, otrzymujemy
$$ 1 \leq \lim_{n \rightarrow \infty}  \sqrt[n]{[z]+\frac{[z]}{2} + \frac{[z]}{3}+ \ldots + \frac{[z]}{n}} \leq 1$$
Podsumowując, z twierdzenia o trzech ciągach:  $$\lim_{n \rightarrow \infty} a_n= 1$$
\rozwStop
\odpStart
$1$
\odpStop
\testStart
A.$\infty$
B.$1$
C.$-\infty$
D.$0$
E.$-1$
G.$[z1]$
H.$[z2]$
\testStop
\kluczStart
B
\kluczStop



\end{document}