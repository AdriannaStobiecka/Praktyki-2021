\documentclass[12pt, a4paper]{article}
\usepackage[utf8]{inputenc}
\usepackage{polski}

\usepackage{amsthm}  %pakiet do tworzenia twierdzeń itp.
\usepackage{amsmath} %pakiet do niektórych symboli matematycznych
\usepackage{amssymb} %pakiet do symboli mat., np. \nsubseteq
\usepackage{amsfonts}
\usepackage{graphicx} %obsługa plików graficznych z rozszerzeniem png, jpg
\theoremstyle{definition} %styl dla definicji
\newtheorem{zad}{} 
\title{Multizestaw zadań}
\author{Robert Fidytek}
%\date{\today}
\date{}
\newcounter{liczniksekcji}
\newcommand{\kategoria}[1]{\section{#1}} %olreślamy nazwę kateforii zadań
\newcommand{\zadStart}[1]{\begin{zad}#1\newline} %oznaczenie początku zadania
\newcommand{\zadStop}{\end{zad}}   %oznaczenie końca zadania
%Makra opcjonarne (nie muszą występować):
\newcommand{\rozwStart}[2]{\noindent \textbf{Rozwiązanie (autor #1 , recenzent #2): }\newline} %oznaczenie początku rozwiązania, opcjonarnie można wprowadzić informację o autorze rozwiązania zadania i recenzencie poprawności wykonania rozwiązania zadania
\newcommand{\rozwStop}{\newline}                                            %oznaczenie końca rozwiązania
\newcommand{\odpStart}{\noindent \textbf{Odpowiedź:}\newline}    %oznaczenie początku odpowiedzi końcowej (wypisanie wyniku)
\newcommand{\odpStop}{\newline}                                             %oznaczenie końca odpowiedzi końcowej (wypisanie wyniku)
\newcommand{\testStart}{\noindent \textbf{Test:}\newline} %ewentualne możliwe opcje odpowiedzi testowej: A. ? B. ? C. ? D. ? itd.
\newcommand{\testStop}{\newline} %koniec wprowadzania odpowiedzi testowych
\newcommand{\kluczStart}{\noindent \textbf{Test poprawna odpowiedź:}\newline} %klucz, poprawna odpowiedź pytania testowego (jedna literka): A lub B lub C lub D itd.
\newcommand{\kluczStop}{\newline} %koniec poprawnej odpowiedzi pytania testowego 
\newcommand{\wstawGrafike}[2]{\begin{figure}[h] \includegraphics[scale=#2] {#1} \end{figure}} %gdyby była potrzeba wstawienia obrazka, parametry: nazwa pliku, skala (jak nie wiesz co wpisać, to wpisz 1)

\begin{document}
\maketitle


\kategoria{Wikieł/Z5.2f}
\zadStart{Zadanie z Wikieł Z 5.2f) moja wersja nr [nrWersji]}
%[a]:[2,4,6,8,10,12,14,16,18]
%[y]:[2,3,4,5,6,7,8,9,10,11,12,15,17]
%[k]=random.randint(2,20)
%[b]=random.randint(2,100)
%[m]=[b]*[a]*(-1)
%[k2]=[a]*[k]
%[w]=round(([a]*3.14159)/[k],2)
%[w1]=round(math.sin([w]),2)
%[w2]=round([m]*[w1],2)
Korzystając z definicji pochodnej, obliczyć wartość wskazanej pochodnej funkcji $f$.\\
 $f(x)=[b]cos([a]x)$, $f'(\frac{\pi}{[k]})$
\zadStop
\rozwStart{Katarzyna Filipowicz}{}
Niech $h=\Delta x$
$$
f'(x)=lim_{h\rightarrow 0} \frac{[b]cos([a](x+h))-[b]cos([a]x)}{h}=
$$ $$
=[b] \cdot lim_{h\rightarrow 0} \frac{cos([a]x+[a]h)-cos([a]x)}{h}=
$$ $$
=[b]\cdot lim_{h\rightarrow 0} \frac{cos([a]x)cos([a]h)-sin([a]x)sin([a]h)-cos([a]x)}{h}=
$$ $$
=[b]\cdot lim_{h\rightarrow 0} \frac{cos([a]x)(cos([a]h)-1)-sin([a]x)sin([a]h)}{h}=
$$ $$
=[b]\cdot lim_{h\rightarrow 0} \frac{cos([a]x)(cos([a]h)-1)}{h}-[b]\cdot lim_{h\rightarrow 0}  \frac{sin([a]x)sin([a]h)}{h}=
$$ $$
=0-[b]\cdot lim_{h\rightarrow 0} sin([a]x)\frac{sin([a]h)}{h}=-[b]\cdot [a]sin([a]x)=[m]sin([a]x)
$$ $$
f'\left(\frac{\pi}{[k]}\right)=[m]sin\left([a]\cdot \frac{\pi}{[k]}\right)=[m] \cdot ([w1])=[w2]
$$
\rozwStop
\odpStart
$f'(\frac{\pi}{[k]})=[w2]$
\odpStop
\testStart
A.$f'(\frac{\pi}{[k]})=[w2]$
B.$f'(\frac{\pi}{[k]})=0$
C.$f'(\frac{\pi}{[k]})=[a]sin([a]x)$
D.$f'(\frac{\pi}{[k]})=[a]cos(x)$
E.$f'(\frac{\pi}{[k]})=[w1]$
F.$f'(\frac{\pi}{[k]})=-[b]$
G.$f'(\frac{\pi}{[k]})=[k]^{\circ}$
H.$f'(\frac{\pi}{[k]})=-1$
I.$f'(\frac{\pi}{[k]})=[y]$
\testStop
\kluczStart
A
\kluczStop



\end{document}