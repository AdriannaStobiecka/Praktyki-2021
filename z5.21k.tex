\documentclass[12pt, a4paper]{article}
\usepackage[utf8]{inputenc}
\usepackage{polski}

\usepackage{amsthm}  %pakiet do tworzenia twierdzeń itp.
\usepackage{amsmath} %pakiet do niektórych symboli matematycznych
\usepackage{amssymb} %pakiet do symboli mat., np. \nsubseteq
\usepackage{amsfonts}
\usepackage{graphicx} %obsługa plików graficznych z rozszerzeniem png, jpg
\theoremstyle{definition} %styl dla definicji
\newtheorem{zad}{} 
\title{Multizestaw zadań}
\author{Robert Fidytek}
%\date{\today}
\date{}
\newcounter{liczniksekcji}
\newcommand{\kategoria}[1]{\section{#1}} %olreślamy nazwę kateforii zadań
\newcommand{\zadStart}[1]{\begin{zad}#1\newline} %oznaczenie początku zadania
\newcommand{\zadStop}{\end{zad}}   %oznaczenie końca zadania
%Makra opcjonarne (nie muszą występować):
\newcommand{\rozwStart}[2]{\noindent \textbf{Rozwiązanie (autor #1 , recenzent #2): }\newline} %oznaczenie początku rozwiązania, opcjonarnie można wprowadzić informację o autorze rozwiązania zadania i recenzencie poprawności wykonania rozwiązania zadania
\newcommand{\rozwStop}{\newline}                                            %oznaczenie końca rozwiązania
\newcommand{\odpStart}{\noindent \textbf{Odpowiedź:}\newline}    %oznaczenie początku odpowiedzi końcowej (wypisanie wyniku)
\newcommand{\odpStop}{\newline}                                             %oznaczenie końca odpowiedzi końcowej (wypisanie wyniku)
\newcommand{\testStart}{\noindent \textbf{Test:}\newline} %ewentualne możliwe opcje odpowiedzi testowej: A. ? B. ? C. ? D. ? itd.
\newcommand{\testStop}{\newline} %koniec wprowadzania odpowiedzi testowych
\newcommand{\kluczStart}{\noindent \textbf{Test poprawna odpowiedź:}\newline} %klucz, poprawna odpowiedź pytania testowego (jedna literka): A lub B lub C lub D itd.
\newcommand{\kluczStop}{\newline} %koniec poprawnej odpowiedzi pytania testowego 
\newcommand{\wstawGrafike}[2]{\begin{figure}[h] \includegraphics[scale=#2] {#1} \end{figure}} %gdyby była potrzeba wstawienia obrazka, parametry: nazwa pliku, skala (jak nie wiesz co wpisać, to wpisz 1)

\begin{document}
\maketitle


\kategoria{Wikieł/Z5.21 k}
\zadStart{Zadanie z Wikieł Z 5.21 k) moja wersja nr [nrWersji]}
%[a]:[4,9,16,25]
%[b]:[2,3,4,5,6,7,8,9]
%[c]=int(math.sqrt([a]))
%[d]=[a]+[b]
%[a]!=0 and [d]!=4 and [d]!=9 and [d]!=16 and [d]!=25
Wyznaczyć przedziały monotoniczności funkcji $f(x)=\ln(x^2-[a])+\frac{[b]}{x^2-[a]}$.
\zadStop
\rozwStart{Joanna Świerzbin}{}
$$f(x)=\ln(x^2-[a])+\frac{[b]}{x^2-[a]}$$
\\
$D_f:$
$$x^2-[a] > 0 $$
$$x^2 > [a] $$
$$x > \sqrt{[a]} \lor x < -\sqrt{[a]} $$
$$x > [c] \lor x < -[c] $$
$$x \in (-\infty,-[c]) \cup ([c], \infty) $$
\\
$$f'(x)=\frac{2x}{x^2-[a]}+\frac{-2\cdot[b]x}{(x^2-[a])^2} = \frac{2x(x^2-[a])-2\cdot[b]x}{(x^2-[a])^2}=$$
$$= \frac{2x^3-2\cdot[a]x-2\cdot[b]x}{(x^2-[a])^2}=\frac{2x^3-2\cdot[d]x}{(x^2-[a])^2}=\frac{2x(x^2-[d])}{(x^2-[a])^2}$$

Znajdźmy miejsca zerowe:
$$2x(x^2-[d])=0$$
$$x_1=0 \lor x_2=\sqrt{[d]} \lor x_3=-\sqrt{[d]} $$

\begin{enumerate}
\item 
$$\frac{2x(x^2-[d])}{(x^2-[a])^2}<0$$
$$2x(x^2-[d])<0$$
$$x\in \left(-\infty, -\sqrt{[d]}\right) \cup \left(0,\sqrt{[d]}\right) \land x \in \left(-\infty,-[c] \right) \cup \left([c], \infty \right) $$
$$x\in \left(-\infty, -\sqrt{[d]}\right) \cup \left([c],\sqrt{[d]}\right)$$
\item 
$$\frac{2x(x^2-[d])}{(x^2-[a])^2}>0$$
$$2x(x^2-[d])>0$$
$$x\in \left(-\sqrt{[d]},0 \right) \cup \left(\sqrt{[d]}, \infty \right) \land x \in \left(-\infty,-[c] \right) \cup \left([c], \infty \right) $$
$$x\in \left(-\sqrt{[d]},-[c] \right) \cup \left(\sqrt{[d]}, \infty \right)$$
\end{enumerate}
Funkcja $f$ jest: \\ rosnąca dla $x\in \left(-\sqrt{[d]},-[c] \right) \cup \left(\sqrt{[d]}, \infty \right)$ \\ malejąca dla $x\in \left(-\infty, -\sqrt{[d]}\right) \cup \left([c],\sqrt{[d]}\right)$.
\rozwStop
\odpStart
$f$ jest: \\ rosnąca dla $x\in \left(-\sqrt{[d]},-[c] \right) \cup \left(\sqrt{[d]}, \infty \right)$ \\ malejąca dla $x\in \left(-\infty, -\sqrt{[d]}\right) \cup \left([c],\sqrt{[d]}\right)$.
\odpStop
\testStart
A. $f$ jest: \\ rosnąca dla $x\in \left(-\sqrt{[d]},-[c] \right) \cup \left(\sqrt{[d]}, \infty \right)$ \\ malejąca dla $x\in \left(-\infty, -\sqrt{[d]}\right) \cup \left([c],\sqrt{[d]}\right)$.\\
B. $f$ jest: \\ rosnąca dla $x\in\left(-\sqrt{[d]},0 \right) \cup \left(\sqrt{[d]}, \infty \right)$ \\ malejąca dla $x\in \left(-\infty, -\sqrt{[d]}\right) \cup \left(0,\sqrt{[d]}\right)$.\\
C. $f$ jest: \\ rosnąca dla $x\in \left(-\sqrt{[d]},0 \right) \cup \left(\sqrt{[d]}, \infty \right)$ \\ malejąca dla $x\in \left(-\infty, -\sqrt{[d]}\right) \cup \left([c],\sqrt{[d]}\right)$.\\
D. $f$ jest: \\ rosnąca dla $x\in \left(-\sqrt{[d]},-[c] \right) \cup \left(\sqrt{[d]}, \infty \right)$ \\ malejąca dla $x\in \left(-\infty, -\sqrt{[d]}\right) \cup \left(0,\sqrt{[d]}\right)$.\\
E. $f$ jest: \\ rosnąca dla $x\in \left(-\sqrt{[d]},-[c] \right) \cup \left(\sqrt{[d]}, \infty \right)$ \\ malejąca dla $x\in \left(-\infty, -\sqrt{[d]}\right) \cup \left(-[c],\sqrt{[d]}\right)$.
\testStop
\kluczStart
A
\kluczStop



\end{document}