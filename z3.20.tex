\documentclass[12pt, a4paper]{article}
\usepackage[utf8]{inputenc}
\usepackage{polski}

\usepackage{amsthm}  %pakiet do tworzenia twierdzeń itp.
\usepackage{amsmath} %pakiet do niektórych symboli matematycznych
\usepackage{amssymb} %pakiet do symboli mat., np. \nsubseteq
\usepackage{amsfonts}
\usepackage{graphicx} %obsługa plików graficznych z rozszerzeniem png, jpg
\theoremstyle{definition} %styl dla definicji
\newtheorem{zad}{} 
\title{Multizestaw zadań}
\author{Robert Fidytek}
%\date{\today}
\date{}
\newcounter{liczniksekcji}
\newcommand{\kategoria}[1]{\section{#1}} %olreślamy nazwę kateforii zadań
\newcommand{\zadStart}[1]{\begin{zad}#1\newline} %oznaczenie początku zadania
\newcommand{\zadStop}{\end{zad}}   %oznaczenie końca zadania
%Makra opcjonarne (nie muszą występować):
\newcommand{\rozwStart}[2]{\noindent \textbf{Rozwiązanie (autor #1 , recenzent #2): }\newline} %oznaczenie początku rozwiązania, opcjonarnie można wprowadzić informację o autorze rozwiązania zadania i recenzencie poprawności wykonania rozwiązania zadania
\newcommand{\rozwStop}{\newline}                                            %oznaczenie końca rozwiązania
\newcommand{\odpStart}{\noindent \textbf{Odpowiedź:}\newline}    %oznaczenie początku odpowiedzi końcowej (wypisanie wyniku)
\newcommand{\odpStop}{\newline}                                             %oznaczenie końca odpowiedzi końcowej (wypisanie wyniku)
\newcommand{\testStart}{\noindent \textbf{Test:}\newline} %ewentualne możliwe opcje odpowiedzi testowej: A. ? B. ? C. ? D. ? itd.
\newcommand{\testStop}{\newline} %koniec wprowadzania odpowiedzi testowych
\newcommand{\kluczStart}{\noindent \textbf{Test poprawna odpowiedź:}\newline} %klucz, poprawna odpowiedź pytania testowego (jedna literka): A lub B lub C lub D itd.
\newcommand{\kluczStop}{\newline} %koniec poprawnej odpowiedzi pytania testowego 
\newcommand{\wstawGrafike}[2]{\begin{figure}[h] \includegraphics[scale=#2] {#1} \end{figure}} %gdyby była potrzeba wstawienia obrazka, parametry: nazwa pliku, skala (jak nie wiesz co wpisać, to wpisz 1)

\begin{document}
\maketitle


\kategoria{Wikieł/Z3.20}
\zadStart{Zadanie z Wikieł Z 3.20 moja wersja nr [nrWersji]}
%[a]:[3,4,5,6,7,8,9,10]
%[b]:[2,3,4,5,6,7,8,9]
%[a2]=[a]*[a]
%[2a]=2*[a]
%[a3]=[a]**3
%[d]=math.gcd([b],[2a])
%[b1]=int([b]/[d])
%[a21]=int([2a]/[d])
%math.gcd([a],[b])==1 and [b1]<[a21] and [b]<[a]
Dana jest funkcja $f(x)=\frac{[a]x}{[a]x-[b]}-\frac{[a2]x^2}{([a]x-[b])^2}+\frac{[a3]x^3}{([a]x-[b])^3}-\cdots$, której prawa strona jest sumą wyrazów nieskończonego ciąg geometrycznego. Znaleźć dziedzinę tej funkcji.
\zadStop
\rozwStart{Adrianna Stobiecka}{}
Mianownik ułamka nie może być równy zero.
$$[a]x-[b]\ne0\qquad\Leftrightarrow\qquad[a]x\ne[b]\qquad\Leftrightarrow\qquad x\ne\frac{[b]}{[a]}$$
Aby suma nieskończonego ciągu geometrycznego mogła istnieć musi być spełnione założenie $|q|<1$. W przypadku naszej funkcji $q=\frac{-[a]x}{[a]x-[b]}$. 
$$|q|<1\qquad\Leftrightarrow\qquad \bigg|\frac{-[a]x}{[a]x-[b]}\bigg|<1\qquad\Leftrightarrow\qquad \frac{-[a]x}{[a]x-[b]}<1 ~~\land~~\frac{-[a]x}{[a]x-[b]}>-1$$
Rozwiążmy najpierw pierwszą z nierówności.
$$\frac{-[a]x}{[a]x-[b]}<1\Leftrightarrow\frac{-[a]x-[a]x+[b]}{[a]x-[b]}<0\Leftrightarrow(-[2a]x+[b])([a]x-[b])<0$$
Otrzymaliśmy parabolę z ramionami skierowanymi w dół oraz miejscami zerowymi w $x=\frac{[b1]}{[a21]}$ oraz $x=\frac{[b]}{[a]}$. Nierówność jest zatem spełniona dla $x\in(-\infty,\frac{[b1]}{[a21]})\cup(\frac{[b]}{[a]},\infty)$.
\\Przejdziemy teraz do drugiej nierówności.
$$\frac{-[a]x}{[a]x-[b]}>-1\Leftrightarrow\frac{-[a]x+[a]x-[b]}{[a]x-[b]}>0\Leftrightarrow\frac{-[b]}{[a]x-[b]}>0$$
$$\Leftrightarrow-[b]([a]x-[b])>0\Leftrightarrow[a]x-[b]<0\Leftrightarrow[a]x<[b]\Leftrightarrow x<\frac{[b]}{[a]}$$
 Nierówność jest zatem spełniona dla $x\in(-\infty,\frac{[b]}{[a]})$.
\\Mamy zatem:
$$x\ne\frac{[b]}{[a]}\qquad\land\qquad x\in\bigg(-\infty,\frac{[b1]}{[a21]}\bigg)\cup\bigg(\frac{[b]}{[a]},\infty\bigg)\qquad\land\qquad x\in\bigg(-\infty,\frac{[b]}{[a]}\bigg)$$
Ostatecznie otrzymujemy, że dziedziną funkcji jest $x\in(-\infty,\frac{[b1]}{[a21]})$.
\rozwStop
\odpStart
$x\in(-\infty,\frac{[b1]}{[a21]})$
\odpStop
\testStart
A.$x\in(\frac{[b]}{[a]},\infty)$
B.$x\in\mathbb{R}$
C.$x\in(\frac{[b1]}{[a21]},\infty)$
D.$x\in(-\infty,\frac{[b1]}{[a21]})$
E.$x\in[\frac{[b1]}{[a21]},\infty)$
F.$x\in(-\infty,\frac{[b1]}{[a21]}]$
G.$x\in\emptyset$
H.$x\in(-\infty,\frac{[b]}{[a]}]$
I.$x\in[\frac{[b]}{[a]},\infty)$
\testStop
\kluczStart
D
\kluczStop



\end{document}
