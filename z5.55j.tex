\documentclass[12pt, a4paper]{article}
\usepackage[utf8]{inputenc}
\usepackage{polski}

\usepackage{amsthm}  %pakiet do tworzenia twierdzeń itp.
\usepackage{amsmath} %pakiet do niektórych symboli matematycznych
\usepackage{amssymb} %pakiet do symboli mat., np. \nsubseteq
\usepackage{amsfonts}
\usepackage{graphicx} %obsługa plików graficznych z rozszerzeniem png, jpg
\theoremstyle{definition} %styl dla definicji
\newtheorem{zad}{} 
\title{Multizestaw zadań}
\author{Robert Fidytek}
%\date{\today}
\date{}
\newcounter{liczniksekcji}
\newcommand{\kategoria}[1]{\section{#1}} %olreślamy nazwę kateforii zadań
\newcommand{\zadStart}[1]{\begin{zad}#1\newline} %oznaczenie początku zadania
\newcommand{\zadStop}{\end{zad}}   %oznaczenie końca zadania
%Makra opcjonarne (nie muszą występować):
\newcommand{\rozwStart}[2]{\noindent \textbf{Rozwiązanie (autor #1 , recenzent #2): }\newline} %oznaczenie początku rozwiązania, opcjonarnie można wprowadzić informację o autorze rozwiązania zadania i recenzencie poprawności wykonania rozwiązania zadania
\newcommand{\rozwStop}{\newline}                                            %oznaczenie końca rozwiązania
\newcommand{\odpStart}{\noindent \textbf{Odpowiedź:}\newline}    %oznaczenie początku odpowiedzi końcowej (wypisanie wyniku)
\newcommand{\odpStop}{\newline}                                             %oznaczenie końca odpowiedzi końcowej (wypisanie wyniku)
\newcommand{\testStart}{\noindent \textbf{Test:}\newline} %ewentualne możliwe opcje odpowiedzi testowej: A. ? B. ? C. ? D. ? itd.
\newcommand{\testStop}{\newline} %koniec wprowadzania odpowiedzi testowych
\newcommand{\kluczStart}{\noindent \textbf{Test poprawna odpowiedź:}\newline} %klucz, poprawna odpowiedź pytania testowego (jedna literka): A lub B lub C lub D itd.
\newcommand{\kluczStop}{\newline} %koniec poprawnej odpowiedzi pytania testowego 
\newcommand{\wstawGrafike}[2]{\begin{figure}[h] \includegraphics[scale=#2] {#1} \end{figure}} %gdyby była potrzeba wstawienia obrazka, parametry: nazwa pliku, skala (jak nie wiesz co wpisać, to wpisz 1)

\begin{document}
\maketitle


\kategoria{Wikieł/Z5.55j}
\zadStart{Zadanie z Wikieł Z 5.55j) moja wersja nr [nrWersji]}
%[a]:[0,1,2,3,4,5,6,7,8,9]
%[b]:[7,8,9,10,11,12,13]
%[e]:[2,3,4,5]
%[c]=[b]-1
%[d]=-[b]
%[bd]=[b]*[d]
%[bb]=[b]*[b]
%[cd]=[c]*[d]
%[cc]=[c]*[c]
Na podstawie podanych wartości $g'([a])=[d],$ $g([a])=[e]$ obliczyć wartość następującej pochodnej $\frac{d}{dx}\left[g^{[b]}(x)\right]\big |_{x=[a]}$.
\zadStop
\rozwStart{Justyna Chojecka}{}
Zauważmy, że $g^{[b]}(x)$ jest złożeniem funkcji $f(x)=x^{[b]}$ oraz $g(x)$. Stąd 
$$g^{[b]}(x)=f(g(x))=(f\circ g)(x).$$
Następnie obliczamy pochodną powyższego złożenia
$$(g^{[b]}(x))'=(f\circ g)'(x)=f'(g(x))\cdot g'(x)$$$$=[b]g^{[c]}(x)\cdot g'(x)=[b](g(x))^{[c]}\cdot g'(x).$$
Obliczamy wartość pochodnej $(g^{[b]}(x))'$ dla $x=[a]$.
$$(g^{[b]}([a]))'=[b](g([a]))^{[c]}\cdot g'([a])=[b]\cdot [e]^{[c]}\cdot ([d])=[bd]\cdot [e]^{[c]}$$
\rozwStop
\odpStart
$[bd]\cdot [e]^{[c]}$
\odpStop
\testStart
A.$[bd]\cdot [e]^{[c]}$
B.$[bb]\cdot [e]^{[c]}$
C.$[bb]\cdot [e]^{[b]}$
D.$[cc]\cdot [e]^{[c]}$
E.$[cd]\cdot [e]^{[c]}$
F.$[bd]\cdot [e]^{[b]}$
G.$[cc]\cdot [e]^{[b]}$
H.$[cd]\cdot [e]^{[b]}$
I.$[bd]\cdot [e]^{-[c]}$
\testStop
\kluczStart
A
\kluczStop



\end{document}