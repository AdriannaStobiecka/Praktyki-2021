\documentclass[12pt, a4paper]{article}
\usepackage[utf8]{inputenc}
\usepackage{polski}

\usepackage{amsthm}  %pakiet do tworzenia twierdzeń itp.
\usepackage{amsmath} %pakiet do niektórych symboli matematycznych
\usepackage{amssymb} %pakiet do symboli mat., np. \nsubseteq
\usepackage{amsfonts}
\usepackage{graphicx} %obsługa plików graficznych z rozszerzeniem png, jpg
\theoremstyle{definition} %styl dla definicji
\newtheorem{zad}{} 
\title{Multizestaw zadań}
\author{Robert Fidytek}
%\date{\today}
\date{}
\newcounter{liczniksekcji}
\newcommand{\kategoria}[1]{\section{#1}} %olreślamy nazwę kateforii zadań
\newcommand{\zadStart}[1]{\begin{zad}#1\newline} %oznaczenie początku zadania
\newcommand{\zadStop}{\end{zad}}   %oznaczenie końca zadania
%Makra opcjonarne (nie muszą występować):
\newcommand{\rozwStart}[2]{\noindent \textbf{Rozwiązanie (autor #1 , recenzent #2): }\newline} %oznaczenie początku rozwiązania, opcjonarnie można wprowadzić informację o autorze rozwiązania zadania i recenzencie poprawności wykonania rozwiązania zadania
\newcommand{\rozwStop}{\newline}                                            %oznaczenie końca rozwiązania
\newcommand{\odpStart}{\noindent \textbf{Odpowiedź:}\newline}    %oznaczenie początku odpowiedzi końcowej (wypisanie wyniku)
\newcommand{\odpStop}{\newline}                                             %oznaczenie końca odpowiedzi końcowej (wypisanie wyniku)
\newcommand{\testStart}{\noindent \textbf{Test:}\newline} %ewentualne możliwe opcje odpowiedzi testowej: A. ? B. ? C. ? D. ? itd.
\newcommand{\testStop}{\newline} %koniec wprowadzania odpowiedzi testowych
\newcommand{\kluczStart}{\noindent \textbf{Test poprawna odpowiedź:}\newline} %klucz, poprawna odpowiedź pytania testowego (jedna literka): A lub B lub C lub D itd.
\newcommand{\kluczStop}{\newline} %koniec poprawnej odpowiedzi pytania testowego 
\newcommand{\wstawGrafike}[2]{\begin{figure}[h] \includegraphics[scale=#2] {#1} \end{figure}} %gdyby była potrzeba wstawienia obrazka, parametry: nazwa pliku, skala (jak nie wiesz co wpisać, to wpisz 1)

\begin{document}
\maketitle


\kategoria{Wikieł/Z2.57}
\zadStart{Zadanie z Wikieł Z 2.57 moja wersja nr [nrWersji]}
%[a]:[2,3,4,5,6,7,8,9,10,11,12,13,14,15,16,17,18,19,20]
%[b]:[2,3,4,5,6,7,8,9,10,11,12,13,14,15,16,17,18,19,20]
%[c]:[14,15,17,18,19,20,21,22,23,24,25,26,27,28,29,30,31,32,33,34,35,36,37,38,39,40,41,42,43,44,45,46,47,48,49,50]
%[bb]=2*[b]
%[ab]=[a]+[b]
%[kbb]=[bb]*[bb]
%[4ab]=4*[ab]
%[4abb]=[4ab]*[b]
%[4abc]=[4ab]*[c]
%[x]=[4abb]-[kbb]
%[mm]=[4abc]/[x]
%[cmm]=int([mm])
%[m]=math.sqrt([cmm])
%[cm]=int([m])
%[x]>0 and [m].is_integer()==True and [mm].is_integer()==True
Podać, dla jakiej wartości m prosta $y=-x+m$ jest styczna do elipsy $[a]x^2+[b]y^2=[c]$.
\zadStop
\rozwStart{Aleksandra Pasińska}{}
$$\left\{ \begin{array}{ll}
[a]x^2+[b]y^2=[c]\\ 
y=-x+m
\end{array} \right.$$
$$[a]x^2+[b](-x+m)^2=[c]$$
$$[a]x^2+[b]x^2-[bb]xm+[b]m^2-[c]=0$$
$$[ab]x^2-[bb]xm+[b]m^2-[c]=0$$
$$\Delta=[kbb]m^2-4\cdot [ab]([b]m^2-[c])=0$$
$$[kbb]m^2-[4abb]m^2+[4abc]=0$$
$$-[x]m^2=-[4abc]$$
$$m^2=[cmm]$$
$$m=\pm [cm]$$
\rozwStop
\odpStart
$m=\pm [cm]$\\
\odpStop
\testStart
A.$m=\pm [cm]$
B.$m=[cm]$
C.$m=-[cm]$
D.$m=[cm]$
E.$m=[cm]$
F.$m=[cm]$
G.$m=1$
H.$ m=0$
I.$m=-1$
\testStop
\kluczStart
A
\kluczStop



\end{document}