\documentclass[12pt, a4paper]{article}
\usepackage[utf8]{inputenc}
\usepackage{polski}

\usepackage{amsthm}  %pakiet do tworzenia twierdzeń itp.
\usepackage{amsmath} %pakiet do niektórych symboli matematycznych
\usepackage{amssymb} %pakiet do symboli mat., np. \nsubseteq
\usepackage{amsfonts}
\usepackage{graphicx} %obsługa plików graficznych z rozszerzeniem png, jpg
\theoremstyle{definition} %styl dla definicji
\newtheorem{zad}{} 
\title{Multizestaw zadań}
\author{Robert Fidytek}
%\date{\today}
\date{}
\newcounter{liczniksekcji}
\newcommand{\kategoria}[1]{\section{#1}} %olreślamy nazwę kateforii zadań
\newcommand{\zadStart}[1]{\begin{zad}#1\newline} %oznaczenie początku zadania
\newcommand{\zadStop}{\end{zad}}   %oznaczenie końca zadania
%Makra opcjonarne (nie muszą występować):
\newcommand{\rozwStart}[2]{\noindent \textbf{Rozwiązanie (autor #1 , recenzent #2): }\newline} %oznaczenie początku rozwiązania, opcjonarnie można wprowadzić informację o autorze rozwiązania zadania i recenzencie poprawności wykonania rozwiązania zadania
\newcommand{\rozwStop}{\newline}                                            %oznaczenie końca rozwiązania
\newcommand{\odpStart}{\noindent \textbf{Odpowiedź:}\newline}    %oznaczenie początku odpowiedzi końcowej (wypisanie wyniku)
\newcommand{\odpStop}{\newline}                                             %oznaczenie końca odpowiedzi końcowej (wypisanie wyniku)
\newcommand{\testStart}{\noindent \textbf{Test:}\newline} %ewentualne możliwe opcje odpowiedzi testowej: A. ? B. ? C. ? D. ? itd.
\newcommand{\testStop}{\newline} %koniec wprowadzania odpowiedzi testowych
\newcommand{\kluczStart}{\noindent \textbf{Test poprawna odpowiedź:}\newline} %klucz, poprawna odpowiedź pytania testowego (jedna literka): A lub B lub C lub D itd.
\newcommand{\kluczStop}{\newline} %koniec poprawnej odpowiedzi pytania testowego 
\newcommand{\wstawGrafike}[2]{\begin{figure}[h] \includegraphics[scale=#2] {#1} \end{figure}} %gdyby była potrzeba wstawienia obrazka, parametry: nazwa pliku, skala (jak nie wiesz co wpisać, to wpisz 1)

\begin{document}
\maketitle

\kategoria{Wikieł/P1.4}

\zadStart{Zadanie z Wikieł P 1.4) moja wersja nr [nrWersji]}
%[a]:[1,2,3,4,5,6,7,8,9,10,11,12,13,14,15,16,17,18,19,20]
Zapisać bez użycia modułu wyrażenie $\mid a - [a] \mid$.
\zadStop

\rozwStart{Natalia Danieluk}{}
$$\mid a - [a] \mid =
\begin{cases} 
a - [a], & \text{dla} \ a - [a] \ge 0 \\
- (a - [a]), & \text{dla}\ a - [a] < 0
\end{cases} 
=
\begin{cases} 
a - [a], & \text{dla} \ a \ge [a] \\
- (a - [a]), & \text{dla}\ a < [a]
\end{cases}
$$
\rozwStop

\odpStart
$
\begin{cases} 
a - [a], & \text{dla} \ a \ge [a] \\
- (a - [a]), & \text{dla}\ a < [a]
\end{cases}
$
\odpStop

\testStart
A. $
\begin{cases} 
a - [a], & \text{dla} \ a \ge [a] \\
- (a - [a]), & \text{dla}\ a < [a]
\end{cases}
$
B. $
\begin{cases} 
a - [a], & \text{dla} \ a < [a] \\
- (a - [a]), & \text{dla}\ a \ge [a]
\end{cases}
$
\testStop

\kluczStart
A
\kluczStop

\end{document}