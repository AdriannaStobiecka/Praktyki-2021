\documentclass[12pt, a4paper]{article}
\usepackage[utf8]{inputenc}
\usepackage{polski}

\usepackage{amsthm}  %pakiet do tworzenia twierdzeń itp.
\usepackage{amsmath} %pakiet do niektórych symboli matematycznych
\usepackage{amssymb} %pakiet do symboli mat., np. \nsubseteq
\usepackage{amsfonts}
\usepackage{graphicx} %obsługa plików graficznych z rozszerzeniem png, jpg
\theoremstyle{definition} %styl dla definicji
\newtheorem{zad}{} 
\title{Multizestaw zadań}
\author{Robert Fidytek}
%\date{\today}
\date{}
\newcounter{liczniksekcji}
\newcommand{\kategoria}[1]{\section{#1}} %olreślamy nazwę kateforii zadań
\newcommand{\zadStart}[1]{\begin{zad}#1\newline} %oznaczenie początku zadania
\newcommand{\zadStop}{\end{zad}}   %oznaczenie końca zadania
%Makra opcjonarne (nie muszą występować):
\newcommand{\rozwStart}[2]{\noindent \textbf{Rozwiązanie (autor #1 , recenzent #2): }\newline} %oznaczenie początku rozwiązania, opcjonarnie można wprowadzić informację o autorze rozwiązania zadania i recenzencie poprawności wykonania rozwiązania zadania
\newcommand{\rozwStop}{\newline}                                            %oznaczenie końca rozwiązania
\newcommand{\odpStart}{\noindent \textbf{Odpowiedź:}\newline}    %oznaczenie początku odpowiedzi końcowej (wypisanie wyniku)
\newcommand{\odpStop}{\newline}                                             %oznaczenie końca odpowiedzi końcowej (wypisanie wyniku)
\newcommand{\testStart}{\noindent \textbf{Test:}\newline} %ewentualne możliwe opcje odpowiedzi testowej: A. ? B. ? C. ? D. ? itd.
\newcommand{\testStop}{\newline} %koniec wprowadzania odpowiedzi testowych
\newcommand{\kluczStart}{\noindent \textbf{Test poprawna odpowiedź:}\newline} %klucz, poprawna odpowiedź pytania testowego (jedna literka): A lub B lub C lub D itd.
\newcommand{\kluczStop}{\newline} %koniec poprawnej odpowiedzi pytania testowego 
\newcommand{\wstawGrafike}[2]{\begin{figure}[h] \includegraphics[scale=#2] {#1} \end{figure}} %gdyby była potrzeba wstawienia obrazka, parametry: nazwa pliku, skala (jak nie wiesz co wpisać, to wpisz 1)

\begin{document}
\maketitle


\kategoria{Wikieł/Z4.13a}
\zadStart{Zadanie z Wikieł Z 4.13 a) moja wersja nr [nrWersji]}
%[a]:[2,3,4,5,6,7,8,9]
%[x0]:[1,2,3,4,5,6,7,8,9]
%[c]:[1,2,3,4,5,6,7,8,9]
%[b]=[x0]
Obliczyć granice jednostronne funkcji w podanych punktach $$f(x)=[a]^{\frac{[c]}{([b]-x)^2}},x_{0}=[x0].$$
\zadStop
\rozwStart{Aleksandra Pasińska}{}
$$f(x)=[a]^{\frac{[c]}{([b]-x)^2}},x_{0}=[x0]$$
$$\lim_{x\rightarrow [x0]^-}[a]^{\frac{[c]}{([b]-x)^2}}=[a]^{\frac{[c]}{0^+}}=[a]^{\infty}=\infty$$ 
$$\lim_{x\rightarrow [x0]^+}[a]^{\frac{[c]}{([b]-x)^2}}=[a]^{\frac{[c]}{0^+}}=[a]^{\infty}=\infty$$ 
\rozwStop
\odpStart
$ L=\lim_{x\rightarrow x_{0}^-}f(x), P=\lim_{x\rightarrow x_{0}^+}f(x)$
$$L=\infty, P=\infty$$
\odpStop
\testStart
A.$ L=\infty, P=\infty $
B.$L=-\infty, P=-\infty$
C.$L=-\infty, P=\infty$
D.$L=\infty, P=-\infty$
E.$L=-1, P=\infty$
F.$L=-\infty, P=0$
G.$L=-\infty, P=1$
H.$L=5, P=\infty$
I.$L=-9, P=\infty$
\testStop
\kluczStart
A
\kluczStop



\end{document}