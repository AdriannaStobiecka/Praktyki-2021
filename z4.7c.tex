\documentclass[12pt, a4paper]{article}
\usepackage[utf8]{inputenc}
\usepackage{polski}

\usepackage{amsthm}  %pakiet do tworzenia twierdzeń itp.
\usepackage{amsmath} %pakiet do niektórych symboli matematycznych
\usepackage{amssymb} %pakiet do symboli mat., np. \nsubseteq
\usepackage{amsfonts}
\usepackage{graphicx} %obsługa plików graficznych z rozszerzeniem png, jpg
\theoremstyle{definition} %styl dla definicji
\newtheorem{zad}{} 
\title{Multizestaw zadań}
\author{Robert Fidytek}
%\date{\today}
\date{}
\newcounter{liczniksekcji}
\newcommand{\kategoria}[1]{\section{#1}} %olreślamy nazwę kateforii zadań
\newcommand{\zadStart}[1]{\begin{zad}#1\newline} %oznaczenie początku zadania
\newcommand{\zadStop}{\end{zad}}   %oznaczenie końca zadania
%Makra opcjonarne (nie muszą występować):
\newcommand{\rozwStart}[2]{\noindent \textbf{Rozwiązanie (autor #1 , recenzent #2): }\newline} %oznaczenie początku rozwiązania, opcjonarnie można wprowadzić informację o autorze rozwiązania zadania i recenzencie poprawności wykonania rozwiązania zadania
\newcommand{\rozwStop}{\newline}                                            %oznaczenie końca rozwiązania
\newcommand{\odpStart}{\noindent \textbf{Odpowiedź:}\newline}    %oznaczenie początku odpowiedzi końcowej (wypisanie wyniku)
\newcommand{\odpStop}{\newline}                                             %oznaczenie końca odpowiedzi końcowej (wypisanie wyniku)
\newcommand{\testStart}{\noindent \textbf{Test:}\newline} %ewentualne możliwe opcje odpowiedzi testowej: A. ? B. ? C. ? D. ? itd.
\newcommand{\testStop}{\newline} %koniec wprowadzania odpowiedzi testowych
\newcommand{\kluczStart}{\noindent \textbf{Test poprawna odpowiedź:}\newline} %klucz, poprawna odpowiedź pytania testowego (jedna literka): A lub B lub C lub D itd.
\newcommand{\kluczStop}{\newline} %koniec poprawnej odpowiedzi pytania testowego 
\newcommand{\wstawGrafike}[2]{\begin{figure}[h] \includegraphics[scale=#2] {#1} \end{figure}} %gdyby była potrzeba wstawienia obrazka, parametry: nazwa pliku, skala (jak nie wiesz co wpisać, to wpisz 1)

\begin{document}
\maketitle


\kategoria{Wikieł/Z4.7c}
\zadStart{Zadanie z Wikieł Z 4.7 c) moja wersja nr [nrWersji]}
%[a]:[2,3,4,5,6,7]
%[b]:[2,3,4,5,6,7]
%[c]:[2,3,4,5,6,7]
%[a]=random.randint(1,12)
%[b]=random.randint(2,12)
%[c]=random.randint(1,12)
%[d]=[a]-[c]
%[a]>[c] and math.gcd([b],2)==1
Obliczyć granice funkcji $\displaystyle{\lim_{x \to \infty}}\bigg(\sqrt{x^2+[a]}-\sqrt{x^2-[b]x+[c]}\bigg)$
\zadStop
\rozwStart{Pascal Nawrocki}{}
Mnożymy poprzez sprzężenie i otrzymujemy:
$$\displaystyle{\lim_{x \to \infty}}\bigg(\frac{\bigg(\sqrt{x^2+[a]}-\sqrt{x^2-[b]x+[c]}\bigg)\bigg(\sqrt{x^2+[a]}+\sqrt{x^2-[b]x+[c]}\bigg)}{\sqrt{x^2+[a]}+\sqrt{x^2-[b]x+[c]}}\bigg)$$
Stosujemy wzór skróconego mnożenia w liczniku $(a-b)(a+b)=a^2-b^2$. Pamiętamy o tym, że $\sqrt{x^2}=|x|$, ale że u nas w przykładzie $x>0$ to pomijamy ten zapis i piszemy samego x.
$$\displaystyle{\lim_{x \to \infty}}\bigg(\frac{x^2+[a]-(x^2-[b]x+[c])}{\sqrt{x^2+[a]}+\sqrt{x^2-[b]x+[c]}}\bigg)=\displaystyle{\lim_{x \to \infty}}\bigg(\frac{[b]x+[d]}{x\sqrt{1+\frac{[a]}{x^2}}+x\sqrt{1-\frac{[b]}{x}+\frac{[c]}{x^2}}}\bigg)$$
Pozostała nam do policzenia prosta granica:
$$\displaystyle{\lim_{x \to \infty}}\bigg(\frac{x([b]+\frac{[d]}{x})}{x\bigg(\sqrt{1+\frac{[a]}{x^2}}+\sqrt{1-\frac{[b]}{x}+\frac{[c]}{x^2}}\bigg)}\bigg)=\frac{[b]}{2}$$
\rozwStop
\odpStart
$\frac{[b]}{2}$
\odpStop
\testStart
A.$\frac{[b]}{2}$
B.$\infty$
C.$-\infty$
D.$-\frac{[b]}{2}$
\testStop
\kluczStart
A
\kluczStop
\end{document}