\documentclass[12pt, a4paper]{article}
\usepackage[utf8]{inputenc}
\usepackage{polski}

\usepackage{amsthm}  %pakiet do tworzenia twierdzeń itp.
\usepackage{amsmath} %pakiet do niektórych symboli matematycznych
\usepackage{amssymb} %pakiet do symboli mat., np. \nsubseteq
\usepackage{amsfonts}
\usepackage{graphicx} %obsługa plików graficznych z rozszerzeniem png, jpg
\theoremstyle{definition} %styl dla definicji
\newtheorem{zad}{} 
\title{Multizestaw zadań}
\author{Robert Fidytek}
%\date{\today}
\date{}
\newcounter{liczniksekcji}
\newcommand{\kategoria}[1]{\section{#1}} %olreślamy nazwę kateforii zadań
\newcommand{\zadStart}[1]{\begin{zad}#1\newline} %oznaczenie początku zadania
\newcommand{\zadStop}{\end{zad}}   %oznaczenie końca zadania
%Makra opcjonarne (nie muszą występować):
\newcommand{\rozwStart}[2]{\noindent \textbf{Rozwiązanie (autor #1 , recenzent #2): }\newline} %oznaczenie początku rozwiązania, opcjonarnie można wprowadzić informację o autorze rozwiązania zadania i recenzencie poprawności wykonania rozwiązania zadania
\newcommand{\rozwStop}{\newline}                                            %oznaczenie końca rozwiązania
\newcommand{\odpStart}{\noindent \textbf{Odpowiedź:}\newline}    %oznaczenie początku odpowiedzi końcowej (wypisanie wyniku)
\newcommand{\odpStop}{\newline}                                             %oznaczenie końca odpowiedzi końcowej (wypisanie wyniku)
\newcommand{\testStart}{\noindent \textbf{Test:}\newline} %ewentualne możliwe opcje odpowiedzi testowej: A. ? B. ? C. ? D. ? itd.
\newcommand{\testStop}{\newline} %koniec wprowadzania odpowiedzi testowych
\newcommand{\kluczStart}{\noindent \textbf{Test poprawna odpowiedź:}\newline} %klucz, poprawna odpowiedź pytania testowego (jedna literka): A lub B lub C lub D itd.
\newcommand{\kluczStop}{\newline} %koniec poprawnej odpowiedzi pytania testowego 
\newcommand{\wstawGrafike}[2]{\begin{figure}[h] \includegraphics[scale=#2] {#1} \end{figure}} %gdyby była potrzeba wstawienia obrazka, parametry: nazwa pliku, skala (jak nie wiesz co wpisać, to wpisz 1)

\begin{document}
\maketitle

\kategoria{Wikieł/Z5.37j}

\zadStart{Zadanie z Wikieł Z 5.37 j) moja wersja nr [nrWersji]}
%[a]:[2,3,4,5,6,7,8,9,10]
%[b]:[2,3,4,5,6,7,8,9,10]
%[c]:[1,2,3,4,5,6,7,8,9,10]
%[d]:[1,2,3,4,5,6,7,8,9,10]
%[e]=2*[a]
Wyznaczyć współrzędne punktów przegięcia wykresu podanej funkcji.
$$y =  [a]x^2 + [b]x + \mid x + [c] \mid + [d]$$
\zadStop

\rozwStart{Natalia Danieluk}{}
Dziedzina funkcji: $\quad \mathcal{D}_f=\mathbb{R}$. \\
Postępujemy według schematu:
\begin{enumerate}
\item Obliczamy pochodne: 
$$f'(x) = [e]x + [b] + \frac{x + [c]}{\mid x + [c] \mid},$$ 
$$f''(x) = [e] + \frac{\mid x + [c] \mid - \frac{x + [c]}{\mid x + [c] \mid}\cdot (x + [c])}{(\mid x + [c] \mid)^2} = [e] + \frac{\frac{(x + [c])^2-(x + [c])^2}{\mid x + [c] \mid}}{(x + [c])^2} = [e]$$
i określamy ich dziedziny: $\quad \mathcal{D}_{f'}=\mathcal{D}_{f''}=\mathbb{R}\backslash\{-[c]\}$. \\
\item Znajdujemy miejsca zerowe $f''$: \\
Druga pochodna jest stała i nie posiada miejsc zerowych.
$$f''(x) > 0 \ \text{dla każdego} \ x \in \mathcal{D}_{f''}$$
\end{enumerate}
Tym samym druga pochodna nie zmienia znaku, a więc wykres funkcji nie ma punktów przegięcia.
\rozwStop

\odpStart
Funkcja nie ma punktów przegięcia.
\odpStop

\testStart
A. Funkcja nie ma punktów przegięcia.
B. Współrzędne punktów przegięcia to: $(0,0)$.
C. Współrzędne punktów przegięcia to:  $([a],0)$.
D. Współrzędne punktów przegięcia to:  $([e],0)$.
\testStop

\kluczStart
A
\kluczStop

\end{document}