\documentclass[12pt, a4paper]{article}
\usepackage[utf8]{inputenc}
\usepackage{polski}

\usepackage{amsthm}  %pakiet do tworzenia twierdzeń itp.
\usepackage{amsmath} %pakiet do niektórych symboli matematycznych
\usepackage{amssymb} %pakiet do symboli mat., np. \nsubseteq
\usepackage{amsfonts}
\usepackage{graphicx} %obsługa plików graficznych z rozszerzeniem png, jpg
\theoremstyle{definition} %styl dla definicji
\newtheorem{zad}{} 
\title{Multizestaw zadań}
\author{Robert Fidytek}
%\date{\today}
\date{}
\newcounter{liczniksekcji}
\newcommand{\kategoria}[1]{\section{#1}} %olreślamy nazwę kateforii zadań
\newcommand{\zadStart}[1]{\begin{zad}#1\newline} %oznaczenie początku zadania
\newcommand{\zadStop}{\end{zad}}   %oznaczenie końca zadania
%Makra opcjonarne (nie muszą występować):
\newcommand{\rozwStart}[2]{\noindent \textbf{Rozwiązanie (autor #1 , recenzent #2): }\newline} %oznaczenie początku rozwiązania, opcjonarnie można wprowadzić informację o autorze rozwiązania zadania i recenzencie poprawności wykonania rozwiązania zadania
\newcommand{\rozwStop}{\newline}                                            %oznaczenie końca rozwiązania
\newcommand{\odpStart}{\noindent \textbf{Odpowiedź:}\newline}    %oznaczenie początku odpowiedzi końcowej (wypisanie wyniku)
\newcommand{\odpStop}{\newline}                                             %oznaczenie końca odpowiedzi końcowej (wypisanie wyniku)
\newcommand{\testStart}{\noindent \textbf{Test:}\newline} %ewentualne możliwe opcje odpowiedzi testowej: A. ? B. ? C. ? D. ? itd.
\newcommand{\testStop}{\newline} %koniec wprowadzania odpowiedzi testowych
\newcommand{\kluczStart}{\noindent \textbf{Test poprawna odpowiedź:}\newline} %klucz, poprawna odpowiedź pytania testowego (jedna literka): A lub B lub C lub D itd.
\newcommand{\kluczStop}{\newline} %koniec poprawnej odpowiedzi pytania testowego 
\newcommand{\wstawGrafike}[2]{\begin{figure}[h] \includegraphics[scale=#2] {#1} \end{figure}} %gdyby była potrzeba wstawienia obrazka, parametry: nazwa pliku, skala (jak nie wiesz co wpisać, to wpisz 1)

\begin{document}
\maketitle


\kategoria{Wikieł/Z5.54}
\zadStart{Zadanie z Wikieł Z 5.54) moja wersja nr [nrWersji]}
%[a]:[1,2,3,4,5,6,7,8,9,10,11]
%[b]:[3,5,7,9,11]
%[c]:[1,2,3,4,5,6]
%[d]:[2,3,4,5,6]
%[cp]=2*[c]
%math.gcd([c],[d])==1 and math.gcd([d],[cp])==1 and [c]<[d] and [d]<[cp]
Dane są $f'(x)=\frac{x}{x^{2}+[a]},$ $g(x)=\sqrt{[b]x-[a]}$. Znaleźć $[f(g(x))]'$ oraz obliczyć $(f\circ g)'\left(\frac{[c]}{[d]}\right).$
\zadStop
\rozwStart{Justyna Chojecka}{}
Korzystamy z twierdzenia o pochodnej funkcji złożonej. Wówczas mamy
$$[f(g(x))]'=f'(g(x))\cdot g'(x)=f'\left(\sqrt{[b]x-[a]}\right)\cdot \frac{[b]}{2\sqrt{[b]x-[a]}}$$$$=\frac{\sqrt{[b]x-[a]}}{(\sqrt{[b]x-[a]})^{2}+[a]}\cdot \frac{[b]}{2\sqrt{[b]x-[a]}}=\frac{[b]}{2([b]x-[a]+[a])}=\frac{[b]}{2\cdot [b]x}=\frac{1}{2x}.$$
Następnie obliczamy $(f\circ g)'\left(\frac{[c]}{[d]}\right)$
$$(f\circ g)'\left(\frac{[c]}{[d]}\right)=\frac{1}{2\cdot\frac{[c]}{[d]}}=\frac{1}{\frac{[cp]}{[d]}}=\frac{[d]}{[cp]}.$$
\rozwStop
\odpStart
$\frac{1}{2x}$, $(f\circ g)'\left(\frac{[c]}{[d]}\right)=\frac{[d]}{[cp]}$
\odpStop
\testStart
A.$\frac{1}{2x}$, $(f\circ g)'\left(\frac{[c]}{[d]}\right)=\frac{[d]}{[cp]}$\\
B.$\frac{1}{3x}$, $(f\circ g)'\left(\frac{[c]}{[d]}\right)=\frac{[d]}{[cp]}$\\
C.$\frac{1}{3x}$, $(f\circ g)'\left(\frac{[c]}{[d]}\right)=\frac{[c]}{[d]}$\\
D.$\frac{1}{5x}$, $(f\circ g)'\left(\frac{[c]}{[d]}\right)=\frac{[d]}{[cp]}$\\
E.$\frac{1}{2x}$, $(f\circ g)'\left(\frac{[c]}{[d]}\right)=\frac{[d]}{[c]}$\\
F.$-\frac{1}{2x}$, $(f\circ g)'\left(\frac{[c]}{[d]}\right)=\frac{[c]}{[d]}$\\
G.$\frac{1}{2x}$, $(f\circ g)'\left(\frac{[c]}{[d]}\right)=\frac{[c]}{[d]}$\\
H.$\frac{1}{4x}$, $(f\circ g)'\left(\frac{[c]}{[d]}\right)=\frac{[d]}{[c]}$\\
I.$-\frac{1}{5x}$, $(f\circ g)'\left(\frac{[c]}{[d]}\right)=\frac{[d]}{[cp]}$
\testStop
\kluczStart
A
\kluczStop



\end{document}