\documentclass[12pt, a4paper]{article}
\usepackage[utf8]{inputenc}
\usepackage{polski}

\usepackage{amsthm}  %pakiet do tworzenia twierdzeń itp.
\usepackage{amsmath} %pakiet do niektórych symboli matematycznych
\usepackage{amssymb} %pakiet do symboli mat., np. \nsubseteq
\usepackage{amsfonts}
\usepackage{graphicx} %obsługa plików graficznych z rozszerzeniem png, jpg
\theoremstyle{definition} %styl dla definicji
\newtheorem{zad}{} 
\title{Multizestaw zadań}
\author{Robert Fidytek}
%\date{\today}
\date{}
\newcounter{liczniksekcji}
\newcommand{\kategoria}[1]{\section{#1}} %olreślamy nazwę kateforii zadań
\newcommand{\zadStart}[1]{\begin{zad}#1\newline} %oznaczenie początku zadania
\newcommand{\zadStop}{\end{zad}}   %oznaczenie końca zadania
%Makra opcjonarne (nie muszą występować):
\newcommand{\rozwStart}[2]{\noindent \textbf{Rozwiązanie (autor #1 , recenzent #2): }\newline} %oznaczenie początku rozwiązania, opcjonarnie można wprowadzić informację o autorze rozwiązania zadania i recenzencie poprawności wykonania rozwiązania zadania
\newcommand{\rozwStop}{\newline}                                            %oznaczenie końca rozwiązania
\newcommand{\odpStart}{\noindent \textbf{Odpowiedź:}\newline}    %oznaczenie początku odpowiedzi końcowej (wypisanie wyniku)
\newcommand{\odpStop}{\newline}                                             %oznaczenie końca odpowiedzi końcowej (wypisanie wyniku)
\newcommand{\testStart}{\noindent \textbf{Test:}\newline} %ewentualne możliwe opcje odpowiedzi testowej: A. ? B. ? C. ? D. ? itd.
\newcommand{\testStop}{\newline} %koniec wprowadzania odpowiedzi testowych
\newcommand{\kluczStart}{\noindent \textbf{Test poprawna odpowiedź:}\newline} %klucz, poprawna odpowiedź pytania testowego (jedna literka): A lub B lub C lub D itd.
\newcommand{\kluczStop}{\newline} %koniec poprawnej odpowiedzi pytania testowego 
\newcommand{\wstawGrafike}[2]{\begin{figure}[h] \includegraphics[scale=#2] {#1} \end{figure}} %gdyby była potrzeba wstawienia obrazka, parametry: nazwa pliku, skala (jak nie wiesz co wpisać, to wpisz 1)

\begin{document}
\maketitle


\kategoria{Wikieł/Z2.41d}
\zadStart{Zadanie z Wikieł Z 2.41 d)  moja wersja nr [nrWersji]}
%[p1]:[2,3,4,5,6,7,8,9,10]
%[p2]:[2,3,4,5,6,7,8,9,10]
%[p3]=random.randint(2,10)
%[p4]:[2,3,4,5,6,7,8,9,10]
%[p5]=random.randint(2,10)
%[p6]=random.randint(2,10)
%[p1p5]=[p1]*[p5]
%[p2p4]=[p2]*[p4]
%[p3p5]=[p3]*[p5]
%[p2p6]=[p2]*[p6]
%[p1p6]=[p1]*[p6]
%[p3p6]=[p3]*[p6]
%[p3p4]=[p3]*[p4]
%math.gcd([p2p4],[p1p5])==1 

Rozwiązać układ równań, w którym $m$ jest parametrem.
$$\left\{\begin{array}{ccc}
[p1]x+[p2]my&=&[p3]\\
\ [p4]mx+[p5]y&=&[p6]m
\end{array} \right.$$

\zadStop
\rozwStart{Maja Szabłowska}{}
Powyższemu układowi równań odpowiadają wyznaczniki:
$$W=\left| \begin{array}{lccr} [p1] & [p2]m \\ \ [p4]m & [p5] \end{array}\right| = [p1]\cdot[p5] - [p2]m\cdot[p4]m=[p1p5]-[p2p4]m^{2}$$

$$W_{x}=\left| \begin{array}{lccr} [p3] & [p2]m \\ \ [p6]m & [p5] \end{array}\right| = [p3]\cdot[p5] - [p2]m\cdot[p6]m=[p3p5]-[p2p6]m^{2}$$

$$W_{y}=\left| \begin{array}{lccr} [p1] & [p3] \\ \ [p4] & [p6]m \end{array}\right| = [p1]\cdot[p6]m - [p3]\cdot[p4]=[p1p6]m-[p3p4]$$

$$W\neq 0 \iff [p1p5]-[p2p4]m^{2} \neq 0  \iff
m^{2}\neq -\frac{[p1p5]}{[p2p4]}, m\in\mathbb{R} $$

$$x=\frac{W_{x}}{W}=\frac{[p3p5]-[p2p6]m^{2}}{[p1p5]-[p2p4]m^{2}}$$

$$y=\frac{W_{y}}{W}=\frac{[p1p6]m-[p3p4]}{[p1p5]-[p2p4]m^{2}}$$
\rozwStop
\odpStart
$x=\frac{[p3p5]-[p2p6]m^{2}}{[p1p5]-[p2p4]m^{2}}, y=\frac{[p1p6]m-[p3p4]}{[p1p5]-[p2p4]m^{2}}$
\odpStop
\testStart
A.$x=\frac{[p3p5]-[p2p6]m^{2}}{[p1p5]-[p2p4]m^{2}}, y=\frac{[p1p6]m-[p3p4]}{[p1p5]-[p2p4]m^{2}}$
B.$x=\frac{-[p3p5]+[p2p6]m}{-[p1p5]+[p2p4]m^{2}}, y=\frac{[p1p6]-[p3p4]m}{[p1p5]+[p2p4]m^{2}}$
C.$x=\frac{[p3p5]+[p2p6]m}{[p1p5]+[p2p4]m^{2}}, y=\frac{[p1p6]+[p3p4]m}{[p1p5]+[p2p4]m^{2}}$
D.$x=\frac{-[p3p5]+[p2p6]m}{-[p1p5]+[p2p4]m^{2}}, y=\frac{[p1p6]-[p3p4]m}{-[p1p5]+[p2p4]m}$
E.$x=\frac{-[p3p5]m+[p2p6]}{-[p1p5]+[p2p4]m^{2}}, y=\frac{[p1p6]-[p3p4]m}{-[p1p5]+[p2p4]m^{2}}$
F.$x=\frac{-[p3p5]+[p2p6]m}{[p1p5]+[p2p4]m^{2}}, y=\frac{[p1p6]-[p3p4]m}{[p1p5]}$
G.$x=\frac{[p3p5]}{-[p1p5]+[p2p4]m^{2}}, y=\frac{[p1p6]}{-[p1p5]+[p2p4]m^{2}}$
H.$x=\frac{-[p3p5]+[p2p6]m}{-[p1p5]+[p2p4]m^{2}}, y=[p1p6]-[p3p4]m$
I.$x=[p1], y=-[p5]$
\testStop
\kluczStart
A
\kluczStop



\end{document}
